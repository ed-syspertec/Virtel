%% Generated by Sphinx.
\def\sphinxdocclass{report}
\documentclass[letterpaper,10pt,english]{sphinxmanual}
\ifdefined\pdfpxdimen
   \let\sphinxpxdimen\pdfpxdimen\else\newdimen\sphinxpxdimen
\fi \sphinxpxdimen=.75bp\relax



\usepackage{cmap}

\usepackage{amsmath,amssymb,amstext}
\usepackage{babel}

\usepackage[Bjarne]{fncychap}
\usepackage[dontkeepoldnames]{sphinx}

\usepackage{geometry}

% Include hyperref last.
\usepackage{hyperref}
% Fix anchor placement for figures with captions.
\usepackage{hypcap}% it must be loaded after hyperref.
% Set up styles of URL: it should be placed after hyperref.
\urlstyle{same}
\addto\captionsenglish{\renewcommand{\contentsname}{Table of Contents:}}

\addto\captionsenglish{\renewcommand{\figurename}{Fig.}}
\addto\captionsenglish{\renewcommand{\tablename}{Table}}
\addto\captionsenglish{\renewcommand{\literalblockname}{Listing}}

\addto\captionsenglish{\renewcommand{\literalblockcontinuedname}{continued from previous page}}
\addto\captionsenglish{\renewcommand{\literalblockcontinuesname}{continues on next page}}

\addto\extrasenglish{\def\pageautorefname{page}}

\setcounter{tocdepth}{2}


% Enable unicode and use Courier New to ensure the card suit
% characters that are part of the 'random' module examples
% appear properly in the PDF output.
\usepackage{fontspec}
\setmonofont{Courier New}


\title{Virtel Messages}
\date{Mar 30, 2018}
\release{4.57}
\author{Syspertec Communications}
\newcommand{\sphinxlogo}{\vbox{}}
\renewcommand{\releasename}{Release}
\makeindex

\begin{document}

\maketitle
\sphinxtableofcontents
\phantomsection\label{\detokenize{messages::doc}}


\sphinxincludegraphics[scale=0.5]{{logo_virtel_web}.png}

\sphinxstylestrong{VIRTEL Messages Guide}

\begin{sphinxadmonition}{danger}{Danger:}
This book is currently under construction. Do not use!
\end{sphinxadmonition}

Version : 4.57

Release Date : 01 Jul 2017 Publication Date : 01/07/2017

Syspertec Communication

196, Bureaux de la Colline 92213 Saint-Cloud Cedex Tél. : +33 (0) 1 46 02 60 42

\sphinxhref{http://www.syspertec.com/}{www.syspertec.com}

\begin{sphinxadmonition}{note}{Note:}
Reproduction, transfer, distribution, or storage, in any form, of all or any part of
the contents of this document, except by prior authorization of SysperTec
Communication, is prohibited.

Every possible effort has been made by SysperTec Communication to ensure that this document
is complete and relevant. In no case can SysperTec Communication be held responsible for
any damages, direct or indirect, caused by errors or omissions in this document.

As SysperTec Communication uses a continuous development methodology; the information
contained in this document may be subject to change without notice. Nothing in this
document should be construed in any manner as conferring a right to use, in whole or in
part, the products or trademarks quoted herein.

“SysperTec Communication” and “VIRTEL” are registered trademarks. Names of other products
and companies mentioned in this document may be trademarks or registered trademarks of
their respective owners.
\end{sphinxadmonition}


\chapter{Virtel Messages}
\label{\detokenize{messages:v457mg-introduction}}\label{\detokenize{messages:virtel457mg}}\label{\detokenize{messages:messages-v4-57}}\label{\detokenize{messages:virtel-messages}}

\section{Virtel user messages}
\label{\detokenize{messages:virtel-user-messages}}

\subsection{ERRM121 - ERRM127}
\label{\detokenize{messages:errm121-errm127}}
\begin{sphinxadmonition}{note}{Note:}
These messages may be displayed in the user’s browser window following an unsuccessful upload of an HTML page, script, or other element to VIRTEL.
\end{sphinxadmonition}


\subsubsection{ERRM121 -UPLOAD NOT AUTHORIZED}
\label{\detokenize{messages:errm121-upload-not-authorized}}\begin{description}
\item[{Module}] \leavevmode
VIR0U12

\item[{Meaning}] \leavevmode
The user is not authorized to upload a page to the specified directory.

\item[{Action}] \leavevmode
Ensure that the directory is defined to VIRTEL (see “Directory Management” in the VIRTEL Web Access Guide). Ensure that the “Copy Up”, “Copy Down”, and “Delete” flags are all set in the directory definition. If using VIRTEL internal security or an external security product, ensure that you have authorization to the resource name which is the same as the VIRTEL directory name.

\end{description}


\subsubsection{ERRM122 - ERROR IN VIRTEL HEADER COMMENT}
\label{\detokenize{messages:errm122-error-in-virtel-header-comment}}\begin{description}
\item[{Module}] \leavevmode
VIR0U12

\item[{Meaning}] \leavevmode
There is a syntax error in the definition of the VIRTEL tag delimiters in the page being uploaded. The format of the definition should be

\end{description}

\begin{sphinxVerbatim}[commandchars=\\\{\}]
\PYGZlt{}!\PYGZhy{}\PYGZhy{}VIRTEL start=\PYGZdq{}\PYGZob{}\PYGZob{}\PYGZob{}\PYGZdq{} end=\PYGZdq{}\PYGZcb{}\PYGZcb{}\PYGZcb{}\PYGZdq{} \PYGZhy{}\PYGZhy{}\PYGZgt{}
\end{sphinxVerbatim}
\begin{description}
\item[{Action}] \leavevmode
Refer to the section “Defining the Tag Delimiters” in the VIRTEL Web Access Guide. Correct the definition and upload the page again.

\end{description}


\subsubsection{ERRM124 - INVALID FIELD}
\label{\detokenize{messages:errm124-invalid-field}}\begin{description}
\item[{Module}] \leavevmode
VIR0U12

\item[{Meaning}] \leavevmode
There is a syntax error in a VIRTEL tag in the page being uploaded.

\item[{Action}] \leavevmode
Refer to the section “Creating HTML and XML template pages” in the VIRTEL Web Access Guide. Correct the definition and upload the page again.

\end{description}


\subsubsection{ERRM125 - INTERNAL ERROR}
\label{\detokenize{messages:errm125-internal-error}}\begin{description}
\item[{Module}] \leavevmode
VIR0U12

\item[{Meaning}] \leavevmode
An unexpected error occurred while uploading a page.

\item[{Action}] \leavevmode
Use the VIRTEL SNAP command to take a snapshot of the internal trace table. Contact technical support.

\end{description}


\subsubsection{ERRM126 - OPEN ERROR}
\label{\detokenize{messages:errm126-open-error}}\begin{description}
\item[{Module}] \leavevmode
VIR0U12

\item[{Meaning}] \leavevmode
An error occurred while attempting to open the directory for a page upload.

\item[{Action}] \leavevmode
Check the directory definition in VIRTEL (see “Directory Management” in the VIRTEL Web Access Guide). Ensure that the DDNAME it refers to is defined in the UFILEn parameter of the VIRTCT and in the VIRTEL started task JCL, and that the file exists and can be opened. Check the console log for any VSAM error messages. Ensure that the “Copy Up”, “Copy Down”, and “Delete” flags are all set in the VIRTEL directory definition. For a newly-defined VSAM cluster, ensure that the file has been loaded with an \$\$\$\$IWS.WORKREC record using the installation job VIR4INST.

\end{description}


\subsubsection{ERRM127 - WRITE ERROR}
\label{\detokenize{messages:errm127-write-error}}\begin{description}
\item[{Module}] \leavevmode
VIR0U12

\item[{Meaning}] \leavevmode
An error occurred while writing into the directory during a page upload.

\item[{Action}] \leavevmode
Check the console log for any VSAM error messages. Use the VIRTEL SNAP command to take a snapshot of the internal trace table. Contact technical support.

\end{description}


\subsection{User Interface Messages}
\label{\detokenize{messages:user-interface-messages}}
\begin{sphinxadmonition}{note}{Note:}
These messages may be displayed in the message area of the user’s terminal when logged on to VIRTEL.
\end{sphinxadmonition}


\subsubsection{0 CONNECTION IN PROGRESS…}
\label{\detokenize{messages:connection-in-progress}}\begin{description}
\item[{Module}] \leavevmode
VIR0015, VIR0021A, VIR0021R, VIR0025, VIR0025T

\item[{Meaning}] \leavevmode
VIRTEL is attempting to connect your terminal to the requested host application.

\item[{Action}] \leavevmode
Wait for the response from the host application.

\end{description}


\subsubsection{1 DATA ENTERED IS INVALID FOR THIS SCREEN}
\label{\detokenize{messages:data-entered-is-invalid-for-this-screen}}
Module
VIR0021, VIR0021A, VIR0021R
Meaning
The option selected is invalid for this screen.
Action
Choose one of the options displayed on the screen.


\subsubsection{2 ERROR CONNECTING TO APPLICATION}
\label{\detokenize{messages:error-connecting-to-application}}\begin{description}
\item[{Module}] \leavevmode
VIR0015, VIR0021H

\item[{Meaning}] \leavevmode
VIRTEL was unable to connect your terminal to the requested host application.

\item[{Action}] \leavevmode
Look for messages on the system console to find the reason for the error.

\end{description}


\subsubsection{3 APPLICATION HAS DISABLED THE SESSION}
\label{\detokenize{messages:application-has-disabled-the-session}}\begin{description}
\item[{Module}] \leavevmode
VIR0015

\item[{Meaning}] \leavevmode
The host application cannot accept the connection from your terminal.

\item[{Action}] \leavevmode
Look for messages on the system console to find the reason for the error.

\end{description}

4 UNKNOWN TERMINAL

This message is no longer issued.


\subsubsection{5 NO SAVED SCREEN FOR THIS SESSION}
\label{\detokenize{messages:no-saved-screen-for-this-session}}\begin{description}
\item[{Module}] \leavevmode
VIR0022, VIR0022O

\item[{Meaning}] \leavevmode
The user requested VIRTEL to redisplay a previously saved screen image, but no screen image has yet been saved for this terminal session.

\item[{Action}] \leavevmode
None.

\end{description}


\subsubsection{8 INSUFFICIENT MEMORY}
\label{\detokenize{messages:insufficient-memory}}\begin{description}
\item[{Module}] \leavevmode
VIR0043, VIR0043H

\item[{Meaning}] \leavevmode
There is insufficient storage to process the application.

\item[{Action}] \leavevmode
Increase VIRTEL region size.

\end{description}


\subsubsection{9 USER NAME UNKNOWN}
\label{\detokenize{messages:user-name-unknown}}\begin{description}
\item[{Module}] \leavevmode
VIR0010

\item[{Meaning}] \leavevmode
Signon was rejected because the security subsystem does not recognize the userid you entered.

\item[{Action}] \leavevmode
Sign on again with a valid userid.

\end{description}


\subsubsection{10 INCORRECT PASSWORD}
\label{\detokenize{messages:incorrect-password}}\begin{description}
\item[{Module}] \leavevmode
VIR0010, VIR0021H

\item[{Meaning}] \leavevmode
Signon was rejected because you did not supply a password, or the password is incorrect.

\item[{Action}] \leavevmode
Sign on again with the correct password.

\end{description}


\subsubsection{11 PASSWORD HAS EXPIRED}
\label{\detokenize{messages:password-has-expired}}\begin{description}
\item[{Module}] \leavevmode
VIR0010

\item[{Meaning}] \leavevmode
Signon was rejected by the security subsystem because your password has expired.

\item[{Action}] \leavevmode
Sign on again specifying your expired password in the “your Password” field, and a password of your choice in the “NEW PASSWORD” field.

\end{description}


\subsubsection{12 NEW PASSWORD IS INVALID}
\label{\detokenize{messages:new-password-is-invalid}}\begin{description}
\item[{Module}] \leavevmode
VIR0010

\item[{Meaning}] \leavevmode
Signon was unsuccessful because the security subsystem rejected your new password.

\item[{Action}] \leavevmode
Contact your security administrator to discover the rules for new passwords. Sign on again specifying a valid password in the “NEW PASSWORD” field.

\end{description}


\subsubsection{13 SIGNON IS SUSPENDED}
\label{\detokenize{messages:signon-is-suspended}}\begin{description}
\item[{Module}] \leavevmode
VIR0010

\item[{Meaning}] \leavevmode
Signon was unsuccessful because your userid has been suspended or revoked by the security subsystem.

\item[{Action}] \leavevmode
Contact your security administrator to reinstate your userid.

\end{description}


\subsubsection{14 USER NOT AUTHORISED TO USE TERMINAL}
\label{\detokenize{messages:user-not-authorised-to-use-terminal}}\begin{description}
\item[{Module}] \leavevmode
VIR0010

\item[{Meaning}] \leavevmode
Signon was unsuccessful because the security subsystem does not allow your userid to log on to this terminal.

\item[{Action}] \leavevmode
Choose a terminal which you are authorized to use.

\end{description}


\subsubsection{15 USER NOT AUTHORISED TO USE APPLICATION}
\label{\detokenize{messages:user-not-authorised-to-use-application}}\begin{description}
\item[{Module}] \leavevmode
VIR0010

\item[{Meaning}] \leavevmode
Signon was unsuccessful because the security subsystem does not allow your userid to log on to this application.

\item[{Action}] \leavevmode
Ask your security administrator to authorize you to log on to the VIRTEL application.

\end{description}


\subsubsection{16 ERROR DURING SIGN-ON PROCESSING}
\label{\detokenize{messages:error-during-sign-on-processing}}\begin{description}
\item[{Module}] \leavevmode
VIR0010

\item[{Meaning}] \leavevmode
The security subsystem rejected the signon for an unknown reason.

\item[{Action}] \leavevmode
Contact technical support. Look for messages in the system log which might indicate the reason for the failure.

\end{description}


\subsubsection{17 TRANSACTION ABEND}
\label{\detokenize{messages:transaction-abend}}\begin{description}
\item[{Module}] \leavevmode
VIR0010

\item[{Meaning}] \leavevmode
A VIRTEL subapplication program has abended.

\item[{Action}] \leavevmode
Look at the VIRTEL system log to detemine the cause of the error.

\end{description}


\subsubsection{18 ENTER YOUR USERID AND PASSWORD}
\label{\detokenize{messages:enter-your-userid-and-password}}\begin{description}
\item[{Module}] \leavevmode
VIR0010

\item[{Meaning}] \leavevmode
The signon module is requesting your userid and password.

\item[{Action}] \leavevmode
Enter your userid and password at the signon screen.

\end{description}


\subsubsection{19 SIGN ON CANCELED AND SESSION ENDED}
\label{\detokenize{messages:sign-on-canceled-and-session-ended}}\begin{description}
\item[{Module}] \leavevmode
VIR0010, VIR0020, VIR0020A, VIR0020L, VIR0020M, VIR0020P

\item[{Meaning}] \leavevmode
You pressed PA2 at the signon screen to cancel signon.

\item[{Action}] \leavevmode
VIRTEL signs off and ends the session.

\end{description}


\subsubsection{20 UNEXPECTED CHOICE}
\label{\detokenize{messages:unexpected-choice}}\begin{description}
\item[{Module}] \leavevmode
VIR0014, VIR0034

\item[{Meaning}] \leavevmode
The service requested is not available from this screen.

\item[{Action}] \leavevmode
Choose one of the services displayed on the screen.

\end{description}


\subsubsection{21 NO MORE PAGES AT THIS LEVEL}
\label{\detokenize{messages:no-more-pages-at-this-level}}\begin{description}
\item[{Module}] \leavevmode
VIR0014, VIR0034

\item[{Meaning}] \leavevmode
There are no more pages at this level of the videotex page hierarchy.

\item[{Action}] \leavevmode
None.

\end{description}


\subsubsection{22 THIS SERVICE IS ACCESS RESTRICTED}
\label{\detokenize{messages:this-service-is-access-restricted}}\begin{description}
\item[{Module}] \leavevmode
VIR0014, VIR0034

\item[{Meaning}] \leavevmode
You must be signed on to access this service.

\item[{Action}] \leavevmode
Sign on with a userid authorized to access the requested service.

\end{description}


\subsubsection{23 YOU ARE NOT AUTHORISED TO ACCESS THIS SERVICE}
\label{\detokenize{messages:you-are-not-authorised-to-access-this-service}}\begin{description}
\item[{Module}] \leavevmode
VIR0014, VIR0034

\item[{Meaning}] \leavevmode
Your userid is not authorized to access the requested service.

\item[{Action}] \leavevmode
Request the security administrator to authorize your userid to access the requested service. See the Security chapter in the Virtel user guide for more information on protecting Virtel resources and services.

\end{description}


\subsubsection{24 SERVICE BRIEFLY INTERRUPTED}
\label{\detokenize{messages:service-briefly-interrupted}}\begin{description}
\item[{Module}] \leavevmode
VIR0014, VIR0034

\item[{Meaning}] \leavevmode
The service is temporarily unavailable.

\item[{Action}] \leavevmode
Wait for the service to become available.

\end{description}


\subsubsection{25 NO PAGES / GUIDE AVAILABLE}
\label{\detokenize{messages:no-pages-guide-available}}\begin{description}
\item[{Module}] \leavevmode
VIR0014, VIR0034

\item[{Meaning}] \leavevmode
You pressed the GUIDE key but there is no guide page associated with this screen.

\item[{Action}] \leavevmode
None.

\end{description}


\subsubsection{26 THE REQUESTED SERVICE IS UNKNOWN}
\label{\detokenize{messages:the-requested-service-is-unknown}}\begin{description}
\item[{Module}] \leavevmode
VIR0014, VIR0034

\item[{Meaning}] \leavevmode
The service requested is not in the index.

\item[{Action}] \leavevmode
None.

\end{description}


\subsubsection{27 PF KEY SELECTED IS INVALID FOR THIS SCREEN}
\label{\detokenize{messages:pf-key-selected-is-invalid-for-this-screen}}\begin{description}
\item[{Module}] \leavevmode
VIR0014, VIR0034, VIR0020, VIR0020A, VIR0020L, VIR0020M, VIR0020P, VIR0022A, VIR0025

\item[{Meaning}] \leavevmode
The PF key selected is invalid for this screen.

\item[{Action}] \leavevmode
Press one of the PF keys displayed on the screen.

\end{description}


\subsubsection{28 USER ID IS SUSPENDED}
\label{\detokenize{messages:user-id-is-suspended}}\begin{description}
\item[{Module}] \leavevmode
VIR0020, VIR0020A, VIR0020L, VIR0020M, VIR0020P

\item[{Meaning}] \leavevmode
Your userid has been suspended.

\item[{Action}] \leavevmode
Ask your security administrator to reinstate your access.

\end{description}


\subsubsection{30 TERMINAL NAME UNKNOWN}
\label{\detokenize{messages:terminal-name-unknown}}\begin{description}
\item[{Module}] \leavevmode
VIR0022, VIR0022O

\item[{Meaning}] \leavevmode
You have requested the display of a terminal which does not exist.

\item[{Action}] \leavevmode
None.

\end{description}


\subsubsection{31 YOU ARE ON THE FIRST PAGE}
\label{\detokenize{messages:you-are-on-the-first-page}}\begin{description}
\item[{Module}] \leavevmode
Various

\item[{Meaning}] \leavevmode
You tried to scroll back to the previous page of data but you are already on the first page.

\item[{Action}] \leavevmode
None.

\end{description}


\subsubsection{32 YOU ARE ON THE LAST PAGE}
\label{\detokenize{messages:you-are-on-the-last-page}}\begin{description}
\item[{Module}] \leavevmode
Various

\item[{Meaning}] \leavevmode
You tried to scroll forward to the next page of data but you are already on the last page.

\item[{Action}] \leavevmode
None.

\end{description}


\subsubsection{33 YOU ARE NOT AUTHORISED TO USE THIS APPLICATION}
\label{\detokenize{messages:you-are-not-authorised-to-use-this-application}}\begin{description}
\item[{Module}] \leavevmode
VIR0021, VIR0021A, VIR0022, VIR0022O, VIR0041A, VIR0043, VIR0043H

\item[{Meaning}] \leavevmode
You requested a VIRTEL subapplication but either you do not have the necessary authorization, or the subapplication has not been enabled in the VIRTCT. When this message is issued by VIR0043 or VIR0043H, it indicates that the VIRTEL directory does not permit the requested operation (upload, download, or delete).

\item[{Action}] \leavevmode
Ask your security administrator to grant you authorization to the requested subapplication. Refer to the VIRTEL Connectivity Guide for details of the authorization mechanism for subapplications. The availability of certain subapplications is governed by the ARBO, RESO, HTVSAM, VIRSECU parameters of the VIRTCT, documented in the VIRTEL Installation Guide. Refer to “Directory Management” in the VIRTEL Web Access Guide for details of VIRTEL directory permissions. Also, see the Security chapter in the Virtel user guide for more information on protecting Virtel resources and services.

\end{description}


\subsubsection{34 UPDATE OK}
\label{\detokenize{messages:update-ok}}\begin{description}
\item[{Module}] \leavevmode
VIR0023, VIR0026, VIR0031, VIR0041, VIR0041A, VIR0042, VIR0044, VIR0045, VIR0046, VIR0047, VIR1001, VIR1002

\item[{Meaning}] \leavevmode
The updated definition has been successfully stored in the VIRARBO file.

\item[{Action}] \leavevmode
None.

\end{description}


\subsubsection{35 CREATION OK}
\label{\detokenize{messages:creation-ok}}\begin{description}
\item[{Module}] \leavevmode
VIR0023, VIR0026, VIR0031, VIR0041, VIR0041A, VIR0042, VIR0044, VIR0045, VIR0046, VIR0047, VIR1001, VIR1002

\item[{Meaning}] \leavevmode
The new definition has been successfully added to the VIRARBO file.

\item[{Action}] \leavevmode
None.

\end{description}


\subsubsection{36 DELETE OK}
\label{\detokenize{messages:delete-ok}}\begin{description}
\item[{Module}] \leavevmode
VIR0023, VIR0026, VIR0031, VIR0033, VIR0041, VIR0041A, VIR0042, VIR0043, VIR0043H, VIR0044, VIR0045, VIR0046, VIR0047, VIR0052, VIR1001, VIR1002

\item[{Meaning}] \leavevmode
The selected definition has been successfully deleted from the VIRARBO file.

\item[{Action}] \leavevmode
None.

\end{description}


\subsubsection{37 RECORD ALREADY EXISTS}
\label{\detokenize{messages:record-already-exists}}\begin{description}
\item[{Module}] \leavevmode
VIR0023, VIR0026, VIR0031, VIR0041, VIR0041A, VIR0042, VIR0044, VIR0045, VIR0046, VIR0047, VIR1001, VIR1002, VIR1005

\item[{Meaning}] \leavevmode
The new definition cannot be added because a record with the same identifier already exists in the VIRARBO file.

\item[{Action}] \leavevmode
Either choose a new identifier, or select the existing record and update it.

\end{description}


\subsubsection{38 RECORD DOES NOT EXIST}
\label{\detokenize{messages:record-does-not-exist}}\begin{description}
\item[{Module}] \leavevmode
VIR0023, VIR0026, VIR0031, VIR0041, VIR0041A, VIR0042, VIR0043, VIR0044, VIR0045, VIR0046, VIR0047, VIR0052, VIR1001, VIR1002, VIR1003, VIR1004, VIR1005, VIR1006

\item[{Meaning}] \leavevmode
The selected definition cannot be deleted from the VIRARBO file because it no longer exists.

\item[{Action}] \leavevmode
None.

\end{description}


\subsubsection{39 INVALID CURSOR POSITION}
\label{\detokenize{messages:invalid-cursor-position}}\begin{description}
\item[{Module}] \leavevmode
VIR0023, VIR0026, VIR0031, VIR0041, VIR0041A, VIR0042, VIR0043, VIR0043H, VIR0044, VIR0045, VIR0046, VIR0047, VIR1003, VIR1004, VIR1006

\item[{Meaning}] \leavevmode
You pressed the “delete” function key (F2) but the cursor was not pointing to a configuration record.

\item[{Action}] \leavevmode
Move the cursor to the record to be deleted.

\end{description}


\subsubsection{40 CONFIRM DELETE}
\label{\detokenize{messages:confirm-delete}}\begin{description}
\item[{Module}] \leavevmode
VIR0023, VIR0026, VIR0031, VIR0033, VIR0041, VIR0041A, VIR0042, VIR0043, VIR0043H, VIR0044, VIR0045, VIR0046, VIR0047, VIR0052, VIR1001, VIR1002

\item[{Meaning}] \leavevmode
You pressed the “delete” function key (F2) to request the deletion of a configuration record.

\item[{Action}] \leavevmode
Press F2 again if you wish to delete the highlighted record. Press any other function key to cancel the delete.

\end{description}


\subsubsection{41 KEY IN DATA AND PRESS ENTER}
\label{\detokenize{messages:key-in-data-and-press-enter}}\begin{description}
\item[{Module}] \leavevmode
VIR0023, VIR0026, VIR0041, VIR0041A, VIR0042, VIR1001

\item[{Meaning}] \leavevmode
An empty screen is displayed into which you can enter the definition of a new configuration record.

\item[{Action}] \leavevmode
Fill in the fields on the screen and press Enter to add the new record to the VIRARBO file.

\end{description}


\subsubsection{42 INCORRECT VALUE}
\label{\detokenize{messages:incorrect-value}}\begin{description}
\item[{Module}] \leavevmode
VIR0023, VIR0031, VIR0041, VIR0041A, VIR0044, VIR0045, VIR0047, VIR1001

\item[{Meaning}] \leavevmode
A value entered in a configuration record is not acceptable. For example, a field contains invalid characters, or the terminal name does not correspond to the terminal prefix of the associated line, or the transaction name does not correspond to the prefix of the associated entry point.

\item[{Action}] \leavevmode
Correct the field in error and press Enter.

\end{description}


\subsubsection{43 PLEASE WAIT….}
\label{\detokenize{messages:please-wait}}\begin{description}
\item[{Module}] \leavevmode
VIR0012, VIR0016, VIR0032, VIR0036

\item[{Meaning}] \leavevmode
VIRTEL is connecting your terminal to the requested application.

\item[{Action}] \leavevmode
Wait for the response from the host application.

\end{description}


\subsubsection{45 EXCESSIVE NUMBER OF ATTEMPTS}
\label{\detokenize{messages:excessive-number-of-attempts}}\begin{description}
\item[{Module}] \leavevmode
VIR0010

\item[{Meaning}] \leavevmode
Signon was rejected by the security subsystem because too many incorrect signon attempts were made.

\item[{Action}] \leavevmode
Contact your security administrator.

\end{description}


\subsubsection{46 PASSWORD HAS BEEN CHANGED}
\label{\detokenize{messages:password-has-been-changed}}\begin{description}
\item[{Module}] \leavevmode
VIR0010

\item[{Meaning}] \leavevmode
The user requested a change of password during signon. The new password has been accepted by the security subsystem.

\item[{Action}] \leavevmode
None.

\end{description}


\subsubsection{47 INACTIVITY TIMEOUT PLEASE ENTER YOUR PASSWORD}
\label{\detokenize{messages:inactivity-timeout-please-enter-your-password}}\begin{description}
\item[{Module}] \leavevmode
VIR0020, VIR0020A, VIR0020L, VIR0020M, VIR0020P

\item[{Meaning}] \leavevmode
VIRTEL has locked your terminal session because it has been inactive for too long.

\item[{Action}] \leavevmode
Enter your password to unlock the terminal.

\end{description}


\subsubsection{48 INVALID FUNCTION KEY}
\label{\detokenize{messages:invalid-function-key}}\begin{description}
\item[{Module}] \leavevmode
VIR1001-1006, VIR2002-2013, VIR2015-2016, VIR2019

\item[{Meaning}] \leavevmode
The PF key selected is invalid for this screen.

\item[{Action}] \leavevmode
Press one of the PF keys displayed on the screen.

\end{description}


\subsubsection{49 NODE TYPE UNSUPPORTED}
\label{\detokenize{messages:node-type-unsupported}}\begin{description}
\item[{Module}] \leavevmode
VIR2002, VIR2014, VIR2019

\item[{Meaning}] \leavevmode
Cannot perform the requested function on this type of node.

\item[{Action}] \leavevmode
None.

\end{description}


\subsubsection{50 NODE TYPE INVALID}
\label{\detokenize{messages:node-type-invalid}}\begin{description}
\item[{Module}] \leavevmode
VIR2002-2019

\item[{Meaning}] \leavevmode
The node type is not recognized by the network management application.

\item[{Action}] \leavevmode
None.

\end{description}


\subsubsection{51 PROGRAM progname UNKNOWN OR DISABLED}
\label{\detokenize{messages:program-progname-unknown-or-disabled}}\begin{description}
\item[{Module}] \leavevmode
VIR0040, VIR1000-1006, VIR2002-2019, VIR4000-4023

\item[{Meaning}] \leavevmode
A VIRTEL subapplication attempted to transfer control to a program which is not available.

\item[{Action}] \leavevmode
Contact technical support.

\end{description}


\subsubsection{52 INVALID CHOICE}
\label{\detokenize{messages:invalid-choice}}\begin{description}
\item[{Module}] \leavevmode
VIR0040, VIR1000, VIR4000-4003, VIR4005

\item[{Meaning}] \leavevmode
You entered an option which is not displayed on the menu.

\item[{Action}] \leavevmode
Choose one of the options displayed on the menu.

\end{description}


\subsubsection{53 FUNCTION RESERVED FOR ADMINISTRATOR}
\label{\detokenize{messages:function-reserved-for-administrator}}\begin{description}
\item[{Module}] \leavevmode
VIR4000

\item[{Meaning}] \leavevmode
You entered an option which requires administrator authorization, but your userid does not have administrator privileges.

\item[{Action}] \leavevmode
Sign on with an administrator userid.

\end{description}


\subsubsection{54 OPTION RESERVED FOR HEAD OF DEPARTMENT}
\label{\detokenize{messages:option-reserved-for-head-of-department}}\begin{description}
\item[{Module}] \leavevmode
VIR0040, VIR4000-4003

\item[{Meaning}] \leavevmode
You entered an option which requires authorization by head of department, but your userid does not have the necessary privileges.

\item[{Action}] \leavevmode
Sign on with a userid which is marked as head of department.

\end{description}


\subsubsection{55 ENTER NAME OF DEPARTMENT}
\label{\detokenize{messages:enter-name-of-department}}\begin{description}
\item[{Module}] \leavevmode
VIR4004, VIR4020, VIR4021

\item[{Meaning}] \leavevmode
The requested operation requires a value in the indicated field but the field is blank.

\item[{Action}] \leavevmode
Enter a value in the indicated field.

\end{description}


\subsubsection{56 ENTER THE DESCRIPTION OF THE deptname DEPARTMENT}
\label{\detokenize{messages:enter-the-description-of-the-deptname-department}}\begin{description}
\item[{Module}] \leavevmode
VIR4020, VIR4021

\item[{Meaning}] \leavevmode
The requested operation requires a value but the field is blank.

\item[{Action}] \leavevmode
Enter a value in the indicated field.

\end{description}


\subsubsection{57 ENTER THE NAME OF THE PERSON RESPONSIBLE}
\label{\detokenize{messages:enter-the-name-of-the-person-responsible}}\begin{description}
\item[{Module}] \leavevmode
VIR4020, VIR4021

\item[{Meaning}] \leavevmode
The requested operation requires a value but the field is blank.

\item[{Action}] \leavevmode
Enter a value in the indicated field.

\end{description}


\subsubsection{58 ENTER THE TITLE OF THE PERSON RESPONSIBLE}
\label{\detokenize{messages:enter-the-title-of-the-person-responsible}}\begin{description}
\item[{Module}] \leavevmode
VIR4020, VIR4021

\item[{Meaning}] \leavevmode
The requested operation requires a value but the field is blank.

\item[{Action}] \leavevmode
Enter a value in the indicated field.

\end{description}


\subsubsection{59 THE DEPARTMENT ALREADY EXISTS}
\label{\detokenize{messages:the-department-already-exists}}\begin{description}
\item[{Module}] \leavevmode
VIR4020

\item[{Meaning}] \leavevmode
You are attempting to add a new department but a department of the same name already exists.

\item[{Action}] \leavevmode
Choose a new department name or edit the existing department.

\end{description}


\subsubsection{60 THE RESPONSIBLE PERSON ALREADY EXISTS}
\label{\detokenize{messages:the-responsible-person-already-exists}}\begin{description}
\item[{Module}] \leavevmode
VIR4020

\item[{Meaning}] \leavevmode
You are attempting to add a new person but a person of the same name already exists.

\item[{Action}] \leavevmode
Choose a new person or edit the existing person.

\end{description}


\subsubsection{61 PRESS PF1 TO CONFIRM CREATION}
\label{\detokenize{messages:press-pf1-to-confirm-creation}}\begin{description}
\item[{Module}] \leavevmode
VIR4006, VIR4010, VIR4014, VIR4020

\item[{Meaning}] \leavevmode
VIRTEL is requesting confirmation that you wish to create a new record.

\item[{Action}] \leavevmode
Press F1 to create the new record. Press any other function key to cancel creation of the new record.

\end{description}


\subsubsection{62 CREATION OK}
\label{\detokenize{messages:id1}}\begin{description}
\item[{Module}] \leavevmode
VIR4006, VIR4010, VIR4014, VIR4020, VIR4022

\item[{Meaning}] \leavevmode
The new record has been created.

\item[{Action}] \leavevmode
None.

\end{description}


\subsubsection{63 END OF FILE}
\label{\detokenize{messages:end-of-file}}\begin{description}
\item[{Module}] \leavevmode
VIR1001-1006, VIR4004, VIR4007-4009, VIR4011-4014, VIR4015-4016, VIR4021-4023

\item[{Meaning}] \leavevmode
The end of file has been reached while browsing forward through the file.

\item[{Action}] \leavevmode
None.

\end{description}


\subsubsection{64 THE DEPARTMENT DOES NOT EXIST}
\label{\detokenize{messages:the-department-does-not-exist}}\begin{description}
\item[{Module}] \leavevmode
VIR4004, VIR4021

\item[{Meaning}] \leavevmode
The department which you are attempting to update no longer exists.

\item[{Action}] \leavevmode
None.

\end{description}


\subsubsection{65 UPDATE OK}
\label{\detokenize{messages:id2}}\begin{description}
\item[{Module}] \leavevmode
VIR4004, VIR4008, VIR4011, VIR4015, VIR4021, VIR4023

\item[{Meaning}] \leavevmode
The updated record has been successfully written to the file.

\item[{Action}] \leavevmode
None.

\end{description}


\subsubsection{66 BEGINNING OF FILE}
\label{\detokenize{messages:beginning-of-file}}\begin{description}
\item[{Module}] \leavevmode
VIR4004, VIR4008, VIR4011, VIR4015, VIR4021, VIR4023

\item[{Meaning}] \leavevmode
The beginning of the file has been reached while browsing backwards through the file.

\item[{Action}] \leavevmode
None.

\end{description}


\subsubsection{67 THE DEPARTMENT STILL CONTAINS PROFILES}
\label{\detokenize{messages:the-department-still-contains-profiles}}\begin{description}
\item[{Module}] \leavevmode
VIR4021

\item[{Meaning}] \leavevmode
The department cannot be deleted because there are profiles associated with this department.

\item[{Action}] \leavevmode
Delete the associated profiles before deleting the department.

\end{description}


\subsubsection{68 THE DEPARTMENT STILL CONTAINS USERS}
\label{\detokenize{messages:the-department-still-contains-users}}\begin{description}
\item[{Module}] \leavevmode
VIR4021

\item[{Meaning}] \leavevmode
The department cannot be deleted because there are users associated with this department.

\item[{Action}] \leavevmode
Delete the associated users before deleting the department.

\end{description}


\subsubsection{69 USE PF2 TO CONFIRM DELETE}
\label{\detokenize{messages:use-pf2-to-confirm-delete}}\begin{description}
\item[{Module}] \leavevmode
VIR4008-4009, VIR4011, VIR4015-4016, VIR4021

\item[{Meaning}] \leavevmode
You requested the deletion of a configuration record.

\item[{Action}] \leavevmode
Press F2 if you wish to delete the record. Press any other function key to cancel the delete.

\end{description}


\subsubsection{70 DELETE OK}
\label{\detokenize{messages:id3}}\begin{description}
\item[{Module}] \leavevmode
VIR4008-4009, VIR4011, VIR4015-4016, VIR4021

\item[{Meaning}] \leavevmode
The requested record has been successfully deleted from the file.

\item[{Action}] \leavevmode
None.

\end{description}


\subsubsection{71 THE RESOURCE ALREADY EXISTS}
\label{\detokenize{messages:the-resource-already-exists}}\begin{description}
\item[{Module}] \leavevmode
VIR4010

\item[{Meaning}] \leavevmode
You are attempting to add a new resource but a resource of the same name already exists.

\item[{Action}] \leavevmode
Choose a new resource name or edit the existing resource.

\end{description}


\subsubsection{72 ENTER THE DESCRIPTION OF THE RESOURCE}
\label{\detokenize{messages:enter-the-description-of-the-resource}}\begin{description}
\item[{Module}] \leavevmode
VIR4010, VIR4011

\item[{Meaning}] \leavevmode
The resource description field is blank.

\item[{Action}] \leavevmode
Enter a value in the resource description field.

\end{description}


\subsubsection{73 THE CURSOR POSITION IS INVALID}
\label{\detokenize{messages:the-cursor-position-is-invalid}}\begin{description}
\item[{Module}] \leavevmode
VIR4009, VIR4011, VIR4013, VIR4016, VIR4022

\item[{Meaning}] \leavevmode
The function you requested requires the cursor to be placed on a record.

\item[{Action}] \leavevmode
Position the cursor on the record you wish to operate upon.

\end{description}


\subsubsection{74 THE RESOURCE DOES NOT EXIST}
\label{\detokenize{messages:the-resource-does-not-exist}}\begin{description}
\item[{Module}] \leavevmode
VIR4011

\item[{Meaning}] \leavevmode
The resource which you are attempting to update no longer exists.

\item[{Action}] \leavevmode
None.

\end{description}


\subsubsection{75 MODIFIED BY ANOTHER USER}
\label{\detokenize{messages:modified-by-another-user}}\begin{description}
\item[{Module}] \leavevmode
VIR4004, VIR4010-4011, VIR4015-4016, VIR4021-4023

\item[{Meaning}] \leavevmode
The record you are attempting to update has been updated by another user.

\item[{Action}] \leavevmode
Return to the previous menu and display the record again.

\end{description}


\subsubsection{76 ENTER THE NAME OF THE PROFILE}
\label{\detokenize{messages:enter-the-name-of-the-profile}}\begin{description}
\item[{Module}] \leavevmode
VIR4006, VIR4008

\item[{Meaning}] \leavevmode
The profile name is blank.

\item[{Action}] \leavevmode
Enter a value in the profile name field.

\end{description}


\subsubsection{77 ENTER THE DESCRIPTION OF THE PROFILE}
\label{\detokenize{messages:enter-the-description-of-the-profile}}\begin{description}
\item[{Module}] \leavevmode
VIR4006, VIR4008

\item[{Meaning}] \leavevmode
The profile description is blank.

\item[{Action}] \leavevmode
Enter a value in the profile description field.

\end{description}


\subsubsection{78 THE PROFILE ALREADY EXISTS}
\label{\detokenize{messages:the-profile-already-exists}}\begin{description}
\item[{Module}] \leavevmode
VIR4006, VIR4008

\item[{Meaning}] \leavevmode
You are attempting to add a new profile record but a profile of the same name already exists.

\item[{Action}] \leavevmode
Enter a different name in the profile name field.

\end{description}


\subsubsection{81 REQUESTED ELEMENT DOES NOT EXIST}
\label{\detokenize{messages:requested-element-does-not-exist}}\begin{description}
\item[{Module}] \leavevmode
VIR1002, VIR4006, VIR4008

\item[{Meaning}] \leavevmode
The record being added or updated references an element which does not exist.

\item[{Action}] \leavevmode
Correct the name of the referenced element.

\end{description}


\subsubsection{82 REQUESTED ELEMENT DOES NOT EXIST IN DEPARTMENT}
\label{\detokenize{messages:requested-element-does-not-exist-in-department}}\begin{description}
\item[{Module}] \leavevmode
VIR4006, VIR4008

\item[{Meaning}] \leavevmode
The record being added or updated references an element which belongs to a different department.

\item[{Action}] \leavevmode
Correct the name of the referenced element.

\end{description}


\subsubsection{83 YOU ARE ON THE FIRST PAGE}
\label{\detokenize{messages:id4}}\begin{description}
\item[{Module}] \leavevmode
VIR4004, VIR4006, VIR4008, VIR4012-4015

\item[{Meaning}] \leavevmode
You tried to scroll back to the previous page of data but you are already on the first page.

\item[{Action}] \leavevmode
None.

\end{description}


\subsubsection{84 YOU ARE ON THE LAST PAGE}
\label{\detokenize{messages:id5}}\begin{description}
\item[{Module}] \leavevmode
VIR4004, VIR4006, VIR4008, VIR4012-4015

\item[{Meaning}] \leavevmode
You tried to scroll forward to the next page of data but you are already on the last page.

\item[{Action}] \leavevmode
None.

\end{description}


\subsubsection{85 THE PROFILE DOES NOT EXIST}
\label{\detokenize{messages:the-profile-does-not-exist}}\begin{description}
\item[{Module}] \leavevmode
VIR4004, VIR4008, VIR4009, VIR4014-4015

\item[{Meaning}] \leavevmode
The profile being updated or deleted does not exist.

\item[{Action}] \leavevmode
None.

\end{description}


\subsubsection{86 THE PROFILE IS USED IN ANOTHER DEPARTMENT}
\label{\detokenize{messages:the-profile-is-used-in-another-department}}\begin{description}
\item[{Module}] \leavevmode
VIR4008

\item[{Meaning}] \leavevmode
The profile being updated does not exist.

\item[{Action}] \leavevmode
None.

\end{description}


\subsubsection{87 THE PROFILE IS EMPLOYED BY A USER}
\label{\detokenize{messages:the-profile-is-employed-by-a-user}}\begin{description}
\item[{Module}] \leavevmode
VIR4008, VIR4009

\item[{Meaning}] \leavevmode
The profile being updated or deleted has been modified by another user since it was last displayed on your terminal.

\item[{Action}] \leavevmode
Display the profile again and re-enter the modifications.

\end{description}


\subsubsection{88 ENTER THE NAME OF THE USER}
\label{\detokenize{messages:enter-the-name-of-the-user}}\begin{description}
\item[{Module}] \leavevmode
VIR4014, VIR4015, VIR4022

\item[{Meaning}] \leavevmode
You are attempting to create or update a user or administrator record, but the user name field is blank.

\item[{Action}] \leavevmode
Enter a valid user name.

\end{description}


\subsubsection{89 ENTER THE DESCRIPTION OF THE USER}
\label{\detokenize{messages:enter-the-description-of-the-user}}
Module
VIR4014, VIR4015
Meaning
You are attempting to create or update a user record, but the user description field is blank.
Action
Enter a value in the description field.


\subsubsection{90 USER ALREADY EXISTS}
\label{\detokenize{messages:user-already-exists}}\begin{description}
\item[{Module}] \leavevmode
VIR4014, VIR4015

\item[{Meaning}] \leavevmode
You are attempting to create or copy a user record, but the a user of the same name already exists.

\item[{Action}] \leavevmode
Enter a different value in the user name field.

\end{description}


\subsubsection{91 ERROR LOGICAL RECORD (name) NOT FOUND}
\label{\detokenize{messages:error-logical-record-name-not-found}}\begin{description}
\item[{Module}] \leavevmode
VIR0021A, VIR0021R, VIR4004, VIR4014, VIR4015

\item[{Meaning}] \leavevmode
Either you are attempting to call an external server name which does not exist, or you are attempting to update or delete a user record name which does not exist in the VIRARBO file.

\item[{Action}] \leavevmode
Specify a valid name.

\end{description}


\subsubsection{92 THE PROFILE DOES NOT APPEAR IN THE DEPARTMENT}
\label{\detokenize{messages:the-profile-does-not-appear-in-the-department}}\begin{description}
\item[{Module}] \leavevmode
VIR4004, VIR4008, VIR4009, VIR4013, VIR4015

\item[{Meaning}] \leavevmode
The profile being updated or deleted does not belong to the same department as the user.

\item[{Action}] \leavevmode
Ensure that the profile matches the user’s department.

\end{description}


\subsubsection{93 AUTHORISED PROFILE LIMIT EXCEEDED}
\label{\detokenize{messages:authorised-profile-limit-exceeded}}\begin{description}
\item[{Module}] \leavevmode
VIR4004, VIR4015

\item[{Meaning}] \leavevmode
An internal table overflow has occurred.

\item[{Action}] \leavevmode
Contact technical support.

\end{description}


\subsubsection{94 USER DOES NOT EXIST}
\label{\detokenize{messages:user-does-not-exist}}\begin{description}
\item[{Module}] \leavevmode
VIR4015, VIR4016, VIR4021-4023

\item[{Meaning}] \leavevmode
The user or administrator being updated or deleted does not exist.

\item[{Action}] \leavevmode
None.

\end{description}


\subsubsection{95 USER DOES NOT APPEAR IN THE DEPARTMENT}
\label{\detokenize{messages:user-does-not-appear-in-the-department}}\begin{description}
\item[{Module}] \leavevmode
VIR4015, VIR4016, VIR4021, VIR4023

\item[{Meaning}] \leavevmode
The user being viewed, updated, deleted, or being used as a model, does not belong to the same department as the administrator.

\item[{Action}] \leavevmode
Only an administrator in the same department as the user can perform the requested operation.

\end{description}


\subsubsection{96 FUNCTION RESERVED FOR HEAD OF THE DEPARTMENT}
\label{\detokenize{messages:function-reserved-for-head-of-the-department}}\begin{description}
\item[{Module}] \leavevmode
VIR4008, VIR4009, VIR4011, VIR4015, VIR4016

\item[{Meaning}] \leavevmode
The user being copied, updated, or deleted does not belong to the same department as the administrator.

\item[{Action}] \leavevmode
Only an administrator in the same department as the user can perform the requested operation.

\end{description}


\subsubsection{97 ENTER Y OR N}
\label{\detokenize{messages:enter-y-or-n}}\begin{description}
\item[{Module}] \leavevmode
VIR4014, VIR4015

\item[{Meaning}] \leavevmode
You are attempting to create or update a user record, and the assistant field must contain either O or N.

\item[{Action}] \leavevmode
Enter O (yes) or N (no) in the indicated field.

\end{description}


\subsubsection{98 FIELD RESERVED FOR RESPONSIBLE OF DEPARTMENT}
\label{\detokenize{messages:field-reserved-for-responsible-of-department}}\begin{description}
\item[{Module}] \leavevmode
VIR4014, VIR4015

\item[{Meaning}] \leavevmode
You are attempting to create or update a user record, and the assistant field contains O (yes) but you are not the head of department.

\item[{Action}] \leavevmode
Only the head of department may put O (yes) in the indicated field.

\end{description}


\subsubsection{99 DELETION OF DEPARTMENT RESPONSIBLE FORBIDDEN}
\label{\detokenize{messages:deletion-of-department-responsible-forbidden}}\begin{description}
\item[{Module}] \leavevmode
VIR4014, VIR4015

\item[{Meaning}] \leavevmode
You are attempting to delete a user record which is marked as head of department.

\item[{Action}] \leavevmode
Update the department record to assign another user as head of department first.

\end{description}


\subsubsection{100 DELETION OF DEPUTY FORBIDDEN}
\label{\detokenize{messages:deletion-of-deputy-forbidden}}\begin{description}
\item[{Module}] \leavevmode
VIR4015

\item[{Meaning}] \leavevmode
You are attempting to delete a user record which is marked as an assistant.

\item[{Action}] \leavevmode
Set the assistant field to N (no) first.

\end{description}


\subsubsection{101 USER IS ALREADY ADMINISTRATOR}
\label{\detokenize{messages:user-is-already-administrator}}\begin{description}
\item[{Module}] \leavevmode
VIR4022

\item[{Meaning}] \leavevmode
You are attempting to designate a user as an administrator, but the user is already an administrator.

\item[{Action}] \leavevmode
None.

\end{description}


\subsubsection{103 REPRODUCTION OK}
\label{\detokenize{messages:reproduction-ok}}\begin{description}
\item[{Module}] \leavevmode
VIR1001, VIR1002, VIR1005, VIR4008, VIR4015

\item[{Meaning}] \leavevmode
The record has been successfully copied.

\item[{Action}] \leavevmode
None.

\end{description}


\subsubsection{104 REPRODUCTION OF DEPUTY FORBIDDEN}
\label{\detokenize{messages:reproduction-of-deputy-forbidden}}\begin{description}
\item[{Module}] \leavevmode
VIR4015

\item[{Meaning}] \leavevmode
You cannot copy a user record which is marked as an assistant.

\item[{Action}] \leavevmode
Set the assistant field to N (no) first, or choose another user to copy.

\end{description}


\subsubsection{105 USER SIGN-ON UNKNOWN}
\label{\detokenize{messages:user-sign-on-unknown}}\begin{description}
\item[{Module}] \leavevmode
VIR4000

\item[{Meaning}] \leavevmode
You attempted to access the security administration panels but you have not signed on to VIRTEL.

\item[{Action}] \leavevmode
Log in with a valid VIRTEL userid.

\end{description}


\subsubsection{106 SIGN-ON RESTORED OK}
\label{\detokenize{messages:sign-on-restored-ok}}\begin{description}
\item[{Module}] \leavevmode
VIR4023

\item[{Meaning}] \leavevmode
You have successfully unblocked a user which was locked out.

\item[{Action}] \leavevmode
None.

\end{description}


\subsubsection{107 CALL REJECTED BY THE NETWORK}
\label{\detokenize{messages:call-rejected-by-the-network}}\begin{description}
\item[{Module}] \leavevmode
VIR0021A, VIR0021R, VIR0025, VIR0025T

\item[{Meaning}] \leavevmode
An attempt to make an outbound X25 call was unsuccessful.

\item[{Action}] \leavevmode
Check for messages in the VIRTEL log which indicate the cause of the error.

\end{description}


\subsubsection{109 PREVIOUS CONNECTION: dd/mm/yy hh:mm:ss termid}
\label{\detokenize{messages:previous-connection-dd-mm-yy-hh-mm-ss-termid}}\begin{description}
\item[{Module}] \leavevmode
VIR00081

\item[{Meaning}] \leavevmode
This message indicates the date, time, and terminal name of the previous logon by your userid using VIRTEL internal security.

\item[{Action}] \leavevmode
None.

\end{description}


\subsubsection{110 THE FILE filename UNKNOWN}
\label{\detokenize{messages:the-file-filename-unknown}}\begin{description}
\item[{Module}] \leavevmode
VIR1001, VIR1002

\item[{Meaning}] \leavevmode
The file filename cannot be found.

\item[{Action}] \leavevmode
Ensure that the file is correctly referenced in the VIRTCT, and that there is a DD statement for the indicated file.

\end{description}


\subsubsection{111 THE FILE filename IS CLOSED}
\label{\detokenize{messages:the-file-filename-is-closed}}\begin{description}
\item[{Module}] \leavevmode
VIR1001, VIR1002

\item[{Meaning}] \leavevmode
VIRTEL cannot open the file filename.

\item[{Action}] \leavevmode
Check the VIRTEL log for messages relating to the indicated file.

\end{description}


\subsubsection{112 I/OERROR ON FILE filename}
\label{\detokenize{messages:i-oerror-on-file-filename}}\begin{description}
\item[{Module}] \leavevmode
VIR0043, VIR0043H, VIR1001, VIR1002

\item[{Meaning}] \leavevmode
There has been an error accessing the file filename.

\item[{Action}] \leavevmode
Check the VIRTEL log for messages relating to the indicated file.

\end{description}


\subsubsection{113 THIS IS NOT A MINITEL NATIVE NODE}
\label{\detokenize{messages:this-is-not-a-minitel-native-node}}\begin{description}
\item[{Module}] \leavevmode
VIR1003

\item[{Meaning}] \leavevmode
The requested record cannot be displayed because it is a sub-server record.

\item[{Action}] \leavevmode
Choose a native node record.

\end{description}


\subsubsection{114 ALL=END}
\label{\detokenize{messages:all-end}}\begin{description}
\item[{Module}] \leavevmode
VIR1003, VIR1004

\item[{Meaning}] \leavevmode
This is an information message indicating that all function keys perform the END function.

\item[{Action}] \leavevmode
None.

\end{description}


\subsubsection{115 PLEASE SUPPLY A VALID NODE NAME}
\label{\detokenize{messages:please-supply-a-valid-node-name}}\begin{description}
\item[{Module}] \leavevmode
VIR2019

\item[{Meaning}] \leavevmode
The network management application requires a non-blank node name.

\item[{Action}] \leavevmode
Enter the name of a valid VTAM node.

\end{description}


\subsubsection{116 THE NODE NAME IS INVALID}
\label{\detokenize{messages:the-node-name-is-invalid}}\begin{description}
\item[{Module}] \leavevmode
VIR2019

\item[{Meaning}] \leavevmode
The node name specified is not the correct type for the network management application.

\item[{Action}] \leavevmode
Enter the name of a valid VTAM node.

\end{description}


\subsubsection{117 CONFIRM CANCELLATION OF THE PASSWORD}
\label{\detokenize{messages:confirm-cancellation-of-the-password}}\begin{description}
\item[{Module}] \leavevmode
VIR0041

\item[{Meaning}] \leavevmode
You pressed PF4 to request that a VIRTELPC password be removed.

\item[{Action}] \leavevmode
Press PF4 again to remove the password, or press any other key to cancel the operation.

\end{description}


\subsubsection{118 THIS FILE IS ALREADY IN THE TARGET DIRECTORY}
\label{\detokenize{messages:this-file-is-already-in-the-target-directory}}\begin{description}
\item[{Module}] \leavevmode
VIR0043, VIR0043H, VIR0052

\item[{Meaning}] \leavevmode
The file you are attempting to copy already exists in the target directory.

\item[{Action}] \leavevmode
Delete the file from the target directory and try again.

\end{description}


\subsubsection{119 COPY COMPLETED}
\label{\detokenize{messages:copy-completed}}\begin{description}
\item[{Module}] \leavevmode
VIR0043, VIR0043H

\item[{Meaning}] \leavevmode
The file you requested has been successfully copied.

\item[{Action}] \leavevmode
None.

\end{description}


\subsubsection{120 THE RECORDED STATUS HAS CHANGED}
\label{\detokenize{messages:the-recorded-status-has-changed}}\begin{description}
\item[{Module}] \leavevmode
VIR0043, VIR0043H

\item[{Meaning}] \leavevmode
The status of the file has been successfully toggled from “Public” to “Private” or vice versa.

\item[{Action}] \leavevmode
None.

\end{description}


\subsubsection{121 FILE ERROR PLEASE SEE THE ADMINISTRATOR}
\label{\detokenize{messages:file-error-please-see-the-administrator}}\begin{description}
\item[{Module}] \leavevmode
VIR0033, VIR0043, VIR0043H, VIR0052

\item[{Meaning}] \leavevmode
An I/O error has occurred on:
- the VIRCMP3 file during compression management;
- a user directory or VIRARBO file during directory management;
- the VIRSWAP file during page capture management.

\item[{Action}] \leavevmode
Check the VIRTEL log for error messages indicating the cause of the error.

\end{description}


\subsubsection{122 FILE TRANSFER IN PROGRESS}
\label{\detokenize{messages:file-transfer-in-progress}}\begin{description}
\item[{Module}] \leavevmode
VIR0043, VIR0043H

\item[{Meaning}] \leavevmode
A file is being transferred between VIRTEL and VIRTELPC.

\item[{Action}] \leavevmode
None.

\end{description}


\subsubsection{123 CONFIRM COPY TO MEMORY}
\label{\detokenize{messages:confirm-copy-to-memory}}\begin{description}
\item[{Module}] \leavevmode
VIR0043, VIR0043H

\item[{Meaning}] \leavevmode
You pressed PF6 to request the copy of a file to the stack.

\item[{Action}] \leavevmode
Press PF6 again to copy the file, or press any other key to cancel the operation.

\end{description}


\subsubsection{124 CONFIRM THE COPY OF THIS PAGE}
\label{\detokenize{messages:confirm-the-copy-of-this-page}}\begin{description}
\item[{Module}] \leavevmode
VIR0043, VIR0043H

\item[{Meaning}] \leavevmode
You pressed PF1 to request that all files should be downloaded.

\item[{Action}] \leavevmode
Press PF1 again to confirm, or press any other key to cancel the operation.

\end{description}


\subsubsection{125 IMPOSSIBLE OPERATION}
\label{\detokenize{messages:impossible-operation}}\begin{description}
\item[{Module}] \leavevmode
VIR0022A

\item[{Meaning}] \leavevmode
Either you pressed PF2 on the VIRTEL Multi-Session sub-menu to request that the application be promoted to the main menu, but the main menu is full; or you pressed PF2 to request that an application should be demoted to the sub-menu, but the application cannot be removed from the main menu because you currently have a active session with this application.

\item[{Action}] \leavevmode
None.

\end{description}


\subsubsection{126 NAME OF PAGE TO LOAD :}
\label{\detokenize{messages:name-of-page-to-load}}\begin{description}
\item[{Module}] \leavevmode
VIR1010

\item[{Meaning}] \leavevmode
VIRTEL is requesting the name of a videotex page to be uploaded to the VIRARBO file.

\item[{Action}] \leavevmode
Enter the name of the page to be uploaded.

\end{description}


\subsubsection{127 THE PAGE ALREADY EXISTS,REPLACE (Y / N)}
\label{\detokenize{messages:the-page-already-exists-replace-y-n}}\begin{description}
\item[{Module}] \leavevmode
VIR1010

\item[{Meaning}] \leavevmode
During a videotex page upload operation, VIRTEL has determined that a page of the same name already exists in the VIRARBO file.

\item[{Action}] \leavevmode
Enter Y to overwrite the page in the VIRARBO file, or N to cancel the operation.

\end{description}


\subsubsection{128 SELECT THE PAGE THEN PRESS SEND}
\label{\detokenize{messages:select-the-page-then-press-send}}\begin{description}
\item[{Module}] \leavevmode
VIR1010

\item[{Meaning}] \leavevmode
During a videotex page upload operation, VIRTEL is ready to receive the page to be uploaded.

\item[{Action}] \leavevmode
Select the page according to the procedure provided by your page composition software, then press the “SEND” (or “ENVOI”) key.

\end{description}


\subsubsection{129 CREATION COMPLETED}
\label{\detokenize{messages:creation-completed}}\begin{description}
\item[{Module}] \leavevmode
VIR1010

\item[{Meaning}] \leavevmode
A new videotex page has been successfully uploaded to the VIRARBO file.

\item[{Action}] \leavevmode
None.

\end{description}


\subsubsection{130 UPDATE COMPLETED}
\label{\detokenize{messages:update-completed}}\begin{description}
\item[{Module}] \leavevmode
VIR1010

\item[{Meaning}] \leavevmode
A replacement videotex page has been successfully uploaded to the VIRARBO file.

\item[{Action}] \leavevmode
None.

\end{description}


\subsubsection{131 PRESS PF1 TO CONFIRM THE UPDATE}
\label{\detokenize{messages:press-pf1-to-confirm-the-update}}\begin{description}
\item[{Module}] \leavevmode
VIR0044, VIR0046

\item[{Meaning}] \leavevmode
As a result of an earlier modification to the definition of a VIRTEL transaction, the associated entry point must now be updated.

\item[{Action}] \leavevmode
Press PF1 to confirm the update.

\end{description}


\subsubsection{132 PLEASE CONFIRM YOUR PASSWORD}
\label{\detokenize{messages:please-confirm-your-password}}\begin{description}
\item[{Module}] \leavevmode
VIR0020, VIR0020A, VIR0020L, VIR0020M, VIR0020P

\item[{Meaning}] \leavevmode
VIRTEL has temporarily locked your terminal because of lack of activity.

\item[{Action}] \leavevmode
Enter your password again to reactivate your terminal.

\end{description}


\subsubsection{133 UNITS SPENT: nnnnn.nn}
\label{\detokenize{messages:units-spent-nnnnn-nn}}\begin{description}
\item[{Module}] \leavevmode
VIR0021A, VIR0025

\item[{Meaning}] \leavevmode
This message shows the number of units consumed by a call to an external server.

\item[{Action}] \leavevmode
None.

\end{description}


\subsubsection{134 POSITION IN QUEUE: nnnn}
\label{\detokenize{messages:position-in-queue-nnnn}}\begin{description}
\item[{Module}] \leavevmode
VIR0025, VIR0025T

\item[{Meaning}] \leavevmode
This message shows your position in the queue for a call to an external server.

\item[{Action}] \leavevmode
None.

\end{description}


\subsubsection{135 ACTIVATION WAS REQUESTED}
\label{\detokenize{messages:activation-was-requested}}\begin{description}
\item[{Module}] \leavevmode
VIR0041A

\item[{Meaning}] \leavevmode
You pressed PF4 in the e-mail correspondent management sub-application to request activation of a correspondent.

\item[{Action}] \leavevmode
None.

\end{description}


\subsubsection{136 DISABLE WAS DONE}
\label{\detokenize{messages:disable-was-done}}\begin{description}
\item[{Module}] \leavevmode
VIR0041A

\item[{Meaning}] \leavevmode
You pressed PF5 in the e-mail correspondent management sub-application to request deactivation of a correspondent.

\item[{Action}] \leavevmode
None.

\end{description}


\subsubsection{137 Should contain the ‘@’ sign}
\label{\detokenize{messages:should-contain-the-sign}}\begin{description}
\item[{Module}] \leavevmode
VIR0041A

\item[{Meaning}] \leavevmode
This message is issued by the e-mail correspondent management sub-application to allow you to verify how your terminal displays the “@” sign. In some countries, the “@” sign may appear differently on a 3270 terminal (for example, “à”).

\item[{Action}] \leavevmode
When you enter an e-mail address, ensure that you use the same symbol as displayed in this message.

\end{description}


\subsubsection{138 Sample command: \&\textbar{}W}
\label{\detokenize{messages:sample-command-w}}\begin{description}
\item[{Module}] \leavevmode
VIR0045

\item[{Meaning}] \leavevmode
This message is issued by the entry point and transaction management sub-application to allow you to verify how your terminal displays the “\&” and “\textbar{}” signs. In some countries, these characters may appear differently on a 3270 terminal (for example, “!” instead of “\textbar{}”).

\item[{Action}] \leavevmode
When you enter script commands in the “TIOA at logon” and “TIOA at logoff” fields, ensure that you use the same symbols as displayed in this message.

\end{description}


\section{Web Access Messages}
\label{\detokenize{messages:web-access-messages}}
\begin{sphinxadmonition}{note}{Note:}
These messages are issued by VIRTEL Web Access scripts and are displayed as alerts in the user’s browser window.
\end{sphinxadmonition}


\subsection{Messages for js01.js}
\label{\detokenize{messages:messages-for-js01-js}}
\sphinxstyleemphasis{Cannot open pop-up window for print data. You may need to disable your pop- up blocker}
\begin{description}
\item[{Module}] \leavevmode
js01.js

\item[{Meaning}] \leavevmode
VIRTEL Web Access print function needs to open a new browser window to display print data, but the function is disallowed by the browser settings.

\item[{Action}] \leavevmode
Adjust your browser settings to allow VIRTEL scripts to open pop-up windows. For Internet Explorer, add the VIRTEL host to the trusted zone. For Firefox, add the VIRTEL host to the exceptions list in Tools \textendash{} Options \textendash{} Content \textendash{} Block pop-up windows.

\end{description}

\sphinxstyleemphasis{Cannot open pop-up window for settings. You may need to disable your pop-up blocker}
\begin{description}
\item[{Module}] \leavevmode
js01.js

\item[{Meaning}] \leavevmode
The VIRTEL Web Access Settings menu cannot be displayed because the browser settings do not permit scripts to open new windows.

\item[{Action}] \leavevmode
Same as previous message.

\end{description}


\subsection{Messages for rsa.js}
\label{\detokenize{messages:messages-for-rsa-js}}
\sphinxstyleemphasis{Message too long for RSA}
\begin{description}
\item[{Module}] \leavevmode
rsa.js

\item[{Meaning}] \leavevmode
An anomaly has been detected by the encryption script.

\item[{Action}] \leavevmode
Contact technical support.

\end{description}

\sphinxstyleemphasis{Invalid RSA public key}
\begin{description}
\item[{Module}] \leavevmode
rsa.js

\item[{Meaning}] \leavevmode
The RSA key supplied by VIRTEL is not valid.

\item[{Action}] \leavevmode
Contact technical support.

\end{description}


\subsection{Messages for vircrypt.js}
\label{\detokenize{messages:messages-for-vircrypt-js}}
\sphinxstyleemphasis{vircrypt.js: VIRTEL CRYPT parameters are missing}
\begin{description}
\item[{Module}] \leavevmode
vircrypt.js

\item[{Meaning}] \leavevmode
VIRTEL did not supply the encryption parameters requested by the page template.

\item[{Action}] \leavevmode
Check that there is a CRYPTn parameter in the VIRTCT whose name matches the name requested by the page template. For the WEB2AJAX.htm template, there must be a CRYPTn parameter whose name subparameter is CRYPT3270.

\end{description}

\sphinxstyleemphasis{vircrypt.js: Unable to obtain public key from VIRTEL}
\begin{description}
\item[{Module}] \leavevmode
vircrypt.js

\item[{Meaning}] \leavevmode
VIRTEL did not supply the RSA public exponent or modulus requested by the page template.

\item[{Action}] \leavevmode
Check the JESMSGLG for the VIRTEL started task to determine why the public key request failed.

\end{description}

\sphinxstyleemphasis{vircrypt.js: Unsupported encryption algorithm: xxx}
\begin{description}
\item[{Module}] \leavevmode
vircrypt.js

\item[{Meaning}] \leavevmode
The symmetric encryption algorithm specified in the CRYPTn parameter of the VIRTCT is not supported by this version of the script.

\item[{Action}] \leavevmode
Clear the browser cache to ensure that you are using the latest version of the script. If the problem persists, contact technical support.

\end{description}

\sphinxstyleemphasis{vircrypt.js: Unsupported encoding\textbar{}chaining\textbar{}padding method: xxx}
\begin{description}
\item[{Module}] \leavevmode
vircrypt.js

\item[{Meaning}] \leavevmode
The encoding method, chaining method, or padding method specified in the CRYPTn parameter of the VIRTCT is not supported by this version of the script.

\item[{Action}] \leavevmode
Clear the browser cache to ensure that you are using the latest version of the script. If the problem persists, contact technical support.

\end{description}

\sphinxstyleemphasis{vircrypt.js: Unsupported PKA algorithm: xxx}
\begin{description}
\item[{Module}] \leavevmode
vircrypt.js

\item[{Meaning}] \leavevmode
The asymmetric encryption algorithm specified in the CRYPTn parameter of the VIRTCT is not supported by this version of the script.

\item[{Action}] \leavevmode
Clear the browser cache to ensure that you are using the latest version of the script. If the problem persists, contact technical support.

\end{description}


\section{VIRTEL console messages}
\label{\detokenize{messages:virtel-console-messages}}

\subsection{Messages VIR000xx}
\label{\detokenize{messages:messages-vir000xx}}

\subsubsection{VIR0000I xxxx Date: Mon, 07 Jun 2004 15:20:23 GMT}
\label{\detokenize{messages:vir0000i-xxxx-date-mon-07-jun-2004-15-20-23-gmt}}\begin{description}
\item[{Module}] \leavevmode
VIR0000

\item[{Meaning}] \leavevmode
These messages indicate the current time from the point of view of the HTTP and SMTP server components of VIRTEL. The times are calculated from the system TOD CLOCK, adjusted by the GMT parameter of the VIRTCT.

\item[{Action}] \leavevmode
None.

\end{description}


\subsubsection{VIR0001W VSAM ERROR ON FILE filename : yy yy (HEX) REQ : zz KEY : keyvalue}
\label{\detokenize{messages:vir0001w-vsam-error-on-file-filename-yy-yy-hex-req-zz-key-keyvalue}}\begin{description}
\item[{Module}] \leavevmode
VIR0001

\item[{Meaning}] \leavevmode
Unexpected VSAM error occurred during access of file filename. yy yy is the VSAM return code, zz is the request being processed, keyvalue is the record access key (16 characters).

\item[{Action}] \leavevmode
Verify the values of the return codes in the appropriate IBM documentation. VSAM error codes are documented in the chapter entitled VSAM Macro Return and Reason Codes in the IBM manual DFSMS Macro Instructions for Data Sets.

\end{description}


\subsubsection{VIR0002W TERM=termid, REQUEST=qq, RTNCD=cc, FDBK2=dd, SENSE=ssss ssss xxxxxxxxx}
\label{\detokenize{messages:vir0002w-term-termid-request-qq-rtncd-cc-fdbk2-dd-sense-ssss-ssss-xxxxxxxxx}}\begin{description}
\item[{Module}] \leavevmode
VIR0009

\item[{Meaning}] \leavevmode
Unexpected VTAM error during dialogue with a terminal: termid is the name of the terminal, qq is the type of VTAM request which encountered the error, cc is the VTAM return code (hexadecimal), dd is the VTAM feedback code (hexadecimal), ssss ssss is the sense code, and xxxxxxx is VIRTEL’s interpretation of the sense code.

\item[{Action}] \leavevmode
Verify the values of the returned sense codes in the appropriate IBM documentation. VTAM codes are documented in the IBM VTAM Messages and Codes or Communications Server IP and SNA Codes manuals.

\end{description}


\subsubsection{VIR0003I xxxxxxxx ENDED}
\label{\detokenize{messages:vir0003i-xxxxxxxx-ended}}\begin{description}
\item[{Module}] \leavevmode
VIR0000

\item[{Meaning}] \leavevmode
VIRTEL termination is complete.

\item[{Action}] \leavevmode
None.

\end{description}


\subsubsection{VIR0004I CLEANUP : luname/ applname}
\label{\detokenize{messages:vir0004i-cleanup-luname-applname}}\begin{description}
\item[{Module}] \leavevmode
VIR0009

\item[{Meaning}] \leavevmode
The session was interrupted between the relay associated with the terminal luname and the application applname.

\item[{Action}] \leavevmode
None.

\end{description}


\subsubsection{VIR0005W UNABLE TO ACTIVATE relayname (termid) ERROR: xx000000}
\label{\detokenize{messages:vir0005w-unable-to-activate-relayname-termid-error-xx000000}}\begin{description}
\item[{Module}] \leavevmode
VIR0000

\item[{Meaning}] \leavevmode
The relay relayname associated with terminal termid cannot be activated. xx is the ACB error code. Commonly encountered codes are:
\begin{itemize}
\item {} 
58 : ACB already in use by another application

\item {} 
5A : ACB not activated in VTAM

\end{itemize}

For the meaning of other codes, see the IBM VTAM Programming manual.

\item[{Action}] \leavevmode
Verify that the VTAM node containing the relay relayname has been activated, verify that the relay is not already activated for an other terminal or application, and that the terminal is correctly defined in VIRTEL.

\end{description}


\subsubsection{VIR0006I DETACHING xxxxxxxx SUBTASK}
\label{\detokenize{messages:vir0006i-detaching-xxxxxxxx-subtask}}\begin{description}
\item[{Module}] \leavevmode
VIR0000

\item[{Meaning}] \leavevmode
VIRTEL is detaching subtask xxxxxxxx before stopping the system.

\item[{Action}] \leavevmode
None.

\end{description}


\subsubsection{VIR0007I luname/applname BIND FAILED}
\label{\detokenize{messages:vir0007i-luname-applname-bind-failed}}\begin{description}
\item[{Module}] \leavevmode
VIR0009

\item[{Meaning}] \leavevmode
A connection request for the terminal luname from the application applname was rejected due to insufficient storage to establish the session, or because a session already exists with this application.

\item[{Action}] \leavevmode
Verify the memory allocation to VIRTEL.

\end{description}


\subsubsection{VIR0008S INVALID RPL}
\label{\detokenize{messages:vir0008s-invalid-rpl}}\begin{description}
\item[{Module}] \leavevmode
VIR0009

\item[{Meaning}] \leavevmode
VTAM has found an invalid RPL. This message is followed by an ABEND U009.

\item[{Action}] \leavevmode
Contact technical support.

\end{description}


\subsubsection{VIR0009I xxxxxxx : SHUT DOWN IN PROGRESS}
\label{\detokenize{messages:vir0009i-xxxxxxx-shut-down-in-progress}}\begin{description}
\item[{Module}] \leavevmode
VIR0000

\item[{Meaning}] \leavevmode
VIRTEL is shutting down. xxxxxxxx represents the name of the VIRTEL primary ACB.

\item[{Action}] \leavevmode
None.

\end{description}


\subsection{Messages VIR001xx}
\label{\detokenize{messages:messages-vir001xx}}

\subsubsection{VIR0010I SUBPOOL SIZE = ssss K}
\label{\detokenize{messages:vir0010i-subpool-size-ssss-k}}\begin{description}
\item[{Module}] \leavevmode
VIR0000

\item[{Meaning}] \leavevmode
Indicates in kilobytes the quantity of memory available to VIRTEL for working storage after loading resident modules.

\item[{Action}] \leavevmode
None.

\end{description}


\subsubsection{VIR0011E INSUFFICIENT MEMORY FOR START UP}
\label{\detokenize{messages:vir0011e-insufficient-memory-for-start-up}}\begin{description}
\item[{Module}] \leavevmode
VIR0000

\item[{Meaning}] \leavevmode
VIRTEL has not acquired the minimum memory required to start and has abandoned its initialisation.

\item[{Action}] \leavevmode
Increase the value of the OSCORE parameter in the VIRTCT, and/or :
- in VSE, increase the partition size,
- in MVS, increase the REGION size.

\end{description}


\subsubsection{VIR0012W INSUFFICIENT MEMORY}
\label{\detokenize{messages:vir0012w-insufficient-memory}}\begin{description}
\item[{Module}] \leavevmode
VIR0000

\item[{Meaning}] \leavevmode
VIRTEL does not have sufficient memory to satisfy a request.

\item[{Action}] \leavevmode
Increase the memory allocated to VIRTEL, and/or increase the value of the OSCORE parameter of the VIRTCT. (See message VIR0011E).

\end{description}


\subsubsection{VIR0013W VTAM SHORT ON STORAGE}
\label{\detokenize{messages:vir0013w-vtam-short-on-storage}}\begin{description}
\item[{Module}] \leavevmode
VIR0009

\item[{Meaning}] \leavevmode
VTAM lacks sufficient memory to satisfy a request.

\item[{Action}] \leavevmode
Ensure that the VTAM definitions are correct.

\end{description}


\subsubsection{VIR0014S FREEMAIN FATAL ERROR}
\label{\detokenize{messages:vir0014s-freemain-fatal-error}}\begin{description}
\item[{Module}] \leavevmode
VIR0000

\item[{Meaning}] \leavevmode
An unexpected error occurred during a call to release memory in the subpool. The system will stop.

\item[{Action}] \leavevmode
Contact technical support.

\end{description}


\subsubsection{VIR0015S ABEND WITHIN VIRTEL KERNEL, TASK=taskname}
\label{\detokenize{messages:vir0015s-abend-within-virtel-kernel-task-taskname}}\begin{description}
\item[{Module}] \leavevmode
VIR0007

\item[{Meaning}] \leavevmode
An unexpected error occurred in the VIRTEL kernel. This message is preceded by message VIR0016W and followed by abend U0007.

\item[{Action}] \leavevmode
In order to deal with this problem, two dumps are printed during VIRTEL termination, one formatted, the other non formatted. Contact technical support.

\end{description}


\subsubsection{VIR0016W ABEND abendtype TERM=termid, PROG=progname, OFFSET=+xxxxx VIR0016W PSW = pppppppp pppppppp nnnncccc aaaaaaaa}
\label{\detokenize{messages:vir0016w-abend-abendtype-term-termid-prog-progname-offset-xxxxx-vir0016w-psw-pppppppp-pppppppp-nnnncccc-aaaaaaaa}}\begin{description}
\item[{Module}] \leavevmode
VIR0004

\item[{Meaning}] \leavevmode
An abend with code abendtype was produced in the program progname at offset xxxxx during a session from terminal termid. ppp…ppp represents the program status word at the time of the abend, nnnn is the instruction length code, cccc is the interruption code, and aaaaaaaa is the translation exception address. For VSE, only the program status word is displayed. A partial formatted DUMP is written to the SYSPRINT file.

\item[{Action}] \leavevmode
Contact technical support.

\end{description}


\subsubsection{VIR0017I LOGON luname/relayname DENIED STATE=ss}
\label{\detokenize{messages:vir0017i-logon-luname-relayname-denied-state-ss}}\begin{description}
\item[{Module}] \leavevmode
VIR0009

\item[{Meaning}] \leavevmode
The terminal luname tried to connect under the name relayname. This connection was refused for one of the following reasons :
\begin{itemize}
\item {} 
the name of the terminal is different to that associated with the relay to which it tried to connect,

\item {} 
the previous connection did not terminate,

\item {} 
a terminal was connected to VIRTEL over a logical channel unknown to VIRTEL.

\end{itemize}

\item[{Action}] \leavevmode
None.

\end{description}


\subsubsection{VIR0018I VIRTEL r.vv HAS THE FOLLOWING PTF(S) APPLIED}
\label{\detokenize{messages:vir0018i-virtel-r-vv-has-the-following-ptf-s-applied}}\begin{description}
\item[{Module}] \leavevmode
VIR0000

\item[{Meaning}] \leavevmode
Indicates the PTF numbers applied to VIRTEL.

\item[{Action}] \leavevmode
None.

\end{description}


\subsubsection{VIR0019I VIRTEL 4.nn HAS NO PTFS APPLIED}
\label{\detokenize{messages:vir0019i-virtel-4-nn-has-no-ptfs-applied}}\begin{description}
\item[{Module}] \leavevmode
VIR0000

\item[{Meaning}] \leavevmode
Indicates that no PTFs are applied to VIRTEL.

\item[{Action}] \leavevmode
None.

\end{description}


\subsection{Messages VIR002xx}
\label{\detokenize{messages:messages-vir002xx}}

\subsubsection{VIR0020E APPLICATION acbname IS ALREADY ACTIVE}
\label{\detokenize{messages:vir0020e-application-acbname-is-already-active}}\begin{description}
\item[{Module}] \leavevmode
VIR0000

\item[{Meaning}] \leavevmode
The application acbname referenced in the APPLID parameter of the VIRTCT is already in use.

\item[{Action}] \leavevmode
Check if another VIRTEL task is already active. Check that the correct APPLID was specified in the VIRTCT.

\end{description}


\subsubsection{VIR0021E ERROR xx OPENING MAIN TASK ACB acbname}
\label{\detokenize{messages:vir0021e-error-xx-opening-main-task-acb-acbname}}\begin{description}
\item[{Module}] \leavevmode
VIR0000

\item[{Meaning}] \leavevmode
VIRTEL has encountered an ACB error code xx when opening the VTAM ACB acbname.

\item[{Action}] \leavevmode
Check that the APPLID is correctly specified in the VIRTCT, and that the VTAM application node for VIRTEL has been activated. The following are commonly encountered ACB error codes:
\begin{itemize}
\item {} 
52 : VTAM is in the process of stopping.

\item {} 
54 : the VIRTEL application is not defined for VTAM.

\item {} 
56 : VIRTEL is defined, but not as an application.

\item {} 
5A : the VIRTEL application is not defined for VTAM.

\item {} 
5C : VTAM is inactive.

\end{itemize}

\end{description}

For values of the ACB error code, refer to the IBM VTAM Programming manual.


\subsubsection{VIR0022E ERROR xx BUILDING VSAM BUFFER POOL}
\label{\detokenize{messages:vir0022e-error-xx-building-vsam-buffer-pool}}\begin{description}
\item[{Module}] \leavevmode
VIR0000

\item[{Meaning}] \leavevmode
Error code xx has occurred when generating the VSAM buffer pools.

\item[{Action}] \leavevmode
Evaluate the following operands BUFDATA, BUFSIZE and STRNO of the VIRTCT. Return code X‘08’ indicates that the memory allocated to VIRTEL is insufficient.

\end{description}


\subsubsection{VIR0023E ERROR xx OPENING FILE filename}
\label{\detokenize{messages:vir0023e-error-xx-opening-file-filename}}\begin{description}
\item[{Module}] \leavevmode
VIR0000

\item[{Meaning}] \leavevmode
VSAM error xx occurred when opening file filename.

\item[{Action}] \leavevmode
Verify in the appropriate IBM documentation the meaning of the returned code. Note that when opening the VIRSWAP file, a return code of X‘5C’ (empty file) is considered normal. The VIRSWAP file is always empty at start-up this does not constitute an error.

\end{description}


\subsubsection{VIR0024I}
\label{\detokenize{messages:vir0024i}}
This message indicates the progress of VIRTEL start-up :

OPENING FILE filename
\begin{description}
\item[{Module}] \leavevmode
VIR0000

\item[{Meaning}] \leavevmode
VIRTEL is opening file filename.

\item[{Action}] \leavevmode
None.

\end{description}

CLOSING FILE filename
\begin{description}
\item[{Module}] \leavevmode
VIR0000

\item[{Meaning}] \leavevmode
VIRTEL is closing file filename.

\item[{Action}] \leavevmode
None.

\end{description}

ATTACHING SUBTASK
\begin{description}
\item[{Module}] \leavevmode
VIR0000

\item[{Meaning}] \leavevmode
VIRTEL is loading its subtasks.

\item[{Action}] \leavevmode
None.

\end{description}

READING VIRARBO
\begin{description}
\item[{Module}] \leavevmode
VIR0000

\item[{Meaning}] \leavevmode
VIRTEL is loading its configuration information stored in the VIRARBO file.

\item[{Action}] \leavevmode
None.

\end{description}

READING TYPES
\begin{description}
\item[{Module}] \leavevmode
VIR0000

\item[{Meaning}] \leavevmode
VIRTEL is loading the screen types for level 3 compression stored in the VIRCMP3 file.

\item[{Action}] \leavevmode
None.

\end{description}

STARTING CRYPTn
\begin{description}
\item[{Module}] \leavevmode
VIR0000

\item[{Meaning}] \leavevmode
VIRTEL is loading the interface modules for the encryption engine specified by the CRYPTn parameter in the VIRTCT.

\item[{Action}] \leavevmode
None.

\end{description}


\subsubsection{VIR0025E ERROR progname IS NOT FOR VIRTEL VERSION vvv}
\label{\detokenize{messages:vir0025e-error-progname-is-not-for-virtel-version-vvv}}\begin{description}
\item[{Module}] \leavevmode
VIR0000

\item[{Meaning}] \leavevmode
The VIRTCT progname was assembled using a version of VIRTEL which was not the same as the running version vvv.

\item[{Action}] \leavevmode
Re-assemble the VIRTCT using version vvv of the VIRTEL MACLIB.

\end{description}


\subsubsection{VIR0025E ERROR ON OVERRIDE: overname RETURN CODE: hhhh SUB CODE: ssssssss}
\label{\detokenize{messages:vir0025e-error-on-override-overname-return-code-hhhh-sub-code-ssssssss}}\begin{description}
\item[{Module}] \leavevmode
VIR0000

\item[{Meaning}] \leavevmode
An error occurred while processing a VTOVER macro in the VIRTCT.  overname is the label of the VTOVER macro       in error, hhhh is the return code (in hexadécimal) specified by the ERRORC parameter of the VTOVER macro, and ssssssss is a code (in hexadécimal) indicating the type of error.

\item[{Action}] \leavevmode
Correct the error, re-assemble the VIRTCT, and restart VIRTEL.

\end{description}


\subsubsection{VIR0026I COMPRESSION 3 IS NOT ACTIVE}
\label{\detokenize{messages:vir0026i-compression-3-is-not-active}}\begin{description}
\item[{Module}] \leavevmode
VIR0000

\item[{Meaning}] \leavevmode
The compression level 3 management system has not been activated.

\item[{Action}] \leavevmode
Verify the validity of the operands FCMP3 and COMPR3 of the VIRTCT.

\end{description}


\subsubsection{VIR0026W termid OPEN SESSION luname1 \textendash{} luname2}
\label{\detokenize{messages:vir0026w-termid-open-session-luname1-luname2}}\begin{description}
\item[{Module}] \leavevmode
VIR0000

\item[{Meaning}] \leavevmode
VIRTEL has stopped while a session with terminal luname is still active.

\item[{Action}] \leavevmode
None.

\end{description}


\subsubsection{VIR0026W linename HAS OPEN OBJECT socknum}
\label{\detokenize{messages:vir0026w-linename-has-open-object-socknum}}\begin{description}
\item[{Module}] \leavevmode
VIR0000

\item[{Meaning}] \leavevmode
VIRTEL has stopped while a connection to the MQSeries message queue manager is still active.

\item[{Action}] \leavevmode
None.

\end{description}


\subsubsection{VIR0026W termid IS DISCONNECTED DUE TO TIME-OUT}
\label{\detokenize{messages:vir0026w-termid-is-disconnected-due-to-time-out}}\begin{description}
\item[{Module}] \leavevmode
VIR0009

\item[{Meaning}] \leavevmode
Terminal termid was disconnected after expiry of the inactivity timer. This message can be suppressed by the SILENCE parameter in the VIRTCT.

\item[{Action}] \leavevmode
None.

\end{description}


\subsubsection{VIR0027I}
\label{\detokenize{messages:vir0027i}}
Screen type management messages :
\begin{description}
\item[{nnnn SCREEN TYPES LOADED}] \leavevmode\begin{description}
\item[{Module}] \leavevmode
VIR0000

\item[{Meaning}] \leavevmode
VIRTEL has located nnnn screen types while initialising level 3 compression.

\item[{Action}] \leavevmode
None.

\end{description}

\item[{SAVING SCREENS}] \leavevmode\begin{description}
\item[{Module}] \leavevmode
VIR0000

\item[{Meaning}] \leavevmode
VIRTEL is saving the screen types in file VIRCMP3 before stopping the system.

\item[{Action}] \leavevmode
None.

\end{description}

\end{description}


\subsubsection{VIR0028I SCREEN screentype ADDED TO LIBRARY}
\label{\detokenize{messages:vir0028i-screen-screentype-added-to-library}}\begin{description}
\item[{Module}] \leavevmode
VIR0000

\item[{Meaning}] \leavevmode
A screen type of screentype was added to the library of screen types.

\item[{Action}] \leavevmode
None.

\end{description}


\subsubsection{VIR0028I SCREEN screentype MODIFIED TO LIBRARY}
\label{\detokenize{messages:vir0028i-screen-screentype-modified-to-library}}\begin{description}
\item[{Module}] \leavevmode
VIR0000

\item[{Meaning}] \leavevmode
A screen of type screentype was modified in the library of screen types.

\item[{Action}] \leavevmode
None.

\end{description}


\subsubsection{VIR0028W SCREEN screentype WAS NOT MOVED TO LIBRARY}
\label{\detokenize{messages:vir0028w-screen-screentype-was-not-moved-to-library}}\begin{description}
\item[{Module}] \leavevmode
VIR0000

\item[{Meaning}] \leavevmode
A screen of type screentype could not be moved to the library of screen types.

\item[{Action}] \leavevmode
Verify that the VIRCMP3 file is not full or damaged.

\end{description}


\subsubsection{VIR0028W WELCOME OF UNDEFINED luname}
\label{\detokenize{messages:vir0028w-welcome-of-undefined-luname}}\begin{description}
\item[{Module}] \leavevmode
VIR0009

\item[{Meaning}] \leavevmode
A terminal luname that was not defined in VIRTEL was connected in “welcome” mode. It will no longer be under the control of VIRTEL once it has selected an application from the menu.

\item[{Action}] \leavevmode
None.

\end{description}


\subsubsection{VIR0029W WELCOME OF UNDEFINED luname DENIED : NO MORE ENTRIES}
\label{\detokenize{messages:vir0029w-welcome-of-undefined-luname-denied-no-more-entries}}\begin{description}
\item[{Module}] \leavevmode
VIR0009

\item[{Meaning}] \leavevmode
The terminal luname which is not defined in VIRTEL attempted to connect in “welcome” mode, but all the available terminal slots are occupied. The connection is refused by VIRTEL.

\item[{Action}] \leavevmode
If you wish to operate the terminal in “relay” mode, define the terminal in VIRTEL with an associated relay defined by a VTAM APPL card. If you wish to continue operating the terminal in “welcome” mode, increase the value of the NBDYNAM parameter in the VIRTCT.

\end{description}


\subsection{Messages VIR003xx}
\label{\detokenize{messages:messages-vir003xx}}

\subsubsection{VIR0030E ERROR xx LOADING progname}
\label{\detokenize{messages:vir0030e-error-xx-loading-progname}}\begin{description}
\item[{Module}] \leavevmode
VIR0000

\item[{Meaning}] \leavevmode
Error xx was detected while loading the module progname.

\item[{Action}] \leavevmode
This was probably an attempt to load an exit that was not assembled, or incorrectly referenced in the VIRTCT.

\end{description}


\subsubsection{VIR0030E ERROR: xx LOADING progname : CROSS MEMORY CANNOT START}
\label{\detokenize{messages:vir0030e-error-xx-loading-progname-cross-memory-cannot-start}}\begin{description}
\item[{Module}] \leavevmode
VIR0000

\item[{Meaning}] \leavevmode
VIRTEL cannot load the VIRXM interface program progname. VIRTEL is unable to start cross-memory (XM) lines.

\item[{Action}] \leavevmode
Ensure that the VIRXM load library is included in the STEPLIB of the VIRTEL STC.

\end{description}


\subsubsection{VIR0030E ERROR: xx LOADING progname : BATCH INTERFACE CANNOT START}
\label{\detokenize{messages:vir0030e-error-xx-loading-progname-batch-interface-cannot-start}}\begin{description}
\item[{Module}] \leavevmode
VIR0000

\item[{Meaning}] \leavevmode
VIRTEL cannot load the batch interface program progname. VIRTEL is unable to start batch (BATCHn) lines.

\item[{Action}] \leavevmode
Ensure that the correct version of the VIRTEL load library is included in the STEPLIB of the VIRTEL STC.

\end{description}


\subsubsection{VIR0030S ERROR OPENING DFHRPL (MVS only)}
\label{\detokenize{messages:vir0030s-error-opening-dfhrpl-mvs-only}}\begin{description}
\item[{Module}] \leavevmode
VIR0003

\item[{Meaning}] \leavevmode
The DFHRPL file was not open. Initialisation is stopped.

\item[{Action}] \leavevmode
Verify the definition of the STC.

\end{description}


\subsubsection{VIR0030W PROGRAM progname NOT FOUND}
\label{\detokenize{messages:vir0030w-program-progname-not-found}}\begin{description}
\item[{Module}] \leavevmode
VIR0003

\item[{Meaning}] \leavevmode
The module progname was not found in the library accessed by VIRTEL.

\item[{Action}] \leavevmode
Contact technical support.

\end{description}


\subsubsection{VIR0031E UNDEFINED TCP TCPn FOR LINE n-xxxxxx}
\label{\detokenize{messages:vir0031e-undefined-tcp-tcpn-for-line-n-xxxxxx}}\begin{description}
\item[{Module}] \leavevmode
VIR0000

\item[{Meaning}] \leavevmode
The definition of line n-xxxxxx specifies line type TCPn, but the TCPn parameter is not defined in the VIRTCT.

\item[{Action}] \leavevmode
Correct the line definition, or define the TCPn parameter in the VIRTCT.

\end{description}


\subsubsection{VIR0031E UNSUPPORTED CROSS-MEMORY XMn (protocol) FOR LINE n-xxxxxx}
\label{\detokenize{messages:vir0031e-unsupported-cross-memory-xmn-protocol-for-line-n-xxxxxx}}\begin{description}
\item[{Module}] \leavevmode
VIR0000

\item[{Meaning}] \leavevmode
The definition of line n-xxxxxx specifies line type XMn, but the XMn parameter is not defined in the VIRTCT.

\item[{Action}] \leavevmode
Correct the line definition, or define the XMn parameter in the VIRTCT.

\end{description}


\subsubsection{VIR0031E UNSUPPORTED MQ SERIES TYPE MQn (protocol) FOR LINE n-xxxxxx}
\label{\detokenize{messages:vir0031e-unsupported-mq-series-type-mqn-protocol-for-line-n-xxxxxx}}\begin{description}
\item[{Module}] \leavevmode
VIR0000

\item[{Meaning}] \leavevmode
The definition of line n-xxxxxx specifies line type MQn, but the MQn parameter is not defined in the VIRTCT.

\item[{Action}] \leavevmode
Correct the line definition, or define the MQn parameter in the VIRTCT.

\end{description}


\subsubsection{VIR0031E UNSUPPORTED BATCH LINE TYPE BATCHn (protocol) FOR LINE n-xxxxxx}
\label{\detokenize{messages:vir0031e-unsupported-batch-line-type-batchn-protocol-for-line-n-xxxxxx}}\begin{description}
\item[{Module}] \leavevmode
VIR0000

\item[{Meaning}] \leavevmode
The definition of line n-xxxxxx specifies line type BATCHn, but the BATCHn parameter is not defined in the VIRTCT.

\item[{Action}] \leavevmode
Correct the line definition, or define the BATCHn parameter in the VIRTCT.

\end{description}


\subsubsection{VIR0031W PROGRAM progname NOT FOUND}
\label{\detokenize{messages:vir0031w-program-progname-not-found}}\begin{description}
\item[{Module}] \leavevmode
VIR0003

\item[{Meaning}] \leavevmode
The module progname was not found in the library accessed by VIRTEL.

\item[{Action}] \leavevmode
In MVS, verify that the DFHRPL DD card in the VIRTEL started task JCL specifies the name of the library that contains the VIRTEL load modules. In DOS, verify that the LIBDEF SEARCH card in the VIRTEL startup JCL references the library that contains the VIRTEL phases.

\end{description}


\subsubsection{VIR0032E PERMANENT I/O ERROR DURING FETCH}
\label{\detokenize{messages:vir0032e-permanent-i-o-error-during-fetch}}\begin{description}
\item[{Module}] \leavevmode
VIR0003

\item[{Meaning}] \leavevmode
An error occurred while attempting to load a module.

\item[{Action}] \leavevmode
Verify the definition of the library containing VIRTEL modules.

\end{description}


\subsubsection{VIR0032W BYPASSING LINE n-xxxxxx : STATUS IS ZERO}
\label{\detokenize{messages:vir0032w-bypassing-line-n-xxxxxx-status-is-zero}}\begin{description}
\item[{Module}] \leavevmode
VIR0000

\item[{Meaning}] \leavevmode
The line whose internal name is n-xxxxxx was not activated at VIRTEL startup, because the “Possible calls” field is set to 0 in the line definition.

\item[{Action}] \leavevmode
None.

\end{description}


\subsubsection{VIR0032W BYPASSING LINE n-xxxxxx : DISABLED IN VIRTCT}
\label{\detokenize{messages:vir0032w-bypassing-line-n-xxxxxx-disabled-in-virtct}}\begin{description}
\item[{Module}] \leavevmode
VIR0000

\item[{Meaning}] \leavevmode
The line whose internal name is n-xxxxxx was not activated at VIRTEL startup, either because its name appears in the IGNLU parameter in the VIRTCT, or because your VIRTEL license does not allow the activation of this type of line.

\item[{Action}] \leavevmode
In the first case, remove the line name from the IGNLU parameter in the VIRTCT. In the second case, contact Syspertec to upgrade your license.

\end{description}


\subsubsection{VIR0033E NO LINE DEFINED FOR termid}
\label{\detokenize{messages:vir0033e-no-line-defined-for-termid}}\begin{description}
\item[{Module}] \leavevmode
VIR0000

\item[{Meaning}] \leavevmode
Terminal termid is defined but is not linked to any active line.

\item[{Action}] \leavevmode
Check that the prefix of the terminal name is referenced in the definition of the appropriate line. This message is normal if the terminal is linked to an inactive line (indicated by message VIR0032W).

\end{description}


\subsubsection{VIR0033W INSUFICIENT MEMORY TO LOAD MODULE progname}
\label{\detokenize{messages:vir0033w-insuficient-memory-to-load-module-progname}}\begin{description}
\item[{Module}] \leavevmode
VIR0003

\item[{Meaning}] \leavevmode
VIRTEL has not loaded the module progname because of memory shortage.

\item[{Action}] \leavevmode
See message VIR0011E and VIR0012W.

\end{description}


\subsubsection{VIR0034E INVALID DEB DURING LOAD}
\label{\detokenize{messages:vir0034e-invalid-deb-during-load}}\begin{description}
\item[{Module}] \leavevmode
VIR0003

\item[{Meaning}] \leavevmode
Unexpected error when loading a module from DFHRPL.

\item[{Action}] \leavevmode
Call technical support.

\end{description}


\subsubsection{VIR0034W BYPASSING RULE rulename}
\label{\detokenize{messages:vir0034w-bypassing-rule-rulename}}\begin{description}
\item[{Module}] \leavevmode
VIR0000

\item[{Meaning}] \leavevmode
Rule rulename could not be loaded.

\item[{Action}] \leavevmode
Check that VIRTEL has enough memory. Obtain a SNAP dump and call technical support.

\end{description}


\subsubsection{VIR0035E UNDEFINED LINE n-xxxxxx FOR RULE rulename}
\label{\detokenize{messages:vir0035e-undefined-line-n-xxxxxx-for-rule-rulename}}\begin{description}
\item[{Module}] \leavevmode
VIR0000

\item[{Meaning}] \leavevmode
Rule rulename is associated with line n-xxxxxx, but the line does not exist or is not active.

\item[{Action}] \leavevmode
This message is normal if the rule is linked to an inactive line (“Possible calls” set to 0, or line specified in the IGNLU parameter in the VIRTCT) or to a user ruleset.

\end{description}


\subsubsection{VIR0036W WARNING : RULE rulename FOR LINE n-xxxxxx HAS AN ACTIVE TRACE}
\label{\detokenize{messages:vir0036w-warning-rule-rulename-for-line-n-xxxxxx-has-an-active-trace}}\begin{description}
\item[{Module}] \leavevmode
VIR0000

\item[{Meaning}] \leavevmode
Rule rulename associated with n-xxxxxx is set to trigger a trace of incoming calls.

\item[{Action}] \leavevmode
If the trace is not wanted, display the definition of rule rulename from the definition panel for line n-xxxxxx. Blank out the “Trace” field and press F1.

\end{description}


\subsubsection{VIR0037E ERREUR xx OUVERTURE SYSPRINT}
\label{\detokenize{messages:vir0037e-erreur-xx-ouverture-sysprint}}\begin{description}
\item[{Module}] \leavevmode
VIR0004

\item[{Meaning}] \leavevmode
Error xx occurred when opening the file SYSPRINT/SYSLST.

\item[{Action}] \leavevmode
Verify the DD card or the DLBL referencing the print file.

\end{description}


\subsubsection{VIR0038I SNAP COMPLETE}
\label{\detokenize{messages:vir0038i-snap-complete}}\begin{description}
\item[{Module}] \leavevmode
VIR0004

\item[{Meaning}] \leavevmode
VIRTEL has written a SNAP dump of the internal trace table to the SYSPRINT/SYSLST file.

\item[{Action}] \leavevmode
None.

\end{description}


\subsubsection{VIR0039E ERROR: THE VIRTCT VIRTCTxx IS INVALID: VIRTEL CANNOT CONTINUE}
\label{\detokenize{messages:vir0039e-error-the-virtct-virtctxx-is-invalid-virtel-cannot-continue}}\begin{description}
\item[{Module}] \leavevmode
VIR0000

\item[{Meaning}] \leavevmode
The length of the VIRTCT does not match the expected length for this release of VIRTEL.

\item[{Action}] \leavevmode
Use job ASMTCT in the VIRTEL CNTL library to reassemble the VIRTCT using the correct level VIRTERM macro.

\end{description}


\subsubsection{VIR0039I trace command VTAM}
\label{\detokenize{messages:vir0039i-trace-command-vtam}}\begin{description}
\item[{Module}] \leavevmode
VIR2020

\item[{Meaning}] \leavevmode
A VTAM command other than a display was issued from the VIRTEL network management system.

\item[{Action}] \leavevmode
None.

\end{description}


\subsection{Messages VIR004xx}
\label{\detokenize{messages:messages-vir004xx}}

\subsubsection{VIR0040E ERROR: THE VIRTCT progname IS INVALID: VIRTEL CANNOT CONTINUE}
\label{\detokenize{messages:vir0040e-error-the-virtct-progname-is-invalid-virtel-cannot-continue}}\begin{description}
\item[{Module}] \leavevmode
VIR0000

\item[{Meaning}] \leavevmode
The VIRTCT progname was assembled using a version of VIRTEL which was not the same as the running version.

\item[{Action}] \leavevmode
Re-assemble the VIRTCT using the current version of the VIRTEL MACLIB.

\end{description}


\subsubsection{VIR0040I GATE : linetype LINE n-xxxxxx ACTIVATED}
\label{\detokenize{messages:vir0040i-gate-linetype-line-n-xxxxxx-activated}}\begin{description}
\item[{Module}] \leavevmode
VIR0005

\item[{Meaning}] \leavevmode
VIRTEL has established communication with the X25 line linename.

\item[{Action}] \leavevmode
None.

\end{description}


\subsubsection{VIR0041I termid : CALL ABORTED ON LINE n-xxxxxx VC cccc CAUSE = xx DIAGNOSTIC = dd}
\label{\detokenize{messages:vir0041i-termid-call-aborted-on-line-n-xxxxxx-vc-cccc-cause-xx-diagnostic-dd}}\begin{description}
\item[{Module}] \leavevmode
VIR0005

\item[{Meaning}] \leavevmode
An outgoing call in GATE mode using terminal termid did not complete. If cccc is greater than the number of SVC’s, it refers to the temporary identification of the outgoing call.

\item[{Action}] \leavevmode
For the meaning of the cause and diagnostic codes, refer to the X.25 documentation supplied by SAPONET.

\end{description}


\subsubsection{VIR0042I GATE : UNSUPPORTED COMMAND = xx ON MCH n-xxxxxx}
\label{\detokenize{messages:vir0042i-gate-unsupported-command-xx-on-mch-n-xxxxxx}}\begin{description}
\item[{Module}] \leavevmode
VIR0005

\item[{Meaning}] \leavevmode
The packet received from the X.25 network is of an unknown type. xx represents the hexadecimal value of the first byte of the packet, n-xxxxxx represents the name of the MCH on which the incident occurred.

\item[{Action}] \leavevmode
Contact technical services if the incident persists.

\end{description}


\subsubsection{VIR0043I GATE : DIAGNOSTIC = xx,yyyyy on MCH n-xxxxxx}
\label{\detokenize{messages:vir0043i-gate-diagnostic-xx-yyyyy-on-mch-n-xxxxxx}}\begin{description}
\item[{Module}] \leavevmode
VIR0005

\item[{Meaning}] \leavevmode
A diagnostic packet was received from the X.25 network. The packet contains values xx and call user data yyyyy. For certain codes, yyyyy represents the number of the virtual circuit concerned.

\item[{Action}] \leavevmode
Refer to the SAPONET documentation.

\end{description}


\subsubsection{VIR0044I termid : COMMAND xx ERROR yy on VC cccccc}
\label{\detokenize{messages:vir0044i-termid-command-xx-error-yy-on-vc-cccccc}}\begin{description}
\item[{Module}] \leavevmode
VIR0005

\item[{Meaning}] \leavevmode
The command xx sent on virtual circuit cccccc has produced an error yy. termid is the name of the terminal concerned.

\item[{Action}] \leavevmode
Refer to the SAPONET documentation.

\end{description}


\subsubsection{VIR0045E termid : NBCVC PARAMETER TOO SMALL}
\label{\detokenize{messages:vir0045e-termid-nbcvc-parameter-too-small}}\begin{description}
\item[{Module}] \leavevmode
VIR0005

\item[{Meaning}] \leavevmode
A call could not complete because there was no VC available. termid is the name of the terminal concerned.

\item[{Action}] \leavevmode
Increase the value of the NBCVC operand in the VIRTCT of VIRTEL.

\end{description}


\subsubsection{VIR0046E GATE : PROTOCOL ERROR ON MCH n-xxxxxx}
\label{\detokenize{messages:vir0046e-gate-protocol-error-on-mch-n-xxxxxx}}\begin{description}
\item[{Module}] \leavevmode
VIR0005

\item[{Meaning}] \leavevmode
An error was encountered on the link with the MCH n-xxxxxx.

\item[{Action}] \leavevmode
This error is different to the lost session error, or the deactivation of the LU MCH. (See the possible associated message VIR0002W ).

\end{description}


\subsubsection{VIR0047W GATE : ERROR ACTIVATING linetype LINE n-xxxxxx}
\label{\detokenize{messages:vir0047w-gate-error-activating-linetype-line-n-xxxxxx}}\begin{description}
\item[{Module}] \leavevmode
VIR0005

\item[{Meaning}] \leavevmode
VIRTEL could not acquire the X.25 line n-xxxxxx.

\item[{Action}] \leavevmode
Check the definition of line n-xxxxxx. The value in the “Partner” field must match the name of the MCH LU generated by NPSI. Verify that the LU is active in VTAM.

\end{description}


\subsubsection{VIR0048W GATE : LINE n-xxxxxx INACTIVATED}
\label{\detokenize{messages:vir0048w-gate-line-n-xxxxxx-inactivated}}\begin{description}
\item[{Module}] \leavevmode
VIR0005

\item[{Meaning}] \leavevmode
The link between VIRTEL and the line n-xxxxxx has terminated.

\item[{Action}] \leavevmode
Verify the cause of the deactivation.

\end{description}


\subsubsection{VIR0049I X25 COMMAND xx RECEIVED FOR TERMINAL termid}
\label{\detokenize{messages:vir0049i-x25-command-xx-received-for-terminal-termid}}\begin{description}
\item[{Module}] \leavevmode
VIR0005

\item[{Meaning}] \leavevmode
VIRTEL has received the X.25 command xx from terminal termid. It is either unknown, or is unexpected at this time, and is ignored.

\item[{Action}] \leavevmode
None.

\end{description}


\subsection{Messages VIR005xx}
\label{\detokenize{messages:messages-vir005xx}}

\subsubsection{VIR0050W INVALID EIB FREEMAIN FOR luname}
\label{\detokenize{messages:vir0050w-invalid-eib-freemain-for-luname}}\begin{description}
\item[{Module}] \leavevmode
VIR0009

\item[{Meaning}] \leavevmode
An unexpected free memory block error associated with terminal luname was encountered.

\item[{Action}] \leavevmode
Contact technical support if the message persists.

\end{description}


\subsubsection{VIR0051I termid CONNECTED TO SERVICE ssssssss}
\label{\detokenize{messages:vir0051i-termid-connected-to-service-ssssssss}}\begin{description}
\item[{Module}] \leavevmode
VIR0014

\item[{Meaning}] \leavevmode
The terminal termid has established contact with VIRTEL. It has been connected to the service ssssssss. This message can be suppressed by the SILENCE parameter in the VIRTCT.

\item[{Action}] \leavevmode
None.

\end{description}


\subsubsection{VIR0052I termid DISCONNECTED AFTER nn MINUTES}
\label{\detokenize{messages:vir0052i-termid-disconnected-after-nn-minutes}}\begin{description}
\item[{Module}] \leavevmode
VIR0014

\item[{Meaning}] \leavevmode
The terminal termid has disconnected after nn minutes of connection. This message can be suppressed by the SILENCE parameter in the VIRTCT.

\item[{Action}] \leavevmode
None.

\end{description}


\subsubsection{VIR0053W MISSING PAGE pagename IN NODE nodename}
\label{\detokenize{messages:vir0053w-missing-page-pagename-in-node-nodename}}\begin{description}
\item[{Module}] \leavevmode
VIR0014

\item[{Meaning}] \leavevmode
Tree structure definition problem. The node nodename referenced a page pagename that does not exist in the VIRARBO file.

\item[{Action}] \leavevmode
Verify the definition of the Minitel tree structure.

\end{description}


\subsubsection{VIR0056S NO MORE OSCORE AVAILABLE}
\label{\detokenize{messages:vir0056s-no-more-oscore-available}}\begin{description}
\item[{Module}] \leavevmode
VIR0009

\item[{Meaning}] \leavevmode
VIRTEL has insufficient memory available to load a module.

\item[{Action}] \leavevmode
Verify the OSCORE parameter in the VIRTCT.

\end{description}


\subsubsection{VIR0059I termid RELAY relayname ACTIVATED}
\label{\detokenize{messages:vir0059i-termid-relay-relayname-activated}}\begin{description}
\item[{Module}] \leavevmode
VIR0000

\item[{Meaning}] \leavevmode
VIRTEL has opened the relay relayname associated with the terminal termid.

\item[{Action}] \leavevmode
None.

\end{description}


\subsection{Messages VIR006xx}
\label{\detokenize{messages:messages-vir006xx}}

\subsubsection{VIR0060W MAPFAIL WAS DETECTED ON TERMINAL luname}
\label{\detokenize{messages:vir0060w-mapfail-was-detected-on-terminal-luname}}\begin{description}
\item[{Module}] \leavevmode
VIR0010

\item[{Meaning}] \leavevmode
Conflict between a VIRTEL program and a sub application screen.

\item[{Action}] \leavevmode
Contact technical support.

\end{description}


\subsubsection{VIR0060W PROGRAM progname IS A NEW COPY}
\label{\detokenize{messages:vir0060w-program-progname-is-a-new-copy}}\begin{description}
\item[{Module}] \leavevmode
VIR0002

\item[{Meaning}] \leavevmode
A NEW command for module progname has successfully refreshed the module in memory.

\item[{Action}] \leavevmode
None.

\end{description}


\subsubsection{VIR0061E MAP mapname NOT FOUND IN MAPSET mapsetname}
\label{\detokenize{messages:vir0061e-map-mapname-not-found-in-mapset-mapsetname}}\begin{description}
\item[{Module}] \leavevmode
VIR0010

\item[{Meaning}] \leavevmode
Conflict between a map mapname and a VIRTEL program.

\item[{Action}] \leavevmode
An abend follows this message. Contact technical support.

\end{description}


\subsubsection{VIR0061W PROGRAM progname NOT IN MEMORY}
\label{\detokenize{messages:vir0061w-program-progname-not-in-memory}}\begin{description}
\item[{Module}] \leavevmode
VIR0000, VIR0002

\item[{Meaning}] \leavevmode
A NEW command, a ZAP command, or a ZAPD instruction did not complete due to the absence of the module progname in memory.

\item[{Action}] \leavevmode
For a NEW command: None. The module will be loaded by VIRTEL when required. For a ZAP command: Correct the command and reenter. For a ZAPD instruction: Correct the instruction and restart VIRTEL.

\end{description}


\subsubsection{VIR0062I termid TRACE ACTIVE}
\label{\detokenize{messages:vir0062i-termid-trace-active}}\begin{description}
\item[{Module}] \leavevmode
VIR0002

\item[{Meaning}] \leavevmode
A TRACE command has activated the trace on terminal or line termid.

\item[{Action}] \leavevmode
None.

\end{description}


\subsubsection{VIR0062I termid TRACE INACTIVE}
\label{\detokenize{messages:vir0062i-termid-trace-inactive}}\begin{description}
\item[{Module}] \leavevmode
VIR0002

\item[{Meaning}] \leavevmode
A NOTRACE command has inactivated the trace on terminal or line termid.

\item[{Action}] \leavevmode
None.

\end{description}


\subsubsection{VIR0062W LINE linename IS UNKNOWN}
\label{\detokenize{messages:vir0062w-line-linename-is-unknown}}\begin{description}
\item[{Module}] \leavevmode
VIR0002

\item[{Meaning}] \leavevmode
A LINE=linename,START or STOP command refers to a line not known to VIRTEL.

\item[{Action}] \leavevmode
Reenter the command specifying a valid linename.

\end{description}


\subsubsection{VIR0063W LINE linename ALREADY ACTIVE}
\label{\detokenize{messages:vir0063w-line-linename-already-active}}\begin{description}
\item[{Module}] \leavevmode
VIR0002

\item[{Meaning}] \leavevmode
A LINE=linename,START command attempted to start a line which was already active.

\item[{Action}] \leavevmode
None.

\end{description}


\subsubsection{VIR0064W LINE linename (n-xxxxxx) START / STOP REQUESTED}
\label{\detokenize{messages:vir0064w-line-linename-n-xxxxxx-start-stop-requested}}\begin{description}
\item[{Module}] \leavevmode
VIR0002

\item[{Meaning}] \leavevmode
VIRTEL has processed a LINE=linename,START or STOP command on the line whose external name is linename and whose internal name is n-xxxxxx.

\item[{Action}] \leavevmode
None.

\end{description}


\subsubsection{VIR0064W ADDRESS aaaa NOW IS :  yyyy yyyy}
\label{\detokenize{messages:vir0064w-address-aaaa-now-is-yyyy-yyyy}}\begin{description}
\item[{Module}] \leavevmode
VIR0000, VIR0002

\item[{Meaning}] \leavevmode
Notifies that the ZAP command or ZAPD instruction was executed successfully in memory at address aaaa.

\item[{Action}] \leavevmode
A modification made by ZAP command is valid until the next restart of VIRTEL. The ZAPH parameter of the VIRTCT can be used to ensure that the modification is reappied at each restart.

\end{description}


\subsubsection{VIR0064W OFFSET LENGTH xxxx IS INVALID}
\label{\detokenize{messages:vir0064w-offset-length-xxxx-is-invalid}}\begin{description}
\item[{Module}] \leavevmode
VIR0000

\item[{Meaning}] \leavevmode
The offset field of a ZAPD instruction in the VIRTCT is invalid.

\item[{Action}] \leavevmode
Correct the ZAPD instruction and restart VIRTEL.

\end{description}


\subsubsection{VIR0065E VERIFY ERROR. ADDRESS aaaa IS : yyyy yyyy}
\label{\detokenize{messages:vir0065e-verify-error-address-aaaa-is-yyyy-yyyy}}\begin{description}
\item[{Module}] \leavevmode
VIR0000, VIR0002

\item[{Meaning}] \leavevmode
A ZAP command or a ZAPD instruction cannot complete because of an error at address aaaa during verify.

\item[{Action}] \leavevmode
None.

\end{description}


\subsubsection{VIR0066I APPLYING ptfid ON progname desc}
\label{\detokenize{messages:vir0066i-applying-ptfid-on-progname-desc}}\begin{description}
\item[{Module}] \leavevmode
VIR0000

\item[{Meaning}] \leavevmode
At startup VIRTEL is preparing to process a ZAPD instruction in the VIRTCT.

\item[{Action}] \leavevmode
None.

\end{description}


\subsubsection{VIR0067I MESSAGES ARE NOW DISPLAYED VIR0067I MESSAGES ARE NOW DISCARDED}
\label{\detokenize{messages:vir0067i-messages-are-now-displayed-vir0067i-messages-are-now-discarded}}\begin{description}
\item[{Module}] \leavevmode
VIR0002

\item[{Meaning}] \leavevmode
As a result of the SILENCE command, connection and deconnection messages will now be displayed or discarded.

\item[{Action}] \leavevmode
None.

\end{description}


\subsubsection{VIR0068E}
\label{\detokenize{messages:vir0068e}}
Invalid system command.
\begin{description}
\item[{INVALID COMMAND}] \leavevmode
Correct the command

\item[{Module}] \leavevmode
VIR0002

\item[{Meaning}] \leavevmode
The command passed to VIRTEL is unknown.

\item[{Action}] \leavevmode
Correct the command

\item[{NOT FOUND}] \leavevmode
Correct the command.

\item[{Module}] \leavevmode
VIR0002

\item[{Meaning}] \leavevmode
A SNAP or TRACE command referenced a terminal unknown to VIRTEL.

\item[{Action}] \leavevmode
Correct the command.

\end{description}


\subsubsection{VIR0068I SNAP WILL FOLLOW msgid1 msgid2}
\label{\detokenize{messages:vir0068i-snap-will-follow-msgid1-msgid2}}\begin{description}
\item[{Module}] \leavevmode
VIR0002

\item[{Meaning}] \leavevmode
As a result of a SNAPMSG command, VIRTEL will produce a SNAP following the first occurrence of one of the messages indicated.

\item[{Action}] \leavevmode
None.

\end{description}


\subsubsection{VIR0069I READY}
\label{\detokenize{messages:vir0069i-ready}}\begin{description}
\item[{Module}] \leavevmode
VIR0002

\item[{Meaning}] \leavevmode
VIRTEL is ready for the next console command (VSE).

\item[{Action}] \leavevmode
None.

\end{description}


\subsection{Messages VIR007xx}
\label{\detokenize{messages:messages-vir007xx}}

\subsubsection{VIR0070I SIMULTANEOUS TRANSACTION AT TERMINAL termid}
\label{\detokenize{messages:vir0070i-simultaneous-transaction-at-terminal-termid}}\begin{description}
\item[{Module}] \leavevmode
VIR0010

\item[{Meaning}] \leavevmode
The terminal termid has started a second transaction while the first one was still active.

\item[{Action}] \leavevmode
If the first transaction results in a Multi-Session menu display, VIRTEL maybe did not obtain from VTAM the status of the menu applications (in cross domain for example). If this is the case, remove status control from this application (PF9 from the general Sub-Applications menu).

\end{description}


\subsection{Messages VIR008xx}
\label{\detokenize{messages:messages-vir008xx}}

\subsubsection{VIR0080W VSAM ERROR ON FILE VIRARBO xx xx REQ : yy, KEY: cccccccc}
\label{\detokenize{messages:vir0080w-vsam-error-on-file-virarbo-xx-xx-req-yy-key-cccccccc}}\begin{description}
\item[{Module}] \leavevmode
Security

\item[{Meaning}] \leavevmode
VSAM error xx xx has appeared on the VIRARBO file for request yy on key cccccccc.

\item[{Action}] \leavevmode
VSAM error codes are documented in the chapter entitled VSAM Macro Return and Reason Codes in the IBM manual DFSMS Macro Instructions for Data Sets.

\end{description}


\subsubsection{VIR0081W NO MORE OSCORE AVAILABLE}
\label{\detokenize{messages:vir0081w-no-more-oscore-available}}\begin{description}
\item[{Module}] \leavevmode
Security

\item[{Meaning}] \leavevmode
VIRTEL does not have sufficient memory to load a security module.

\item[{Action}] \leavevmode
Check the value of the OSCORE parameter in the VIRTCT.

\end{description}


\subsubsection{VIR0082W UNAUTHORIZED USER TERMINAL : luname NAME : username}
\label{\detokenize{messages:vir0082w-unauthorized-user-terminal-luname-name-username}}\begin{description}
\item[{Module}] \leavevmode
Security

\item[{Meaning}] \leavevmode
VIRTEL security. A password was erroneously used more than three times for the same user username from the terminal luname.

\item[{Action}] \leavevmode
The user username is revoked and may not be re established except by the security administrator of VIRTEL.

\end{description}


\subsubsection{VIR0083W opcode OF ELEMENT xxxxxxxx BY username FROM luname}
\label{\detokenize{messages:vir0083w-opcode-of-element-xxxxxxxx-by-username-from-luname}}\begin{description}
\item[{Module}] \leavevmode
Security

\item[{Meaning}] \leavevmode
VIRTEL security. Trace a modification of the security of VIRTEL. The user username has proceeded with the modification opcode (ADD, UPDATE, DELETION) of security element xxxxxxxx USER, RESSOURCE, PROFIL, DEPARTMENT from terminal luname.

\item[{Action}] \leavevmode
None.

\end{description}


\subsubsection{VIR0084W ELEMENT REFERENCE ERROR xxxxxxxx ELEMENT yyyyyyyy}
\label{\detokenize{messages:vir0084w-element-reference-error-xxxxxxxx-element-yyyyyyyy}}\begin{description}
\item[{Module}] \leavevmode
Security

\item[{Meaning}] \leavevmode
VIRTEL security. A referencing problem in the security of VIRTEL. For example the element xxxxxxxx referenced element yyyyyyyy that does not exist.

\item[{Action}] \leavevmode
Modify the element reference in the security program of VIRTEL.

\end{description}


\subsubsection{VIR0085E INVALID MEMORY FREE REQUEST. ADDR=aaaaaaaa. SUBPOOL=ss. CALLER=cccccccc}
\label{\detokenize{messages:vir0085e-invalid-memory-free-request-addr-aaaaaaaa-subpool-ss-caller-cccccccc}}\begin{description}
\item[{Module}] \leavevmode
VIR0000

\item[{Meaning}] \leavevmode
The VIRTEL memory request is invalid because, either the subpool is invalid or the address to be freed is outside the associated subpool pages, or the address to be freed is not found in the DSA table.

\item[{Action}] \leavevmode
Virtel continues. If you get a significant number of these messages you consider a schedule a restart of Virtel.

\end{description}


\subsubsection{VIR0086E GETMAIN FAILED. MEMORY DSA DISABLED}
\label{\detokenize{messages:vir0086e-getmain-failed-memory-dsa-disabled}}\begin{description}
\item[{Module}] \leavevmode
VIR0000

\item[{Meaning}] \leavevmode
Virtel was unable to GETMAIN storage “above the BAR” for the DSA table.

\item[{Action}] \leavevmode
Increase MEMLIMIT= in the JCL to provide more “above the BAR” storage. Virtel continues but you should contact support if the problem continues. Suggest a default of MEMLIMIT=2G.

\end{description}


\subsubsection{VIRT0088E DSA TABLE FULL. MEMORY DIAGS. DISABLED}
\label{\detokenize{messages:virt0088e-dsa-table-full-memory-diags-disabled}}\begin{description}
\item[{Module}] \leavevmode
VIR0000

\item[{Meaning}] \leavevmode
The DSA memory table is full and cannot store further entries. Memory diagnostics disabled.

\item[{Action}] \leavevmode
Virtel contines but you should contact support support. Schedule a restart of Virtel.

\end{description}


\subsubsection{VIR0086E GETMAIN FAILED. MEMORY DSA DISABLED}
\label{\detokenize{messages:id6}}\begin{description}
\item[{Module}] \leavevmode
VIR0000

\item[{Meaning}] \leavevmode
Virtel was unable to GETMAIN storage “above the BAR” for the DSA table.

\item[{Action}] \leavevmode
Increase MEMLIMIT= in the JCL to provide more “above the BAR” storage. Virtel continues but you should contact support if the problem continues. Suggest a default of MEMLIMIT=2G.

\end{description}


\subsubsection{VIR0089I VIRTEL RUNNING AUTHORIZED}
\label{\detokenize{messages:vir0089i-virtel-running-authorized}}\begin{description}
\item[{Module}] \leavevmode
Security

\item[{Meaning}] \leavevmode
VIRTEL security. Virtel is running as an authorized task.

\item[{Action}] \leavevmode
None. Information only.

\end{description}


\subsection{Messages VIR009xx}
\label{\detokenize{messages:messages-vir009xx}}

\subsubsection{VIR0090E VIRSV INITIALIZATION ERROR -VSVPOPTR R15 : xxxxxxxx (dddddddd)}
\label{\detokenize{messages:vir0090e-virsv-initialization-error-vsvpoptr-r15-xxxxxxxx-dddddddd}}\begin{description}
\item[{Module}] \leavevmode
VIR0000

\item[{Meaning}] \leavevmode
VIRTEL was unable to initialize the VIRSV service environment. xxxxxxxx is the hexadecimal return code from program VSVPOPTR, and dddddddd is the decimal equivalent of the low-order byte of the return code.

\item[{Action}] \leavevmode
Refer to the VIRSV User’s Guide manual to determine the meaning of the VSVPOPTR return code. Check the VIRTEL log and the VSVTRACE file for additional messages. Check the VIRTEL started task JCL to ensure that the VIRSV load library is referenced in both the STEPLIB and the SERVLIB concatenations. Check that the VIRSV load library is APF- authorized.

\end{description}


\subsubsection{VIR0091E ERROR: VIRTEL MUST BE APF AUTHORIZED WHEN SECUR=RACROUTE}
\label{\detokenize{messages:vir0091e-error-virtel-must-be-apf-authorized-when-secur-racroute}}\begin{description}
\item[{Module}] \leavevmode
VIR0000

\item[{Meaning}] \leavevmode
VIRTEL cannot start because RACROUTE security was requested by the SECUR parameter of the VIRTCT, but the VIRTEL address space is not running as an APF-authorized jobstep.

\item[{Action}] \leavevmode
Check that all of the libraries referenced by the STEPLIB, DFHRPL, and SERVLIB (if present) statements in the VIRTEL started task JCL are defined as APF-authorized libraries in the MVS system APF-list.

\end{description}


\subsubsection{VIR0092E GNAMEADD FAILED FOR VTAM GENERIC RESOURCE grname}
\label{\detokenize{messages:vir0092e-gnameadd-failed-for-vtam-generic-resource-grname}}\begin{description}
\item[{Module}] \leavevmode
VIR0000

\item[{Meaning}] \leavevmode
VIRTEL was unable to identify itself to VTAM as a generic resource. grname is the generic resource name specified in the GRNAME parameter of the VIRTCT.

\item[{Action}] \leavevmode
Check the console log for preceding message VIR0002W REQ=SETLOGON RTNCD=cc FDBK2=dd. Take action according to the return code and feedback code indicated in message VIR0002W. Commonly encountered codes are:
- RTNCD=10 FDBK2=19 Sysplex coupling facility does not exist; CFRM policy for the required coupling facility structure was not active; VTAM is not defined as an APPN node; or VTAM has lost connectivity to the required coupling facility structure.

\end{description}


\subsubsection{VIR0093I VTAM GENERIC RESOURCE NAME IS grname}
\label{\detokenize{messages:vir0093i-vtam-generic-resource-name-is-grname}}\begin{description}
\item[{Module}] \leavevmode
VIR0000

\item[{Meaning}] \leavevmode
VIRTEL has successfully identified itself to VTAM as a generic resource. grname is the generic resource name specified in the GRNAME parameter of the VIRTCT.

\item[{Action}] \leavevmode
None.

\end{description}


\subsubsection{VIR0094E PRODID {[}DEFINE \textbar{} AUTH{]} ERROR: RC=code}
\label{\detokenize{messages:vir0094e-prodid-define-auth-error-rc-code}}\begin{description}
\item[{Module}] \leavevmode
VIR0000

\item[{Meaning}] \leavevmode
While attempting to identify itself to z/VSE, VIRTEL encountered an unexpected return code.

\item[{Action}] \leavevmode
Contact technical support.

\end{description}


\subsubsection{VIR0095I PRODID AUTHORIZATION SUCCESSFUL}
\label{\detokenize{messages:vir0095i-prodid-authorization-successful}}\begin{description}
\item[{Module}] \leavevmode
VIR0000

\item[{Meaning}] \leavevmode
VIRTEL has successfully identified itself to z/VSE as a vendor product.

\item[{Action}] \leavevmode
None.

\end{description}


\subsubsection{VIR0096I VIRTEL IS USING VIRTCT ‘VIRTCTnn’}
\label{\detokenize{messages:vir0096i-virtel-is-using-virtct-virtctnn}}\begin{description}
\item[{Module}] \leavevmode
VIR0000

\item[{Meaning}] \leavevmode
As a result of the parameter TCT=nn specified in the startup JCL, VIRTEL is using the parameter table VIRTCTnn.

\item[{Action}] \leavevmode
None.

\end{description}


\subsubsection{VIR0097E ERROR ALLOCATING MEMORY FOR termid}
\label{\detokenize{messages:vir0097e-error-allocating-memory-for-termid}}\begin{description}
\item[{Module}] \leavevmode
VIR0000

\item[{Meaning}] \leavevmode
VIRTEL has insufficient memory to allocate a storage area for terminal termid.

\item[{Action}] \leavevmode
Increase the VIRTEL region size or partition size.

\end{description}


\subsubsection{VIR0098I VIRTEL RUNNING AS A SUBTASK. LINKED FROM mmmmmmmm}
\label{\detokenize{messages:vir0098i-virtel-running-as-a-subtask-linked-from-mmmmmmmm}}\begin{description}
\item[{Module}] \leavevmode
VIR0000

\item[{Meaning}] \leavevmode
Indicates that VIRTEL has been attached and called by another process. The module mmmmmmm is calling routine.

\item[{Action}] \leavevmode
None

\end{description}


\subsubsection{VIR0099I applid STARTED AT dd/mm/yy hh:mm:ss , VERSION vvvv}
\label{\detokenize{messages:vir0099i-applid-started-at-dd-mm-yy-hh-mm-ss-version-vvvv}}\begin{description}
\item[{Module}] \leavevmode
VIR0000

\item[{Meaning}] \leavevmode
Initialisation of VIRTEL version vvvv is complete. VTAM application applid is now available.

\item[{Action}] \leavevmode
None.

\end{description}


\subsection{Messages VIR02xxx}
\label{\detokenize{messages:messages-vir02xxx}}

\subsubsection{VIR0200I, VIR0201I}
\label{\detokenize{messages:vir0200i-vir0201i}}\begin{description}
\item[{Module}] \leavevmode
VIR0002

\item[{Meaning}] \leavevmode
This message serves as an audit trail of a VIRTEL command entered at the operator console.

\item[{Action}] \leavevmode
None.

\end{description}


\subsubsection{VIR0201I VIRTEL 4.xx APPLID=applid LINES {[}ACT/INACT{]}}
\label{\detokenize{messages:vir0201i-virtel-4-xx-applid-applid-lines-act-inact}}\begin{description}
\item[{Module}] \leavevmode
VIR0002

\item[{Meaning}] \leavevmode
This message is displayed in response to a VIRTEL LINES command. 4.xx is the VIRTEL version number and applid is the name of the main VIRTEL application ACB.

\item[{Action}] \leavevmode
None.

\end{description}
\begin{description}
\item[{Module}] \leavevmode
VIR0002

\item[{Meaning}] \leavevmode
This message is the output from a VIRTEL LINES command. For each line displayed, n-xxxxxx is the internal name of the line, linename is the external name of the line, type is the line type (GATE, FASTC, /GATE, /FASTC, /PCNE, APPCn, TCPn), and locaddr is the VTAM LU name or IP address of the line specified in the “Local ident” field of the line definition. An asterisk before the line type indicates that the line is currently inactive.

\item[{Action}] \leavevmode
None.

\end{description}
\begin{description}
\item[{Module}] \leavevmode
VIR0002

\item[{Meaning}] \leavevmode
This message is the output from a VIRTEL LINE=linename,DISPLAY command. It displays a list of terminals associated with the line. linename is the external name of the line, and n-xxxxxx is the internal name of the line. For each terminal displayed, termid is the terminal name, “+” indicates that this terminal is pointed to by the MCH 1st LU pointer, luname is the relay LU name, acbstat is the relay ACB status which can be blank (ACB closed), ACTIV (ACB open), P-RQS (VIRTEL is awaiting the response from the application for a pending session request), or ACT/S (ACB    in session). cvcstat is the CVC status which can be blank (terminal is free), SERV (terminal is occupied by a service transaction), or INUSE (terminal is occupied by a call). tranid is the external name of the transaction using the terminal. termstat is the terminal status which can be LINKED (terminal is owned by and linked to this line), NOT LK (terminal is owned by this line but not yet linked) OWNED BY m-yyyyyy (terminal prefix matches this line but terminal is owned by another line), UNOWNED (terminal prefix matches this line but terminal is not owned by any line).

\item[{Action}] \leavevmode
None.

\end{description}
\begin{description}
\item[{Module}] \leavevmode
VIR0002

\item[{Meaning}] \leavevmode
This message is appended to the output from a VIRTEL LINE=linename,DISPLAY command when one or more of the terminals attached to the line references a logical pool. It displays a list of terminals belonging to the logical pool.  poolname is the name of the logical pool. For each terminal displayed, termid1 is the name of the terminal in the logical pool, luname is the relay LU name associated with this terminal, printlu is the LU name of the associated virtual printer, and termid2 is the name of the terminal (if any) which is currently using this relay.

\item[{Action}] \leavevmode
None.

\end{description}


\subsubsection{VIR0205I NO TERMINALS ASSOCIATED WITH LINE n-xxxxxx}
\label{\detokenize{messages:vir0205i-no-terminals-associated-with-line-n-xxxxxx}}\begin{description}
\item[{Module}] \leavevmode
VIR0002

\item[{Meaning}] \leavevmode
As a result of a VIRTEL LINE=linename,DISPLAY command, no terminals were found which match the prefix of the requested line.

\item[{Action}] \leavevmode
None.

\end{description}


\subsubsection{VIR0206I LINE n-xxxxxx linetype linestat acbname acbstat}
\label{\detokenize{messages:vir0206i-line-n-xxxxxx-linetype-linestat-acbname-acbstat}}\begin{description}
\item[{Module}] \leavevmode
VIR0002

\item[{Meaning}] \leavevmode
This message is part of the output from a VIRTEL LINE=linename,DISPLAY command lines which have an associated SNA LU. It displays the status of the line. linename is the external name of the line, and n-xxxxxx is the internal  name of the line. linetype is the type of line (/GATE, /FASTC), linestat is the status of the line (STARTED or STOPPED),
acbname is the LU name of the line, acbstat is the line LU status which can be blank (ACB closed), ACTIV (ACB open), or ACT/S (ACB in session).

\item[{Action}] \leavevmode
None.

\end{description}


\subsubsection{VIR0207I LINE n-xxxxxx linetype lineprot linestat}
\label{\detokenize{messages:vir0207i-line-n-xxxxxx-linetype-lineprot-linestat}}\begin{description}
\item[{Module}] \leavevmode
VIR0002

\item[{Meaning}] \leavevmode
This message is part of the output from a VIRTEL LINE=linename,DISPLAY command for lines which do not have an associated SNA LU. It displays the status of the line. linename is the external name of the line, and n-xxxxxx is the internal name of the line. linetype is the type of line (GATE, FASTC, /PCNE, TCPn), lineprot is the protocol (XOT, HTTP, SMTP, PASS, PSIT, NEOX), linestat is the status of the line (STARTED or STOPPED).

\item[{Action}] \leavevmode
None.

\end{description}


\subsubsection{VIR0210E command NOT VALID FOR linetype LINE n-xxxxxx}
\label{\detokenize{messages:vir0210e-command-not-valid-for-linetype-line-n-xxxxxx}}\begin{description}
\item[{Module}] \leavevmode
VIR0002

\item[{Meaning}] \leavevmode
The VIRTEL LINE=linename,START and STOP commands are not valid for lines of type APPCn, GATE or FASTC. This message may also occur for lines of type TCPn or XMn when the corresponding TCPn or XMn parameter is not coded in the VIRTCT.

\item[{Action}] \leavevmode
None.

\end{description}


\subsubsection{VIR0211I LINE linename TRACE ACTIVE VIR0211I TERM termid TRACE ACTIVE}
\label{\detokenize{messages:vir0211i-line-linename-trace-active-vir0211i-term-termid-trace-active}}\begin{description}
\item[{Module}] \leavevmode
VIR0002

\item[{Meaning}] \leavevmode
In response to a TRACE,DISPLAY command, VIRTEL displays a list of lines and terminals for which tracing is active.

\item[{Action}] \leavevmode
Use the NOTRACE command to deactivate tracing if desired.

\end{description}


\subsubsection{VIR0212I LINE linename TRACE INACTIVATED VIR0212I TERM termid TRACE INACTIVATED}
\label{\detokenize{messages:vir0212i-line-linename-trace-inactivated-vir0212i-term-termid-trace-inactivated}}\begin{description}
\item[{Module}] \leavevmode
VIR0002

\item[{Meaning}] \leavevmode
In response to a NOTRACE,ALL command, VIRTEL displays a list of lines and terminals for which tracing has been deactivated.

\item[{Action}] \leavevmode
None.

\end{description}


\subsubsection{VIR0213I NO ACTIVE TRACES}
\label{\detokenize{messages:vir0213i-no-active-traces}}\begin{description}
\item[{Module}] \leavevmode
VIR0002

\item[{Meaning}] \leavevmode
A NOTRACE,ALL command was entered, but no traces were active.

\item[{Action}] \leavevmode
None.

\end{description}
\begin{description}
\item[{Module}] \leavevmode
VIR0002

\item[{Meaning}] \leavevmode
This message is displayed in response to a VIRTEL RELAYS command. 4.xx is the VIRTEL version number and applid is the name of the main VIRTEL application ACB. For each terminal displayed, termid is the name of the terminal in the logical pool, luname is the relay LU name associated with this terminal, applname is the name of the host application to which the relay LU is connected, and n.n.n.n is the IP address of the client workstation (if any).

\item[{Action}] \leavevmode
None.

\end{description}


\subsubsection{VIR0215I NO ACTIVE RELAY ACBS FOR VIRTEL 4.xx APPLID=applid}
\label{\detokenize{messages:vir0215i-no-active-relay-acbs-for-virtel-4-xx-applid-applid}}\begin{description}
\item[{Module}] \leavevmode
VIR0002

\item[{Meaning}] \leavevmode
This message is displayed in response to a VIRTEL RELAYS command when VIRTEL has no VTAM ACB open except for the main VIRTEL application ACB applid.

\item[{Action}] \leavevmode
None.

\end{description}


\subsubsection{VIR0220I termid SCENARIO STOP REQUESTED}
\label{\detokenize{messages:vir0220i-termid-scenario-stop-requested}}\begin{description}
\item[{Module}] \leavevmode
VIR0002

\item[{Meaning}] \leavevmode
A KILL command was entered for terminal termid.

\item[{Action}] \leavevmode
None.

\end{description}


\subsubsection{VIR0230I DYNAMIC ALLOCATION FAILED FOR XXXXXXXX}
\label{\detokenize{messages:vir0230i-dynamic-allocation-failed-for-xxxxxxxx}}\begin{description}
\item[{Module}] \leavevmode
VIR0002A

\item[{Meaning}] \leavevmode
Dynamic allocation has failed for a dataset. Log option will be set to CONSOLE.

\item[{Action}] \leavevmode
Check the VIRTEL log and SYSLOG for further information.

\end{description}


\subsubsection{VIR0231I ERROR OPENING DDNAME XXXXXXXX}
\label{\detokenize{messages:vir0231i-error-opening-ddname-xxxxxxxx}}\begin{description}
\item[{Module}] \leavevmode
VIR0002A

\item[{Meaning}] \leavevmode
An error has occurred trying to open the log output destination. Log option will be set to CONSOLE.

\item[{Action}] \leavevmode
Check the VIRTEL log and SYSLOG for further information.

\end{description}


\subsubsection{VIR0233I VIRTEL XXXXXXXX DATASET HAS BEEN SPUN OFF}
\label{\detokenize{messages:vir0233i-virtel-xxxxxxxx-dataset-has-been-spun-off}}\begin{description}
\item[{Module}] \leavevmode
VIR0002A

\item[{Meaning}] \leavevmode
A log output destination file has been closed and made available.

\item[{Action}] \leavevmode
None.

\end{description}


\subsubsection{VIR0234I CLOSE FAILED FOR XXXXXXXX DATASET}
\label{\detokenize{messages:vir0234i-close-failed-for-xxxxxxxx-dataset}}\begin{description}
\item[{Module}] \leavevmode
VIR0002A

\item[{Meaning}] \leavevmode
An error has occurred closing a log output destination. Log option will be set to CONSOLE.

\item[{Action}] \leavevmode
Check the VIRTEL log and SYSLOG for further information.

\end{description}


\subsubsection{VIR0235I VIRTEL LOG RECORDING TO XXXXXXXX}
\label{\detokenize{messages:vir0235i-virtel-log-recording-to-xxxxxxxx}}\begin{description}
\item[{Module}] \leavevmode
VIR0002A

\item[{Meaning}] \leavevmode
Reports reporing destination for Virtel messages. XXXXXXXX = CONSOLE, LOGFILEx, LOGSTREAM or SYSOUT

\item[{Action}] \leavevmode
None.

\end{description}


\subsubsection{VIR0236I VIRTEL LOGFILE SWITCHED FROM XXXXXXXX TO XXXXXXXX}
\label{\detokenize{messages:vir0236i-virtel-logfile-switched-from-xxxxxxxx-to-xxxxxxxx}}\begin{description}
\item[{Module}] \leavevmode
VIR0002A

\item[{Meaning}] \leavevmode
A log file switch has occurred to the inactive LOGFILE.

\item[{Action}] \leavevmode
None.

\end{description}


\subsubsection{VIR0237I VIRTEL LOG ROUTINE VIR0002A LOADED}
\label{\detokenize{messages:vir0237i-virtel-log-routine-vir0002a-loaded}}\begin{description}
\item[{Module}] \leavevmode
VIR0002

\item[{Meaning}] \leavevmode
“The Virtel log module has been successfully loaded.

\item[{Action}] \leavevmode
None.

\end{description}


\subsubsection{VIR0238I VIRTEL LOG RECORDING TO LOGFILE}
\label{\detokenize{messages:vir0238i-virtel-log-recording-to-logfile}}\begin{description}
\item[{Module}] \leavevmode
VIR0002A

\item[{Meaning}] \leavevmode
Reports reporing destination for Virtel LOGFILE messages.

\item[{Action}] \leavevmode
Use to LOG,D command to display the active LOGFILE.

\end{description}


\subsubsection{VIR0239I VIRTEL LOGGER: xxxxxxxx STRNAME= xxxxxxxx FULL. STRUCTURE OFFLOADED}
\label{\detokenize{messages:vir0239i-virtel-logger-xxxxxxxx-strname-xxxxxxxx-full-structure-offloaded}}\begin{description}
\item[{Module}] \leavevmode
VIR0002A

\item[{Meaning}] \leavevmode
The Virtel logsream has been offloaded.

\item[{Action}] \leavevmode
None.

\end{description}


\subsubsection{VIR0260W SERVICE servname IS A NEW COPY}
\label{\detokenize{messages:vir0260w-service-servname-is-a-new-copy}}\begin{description}
\item[{Module}] \leavevmode
VIR0002

\item[{Meaning}] \leavevmode
A VIRSV,NEW command for service servname has successfully stopped the service.

\item[{Action}] \leavevmode
None.

\end{description}


\subsubsection{VIR0261W SERVICE servname NOT IN MEMORY}
\label{\detokenize{messages:vir0261w-service-servname-not-in-memory}}\begin{description}
\item[{Module}] \leavevmode
VIR0002

\item[{Meaning}] \leavevmode
A VIRSV,NEW command did not complete because the service servname was not started.

\item[{Action}] \leavevmode
None. The service will be started by VIRTEL when required.

\end{description}


\subsubsection{VIR0262W SNAPMSG TRIGGERED - VIRTEL ABENDED \textbar{} SNAP TAKEN}
\label{\detokenize{messages:vir0262w-snapmsg-triggered-virtel-abended-snap-taken}}\begin{description}
\item[{Module}] \leavevmode
VIR0002

\item[{Meaning}] \leavevmode
A message has been trapped by the SNAPMSG function. Action taken will be either a SNAP dump or an ABEND of Virtel.

\item[{Action}] \leavevmode
Determined by TCT parameters or the action set in the SNAPMSG= command.

\end{description}


\subsubsection{VIR0270I DISPLAY 938 VIRTEL TCT=VIRTCTRJ:}
\label{\detokenize{messages:vir0270i-display-938-virtel-tct-virtctrj}}\begin{description}
\item[{Module}] \leavevmode
VIR0002

\item[{Meaning}] \leavevmode
A TCT command has been issued to display the content of some of the VIRTCT parameters. This message shows the parameter details.

\item[{Action}] \leavevmode
None.

\end{description}


\subsubsection{VIR0280I END}
\label{\detokenize{messages:vir0280i-end}}\begin{description}
\item[{Module}] \leavevmode
VIR0002

\item[{Meaning}] \leavevmode
This marks the end of a VIR0270I message.

\item[{Action}] \leavevmode
None.

\end{description}


\subsection{Messages VIR05xxx}
\label{\detokenize{messages:messages-vir05xxx}}

\subsubsection{VIR0504I ACQUIRING TERMINAL luname(relayname) TO LINK n-xxxxxx}
\label{\detokenize{messages:vir0504i-acquiring-terminal-luname-relayname-to-link-n-xxxxxx}}\begin{description}
\item[{Module}] \leavevmode
VIR0005

\item[{Meaning}] \leavevmode
VIRTEL is entering into session with terminal luname.

\item[{Action}] \leavevmode
None.

\end{description}


\subsubsection{VIR0505I  LINKING TERMINAL termid TO n-xxxxxx}
\label{\detokenize{messages:vir0505i-linking-terminal-termid-to-n-xxxxxx}}\begin{description}
\item[{Module}] \leavevmode
VIR0005

\item[{Meaning}] \leavevmode
VIRTEL has linked terminal termid to the line whose internal name is n-xxxxxx. This message can be suppressed by the SILENCE parameter in the VIRTCT.

\item[{Action}] \leavevmode
None.

\end{description}


\subsubsection{VIR0506E ERROR LINKING TERMINAL termid TO n-xxxxxx}
\label{\detokenize{messages:vir0506e-error-linking-terminal-termid-to-n-xxxxxx}}\begin{description}
\item[{Module}] \leavevmode
VIR0005

\item[{Meaning}] \leavevmode
VIRTEL was unable to link terminal termid to the line whose internal name is n-xxxxxx.

\item[{Action}] \leavevmode
Check the terminal definition in VIRTEL and in VTAM.

\end{description}


\subsubsection{VIR0507I LINKING TERMINAL termid TO n-xxxxxx RELAY relayname}
\label{\detokenize{messages:vir0507i-linking-terminal-termid-to-n-xxxxxx-relay-relayname}}\begin{description}
\item[{Module}] \leavevmode
VIR0005

\item[{Meaning}] \leavevmode
VIRTEL linked terminal termid to the line whose internal name is n-xxxxxx. The associated relay is relayname (if specified). This message can be suppressed by the SILENCE parameter in the VIRTCT.

\item[{Action}] \leavevmode
None.

\end{description}


\subsubsection{VIR0508E ERROR: NO AVAILABLE PSEUDO TERMINALS WERE FOUND FOR LINE : n-xxxxxx}
\label{\detokenize{messages:vir0508e-error-no-available-pseudo-terminals-were-found-for-line-n-xxxxxx}}\begin{description}
\item[{Module}] \leavevmode
VIR0005

\item[{Meaning}] \leavevmode
VIRTEL could not activate the line whose internal name is n-xxxxxx because there were no associated terminals defined, or the terminals were owned by another line or belonged to a pool.

\item[{Action}] \leavevmode
Check the definition of line n-xxxxxx. Use the VIRTEL console command LINE=n-xxxxxx,DISPLAY to display the associated terminals. Check that terminals are defined whose name starts with the prefix specified in the line definitions, and that these terminals do not belong to another line or to a pool.

\end{description}


\subsubsection{VIR0509E termid SERVER servname HAS MISSING OR INVALID DIALNO xxxx}
\label{\detokenize{messages:vir0509e-termid-server-servname-has-missing-or-invalid-dialno-xxxx}}\begin{description}
\item[{Module}] \leavevmode
VIR0005

\item[{Meaning}] \leavevmode
A call to server servname from terminal termid failed because the called number is blank or non-numeric.

\item[{Action}] \leavevmode
Check the definition of external server servname. Check that the “Number” field is valid (see “External Servers” in the VIRTEL Connectivity Reference manual). If the value of the “Number” field is blank or “=”, check the called number supplied by the application (CFT, Inter.PEL, STI) which initiated the call. In the case of a VIRKIX application, check that the entry point has a “Mirror” transaction as the first transaction.

\end{description}


\subsubsection{VIR0526W termid IS DISCONNECTED DUE TO TIME-OUT}
\label{\detokenize{messages:vir0526w-termid-is-disconnected-due-to-time-out}}\begin{description}
\item[{Module}] \leavevmode
VIR0005

\item[{Meaning}] \leavevmode
The terminal termid was disconnected because it exceeded the inactivity delay.

\item[{Action}] \leavevmode
None.

\end{description}


\subsubsection{VIR0527E termid CALLER x25callernumber REJECTED AT ENTRY POINT epname}
\label{\detokenize{messages:vir0527e-termid-caller-x25callernumber-rejected-at-entry-point-epname}}\begin{description}
\item[{Module}] \leavevmode
VIR0005

\item[{Meaning}] \leavevmode
An incoming call from X25 number x25callernumber was rejected. The required entry point epname does not exist.

\item[{Action}] \leavevmode
Verify the call user data used in the connection. If it is acceptable, define the corresponding entry point in VIRTEL.

\end{description}


\subsubsection{VIR0528I termid CALLER x25callernumber GETS ENTRY POINT ‘epname’ FROM RULE ‘rulename’}
\label{\detokenize{messages:vir0528i-termid-caller-x25callernumber-gets-entry-point-epname-from-rule-rulename}}\begin{description}
\item[{Module}] \leavevmode
VIR0005

\item[{Meaning}] \leavevmode
VIRTEL allocated entry point epname to an incoming call from X25 number x25callernumber using rule rulename. termid is the name of the VIRTEL terminal which represents the virtual circuit.

\item[{Action}] \leavevmode
None.

\end{description}


\subsubsection{VIR0529E termid REJECTING CONFLICTING IN AND OUT CALLS}
\label{\detokenize{messages:vir0529e-termid-rejecting-conflicting-in-and-out-calls}}\begin{description}
\item[{Module}] \leavevmode
VIR0005

\item[{Meaning}] \leavevmode
After VIRTEL sent an outgoing call packet on terminal termid, an incoming call arrived on the same virtual circuit.

\item[{Action}] \leavevmode
VIRTEL refuses the incoming call. VIRTEL considers the outgoing call as refused.

\end{description}


\subsubsection{VIR0530E termid FAILED CALL TO ‘servname’ THRU LINE n-xxxxxx}
\label{\detokenize{messages:vir0530e-termid-failed-call-to-servname-thru-line-n-xxxxxx}}\begin{description}
\item[{Module}] \leavevmode
VIR0005

\item[{Meaning}] \leavevmode
An incoming call on terminal termid was unable to be rerouted to the line whose internal name is n-xxxxxx. Either the line does not exist, or it is not started, or it is not connected to its partner application, or line n-xxxxxx is configured to disallow outgoing calls.

\item[{Action}] \leavevmode
Display the definition of the external server servname, and check that the line name is correctly specified. Check that the line is in session with the CTCP or partner application. Check the definition of line n-xxxxxx and ensure that the “Possible calls” field is set to 2 or 3.

\end{description}


\subsubsection{VIR0531E termid NO OUTPUT TERMINAL AVAILABLE ON LINE n-xxxxxx}
\label{\detokenize{messages:vir0531e-termid-no-output-terminal-available-on-line-n-xxxxxx}}\begin{description}
\item[{Module}] \leavevmode
VIR0005

\item[{Meaning}] \leavevmode
An incoming call on terminal termid could not be rerouted to the line whose internal name is n-xxxxxx because there were insufficient virtual circuits available.

\item[{Action}] \leavevmode
Display the definition of line n-xxxxxx and press F4. Increase the number of terminals where the “I/O” (“Possible calls”) field is set to 2 or 3.

\end{description}


\subsubsection{VIR0532E termid OUTPUT CALL REFUSED BY EXIT5}
\label{\detokenize{messages:vir0532e-termid-output-call-refused-by-exit5}}\begin{description}
\item[{Module}] \leavevmode
VIR0005

\item[{Meaning}] \leavevmode
An incoming call on terminal termid was unable to be rerouted to the output line because Exit 5 disallowed the outgoing call.

\item[{Action}] \leavevmode
None.

\end{description}


\subsubsection{VIR0533E termid CALL IN PROGRESS}
\label{\detokenize{messages:vir0533e-termid-call-in-progress}}\begin{description}
\item[{Module}] \leavevmode
VIR0005

\item[{Meaning}] \leavevmode
An incoming call on terminal termid was unable to be rerouted to the output line because another call was in progress on the same terminal.

\item[{Action}] \leavevmode
Retry the call later.

\end{description}


\subsubsection{VIR0534E termid EXTERNAL SERVER servname SPECIFIES NON-EXISTENT LINE n-xxxxxx}
\label{\detokenize{messages:vir0534e-termid-external-server-servname-specifies-non-existent-line-n-xxxxxx}}\begin{description}
\item[{Module}] \leavevmode
VIR0005

\item[{Meaning}] \leavevmode
An incoming call on terminal termid was unable to be rerouted because the output line n-xxxxxx specified in external server servname does not exist.

\item[{Action}] \leavevmode
Display the definition of the external server servname, and check that the line name is correctly specified.

\end{description}


\subsubsection{VIR0535E termid LINE n-xxxxxx(linename) DOES NOT PERMIT CALLS TO SERVER}
\label{\detokenize{messages:vir0535e-termid-line-n-xxxxxx-linename-does-not-permit-calls-to-server}}\begin{description}
\item[{Module}] \leavevmode
VIR0005

\item[{Meaning}] \leavevmode
An incoming call on terminal termid was unable to be rerouted to the line whose internal name is n-xxxxxx and whose external name is linename. Line n-xxxxxx is configured to disallow outgoing calls.

\item[{Action}] \leavevmode
Check the definition of line n-xxxxxx and ensure that the “Possible calls” field is set to 2 or 3.

\end{description}


\subsubsection{VIR0536E termid LINE n-xxxxxx(linename) IS NOT STARTED}
\label{\detokenize{messages:vir0536e-termid-line-n-xxxxxx-linename-is-not-started}}\begin{description}
\item[{Module}] \leavevmode
VIR0005

\item[{Meaning}] \leavevmode
An incoming call on terminal termid was unable to be rerouted to the line whose internal name is n-xxxxxx and whose external name is linename. The line is not started or it is not connected to its partner application.

\item[{Action}] \leavevmode
Check that the line is started and in session with the CTCP or partner application.

\end{description}


\subsubsection{VIR0537E termid LINE n-xxxxxx(linename) HAS NO TERMINALS LINKED}
\label{\detokenize{messages:vir0537e-termid-line-n-xxxxxx-linename-has-no-terminals-linked}}\begin{description}
\item[{Module}] \leavevmode
VIR0005

\item[{Meaning}] \leavevmode
An incoming call on terminal termid was unable to be rerouted to the line whose internal name is n-xxxxxx and whose external name is linename. The line has no terminals linked.

\item[{Action}] \leavevmode
Enter the VIRTEL command L=linename,D at the console to determine if the line has terminals and why they are   not linked. Check the definition of line n-xxxxxx and ensure that the appropriate terminals are defined and do not duplicate those of another line.

\end{description}


\subsubsection{VIR0538I termid GETS ENTRY POINT ‘epname’ FROM RULE ‘rulename’}
\label{\detokenize{messages:vir0538i-termid-gets-entry-point-epname-from-rule-rulename}}\begin{description}
\item[{Module}] \leavevmode
VIR0005

\item[{Meaning}] \leavevmode
VIRTEL allocated entry point epname to an outgoing call from a GATE or PCNE application. termid is the name of the VIRTEL terminal which represents the virtual circuit.

\item[{Action}] \leavevmode
None.

\end{description}


\subsubsection{VIR0539I termid CALLER x25callernumber GETS DEFAULT ENTRY POINT FOR LINE n-xxxxxx}
\label{\detokenize{messages:vir0539i-termid-caller-x25callernumber-gets-default-entry-point-for-line-n-xxxxxx}}\begin{description}
\item[{Module}] \leavevmode
VIR0005

\item[{Meaning}] \leavevmode
An incoming call from X25 number x25callernumber did not match any of the rules attached to the line whose internal name is n-xxxxxx. termid is the name of the VIRTEL terminal which represents the virtual circuit.

\item[{Action}] \leavevmode
VIRTEL uses the default entry point associated with line n-xxxxxx.

\end{description}


\subsubsection{VIR0540I termid GETS DEFAULT ENTRY POINT FOR LINE linename}
\label{\detokenize{messages:vir0540i-termid-gets-default-entry-point-for-line-linename}}\begin{description}
\item[{Module}] \leavevmode
VIR0005

\item[{Meaning}] \leavevmode
An outgoing call from a GATE or PCNE application did not match any of the rules attached to the line whose external name is linename. termid is the name of the VIRTEL terminal which represents the virtual circuit.

\item[{Action}] \leavevmode
VIRTEL uses the default entry point associated with line linename.

\end{description}


\subsubsection{VIR0541I termid OUTBOUND GATE\textbar{}FAST CALL FROM ctcpappl VIA mchlu TO x25callednumber}
\label{\detokenize{messages:vir0541i-termid-outbound-gate-fast-call-from-ctcpappl-via-mchlu-to-x25callednumber}}\begin{description}
\item[{Module}] \leavevmode
VIR0005

\item[{Meaning}] \leavevmode
The CTCP application ctcpappl sent an outgoing call in GATE or Fast Connect mode via MCH mchlu to the X25 number x25callednumber. termid is the name of the VIRTEL terminal which represents the virtual circuit.

\item[{Action}] \leavevmode
None.

\end{description}


\subsubsection{VIR0542I termid OUTBOUND PCNE CALL FROM pcneappl VIA pcnelu}
\label{\detokenize{messages:vir0542i-termid-outbound-pcne-call-from-pcneappl-via-pcnelu}}\begin{description}
\item[{Module}] \leavevmode
VIR0005

\item[{Meaning}] \leavevmode
The PCNE application pcneappl sent an outgoing call via LU pcnelu. The X25 called number will be determined by VIRTEL using the external server definition pcnelu. termid is the name of the VIRTEL terminal which represents the virtual circuit.

\item[{Action}] \leavevmode
None.

\end{description}


\subsubsection{VIR0545I termid CALL CLEARED BY VIRTEL CAUSE=xx DIAG=yy}
\label{\detokenize{messages:vir0545i-termid-call-cleared-by-virtel-cause-xx-diag-yy}}\begin{description}
\item[{Module}] \leavevmode
VIR0005

\item[{Meaning}] \leavevmode
VIRTEL terminated the call on terminal termid. xx and yy are the cause and diagnostic codes generated by VIRTEL. The following codes are possible:
- Cause=00 Diag=02 : No terminals available on output line
- Cause=0D Diag=40 : Call could not be routed to output line
- Cause=xx Diag=yy : Error sending to local PCNE or GATE application (xx,yy = VTAM RTNCD/FDBK codes)
- Cause=EE Diag=EE : Call refused, no master (QLLC 3174 emulation)

\item[{Action}] \leavevmode
For Cause=00, Diag=02: Check the definition of the line. For other codes, examine preceding error messages to determine the cause.

\end{description}


\subsubsection{VIR0550I ROUTAGE REJECTED FOR LUNAME luname CAUSE=xx DIAGNOSTIC=dd}
\label{\detokenize{messages:vir0550i-routage-rejected-for-luname-luname-cause-xx-diagnostic-dd}}\begin{description}
\item[{Module}] \leavevmode
VIR0005

\item[{Meaning}] \leavevmode
TELETEL re-routing was refused by the TELETEL PAD for terminal luname.

\item[{Action}] \leavevmode
Refer to the SAPONET documentation for the values of the codes xx and dd also check that your TELETEL subscription supports re-rerouting .

\end{description}


\subsubsection{VIR0551I termid CONNECTED TO EXTERNAL SERVICE ssssssss}
\label{\detokenize{messages:vir0551i-termid-connected-to-external-service-ssssssss}}\begin{description}
\item[{Module}] \leavevmode
VIR0005

\item[{Meaning}] \leavevmode
The terminal termid is connected to the external server sssssss via a logical channel used in fast connect mode.

\item[{Action}] \leavevmode
None.

\end{description}


\subsubsection{VIR0552I termid DISCONNECTED AFTER nn MINUTES}
\label{\detokenize{messages:vir0552i-termid-disconnected-after-nn-minutes}}\begin{description}
\item[{Module}] \leavevmode
VIR0005

\item[{Meaning}] \leavevmode
The terminal termid used in Fast-Connect mode was disconnected after nn minutes of connetion time.

\item[{Action}] \leavevmode
None.

\end{description}


\subsubsection{VIR0553I RESET REQUEST RECEIVED FROM luname CAUSE=xx DIAGNOSTIC=dd}
\label{\detokenize{messages:vir0553i-reset-request-received-from-luname-cause-xx-diagnostic-dd}}\begin{description}
\item[{Module}] \leavevmode
VIR0005

\item[{Meaning}] \leavevmode
A re-initialisation request was received by the PAD for the terminal luname.

\item[{Action}] \leavevmode
Refer to the SAPONET documentation to determine the cause and diagnostic.

\end{description}


\subsubsection{VIR0554I pseudo-terminal LINKED TO service THRU terminal}
\label{\detokenize{messages:vir0554i-pseudo-terminal-linked-to-service-thru-terminal}}\begin{description}
\item[{Module}] \leavevmode
VIR0005

\item[{Meaning}] \leavevmode
Initialisation of service indicated on the pseudo terminal indicated via the terminal indicated.

\item[{Action}] \leavevmode
None.

\end{description}


\subsubsection{VIR0555E INVALID RULE rulename ENTRY POINT ‘epname’}
\label{\detokenize{messages:vir0555e-invalid-rule-rulename-entry-point-epname}}\begin{description}
\item[{Module}] \leavevmode
VIR0005

\item[{Meaning}] \leavevmode
Rule rulename used by an X25 line specifies a non-existent entry point epname.

\item[{Action}] \leavevmode
Check the entry point name specified in the rule definition.

\end{description}


\subsection{Messages VIR06xxx}
\label{\detokenize{messages:messages-vir06xxx}}

\subsubsection{VIR0601I VIRSTATx status DSN=dsname}
\label{\detokenize{messages:vir0601i-virstatx-status-dsn-dsname}}\begin{description}
\item[{Module}] \leavevmode
VIR0006

\item[{Meaning}] \leavevmode
This message displays the status of a VIRSTATx file in response to a STAT,D command. The status can be:
\begin{itemize}
\item {} 
\sphinxstylestrong{* IN USE *} : VIRTEL is currently recording statistics to this file

\item {} 
AVAILABLE : This file is available for use

\item {} 
OFFLOAD REQUIRED : This file contains statistical data and cannot be reused until the data has been offloaded using the STATCOPY batch job

\item {} 
NOT AVAILABLE : This file cannot be allocated

\end{itemize}

\item[{Action}] \leavevmode
None.

\end{description}


\subsubsection{VIR0603I VIRSTATx OFFLOAD REQUIRED DSN=dsname}
\label{\detokenize{messages:vir0603i-virstatx-offload-required-dsn-dsname}}\begin{description}
\item[{Module}] \leavevmode
VIR0006

\item[{Meaning}] \leavevmode
VIRTEL has determined that the VIRSTATx file is full.

\item[{Action}] \leavevmode
Run the STATCOPY batch job to empty the file. This message is designed to be trapped by an automated operator.

\end{description}


\subsubsection{VIR0604I VIRSTAT NOW RECORDING ON VIRSTATx DSN=dsname}
\label{\detokenize{messages:vir0604i-virstat-now-recording-on-virstatx-dsn-dsname}}\begin{description}
\item[{Module}] \leavevmode
VIR0006

\item[{Meaning}] \leavevmode
VIRTEL has started recording statistics on the VIRSTATx file.

\item[{Action}] \leavevmode
None.

\end{description}


\subsubsection{VIR0605E NO AVAILABLE VIRSTAT FILES -VIRTEL TERMINATING}
\label{\detokenize{messages:vir0605e-no-available-virstat-files-virtel-terminating}}\begin{description}
\item[{Module}] \leavevmode
VIR0006

\item[{Meaning}] \leavevmode
All of the VIRSTATx files are full or unavailable, and the VIRTCT specifies STATS=(MULTI,TERMINATE).

\item[{Action}] \leavevmode
VIRTEL terminates.

\end{description}


\subsubsection{VIR0606I VIRSTAT RECORDING SUSPENDED - RUN OFFLOAD AND ISSUE F virtelstc,STAT,I}
\label{\detokenize{messages:vir0606i-virstat-recording-suspended-run-offload-and-issue-f-virtelstc-stat-i}}\begin{description}
\item[{Module}] \leavevmode
VIR0006

\item[{Meaning}] \leavevmode
All of the VIRSTATx files are full or unavailable, and the VIRTCT specifies STATS=(MULTI,CONTINUE).

\item[{Action}] \leavevmode
VIRTEL continues processing, but statistics will no longer be recorded. To restart statistics recording, use the VIRTEL STAT,I command.

\end{description}


\subsubsection{VIR0607E VIRSTATx ALLOC ERR=errc-infc DSN=dsname}
\label{\detokenize{messages:vir0607e-virstatx-alloc-err-errc-infc-dsn-dsname}}\begin{description}
\item[{Module}] \leavevmode
VIR0006

\item[{Meaning}] \leavevmode
VIRTEL cannot allocate the VIRSTATx file. errc is the error code, and infc is the info code from dynamic allocation.  For the meaning of these codes, see OS/390 MVS Programming: Authorized Assembler Services Guide under the heading Interpreting DYNALLOC Return Codes. Commonly encountered codes are: 0210-0000 File in use by another job; 0218-0000 DASD volume not mounted; 1708-0002 File not cataloged.

\item[{Action}] \leavevmode
VIRTEL uses the next VIRSTATx file.

\end{description}


\subsubsection{VIR0608E VIRSTATx DEALC ERR=errc-infc DSN=dsname}
\label{\detokenize{messages:vir0608e-virstatx-dealc-err-errc-infc-dsn-dsname}}\begin{description}
\item[{Module}] \leavevmode
VIR0006

\item[{Meaning}] \leavevmode
An error occurred during deallocation of the VIRSTATx file. errc is the error code, and infc is the info code from dynamic allocation.

\item[{Action}] \leavevmode
See OS/390 MVS Programming: Authorized Assembler Services Guide under the heading Interpreting DYNALLOC Return Codes.

\end{description}


\subsubsection{VIR0609E VIRSTATx synadaf DSN=dsname}
\label{\detokenize{messages:vir0609e-virstatx-synadaf-dsn-dsname}}\begin{description}
\item[{Module}] \leavevmode
VIR0006

\item[{Meaning}] \leavevmode
An error occurred when VIRTEL attempted to write to the VIRSTATx file. synadaf is the message generated by MVS.

\item[{Action}] \leavevmode
Check the DCB attributes of the file.

\end{description}


\subsubsection{VIR0610E VIRSTATx OPEN/CLOSE/GET/WRITE ABEND=ccc-rc DSN=dsname}
\label{\detokenize{messages:vir0610e-virstatx-open-close-get-write-abend-ccc-rc-dsn-dsname}}\begin{description}
\item[{Module}] \leavevmode
VIR0006

\item[{Meaning}] \leavevmode
An error occurred when VIRTEL attempted to access the VIRSTATx file. ccc is the system completion code, and rc is the return code. For the meaning of these codes, refer to the OS/390 MVS System Codes manual under the heading System Completion Codes. The most commonly encountered codes are B37 or D37, which indicate that the file is full.

\item[{Action}] \leavevmode
VIRTEL switches automatically to the next VIRSTATx file.

\end{description}


\subsubsection{VIR0611I VIRSTAT NOW RECORDING TO SMF}
\label{\detokenize{messages:vir0611i-virstat-now-recording-to-smf}}\begin{description}
\item[{Module}] \leavevmode
VIR0006

\item[{Meaning}] \leavevmode
VIRTEL is recording the VIRSTAT statistics to SMF.

\item[{Action}] \leavevmode
None.

\end{description}


\subsubsection{VIR0612I VIRSTAT SMFWTM FAILED.RC=rc}
\label{\detokenize{messages:vir0612i-virstat-smfwtm-failed-rc-rc}}\begin{description}
\item[{Module}] \leavevmode
VIR0006

\item[{Meaning}] \leavevmode
An error occurred when VIRTEL attempted to write a SMF record with the IBM SMFWTM macro. For the meaning of these codes, refer to the z/OS IBM MVS System Management Facilities Guide.

\item[{Action}] \leavevmode
VIRTEL STATS SMF recording is disabled.

\end{description}


\subsection{Messages VIR07xxx}
\label{\detokenize{messages:messages-vir07xxx}}

\subsubsection{VIR0700W ERROR IN TASK taskname PSW= xxxxxxxx xxxxxxxx}
\label{\detokenize{messages:vir0700w-error-in-task-taskname-psw-xxxxxxxx-xxxxxxxx}}\begin{description}
\item[{Module}] \leavevmode
VIR0007

\item[{Meaning}] \leavevmode
A fatal error occurred in taskname (MAIN, VSAM, STAT, LOAD) of VIRTEL. VIRTEL will shut down after printing a dump.

\item[{Action}] \leavevmode
See message VIR0015S and VIR0016W.

\end{description}


\subsection{Messages VIR08xxx}
\label{\detokenize{messages:messages-vir08xxx}}

\subsubsection{VIR0800I VIRTEL IS USING NO SECURITY}
\label{\detokenize{messages:vir0800i-virtel-is-using-no-security}}\begin{description}
\item[{Module}] \leavevmode
Security

\item[{Meaning}] \leavevmode
As a result of the SECUR=NO parameter specified in the VIRTCT at startup, VIRTEL is running without security.

\item[{Action}] \leavevmode
None.

\end{description}


\subsubsection{VIR0810I VIRTEL IS USING INTERNAL SECURITY}
\label{\detokenize{messages:vir0810i-virtel-is-using-internal-security}}\begin{description}
\item[{Module}] \leavevmode
Security

\item[{Meaning}] \leavevmode
As a result of the SECUR=VIRTEL parameter specified in the VIRTCT at startup, VIRTEL is using its own internal security management.

\item[{Action}] \leavevmode
None.

\end{description}


\subsubsection{VIR0820I VIRTEL IS USING TSS SECURITY}
\label{\detokenize{messages:vir0820i-virtel-is-using-tss-security}}\begin{description}
\item[{Module}] \leavevmode
Security

\item[{Meaning}] \leavevmode
As a result of the SECUR=TOPS parameter specified in the VIRTCT at startup, VIRTEL is using the TOP SECRET security system without RACROUTE.

\item[{Action}] \leavevmode
None.

\end{description}


\subsubsection{VIR0830I VIRTEL IS USING RACF SECURITY}
\label{\detokenize{messages:vir0830i-virtel-is-using-racf-security}}\begin{description}
\item[{Module}] \leavevmode
Security

\item[{Meaning}] \leavevmode
As a result of the SECUR=RACF parameter specified in the VIRTCT at startup, VIRTEL is using the RACF security system without RACROUTE.

\item[{Action}] \leavevmode
None.

\end{description}


\subsubsection{VIR0840I VIRTEL IS USING RACROUTE SECURITY WITH TSS}
\label{\detokenize{messages:vir0840i-virtel-is-using-racroute-security-with-tss}}\begin{description}
\item[{Module}] \leavevmode
Security

\item[{Meaning}] \leavevmode
As a result of the SECUR=(RACROUTE,TOPS) parameter specified in the VIRTCT at startup, VIRTEL is using the TOP SECRET security system with RACROUTE.

\item[{Action}] \leavevmode
None.

\end{description}


\subsubsection{VIR0843I MIXED-CASE PASSWORD SUPPORT IS ACTIVE}
\label{\detokenize{messages:vir0843i-mixed-case-password-support-is-active}}\begin{description}
\item[{Module}] \leavevmode
Security

\item[{Meaning}] \leavevmode
The user has defined the MIXEDCASE subparameter in the SECUR TCT keyword for TOP SECRET. This indicates to VIRTEL that it should support lower-case characters in passwords. Consequently VIRTEL will pass passwords to the security manager exactly as entered by the user, without translating them to upper case.

\item[{Action}] \leavevmode
None.

\end{description}


\subsubsection{VIR0850I VIRTEL IS USING ACF2 SECURITY}
\label{\detokenize{messages:vir0850i-virtel-is-using-acf2-security}}\begin{description}
\item[{Module}] \leavevmode
Security

\item[{Meaning}] \leavevmode
VIRTEL is running under VM with the SECUR=ACF2 parameter specified in the VIRTCT.

\item[{Action}] \leavevmode
None.

\end{description}


\subsubsection{VIR0860I VIRTEL IS USING RACROUTE SECURITY}
\label{\detokenize{messages:vir0860i-virtel-is-using-racroute-security}}\begin{description}
\item[{Module}] \leavevmode
Security

\item[{Meaning}] \leavevmode
As a result of the SECUR=RACROUTE parameter specified in the VIRTCT at startup, VIRTEL is using the RACROUTE interface to RACF, ACF2, or other compatible security sub-system.

\item[{Action}] \leavevmode
None.

\end{description}


\subsubsection{VIR0861I MIXED-CASE PASSWORD SUPPORT IS ACTIVE}
\label{\detokenize{messages:vir0861i-mixed-case-password-support-is-active}}\begin{description}
\item[{Module}] \leavevmode
Security

\item[{Meaning}] \leavevmode
The security manager (RACF or compatible sub-system) has indicated to VIRTEL that it supports lower-case characters in passwords. Consequently VIRTEL will pass passwords to the security manager exactly as entered by the user, without translating them to upper case.

\item[{Action}] \leavevmode
None.

\end{description}


\subsubsection{VIR0863I MIXED-CASE PASSWORD SUPPORT IS ACTIVE}
\label{\detokenize{messages:vir0863i-mixed-case-password-support-is-active}}\begin{description}
\item[{Module}] \leavevmode
Security

\item[{Meaning}] \leavevmode
The user has defined the MIXEDCASE subparameter in the SECUR TCT keyword. This indicates to VIRTEL that it should support lower-case characters in passwords. Consequently VIRTEL will pass passwords to the security manager exactly as entered by the user, without translating them to upper case.

\item[{Action}] \leavevmode
None.

\end{description}


\subsection{Messages VIR09xxx}
\label{\detokenize{messages:messages-vir09xxx}}

\subsubsection{VIR0900I LICENCE VIRTEL licence information}
\label{\detokenize{messages:vir0900i-licence-virtel-licence-information}}\begin{description}
\item[{Module}] \leavevmode
VIR0009

\item[{Meaning}] \leavevmode
VIRTEL licence information messagee.

\item[{Action}] \leavevmode
None.

\end{description}


\subsubsection{VIR0901I LICENCE IS ABOUT TO EXPIRE}
\label{\detokenize{messages:vir0901i-licence-is-about-to-expire}}\begin{description}
\item[{Module}] \leavevmode
VIR0009

\item[{Meaning}] \leavevmode
VIRTEL licence renewal has been triggered as determined by the TCT option WARNING=.

\item[{Action}] \leavevmode
Contact Virtel Support to renew licence.

\end{description}


\subsubsection{VIR0904I ACQUIRING TERMINAL relayname(termid)}
\label{\detokenize{messages:vir0904i-acquiring-terminal-relayname-termid}}\begin{description}
\item[{Module}] \leavevmode
VIR0009

\item[{Meaning}] \leavevmode
VIRTEL is entering into session with terminal termid associated with relay relayname.

\item[{Action}] \leavevmode
None.

\end{description}


\subsubsection{VIR0905I termid RELEASED}
\label{\detokenize{messages:vir0905i-termid-released}}\begin{description}
\item[{Module}] \leavevmode
VIR0009

\item[{Meaning}] \leavevmode
The terminal termid is released.

\item[{Action}] \leavevmode
None.

\end{description}


\subsubsection{VIR0905W UNABLE TO ACTIVATE RELAY relayname(termid)}
\label{\detokenize{messages:vir0905w-unable-to-activate-relay-relayname-termid}}\begin{description}
\item[{Module}] \leavevmode
VIR0009

\item[{Meaning}] \leavevmode
The relay relayname linked to the terminal termid cannot be activated.

\item[{Action}] \leavevmode
Check that the VTAM node containing the relay relayname has been activated, verify that the relay is not already activated for another terminal, and that the terminal is correctly defined in VIRTEL.

\end{description}


\subsubsection{VIR0906I applid CONNECTING LUTYPE n PRINTER prterm(luname) TO termid}
\label{\detokenize{messages:vir0906i-applid-connecting-lutype-n-printer-prterm-luname-to-termid}}\begin{description}
\item[{Module}] \leavevmode
VIR0009

\item[{Meaning}] \leavevmode
Application applname has connected printer of LU type n terminal name prterm and LU name luname to VIRTEL terminal name termid.

\item[{Action}] \leavevmode
None.

\end{description}


\subsubsection{VIR0914E termid ERROR R15=xx R0=yy CONNECTING luname TO applid}
\label{\detokenize{messages:vir0914e-termid-error-r15-xx-r0-yy-connecting-luname-to-applid}}\begin{description}
\item[{Module}] \leavevmode
VIR0009

\item[{Meaning}] \leavevmode
VIRTEL could not establish a session between VIRTEL LU luname and the ACB applid of a partner application. xx and yy are the hexadecimal error codes from REQSESS.

\item[{Action}] \leavevmode
If R15=00000020, activate the LU luname in VTAM, then enter a VIRTEL START command for the line. For any other value of R15, check that the application applid is active and ready to receive connections, and that the LU luname is in CONCT status.

\end{description}


\subsubsection{VIR0915E termid(luname) SESSION REQUEST REFUSED BY applid SENSE=xxxxxxx}
\label{\detokenize{messages:vir0915e-termid-luname-session-request-refused-by-applid-sense-xxxxxxx}}\begin{description}
\item[{Module}] \leavevmode
VIR0009

\item[{Meaning}] \leavevmode
VIRTEL terminal termid could not establish a session between VIRTEL LU luname and partner application applid. The application rejected the session request with sense code xxxxxxxx.

\item[{Action}] \leavevmode
Check the message log of the partner application (for example, the CICS MSGUSR file) to determine why the application rejected the session request. Refer to the IBM SNA Formats manual to determine the meaning of the sense code. Sense code 08010000 may indicate that the CICS terminal is out of service.

\end{description}


\subsubsection{VIR0918W termid RELAY relayname INACTIVATED}
\label{\detokenize{messages:vir0918w-termid-relay-relayname-inactivated}}\begin{description}
\item[{Module}] \leavevmode
VIR0009

\item[{Meaning}] \leavevmode
VIRTEL has closed the relay relayname associated with the terminal termid.

\item[{Action}] \leavevmode
None.

\end{description}


\subsubsection{VIR0919I termid RELAY relayname ACTIVATED}
\label{\detokenize{messages:vir0919i-termid-relay-relayname-activated}}\begin{description}
\item[{Module}] \leavevmode
VIR0009

\item[{Meaning}] \leavevmode
VIRTEL has opened the relay relayname associated with the terminal termid.

\item[{Action}] \leavevmode
None.

\end{description}


\subsubsection{VIR0920E RELAY relayname ALREADY ACTIVE}
\label{\detokenize{messages:vir0920e-relay-relayname-already-active}}\begin{description}
\item[{Module}] \leavevmode
VIR0009

\item[{Meaning}] \leavevmode
The application relay relayname is already active and in session with another terminal.

\item[{Action}] \leavevmode
Check that the same relay has not been allocated to two different terminals.

\end{description}


\subsubsection{VIR0921W NO MORE RELAY AVAILABLE FOR termid POOL ‘poolname’ PREFIX ‘prefix’}
\label{\detokenize{messages:vir0921w-no-more-relay-available-for-termid-pool-poolname-prefix-prefix}}\begin{description}
\item[{Module}] \leavevmode
VIR0009

\item[{Meaning}] \leavevmode
Terminal termid attempted to connect, but there is no active relay in pool poolname, or all the relays in the pool are in use, or there is no available relay with the specified prefix. The connection is refused by VIRTEL.

\item[{Action}] \leavevmode
Check that the VTAM node containing the relays is activated. Increase the number of relays in the pool poolname.  If the message contains a PREFIX then there is a conflict between the prefix specified in the transaction and the terminal names of the relays in the pool.

\end{description}


\subsubsection{VIR0922W NBTERM IS TOO SMALL}
\label{\detokenize{messages:vir0922w-nbterm-is-too-small}}\begin{description}
\item[{Module}] \leavevmode
VIR0009

\item[{Meaning}] \leavevmode
The number of events awaiting simultaneous processing has exceeded the estimated maximum.

\item[{Action}] \leavevmode
VIRTEL automatically allocates additional memory and continues processing. To avoid this message, increase the value of the NBTERM parameter in the VIRTCT.

\end{description}


\subsubsection{VIR0923E NO RELAY AVAILABLE FOR termid POOL ‘poolname’ WITH NAME ‘luname’}
\label{\detokenize{messages:vir0923e-no-relay-available-for-termid-pool-poolname-with-name-luname}}\begin{description}
\item[{Module}] \leavevmode
VIR0009

\item[{Meaning}] \leavevmode
Terminal termid could not allocate the LU luname required by the incoming call because pool poolname does not contain the relay luname.

\item[{Action}] \leavevmode
Check the definitions of terminal termid and pool poolname (F2 from the configuration menu, or F5 from the sub-application system services menu). Check the definition of the rule which matched the incoming call. The relay specified in the “Parameter” field of the rule must exist in the pool.

\end{description}


\subsubsection{VIR0924E termid RELAY relayname COULD NOT BE ACTIVATED}
\label{\detokenize{messages:vir0924e-termid-relay-relayname-could-not-be-activated}}\begin{description}
\item[{Module}] \leavevmode
VIR0009

\item[{Meaning}] \leavevmode
VIRTEL could not allocate the relay LU named relayname. Either the terminal required by the transaction does not exist in the pool, or the relay is unavailable or is allocated to another line.

\item[{Action}] \leavevmode
Check the definitions of the terminal and transaction.

\end{description}


\subsubsection{VIR0925E UNABLE TO ACTIVATE luname (termid) ERROR: errcode}
\label{\detokenize{messages:vir0925e-unable-to-activate-luname-termid-error-errcode}}\begin{description}
\item[{Module}] \leavevmode
VIR0009

\item[{Meaning}] \leavevmode
VIRTEL could not open the relay luname for the terminal termid. errcode is the ACBERFLG in hexadecimal: 58=APPL already opened by another application, 5A=APPL inactive.

\item[{Action}] \leavevmode
Check that the VTAM node containing the relay has been activated. The command D NET,ID=luname,E should show luname defined as a VTAM APPL in CONCT status.

\end{description}


\subsubsection{VIR0951I termid CONNECTED TO EXTERNAL SERVICE ssssssss}
\label{\detokenize{messages:vir0951i-termid-connected-to-external-service-ssssssss}}\begin{description}
\item[{Module}] \leavevmode
VIR0009

\item[{Meaning}] \leavevmode
The terminal termid is connected to external server ssssssss via a logical channel using non Fast-Connect mode.

\item[{Action}] \leavevmode
None.

\end{description}


\subsubsection{VIR0956S NO MORE OSCORE AVAILABLE}
\label{\detokenize{messages:vir0956s-no-more-oscore-available}}\begin{description}
\item[{Module}] \leavevmode
VIR0009

\item[{Meaning}] \leavevmode
VIRTEL does not have sufficient memory to load a module.

\item[{Action}] \leavevmode
Check the OSCORE parameter of the VIRTCT.

\end{description}


\subsubsection{VIR0959E termid CANNOT CONNECT : THIS IS A DEMONSTRATION SYSTEM WITH LIMITED RESOURCES}
\label{\detokenize{messages:vir0959e-termid-cannot-connect-this-is-a-demonstration-system-with-limited-resources}}\begin{description}
\item[{Module}] \leavevmode
VIR0009

\item[{Meaning}] \leavevmode
The terminal termid has attempted to establish a session with a VTAM application, but the limit on the number of active sessions permitted by your VIRTEL license has already been reached.

\item[{Action}] \leavevmode
Wait until another terminal disconnects, or contact Syspertec to upgrade your license.

\end{description}


\subsection{Messages VIR10xxx}
\label{\detokenize{messages:messages-vir10xxx}}

\subsubsection{VIR1021I VIRARBO type RECORD name ADDED/UPDATED/DELETED BY USER userid AT TERMINAL termid}
\label{\detokenize{messages:vir1021i-virarbo-type-record-name-added-updated-deleted-by-user-userid-at-terminal-termid}}\begin{description}
\item[{Module}] \leavevmode
VIR0010

\item[{Meaning}] \leavevmode
User userid at terminal termid has added, updated, or deleted a record in the VIRTEL configuration file. type is the record type (LINE, TERMINAL, SERVER, …) and name is the entity name.

\item[{Action}] \leavevmode
None.

\end{description}


\subsubsection{VIR1062W ERROR ACCESSING FILE filename}
\label{\detokenize{messages:vir1062w-error-accessing-file-filename}}\begin{description}
\item[{Module}] \leavevmode
VIR0010

\item[{Meaning}] \leavevmode
VIRTEL has detected an error accessing file filename.

\item[{Action}] \leavevmode
Contact technical support.

\end{description}


\subsubsection{VIR1063W ERROR : FILE filename NOT FOUND IN VIRTCT}
\label{\detokenize{messages:vir1063w-error-file-filename-not-found-in-virtct}}\begin{description}
\item[{Module}] \leavevmode
VIR0010

\item[{Meaning}] \leavevmode
File filename is not defined in the VIRTCT.

\item[{Action}] \leavevmode
Add the definition and reassemble the VIRTCT.

\end{description}


\subsubsection{VIR1064E ERROR : TERMINAL termid FAILED TO SEND MAIL}
\label{\detokenize{messages:vir1064e-error-terminal-termid-failed-to-send-mail}}\begin{description}
\item[{Module}] \leavevmode
VIR0010

\item[{Meaning}] \leavevmode
VIRTEL detected an SMTP protocol error.

\item[{Action}] \leavevmode
Contact technical support.

\end{description}


\subsubsection{VIR1065W LINE linename IS UNKNOWN}
\label{\detokenize{messages:vir1065w-line-linename-is-unknown}}\begin{description}
\item[{Module}] \leavevmode
VIR0010

\item[{Meaning}] \leavevmode
A START or STOP command entered at the “Status of Lines” screen refers to a line unknown to VIRTEL.

\item[{Action}] \leavevmode
Contact technical support.

\end{description}


\subsubsection{VIR1066W LINE linename ALREADY ACTIVE}
\label{\detokenize{messages:vir1066w-line-linename-already-active}}\begin{description}
\item[{Module}] \leavevmode
VIR0010

\item[{Meaning}] \leavevmode
A START command was entered at the “Status of Lines” screen for a line which was already started.

\item[{Action}] \leavevmode
None.

\end{description}


\subsubsection{VIR1067W LINE linename (n-xxxxxx) START / STOP REQUESTED}
\label{\detokenize{messages:vir1067w-line-linename-n-xxxxxx-start-stop-requested}}\begin{description}
\item[{Module}] \leavevmode
VIR0010

\item[{Meaning}] \leavevmode
VIRTEL processed a START or STOP command from the “Status of Lines” screen on the line with external name linename and internal name n-xxxxxx.

\item[{Action}] \leavevmode
None.

\end{description}


\subsubsection{VIR1068E command NOT VALID FOR linetype LINE n-xxxxxx}
\label{\detokenize{messages:vir1068e-command-not-valid-for-linetype-line-n-xxxxxx}}\begin{description}
\item[{Module}] \leavevmode
VIR0010

\item[{Meaning}] \leavevmode
START and STOP commands entered at the “Status of Lines” screen are not valid for lines of type APPCn, GATE or FASTC. This message may also be issued for lines of type TCPn or XMn if the corresponding TCPn or XMn parameter is not coded in the VIRTCT.

\item[{Action}] \leavevmode
None.

\end{description}


\subsection{Messages VIR11xxx}
\label{\detokenize{messages:messages-vir11xxx}}

\subsubsection{VIR11D1W termid ERROR LOADING SCRIPT scriptnam IN TRANSACTION tranid}
\label{\detokenize{messages:vir11d1w-termid-error-loading-script-scriptnam-in-transaction-tranid}}\begin{description}
\item[{Module}] \leavevmode
VIR0011D

\item[{Meaning}] \leavevmode
Transaction tranid at terminal termid called for a scenario module which could not be loaded.

\item[{Action}] \leavevmode
Check the log for a previous message VIR0031W. Check the module name specified in the “Input Message Exit” or “Output Message Exit” field of transaction tranid. Ensure that this module exists in the VIRTEL load library and is a valid scenario module. Recompile the scenario using the current version of the VIRTEL SCRNAPI macro library.

\end{description}


\subsubsection{VIR11D2W termid INVALID FA39 SCRIPT RECEIVED}
\label{\detokenize{messages:vir11d2w-termid-invalid-fa39-script-received}}\begin{description}
\item[{Module}] \leavevmode
VIR0011D

\item[{Meaning}] \leavevmode
The transaction at terminal termid called for a scenario module which was not valid.

\item[{Action}] \leavevmode
Check the module name specified in the “Input Message Exit” or “Output Message Exit” field of the transaction. Ensure that this module contains a valid scenario of the requested type. Recompile the scenario using the current version of the VIRTEL SCRNAPI macro library.

\end{description}


\subsection{Messages VIR15xxx}
\label{\detokenize{messages:messages-vir15xxx}}

\subsubsection{VIR1501E termid REQSESS FAILED, NO RELAY DEFINED}
\label{\detokenize{messages:vir1501e-termid-reqsess-failed-no-relay-defined}}\begin{description}
\item[{Module}] \leavevmode
VIR0015

\item[{Meaning}] \leavevmode
Terminal termid could not be connected to a VTAM application because the terminal has no relay defined.

\item[{Action}] \leavevmode
Specify the name of a VIRTEL relay LU in the “Relay” field of the terminal definition. The relay LU must also be defined in a VTAM APPL statement.

\end{description}


\subsubsection{VIR1502E termid PASSTICKET ERROR FOR applname / userid SAF RC: ‘safrc’ RACF RC: ‘racrc’ RACF REASON: ‘racreas’}
\label{\detokenize{messages:vir1502e-termid-passticket-error-for-applname-userid-saf-rc-safrc-racf-rc-racrc-racf-reason-racreas}}\begin{description}
\item[{Module}] \leavevmode
VIR0015

\item[{Meaning}] \leavevmode
VIRTEL does not have sufficient access rights to create or validate a passticket allowing user userid at terminal termid to access application applname. This message is usually preceded by message ICH408I which shows the name of the resource to which VIRTEL must be granted access.

\item[{Action}] \leavevmode
Examine the SAF and RACF return codes and the RACF reason code to determine the cause. Check that VIRTEL has access to resource IRR.RTICKETSERV in the FACILITY class, and also to resource IRRPTAUTH.applname.userid in the PTKTDATA class. The generic resource IRRPTAUTH.** may be used to permit VIRTEL to generate passtickets for all applications.
For an explanation of the return codes and reason codes, see z/OS Security Server RACF Callable Services chapter 2 “R\_ticketserv”. Some common codes are:

\end{description}


\subsubsection{VIR1551I termid(luname) CONNECTED TO “applname”}
\label{\detokenize{messages:vir1551i-termid-luname-connected-to-applname}}\begin{description}
\item[{Module}] \leavevmode
VIR0015

\item[{Meaning}] \leavevmode
Terminal termid has been connected to application applname via the VIRTEL relay LU luname. This message can be suppressed by the SILENCE parameter in the VIRTCT.

\item[{Action}] \leavevmode
None.

\end{description}


\subsubsection{VIR1552I termid DISCONNECTED AFTER nn MINUTES}
\label{\detokenize{messages:vir1552i-termid-disconnected-after-nn-minutes}}\begin{description}
\item[{Module}] \leavevmode
VIR0015

\item[{Meaning}] \leavevmode
Terminal termid has been disconnected from an application after nn minutes of connection time.

\item[{Action}] \leavevmode
None.

\end{description}


\subsection{Messages VIR17xxx}
\label{\detokenize{messages:messages-vir17xxx}}

\subsubsection{VIR1705W UNABLE TO ACTIVATE relayname(termid)}
\label{\detokenize{messages:vir1705w-unable-to-activate-relayname-termid}}\begin{description}
\item[{Module}] \leavevmode
VIR0017

\item[{Meaning}] \leavevmode
The relay relayname linked to terminal termid has not been activated.

\item[{Action}] \leavevmode
Check the activation of the VTAM book containing the relay relayname, check also that the relay is not already activated for another terminal and that the relay is correctly defined in VIRTEL.

\end{description}


\subsubsection{VIR1724E FILE NOT FOUND filename}
\label{\detokenize{messages:vir1724e-file-not-found-filename}}\begin{description}
\item[{Module}] \leavevmode
VIR0017, VIR0B17

\item[{Meaning}] \leavevmode
The file filename cannot be accessed by VIRTEL.

\item[{Action}] \leavevmode
Check the state of the file and its definition in VIRTEL.

\end{description}


\subsubsection{VIR1725E linename: ERROR ACCESSING ‘key’ ON filename}
\label{\detokenize{messages:vir1725e-linename-error-accessing-key-on-filename}}\begin{description}
\item[{Module}] \leavevmode
VIR0017

\item[{Meaning}] \leavevmode
A VSAM error occurred when VIRTEL attempted to access file filename.

\item[{Action}] \leavevmode
Check the state of the file and its definition in VIRTEL.

\end{description}


\subsubsection{VIR1726E termid ERROR ADDING TO FILE ‘filename’}
\label{\detokenize{messages:vir1726e-termid-error-adding-to-file-filename}}\begin{description}
\item[{Module}] \leavevmode
VIR0017

\item[{Meaning}] \leavevmode
A VSAM error occurred when VIRTEL attempted to write to file filename.

\item[{Action}] \leavevmode
Check the SYSLOG for preceding message IEC070I.

\end{description}


\subsubsection{VIR1729E LINE linename NOT FOUND}
\label{\detokenize{messages:vir1729e-line-linename-not-found}}\begin{description}
\item[{Module}] \leavevmode
VIR0B17

\item[{Meaning}] \leavevmode
An instruction in a scenario (SEND\$ TO-LINE, SEND\$ VARIABLE-TO-LINE), or a tag in an HTML page, for example: \{\{\{IP- ADDR “linename”\}\}\}, refers to a non-existent line.

\item[{Action}] \leavevmode
Specify a valid line name.

\end{description}


\subsubsection{VIR1756S NO MORE OSCORE AVAILABLE}
\label{\detokenize{messages:vir1756s-no-more-oscore-available}}\begin{description}
\item[{Module}] \leavevmode
VIR0017

\item[{Meaning}] \leavevmode
VIRTEL cannot load a program in memory due to insufficient space.

\item[{Action}] \leavevmode
Check the OCORE parameter of the VIRTCT.

\end{description}


\subsection{Messages VIR19xxx}
\label{\detokenize{messages:messages-vir19xxx}}

\subsubsection{VIR1952I luname DISCONNECTED AFTER nn MINUTES}
\label{\detokenize{messages:vir1952i-luname-disconnected-after-nn-minutes}}\begin{description}
\item[{Module}] \leavevmode
VIR0019

\item[{Meaning}] \leavevmode
The terminal luname has been disconnected after nn minutes of connection time.

\item[{Action}] \leavevmode
None.

\end{description}


\subsection{Messages VIR21xxx}
\label{\detokenize{messages:messages-vir21xxx}}

\subsubsection{VIR2121E epname HAS NO TRANSACTIONS}
\label{\detokenize{messages:vir2121e-epname-has-no-transactions}}\begin{description}
\item[{Module}] \leavevmode
VIR0021B

\item[{Meaning}] \leavevmode
Entry point epname has no transactions defined.

\item[{Action}] \leavevmode
Define at least one transaction under entry point epname.

\end{description}


\subsubsection{VIR2151E epname HAS NO TRANSACTION NAMED ‘tranid’}
\label{\detokenize{messages:vir2151e-epname-has-no-transaction-named-tranid}}\begin{description}
\item[{Module}] \leavevmode
VIR0021E

\item[{Meaning}] \leavevmode
An incoming X25 call specifies the name tranid in the first 8 bytes of its preconnection message, but entry point epname has no transaction of that name. The call is cleared.

\item[{Action}] \leavevmode
Define a transaction with external name tranid under entry point epname. See the description of VIR0021E in the VIRTEL Connectivity Reference manual.

\end{description}


\subsubsection{VIR2161E epname HAS NO TRANSACTION FOR LOGON DATA logonmsg}
\label{\detokenize{messages:vir2161e-epname-has-no-transaction-for-logon-data-logonmsg}}\begin{description}
\item[{Module}] \leavevmode
VIR0021F

\item[{Meaning}] \leavevmode
Entry point epname has no transaction whose “Logon message” field matches the preconnection message (logonmsg) of an incoming X25 call. The call is cleared.

\item[{Action}] \leavevmode
Under entry point epname, define a transaction whose “Logon message” field matches the start of logonmsg. Check that the contents of this field are within apostrophes. See the description of VIR0021F in the VIRTEL Connectivity Reference manual.

\end{description}


\subsubsection{VIR2162E epname TRANSACTION tranid HAS INCORRECT DATA IN LOGON MESSAGE FIELD}
\label{\detokenize{messages:vir2162e-epname-transaction-tranid-has-incorrect-data-in-logon-message-field}}\begin{description}
\item[{Module}] \leavevmode
VIR0021F

\item[{Meaning}] \leavevmode
The contents of the “Logon message” field in the definition of transaction tranid is not in the format required by program VIR0021F, which is specified as the menu program for entry point epname.

\item[{Action}] \leavevmode
VIRTEL ignores this transaction definition. Check that the “Logon message” field contains a character string or hexadecimal string enclosed in apostrophes. See the description of VIR0021F in the VIRTEL Connectivity Reference manual.

\end{description}


\subsubsection{VIR2171E epname HAS NO TRANSACTION NAMED USSMSG01}
\label{\detokenize{messages:vir2171e-epname-has-no-transaction-named-ussmsg01}}\begin{description}
\item[{Module}] \leavevmode
VIR0021G

\item[{Meaning}] \leavevmode
Entry point epname has no transaction whose “Logon message” field matches the preconnection message of an incoming X25 call, and VIRTEL cannot send an error message to the terminal because there is no transaction with external name USSMSG01. The call is cleared.

\item[{Action}] \leavevmode
Define a transaction with external name USSMSG01 under entry point epname. See the description of VIR0021G in the VIRTEL Connectivity Reference manual.

\end{description}


\subsubsection{VIR2172E epname TRANSACTION tranid HAS INCORRECT DATA IN LOGON MESSAGE FIELD}
\label{\detokenize{messages:vir2172e-epname-transaction-tranid-has-incorrect-data-in-logon-message-field}}\begin{description}
\item[{Module}] \leavevmode
VIR0021G

\item[{Meaning}] \leavevmode
The contents of the “Logon message” field in the definition of transaction tranid is not in the format required by program VIR0021G, which is specified as the menu program for entry point epname.

\item[{Action}] \leavevmode
VIRTEL ignores this transaction definition. Check that the “Logon message” field contains a character string or hexadecimal string enclosed in apostrophes. See the description of VIR0021G in the VIRTEL Connectivity Reference manual.

\end{description}


\subsubsection{VIR21J1E epname HAS NO AVAILABLE TRANSACTION-ERROR=errcode}
\label{\detokenize{messages:vir21j1e-epname-has-no-available-transaction-error-errcode}}\begin{description}
\item[{Module}] \leavevmode
VIR0021J

\item[{Meaning}] \leavevmode
VIRTEL did not find any available transaction in entry point epname. The error code errcode indicates the reason:
\begin{itemize}
\item {} 
1 : the entry point has no transactions

\item {} 
2 : none of the VTAM applications referenced by the transactions of the entry point are active.

\end{itemize}

\item[{Action}] \leavevmode
For code 1, define at least one transaction under entry point epname. For code 2, start at least one of the VTAM applications referenced by the transactions under entry point epname.

\end{description}


\subsection{Messages VIR27xxx}
\label{\detokenize{messages:messages-vir27xxx}}

\subsubsection{VIR2701W USER userid SENT ‘c’ TO LINE ‘n-xxxxxx’ FROM TERMINAL ‘termid’}
\label{\detokenize{messages:vir2701w-user-userid-sent-c-to-line-n-xxxxxx-from-terminal-termid}}\begin{description}
\item[{Module}] \leavevmode
VIR0027

\item[{Meaning}] \leavevmode
User userid at terminal termid sent command c to the line whose internal name is n-xxxxxx from the “State of lines” screen.

\item[{Action}] \leavevmode
None.

\end{description}


\subsection{Messages VIR31xxx}
\label{\detokenize{messages:messages-vir31xxx}}

\subsubsection{VIR3101W WARNING: LECAM SERVER servname MODIFIED}
\label{\detokenize{messages:vir3101w-warning-lecam-server-servname-modified}}\begin{description}
\item[{Module}] \leavevmode
VIR0031

\item[{Meaning}] \leavevmode
This message indicate that the definition of external server servname has been modified. The server specifies LECAM emulation.

\item[{Action}] \leavevmode
Following this type of modification, any previous “service proposition” from a LECAM PC is no longer available for use by new clients. A new “service proposition” must be generated by restarting the LECAM service application on the PC.

\end{description}


\subsection{Messages VIR35xxx}
\label{\detokenize{messages:messages-vir35xxx}}

\subsubsection{VIR3551I termid CONNECTING AS PERSONAL COMPUTER “xxxxxxxx”}
\label{\detokenize{messages:vir3551i-termid-connecting-as-personal-computer-xxxxxxxx}}\begin{description}
\item[{Module}] \leavevmode
VIR0035

\item[{Meaning}] \leavevmode
Terminal termid running VIRTEL/PC has connected to VIRTEL. The PC identification is xxxxxxxx.

\item[{Action}] \leavevmode
None.

\end{description}


\subsubsection{VIR3552I termid DISCONNECTED AFTER nn MINUTES}
\label{\detokenize{messages:vir3552i-termid-disconnected-after-nn-minutes}}\begin{description}
\item[{Module}] \leavevmode
VIR0035

\item[{Meaning}] \leavevmode
Terminal termid has disconnected after nn minutes of connection.

\item[{Action}] \leavevmode
None.

\end{description}


\subsubsection{VIR3553I termid IDENTIFICATION ERROR FOR PERSONAL COMPUTER “xxxxxxxx”}
\label{\detokenize{messages:vir3553i-termid-identification-error-for-personal-computer-xxxxxxxx}}\begin{description}
\item[{Module}] \leavevmode
VIR0035

\item[{Meaning}] \leavevmode
The terminal connected to VIRTELPC will be disconnected at the data processing center. The entry point used requires identification of the PC. This PC has not been defined at the host site.

\item[{Action}] \leavevmode
Check the definition of the PC copy at the host site. If it exists purge the associated sign-on.

\end{description}


\subsubsection{VIR3554I Input call}
\label{\detokenize{messages:vir3554i-input-call}}\begin{description}
\item[{Module}] \leavevmode
VIR0035

\item[{Meaning}] \leavevmode
A request for asynchronous connection is being processed.

\item[{Action}] \leavevmode
None.

\end{description}


\subsection{Messages VIR39xxx}
\label{\detokenize{messages:messages-vir39xxx}}

\subsubsection{VIR3952I luname DISCONNECTED AFTER nn MINUTES}
\label{\detokenize{messages:vir3952i-luname-disconnected-after-nn-minutes}}\begin{description}
\item[{Module}] \leavevmode
VIR0039

\item[{Meaning}] \leavevmode
The terminal luname has disconnected after having been connected for nn minutes.

\item[{Action}] \leavevmode
None.

\end{description}


\subsection{Messages VIR60xxx}
\label{\detokenize{messages:messages-vir60xxx}}

\subsubsection{VIR6017I FORCIBLY DETACHING VIRTEL}
\label{\detokenize{messages:vir6017i-forcibly-detaching-virtel}}\begin{description}
\item[{Module}] \leavevmode
VIR6000

\item[{Meaning}] \leavevmode
After the VIRSV subtask terminates, VIR6000 waits up to 5 seconds, and then detaches the VIRTEL subtask if it has not already terminated by itself.

\item[{Action}] \leavevmode
None.

\end{description}


\subsection{Messages VIR62xxx}
\label{\detokenize{messages:messages-vir62xxx}}

\subsubsection{VIR6202W LU 6.2 SESSION STARTED WITH applname (luname)}
\label{\detokenize{messages:vir6202w-lu-6-2-session-started-with-applname-luname}}\begin{description}
\item[{Module}] \leavevmode
VIR0062

\item[{Meaning}] \leavevmode
An LU 6.2 session has been opened between VIRTEL LU luname and partner application applname.

\item[{Action}] \leavevmode
None.

\end{description}


\subsubsection{VIR6203W LU 6.2 SESSION STARTED WITH applname}
\label{\detokenize{messages:vir6203w-lu-6-2-session-started-with-applname}}\begin{description}
\item[{Module}] \leavevmode
VIR0062

\item[{Meaning}] \leavevmode
LU 6.2 CNOS negotiation with partner LU applname was successful.

\item[{Action}] \leavevmode
None.

\end{description}


\subsubsection{VIR6204W LU 6.2 SESSION REQUESTED BY UNDEFINED pseudolu}
\label{\detokenize{messages:vir6204w-lu-6-2-session-requested-by-undefined-pseudolu}}\begin{description}
\item[{Module}] \leavevmode
VIR0062

\item[{Meaning}] \leavevmode
A request for an LU 6.2 session has been issued intended for a pseudo terminal pseudolu that is not under the control of VIRTEL.

\item[{Action}] \leavevmode
Check the parameters associated with the request.

\end{description}


\subsubsection{VIR6205W UNABLE TO ACTIVATE luname (n-xxxxxx) ERROR: xx}
\label{\detokenize{messages:vir6205w-unable-to-activate-luname-n-xxxxxx-error-xx}}\begin{description}
\item[{Module}] \leavevmode
VIR0062

\item[{Meaning}] \leavevmode
LU luname associated with the line whose internal name is n-xxxxxx could not be activated. xx is the ACB error code (see the IBM VTAM Programming manual)

\item[{Action}] \leavevmode
Check that the VTAM node containing LU luname has been activated, check that the LU is not already in use on another line, and that the line is correctly defined in VIRTEL.

\end{description}


\subsubsection{VIR6206W LU 6.2 SESSION RESET FOR LU applname (luname)}
\label{\detokenize{messages:vir6206w-lu-6-2-session-reset-for-lu-applname-luname}}\begin{description}
\item[{Module}] \leavevmode
VIR0062

\item[{Meaning}] \leavevmode
An LU 6.2 session has been reinitialised between VIRTEL LU luname and partner application applname.

\item[{Action}] \leavevmode
None.

\end{description}


\subsubsection{VIR6208W LU 6.2 CONVERSATION (luname \textendash{} applname) STARTING ON n-xxxxxx}
\label{\detokenize{messages:vir6208w-lu-6-2-conversation-luname-applname-starting-on-n-xxxxxx}}\begin{description}
\item[{Module}] \leavevmode
VIR0062

\item[{Meaning}] \leavevmode
A conversation has begun between VIRTEL LU luname and a partner application applname on the line with internal name n-xxxxxx.

\item[{Action}] \leavevmode
None.

\end{description}


\subsubsection{VIR6210W LU 6.2 CONVERSATION REQUESTED BY UNDEFINED pseudolu}
\label{\detokenize{messages:vir6210w-lu-6-2-conversation-requested-by-undefined-pseudolu}}\begin{description}
\item[{Module}] \leavevmode
VIR0062

\item[{Meaning}] \leavevmode
An LU 6.2 request was sent intended for a pseudo terminal pseudolu not under the control of VIRTEL.

\item[{Action}] \leavevmode
Check the parameters associated with the request.

\end{description}


\subsubsection{VIR6212W CONVERSATION cccccccc STARTED  ON n-xxxxxx}
\label{\detokenize{messages:vir6212w-conversation-cccccccc-started-on-n-xxxxxx}}\begin{description}
\item[{Module}] \leavevmode
VIR0062

\item[{Meaning}] \leavevmode
The conversation number cccccccc has begun on the line with internal name n-xxxxxx.

\item[{Action}] \leavevmode
None.

\end{description}


\subsubsection{VIR6216W luname(applname) INACTIVATED}
\label{\detokenize{messages:vir6216w-luname-applname-inactivated}}\begin{description}
\item[{Module}] \leavevmode
VIR0062

\item[{Meaning}] \leavevmode
The LU6.2 session between LU luname and the application applname has been deactivated.

\item[{Action}] \leavevmode
None.

\end{description}


\subsubsection{VIR6218S NO MORE OSCORE AVAILABLE}
\label{\detokenize{messages:vir6218s-no-more-oscore-available}}\begin{description}
\item[{Module}] \leavevmode
VIR0062

\item[{Meaning}] \leavevmode
VIRTEL cannot load a program in memory due to insufficient space.

\item[{Action}] \leavevmode
Check the OSCORE parameter of the VIRTCT.

\end{description}


\subsubsection{VIR6220W LU 6.2 SESSION LOST FOR LINE n-xxxxxx (luname \textendash{} applname)}
\label{\detokenize{messages:vir6220w-lu-6-2-session-lost-for-line-n-xxxxxx-luname-applname}}\begin{description}
\item[{Module}] \leavevmode
VIR0062

\item[{Meaning}] \leavevmode
The LU 6.2 session between VIRTEL LU luname and partner application applname has been lost. n-xxxxxx is the internal name of the APPC line.

\item[{Action}] \leavevmode
None.

\end{description}


\subsubsection{VIR6222E ERROR ON applname - R15-R0 : yyyy zzzz RCPRI: pri RCSEC: sec REQ: req QUAL: qual STATE: stat SENSE: sens}
\label{\detokenize{messages:vir6222e-error-on-applname-r15-r0-yyyy-zzzz-rcpri-pri-rcsec-sec-req-req-qual-qual-state-stat-sense-sens}}\begin{description}
\item[{Module}] \leavevmode
VIR0062

\item[{Meaning}] \leavevmode
An unexpected error has occurred during dialogue in LU 6.2 mode.

\item[{Action}] \leavevmode
Contact technical support.

\end{description}


\subsubsection{VIR6223E ERROR ON applname - R15-R0 : yyyy zzzz RCPRI: pri RCSEC: sec REQ: req QUAL: qual STATE: stat SENSE: sens}
\label{\detokenize{messages:vir6223e-error-on-applname-r15-r0-yyyy-zzzz-rcpri-pri-rcsec-sec-req-req-qual-qual-state-stat-sense-sens}}\begin{description}
\item[{Module}] \leavevmode
VIR0062

\item[{Meaning}] \leavevmode
An unexpected error has occurred during dialogue in LU 6.2 mode between application applname and VIRTEL. This message has no significance excepting if R15 = 00 (yyyy) and R0 = 0B (zzzz).

\item[{Action}] \leavevmode
See return code RCPRI and RCSEC in the IBM VTAM Programming for LU 6.2 manual.

\end{description}


\subsubsection{VIR6224W ENDING CONVERSATION cccccccc WITH applname}
\label{\detokenize{messages:vir6224w-ending-conversation-cccccccc-with-applname}}\begin{description}
\item[{Module}] \leavevmode
VIR0062

\item[{Meaning}] \leavevmode
APPC conversation cccccccc with partner application applname has ended.

\item[{Action}] \leavevmode
None.

\end{description}


\subsection{Messages VIR65xxx}
\label{\detokenize{messages:messages-vir65xxx}}

\subsubsection{VIR6599E linename CANNOT START -DEFINITION IS INCOMPLETE}
\label{\detokenize{messages:vir6599e-linename-cannot-start-definition-is-incomplete}}\begin{description}
\item[{Module}] \leavevmode
VIR0615

\item[{Meaning}] \leavevmode
The line whose external name is linename contains a definition error (for example, protocol program not defined) and cannot be started.

\item[{Action}] \leavevmode
Correct the line definition.

\end{description}


\subsection{Messages VIR75xxx}
\label{\detokenize{messages:messages-vir75xxx}}

\subsubsection{VIR7551I applname CONNECTING pseudolu}
\label{\detokenize{messages:vir7551i-applname-connecting-pseudolu}}\begin{description}
\item[{Module}] \leavevmode
VIR0715

\item[{Meaning}] \leavevmode
The application applname is connected to the pseudo terminal pseudolu in LU 6.2 single session mode (APPC1).

\item[{Action}] \leavevmode
None.

\end{description}


\subsubsection{VIR7552I applname DISCONNECTING pseudolu}
\label{\detokenize{messages:vir7552i-applname-disconnecting-pseudolu}}\begin{description}
\item[{Module}] \leavevmode
VIR0715

\item[{Meaning}] \leavevmode
The LU 6.2 application applname has disconnected from the pseudo terminal pseudolu.

\item[{Action}] \leavevmode
None.

\end{description}


\subsubsection{VIR7599I applname CONNECTED}
\label{\detokenize{messages:vir7599i-applname-connected}}\begin{description}
\item[{Module}] \leavevmode
VIR0715

\item[{Meaning}] \leavevmode
The LU 6.2 application applname is connected to VIRTEL in single session mode (APPC1).

\item[{Action}] \leavevmode
None.

\end{description}


\subsection{Messages VIR85xxx}
\label{\detokenize{messages:messages-vir85xxx}}

\subsubsection{VIR8551I applname CONNECTING pseudolu}
\label{\detokenize{messages:vir8551i-applname-connecting-pseudolu}}\begin{description}
\item[{Module}] \leavevmode
VIR0815

\item[{Meaning}] \leavevmode
The application applname is connected to the pseudo terminal pseudolu in shared LU 6.2 mode (APPC2).

\item[{Action}] \leavevmode
None.

\end{description}


\subsubsection{VIR8552I applname DISCONNECTING pseudolu}
\label{\detokenize{messages:vir8552i-applname-disconnecting-pseudolu}}\begin{description}
\item[{Module}] \leavevmode
VIR0815

\item[{Meaning}] \leavevmode
The LU 6.2 application applname has disconnected from the pseudo terminal pseudolu.

\item[{Action}] \leavevmode
None.

\end{description}


\subsubsection{VIR8553I applname ACCEPTED pseudolu}
\label{\detokenize{messages:vir8553i-applname-accepted-pseudolu}}\begin{description}
\item[{Module}] \leavevmode
VIR0815

\item[{Meaning}] \leavevmode
The application applname has accepted the pseudo terminal pseudolu.

\item[{Action}] \leavevmode
None.

\end{description}


\subsubsection{VIR8599I applname (n-xxxxxx) CONNECTED}
\label{\detokenize{messages:vir8599i-applname-n-xxxxxx-connected}}\begin{description}
\item[{Module}] \leavevmode
VIR0815

\item[{Meaning}] \leavevmode
The LU 6.2 application applname is connected to VIRTEL in shared session mode (APPC2) on the line whose internal name is n-xxxxxx.

\item[{Action}] \leavevmode
None.

\end{description}


\subsection{Messages VIR91xxx}
\label{\detokenize{messages:messages-vir91xxx}}

\subsubsection{VIR9151I applid CONNECTING LUTYPE n PRINTER prname(luname) TO termid}
\label{\detokenize{messages:vir9151i-applid-connecting-lutype-n-printer-prname-luname-to-termid}}\begin{description}
\item[{Module}] \leavevmode
VIR0915I

\item[{Meaning}] \leavevmode
While connecting a non-predefined VTAM LU, application applid connected printer LU type n with terminal name prname and LU name luname to terminal termid

\item[{Action}] \leavevmode
Aucune.

\end{description}


\subsection{Messages VIR99xxx}
\label{\detokenize{messages:messages-vir99xxx}}

\subsubsection{VIR9901E termid: ERROR nnnnnnnn CALLING TRANSACTION tranname FROM ENTRY POINT epname}
\label{\detokenize{messages:vir9901e-termid-error-nnnnnnnn-calling-transaction-tranname-from-entry-point-epname}}\begin{description}
\item[{Module}] \leavevmode
VIR0099

\item[{Meaning}] \leavevmode
Transaction tranname associated with the entry point epname and started by terminal luname cannot be found and shows error nnnnnnnn.

\item[{Action}] \leavevmode
Check the definition of the entry point and transactions to ensure that the terminal has been defined.

\end{description}


\subsubsection{VIR9905W LU 6.2 relayname (applname) ACTIVATED}
\label{\detokenize{messages:vir9905w-lu-6-2-relayname-applname-activated}}\begin{description}
\item[{Module}] \leavevmode
VIR0000, VIR0062

\item[{Meaning}] \leavevmode
The LU6.2 session between the relay relayname and the application applname has been activated.

\item[{Action}] \leavevmode
None.

\end{description}


\subsubsection{VIR9999S GETMAIN FATAL ERROR}
\label{\detokenize{messages:vir9999s-getmain-fatal-error}}\begin{description}
\item[{Module}] \leavevmode
VIR0000

\item[{Meaning}] \leavevmode
An unexpected error has occurred while requesting memory. The system will stop.

\item[{Action}] \leavevmode
Contact technical support.

\end{description}


\subsubsection{VIR9999I INVITATION A LIBERER RECUE DE termname}
\label{\detokenize{messages:vir9999i-invitation-a-liberer-recue-de-termname}}\begin{description}
\item[{Module}] \leavevmode
VIR0005

\item[{Meaning}] \leavevmode
Terminal termname has sent a REQUEST FREE command and will be force disconnected.

\item[{Action}] \leavevmode
None.

\end{description}


\subsection{Messages VIRB1xxx}
\label{\detokenize{messages:messages-virb1xxx}}

\subsubsection{VIRB171I LINE linename (n-xxxxxx) IS WAITING FOR m-yyyyyy ACTIVATION}
\label{\detokenize{messages:virb171i-line-linename-n-xxxxxx-is-waiting-for-m-yyyyyy-activation}}\begin{description}
\item[{Module}] \leavevmode
VIR0B17

\item[{Meaning}] \leavevmode
Initialisation of the line with external name linename and internal name n-xxxxxx is waiting for line m-yyyyyy to start, because WAIT-LINE(m-yyyyyy) was specified in the “Startup prerequisite” field in the definition of line n-xxxxxx.

\item[{Action}] \leavevmode
Activate line m-yyyyyy, or wait until VIRTEL activates it.

\end{description}


\subsubsection{VIRB172I LINE linename (n-xxxxxx) FAILED TO PROCESS: condition}
\label{\detokenize{messages:virb172i-line-linename-n-xxxxxx-failed-to-process-condition}}\begin{description}
\item[{Module}] \leavevmode
VIR0B17

\item[{Meaning}] \leavevmode
There is an invalid value condition in the “Startup prerequisite” field in the definition of the line whose external name is linename and whose internal name is n-xxxxxx, or condition refers to an unknown line name.

\item[{Action}] \leavevmode
Correct the definition of line n-xxxxxx.

\end{description}


\subsubsection{VIRB173I LINE linename (n-xxxxxx) STARTUP IS NOT AUTOMATIC}
\label{\detokenize{messages:virb173i-line-linename-n-xxxxxx-startup-is-not-automatic}}\begin{description}
\item[{Module}] \leavevmode
VIR0B17

\item[{Meaning}] \leavevmode
Initialisation of the line whose external name is linename and whose internal name is n-xxxxxx is waiting for a START command, because WAIT-COMMAND is specified in the “Startup prerequisite” field in the definition of the line.

\item[{Action}] \leavevmode
To activate the line, enter the VIRTEL command LINE=n-xxxxxx,START at the system console.

\end{description}


\subsubsection{VIRB174I LINE linename (n-xxxxxx) STARTUP IS PASSIVE}
\label{\detokenize{messages:virb174i-line-linename-n-xxxxxx-startup-is-passive}}\begin{description}
\item[{Module}] \leavevmode
VIR0B17

\item[{Meaning}] \leavevmode
Initialisation of the line whose external name is linename and whose internal name is n-xxxxxx is waiting for a VTAM BIND from its partner LU, because WAIT-PARTNER is specified in the “Startup prerequisite” field in the definition of the line.

\item[{Action}] \leavevmode
To activate the line, start the partner application.

\end{description}


\subsubsection{VIRB176I LINE linename (n-xxxxxx), RESTARTED BY m-yyyyyy}
\label{\detokenize{messages:virb176i-line-linename-n-xxxxxx-restarted-by-m-yyyyyy}}\begin{description}
\item[{Module}] \leavevmode
VIR0B17

\item[{Meaning}] \leavevmode
Following activation of line m-yyyyyy, VIRTEL has begun initialisation of the line whose external name is linename and whose internal name is n-xxxxxx.

\item[{Action}] \leavevmode
None.

\end{description}


\subsubsection{VIRB177I LINE linename (n-xxxxxx) IS WAITING FOR m-yyyyyy ACTIVATION}
\label{\detokenize{messages:virb177i-line-linename-n-xxxxxx-is-waiting-for-m-yyyyyy-activation}}\begin{description}
\item[{Module}] \leavevmode
VIR0B17

\item[{Meaning}] \leavevmode
Initialisation of the line whose external name is linename and whose internal name is n-xxxxxx is waiting for line m- yyyyyy to start, because MIMIC-LINE(m-yyyyyy) is specified in the “Startup prerequisite” field in the definition of line n-xxxxxx.

\item[{Action}] \leavevmode
Activate line m-yyyyyy, or wait until VIRTEL activates it.

\end{description}


\subsubsection{VIRB178I LINE linename (n-xxxxxx), STOPPED BY m-yyyyyy}
\label{\detokenize{messages:virb178i-line-linename-n-xxxxxx-stopped-by-m-yyyyyy}}\begin{description}
\item[{Module}] \leavevmode
VIR0B17

\item[{Meaning}] \leavevmode
Following the deactivation of line m-yyyyyy, VIRTEL has stopped the line whose external name is linename and whose internal name is n-xxxxxx, because MIMIC-LINE(m-yyyyyy) is specified in the “Startup prerequisite” field in the definition of line n-xxxxxx.

\item[{Action}] \leavevmode
None.

\end{description}


\subsubsection{VIRB179E ERROR ON: luname ALLOCATING 64 BITS STORAGE - RETCODE: xxxxxxxx - REASON CODE: xxxxxxxx}
\label{\detokenize{messages:virb179e-error-on-luname-allocating-64-bits-storage-retcode-xxxxxxxx-reason-code-xxxxxxxx}}\begin{description}
\item[{Module}] \leavevmode
VIR0B17

\item[{Meaning}] \leavevmode
An memory allocation attempt failed while trying to store a VIRTEL variable above ‘above the BAR’ (ie in 64 bits storage). The hexadecimal return code and reason code could be researched in the DC2 code in the IBM “z/OS   MVS System Codes” manual or possibly in the IBM IARV64 service documentation in the “MVS Assembler Services Reference” manual. Virtel currently limits the maximum size of a Virtel variable to 2 gigabytes. The maximum amount of 64-bit private virtual memory available to the Virtel address space can be controles by the MEMLIMIT JCL parameter.

\item[{Action}] \leavevmode
Use an appropriate value in the MEMLIMIT JCL parameter.

\end{description}


\subsection{Messages VIRB4xxx}
\label{\detokenize{messages:messages-virb4xxx}}

\subsubsection{VIRB411E termid UPLOAD FAILED FOR USER userid}
\label{\detokenize{messages:virb411e-termid-upload-failed-for-user-userid}}\begin{description}
\item[{Module}] \leavevmode
VIR0041B

\item[{Meaning}] \leavevmode
Upload of an HTML page has failed. The page was received from user userid on terminal termid.

\item[{Action}] \leavevmode
See message sent by VIRTEL to the user.

\end{description}


\subsection{Messages VIRB9xxx}
\label{\detokenize{messages:messages-virb9xxx}}

\subsubsection{VIRB903W LINE x-nnnnnn TAKES INPUT FROM: indd AND OUTPUTS TO: outdd}
\label{\detokenize{messages:virb903w-line-x-nnnnnn-takes-input-from-indd-and-outputs-to-outdd}}\begin{description}
\item[{Module}] \leavevmode
VIR0B09

\item[{Meaning}] \leavevmode
The batch line with internal name x-nnnnnn has started. The input data for this line will be taken from ddname indd and the output data will be written to ddname outdd.

\item[{Action}] \leavevmode
None.

\end{description}


\subsubsection{VIRB904W linename PROCESSING .GET\textbar{}.POST\textbar{}.RAW\textbar{}.END\textbar{}.EOJ COMMAND}
\label{\detokenize{messages:virb904w-linename-processing-get-post-raw-end-eoj-command}}\begin{description}
\item[{Module}] \leavevmode
VIR0B09

\item[{Meaning}] \leavevmode
The batch line with external name linename is processing the indicated command. This message is not displayed if SILENCE=YES is specified in the VIRTCT.

\item[{Action}] \leavevmode
None.

\end{description}


\subsubsection{VIRB906W LINE x-nnnnnn CLOSING indd AND outdd}
\label{\detokenize{messages:virb906w-line-x-nnnnnn-closing-indd-and-outdd}}\begin{description}
\item[{Module}] \leavevmode
VIR0B09

\item[{Meaning}] \leavevmode
The batch line with internal name x-nnnnnn is terminating. The input ddname indd and the output ddname outdd are being closed.

\item[{Action}] \leavevmode
None.

\end{description}


\subsubsection{VIRB907W linename ENDING WITH RETURN CODE xxxxxxxx (dddddddd)}
\label{\detokenize{messages:virb907w-linename-ending-with-return-code-xxxxxxxx-dddddddd}}\begin{description}
\item[{Module}] \leavevmode
VIR0B09

\item[{Meaning}] \leavevmode
The batch line with external name linename is terminating. xxxxxxxx is the hexadecimal return code of the batch job step, and dddddddd is the return code in decimal.

\item[{Action}] \leavevmode
None.

\end{description}


\subsubsection{VIRB908E x-nnnnnn INVALID COMMAND ‘xxxx’ REPLACED BY ‘.EOJ’}
\label{\detokenize{messages:virb908e-x-nnnnnn-invalid-command-xxxx-replaced-by-eoj}}\begin{description}
\item[{Module}] \leavevmode
VIR0B09

\item[{Meaning}] \leavevmode
The batch line with internal name x-nnnnnn has encountered an unknown command xxxx in its input data file. The command is processed as if it were an end of job command, and the batch job terminates with return code 16.

\item[{Action}] \leavevmode
Check the input data for the batch line.

\end{description}


\subsubsection{VIRB909E x-nnnnnn OPEN ERROR, EOJ REQUESTED}
\label{\detokenize{messages:virb909e-x-nnnnnn-open-error-eoj-requested}}\begin{description}
\item[{Module}] \leavevmode
VIR0B09

\item[{Meaning}] \leavevmode
The batch line with internal name x-nnnnnn was unable to open its input or output file. The batch job terminates with return code 16.

\item[{Action}] \leavevmode
Check the console log for error messages. Check that the job contains DD statements for the input and output files associated with this batch line.

\end{description}


\subsubsection{VIRB912W termid OBJECT nnnnnnnn STARTED}
\label{\detokenize{messages:virb912w-termid-object-nnnnnnnn-started}}\begin{description}
\item[{Module}] \leavevmode
VIR0B09

\item[{Meaning}] \leavevmode
Terminal termid associated with a batch line has begun processing.

\item[{Action}] \leavevmode
None.

\end{description}


\subsubsection{VIRB922W termid ENDING OBJECT nnnnnnnn}
\label{\detokenize{messages:virb922w-termid-ending-object-nnnnnnnn}}\begin{description}
\item[{Module}] \leavevmode
VIR0B09

\item[{Meaning}] \leavevmode
Terminal termid associated with a batch line has finished processing.

\item[{Action}] \leavevmode
None.

\end{description}


\subsubsection{VIRB923E termid REQ\textbar{}PARAM xxxx COMPLETION CODE code REASON CODE xxxxxxxx (dddddddd) LINE linename}
\label{\detokenize{messages:virb923e-termid-req-param-xxxx-completion-code-code-reason-code-xxxxxxxx-dddddddd-line-linename}}\begin{description}
\item[{Module}] \leavevmode
VIR0B09

\item[{Meaning}] \leavevmode
A batch request has terminated abnormally. linename is the external name of the batch line, termid is the associated terminal name, type is the type of request, code is the abend code, xxxxxxxx is the reason code in hexadecimal and dddddddd is the decimal equivalent.

\item[{Action}] \leavevmode
Inspect the console log for other error messages which may explain the cause. Contact technical support.

\end{description}


\subsubsection{VIRB924E termid OBJECT\textbar{}PARAM nnnnnnnn REQ type COMPLETION CODE code REASON CODE xxxxxxxx (dddddddd) LINE linename}
\label{\detokenize{messages:virb924e-termid-object-param-nnnnnnnn-req-type-completion-code-code-reason-code-xxxxxxxx-dddddddd-line-linename}}\begin{description}
\item[{Module}] \leavevmode
VIR0B09

\item[{Meaning}] \leavevmode
An error or unexpected end of file was detected while reading the input file for a batch line. linename is the external name of the batch line, termid is the associated terminal name, type is the type of request, code is the abend code, xxxxxxxx is the reason code in hexadecimal and dddddddd is the decimal equivalent.

\item[{Action}] \leavevmode
Check that the input file is correct.

\end{description}


\subsection{Messages VIRC1xxx}
\label{\detokenize{messages:messages-virc1xxx}}

\subsubsection{VIRC121E PAGE NOT FOUND FOR termid ENTRY POINT ‘epname’ DIRECTORY ‘tranid’(dirname dirkey) PAGE ‘filename’ URL’url’}
\label{\detokenize{messages:virc121e-page-not-found-for-termid-entry-point-epname-directory-tranid-dirname-dirkey-page-filename-url-url}}\begin{description}
\item[{Module}] \leavevmode
VIR0C12

\item[{Meaning}] \leavevmode
HTML page filename requested by terminal termid does not exist in the directory dirname specified by the transaction whose external name is tranid linked to entry point epname. The directory key is dirkey. VIRTEL sent a 404 NOT FOUND reply to the terminal.

\item[{Action}] \leavevmode
Check that the browser requested the correct page. Upload the page into the directory dirname.

\end{description}


\subsubsection{VIRC122E ERROR termid IS SENDING A SCENARIO PF KEY BUT SCENARIO IS MISSING IN TRANSACTION ‘tranid’ ENTRY POINT ‘epname’}
\label{\detokenize{messages:virc122e-error-termid-is-sending-a-scenario-pf-key-but-scenario-is-missing-in-transaction-tranid-entry-point-epname}}\begin{description}
\item[{Module}] \leavevmode
VIR0C12

\item[{Meaning}] \leavevmode
Terminal termid sent an HTTP request containing a PF=SCENARIO parameter (see “Function key management” in  the VIRTEL Web Access Guide) but the transaction whose external name is tranid does not have an input scenario specified.

\item[{Action}] \leavevmode
Add the name of an input scenario to the transaction tranid defined under entry point epname.

\end{description}


\subsection{Messages VIRC4xxx}
\label{\detokenize{messages:messages-virc4xxx}}

\subsubsection{VIRC411E termid UPLOAD FAILED FOR USER userid}
\label{\detokenize{messages:virc411e-termid-upload-failed-for-user-userid}}\begin{description}
\item[{Module}] \leavevmode
VIR0041C

\item[{Meaning}] \leavevmode
Upload of an HTML page has failed. The page was received from user userid on terminal termid.

\item[{Action}] \leavevmode
See message sent by VIRTEL to the user.

\end{description}


\subsection{Messages VIRCAxxx}
\label{\detokenize{messages:messages-vircaxxx}}

\subsubsection{VIRCA01W CRYn INITIALISING CRYPTOGRAPHY WITH PARAMETERS: ’name1’,’algs’,’algp’,’engine’,’encoding’,’chaining’,’padding’}
\label{\detokenize{messages:virca01w-cryn-initialising-cryptography-with-parameters-name1-algs-algp-engine-encoding-chaining-padding}}\begin{description}
\item[{Module}] \leavevmode
VIR0CA0

\item[{Meaning}] \leavevmode
VIRTEL is initializing the cryptographic engine specified by the CRYPTn parameter of the VIRTCT. Refer to the VIRTEL Installation Guide for the meaning of the parameters.

\item[{Action}] \leavevmode
None.

\end{description}


\subsubsection{VIRCA02W CRYn termid REQUEST FOR PUBLIC KEY}
\label{\detokenize{messages:virca02w-cryn-termid-request-for-public-key}}\begin{description}
\item[{Module}] \leavevmode
VIR0CA0

\item[{Meaning}] \leavevmode
An HTML page delivered to terminal termid has requested VIRTEL to generate a public key using the method specified by the CRYPTn parameter of the VIRTCT.

\item[{Action}] \leavevmode
None.

\end{description}


\subsubsection{VIRCA03W CRYn termid DECRYPTING SESSION KEY}
\label{\detokenize{messages:virca03w-cryn-termid-decrypting-session-key}}\begin{description}
\item[{Module}] \leavevmode
VIR0CA0

\item[{Meaning}] \leavevmode
VIRTEL has received an encrypted session key from the terminal termid and is decrypting the key according to the method specified by the CRYPTn parameter of the VIRTCT.

\item[{Action}] \leavevmode
None.

\end{description}


\subsubsection{VIRCA04W CRYn termid ENCRYPTING A MESSAGE}
\label{\detokenize{messages:virca04w-cryn-termid-encrypting-a-message}}\begin{description}
\item[{Module}] \leavevmode
VIR0CA0

\item[{Meaning}] \leavevmode
A message to be sent to terminal termid is being encrypted according to the method specified by the CRYPTn parameter of the VIRTCT.

\item[{Action}] \leavevmode
None.

\end{description}


\subsubsection{VIRCA04W CRYn termid DECRYPTING A MESSAGE}
\label{\detokenize{messages:virca04w-cryn-termid-decrypting-a-message}}\begin{description}
\item[{Module}] \leavevmode
VIR0CA0

\item[{Meaning}] \leavevmode
A message received from terminal termid is being decrypted according to the method specified by the CRYPTn parameter of the VIRTCT.

\item[{Action}] \leavevmode
None.

\end{description}


\subsubsection{VIRCA11W CRYn INITIALISING CRYPTOGRAPHY WITH PARAMETERS: ’name1’,’algs’,’algp’,’engine’,’encoding’,’chaining’,’padding’}
\label{\detokenize{messages:virca11w-cryn-initialising-cryptography-with-parameters-name1-algs-algp-engine-encoding-chaining-padding}}\begin{description}
\item[{Module}] \leavevmode
VIR0CA1

\item[{Meaning}] \leavevmode
VIRTEL is initializing the cryptographic engine specified by the CRYPTn parameter of the VIRTCT. Refer to the VIRTEL Installation Guide for the meaning of the parameters.

\item[{Action}] \leavevmode
None.

\end{description}


\subsubsection{VIRCA12W termid CRYn REQUEST FOR PUBLIC KEY}
\label{\detokenize{messages:virca12w-termid-cryn-request-for-public-key}}\begin{description}
\item[{Module}] \leavevmode
VIR0CA1

\item[{Meaning}] \leavevmode
An HTML page delivered to terminal termid has requested VIRTEL to generate a public key using the method specified by the CRYPTn parameter of the VIRTCT.

\item[{Action}] \leavevmode
None.

\end{description}


\subsubsection{VIRCA13W termid CRYn DECRYPTING SESSION KEY}
\label{\detokenize{messages:virca13w-termid-cryn-decrypting-session-key}}\begin{description}
\item[{Module}] \leavevmode
VIR0CA1

\item[{Meaning}] \leavevmode
VIRTEL has received an encrypted session key from the terminal termid and is decrypting the key according to the method specified by the CRYPTn parameter of the VIRTCT.

\item[{Action}] \leavevmode
None.

\end{description}


\subsubsection{VIRCA17E termid CRYn \sphinxstylestrong{Error} servname retc=xxxx reas=yyyy}
\label{\detokenize{messages:virca17e-termid-cryn-error-servname-retc-xxxx-reas-yyyy}}\begin{description}
\item[{Module}] \leavevmode
VIR0CA1

\item[{Meaning}] \leavevmode
A call to an ICSF cryptographic service servname failed with return code xxxx and reason code yyyy. The return code and reason code are shown in hexadecimal.

\item[{Action}] \leavevmode
Refer to SA22-7522 z/OS Cryptographic Services ICSF Application Programmer’s Guide Appendix A for the meaning of ICSF return codes and reason codes. Return code 0000000C means that ICSF services are not available, usually because the CSF started task is not correctly initialized.

\end{description}


\subsubsection{VIRCA18E termid keylen=nnnn fieldlen=nnnn}
\label{\detokenize{messages:virca18e-termid-keylen-nnnn-fieldlen-nnnn}}\begin{description}
\item[{Module}] \leavevmode
VIR0CA1

\item[{Meaning}] \leavevmode
This message is issued in conjunction with message VIRCA17E. It indicates length of the symmetric key and the length of the field being encrypted or decrypted when the error occurred.

\item[{Action}] \leavevmode
Refer to preceding message VIRCA17E.

\end{description}


\subsection{Messages VIRCFxxx}
\label{\detokenize{messages:messages-vircfxxx}}

\subsubsection{VIRCF27E SYNTAX ERROR IN EXEC PARAMETER INVALID EXEC PARAMETER - RC = 16}
\label{\detokenize{messages:vircf27e-syntax-error-in-exec-parameter-invalid-exec-parameter-rc-16}}\begin{description}
\item[{Module}] \leavevmode
VIRCONF

\item[{Meaning}] \leavevmode
One of the value specified in the PARM parameter of the EXEC card is invalid. Valid values are: LOAD, UNLOAD, SCAN or LANG=

\item[{Action}] \leavevmode
Correct the parameter.

\end{description}


\subsubsection{VIRCF28E NOT AUTHORIZED TO UNLOAD PLAINTXT SECURITY (eg RACF) ERROR - RC = 12}
\label{\detokenize{messages:vircf28e-not-authorized-to-unload-plaintxt-security-eg-racf-error-rc-12}}\begin{description}
\item[{Module}] \leavevmode
VIRCONF

\item[{Meaning}] \leavevmode
The user code assign to the job is not authorized to access the ARBO file according to the operation specified in the PARM parameter.

\item[{Action}] \leavevmode
Contact security department to obtain sufficient permissions to perform the desired operation.

\end{description}


\subsubsection{VIRCF47E OPEN FAILED FOR SYSPUNCH DDNAME SYSPUNCH OPEN ERROR - RC = 16}
\label{\detokenize{messages:vircf47e-open-failed-for-syspunch-ddname-syspunch-open-error-rc-16}}\begin{description}
\item[{Module}] \leavevmode
VIRCONF

\item[{Meaning}] \leavevmode
Open failed on the SYSPUNCH entry.

\item[{Action}] \leavevmode
Check the SYSPUNCH specification.

\end{description}


\subsubsection{VIRCF48E OPEN FAILED FOR SYSPRINT DDNAME SYSPRINT OPEN ERROR - RC = 16}
\label{\detokenize{messages:vircf48e-open-failed-for-sysprint-ddname-sysprint-open-error-rc-16}}\begin{description}
\item[{Module}] \leavevmode
VIRCONF

\item[{Meaning}] \leavevmode
Open failed on the SYSPRINT entry.

\item[{Action}] \leavevmode
Check the SYSPRINT specification.

\end{description}


\subsubsection{VIRCF50E OPEN FAILED FOR SYSIN DDNAME SYSPIN OPEN ERROR - RC = 16}
\label{\detokenize{messages:vircf50e-open-failed-for-sysin-ddname-syspin-open-error-rc-16}}\begin{description}
\item[{Module}] \leavevmode
VIRCONF

\item[{Meaning}] \leavevmode
Open failed on the SYSIN entry.

\item[{Action}] \leavevmode
Check the SYSPIN specification.

\end{description}


\subsubsection{VIRCF52E VSAM OPEN ERROR DDNAME=VIRARBO, R15=XXXXXXXX, ACBERFLG=XXXXXXXX VSAM error code opening VIRARBO - RC = 16}
\label{\detokenize{messages:vircf52e-vsam-open-error-ddname-virarbo-r15-xxxxxxxx-acberflg-xxxxxxxx-vsam-error-code-opening-virarbo-rc-16}}\begin{description}
\item[{Module}] \leavevmode
VIRCONF

\item[{Meaning}] \leavevmode
Unexpected VSAM error occurred during access of VIRARBO file. R15 and ACBERFLG gives the error and the reason of the error.

\item[{Action}] \leavevmode
Verify the values of the return codes in the appropriate IBM documentation. VSAM error codes are documented in the chapter entitled VSAM Macro Return and Reason Codes in the IBM manual DFSMS Macro Instructions for Data Sets.

\end{description}


\subsection{Messages VIRCTxxx}
\label{\detokenize{messages:messages-virctxxx}}

\subsubsection{VIRCT01E CRYn ERROR INSTALLING ‘NO-ENCRYPTION’ SUBTASK}
\label{\detokenize{messages:virct01e-cryn-error-installing-no-encryption-subtask}}\begin{description}
\item[{Module}] \leavevmode
VIR0CT0

\item[{Meaning}] \leavevmode
An unexpected error occurred while initializing the VIRTEL subtask for the no-encryption module specified by the CRYPTn parameter of the VIRTCT.

\item[{Action}] \leavevmode
Contact technical support.

\end{description}


\subsubsection{VIRCT02W CRYn ‘NO-ENCRYPTION’ SUBTASK STARTED}
\label{\detokenize{messages:virct02w-cryn-no-encryption-subtask-started}}\begin{description}
\item[{Module}] \leavevmode
VIR0CT0

\item[{Meaning}] \leavevmode
Successful initialization of the VIRTEL subtask for the no-encryption module specified by the CRYPTn parameter of the VIRTCT.

\item[{Action}] \leavevmode
None.

\end{description}


\subsubsection{VIRCT03W CRYn ‘NO-ENCRYPTION’ SUBTASK ENDED}
\label{\detokenize{messages:virct03w-cryn-no-encryption-subtask-ended}}\begin{description}
\item[{Module}] \leavevmode
VIR0CT0

\item[{Meaning}] \leavevmode
Termination of the VIRTEL subtask for the no-encryption module specified by the CRYPTn parameter of the VIRTCT.

\item[{Action}] \leavevmode
None.

\end{description}


\subsubsection{VIRCT05W CRYn ‘NO-ENCRYPTION’ SESSION KEY READY}
\label{\detokenize{messages:virct05w-cryn-no-encryption-session-key-ready}}\begin{description}
\item[{Module}] \leavevmode
VIR0CT0

\item[{Meaning}] \leavevmode
VIRTEL has established a session key for a terminal using the no-encryption module specified by the CRYPTn parameter of the VIRTCT.

\item[{Action}] \leavevmode
None.

\end{description}


\subsubsection{VIRCT101}
\label{\detokenize{messages:virct101}}
\begin{sphinxVerbatim}[commandchars=\\\{\}]
\PYG{n}{VIRCT10I} \PYG{n}{ICSFSTAT} \PYG{p}{:} \PYG{n}{FMID}\PYG{o}{=}\PYG{n}{fmid} \PYG{n}{STATUS1}\PYG{o}{=}\PYG{n}{n} \PYG{n}{STATUS2}\PYG{o}{=}\PYG{n}{n} \PYG{n}{CPACF}\PYG{o}{=}\PYG{n}{n} \PYG{n}{AES}\PYG{o}{=}\PYG{n}{n} \PYG{n}{DSA}\PYG{o}{=}\PYG{n}{n} \PYG{n}{RSA1}\PYG{o}{=}\PYG{n}{n} \PYG{n}{RSA2}\PYG{o}{=}\PYG{n}{n} \PYG{n}{RSA3}\PYG{o}{=}\PYG{n}{n} \PYG{n}{ACC}\PYG{o}{=}\PYG{n}{n}
\PYG{n}{VIRCT10I} \PYG{n}{ICSFST2} \PYG{p}{:} \PYG{n}{VERSION}\PYG{o}{=}\PYG{n}{vers} \PYG{n}{FMID}\PYG{o}{=}\PYG{n}{fmid} \PYG{n}{STATUS1}\PYG{o}{=}\PYG{n}{n} \PYG{n}{STATUS2}\PYG{o}{=}\PYG{n}{n} \PYG{n}{STATUS3}\PYG{o}{=}\PYG{n}{n} \PYG{n}{STATUS4}\PYG{o}{=}\PYG{n}{n}
\PYG{n}{VIRCT10I} \PYG{n}{STATAES} \PYG{p}{:} \PYG{n}{NMK}\PYG{o}{\PYGZhy{}}\PYG{n}{STATUS}\PYG{o}{=}\PYG{n}{n} \PYG{n}{CMK}\PYG{o}{\PYGZhy{}}\PYG{n}{STATUS}\PYG{o}{=}\PYG{n}{n} \PYG{n}{OMK}\PYG{o}{\PYGZhy{}}\PYG{n}{STATUS}\PYG{o}{=}\PYG{n}{n} \PYG{n}{KEYLEN}\PYG{o}{=}\PYG{n}{n}
\PYG{n}{VIRCT10I} \PYG{n}{STATCCA} \PYG{p}{:} \PYG{n}{NMK}\PYG{o}{\PYGZhy{}}\PYG{n}{STATUS}\PYG{o}{=}\PYG{n}{n} \PYG{n}{CMK}\PYG{o}{\PYGZhy{}}\PYG{n}{STATUS}\PYG{o}{=}\PYG{n}{n} \PYG{n}{OMK}\PYG{o}{\PYGZhy{}}\PYG{n}{STATUS}\PYG{o}{=}\PYG{n}{n} \PYG{n}{CCA}\PYG{o}{\PYGZhy{}}\PYG{n}{APP}\PYG{o}{\PYGZhy{}}\PYG{n}{VERS}\PYG{o}{=}\PYG{n}{n} \PYG{n}{CCA}\PYG{o}{\PYGZhy{}}\PYG{n}{APP}\PYG{o}{\PYGZhy{}}\PYG{n}{BUILD}\PYG{o}{=}\PYG{n}{n} \PYG{n}{USER}\PYG{o}{\PYGZhy{}}\PYG{n}{ROLE}\PYG{o}{=}\PYG{n}{n}
\PYG{n}{VIRCT10I} \PYG{n}{STATCARD} \PYG{p}{:} \PYG{n}{ADAPTERS}\PYG{o}{=}\PYG{n}{n} \PYG{n}{DES}\PYG{o}{=}\PYG{n}{n} \PYG{n}{RSA}\PYG{o}{=}\PYG{n}{n} \PYG{n}{POST}\PYG{o}{=}\PYG{n}{n} \PYG{n}{n} \PYG{n}{CP}\PYG{o}{\PYGZhy{}}\PYG{n}{OS}\PYG{o}{=}\PYG{n}{osname} \PYG{n}{VERS}\PYG{o}{=}\PYG{n}{osver} \PYG{n}{PART}\PYG{o}{=}\PYG{n}{partno} \PYG{n}{EC}\PYG{o}{=}\PYG{n}{eclevel} \PYG{n}{BOOT}\PYG{o}{=}\PYG{n}{n} \PYG{n}{n}
\PYG{n}{VIRCT10I} \PYG{n}{STATDIAG} \PYG{p}{:} \PYG{n}{BATTERY}\PYG{o}{=}\PYG{n}{n} \PYG{n}{INTRUSION}\PYG{o}{=}\PYG{n}{n} \PYG{n}{ERROR}\PYG{o}{\PYGZhy{}}\PYG{n}{LOG}\PYG{o}{=}\PYG{n}{n} \PYG{n}{MESH}\PYG{o}{=}\PYG{n}{n} \PYG{n}{LOW}\PYG{o}{\PYGZhy{}}\PYG{n}{VOLT}\PYG{o}{=}\PYG{n}{n} \PYG{n}{HIGH}\PYG{o}{\PYGZhy{}}\PYG{n}{VOLT}\PYG{o}{=}\PYG{n}{n} \PYG{n}{TEMPERATURE}\PYG{o}{=}\PYG{n}{n} \PYG{n}{RADIATION}\PYG{o}{=}\PYG{n}{n}
\PYG{n}{VIRCT10I} \PYG{n}{STATEXPT} \PYG{p}{:} \PYG{n}{CCA}\PYG{o}{=}\PYG{n}{n} \PYG{n}{CMDF}\PYG{o}{=}\PYG{n}{n} \PYG{l+m+mi}{56}\PYG{o}{\PYGZhy{}}\PYG{n}{bit}\PYG{o}{\PYGZhy{}}\PYG{n}{DES}\PYG{o}{=}\PYG{n}{n} \PYG{n}{Triple}\PYG{o}{\PYGZhy{}}\PYG{n}{DES}\PYG{o}{=}\PYG{n}{n} \PYG{n}{SET}\PYG{o}{=}\PYG{n}{n} \PYG{n}{MAX}\PYG{o}{\PYGZhy{}} \PYG{n}{SYMMETRIC}\PYG{o}{\PYGZhy{}}\PYG{n}{MODULUS}\PYG{o}{=}\PYG{n}{nnnn}
\end{sphinxVerbatim}
\begin{description}
\item[{Module}] \leavevmode
VIR0CT1

\item[{Meaning}] \leavevmode
This message provides diagnostic information about the facilities provided by the underlying ICSF software and cryptographic hardware. The meaning of each of the fields is shown below. For further information, refer to CSFIQF in SA22-7522 z/OS Cryptographic Services ICSF Application Programmer’s Guide.

\item[{ICSFSTAT}] \leavevmode
FMID : Indicates the version of ICSF installed
STATUS1 : 0=ICSF started, 1=ICSF initialized, 2=SYM-MK valid, 3=PKA enabled STATUS2 : 0=31-bit, 1=64-bit, 2=PKCS\#11 available
CPACF : CPACF services available: 0=None, 1=SHA-1, 2=DES/TDES, 3=SHA-224/256, 4=SHA-224/256/DES/TDES, 5=SHA-384/512, 6=SHA-384/512/DES/TDES
AES : AES availability: 0=No, 1=Software, 2=AES-128, 3=AES-192/256 DSA : DSA availability: 0=No, 1=DSA-1024, 2=DSA-2048
RSA1 : RSA signature key length: 0=No, 1=1024, 2=2048, 3=4096 RSA2 : RSA key management length: 0=No, 1=1024, 2=2048, 3=4096 RSA3 : RSA key generate service: 0=No, 1=2048, 2=4096
ACC : RSA key accelerators available: 0=No, 1=Yes

\item[{ICSFST2}] \leavevmode
VERSION : ICSFST2 version number
FMID : Indicates the version of ICSF installed STATUS1 : PKA services: 0=Disabled, 1=Enabled STATUS2 : PKCS\#11 available: 0=No, 1=Yes
STATUS3 : ICSF status: 0=Started, 1=Initialized, 2=AES master key valid STATUS4 : Secure AES key available: 0=No, 1=Yes
STATAES
NMK-STATUS AES new master key : 1=Clear, 2=Partial, 3=Complete CMK-STATUS AES current master key : 1=Clear, 2=Key
OMK-STATUS AES old master key : 1=Clear, 2=Key KEYLEN : Maximum AES key length

\item[{STATCCA}] \leavevmode
NMK-STATUS DES new master key : 1=Clear, 2=Partial, 3=Complete CMK-STATUS DES current master key : 1=Clear, 2=Key
OMK-STATUS DES old master key : 1=Clear, 2=Key
CCA-APP-VERS : Version of CCA application in co-processor CCA-APP-BUILD : Build date of CCA application in co-processor USER-ROLE : Host application user’s authority role identifier

\item[{STATCARD}] \leavevmode
ADAPTERS : Number of installed cryptographic adapters DES : DES hardware level
RSA : RSA hardware level
POST : Power-On Self Test version numbers
CP-OS : Crypto co-processor operating system name VERS : Operating system version number
PART : Crypto co-processor part number
EC : Crypto co-processor engineering change level BOOT : Crypto co-processor miniboot version numbers

\item[{STATDIAG}] \leavevmode
BATTERY : Battery state: 1=Good, 2=Replace INTRUSION : Intrusion latch state: 1=Cleared, 2=Set ERROR-LOG : Error log status: 1=Empty, 2=Data, 3=Full MESH : Mesh intrusion: 1=No, 2=Intrusion
LOW-VOLT : Low-voltage detected: 1=No, 2=Yes HIGH-VOLT : High-voltage detected: 1=No, 2=Yes
TEMPERATURE : Temperature range exceeded: 1=No, 2=Yes RADIATION : Radiation detected: 1=No, 2=Yes
STATEXPT
CCA : Base CCA services available: 0=No, 1=Yes CMDF : CDMF encryption available: 0=No, 1=Yes
56-bit-DES : 56-bit DES encryption available: 0=No, 1=Yes Triple-DES : Triple DES encryption available: 0=No, 1=Yes SET : Secure Electronic Transaction available: 0=No, 1=Yes
MAX-SYMMERTIC-MODULUS : Maximum modulus size enabled for asymmetric encryption of symmetric keys

\item[{Action}] \leavevmode
None.

\end{description}


\subsubsection{VIRCT11E CRYn ERROR INSTALLING ‘ICSF’ SUBTASK}
\label{\detokenize{messages:virct11e-cryn-error-installing-icsf-subtask}}
Module
VIR0CT1
Meaning
An unexpected error occurred while initializing the VIRTEL subtask for the ICSF interface specified by the CRYPTn parameter of the VIRTCT.
Action
Contact technical support.


\subsubsection{VIRCT12W CRYn ‘ICSF’ SUBTASK STARTED}
\label{\detokenize{messages:virct12w-cryn-icsf-subtask-started}}\begin{description}
\item[{Module}] \leavevmode
VIR0CT1

\item[{Meaning}] \leavevmode
Successful initialization of the VIRTEL subtask for the ICSF interface specified by the CRYPTn parameter of the VIRTCT.

\item[{Action}] \leavevmode
None.

\end{description}


\subsubsection{VIRCT13W CRYn ‘ICSF’ SUBTASK ENDED}
\label{\detokenize{messages:virct13w-cryn-icsf-subtask-ended}}\begin{description}
\item[{Module}] \leavevmode
VIR0CT1

\item[{Meaning}] \leavevmode
Termination of the VIRTEL subtask for the ICSF interface specified by the CRYPTn parameter of the VIRTCT.

\item[{Action}] \leavevmode
None.

\end{description}


\subsubsection{VIRCT15W CRYn ‘ICSF’ SESSION KEY READY}
\label{\detokenize{messages:virct15w-cryn-icsf-session-key-ready}}\begin{description}
\item[{Module}] \leavevmode
VIR0CT1

\item[{Meaning}] \leavevmode
VIRTEL has established a session key for a terminal using the ICSF interface specified by the CRYPTn parameter of the VIRTCT.

\item[{Action}] \leavevmode
None.

\end{description}


\subsubsection{VIRCT16E CRYn \sphinxstylestrong{Error} servname retc=xxxx reas=yyyy}
\label{\detokenize{messages:virct16e-cryn-error-servname-retc-xxxx-reas-yyyy}}\begin{description}
\item[{Module}] \leavevmode
VIR0CT1

\item[{Meaning}] \leavevmode
A call to an ICSF cryptographic service servname failed with return code xxxx and reason code yyyy. The return code and reason code are shown in hexadecimal.

\item[{Action}] \leavevmode
Refer to SA22-7522 z/OS Cryptographic Services ICSF Application Programmer’s Guide Appendix A for the meaning of ICSF return codes and reason codes. Return code 0000000C means that ICSF services are not available, usually because the CSF started task is not correctly initialized.

\end{description}


\subsubsection{VIRCT17E CRYn ERROR: text}
\label{\detokenize{messages:virct17e-cryn-error-text}}\begin{description}
\item[{Module}] \leavevmode
VIR0CT1

\item[{Meaning}] \leavevmode
This message is issued in conjunction with message VIRCT16E. It provides an explanation for the reason code indicated in the previous message.

\item[{Action}] \leavevmode
Refer to preceding message VIRCT16E.

\end{description}


\subsection{Messages VIRHTxxx}
\label{\detokenize{messages:messages-virhtxxx}}

\subsubsection{VIRHT01I HTTP INITIALISATION FOR linename (n-xxxxxx), VERSION x.xx}
\label{\detokenize{messages:virht01i-http-initialisation-for-linename-n-xxxxxx-version-x-xx}}\begin{description}
\item[{Module}] \leavevmode
VIRHTTP

\item[{Meaning}] \leavevmode
Initialisation of HTTP line with external name linename and internal name n-xxxxxx.

\item[{Action}] \leavevmode
None.

\end{description}


\subsubsection{VIRHT02I LINE linename (n-xxxxxx) HAS URL http://ipaddr:port}
\label{\detokenize{messages:virht02i-line-linename-n-xxxxxx-has-url-http-ipaddr-port}}\begin{description}
\item[{Module}] \leavevmode
VIRHTTP

\item[{Meaning}] \leavevmode
The HTTP line with external name linename and internal name n-xxxxxx is ready to receive requests at the indicated URL.

\item[{Action}] \leavevmode
None.

\end{description}


\subsubsection{VIRHT04I HTTP LINE linename WAS STOPPED}
\label{\detokenize{messages:virht04i-http-line-linename-was-stopped}}\begin{description}
\item[{Module}] \leavevmode
VIRHTTP

\item[{Meaning}] \leavevmode
HTTP line with external name linename was stopped, either by console command, or by setting the “Possible calls” field to 0 in the line definition.

\item[{Action}] \leavevmode
None.

\end{description}


\subsubsection{VIRHT27E linename IS REJECTED AT ENTRY POINT epname}
\label{\detokenize{messages:virht27e-linename-is-rejected-at-entry-point-epname}}\begin{description}
\item[{Module}] \leavevmode
VIRHTTP

\item[{Meaning}] \leavevmode
VIRTEL received an incoming call on the HTTP line whose external name is linename, but the entry point epname specified in the line definition does not exist.

\item[{Action}] \leavevmode
Check the HTTP line definition.

\end{description}


\subsubsection{VIRHT28I linename CALLER ipaddr:port GETS ENTRY POINT ‘epname’ FROM RULE ‘rulename’}
\label{\detokenize{messages:virht28i-linename-caller-ipaddr-port-gets-entry-point-epname-from-rule-rulename}}\begin{description}
\item[{Module}] \leavevmode
VIRHTTP

\item[{Meaning}] \leavevmode
VIRTEL chose entry point epname for an incoming call on the HTTP line whose external name is linename, because the call matched the rule named rulename.

\item[{Action}] \leavevmode
None.

\end{description}


\subsubsection{VIRHT29E n-xxxxxx DEFAULT ENTRY POINT MISSING OR INVALID}
\label{\detokenize{messages:virht29e-n-xxxxxx-default-entry-point-missing-or-invalid}}\begin{description}
\item[{Module}] \leavevmode
VIRHTTP

\item[{Meaning}] \leavevmode
An incoming call on the HTTP line whose internal name is n-xxxxxx did not match any of the rules of the line, and the default entry point specified in the line definition is either blank or does not exist.

\item[{Action}] \leavevmode
Either define a default rule for the HTTP line, or specify a valid default entry point in the line definition.

\end{description}


\subsubsection{VIRHT51I linename CONNECTING termname TO ipaddr:port}
\label{\detokenize{messages:virht51i-linename-connecting-termname-to-ipaddr-port}}\begin{description}
\item[{Module}] \leavevmode
VIRHTTP

\item[{Meaning}] \leavevmode
An incoming or outgoing HTTP call is being processed. For an incoming HTTP call, linename is the external name of the HTTP line, termname is the name of the VIRTEL terminal allocated to process the call, and ipaddr:port is the IP address and port number of the client browser. For an outgoing HTTP call, linename is the external name of the HTTP outbound line, termname is the name of the VIRTEL terminal making the call, and ipaddr:port is the IP address and port number of the remote HTTP server. This message can be suppressed by the SILENCE parameter in the VIRTCT.

\item[{Action}] \leavevmode
None.

\end{description}


\subsubsection{VIRHT52I linename DECONNECTING termname}
\label{\detokenize{messages:virht52i-linename-deconnecting-termname}}\begin{description}
\item[{Module}] \leavevmode
VIRHTTP

\item[{Meaning}] \leavevmode
Terminal termname has been disconnected from the HTTP line whose external name is linename.

\item[{Action}] \leavevmode
None.

\end{description}


\subsubsection{VIRHT53E linename NO MORE PSEUDO FOUND WITH PREFIX xxxx FOR TRANSACTION tranname(tranid)}
\label{\detokenize{messages:virht53e-linename-no-more-pseudo-found-with-prefix-xxxx-for-transaction-tranname-tranid}}\begin{description}
\item[{Module}] \leavevmode
VIRHTTP

\item[{Meaning}] \leavevmode
VIRTEL tried to open a terminal whose name matches the prefix xxxx specified in the VIRTEL transaction with external name tranname and internal name tranid, but this prefix conflicts with the prefix specified on the definition of the HTTP line whose external name is linename.

\item[{Action}] \leavevmode
Check the “Prefix” field of the HTTP line and the “Pseudo-terminals” field of the transaction.

\end{description}


\subsubsection{VIRHT54E INVALID REQUEST ON linename ENTRY POINT ‘epname’ DIRECTORY ‘dirname’ PAGE ‘filename’ URL ‘url’ TRANSACTION ‘tranid’ CALLER ipaddr:port reason}
\label{\detokenize{messages:virht54e-invalid-request-on-linename-entry-point-epname-directory-dirname-page-filename-url-url-transaction-tranid-caller-ipaddr-port-reason}}\begin{description}
\item[{Module}] \leavevmode
VIRHTTP

\item[{Meaning}] \leavevmode
VIRTEL received a request on the HTTP line whose external name is linename originating from the browser at IP address ipaddr:port. The browser requested HTML page filename using URL url. Either the directory path dirname does not exist, or there is no transaction named tranid under the entry point epname.

\item[{Action}] \leavevmode
Check that the browser requested the correct URL. Check the definition of the entry point epname. If reason is “missing directory : dirname”, ensure that there is a type 4 transaction with external name dirname. If reason is “rejected transaction : tranid”, ensure that there is a type 1 transaction with external name tranid. See “VIRTEL URL formats” in the VIRTEL Web Access Guide.

\end{description}


\subsubsection{VIRHT55E INVALID RULE rulename ENTRY POINT ‘epname’}
\label{\detokenize{messages:virht55e-invalid-rule-rulename-entry-point-epname}}\begin{description}
\item[{Module}] \leavevmode
VIRHTTP

\item[{Meaning}] \leavevmode
Rule rulename used by an HTTP line specifies a non-existent entry point epname.

\item[{Action}] \leavevmode
Check the entry point name specified in the rule definition.

\end{description}


\subsubsection{VIRHT56I n-xxxxxx CALLER real-ip WAS FORWARDED BY proxy-ip}
\label{\detokenize{messages:virht56i-n-xxxxxx-caller-real-ip-was-forwarded-by-proxy-ip}}\begin{description}
\item[{Module}] \leavevmode
VIRHTTP

\item[{Meaning}] \leavevmode
An HTTP request was received from proxy-ip on the HTTP line whose internal name is n-xxxxxx. The source IP address proxy-ip appears in the “Calling DTE” field of a rule attached to the HTTP line, and the request matches the rule in all other respects. VIRTEL considers the IP address real-ip, which is specified in the “forwarding header” inserted by the proxy, to be the real originating IP address. Refer to “Calling DTE” in the “Rules” chapter of the VIRTEL Configuration Reference manual for further information.

\item[{Action}] \leavevmode
VIRTEL henceforth treats the call as if it had been received from real-ip instead of from proxy-ip. The real-ip address will be displayed in console messages and recorded in the VIRLOG.

\end{description}


\subsubsection{VIRHT57E linename RECEIVED AN ERRONEOUS REQUEST FROM ipaddr:port : LINE IS NOT SET UP FOR HTTPS}
\label{\detokenize{messages:virht57e-linename-received-an-erroneous-request-from-ipaddr-port-line-is-not-set-up-for-https}}\begin{description}
\item[{Module}] \leavevmode
VIRHTTP

\item[{Meaning}] \leavevmode
VIRTEL received a request on the HTTP line whose external name is linename originating from the browser at IP address ipaddr:port. The browser has requested that the session should be encrypted using the SSL or TLS protocols, but encryption is not activated for this line.

\item[{Action}] \leavevmode
Either resubmit the request without encryption (use http instead of https), or activate encryption on this line (see “Data encryption” in the “VIRTEL Web Access Security” chapter of the VIRTEL Web Access Guide.

\end{description}


\subsubsection{VIRHT58E linename IS NOT AN OUTPUT LINE: CANNOT CONNECT termid}
\label{\detokenize{messages:virht58e-linename-is-not-an-output-line-cannot-connect-termid}}\begin{description}
\item[{Module}] \leavevmode
VIRHTTP

\item[{Meaning}] \leavevmode
A scenario invoked by terminal termid has attempted to make an outbound HTTP request via the HTTP line whose external name is linename, but the line definition does not permit outbound calls.

\item[{Action}] \leavevmode
Modify the definition of line linename setting the “Possible calls” field to 2 or 3.

\end{description}


\subsubsection{VIRHT59E LINE linename IS NOT OPEN: CANNOT CONNECT termid}
\label{\detokenize{messages:virht59e-line-linename-is-not-open-cannot-connect-termid}}\begin{description}
\item[{Module}] \leavevmode
VIRHTTP

\item[{Meaning}] \leavevmode
A scenario invoked by terminal termid has attempted to make an outbound HTTP request via the HTTP line whose external name is linename, but the line is stopped.

\item[{Action}] \leavevmode
Start the line by entering the VIRTEL command: line=linename,S

\end{description}


\subsubsection{VIRHT60E linename HAS NO PSEUDO TERMINALS: CANNOT CONTINUE}
\label{\detokenize{messages:virht60e-linename-has-no-pseudo-terminals-cannot-continue}}\begin{description}
\item[{Module}] \leavevmode
VIRHTTP

\item[{Meaning}] \leavevmode
No terminals are defined for the HTTP line whose external name is linename.

\item[{Action}] \leavevmode
If this is a batch line then VIRTEL terminates; otherwise VIRTEL continues but the line cannot be used.

\end{description}


\subsubsection{VIRHT61E linename A PARAMETER IS NEEDED FOR PREFIX \$LINE\$ FOR TRANSACTION ttttttt (xxxxx)}
\label{\detokenize{messages:virht61e-linename-a-parameter-is-needed-for-prefix-line-for-transaction-ttttttt-xxxxx}}\begin{description}
\item[{Module}] \leavevmode
VIRHTTP

\item[{Meaning}] \leavevmode
The parameter \&luname= and his value were not specified in the URL received by VIRTEL on line linename for transaction ttttttt.

\item[{Action}] \leavevmode
Check the URL value if possible or run a /F stsname,snapmsg,all command at the console.

\end{description}


\subsubsection{VIRHT62E}
\label{\detokenize{messages:virht62e}}\begin{description}
\item[{Module}] \leavevmode
VIRHTTP

\item[{Meaning}] \leavevmode
The parameter \&luname= and his value were not specified in the URL received by VIRTEL on line linename for transaction ttttttt.

\item[{Action}] \leavevmode
Check the URL value if possible or run a /F stsname,snapmsg,all command at the console.

\end{description}


\subsubsection{VIRHT63E linename ERROR STARTING PSEUDO FOR PREFIX \$LINE\$ FOR TRANSACTION ttttttt (xxxxx)}
\label{\detokenize{messages:virht63e-linename-error-starting-pseudo-for-prefix-line-for-transaction-ttttttt-xxxxx}}\begin{description}
\item[{Module}] \leavevmode
VIRHTTP

\item[{Meaning}] \leavevmode
The VTAM terminal (relay) to use for a request is already active for another session. The request was received on line linename for transaction ttttttt.

\item[{Action}] \leavevmode
The request cannot be satisfied. The connection is closed. Run a /F stcname,snapmsg,all command at the console.

\end{description}


\subsection{Messages VIRI9xxx}
\label{\detokenize{messages:messages-viri9xxx}}

\subsubsection{VIRI902W TERM=termid REQ=req RTNCD=rtncd FDBK2=fdbk2 SENSE=sense}
\label{\detokenize{messages:viri902w-term-termid-req-req-rtncd-rtncd-fdbk2-fdbk2-sense-sense}}\begin{description}
\item[{Module}] \leavevmode
VIR0I09

\item[{Meaning}] \leavevmode
VTAM error on terminal termid. The RPL request code is req, and the VTAM return codes are rtncd, fdbk2, sense.

\item[{Action}] \leavevmode
Refer to the IBM VTAM Messages and Codes or Communications Server IP and SNA Codes manual for an explanation of the VTAM codes.

\end{description}


\subsubsection{VIRI903I luname(termid) RECEIVED BIND FROM applname}
\label{\detokenize{messages:viri903i-luname-termid-received-bind-from-applname}}\begin{description}
\item[{Module}] \leavevmode
VIR0I09

\item[{Meaning}] \leavevmode
The relay LU luname associated with VIRTEL terminal termid has received an SNA BIND command from partner application ACB applname. This message appears only if a terminal or line trace is active for terminal termid.

\item[{Action}] \leavevmode
None.

\end{description}


\subsubsection{VIRI904I CLEANUP: linename / applid}
\label{\detokenize{messages:viri904i-cleanup-linename-applid}}\begin{description}
\item[{Module}] \leavevmode
VIR0I09

\item[{Meaning}] \leavevmode
Loss of session with partner LU applid on the line whose external name is linename.

\item[{Action}] \leavevmode
None.

\end{description}


\subsubsection{VIRI905I luname RELEASED}
\label{\detokenize{messages:viri905i-luname-released}}\begin{description}
\item[{Module}] \leavevmode
VIR0I09

\item[{Meaning}] \leavevmode
At the request of another application (VTAM RELREQ), VIRTEL released the LU luname.

\item[{Action}] \leavevmode
None.

\end{description}


\subsubsection{VIRI905W LU luname (n-xxxxxx) ACTIVATED}
\label{\detokenize{messages:viri905w-lu-luname-n-xxxxxx-activated}}\begin{description}
\item[{Module}] \leavevmode
VIR0I09

\item[{Meaning}] \leavevmode
Initialisation of LU luname associated with the line whose internal name is n-xxxxxx.

\item[{Action}] \leavevmode
None.

\end{description}


\subsubsection{VIRI906W LU luname (n-xxxxxx) INACTIVATED}
\label{\detokenize{messages:viri906w-lu-luname-n-xxxxxx-inactivated}}\begin{description}
\item[{Module}] \leavevmode
VIR0I09

\item[{Meaning}] \leavevmode
Inactivation of LU luname associated with the line whose internal name is n-xxxxxx.

\item[{Action}] \leavevmode
None.

\end{description}


\subsubsection{VIRI907I luname(termid) REJECTED BIND FROM applname}
\label{\detokenize{messages:viri907i-luname-termid-rejected-bind-from-applname}}\begin{description}
\item[{Module}] \leavevmode
VIR0I09

\item[{Meaning}] \leavevmode
The relay LU luname associated with VIRTEL terminal termid received an unsolicited SNA BIND command from application applname. VIRTEL rejected the BIND because the LU was not expecting to receive a session request from applname. This can occur if a PCNE application is attempting to place an outgoing call to an LU reserved for incoming calls.

\item[{Action}] \leavevmode
Check that the application has specified the correct LU name. Check that “possible calls” is not set to “2” in the VIRTEL terminal definition.

\end{description}


\subsubsection{VIRI908S INVALID RPL}
\label{\detokenize{messages:viri908s-invalid-rpl}}\begin{description}
\item[{Module}] \leavevmode
VIR0I09

\item[{Meaning}] \leavevmode
The VTAM SYNAD / LERAD exit was entered with an invalid recovery action return code (RTNCD) in register 0. This message is followed by a dump with abend code U0009.

\item[{Action}] \leavevmode
Contact technical support.

\end{description}


\subsubsection{VIRI909E UNABLE TO ACTIVATE luname (n-xxxxxx) ERROR: xx000000}
\label{\detokenize{messages:viri909e-unable-to-activate-luname-n-xxxxxx-error-xx000000}}\begin{description}
\item[{Module}] \leavevmode
VIR0I09

\item[{Meaning}] \leavevmode
LU luname associated with the line whose internal name is n-xxxxxx could not be activated. xx represents the ACB error code:
- 58 : ACB already active by another application
- 5A : ACB inactive under VTAM

\end{description}

For the meaning of other codes, see the IBM VTAM Programming manual.
\begin{description}
\item[{Action}] \leavevmode
Check that the VTAM node containing LU luname has been activated, check if the LU is not already active on another line, and that the line is correctly defined in VIRTEL.

\end{description}


\subsubsection{VIRI910I luname(termid) RECEIVED UNBIND FROM applname}
\label{\detokenize{messages:viri910i-luname-termid-received-unbind-from-applname}}\begin{description}
\item[{Module}] \leavevmode
VIR0I09

\item[{Meaning}] \leavevmode
The relay LU luname associated with VIRTEL terminal termid has received an SNA UNBIND command from application ACB applname. This message appears only if a terminal or line trace is active for terminal termid.

\item[{Action}] \leavevmode
None.

\end{description}


\subsubsection{VIRI911I luname(termid) REQUESTING SESSION WITH applname}
\label{\detokenize{messages:viri911i-luname-termid-requesting-session-with-applname}}\begin{description}
\item[{Module}] \leavevmode
VIR0I09

\item[{Meaning}] \leavevmode
The VIRTEL terminal termid has requested VTAM to establish a session between its associated relay LU luname and the application ACB applname. This message appears only if a terminal or line trace is active for terminal termid.

\item[{Action}] \leavevmode
None.

\end{description}


\subsubsection{VIRI913W VTAM SHORT ON STORAGE}
\label{\detokenize{messages:viri913w-vtam-short-on-storage}}\begin{description}
\item[{Module}] \leavevmode
VIR0I09

\item[{Meaning}] \leavevmode
The VTAM action code supplied to the SYNAD exit indicates that there is insufficient memory to process the operation.

\item[{Action}] \leavevmode
VIRTEL retries the operation after 1 second.

\end{description}


\subsubsection{VIRI914E termid ERROR R15=xx R0=yy CONNECTING luname TO applid}
\label{\detokenize{messages:viri914e-termid-error-r15-xx-r0-yy-connecting-luname-to-applid}}\begin{description}
\item[{Module}] \leavevmode
VIR0I09

\item[{Meaning}] \leavevmode
VIRTEL could not establish a session between VIRTEL LU luname and the ACB applid of a partner application. xx et yy are the hexadecimal error codes from REQSESS.

\item[{Action}] \leavevmode
If R15=00000020, activate the LU luname in VTAM, then enter a VIRTEL START command for the line. For any other value of R15, check that the application applid is active and ready to receive connections, and that the LU luname is in CONCT status.

\end{description}


\subsubsection{VIRI915E termid ERROR ACTIVATING luname}
\label{\detokenize{messages:viri915e-termid-error-activating-luname}}\begin{description}
\item[{Module}] \leavevmode
VIR0I09

\item[{Meaning}] \leavevmode
VIRTEL could not establish a session between VIRTEL LU luname and the ACB of its partner application.

\item[{Action}] \leavevmode
Refer to the SYSLOG. This message is normally accompanied by VTAM message IST663I which gives more information about the cause of the error.

\end{description}


\subsubsection{VIRI918W RELAY luname(termid) INACTIVATED}
\label{\detokenize{messages:viri918w-relay-luname-termid-inactivated}}\begin{description}
\item[{Module}] \leavevmode
VIR0I09

\item[{Meaning}] \leavevmode
VIRTEL closed the relay luname for terminal termid following inactivation of the LU by VTAM.

\item[{Action}] \leavevmode
Further connections using this relay will only be possible after the LU has been reactivated under VTAM, and a VIRTEL START command has been issued for the associated line, either from the “Status of lines” screen (F9 from the configuration menu, or F10 from the sub-applications system services menu), or by the console command LINE=linename,START.

\end{description}


\subsubsection{VIRI956S NO MORE OSCORE AVAILABLE}
\label{\detokenize{messages:viri956s-no-more-oscore-available}}\begin{description}
\item[{Module}] \leavevmode
VIR0I09

\item[{Meaning}] \leavevmode
VIRTEL has insufficient memory to start a new VTAM task.

\item[{Action}] \leavevmode
Check the OSCORE parameter in the VIRTCT.

\end{description}


\subsection{Messages VIRICxxx}
\label{\detokenize{messages:messages-viricxxx}}

\subsubsection{VIRIC01I INITIALISATION FOR linename (n-xxxxxx), VERSION x.xx}
\label{\detokenize{messages:viric01i-initialisation-for-linename-n-xxxxxx-version-x-xx}}\begin{description}
\item[{Module}] \leavevmode
VIR0ICON

\item[{Meaning}] \leavevmode
The IMS Connect line with external name linename and internal name n-xxxxxx has been initialised.

\item[{Action}] \leavevmode
None.

\end{description}


\subsubsection{VIRIC04I LINE linename WAS STOPPED}
\label{\detokenize{messages:viric04i-line-linename-was-stopped}}\begin{description}
\item[{Module}] \leavevmode
VIR0ICON

\item[{Meaning}] \leavevmode
The IMS Connect line with external name linename was stopped, either by console command, or by setting the “Possible calls” parameter to 0 in the line definition.

\item[{Action}] \leavevmode
None.

\end{description}


\subsubsection{VIRIC27E linename IS REJECTED AT ENTRY POINT epname}
\label{\detokenize{messages:viric27e-linename-is-rejected-at-entry-point-epname}}\begin{description}
\item[{Module}] \leavevmode
VIR0ICON

\item[{Meaning}] \leavevmode
VIRTEL was unable to start the IMS Connect line whose external name is linename because the entry point epname specified on the line definition does not exist.

\item[{Action}] \leavevmode
Correct the entry point name specified in the line definition.

\end{description}


\subsubsection{VIRIC28I linename CALLER ipaddr:port GETS ENTRY POINT ‘epname’ FROM RULE ‘rulename’}
\label{\detokenize{messages:viric28i-linename-caller-ipaddr-port-gets-entry-point-epname-from-rule-rulename}}\begin{description}
\item[{Module}] \leavevmode
VIR0ICON

\item[{Meaning}] \leavevmode
VIRTEL selected entry point epname for an incoming call on the IMS Connect line whose external name is linename, because the call matched the rule rulename.

\item[{Action}] \leavevmode
None.

\end{description}


\subsubsection{VIRIC29I linename HAS NO MORE PSEUDO TERMINALS}
\label{\detokenize{messages:viric29i-linename-has-no-more-pseudo-terminals}}\begin{description}
\item[{Module}] \leavevmode
VIR0ICON

\item[{Meaning}] \leavevmode
There are no more terminals associated with the IMS Connect line whose external name is linename.

\item[{Action}] \leavevmode
Check the definition of the line. Add more terminals if necessary.

\end{description}


\subsubsection{VIRIC51I linename CONNECTION STARTING ON termname TO ENTRY POINT ‘epname’ TRANSACTION ‘tranname’}
\label{\detokenize{messages:viric51i-linename-connection-starting-on-termname-to-entry-point-epname-transaction-tranname}}\begin{description}
\item[{Module}] \leavevmode
VIR0ICON

\item[{Meaning}] \leavevmode
VIRTEL has received a request from IMS Connect on the line with external name linename using terminal termname. The request is being processed by transaction tranname in entry point epname.

\item[{Action}] \leavevmode
None.

\end{description}


\subsubsection{VIRIC52I linename DECONNECTING termname}
\label{\detokenize{messages:viric52i-linename-deconnecting-termname}}\begin{description}
\item[{Module}] \leavevmode
VIR0ICON

\item[{Meaning}] \leavevmode
Terminal termname has been disconnected from the IMS Connect line whose external name is linename.

\item[{Action}] \leavevmode
None.

\end{description}


\subsubsection{VIRIC53E linename ENTRY POINT ‘epname’ WAS NOT FOUND}
\label{\detokenize{messages:viric53e-linename-entry-point-epname-was-not-found}}\begin{description}
\item[{Module}] \leavevmode
VIR0ICON

\item[{Meaning}] \leavevmode
VIRTEL received a request on the IMS Connect line whose external name is linename. The request could not be processed because the entry point epname does not exist.

\item[{Action}] \leavevmode
Correct the entry point name specified in the line definition.

\end{description}


\subsubsection{VIRIC54E linename IMS STATUS ERROR ‘retcode’ REASON ‘reascode’ text}
\label{\detokenize{messages:viric54e-linename-ims-status-error-retcode-reason-reascode-text}}\begin{description}
\item[{Module}] \leavevmode
VIR0ICON

\item[{Meaning}] \leavevmode
VIRTEL attempted to start the IMS Connect line whose external name is linename. The RESUME TPIPE request was rejected by IMS Connect with return code retcode and reason code reascode. A brief explanation of the code is displayed as text if the return code and reason code are known to VIRTEL.

\item[{Action}] \leavevmode
Check the OTMAPRM parameter in the VIRTCT. Check the IMS Connect log for error messages. If the error is NACK\_RSM\_TPIPE\_SEC\_FAILED check for ICH408I and IRR012I messages in the system log.

\end{description}


\subsubsection{VIRIC55E INVALID RULE rulename ENTRY POINT ‘epname’}
\label{\detokenize{messages:viric55e-invalid-rule-rulename-entry-point-epname}}\begin{description}
\item[{Module}] \leavevmode
VIR0ICON

\item[{Meaning}] \leavevmode
Rule rulename attached to an IMS Connect line specifies an unknown entry point epname.

\item[{Action}] \leavevmode
Check the entry point name specified in the rule definition.

\end{description}


\subsubsection{VIRIC56W linename IS REQUESTING CLOSE}
\label{\detokenize{messages:viric56w-linename-is-requesting-close}}\begin{description}
\item[{Module}] \leavevmode
VIR0ICON

\item[{Meaning}] \leavevmode
IMS Connect is requesting termination of the RESUME TPIPE connection on the IMS Connect line whose external name is linename.

\item[{Action}] \leavevmode
None.

\end{description}


\subsubsection{VIRIC57E linename IN ENTRY POINT ‘epname’, THE TRANSACTION ‘tranname’ WAS NOT FOUND, AND THERE IS NO DEFAULT TRANSACTION}
\label{\detokenize{messages:viric57e-linename-in-entry-point-epname-the-transaction-tranname-was-not-found-and-there-is-no-default-transaction}}\begin{description}
\item[{Module}] \leavevmode
VIR0ICON

\item[{Meaning}] \leavevmode
VIRTEL received a request on the IMS Connect line whose external name is linename. The request header specified transaction name tranname. The request could not be processed because the entry point epname does not contain a transaction whose external name is tranname, nor does it contain a default transaction whose external name is the same as the entry point name.

\item[{Action}] \leavevmode
Correct the transaction name specified by the calling IMS program. Alternatively, in entry point epname add a transaction named either tranname or epname.

\end{description}


\subsection{Messages VIRIExxx}
\label{\detokenize{messages:messages-viriexxx}}

\subsubsection{VIRIE01I INVERSE PCNE INITIALISATION FOR linename (n-xxxxxx)}
\label{\detokenize{messages:virie01i-inverse-pcne-initialisation-for-linename-n-xxxxxx}}\begin{description}
\item[{Module}] \leavevmode
VIR00IE

\item[{Meaning}] \leavevmode
The AntiPCNE line with external name linename and internal name n-xxxxxx has been initialised.

\item[{Action}] \leavevmode
None.

\end{description}


\subsubsection{VIRIE02I termid luname CONNECTED TO applname}
\label{\detokenize{messages:virie02i-termid-luname-connected-to-applname}}\begin{description}
\item[{Module}] \leavevmode
VIR00IE

\item[{Meaning}] \leavevmode
An SNA session has been established between the relay LU luname (associated with the AntiPCNE terminal termid) and the partner application applname.

\item[{Action}] \leavevmode
None.

\end{description}


\subsubsection{VIRIE03I termid luname DISCONNECTED BY applname}
\label{\detokenize{messages:virie03i-termid-luname-disconnected-by-applname}}\begin{description}
\item[{Module}] \leavevmode
VIR00IE

\item[{Meaning}] \leavevmode
Partner application applname has terminated its SNA session with the relay LU luname associated with the AntiPCNE terminal termid.

\item[{Action}] \leavevmode
None.

\end{description}


\subsubsection{VIRIE04I PCNE LINE linename WAS STOPPED}
\label{\detokenize{messages:virie04i-pcne-line-linename-was-stopped}}\begin{description}
\item[{Module}] \leavevmode
VIR00IE

\item[{Meaning}] \leavevmode
The AntiPCNE line with external name linename was stopped, either by console command, or by setting the “Possible calls” parameter to 0 in the line definition.

\item[{Action}] \leavevmode
None.

\end{description}


\subsubsection{VIRIE05I termid RELAY CHANGED FROM oldlu TO newlu}
\label{\detokenize{messages:virie05i-termid-relay-changed-from-oldlu-to-newlu}}\begin{description}
\item[{Module}] \leavevmode
VIR00IE

\item[{Meaning}] \leavevmode
The terminal termid attached to an AntiPCNE line was previously active using relay LU name oldlu. Following an update to the definition and a restart of the AntiPCNE line, the terminal was restarted with a new relay LU name newlu.

\item[{Action}] \leavevmode
None.

\end{description}


\subsubsection{VIRIE45I termid CALL CLEARED BY VIRTEL CAUSE=xx DIAG=yy}
\label{\detokenize{messages:virie45i-termid-call-cleared-by-virtel-cause-xx-diag-yy}}\begin{description}
\item[{Module}] \leavevmode
VIR00IE

\item[{Meaning}] \leavevmode
VIRTEL terminated the call on AntiPCNE terminal termid. xx and yy are the cause and diagnostic codes generated by VIRTEL. The following codes are possible:
\begin{itemize}
\item {} 
Cause=00 Diag=00 : Call was terminated by local PCNE application

\item {} 
Cause=13 Diag=81 : Call could not be connected to the local PCNE application

\item {} 
Cause=xx Diag=yy : Error communicating with local PCNE application (xx,yy = VTAM RTNCD/FDBK codes)

\end{itemize}

\item[{Action}] \leavevmode
For Cause=00, Diag=00: None. For other codes, examine preceding error messages to determine the cause.

\end{description}


\subsubsection{VIRIE52I linename DECONNECTING termname}
\label{\detokenize{messages:virie52i-linename-deconnecting-termname}}\begin{description}
\item[{Module}] \leavevmode
VIR00IE

\item[{Meaning}] \leavevmode
Terminal termname disconnected from the AntiPCNE line whose external name is linename.

\item[{Action}] \leavevmode
None.

\end{description}


\subsection{Messages VIRIFxxx}
\label{\detokenize{messages:messages-virifxxx}}

\subsubsection{VIRIF01I INVERSE FASTC INITIALISATION FOR linename (n-xxxxxx)}
\label{\detokenize{messages:virif01i-inverse-fastc-initialisation-for-linename-n-xxxxxx}}\begin{description}
\item[{Module}] \leavevmode
VIR00IF

\item[{Meaning}] \leavevmode
The AntiFastConnect line with external name linename and internal name n-xxxxxx has been initialised..

\item[{Action}] \leavevmode
None.

\end{description}


\subsubsection{VIRIF02I linename (n-xxxxxx) CONNECTED TO ctcpname}
\label{\detokenize{messages:virif02i-linename-n-xxxxxx-connected-to-ctcpname}}\begin{description}
\item[{Module}] \leavevmode
VIR00IF

\item[{Meaning}] \leavevmode
The AntiFastConnect line with external name linename and internal name n-xxxxxx has established a connection with the partner CTCP ctcpname.

\item[{Action}] \leavevmode
None.

\end{description}


\subsubsection{VIRIF03I linename (n-xxxxxx) DISCONNECTED}
\label{\detokenize{messages:virif03i-linename-n-xxxxxx-disconnected}}\begin{description}
\item[{Module}] \leavevmode
VIR00IF

\item[{Meaning}] \leavevmode
The connection between the AntiFastConnect line with external name linename and internal name n-xxxxxx has been terminated by the partner CTCP.

\item[{Action}] \leavevmode
None.

\end{description}


\subsubsection{VIRIF04I FASTC LINE linename WAS STOPPED}
\label{\detokenize{messages:virif04i-fastc-line-linename-was-stopped}}\begin{description}
\item[{Module}] \leavevmode
VIR00IF

\item[{Meaning}] \leavevmode
The AntiFastConnect line with external name linename was stopped, either by console command, or by setting the “Possible calls” parameter to 0 in the line definition.

\item[{Action}] \leavevmode
None.

\end{description}


\subsubsection{VIRIF42I FASTC : UNSUPPORTED COMMAND = xx ON MCH linename}
\label{\detokenize{messages:virif42i-fastc-unsupported-command-xx-on-mch-linename}}\begin{description}
\item[{Module}] \leavevmode
VIR00IF

\item[{Meaning}] \leavevmode
A packet received from the partner CTCP is of unknown type. xx represents the hexadecimal value of the first byte of the packet, linename represents the external name of the AntiFastConnect line on which the incident occurred.

\item[{Action}] \leavevmode
Obtain a SNAP. Contact technical support if the incident persists.

\end{description}


\subsubsection{VIRIF52I linename DECONNECTING termname}
\label{\detokenize{messages:virif52i-linename-deconnecting-termname}}\begin{description}
\item[{Module}] \leavevmode
VIR00IF

\item[{Meaning}] \leavevmode
Terminal termname disconnected from the AntiFastConnect line with external name linename.

\item[{Action}] \leavevmode
None.

\end{description}


\subsection{Messages VIRIGxxx}
\label{\detokenize{messages:messages-virigxxx}}

\subsubsection{VIRIG01I INVERSE GATE INITIALISATION FOR linename (n-xxxxxx)}
\label{\detokenize{messages:virig01i-inverse-gate-initialisation-for-linename-n-xxxxxx}}\begin{description}
\item[{Module}] \leavevmode
VIR00IG

\item[{Meaning}] \leavevmode
The AntiGATE line with external name linename and internal name n-xxxxxx has been initialised.

\item[{Action}] \leavevmode
None.

\end{description}


\subsubsection{VIRIG02I n-xxxxxx luname CONNECTED TO applname}
\label{\detokenize{messages:virig02i-n-xxxxxx-luname-connected-to-applname}}\begin{description}
\item[{Module}] \leavevmode
VIR00IG

\item[{Meaning}] \leavevmode
An SNA session has been established between the partner CTCP application applname and the VIRTEL MCH LU luname for the AntiGATE line whose internal name is n-xxxxxx.

\item[{Action}] \leavevmode
None.

\end{description}


\subsubsection{VIRIG03I n-xxxxxx luname DISCONNECTED BY applname}
\label{\detokenize{messages:virig03i-n-xxxxxx-luname-disconnected-by-applname}}\begin{description}
\item[{Module}] \leavevmode
VIR00IG

\item[{Meaning}] \leavevmode
The partner CTCP application applname has terminated its SNA session with the VIRTEL MCH LU luname for the AntiGATE line whose internal name is n-xxxxxx.

\item[{Action}] \leavevmode
None.

\end{description}


\subsubsection{VIRIG04I n-xxxxxx GATE LINE linename WAS STOPPED}
\label{\detokenize{messages:virig04i-n-xxxxxx-gate-line-linename-was-stopped}}\begin{description}
\item[{Module}] \leavevmode
VIR00IG

\item[{Meaning}] \leavevmode
The AntiGATE line with external name linename and internal name n-xxxxxx was stopped, either by console command, or by setting the “Possible calls” parameter to 0 in the line definition.

\item[{Action}] \leavevmode
None.

\end{description}


\subsubsection{VIRIG05I n-xxxxxx(linename) LOCAL LU NAME CHANGED FROM oldlu TO newlu}
\label{\detokenize{messages:virig05i-n-xxxxxx-linename-local-lu-name-changed-from-oldlu-to-newlu}}\begin{description}
\item[{Module}] \leavevmode
VIR00IG

\item[{Meaning}] \leavevmode
The AntiGATE line with external name linename and internal name n-xxxxxx was previously active using MCH name oldlu. Following an update of the field “Local ident” in the line definition and a restart of the line, the MCH name is now newlu.

\item[{Action}] \leavevmode
None.

\end{description}


\subsubsection{VIRIG12I termid luname CONNECTED TO applname}
\label{\detokenize{messages:virig12i-termid-luname-connected-to-applname}}\begin{description}
\item[{Module}] \leavevmode
VIR00IG

\item[{Meaning}] \leavevmode
An SNA session has been established between the partner CTCP application applname and the VIRTEL CVC relay LU luname associated with the AntiGATE terminal termid. This message can be suppressed by the SILENCE parameter in the VIRTCT.

\item[{Action}] \leavevmode
None.

\end{description}


\subsubsection{VIRIG13I termid luname DISCONNECTED BY applname}
\label{\detokenize{messages:virig13i-termid-luname-disconnected-by-applname}}\begin{description}
\item[{Module}] \leavevmode
VIR00IG

\item[{Meaning}] \leavevmode
The partner CTCP application applname has terminated its SNA session with the VIRTEL CVC relay LU luname associated with the AntiGATE terminal termid. This message can be suppressed by the SILENCE parameter in the VIRTCT.

\item[{Action}] \leavevmode
None.

\end{description}


\subsubsection{VIRIG42I n-xxxxxx(linename) RECEIVED UNSUPPORTED COMMAND xx FROM applname ON MCH luname}
\label{\detokenize{messages:virig42i-n-xxxxxx-linename-received-unsupported-command-xx-from-applname-on-mch-luname}}\begin{description}
\item[{Module}] \leavevmode
VIR00IG

\item[{Meaning}] \leavevmode
A packet received from the partner CTCP application applname is of unknown type. xx represents the hexadecimal value of the first byte of the packet, linename represents the external name of the AntiGATE line on which the incident occurred, n-xxxxxx is the internal name of the AntiGATE line, and luname is the LU name of the associated MCH.

\item[{Action}] \leavevmode
Obtain a SNAP. Contact technical support if the incident persists.

\end{description}


\subsubsection{VIRIG43I termid luname CLEAR RECEIVED FROM LOCAL applname CAUSE=xx DIAG=yy}
\label{\detokenize{messages:virig43i-termid-luname-clear-received-from-local-applname-cause-xx-diag-yy}}\begin{description}
\item[{Module}] \leavevmode
VIR00IG

\item[{Meaning}] \leavevmode
The VIRTEL AntiGATE line with internal name n-xxxxxx received an X25 CLEAR command from the CTCP application applname on the control session with MCH LU luname. VIRTEL determined that the command was associated with CVC terminal termid. xx and yy are the X25 cause and diagnostic codes in hexadecimal. This message is suppressed if the cause and diagnostic codes are zero and the SILENCE parameter is specified in the VIRTCT.

\item[{Action}] \leavevmode
If the cause and diagnostic codes are zero, this indicates a normal end of call. Otherwise, use the cause and diagnostic codes to determine why the CTCP application issued the CLEAR.

\end{description}


\subsubsection{VIRIG44I termid luname CLEAR RECEIVED FROM LOCAL applname CAUSE=xx DIAG=yy}
\label{\detokenize{messages:virig44i-termid-luname-clear-received-from-local-applname-cause-xx-diag-yy}}\begin{description}
\item[{Module}] \leavevmode
VIR00IG

\item[{Meaning}] \leavevmode
VIRTEL received an X25 CLEAR command from the CTCP application applname. The command was received on the data session with CVC relay LU name luname associated with AntiGATE terminal termid. xx and yy are the X25 cause and diagnostic codes in hexadecimal. This message is suppressed if the cause and diagnostic codes are zero and the SILENCE parameter is specified in the VIRTCT.

\item[{Action}] \leavevmode
If the cause and diagnostic codes are zero, this indicates a normal end of call. Otherwise, use the cause and diagnostic codes to determine why the CTCP application issued the CLEAR.

\end{description}


\subsubsection{VIRIG45I n-xxxxxx(linename) CALL FROM ctcpappl CLEARED BY VIRTEL CAUSE=xx DIAG=yy}
\label{\detokenize{messages:virig45i-n-xxxxxx-linename-call-from-ctcpappl-cleared-by-virtel-cause-xx-diag-yy}}\begin{description}
\item[{Module}] \leavevmode
VIR00IG

\item[{Meaning}] \leavevmode
VIRTEL terminated the outbound call by the CTCP application ctcpappl on the AntiGATE line whose internal name is n-xxxxxx and whose external name is linename. xx and yy are the cause and diagnostic codes generated by VIRTEL. The following codes are possible:
\begin{itemize}
\item {} 
Cause=01 Diag=E5 : Line n-xxxxxx has no available terminals which accept calls from CTCP

\item {} 
Cause=02 Diag=E5 : Could not connect CVC LU to ctcpappl

\item {} 
Cause=03 Diag=E5 : Error sending Call Connected packet to ctcpappl

\end{itemize}

\item[{Action}] \leavevmode
For Cause=01, Diag=E5: Check the definition of the terminals associated with AntiGATE line n-xxxxxx and verify that the “Possible calls” field is set to 1 or 3. For other codes, examine preceding error messages to determine the cause.

\end{description}


\subsubsection{VIRIG52I linename DECONNECTING termname}
\label{\detokenize{messages:virig52i-linename-deconnecting-termname}}\begin{description}
\item[{Module}] \leavevmode
VIR00IG

\item[{Meaning}] \leavevmode
Terminal termname disconnected from the AntiGATE line with external name linename.

\item[{Action}] \leavevmode
None.

\end{description}


\subsection{Messages VIRIPAxx}
\label{\detokenize{messages:messages-viripaxx}}

\subsubsection{VIRIPA1I linename CONNECTING termid}
\label{\detokenize{messages:viripa1i-linename-connecting-termid}}\begin{description}
\item[{Module}] \leavevmode
VIR0PASS

\item[{Meaning}] \leavevmode
An incoming call on the VIRPASS/NT line whose external name is linename has been allocated to terminal termid.

\item[{Action}] \leavevmode
None.

\end{description}


\subsubsection{VIRIPA2I linename DECONNECTING termid}
\label{\detokenize{messages:viripa2i-linename-deconnecting-termid}}\begin{description}
\item[{Module}] \leavevmode
VIR0PASS

\item[{Meaning}] \leavevmode
Terminal termid has disconnected from the VIRPASS/NT line whose external name is linename.

\item[{Action}] \leavevmode
None.

\end{description}


\subsubsection{VIRIPA3I linename ACCEPTED termid}
\label{\detokenize{messages:viripa3i-linename-accepted-termid}}\begin{description}
\item[{Module}] \leavevmode
VIR0PASS

\item[{Meaning}] \leavevmode
An incoming call on the VIRPASS/NT line whose external name is linename has been accepted on terminal termid.

\item[{Action}] \leavevmode
None.

\end{description}


\subsubsection{VIRIPA4I linename NO OUTPUT LINE FOUND FOR termid}
\label{\detokenize{messages:viripa4i-linename-no-output-line-found-for-termid}}\begin{description}
\item[{Module}] \leavevmode
VIR0PASS

\item[{Meaning}] \leavevmode
VIRTEL was unable to connect output terminal termid on the VIRPASS/NT line whose external name is linename.

\item[{Action}] \leavevmode
Wait until the partner VIRNT server retries the connection. Restart the connection at the VIRNT server if necessary. Check the definition of line linename and the terminals associated with line linename.

\end{description}


\subsubsection{VIRIPA6E linename PROTOCOL ERROR - SNAP TAKEN}
\label{\detokenize{messages:viripa6e-linename-protocol-error-snap-taken}}\begin{description}
\item[{Module}] \leavevmode
VIR0PASS

\item[{Meaning}] \leavevmode
VIRTEL received a message received on the VIRPASS/NT line whose external name is linename. The message did not contain a valid VIRPASS identifier.

\item[{Action}] \leavevmode
Contact technical support.

\end{description}


\subsubsection{VIRIPA7I linename HAS NO MORE PSEUDO TERMINALS}
\label{\detokenize{messages:viripa7i-linename-has-no-more-pseudo-terminals}}\begin{description}
\item[{Module}] \leavevmode
VIR0PASS

\item[{Meaning}] \leavevmode
There are no more terminals associated with the VIRPASS/NT line whose external name is linename.

\item[{Action}] \leavevmode
Check the definition of the line. Add more terminals if necessary.

\end{description}


\subsubsection{VIRIPA8E linename UNKNOWN CORRELATOR xxxxxxxx IN COMMAND c}
\label{\detokenize{messages:viripa8e-linename-unknown-correlator-xxxxxxxx-in-command-c}}\begin{description}
\item[{Module}] \leavevmode
VIR0PASS

\item[{Meaning}] \leavevmode
VIRTEL received a message with an unknown correlator on the VIRPASS/NT line whose external name is linename.

\item[{Action}] \leavevmode
Contact technical support.

\end{description}


\subsubsection{VIRIPA9I INITIALISATION FOR linename (n-xxxxxx), VERSION 4.xx}
\label{\detokenize{messages:viripa9i-initialisation-for-linename-n-xxxxxx-version-4-xx}}\begin{description}
\item[{Module}] \leavevmode
VIRPASS

\item[{Meaning}] \leavevmode
The VIRPASS/NT line with external name linename and internal name n-xxxxxx has been initialised.

\item[{Action}] \leavevmode
None.

\end{description}


\subsection{Messages VIRNAxxx}
\label{\detokenize{messages:messages-virnaxxx}}

\subsubsection{VIRNA01I NATIVE IP INITIALISATION FOR linename (n-xxxxxx), VERSION x.xx}
\label{\detokenize{messages:virna01i-native-ip-initialisation-for-linename-n-xxxxxx-version-x-xx}}\begin{description}
\item[{Module}] \leavevmode
VIRNATV

\item[{Meaning}] \leavevmode
The line with external name linename and internal name n-xxxxxx has been initialised with the TCP/IP NATIVE protocol.

\item[{Action}] \leavevmode
None.

\end{description}


\subsubsection{VIRNA04I NATIVE IP LINE linename WAS STOPPED}
\label{\detokenize{messages:virna04i-native-ip-line-linename-was-stopped}}\begin{description}
\item[{Module}] \leavevmode
VIRNATV

\item[{Meaning}] \leavevmode
The line with external name linename was stopped, either by console command, or by setting the “Possible calls” parameter to 0 in the line definition.

\item[{Action}] \leavevmode
None.

\end{description}


\subsubsection{VIRNA05E linename UNDEFINED PARTNER n.n.n.n:p IS CALLING m.m.m.m:q}
\label{\detokenize{messages:virna05e-linename-undefined-partner-n-n-n-n-p-is-calling-m-m-m-m-q}}\begin{description}
\item[{Module}] \leavevmode
VIRNATV

\item[{Meaning}] \leavevmode
VIRTEL has rejected an incoming call to a native IP line. n.n.n.n:p represents the IP address and port of the caller, m.m.m.m:q represents the called IP address and port in VIRTEL. VIRTEL rejected the call because none of the TCP/IP lines listening on address/port m.m.m.m:q specifies address n.n.n.n as partner.

\item[{Action}] \leavevmode
Define an XOT line specifying m.m.m.m:q in the “Local ident” field and n.n.n.n:p in the “Remote ident” field

\end{description}


\subsubsection{VIRNA06I linename (n-xxxxxx) IS USING A GENERIC ADDRESS}
\label{\detokenize{messages:virna06i-linename-n-xxxxxx-is-using-a-generic-address}}\begin{description}
\item[{Module}] \leavevmode
VIRNATV

\item[{Meaning}] \leavevmode
The IP address and port number of the line with external name linename and internal name n-xxxxxx is treated as sharable with other TCP/IP lines as a result of the SHAREDA=Y parameter specified in the definition of line linename.

\item[{Action}] \leavevmode
None.

\end{description}


\subsubsection{VIRNA07I linename (n-xxxxxx) IS USING A GENERIC ADDRESS}
\label{\detokenize{messages:virna07i-linename-n-xxxxxx-is-using-a-generic-address}}\begin{description}
\item[{Module}] \leavevmode
VIRNATV

\item[{Meaning}] \leavevmode
The IP address and port number of the line with external name linename and internal name n-xxxxxx is shared with another TCP/IP line. VIRTEL found the same address and port number specified in the definition of another line. Neither SHAREDA nor UNIQUEP were specified in the definition of line linename.

\item[{Action}] \leavevmode
None.

\end{description}


\subsubsection{VIRNA26E linename HAS NO PSEUDO TERMINALS: CANNOT CONTINUE}
\label{\detokenize{messages:virna26e-linename-has-no-pseudo-terminals-cannot-continue}}\begin{description}
\item[{Module}] \leavevmode
VIRNATV

\item[{Meaning}] \leavevmode
No terminals are defined for the native IP line whose external name is linename.

\item[{Action}] \leavevmode
VIRTEL continues but the line cannot be used. Define the associated terminals, then stop and start the line.

\end{description}


\subsubsection{VIRNA27E linename IS REJECTED AT ENTRY POINT epname}
\label{\detokenize{messages:virna27e-linename-is-rejected-at-entry-point-epname}}\begin{description}
\item[{Module}] \leavevmode
VIRNATV

\item[{Meaning}] \leavevmode
VIRTEL received an incoming call on the line whose external name is linename, but the entry point epname specified on the line definition does not exist.

\item[{Action}] \leavevmode
Check the definition of the line.

\end{description}


\subsubsection{VIRNA28I linename CALLER ipaddr:port GETS ENTRY POINT ‘epname’ FROM RULE ‘rulename’}
\label{\detokenize{messages:virna28i-linename-caller-ipaddr-port-gets-entry-point-epname-from-rule-rulename}}\begin{description}
\item[{Module}] \leavevmode
VIRNATV

\item[{Meaning}] \leavevmode
VIRTEL selected entry point epname for an incoming call on the line whose external name is linename, because the call matched the rule rulename.

\item[{Action}] \leavevmode
None.

\end{description}


\subsubsection{VIRNA51I linename CONNECTING termname}
\label{\detokenize{messages:virna51i-linename-connecting-termname}}\begin{description}
\item[{Module}] \leavevmode
VIRNATV

\item[{Meaning}] \leavevmode
An incoming call on the line whose external name is linename has been allocated to terminal termname. This message can be suppressed by the SILENCE parameter in the VIRTCT.

\item[{Action}] \leavevmode
None.

\end{description}


\subsubsection{VIRNA52I linename DECONNECTING termname}
\label{\detokenize{messages:virna52i-linename-deconnecting-termname}}\begin{description}
\item[{Module}] \leavevmode
VIRNATV

\item[{Meaning}] \leavevmode
Terminal termname has been disconnected from the line whose external name is linename.

\item[{Action}] \leavevmode
None.

\end{description}


\subsubsection{VIRNA53E linename HAS NO MORE PSEUDO TERMINALS}
\label{\detokenize{messages:virna53e-linename-has-no-more-pseudo-terminals}}\begin{description}
\item[{Module}] \leavevmode
VIRNATV

\item[{Meaning}] \leavevmode
There are no more terminals associated with the line whose external name is linename.

\item[{Action}] \leavevmode
Check the definition of the line. Add more terminals if necessary.

\end{description}


\subsubsection{VIRNA55E INVALID RULE rulename ENTRY POINT ‘epname’}
\label{\detokenize{messages:virna55e-invalid-rule-rulename-entry-point-epname}}\begin{description}
\item[{Module}] \leavevmode
VIRNATV

\item[{Meaning}] \leavevmode
Rule rulename attached to a native IP line specifies an unknown entry point epname.

\item[{Action}] \leavevmode
Check the entry point name specified in the rule definition.

\end{description}


\subsubsection{VIRNA98E linename ERROR errcode RECEIVING FROM SOCKET sockno}
\label{\detokenize{messages:virna98e-linename-error-errcode-receiving-from-socket-sockno}}\begin{description}
\item[{Module}] \leavevmode
VIRNATV

\item[{Meaning}] \leavevmode
An error occurred when receiving a message from the native IP line whose external name is linename. errcode is an error code whose meaning is:
- 00000001 : No session exists for this line.
- 00000002 : Unable to identify incoming message with an existing session.
- 00000003 : No more pseudo terminals, or line not linked to TCP.
- 00000004 : TCP/IP receive error; see preceding message VIRT923E or VIRT924E.
- 00000005 : Socket is closed.
- 00000006 : A message received on the line does not conform to the VIRTEL native IP protocol. The message length is invalid or exceeds the maximum packet size.

\item[{Action}] \leavevmode
If the cause cannot be determined by preceding messages, use the VIRTEL SNAP command to produce a dump of the internal trace table. Determine whether the partner sent an invalid message, and if so, perform the appropriate diagnostic procedures. Otherwise contact technical support.

\end{description}


\subsubsection{VIRNA99I linename ERROR errcode SENDING TO SOCKET socknum}
\label{\detokenize{messages:virna99i-linename-error-errcode-sending-to-socket-socknum}}\begin{description}
\item[{Module}] \leavevmode
VIRNATV

\item[{Meaning}] \leavevmode
An error occurred when sending a message to the native IP line whose external name is linename. errcode is an error code whose meaning is:
- 00000001 : Invalid pseudo terminal.
- 00000002 : Message not identified with an active session.
- 00000003 : Line not linked to TCP stack.
- 00000004 : TCP/IP send error; see preceding message VIRT923E or VIRT924E.
- 00000005 : VIRTEL internal task dispatching error.
- 00000006 : Session cleared before message could be sent.

\item[{Action}] \leavevmode
If the cause cannot be determined by preceding messages, use the VIRTEL SNAP command to produce a dump of the internal trace table and contact technical support.

\end{description}


\subsection{Messages VIRP1xxx}
\label{\detokenize{messages:messages-virp1xxx}}

\subsubsection{VIRP121E PAGE NOT FOUND FOR termid ENTRY POINT ‘epname’ SCENARIO ‘scenname’ DIRECTORY ‘tranid’(dirname) NAME ‘filename’ URL’url’}
\label{\detokenize{messages:virp121e-page-not-found-for-termid-entry-point-epname-scenario-scenname-directory-tranid-dirname-name-filename-url-url}}\begin{description}
\item[{Module}] \leavevmode
VIR0P12

\item[{Meaning}] \leavevmode
The scenario scenname invoked from terminal termid requested a file named filename which does not exist. dirname is the name of the current VIRTEL directory and tranid is the external name of the corresponding transaction which belongs to entry point epname. VIRTEL sent a 404 NOT FOUND reply to the terminal with the text “Failed on missing page:filename”.

\item[{Action}] \leavevmode
Change the scenario to specify the correct file name.

\end{description}


\subsubsection{VIRP122E VARIABLE NOT FOUND FOR termid ENTRY POINT ‘epname’ SCENARIO ‘scenname’ DIRECTORY ‘tranid’(dirname) NAME ‘varname’ URL’url’}
\label{\detokenize{messages:virp122e-variable-not-found-for-termid-entry-point-epname-scenario-scenname-directory-tranid-dirname-name-varname-url-url}}\begin{description}
\item[{Module}] \leavevmode
VIR0P12

\item[{Meaning}] \leavevmode
The scenario scenname invoked from terminal termid requested a VIRTEL variable named varname which does not exist. dirname is the name of the current VIRTEL directory and tranid is the external name of the corresponding transaction which belongs to entry point epname. VIRTEL sent a 404 NOT FOUND reply to the terminal with the text “Failed on missing variable:varname”.

\item[{Action}] \leavevmode
Change the scenario to specify the correct variable name.

\end{description}


\subsubsection{VIRP123E SCENARIO NOT FOUND FOR termid ENTRY POINT ‘epname’ DIRECTORY ‘dirname’ NAME ‘filename’ URL ‘url’}
\label{\detokenize{messages:virp123e-scenario-not-found-for-termid-entry-point-epname-directory-dirname-name-filename-url-url}}\begin{description}
\item[{Module}] \leavevmode
VIR0P12

\item[{Meaning}] \leavevmode
The scenario filename (filetype .390) invoked from terminal termid for URL url was not found in the scenario directory dirname defined in entry point epname.

\item[{Action}] \leavevmode
Use Virtel Studio to upload the scenario to the directory specified by the entry point.

\end{description}


\subsubsection{VIRP1251I termid CONNECTED TO “epname”}
\label{\detokenize{messages:virp1251i-termid-connected-to-epname}}\begin{description}
\item[{Module}] \leavevmode
VIR0P12

\item[{Meaning}] \leavevmode
An input message from terminal termid has been assigned to entry point epname.

\item[{Action}] \leavevmode
None.

\end{description}


\subsection{Messages VIRPAxxx}
\label{\detokenize{messages:messages-virpaxxx}}

\subsubsection{VIRPA54E termid INVALID REQUEST ON linename FROM CALLER ipaddr:port ENTRY POINT ‘epname’ TRANSACTION OR SERVER ‘servid’}
\label{\detokenize{messages:virpa54e-termid-invalid-request-on-linename-from-caller-ipaddr-port-entry-point-epname-transaction-or-server-servid}}\begin{description}
\item[{Module}] \leavevmode
VIR0PASS

\item[{Meaning}] \leavevmode
VIRTEL received an incoming call on terminal termid associated with the VIRPASS line whose external name is linename. The call originated from IP address ipaddr port port. The service requested was servid. But the chosen entry point epname contains no transaction whose external name is servid, nor does it contain a default transaction capable of invoking the external server named servid.

\item[{Action}] \leavevmode
Check that entry point epname has been defined in accordance with the documentation for setting up the VIRTEL Videotex Plugin.

\end{description}


\subsection{Messages VIRPExxx}
\label{\detokenize{messages:messages-virpexxx}}

\subsubsection{VIRPE27E termid IS REJECTED AT ENTRY POINT epname}
\label{\detokenize{messages:virpe27e-termid-is-rejected-at-entry-point-epname}}\begin{description}
\item[{Module}] \leavevmode
VIR00PE

\item[{Meaning}] \leavevmode
VIRTEL received an incoming call from terminal termid, but the requested entry point epname does not exist.

\item[{Action}] \leavevmode
Check the entry point specified in the logdata (for SNA) or in the call user data (for X25).

\end{description}


\subsubsection{VIRPE52I termid DISCONNECTED AFTER nn MINUTES}
\label{\detokenize{messages:virpe52i-termid-disconnected-after-nn-minutes}}\begin{description}
\item[{Module}] \leavevmode
VIR00PE

\item[{Meaning}] \leavevmode
Terminal termid was connected to an application and has disconnected after nn minutes of connection time.

\item[{Action}] \leavevmode
None.

\end{description}


\subsection{Messages VIRPFxxx}
\label{\detokenize{messages:messages-virpfxxx}}

\subsubsection{VIRPF01I INITIALISATION FOR linename (n-xxxxxx), VERSION x.xx}
\label{\detokenize{messages:virpf01i-initialisation-for-linename-n-xxxxxx-version-x-xx}}\begin{description}
\item[{Module}] \leavevmode
VIR0PRF

\item[{Meaning}] \leavevmode
The line with external name linename and internal name n-xxxxxx has been initialised with the PREFIXED protocol.

\item[{Action}] \leavevmode
None.

\end{description}


\subsubsection{VIRPF04I LINE linename WAS STOPPED}
\label{\detokenize{messages:virpf04i-line-linename-was-stopped}}\begin{description}
\item[{Module}] \leavevmode
VIR0PRF

\item[{Meaning}] \leavevmode
The line with external name linename was stopped, either by console command, or by setting the “Possible calls” parameter to 0 in the line definition.

\item[{Action}] \leavevmode
None.

\end{description}


\subsubsection{VIRPF27E linename IS REJECTED AT ENTRY POINT epname}
\label{\detokenize{messages:virpf27e-linename-is-rejected-at-entry-point-epname}}\begin{description}
\item[{Module}] \leavevmode
VIR0PRF

\item[{Meaning}] \leavevmode
VIRTEL received an incoming call on the line whose external name is linename, but the entry point epname specified on the line definition does not exist.

\item[{Action}] \leavevmode
Check the definition of the line.

\end{description}


\subsubsection{VIRPF28I linename CALLER ipaddr:port GETS ENTRY POINT ‘epname’ FROM RULE ‘rulename’}
\label{\detokenize{messages:virpf28i-linename-caller-ipaddr-port-gets-entry-point-epname-from-rule-rulename}}\begin{description}
\item[{Module}] \leavevmode
VIR0PRF

\item[{Meaning}] \leavevmode
VIRTEL selected entry point epname for an incoming call on the line whose external name is linename, because the call matched the rule rulename. This message can be suppressed by the SILENCE parameter in the VIRTCT.

\item[{Action}] \leavevmode
None.

\end{description}


\subsubsection{VIRPF29I linename HAS NO MORE PSEUDO TERMINALS}
\label{\detokenize{messages:virpf29i-linename-has-no-more-pseudo-terminals}}\begin{description}
\item[{Module}] \leavevmode
VIR0PRF

\item[{Meaning}] \leavevmode
There are no more terminals associated with the line whose external name is linename.

\item[{Action}] \leavevmode
Check the definition of the line. Add more terminals if necessary.

\end{description}


\subsubsection{VIRPF51I linename CONNECTING termname}
\label{\detokenize{messages:virpf51i-linename-connecting-termname}}\begin{description}
\item[{Module}] \leavevmode
VIR0PRF

\item[{Meaning}] \leavevmode
An incoming call on the line whose external name is linename has been allocated to terminal termname. This message can be suppressed by the SILENCE parameter in the VIRTCT.

\item[{Action}] \leavevmode
None.

\end{description}


\subsubsection{VIRPF52I linename DECONNECTING termname}
\label{\detokenize{messages:virpf52i-linename-deconnecting-termname}}\begin{description}
\item[{Module}] \leavevmode
VIR0PRF

\item[{Meaning}] \leavevmode
Terminal termname has been disconnected from the line whose external name is linename. This message can be suppressed by the SILENCE parameter in the VIRTCT.

\item[{Action}] \leavevmode
None.

\end{description}


\subsubsection{VIRPF54E INVALID REQUEST ON linename ENTRY POINT ‘epname’ DIRECTORY ‘dirname’}
\label{\detokenize{messages:virpf54e-invalid-request-on-linename-entry-point-epname-directory-dirname}}\begin{description}
\item[{Module}] \leavevmode
VIR0PRF

\item[{Meaning}] \leavevmode
VIRTEL received a request on the line whose external name is linename. The requested directory path dirname does not exist under the entry point epname.

\item[{Action}] \leavevmode
Check the definition of the entry point epname. Ensure that there is a type 4 transaction with external name dirname.

\end{description}


\subsubsection{VIRPF55E INVALID RULE rulename ENTRY POINT ‘epname’}
\label{\detokenize{messages:virpf55e-invalid-rule-rulename-entry-point-epname}}\begin{description}
\item[{Module}] \leavevmode
VIR0PRF

\item[{Meaning}] \leavevmode
Rule rulename attached to an MQ line specifies an unknown entry point epname.

\item[{Action}] \leavevmode
Check the entry point name specified in the rule definition.

\end{description}


\subsubsection{VIRPF99I linename MESSAGE RECEIVED FROM SOCKET socknum}
\label{\detokenize{messages:virpf99i-linename-message-received-from-socket-socknum}}\begin{description}
\item[{Module}] \leavevmode
VIR0PRF

\item[{Meaning}] \leavevmode
Input data has been received by the line whose external name is linename.

\item[{Action}] \leavevmode
None.

\end{description}


\subsubsection{VIRPF99I linename SEND TO SOCKET socknum}
\label{\detokenize{messages:virpf99i-linename-send-to-socket-socknum}}\begin{description}
\item[{Module}] \leavevmode
VIR0PRF

\item[{Meaning}] \leavevmode
Output data has been sent to the line whose external name is linename. This message can be suppressed by the SILENCE parameter in the VIRTCT.

\item[{Action}] \leavevmode
None.

\end{description}


\subsection{Messages VIRPSxxx}
\label{\detokenize{messages:messages-virpsxxx}}

\subsubsection{VIRPS01I INITIALISATION FOR linename (n-xxxxxx), VERSION nn.nn}
\label{\detokenize{messages:virps01i-initialisation-for-linename-n-xxxxxx-version-nn-nn}}\begin{description}
\item[{Module}] \leavevmode
VIRPESIT

\item[{Meaning}] \leavevmode
The VIRPESIT line with external name linename and internal name n-xxxxxx has been initialised.

\item[{Action}] \leavevmode
None.

\end{description}


\subsubsection{VIRPS02I linename DECONNECTING termname}
\label{\detokenize{messages:virps02i-linename-deconnecting-termname}}\begin{description}
\item[{Module}] \leavevmode
VIRPESIT

\item[{Meaning}] \leavevmode
Terminal termname has been disconnected from the VIRPESIT line whose external name is linename.

\item[{Action}] \leavevmode
None.

\end{description}


\subsubsection{VIRPS04I VIRPESIT LINE linename WAS STOPPED}
\label{\detokenize{messages:virps04i-virpesit-line-linename-was-stopped}}\begin{description}
\item[{Module}] \leavevmode
VIRPESIT

\item[{Meaning}] \leavevmode
VIRPESIT line with external name linename was stopped, either by console command, or by setting the “Possible calls” field to 0 in the line definition.

\item[{Action}] \leavevmode
None.

\end{description}


\subsubsection{VIRPS06E linename HAS NO PSEUDO TERMINALS: CANNOT CONTINUE}
\label{\detokenize{messages:virps06e-linename-has-no-pseudo-terminals-cannot-continue}}\begin{description}
\item[{Module}] \leavevmode
VIRPESIT

\item[{Meaning}] \leavevmode
No terminals are defined for the VIRPESIT line whose external name is linename.

\item[{Action}] \leavevmode
VIRTEL continues but the line cannot be used. Define the associated terminals, then stop and start the line.

\end{description}


\subsubsection{VIRPS07I linename HAS NO MORE PSEUDO TERMINALS}
\label{\detokenize{messages:virps07i-linename-has-no-more-pseudo-terminals}}\begin{description}
\item[{Module}] \leavevmode
VIRPESIT

\item[{Meaning}] \leavevmode
There are no more terminals associated with the VIRPESIT line whose external name is linename.

\item[{Action}] \leavevmode
Check the line definition. Add more terminals if necessary.

\end{description}


\subsubsection{VIRPS08E linename PSEUDO termname LOST}
\label{\detokenize{messages:virps08e-linename-pseudo-termname-lost}}\begin{description}
\item[{Module}] \leavevmode
VIRPESIT

\item[{Meaning}] \leavevmode
VIRTEL was unable to identify the terminal termname associated with the VIRPESIT line whose external name is linename.

\item[{Action}] \leavevmode
Contact technical support.

\end{description}


\subsubsection{VIRPS11I linename PSEUDO termname RECEIVED ABORT REQUEST FROM emitter TO receiver WITH FIRST ABORT PID : pid}
\label{\detokenize{messages:virps11i-linename-pseudo-termname-received-abort-request-from-emitter-to-receiver-with-first-abort-pid-pid}}\begin{description}
\item[{Module}] \leavevmode
VIRPESIT

\item[{Meaning}] \leavevmode
VIRTEL received an abort request of type pid via terminal termname associated with the VIRPESIT line whose external name is linename. Emitter is the name of the file transfer requester and receiver is the name of the destination partner.

\item[{Action}] \leavevmode
The file transfer terminates. Inspect the emitter logs to determine why the partner application terminated the transfer.

\end{description}


\subsubsection{VIRPS12I linename RECEIVED ABORT REQUEST FOR termname (partnerid)}
\label{\detokenize{messages:virps12i-linename-received-abort-request-for-termname-partnerid}}\begin{description}
\item[{Module}] \leavevmode
VIRPESIT

\item[{Meaning}] \leavevmode
VIRTEL received an abort request via the VIRPESIT line whose external name is linename. termname is the terminal name and partnerid is the name of the destination partner.

\item[{Action}] \leavevmode
The file transfer terminates. Inspect the logs of the file transfer application to determine why the application terminated the transfer.

\end{description}


\subsubsection{VIRPS13I linename OUTBOUND CONNECTION STARTING ON termname \textbar{} linename CLIENT ‘requesterid’ SERVER ‘serverid’ PROTOCOL ‘protocol’}
\label{\detokenize{messages:virps13i-linename-outbound-connection-starting-on-termname-linename-client-requesterid-server-serverid-protocol-protocol}}\begin{description}
\item[{Module}] \leavevmode
VIRPESIT

\item[{Meaning}] \leavevmode
VIRTEL is starting an outbound file transfer session via the VIRPESIT line whose external name is linename. termname is the terminal name, requesterid is the name of the requesting partner defined in the file transfer application, serverid is the name of the destination partner, and protocol is the protocol name (for example, PESIT).

\item[{Action}] \leavevmode
None.

\end{description}


\subsubsection{VIRPS14I linename INBOUND CONNECTION STARTING ON termname TO ENTRY POINT ‘epname’ TRANSACTION ‘tranid’ \textbar{} linename CLIENT ‘’requesterid’ SERVER ‘serverid’ PROTOCOL ‘protocol’}
\label{\detokenize{messages:virps14i-linename-inbound-connection-starting-on-termname-to-entry-point-epname-transaction-tranid-linename-client-requesterid-server-serverid-protocol-protocol}}\begin{description}
\item[{Module}] \leavevmode
VIRPESIT

\item[{Meaning}] \leavevmode
VIRTEL is has received an inbound request for a file transfer session via terminal termname on the VIRPESIT line whose external name is linename. The request has been assigned to entry point epname and transaction tranid. requesterid is the name of the requesting partner defined in the file transfer application, serverid is the name of the destination partner, and protocol is the protocol name (for example, PESIT).

\item[{Action}] \leavevmode
None.

\end{description}


\subsubsection{VIRPS15I linename RECEIVED REJECT CONNECT REQUEST FROM ‘serverid’ TO ‘requesterid’ WITH DIAGNOSTIC :’hexcode’ (‘deccode’)}
\label{\detokenize{messages:virps15i-linename-received-reject-connect-request-from-serverid-to-requesterid-with-diagnostic-hexcode-deccode}}\begin{description}
\item[{Module}] \leavevmode
VIRPESIT

\item[{Meaning}] \leavevmode
A file transfer request via the VIRPESIT line whose external name is linename has been rejected by the partner file transfer application. serverid is the name of the destination partner, and requesterid is the name of the requesting partner. hexcode is the hexadecimal abort code and deccode is the decimal abort code.

\item[{Action}] \leavevmode
Inspect the logs of the file transfer application to determine why the application terminated the transfer. Refer to the file transfer application’s protocol documentation to determine the meaning of the abort codes.

\end{description}


\subsubsection{VIRPS16I linename RECEIVED REJECT CONNECT REQUEST FOR ‘termname’ WITH DIAGNOSTIC :’hexcode’ (‘deccode’)}
\label{\detokenize{messages:virps16i-linename-received-reject-connect-request-for-termname-with-diagnostic-hexcode-deccode}}\begin{description}
\item[{Module}] \leavevmode
VIRPESIT

\item[{Meaning}] \leavevmode
A file transfer request via the VIRPESIT line whose external name is linename has been rejected by the partner file transfer application. termname is the terminal name, hexcode is the hexadecimal abort code and deccode is the decimal abort code.

\item[{Action}] \leavevmode
Inspect the logs of the file transfer application to determine why the application terminated the transfer. Refer to the file transfer application’s protocol documentation to determine the meaning of the abort codes.

\end{description}


\subsubsection{VIRPS17I linename CONNECTION ENDING FOR ‘termname’}
\label{\detokenize{messages:virps17i-linename-connection-ending-for-termname}}\begin{description}
\item[{Module}] \leavevmode
VIRPESIT

\item[{Meaning}] \leavevmode
An inbound file transfer session via terminal termname on the VIRPESIT line whose external name is linename has been terminated by the partner file transfer application.

\item[{Action}] \leavevmode
None.

\end{description}


\subsubsection{VIRPS18I linename CONNECTION ENDING FOR ‘termname’}
\label{\detokenize{messages:virps18i-linename-connection-ending-for-termname}}\begin{description}
\item[{Module}] \leavevmode
VIRPESIT

\item[{Meaning}] \leavevmode
An outbound file transfer session via terminal termname on the VIRPESIT line whose external name is linename has been terminated by the partner file transfer application.

\item[{Action}] \leavevmode
None.

\end{description}


\subsubsection{VIRPS28I linename CALLER ipaddr:port GETS ENTRY POINT ‘epname’ FROM RULE ‘rulename’}
\label{\detokenize{messages:virps28i-linename-caller-ipaddr-port-gets-entry-point-epname-from-rule-rulename}}\begin{description}
\item[{Module}] \leavevmode
VIRPESIT

\item[{Meaning}] \leavevmode
VIRTEL chose entry point epname for an incoming call on the VIRPESIT line whose external name is linename, because the call matched the rule named rulename. ipaddr:port is the IP address and port number of the calling application.

\item[{Action}] \leavevmode
None.

\end{description}


\subsubsection{VIRPS29E INVALID RULE rulename ENTRY POINT ‘epname’}
\label{\detokenize{messages:virps29e-invalid-rule-rulename-entry-point-epname}}\begin{description}
\item[{Module}] \leavevmode
VIRPESIT

\item[{Meaning}] \leavevmode
Rule rulename used by a VIRPESIT line specifies a non-existent entry point epname.

\item[{Action}] \leavevmode
Check the entry point name specified in the rule definition.

\end{description}


\subsubsection{VIRPS50E linename CANNOT CONNECT termname TO CALLER ipaddr:port}
\label{\detokenize{messages:virps50e-linename-cannot-connect-termname-to-caller-ipaddr-port}}\begin{description}
\item[{Module}] \leavevmode
VIRPESIT

\item[{Meaning}] \leavevmode
An inbound file transfer session request was received on the VIRPESIT line whose external name is linename. The request originated from IP address ipaddr  and port number port and the VIRTEL terminal  name is termname.     The request could not be processed because the entry point was not valid or there was no transaction definition corresponding to the request.

\item[{Action}] \leavevmode
Refer to subsequent message displayed on the console (VIRPS52E through VIRPS57E) for detailed explanation.

\end{description}


\subsubsection{VIRPS51I linename PSEUDO termname CALLING ipaddr:port (servname)}
\label{\detokenize{messages:virps51i-linename-pseudo-termname-calling-ipaddr-port-servname}}\begin{description}
\item[{Module}] \leavevmode
VIRPESIT

\item[{Meaning}] \leavevmode
VIRTEL is sending an outbound session request to the file transfer application at IP address ipaddr and port number port. linename is the external name of the VIRPESIT line, termname is the terminal name, and servname is the name of the external server.

\item[{Action}] \leavevmode
None.

\end{description}


\subsubsection{VIRPS52E linename ENTRY POINT ‘epname’ WAS NOT FOUND}
\label{\detokenize{messages:virps52e-linename-entry-point-epname-was-not-found}}\begin{description}
\item[{Module}] \leavevmode
VIRPESIT

\item[{Meaning}] \leavevmode
An inbound file transfer session request was received on the VIRPESIT line whose external name is linename. The request could not be processed because the default entry point name specified in the line definition does not exist.

\item[{Action}] \leavevmode
Correct the default entry point name specified in the line definition.

\end{description}


\subsubsection{VIRPS53E linename DEFAULT ENTRY POINT MISSING OR INVALID}
\label{\detokenize{messages:virps53e-linename-default-entry-point-missing-or-invalid}}\begin{description}
\item[{Module}] \leavevmode
VIRPESIT

\item[{Meaning}] \leavevmode
An inbound file transfer session request was received on the VIRPESIT line whose external name is linename. The request could not be processed because it did not match any rule and the default entry point name specified in the line definition is blank.

\item[{Action}] \leavevmode
Add a rule or specify a default entry point name in the line definition.

\end{description}


\subsubsection{VIRPS54E linename IN ENTRY POINT ‘epname’, TRANSACTION ‘tranid’, THE SERVER ‘servname’ WAS NOT FOUND}
\label{\detokenize{messages:virps54e-linename-in-entry-point-epname-transaction-tranid-the-server-servname-was-not-found}}\begin{description}
\item[{Module}] \leavevmode
VIRPESIT

\item[{Meaning}] \leavevmode
An inbound file transfer session request was received on the VIRPESIT line whose external name is linename.       The request was assigned to entry point epname and transaction tranid. The transaction calls for a server named servname (either explicitly, or by reference to the destination partner name specified by the file transfer application), but there is no external server definition of that name.

\item[{Action}] \leavevmode
Add an external server definition for servname.

\end{description}


\subsubsection{VIRPS56E linename IN ENTRY POINT ‘epname’, THE TRANSACTION ‘partnerid’ WAS NOT FOUND, AND THERE IS NO DEFAULT TRANSACTION}
\label{\detokenize{messages:virps56e-linename-in-entry-point-epname-the-transaction-partnerid-was-not-found-and-there-is-no-default-transaction}}\begin{description}
\item[{Module}] \leavevmode
VIRPESIT

\item[{Meaning}] \leavevmode
An inbound file transfer session request was received on the VIRPESIT line whose external name is linename. The request was assigned to entry point epname, but the entry point has no transaction corresponding to the destination partner partnerid.

\item[{Action}] \leavevmode
Correct the definition of the entry point, either by adding a transaction whose external name is partnerid, or by adding a default transaction whose external name is the same as the entry point name epname.

\end{description}


\subsubsection{VIRPS57E linename IN ENTRY POINT ‘epname’, THE EMULATION ‘emtype’ IS NOT SUPPORTED, SHOULD BE ‘\$NONE\$’}
\label{\detokenize{messages:virps57e-linename-in-entry-point-epname-the-emulation-emtype-is-not-supported-should-be-none}}\begin{description}
\item[{Module}] \leavevmode
VIRPESIT

\item[{Meaning}] \leavevmode
An inbound file transfer session request was received on the VIRPESIT line whose external name is linename. The request was assigned to entry point epname, but the entry point specifies emulation type emtype which is not compatible with the VIRPESIT protocol.

\item[{Action}] \leavevmode
Ensure that only entry points with emulation type \$NONE\$ are specified in the rules and in the line definition for a VIRPESIT line.

\end{description}


\subsection{Messages VIRQ9xxx}
\label{\detokenize{messages:messages-virq9xxx}}

\subsubsection{VIRQ903W LINE linename HAS A SESSION STARTED WITH MQM mqmname}
\label{\detokenize{messages:virq903w-line-linename-has-a-session-started-with-mqm-mqmname}}\begin{description}
\item[{Module}] \leavevmode
VIR0Q09

\item[{Meaning}] \leavevmode
The line with external name linename has successfully connected to the message-queue manager named mqmname

\item[{Action}] \leavevmode
None.

\end{description}


\subsubsection{VIRQ904W linename OBJECT socknum WAITING FOR INPUT}
\label{\detokenize{messages:virq904w-linename-object-socknum-waiting-for-input}}\begin{description}
\item[{Module}] \leavevmode
VIR0Q09

\item[{Meaning}] \leavevmode
The line with external name linename is waiting for input data to arrive. Message VIRQ904W can be suppressed by the SILENCE parameter in the VIRTCT.

\item[{Action}] \leavevmode
None.

\end{description}


\subsubsection{VIRQ904W linename OBJECT socknum HAS MESSAGE OF LENGTH ‘hexlen’ WAITING}
\label{\detokenize{messages:virq904w-linename-object-socknum-has-message-of-length-hexlen-waiting}}\begin{description}
\item[{Module}] \leavevmode
VIR0Q09

\item[{Meaning}] \leavevmode
Input data has arrived on the line with external name linename. The length of the input data (in hexadecimal) is hexlen. Message VIRQ904W can be suppressed by the SILENCE parameter in the VIRTCT.

\item[{Action}] \leavevmode
None.

\end{description}


\subsubsection{VIRQ912W linename OBJECT socknum STARTED}
\label{\detokenize{messages:virq912w-linename-object-socknum-started}}\begin{description}
\item[{Module}] \leavevmode
VIR0Q09

\item[{Meaning}] \leavevmode
The line with external name linename has opened the message queue. Message VIRQ912W can be suppressed by the SILENCE parameter in the VIRTCT.

\item[{Action}] \leavevmode
None.

\end{description}


\subsubsection{VIRQ922W linename ENDING OBJECT socknum FOR mqmname}
\label{\detokenize{messages:virq922w-linename-ending-object-socknum-for-mqmname}}\begin{description}
\item[{Module}] \leavevmode
VIR0Q09

\item[{Meaning}] \leavevmode
The line with external name linename has closed the message queue. Message VIRQ922W can be suppressed by the SILENCE parameter in the VIRTCT.

\item[{Action}] \leavevmode
None.

\end{description}


\subsubsection{VIRQ923E linename REQ reqtype COMPLETION CODE ccc REASON CODE xx (ddd) MQM mqmname \textbar{} linename PARAM queuename}
\label{\detokenize{messages:virq923e-linename-req-reqtype-completion-code-ccc-reason-code-xx-ddd-mqm-mqmname-linename-param-queuename}}\begin{description}
\item[{Module}] \leavevmode
VIR0Q09

\item[{Meaning}] \leavevmode
A message queue manager request of type reqtype for the line whose external name is linename was not successful. mqmname is the name of the message queue manager, ccc is the message queue manager completion code, and xx and ddd are respectively the hexadecimal and decimal reason codes. queuename is the name of the message queue, if any, associated with the error.

\item[{Action}] \leavevmode
For an explanation of the error codes, refer to Appendix A of the IBM manual MQSeries Application Programming Reference.
The following codes are commonly encountered:
- REQ MQCONN COMPLETION CODE 00000002 REASON CODE 0000080A (00002058) : indicates that the message queue manager name is not known. Specify the correct name in the MQn parameter of the VIRTCT.
- REQ MQOPEN COMPLETION CODE 00000002 REASON CODE 00000825 (00002085) : indicates that the message queue name is not known. Check the “Local address” field in the definition of the MQ line, and the queue name prefix in the MQn parameter of the VIRTCT.
- REQ MQCONN COMPLETION CODE 00000002 REASON CODE 00000851 (00002129) : indicates that the required MQSeries libraries are not present in the STEPLIB concatenation of the VIRTEL started task JCL. Refer to “Executing VIRTEL in an MVS environment” in the VIRTEL Installation Guide.

\end{description}


\subsubsection{VIRQ924E linename OBJECT socknum REQ reqtype COMPLETION CODE ccc REASON CODE xx (ddd) MQM mqmname \textbar{} linename PARAM queuename}
\label{\detokenize{messages:virq924e-linename-object-socknum-req-reqtype-completion-code-ccc-reason-code-xx-ddd-mqm-mqmname-linename-param-queuename}}\begin{description}
\item[{Module}] \leavevmode
VIR0Q09

\item[{Meaning}] \leavevmode
A message queue manager request of type reqtype for the line whose external name is linename was not successful. mqmname is the name of the message queue manager, ccc is the message queue manager completion code, and xx and ddd are respectively the hexadecimal and decimal reason codes. queuename is the name of the message queue, if any, associated with the error.

\item[{Action}] \leavevmode
For an explanation of the error codes, refer to Appendix A of the IBM manual MQSeries Application Programming Reference.

\end{description}


\subsubsection{VIRQ925W linename MQM mqmname IS QUIESCING}
\label{\detokenize{messages:virq925w-linename-mqm-mqmname-is-quiescing}}\begin{description}
\item[{Module}] \leavevmode
VIR0Q09

\item[{Meaning}] \leavevmode
The message queue manager mqmname is terminating. linename is the external name of the associated VIRTEL line. MQ services will not be available until the queue manager is restarted.

\item[{Action}] \leavevmode
Restart the queue manager.

\end{description}


\subsection{Messages VIRR2xxx}
\label{\detokenize{messages:messages-virr2xxx}}

\subsubsection{VIRR211E termid INVALID SERVER CALLED: ‘servid’ FROM TRANSACTION tranid BY USER userid}
\label{\detokenize{messages:virr211e-termid-invalid-server-called-servid-from-transaction-tranid-by-user-userid}}\begin{description}
\item[{Module}] \leavevmode
VIR0021R

\item[{Meaning}] \leavevmode
Transaction tranid either specifies a non-existent external server servid in its “application” field, or the “application type” field in the transaction definition is not valid.

\item[{Action}] \leavevmode
Correct the definition of transaction tranid.

\end{description}


\subsection{Messages VIRRWxxx}
\label{\detokenize{messages:messages-virrwxxx}}

\subsubsection{VIRRW01I INITIALISATION FOR linename (n-xxxxxx), VERSION x.xx}
\label{\detokenize{messages:virrw01i-initialisation-for-linename-n-xxxxxx-version-x-xx}}\begin{description}
\item[{Module}] \leavevmode
VIR0RAW

\item[{Meaning}] \leavevmode
The line with external name linename and internal name n-xxxxxx has been initialised with the RAW protocol.

\item[{Action}] \leavevmode
None.

\end{description}


\subsubsection{VIRRW04I LINE linename WAS STOPPED}
\label{\detokenize{messages:virrw04i-line-linename-was-stopped}}\begin{description}
\item[{Module}] \leavevmode
VIR0RAW

\item[{Meaning}] \leavevmode
The line with external name linename was stopped, either by console command, or by setting the “Possible calls” parameter to 0 in the line definition.

\item[{Action}] \leavevmode
None.

\end{description}


\subsubsection{VIRRW27E linename IS REJECTED AT ENTRY POINT epname}
\label{\detokenize{messages:virrw27e-linename-is-rejected-at-entry-point-epname}}\begin{description}
\item[{Module}] \leavevmode
VIR0RAW

\item[{Meaning}] \leavevmode
VIRTEL received an incoming call on the line whose external name is linename, but the entry point epname specified on the line definition does not exist.

\item[{Action}] \leavevmode
Check the definition of the line.

\end{description}


\subsubsection{VIRRW28I linename CALLER ipaddr:port GETS ENTRY POINT ‘epname’ FROM RULE ‘rulename’}
\label{\detokenize{messages:virrw28i-linename-caller-ipaddr-port-gets-entry-point-epname-from-rule-rulename}}\begin{description}
\item[{Module}] \leavevmode
VIR0RAW

\item[{Meaning}] \leavevmode
VIRTEL selected entry point epname for an incoming call on the line whose external name is linename, because the call matched the rule rulename.

\item[{Action}] \leavevmode
None.

\end{description}


\subsubsection{VIRRW29I linename HAS NO MORE PSEUDO TERMINALS}
\label{\detokenize{messages:virrw29i-linename-has-no-more-pseudo-terminals}}\begin{description}
\item[{Module}] \leavevmode
VIR0RAW

\item[{Meaning}] \leavevmode
There are no more terminals associated with the line whose external name is linename.

\item[{Action}] \leavevmode
Check the definition of the line. Add more terminals if necessary.

\end{description}


\subsubsection{VIRRW30I linename REQUEST FROM termname SENT AT timestamp1 CORR=correlator}
\label{\detokenize{messages:virrw30i-linename-request-from-termname-sent-at-timestamp1-corr-correlator}}\begin{description}
\item[{Module}] \leavevmode
VIR0RAW

\item[{Meaning}] \leavevmode
A request has been sent from terminal termname on the line whose external name is linename. The time of the request is timestamp1 and the correlator is correlator.

\item[{Action}] \leavevmode
None.

\end{description}


\subsubsection{VIRRW30I linename REPLY timestamp1 AT timestamp2 CORR=correlator Q=qname}
\label{\detokenize{messages:virrw30i-linename-reply-timestamp1-at-timestamp2-corr-correlator-q-qname}}\begin{description}
\item[{Module}] \leavevmode
VIR0RAW

\item[{Meaning}] \leavevmode
A response to a message sent by terminal termname on the line whose external name is linename has been received. The time of the request is timestamp1. The time of the response is timestamp2. The correlator is correlator. The queue name is qname.

\item[{Action}] \leavevmode
None.

\end{description}


\subsubsection{VIRRW31I linename REQUEST timestamp1 CORR=correlator TIMEOUT Q=qname}
\label{\detokenize{messages:virrw31i-linename-request-timestamp1-corr-correlator-timeout-q-qname}}\begin{description}
\item[{Module}] \leavevmode
VIR0RAW

\item[{Meaning}] \leavevmode
VIRTEL has not received a response to a message sent by terminal termname on the line whose external name is linename. The time of the request is timestamp1. The correlator is correlator. The queue name is qname.

\item[{Action}] \leavevmode
Check the message queue manager log to determine why no response was received.

\end{description}


\subsubsection{VIRRW51I linename CONNECTING termname}
\label{\detokenize{messages:virrw51i-linename-connecting-termname}}\begin{description}
\item[{Module}] \leavevmode
VIR0RAW

\item[{Meaning}] \leavevmode
An incoming call on the line whose external name is linename has been allocated to terminal termname.

\item[{Action}] \leavevmode
None.

\end{description}


\subsubsection{VIRRW52I linename DECONNECTING termname}
\label{\detokenize{messages:virrw52i-linename-deconnecting-termname}}\begin{description}
\item[{Module}] \leavevmode
VIR0RAW

\item[{Meaning}] \leavevmode
Terminal termname has been disconnected from the line whose external name is linename.

\item[{Action}] \leavevmode
None.

\end{description}


\subsubsection{VIRRW54E INVALID REQUEST ON linename ENTRY POINT ‘epname’ DIRECTORY ‘dirname’}
\label{\detokenize{messages:virrw54e-invalid-request-on-linename-entry-point-epname-directory-dirname}}\begin{description}
\item[{Module}] \leavevmode
VIR0RAW

\item[{Meaning}] \leavevmode
VIRTEL received a request on the line whose external name is linename. The requested directory path dirname does not exist under the entry point epname.

\item[{Action}] \leavevmode
Check the definition of the entry point epname. Ensure that there is a type 4 transaction with external name dirname.

\end{description}


\subsubsection{VIRRW55E INVALID RULE rulename ENTRY POINT ‘epname’}
\label{\detokenize{messages:virrw55e-invalid-rule-rulename-entry-point-epname}}\begin{description}
\item[{Module}] \leavevmode
VIR0RAW

\item[{Meaning}] \leavevmode
Rule rulename attached to an MQ line specifies an unknown entry point epname.

\item[{Action}] \leavevmode
Check the entry point name specified in the rule definition.

\end{description}


\subsubsection{VIRRW99I linename MESSAGE RECEIVED FROM SOCKET socknum}
\label{\detokenize{messages:virrw99i-linename-message-received-from-socket-socknum}}\begin{description}
\item[{Module}] \leavevmode
VIR0RAW

\item[{Meaning}] \leavevmode
Input data has been received by the line whose external name is linename.

\item[{Action}] \leavevmode
None.

\end{description}


\subsubsection{VIRRW99I linename SEND TO SOCKET socknum}
\label{\detokenize{messages:virrw99i-linename-send-to-socket-socknum}}\begin{description}
\item[{Module}] \leavevmode
VIR0RAW

\item[{Meaning}] \leavevmode
Output data has been sent to the line whose external name is linename.

\item[{Action}] \leavevmode
None.

\end{description}


\subsection{Messages VIRS1xxx}
\label{\detokenize{messages:messages-virs1xxx}}

\subsubsection{VIRS121W termid ERROR LOADING SCENARIO modname IN TRANSACTION tranid}
\label{\detokenize{messages:virs121w-termid-error-loading-scenario-modname-in-transaction-tranid}}\begin{description}
\item[{Module}] \leavevmode
VIR0S12

\item[{Meaning}] \leavevmode
Transaction tranid at terminal termid called for a presentation module modname which could not be loaded.

\item[{Action}] \leavevmode
Check the log for a previous message VIR0031W. Check the module name specified in the “Initial Scenario”, “Final Scenario”, “Input Scenario” or “Output Scenario” field of transaction tranid. Ensure that this module exists in the VIRTEL load library and that it contains a valid scenario of the requested type. Recompile the module using the current version of the VIRTEL SCRNAPI macro library.

\end{description}


\subsubsection{VIRS122W termid INVALID FA39 SCENARIO RECEIVED}
\label{\detokenize{messages:virs122w-termid-invalid-fa39-scenario-received}}\begin{description}
\item[{Module}] \leavevmode
VIR0S12

\item[{Meaning}] \leavevmode
The transaction at terminal termid called for a presentation module which was not valid.

\item[{Action}] \leavevmode
Check the module name specified in the “Initial Scenario”, “Final Scenario”, “Input Scenario” or “Output Scenario” field of the transaction. Ensure that this module contains a valid scenario of the requested type. Recompile the module using the current version of the VIRTEL SCRNAPI macro library.

\end{description}


\subsubsection{VIRS123W termid ERROR LOADING SCENARIO modname IN ENTRY POINT epname}
\label{\detokenize{messages:virs123w-termid-error-loading-scenario-modname-in-entry-point-epname}}\begin{description}
\item[{Module}] \leavevmode
VIR0S12

\item[{Meaning}] \leavevmode
Entry point epname invoked from terminal termid called for a presentation module modname which could not be loaded.

\item[{Action}] \leavevmode
Check the log for a previous message VIR0031W. Check the module name specified in the “Identification Scenario” field of entry point epname. Ensure that this module exists in the VIRTEL load library and that it contains a valid scenario of the requested type. Recompile the module using the current version of the VIRTEL SCRNAPI macro library.

\end{description}


\subsubsection{VIRS124E termid VIRSV CREATE ERROR \textendash{} R15 : xxxxxxxx (dddddddd)}
\label{\detokenize{messages:virs124e-termid-virsv-create-error-r15-xxxxxxxx-dddddddd}}\begin{description}
\item[{Module}] \leavevmode
VIR0S12

\item[{Meaning}] \leavevmode
A scenario invoked from terminal termid was unable to call a service program using the VIRSV\$ macro instruction. xxxxxxxx is the hexadecimal return code from program VSVPCREA, and dddddddd is the decimal equivalent of the low-order byte of the return code.

\item[{Action}] \leavevmode
Refer to the VIRSV User’s Guide manual to determine the meaning of the VSVPCREA return code. Check the VIRTEL log and the VSVTRACE file for additional messages. Check the VIRTEL started task JCL to ensure that the VIRSV load library is referenced in both the STEPLIB and the SERVLIB concatenations. If VIRTEL runs as an APF-authorized task, check that the VIRSV load library is APF-authorized.

\end{description}


\subsubsection{VIRS125E termid VIRSV REQUEST ERROR \textendash{} R15 : xxxxxxxx (dddddddd)}
\label{\detokenize{messages:virs125e-termid-virsv-request-error-r15-xxxxxxxx-dddddddd}}\begin{description}
\item[{Module}] \leavevmode
VIR0S12

\item[{Meaning}] \leavevmode
A scenario invoked from terminal termid was unable to call a service program using the VIRSV\$ macro instruction. xxxxxxxx is the hexadecimal return code from program VSVPREQS, and dddddddd is the decimal equivalent of the low-order byte of the return code.

\item[{Action}] \leavevmode
Refer to the VIRSV User’s Guide manual to determine the meaning of the VSVPREQS return code. Check the VIRTEL log and the VSVTRACE file for additional messages. Check that the requested service program exists in a load library in the SERVLIB concatenation of the VIRTEL started task JCL. If VIRTEL runs as an APF-authorized task, check that the load library containing the service program is APF-authorized.

\end{description}


\subsubsection{VIRS126E termid VIRSV APPLICATION ERROR CODE : xxxxxxxx (dddddddd)}
\label{\detokenize{messages:virs126e-termid-virsv-application-error-code-xxxxxxxx-dddddddd}}\begin{description}
\item[{Module}] \leavevmode
VIR0S12

\item[{Meaning}] \leavevmode
A scenario invoked from terminal termid called a service program using the VIRSV\$ macro instruction. The service program completed with a non-zero return code. xxxxxxxx is the hexadecimal return code from the service program, and dddddddd is the decimal equivalent.

\item[{Action}] \leavevmode
Determine the meaning of the service program return code and take appropriate action.

\end{description}


\subsubsection{VIRS127E termid VIRSV SERVICES UNAVAILABLE}
\label{\detokenize{messages:virs127e-termid-virsv-services-unavailable}}\begin{description}
\item[{Module}] \leavevmode
VIR0S12

\item[{Meaning}] \leavevmode
A scenario invoked from terminal termid was unable to call a service program using the VIRSV\$ macro instruction because the VIRSV service call facility is not available to VIRTEL.

\item[{Action}] \leavevmode
Check that the VIRTCT contains a VIRSV1 parameter. If so, check the VIRTEL log to determine why the VIRSV facility failed to initialise at VIRTEL startup. Refer to message VIR0090E.

\end{description}


\subsubsection{VIRS129E termid ERROR: SCENARIO scenname ABENDED AT OFFSET ‘xxxxxxxx’ ON INSTRUCTION STARTING WITH ‘yyyyyyyy’ {[}INPUT\textbar{}OUTPUT\textbar{}IDENTIFICATION SCENARIO IS DISCARDED{]}}
\label{\detokenize{messages:virs129e-termid-error-scenario-scenname-abended-at-offset-xxxxxxxx-on-instruction-starting-with-yyyyyyyy-input-output-identification-scenario-is-discarded}}\begin{description}
\item[{Module}] \leavevmode
VIR0S12

\item[{Meaning}] \leavevmode
The scenario scenname invoked from terminal termid contains an instruction which is invalid, or which attempted to perform an invalid operation such as copying into a protected field on the 3270 screen, or a MAP\$ ABEND instruction was executed. xxxxxxxx is the hexadecimal offset within the module of the instruction. yyyyyyyy represents the first four bytes (in hexadecimal) of the instruction. The scenario is abandoned (as if SCENARIO DISCARD had been coded).

\item[{Action}] \leavevmode
Check the assembly listing to verify that the scenario has been correctly assembled. Check that the correct level macro library was used.

\end{description}


\subsubsection{VIRS12AI termid ERROR: SCENARIO scenname WAS ABNORMALLY STOPPED AT OFFSET ‘xxxxxxxx’}
\label{\detokenize{messages:virs12ai-termid-error-scenario-scenname-was-abnormally-stopped-at-offset-xxxxxxxx}}\begin{description}
\item[{Module}] \leavevmode
VIR0S12

\item[{Meaning}] \leavevmode
The scenario scenname invoked from terminal termid was terminated abnormally using the KILL command. xxxxxxxx is the hexadecimal offset within the module.

\item[{Action}] \leavevmode
None.

\end{description}


\subsubsection{VIRS12BE SCENARIO REQUIRES HEADER ‘headername’ NOT FOUND IN TCT : ABORTING}
\label{\detokenize{messages:virs12be-scenario-requires-header-headername-not-found-in-tct-aborting}}\begin{description}
\item[{Module}] \leavevmode
VIR0S12

\item[{Meaning}] \leavevmode
A COPY\$ SYSTEM-TO-VARIABLE instruction in a scenario is attempting to obtain the value of the HTTP header headername but the name was not declared in the VIRTCT. A subsequent VIRS129E instruction shows the name of the scenario and the location of the instruction in error.

\item[{Action}] \leavevmode
Add headername to the HTHEADR parameter of the VIRTCT.

\end{description}


\subsubsection{VIRS12CE termid VIRSV REQUEST ERROR \textendash{} R15 : xxxxxxxx (dddddddd) FROM progname}
\label{\detokenize{messages:virs12ce-termid-virsv-request-error-r15-xxxxxxxx-dddddddd-from-progname}}\begin{description}
\item[{Module}] \leavevmode
VIR0S12

\item[{Meaning}] \leavevmode
A call from a scenario to VIRSV returned an error. xxxxxxxx is the hexadecimal return code from VIRSV module progname and dddddddd is the decimal equivalent.

\item[{Action}] \leavevmode
Refer to the VIRSV documentation for the meaning of the return code.

\end{description}


\subsubsection{VIRS12DE termid VIRSV TRANSACTION ERROR CODE : xxxxxxxx (dddddddd) FROM progname}
\label{\detokenize{messages:virs12de-termid-virsv-transaction-error-code-xxxxxxxx-dddddddd-from-progname}}\begin{description}
\item[{Module}] \leavevmode
VIR0S12

\item[{Meaning}] \leavevmode
A call from a scenario to VIRSV returned an error. xxxxxxxx is the hexadecimal return code from VIRSV module progname and dddddddd is the decimal equivalent.

\item[{Action}] \leavevmode
Refer to the VIRSV documentation for the meaning of the return code.

\end{description}


\subsubsection{VIRS12EE termid SCENARIO scenname RETURNED ERROR : ‘dddddddd’}
\label{\detokenize{messages:virs12ee-termid-scenario-scenname-returned-error-dddddddd}}\begin{description}
\item[{Module}] \leavevmode
VIR0S12

\item[{Meaning}] \leavevmode
The scenario scenname invoked from terminal termid terminated using the ERROR\$ macro instruction. dddddddd is the decimal return code.

\item[{Action}] \leavevmode
Determine the meaning of the scenario return code and take appropriate action.

\end{description}


\subsubsection{VIRS12FE termid ERROR: SCENARIO ‘scenname’ CALLS UNDEFINED TRANSACTION ‘tranname’ AT OFFSET ‘xxxxxxxx’}
\label{\detokenize{messages:virs12fe-termid-error-scenario-scenname-calls-undefined-transaction-tranname-at-offset-xxxxxxxx}}\begin{description}
\item[{Module}] \leavevmode
VIR0S12

\item[{Meaning}] \leavevmode
The scenario scenname invoked from terminal termid issued a SEND\$ TO or SEND\$ VARIABLE-TO instruction specifying an invalid transaction name tranname. xxxxxxxx is the hexadecimal offset of the SEND\$ instruction within the module.

\item[{Action}] \leavevmode
Verify that the entry point contains a type 5 transaction whose external name is tranname, or correct the transaction name in the scenario.

\end{description}


\subsubsection{VIRS12GE CANNOT RESOLVE PASSTICKET OF TYPE ‘applname’ ABORTING SCENARIO}
\label{\detokenize{messages:virs12ge-cannot-resolve-passticket-of-type-applname-aborting-scenario}}\begin{description}
\item[{Module}] \leavevmode
VIR0S12

\item[{Meaning}] \leavevmode
VIRTEL was unable to process an instruction of type PASSTICKET-FOR-TRANSACTION contained in a scenario.

\item[{Action}] \leavevmode
Possible causes are: (1) Not signed on (2) Transaction not found (3) No application name in transaction (4) PASSTCK not specified in VIRTCT (5) Application name not found in PASSTCK table (6) Request is neither GENERATE nor EVALUATE (7) Request specifies GENERATE or EVALUATE but does not match corresponding entry in the PASSTCK table (8) Insufficient storage for workarea (9) Error linking to module IRRSPK00

\end{description}


\subsubsection{VIRS12HE termid ERROR: SCENARIO scenname RACF ticketserv GENERATE\textbar{}EVALUATE SAF RC: ‘safrc’ RACF RC: ‘racfrc’ RACF reason: ‘racfreas’}
\label{\detokenize{messages:virs12he-termid-error-scenario-scenname-racf-ticketserv-generate-evaluate-saf-rc-safrc-racf-rc-racfrc-racf-reason-racfreas}}\begin{description}
\item[{Module}] \leavevmode
VIR0S12

\item[{Meaning}] \leavevmode
The security subsystem returned an error code during processing of an instruction of type PASSTICKET-FOR- TRANSACTION contained in scenario scenname for terminal termid.

\item[{Action}] \leavevmode
For an explanation of the return codes and reason codes, refer to the IBM manual z/OS Security Server RACF Callable Services chapter 2: “R\_ticketserv”.

\end{description}


\subsubsection{VIRS12IE termid LOOP WAS DETECTED IN SCENARIO scenname}
\label{\detokenize{messages:virs12ie-termid-loop-was-detected-in-scenario-scenname}}\begin{description}
\item[{Module}] \leavevmode
VIR0S12

\item[{Meaning}] \leavevmode
A possible loop has been detected in scenario scenname. The scenario is abended.

\item[{Action}] \leavevmode
For an explanation on the way to identify or manage loops in scenario, refer to the HANDLE\$ LOOP instrauction in the VIRTEL Web Access documentation.

\end{description}


\subsection{Messages VIRSMxxx}
\label{\detokenize{messages:messages-virsmxxx}}

\subsubsection{VIRSM01I SMTP INITIALISATION FOR linename (n-xxxxxx), VERSION x.xx}
\label{\detokenize{messages:virsm01i-smtp-initialisation-for-linename-n-xxxxxx-version-x-xx}}\begin{description}
\item[{Module}] \leavevmode
VIRSMTP

\item[{Meaning}] \leavevmode
The SMTP line with external name linename and internal name n-xxxxxx has been initialised.

\item[{Action}] \leavevmode
None.

\end{description}


\subsubsection{VIRSM04I SMTP LINE linename WAS STOPPED}
\label{\detokenize{messages:virsm04i-smtp-line-linename-was-stopped}}\begin{description}
\item[{Module}] \leavevmode
VIRSMTP

\item[{Meaning}] \leavevmode
The SMTP line with external name linename was stopped, either by console command, or by setting the “Possible calls” parameter to 0 in the line definition.

\item[{Action}] \leavevmode
None.

\end{description}


\subsubsection{VIRSM27E linename IS REJECTED AT ENTRY POINT epname}
\label{\detokenize{messages:virsm27e-linename-is-rejected-at-entry-point-epname}}\begin{description}
\item[{Module}] \leavevmode
VIRSMTP

\item[{Meaning}] \leavevmode
VIRTEL received an incoming call on the SMTP line whose external name is linename, but the entry point epname specified on the line definition does not exist.

\item[{Action}] \leavevmode
Check the definition of the SMTP line.

\end{description}


\subsubsection{VIRSM28I linename CALLER ipaddr:port GETS ENTRY POINT ‘epname’ FROM RULE ‘rulename’}
\label{\detokenize{messages:virsm28i-linename-caller-ipaddr-port-gets-entry-point-epname-from-rule-rulename}}\begin{description}
\item[{Module}] \leavevmode
VIRSMTP

\item[{Meaning}] \leavevmode
VIRTEL selected entry point epname for an incoming call on the SMTP line whose external name is linename, because the call matched the rule rulename.

\item[{Action}] \leavevmode
None.

\end{description}


\subsubsection{VIRSM29I linename HAS NO MORE PSEUDO TERMINALS}
\label{\detokenize{messages:virsm29i-linename-has-no-more-pseudo-terminals}}\begin{description}
\item[{Module}] \leavevmode
VIRSMTP

\item[{Meaning}] \leavevmode
There are no more terminals associated with the SMTP line whose external name is linename.

\item[{Action}] \leavevmode
Check the definition of the line. Add more terminals if necessary.

\end{description}


\subsubsection{VIRSM51I linename CONNECTING termname}
\label{\detokenize{messages:virsm51i-linename-connecting-termname}}\begin{description}
\item[{Module}] \leavevmode
VIRSMTP

\item[{Meaning}] \leavevmode
An incoming call on the SMTP line whose external name is linename has been allocated to terminal termname.

\item[{Action}] \leavevmode
None.

\end{description}


\subsubsection{VIRSM52I linename DECONNECTING termname}
\label{\detokenize{messages:virsm52i-linename-deconnecting-termname}}\begin{description}
\item[{Module}] \leavevmode
VIRSMTP

\item[{Meaning}] \leavevmode
Terminal termname has been disconnected from the SMTP line whose external name is linename.

\item[{Action}] \leavevmode
None.

\end{description}


\subsubsection{VIRSM55E INVALID RULE rulename ENTRY POINT ‘epname’}
\label{\detokenize{messages:virsm55e-invalid-rule-rulename-entry-point-epname}}\begin{description}
\item[{Module}] \leavevmode
VIRSMTP

\item[{Meaning}] \leavevmode
Rule rulename attached to an SMTP line specifies an unknown entry point epname.

\item[{Action}] \leavevmode
Check the entry point name specified in the rule definition.

\end{description}


\subsubsection{VIRSM571 MESSAGE FROM xxx@xxx.xxx TO xxx@xxx.xxx WAS REJECTED}
\label{\detokenize{messages:virsm571-message-from-xxx-xxx-xxx-to-xxx-xxx-xxx-was-rejected}}\begin{description}
\item[{Module}] \leavevmode
VIRSMTP

\item[{Meaning}] \leavevmode
VIRTEL received an incomplete message on an SMTP line. The message does not contain the required headers In-Reply-To: (or alternatively References:) and Mime-Version:. The message has been rejected with response 571 Delivery not authorized, message refused sent to the SMTP requester.

\item[{Action}] \leavevmode
To upload an HTML page, you must reply to a message sent by VIRTEL and include an attached file.

\end{description}


\subsection{Messages VIRT2xxx}
\label{\detokenize{messages:messages-virt2xxx}}

\subsubsection{VIRT251E termid INVALID SERVER CALLED: ‘servid’ FROM TRANSACTION tranid BY USER userid}
\label{\detokenize{messages:virt251e-termid-invalid-server-called-servid-from-transaction-tranid-by-user-userid}}\begin{description}
\item[{Module}] \leavevmode
VIR0025T

\item[{Meaning}] \leavevmode
Transaction tranid either specifies a non-existent external server servid in its “application” field, or the “application type” field in the transaction definition is not valid.

\item[{Action}] \leavevmode
Correct the definition of transaction tranid.

\end{description}


\subsection{Messages VIRT9xxx}
\label{\detokenize{messages:messages-virt9xxx}}

\subsubsection{VIRT903W LINE linename HAS A SESSION STARTED WITH TCP/IP tcpname HIGHEST SOCKET IS ‘xxxxxxxx’ (dddddddd)}
\label{\detokenize{messages:virt903w-line-linename-has-a-session-started-with-tcp-ip-tcpname-highest-socket-is-xxxxxxxx-dddddddd}}\begin{description}
\item[{Module}] \leavevmode
VIR0T09

\item[{Meaning}] \leavevmode
The line with external name linename has opened a session with TCP/IP name tcpname. xxxxxxxx is the highest socket number (in hexadecimal) for this line. dddddddd is the highest socket number in decimal. The highest socket number is assigned by the TCP/IP stack and will normally be one less than the maxsock subparameter of the TCPn parameter in the VIRTCT.

\item[{Action}] \leavevmode
None

\end{description}


\subsubsection{VIRT904I LINE linename (n-xxxxxx) IS USING TCP/IP KEEPALIVE}
\label{\detokenize{messages:virt904i-line-linename-n-xxxxxx-is-using-tcp-ip-keepalive}}\begin{description}
\item[{Module}] \leavevmode
VIR0T09

\item[{Meaning}] \leavevmode
VIRTEL has requested the TCP/IP  stack to send “keepalive” messages on the line with external name linename     and internal name n-xxxxxx. You must also activate keepalive in the TCP/IP stack parameter file. See the “Action” parameter of the Line Definition in the VIRTEL Connectivity Reference manual.

\item[{Action}] \leavevmode
None.

\end{description}


\subsubsection{VIRT905I linename SOCKET socknum LISTENING ipaddr:port}
\label{\detokenize{messages:virt905i-linename-socket-socknum-listening-ipaddr-port}}\begin{description}
\item[{Module}] \leavevmode
VIR0T09

\item[{Meaning}] \leavevmode
VIRTEL has opened a TCP/IP listening socket for the line with external name linename. port is the listening port number, and ipaddr is the IP address on which VIRTEL will accept incoming connections. If ipaddr is 000.000.000.000 then VIRTEL accepts connections on any of the host’s home IP addresses.

\item[{Action}] \leavevmode
None.

\end{description}


\subsubsection{VIRT906I linename SOCKET socknum CALL FROM ipaddr:port}
\label{\detokenize{messages:virt906i-linename-socket-socknum-call-from-ipaddr-port}}\begin{description}
\item[{Module}] \leavevmode
VIR0T09

\item[{Meaning}] \leavevmode
VIRTEL has accepted an incoming connection on the line with external name linename. socknum is the TCP/IP socket number, and ipaddr:port is the client’s IP address and port number.

\item[{Action}] \leavevmode
None.

\end{description}


\subsubsection{VIRT907I linename SOCKET socknum CALLING ipaddr:port}
\label{\detokenize{messages:virt907i-linename-socket-socknum-calling-ipaddr-port}}\begin{description}
\item[{Module}] \leavevmode
VIR0T09

\item[{Meaning}] \leavevmode
VIRTEL is making an outgoing connection on the line with external name linename. socknum is the TCP/IP socket number, and ipaddr:port is the destination IP address and port number.

\item[{Action}] \leavevmode
None.

\end{description}


\subsubsection{VIRT912W linename SOCKET socknum STARTED FOR ipaddr:port}
\label{\detokenize{messages:virt912w-linename-socket-socknum-started-for-ipaddr-port}}\begin{description}
\item[{Module}] \leavevmode
VIR0T09

\item[{Meaning}] \leavevmode
VIRTEL has opened TCP/IP socket socknum on the line with external name linename. If socknum is 00000000 then this is a listening socket, port is the listening port number, and ipaddr is the IP address on which VIRTEL accepts incoming connections (000.000.000.000 means that VIRTEL accepts connections on any of the host’s home IP addresses). If socknum is non-zero and port is non-zero, then this is an incoming connection, and ipaddr:port is the client’s IP address and port number. If socknum is non-zero and port is 00000, then this is an outgoing connection, and ipaddr is VIRTEL’s  IP address (000.000.000.000 means that VIRTEL is using the host’s default home address).  For VIRTEL 4.33 onwards, this message is replaced by messages VIRT905I, VIRT906I and VIRT907I respectively. This message can be suppressed by the SILENCE parameter in the VIRTCT.

\item[{Action}] \leavevmode
None.

\end{description}


\subsubsection{VIRT918S NO MORE OSCORE AVAILABLE}
\label{\detokenize{messages:virt918s-no-more-oscore-available}}\begin{description}
\item[{Module}] \leavevmode
VIR0T09

\item[{Meaning}] \leavevmode
VIRTEL has insufficient memory to start a new TCP/IP task.

\item[{Action}] \leavevmode
Check the OSCORE parameter in the VIRTCT.

\end{description}


\subsubsection{VIRT920W TCP SESSION LOST FOR LU linename}
\label{\detokenize{messages:virt920w-tcp-session-lost-for-lu-linename}}\begin{description}
\item[{Module}] \leavevmode
VIR0T09

\item[{Meaning}] \leavevmode
Loss of TCP/IP session for the line with external name linename.

\item[{Action}] \leavevmode
Check the partner system to determine the cause of the error.

\end{description}


\subsubsection{VIRT922W linename SOCKET socknum ENDED FOR ipaddr:port}
\label{\detokenize{messages:virt922w-linename-socket-socknum-ended-for-ipaddr-port}}\begin{description}
\item[{Module}] \leavevmode
VIR0T09

\item[{Meaning}] \leavevmode
VIRTEL has closed the TCP/IP socket socknum on the line with external name linename on the indicated IP address ipaddr and the indicated port number port. This message can be suppressed by the SILENCE parameter in the VIRTCT.

\item[{Action}] \leavevmode
None.

\end{description}


\subsubsection{VIRT923E ERROR ON: linename REQ: reqtype - RETCODE: FFFFFFFF ERRNO: xx (ddd)}
\label{\detokenize{messages:virt923e-error-on-linename-req-reqtype-retcode-ffffffff-errno-xx-ddd}}\begin{description}
\item[{Module}] \leavevmode
VIR0T09

\item[{Meaning}] \leavevmode
A request of type reqtype in TCP/IP mode for the line indicated by the external name linename was not successful. xx and ddd are respectively the hexadecimal and decimal error codes.

\item[{Action}] \leavevmode
For an explanation of the error codes, refer to Appendix B of the IBM manual z/OS Communications Server: IP Sockets Application Programming Interface Guide and Reference. For VSE, refer to “ERRNO values” in chapter 8 of the IBM manual TCP/IP for VSE/ESA IBM Program Setup and Supplementary Information.
The following codes are commonly encountered:
- REQ: SOCKET - RETCODE: FFFFFFFF ERRNO:00000018 (00000024) : indicates that the value of the MAXFILEPROC parameter in PARMLIB member BPXPRMxx is insufficient for the number of TCP/IP connections. Use the console command D OMVS,O to determine the current value of MAXFILEPROC. The command SETOMVS MAXFILEPROC=nnnn can be used to increase the value temporarily until the next IPL.
- REQ: SOCKET - RETCODE: FFFFFFFF ERRNO: 0000045D (00001117) : indicates that TCP/IP for VSE is not started.
- REQ: GTHOSTID - RETCODE: FFFFFFFF ERRNO: 00000000 (00000000) : may indicate that TCP/IP for VSE is not started.
- REQ: GTHOSTID - RETCODE: FFFFFFFF ERRNO: 0000009C (00000156) : may indicate that the USER assigned to the VIRTEL started task does not have the necessary authorization (RACF, TOP SECRET) to access TCP/IP facilities via MVS OPEN-EDITION.
- REQ: ACCEPT - RETCODE: FFFFFFFF ERRNO: 000003EA (00001002) : indicates that the maxsock subparameter of the TCPn parameter in the VIRTCT is insufficient for the number of incoming calls received.
- REQ: INITAPI - RETCODE: FFFFFFFF ERRNO: 000003F3 (00001011) : indicates that the name specified in the TCPn parameter in the VIRTCT does not match the name of the z/OS TCP/IP stack.
- REQ: INITAPI - RETCODE: FFFFFFFF ERRNO: 000027D6 (00010198) : indicates that the maxsock subparameter of the TCPn parameter in the VIRTCT exceeds the maximum allowed by the TCP/IP stack.

\end{description}


\subsubsection{VIRT924E ERROR ON: linename SOCKET: socknum REQ: reqtype - RETCODE: FFFFFFFF ERRNO: xx (ddd)}
\label{\detokenize{messages:virt924e-error-on-linename-socket-socknum-req-reqtype-retcode-ffffffff-errno-xx-ddd}}\begin{description}
\item[{Module}] \leavevmode
VIR0T09

\item[{Meaning}] \leavevmode
Socket socknum has not been successful in establishing a TCP/IP  session. reqtype is the TCP/IP  API call which  failed, linename is the external name of the line, and xx and ddd are respectively the hexadecimal and decimal representations of the TCP/IP error code. This message can appear several times, as VIRTEL retries periodically to obtain the connection.

\item[{Action}] \leavevmode
For an explanation of the error codes, refer to Appendix B of the IBM manual z/OS Communications Server: IP Sockets Application Programming Interface Guide and Reference. For VSE, refer to “ERRNO values” in chapter 8 of the IBM manual TCP/IP for VSE/ESA IBM Program Setup and Supplementary Information.
The following codes are commonly encountered:
- REQ :BIND - RETCODE: FFFFFFFF ERRNO: 00000030 (00000048) : indicates that the field “Local ident” in the line definition specified a port number which is already in use by another application or by another VIRTEL line.
- REQ :BIND - RETCODE: FFFFFFFF ERRNO: 00000031 (00000049) : indicates the the field “Local ident” in the line definition specifies an IP address which is not accepted by the TCP/IP stack as a valid home address.

\end{description}


\subsubsection{VIRT925E ERROR CALLING PORT port AT SERVER servname}
\label{\detokenize{messages:virt925e-error-calling-port-port-at-server-servname}}\begin{description}
\item[{Module}] \leavevmode
VIR0T09

\item[{Meaning}] \leavevmode
An outbound HTTP call to port port at server servname has failed.

\item[{Action}] \leavevmode
Check preceding messages in the log. Check whether a firewall has blocked access to the specified port.

\end{description}


\subsection{Messages VIRU1xxx}
\label{\detokenize{messages:messages-viru1xxx}}

\subsubsection{VIRU121E termid FILE UPLOAD FAILED : ENTRY POINT ‘epname’ DIRECTORY ‘dirname’ FILE ‘filename’ USER ‘username’}
\label{\detokenize{messages:viru121e-termid-file-upload-failed-entry-point-epname-directory-dirname-file-filename-user-username}}\begin{description}
\item[{Module}] \leavevmode
VIR0U12

\item[{Meaning}] \leavevmode
A request by user username to upload the file filename into the directory dirname at VIRTEL entry point epname failed.

\item[{Action}] \leavevmode
Check, if necessary, that the user has successfully installed a cookie (see “Uploading pages via HTML secured by cookie” in the VIRTEL Web Access Guide). Check the definition of entry point epname. Check (using F6 in the VIRTEL configuration menu) that the directory has permissions “Copy to”, “Copy from”, and “Delete” all marked with “X”. Check that the DD name specified in the directory definition is included in the VIRTEL started task JCL.

\end{description}


\subsubsection{VIRU122I termid FILE UPLOAD : ENTRY POINT ‘epname’ DIRECTORY ‘dirname’ FILE ‘filename’ USER ‘username’}
\label{\detokenize{messages:viru122i-termid-file-upload-entry-point-epname-directory-dirname-file-filename-user-username}}\begin{description}
\item[{Module}] \leavevmode
VIR0U12

\item[{Meaning}] \leavevmode
A request by user username to upload the file filename into the directory dirname at VIRTEL entry point epname was successful.

\item[{Action}] \leavevmode
None.

\end{description}


\subsection{Messages VIRUCSxx}
\label{\detokenize{messages:messages-virucsxx}}

\subsubsection{VIRUCS1E termid COULD NOT FIND TRANSLATION TABLE tablename}
\label{\detokenize{messages:virucs1e-termid-could-not-find-translation-table-tablename}}\begin{description}
\item[{Module}] \leavevmode
VIR0UCS

\item[{Meaning}] \leavevmode
Translation table tablename requested by terminal termid was not found.

\item[{Action}] \leavevmode
Check that the CODEPAGE parameter of the URL contains a valid translation table name, and that the table has been loaded by VIRTEL at startup, either as standard, or by the CHARSET parameter of the VIRTCT (see “Parameters of the VIRTCT” in the VIRTEL Installation Guide).

\end{description}


\subsection{Messages VIRV1xxx}
\label{\detokenize{messages:messages-virv1xxx}}

\subsubsection{VIRV121E termid COULD NOT FIND TRANSLATION TABLE ‘tablname’ FOR PAGE ‘pagename’ DIRECTORY ‘tranid’ (dirname)}
\label{\detokenize{messages:virv121e-termid-could-not-find-translation-table-tablname-for-page-pagename-directory-tranid-dirname}}\begin{description}
\item[{Module}] \leavevmode
VIR0V12

\item[{Meaning}] \leavevmode
The HTML page pagename in the directory referenced by the transaction with external name tranid and internal name dirname specifies an unknown translation table tablname.

\item[{Action}] \leavevmode
Check that the table tablname specified in the SET-OUTPUT-ENCODING-UTF-8 instruction in the HTML page was loaded into VIRTEL at startup time, either as standard, or by being specified in the CHARSET parameter of the VIRTCT (see the VIRTEL Installation Guide manual).

\end{description}


\subsubsection{VIRV122E termid WAS STOPPED}
\label{\detokenize{messages:virv122e-termid-was-stopped}}\begin{description}
\item[{Module}] \leavevmode
VIR0V12

\item[{Meaning}] \leavevmode
As a result of a KILL command, the scenario which was active on terminal termid has been abnormally terminated.

\item[{Action}] \leavevmode
None.

\end{description}


\subsection{Messages VIRX9xxx}
\label{\detokenize{messages:messages-virx9xxx}}

\subsubsection{VIRX903W linename STARTING A VIRPASS SERVICE OF LOCAL NAME xmident REMOTE NAME relayname IN JOB jobname}
\label{\detokenize{messages:virx903w-linename-starting-a-virpass-service-of-local-name-xmident-remote-name-relayname-in-job-jobname}}\begin{description}
\item[{Module}] \leavevmode
VIR0X09

\item[{Meaning}] \leavevmode
VIRTEL is starting a VIRPASS XM (cross-memory) connection to VIRKIX running in a CICS region in the same MVS image. linename is the name of the XM line, jobname is the job name of the CICS region, relayname is the name of the VIRPASS XM relay in VIRKIX, and xmident is the link name defined in the VIRPASS XM relay.

\item[{Action}] \leavevmode
None.

\end{description}


\subsubsection{VIRX907I linename DETACHING VXMTRECV SUBTASK}
\label{\detokenize{messages:virx907i-linename-detaching-vxmtrecv-subtask}}\begin{description}
\item[{Module}] \leavevmode
VIR0X09

\item[{Meaning}] \leavevmode
VIRTEL is terminating a VIRPASS XM (cross-memory) connection as a result of a TERMAPI request. linename is the name of the XM line.

\item[{Action}] \leavevmode
None.

\end{description}


\subsubsection{VIRX908I linename DETACHING VXMTRECV SUBTASK}
\label{\detokenize{messages:virx908i-linename-detaching-vxmtrecv-subtask}}\begin{description}
\item[{Module}] \leavevmode
VIR0X09

\item[{Meaning}] \leavevmode
VIRTEL is terminating a VIRPASS XM (cross-memory) connection as a result of a CLOSE request. linename is the name of the XM line.

\item[{Action}] \leavevmode
None.

\end{description}


\subsubsection{VIRX909I linename DETACHING VXMTRECV SUBTASK}
\label{\detokenize{messages:virx909i-linename-detaching-vxmtrecv-subtask}}\begin{description}
\item[{Module}] \leavevmode
VIR0X09

\item[{Meaning}] \leavevmode
VIRTEL is terminating a VIRPASS XM (cross-memory) connection as a result of a CANCEL request. linename is the name of the XM line.

\item[{Action}] \leavevmode
None.

\end{description}


\subsubsection{VIRX912W linename CONTEXT: sessionid STARTED}
\label{\detokenize{messages:virx912w-linename-context-sessionid-started}}\begin{description}
\item[{Module}] \leavevmode
VIR0X09

\item[{Meaning}] \leavevmode
A session has started on cross-memory line linename. The session identifier is sessionid.

\item[{Action}] \leavevmode
None.

\end{description}


\subsubsection{VIRX922W linename ENDING CONTEXT sessionid FOR jobname}
\label{\detokenize{messages:virx922w-linename-ending-context-sessionid-for-jobname}}\begin{description}
\item[{Module}] \leavevmode
VIR0X09

\item[{Meaning}] \leavevmode
A session has ended on cross-memory line linename. The session identifier is sessionid. The job name of the session partner is jobname.

\item[{Action}] \leavevmode
None.

\end{description}


\subsubsection{VIRX923E ERROR ON linename REQ reqtype R15 r15code RETCODE retcode ERRNO xxxxxxxx (dddddddd)}
\label{\detokenize{messages:virx923e-error-on-linename-req-reqtype-r15-r15code-retcode-retcode-errno-xxxxxxxx-dddddddd}}\begin{description}
\item[{Module}] \leavevmode
VIR0X09

\item[{Meaning}] \leavevmode
An error has occurred on cross-memory line linename. reqtype is the VIRXM request type, r15code and retcode are hexadecimal return codes, xxxxxxxx is the hexadecimal error code, and dddddddd is the decimal value of the low- order byte of the error code.

\item[{Action}] \leavevmode
Refer to the VIRXM User Guide manual to determine the meaning of the return code and error code for the specified request type.

\end{description}


\subsubsection{VIRX924E ERROR ON linename CONTEXT sessionid REQ reqtype RETCODE retcode ERRNO xxxxxxxx (dddddddd)}
\label{\detokenize{messages:virx924e-error-on-linename-context-sessionid-req-reqtype-retcode-retcode-errno-xxxxxxxx-dddddddd}}\begin{description}
\item[{Module}] \leavevmode
VIR0X09

\item[{Meaning}] \leavevmode
An error has occurred on cross-memory line linename. with session identifier sessionid. reqtype is the VIRXM request type, retcode is the decimal return code, xxxxxxxx is the hexadecimal error code, and dddddddd is the decimal value of the low-order byte of the error code.

\item[{Action}] \leavevmode
Refer to the VIRXM User Guide manual to determine the meaning of the return code and error code for the specified request type.

\end{description}


\subsection{Messages VIRXOxxx}
\label{\detokenize{messages:messages-virxoxxx}}

\subsubsection{VIRXO01I XOT INITIALISATION FOR linename (n-xxxxxx), VERSION 4.xx}
\label{\detokenize{messages:virxo01i-xot-initialisation-for-linename-n-xxxxxx-version-4-xx}}\begin{description}
\item[{Module}] \leavevmode
VIRXOT

\item[{Meaning}] \leavevmode
The XOT line with external name linename and internal name n-xxxxxx has been initialised.

\item[{Action}] \leavevmode
None.

\end{description}


\subsubsection{VIRXO04I XOT LINE linename WAS STOPPED}
\label{\detokenize{messages:virxo04i-xot-line-linename-was-stopped}}\begin{description}
\item[{Module}] \leavevmode
VIRXOT

\item[{Meaning}] \leavevmode
The XOT line with external name linename was stopped, either by console command, or by setting the “Possible calls” parameter to 0 in the line definition.

\item[{Action}] \leavevmode
None.

\end{description}


\subsubsection{VIRXO05E linename UNDEFINED PARTNER n.n.n.n:p IS CALLING m.m.m.m:q}
\label{\detokenize{messages:virxo05e-linename-undefined-partner-n-n-n-n-p-is-calling-m-m-m-m-q}}\begin{description}
\item[{Module}] \leavevmode
VIRXOT

\item[{Meaning}] \leavevmode
VIRTEL has rejected an incoming call from an XOT line. n.n.n.n:p represents the IP address and port of the caller (usually an XOT router), m.m.m.m:q represents the called IP address and port in VIRTEL. VIRTEL rejected the call because none of the XOT lines listening on address/port m.m.m.m:q specifies address n.n.n.n as partner.

\item[{Action}] \leavevmode
Define an XOT line specifying m.m.m.m:q in the “Local ident” field and n.n.n.n:1998 in the “Remote ident” field. If necessary, use the xot-source parameter of the x25 route instruction in the router to ensure that the router presents the expected address to VIRTEL in its incoming calls. For additional information, refer to the section “Defining an XOT line” in the VIRTEL Connectivity Reference manual.

\end{description}


\subsubsection{VIRXO06I linename (n-xxxxxx) IS USING A GENERIC ADDRESS}
\label{\detokenize{messages:virxo06i-linename-n-xxxxxx-is-using-a-generic-address}}\begin{description}
\item[{Module}] \leavevmode
VIRXOT

\item[{Meaning}] \leavevmode
The IP address and port number of the line with external name linename and internal name n-xxxxxx is treated as sharable with other TCP/IP lines as a result of the SHAREDA=Y parameter specified in the definition of line linename.

\item[{Action}] \leavevmode
None.

\end{description}


\subsubsection{VIRXO07I linename (n-xxxxxx) IS USING A GENERIC ADDRESS}
\label{\detokenize{messages:virxo07i-linename-n-xxxxxx-is-using-a-generic-address}}\begin{description}
\item[{Module}] \leavevmode
VIRXOT

\item[{Meaning}] \leavevmode
The IP address and port number of the line with external name linename and internal name n-xxxxxx is shared with another TCP/IP line. VIRTEL found the same address and port number specified in the definition of another line. Neither SHAREDA nor UNIQUEP were specified in the definition of line linename.

\item[{Action}] \leavevmode
None.

\end{description}


\subsubsection{VIRXO11I termid CLEAR RECEIVED FROM REMOTE linename CAUSE=xx DIAG=yy}
\label{\detokenize{messages:virxo11i-termid-clear-received-from-remote-linename-cause-xx-diag-yy}}\begin{description}
\item[{Module}] \leavevmode
VIRXOT

\item[{Meaning}] \leavevmode
VIRTEL received an X25 CLEAR command from the router attached to the XOT line with external name linename. VIRTEL determined that the command was associated with CVC terminal termid. xx and yy are the X25 cause and diagnostic codes in hexadecimal. This message is suppressed if the cause and diagnostic codes are zero and the SILENCE parameter is specified in the VIRTCT.

\item[{Action}] \leavevmode
If the cause and diagnostic codes are zero, this indicates a normal end of call from the remote partner. Otherwise, use the cause and diagnostic codes to determine why the router or the remote partner application issued the CLEAR.

\end{description}


\subsubsection{VIRXO51I termid CALLING x25callednumber VIA ROUTER ipaddr:port}
\label{\detokenize{messages:virxo51i-termid-calling-x25callednumber-via-router-ipaddr-port}}\begin{description}
\item[{Module}] \leavevmode
VIRXOT

\item[{Meaning}] \leavevmode
VIRTEL is placing an outbound call to X25 number x25callednumber via the router whose IP address and port number is ipaddr:port. termid is the name of the VIRTEL terminal which represents the virtual circuit.

\item[{Action}] \leavevmode
None.

\end{description}


\subsubsection{VIRXO52I linename DECONNECTING termid}
\label{\detokenize{messages:virxo52i-linename-deconnecting-termid}}\begin{description}
\item[{Module}] \leavevmode
VIRXOT

\item[{Meaning}] \leavevmode
Terminal termid has been disconnected from the XOT line whose external name is linename.

\item[{Action}] \leavevmode
None.

\end{description}


\subsubsection{VIRXO53E linename HAS NO MORE PSEUDO TERMINALS}
\label{\detokenize{messages:virxo53e-linename-has-no-more-pseudo-terminals}}\begin{description}
\item[{Module}] \leavevmode
VIRXOT

\item[{Meaning}] \leavevmode
There are no more terminals associated with the XOT line whose external name is linename.

\item[{Action}] \leavevmode
Check the line definition. Add more terminals if necessary.

\end{description}


\subsubsection{VIRXO98E termid ERROR errcode RECEIVING FROM SOCKET sockno}
\label{\detokenize{messages:virxo98e-termid-error-errcode-receiving-from-socket-sockno}}\begin{description}
\item[{Module}] \leavevmode
VIRXOT

\item[{Meaning}] \leavevmode
An error occurred when receiving a message from the router on an XOT line. termid is the name of the virtual circuit, if available, otherwise it is the external name of the XOT line. errcode is an error code whose meaning is:
\begin{itemize}
\item {} 
000D0001 : No session exists for this line.

\item {} 
000D0002 : Unable to identify incoming message with an existing session.

\item {} 
000D0003 : No more pseudo terminals, or line not linked to TCP.

\item {} 
000D0004 : TCP/IP receive error; see preceding message VIRT923E or VIRT924E.

\item {} 
000D0005 : Socket is closed.

\item {} 
000D0006 : A message received on the XOT line does not conform to the XOT protocol. Either the version number in the XOT header is invalid, or the message length exceeds the negotiated packet size.

\end{itemize}

\item[{Action}] \leavevmode
If the cause cannot be determined by preceding messages, use the VIRTEL SNAP command to produce a dump of the internal trace table. Determine whether the router sent an invalid XOT message, and if so, perform the appropriate router diagnostic procedures. Otherwise contact technical support.

\end{description}


\subsubsection{VIRXO99E termid ERROR errcode SENDING TO SOCKET sockno}
\label{\detokenize{messages:virxo99e-termid-error-errcode-sending-to-socket-sockno}}\begin{description}
\item[{Module}] \leavevmode
VIRXOT

\item[{Meaning}] \leavevmode
An error occurred when sending a message to the router on an XOT line. termid is the name of the virtual circuit, if available, otherwise it is the external name of the XOT line. errcode is an error code whose meaning is:
\begin{itemize}
\item {} 
000B0001 : Invalid pseudo terminal.

\item {} 
000B0002 : Message not identified with an active session.

\item {} 
000B0003 : Line not linked to TCP stack.

\item {} 
000B0004 : TCP/IP send error; see preceding message VIRT923E or VIRT924E.

\item {} 
000B0005 : VIRTEL internal task dispatching error.

\item {} 
000B0006 : Session cleared before message could be sent.

\end{itemize}

\item[{Action}] \leavevmode
If the cause cannot be determined by preceding messages, use the VIRTEL SNAP command to produce a dump of the internal trace table and contact technical support.

\end{description}



\renewcommand{\indexname}{Index}
\printindex
\end{document}