%% Generated by Sphinx.
\def\sphinxdocclass{report}
\documentclass[letterpaper,10pt,english]{sphinxmanual}
\ifdefined\pdfpxdimen
   \let\sphinxpxdimen\pdfpxdimen\else\newdimen\sphinxpxdimen
\fi \sphinxpxdimen=.75bp\relax
\ifdefined\pdfimageresolution
    \pdfimageresolution= \numexpr \dimexpr1in\relax/\sphinxpxdimen\relax
\fi
%% let collapsible pdf bookmarks panel have high depth per default
\PassOptionsToPackage{bookmarksdepth=5}{hyperref}

\PassOptionsToPackage{booktabs}{sphinx}
\PassOptionsToPackage{colorrows}{sphinx}

\PassOptionsToPackage{warn}{textcomp}


\usepackage{cmap}

\usepackage{amsmath,amssymb,amstext}
\usepackage{babel}





\usepackage[Bjarne]{fncychap}
\usepackage[,numfigreset=1,mathnumfig]{sphinx}

\fvset{fontsize=auto}
\usepackage{geometry}


% Include hyperref last.
\usepackage{hyperref}
% Fix anchor placement for figures with captions.
\usepackage{hypcap}% it must be loaded after hyperref.
% Set up styles of URL: it should be placed after hyperref.
\urlstyle{same}

\addto\captionsenglish{\renewcommand{\contentsname}{Table of Contents:}}

\usepackage{sphinxmessages}
\setcounter{tocdepth}{2}


% Enable unicode and use Courier New to ensure the card suit
% characters that are part of the 'random' module examples
% appear properly in the PDF output.
\usepackage{fontspec}
\setmonofont{Courier New}


\title{Virtel Customization}
\date{Dec 14, 2023}
\release{4.61}
\author{Syspertec Communications}
\newcommand{\sphinxlogo}{\vbox{}}
\renewcommand{\releasename}{Release}
\makeindex
\begin{document}

\ifdefined\shorthandoff
  \ifnum\catcode`\=\string=\active\shorthandoff{=}\fi
  \ifnum\catcode`\"=\active\shorthandoff{"}\fi
\fi

\pagestyle{empty}
\sphinxmaketitle
\pagestyle{plain}
\sphinxtableofcontents
\pagestyle{normal}
\phantomsection\label{\detokenize{Customization::doc}}


\sphinxAtStartPar
\sphinxincludegraphics[scale=0.5]{{logo_virtel_web}.png}

\sphinxAtStartPar
\sphinxstylestrong{VIRTEL Customization}

\begin{sphinxadmonition}{warning}{Warning:}
\sphinxAtStartPar
This is a draft version of the document.
\end{sphinxadmonition}

\sphinxAtStartPar
Version : 4.61 Draft

\sphinxAtStartPar
Release Date : TBA. Publication Date : 10/10/2021

\sphinxAtStartPar
Syspertec Communication

\sphinxAtStartPar
196, Bureaux de la Colline 92213 Saint\sphinxhyphen{}Cloud Cedex Tél. : +33 (0) 1 46 02 60 42

\sphinxAtStartPar
\sphinxhref{http://www.syspertec.com/}{www.syspertec.com}

\begin{sphinxadmonition}{note}{Note:}
\sphinxAtStartPar
Reproduction, transfer, distribution, or storage, in any form, of all or any part of
the contents of this document, except by prior authorization of SysperTec
Communication, is prohibited.

\sphinxAtStartPar
Every possible effort has been made by SysperTec Communication to ensure that this document
is complete and relevant. In no case can SysperTec Communication be held responsible for
any damages, direct or indirect, caused by errors or omissions in this document.

\sphinxAtStartPar
As SysperTec Communication uses a continuous development methodology; the information
contained in this document may be subject to change without notice. Nothing in this
document should be construed in any manner as conferring a right to use, in whole or in
part, the products or trademarks quoted herein.

\sphinxAtStartPar
“SysperTec Communication” and “VIRTEL” are registered trademarks. Names of other products
and companies mentioned in this document may be trademarks or registered trademarks of
their respective owners.
\end{sphinxadmonition}

\newpage


\chapter{Summary of Amendments}
\label{\detokenize{Customization:summary-of-amendments}}\label{\detokenize{Customization:virtel461cs-summary-of-amendments}}

\section{Virtel version 4.61 (11th Oct 2021)}
\label{\detokenize{Customization:virtel-version-4-61-11th-oct-2021}}
\begin{sphinxadmonition}{note}{Note:}
\sphinxAtStartPar
For further details see the Virtel Technical Newsletter TN202101: Whats new in Virtel 4.61.
\end{sphinxadmonition}


\chapter{IPV6 Support}
\label{\detokenize{Customization:ipv6-support}}
\sphinxAtStartPar
Virtel 4.60 introduces support of IPV6. Throughout this document IP addresses default to the IPV4 construct \sphinxhyphen{} 192.168.048.002:41001 but this can also be read as an IPV6 construct. For example {[}\sphinxurl{http://{[}fd10:15c1:1921:a1a6::31{]}:41001}. In all cases the square brackets are essential.


\chapter{W2H Customization}
\label{\detokenize{Customization:w2h-customization}}
\sphinxAtStartPar
Customization of Virtel can be implemented through customization of the following elements:
\begin{itemize}
\item {} 
\sphinxAtStartPar
Custom CSS files, including Print Style CSS files.

\item {} 
\sphinxAtStartPar
Custom JavaScript files.

\item {} 
\sphinxAtStartPar
Custom setting parameters.

\item {} 
\sphinxAtStartPar
Custom Help pages.

\end{itemize}

\sphinxAtStartPar
As delivered, Virtel provides a dummy settings member, W2HPARM.JS, which is located in both the W2H\sphinxhyphen{}DIR and CLI\sphinxhyphen{}DIR. Note that any customization should only be made to the CLI\sphinxhyphen{}DIR or any other customer directories. No changes should be made to the W2H\sphinxhyphen{}DIR as these changes will be lost during any Virtel upgrade or migration.


\section{Levels of customization}
\label{\detokenize{Customization:levels-of-customization}}
\sphinxAtStartPar
Depending on how Virtel is implemented at a site will determine what level of customization is required. For most users, simple modification of the default settings may be all that is required. This can be achieved just by modifying the supplied dummy W2HPARM.JS in the CLI\sphinxhyphen{}DIR assuming that this is the only directory in use for any “user” transactions. The setting defaults are listed in Appendix A of the Virtel User Guide.

\sphinxAtStartPar
More complex installations may require different levels of customization invoking bespoke CSS and JavaScript members. Furthermore, different customization levels may be required for different Entry points and underlying transactions. The “Option” level of customization meets these requirements.

\index{Customization@\spxentry{Customization}!W2HPARM.JS@\spxentry{W2HPARM.JS}}\ignorespaces 

\subsection{Settings member W2HPARM.JS}
\label{\detokenize{Customization:settings-member-w2hparm-js}}\label{\detokenize{Customization:index-0}}
\sphinxAtStartPar
The Virtel settings member is a JavaScript object consisting of a key:value construct. The key representing the Virtel setting and the value the associated setting. Note that a key value can be another object as is the case in the “global\sphinxhyphen{}settings” and “useVirtelMacros” values. Keys names are case sensitive \sphinxhyphen{} key “w2hparm” is different to “W2hparm”. Invalid keys name will be ignored by Virtel. Key names are enclosed in quotes when defined within the w2hparm object, for example:

\begin{sphinxVerbatim}[commandchars=\\\{\}]
\PYG{n}{var} \PYG{n}{w2hparm} \PYG{o}{=} \PYG{p}{\PYGZob{}}
    \PYG{l+s+s2}{\PYGZdq{}}\PYG{l+s+s2}{ctrl}\PYG{l+s+s2}{\PYGZdq{}}\PYG{p}{:}\PYG{l+s+s2}{\PYGZdq{}}\PYG{l+s+s2}{Newline}\PYG{l+s+s2}{\PYGZdq{}}\PYG{p}{,}
    \PYG{p}{\PYGZcb{}}\PYG{p}{;}
\end{sphinxVerbatim}

\sphinxAtStartPar
but if a property is defined outside of the object definition, no quotes are specified. For example:

\begin{sphinxVerbatim}[commandchars=\\\{\}]
\PYG{n}{w2hparm}\PYG{o}{.}\PYG{n}{ctrl} \PYG{o}{=} \PYG{l+s+s2}{\PYGZdq{}}\PYG{l+s+s2}{Newline}\PYG{l+s+s2}{\PYGZdq{}}\PYG{p}{;}
\end{sphinxVerbatim}

\sphinxAtStartPar
See Appendix A of the Virtel Users Guide for a full list of “key” names and associated default values.

\index{Customization@\spxentry{Customization}!W2HPARM.JS@\spxentry{W2HPARM.JS}}\index{jokerkey@\spxentry{jokerkey}}\ignorespaces 

\subsection{Setting a Joker Key}
\label{\detokenize{Customization:setting-a-joker-key}}\label{\detokenize{Customization:index-1}}
\sphinxAtStartPar
The JokeyKey setting may be used to redefine keys on a keyboard to a one of the following functions \sphinxhyphen{} Reset, Ins, ATTN, Clear. For example, to redefine a key as an Insert key for keyboards that do not have an Insert key.
\begin{enumerate}
\sphinxsetlistlabels{\arabic}{enumi}{enumii}{}{.}%
\item {} 
\sphinxAtStartPar
First set which key will be the Joker Key in w2hparm.js

\end{enumerate}

\begin{sphinxVerbatim}[commandchars=\\\{\}]
\PYG{n}{w2hparm}\PYG{o}{.}\PYG{n}{jokeykey}\PYG{o}{=}\PYG{l+m+mi}{192}\PYG{p}{;}   \PYG{o}{/}\PYG{o}{/} \PYG{n}{Leftmost} \PYG{n}{key} \PYG{n}{on} \PYG{n}{second} \PYG{n}{row} \PYG{n}{of} \PYG{n}{US} \PYG{n}{keyboard}
\end{sphinxVerbatim}
\begin{enumerate}
\sphinxsetlistlabels{\arabic}{enumi}{enumii}{}{.}%
\setcounter{enumi}{1}
\item {} 
\sphinxAtStartPar
To determine the scancode of a key use the scancode tester by entering the following URL

\end{enumerate}

\begin{sphinxVerbatim}[commandchars=\\\{\}]
\PYG{n}{http}\PYG{p}{:}\PYG{o}{/}\PYG{o}{/}\PYG{l+m+mf}{192.168}\PYG{l+m+mf}{.170}\PYG{l+m+mf}{.48}\PYG{p}{:}\PYG{l+m+mi}{41001}\PYG{o}{/}\PYG{n}{w2h}\PYG{o}{/}\PYG{n}{scancode}\PYG{o}{.}\PYG{n}{htm}
\end{sphinxVerbatim}
\begin{enumerate}
\sphinxsetlistlabels{\arabic}{enumi}{enumii}{}{.}%
\setcounter{enumi}{2}
\item {} 
\sphinxAtStartPar
Administrators can preset the joker key function with

\end{enumerate}
\begin{description}
\sphinxlineitem{::}
\sphinxAtStartPar
w2hparm.joker | shiftjoker | altjoker = “Ins”

\end{description}

\begin{sphinxadmonition}{note}{Note:}
\sphinxAtStartPar
A user can set the function assigned to the Joker Key through the user settings.
\end{sphinxadmonition}

\sphinxAtStartPar
\sphinxincludegraphics{{image69}.png}
\sphinxstyleemphasis{Settings Panel showing SHIFT Joker key assignments}

\index{Customization@\spxentry{Customization}!Displaying customized members@\spxentry{Displaying customized members}}\ignorespaces 

\subsection{Displaying customized members}
\label{\detokenize{Customization:displaying-customized-members}}\label{\detokenize{Customization:index-2}}
\sphinxAtStartPar
To display a settings member use the following URLs as examples:

\begin{sphinxVerbatim}[commandchars=\\\{\}]
\PYG{n}{http}\PYG{p}{:}\PYG{o}{/}\PYG{o}{/}\PYG{l+m+mf}{192.168}\PYG{l+m+mf}{.170}\PYG{l+m+mf}{.48}\PYG{p}{:}\PYG{l+m+mi}{41002}\PYG{o}{/}\PYG{n}{w2h}\PYG{o}{/}\PYG{n}{w2hparm}\PYG{o}{\PYGZhy{}}\PYG{n}{js}\PYG{o}{/}\PYG{n}{w2hparm}\PYG{o}{.}\PYG{n}{js}
\end{sphinxVerbatim}

\sphinxAtStartPar
\sphinxstyleemphasis{Example of displaying the w2hparm.js member}

\sphinxAtStartPar
To display a customized option member use the following URL:

\begin{sphinxVerbatim}[commandchars=\\\{\}]
\PYG{n}{http}\PYG{p}{:}\PYG{o}{/}\PYG{o}{/}\PYG{l+m+mf}{192.168}\PYG{l+m+mf}{.170}\PYG{l+m+mf}{.48}\PYG{p}{:}\PYG{l+m+mi}{41002}\PYG{o}{/}\PYG{n}{option}\PYG{o}{/}\PYG{n}{w2hparm}\PYG{o}{.}\PYG{k}{global}\PYG{o}{.}\PYG{n}{js}
\end{sphinxVerbatim}

\sphinxAtStartPar
\sphinxstyleemphasis{Example of displaying a global option member}

\index{Customization@\spxentry{Customization}!Common settings member@\spxentry{Common settings member}}\ignorespaces 

\subsection{Common settings member}
\label{\detokenize{Customization:common-settings-member}}\label{\detokenize{Customization:index-3}}
\sphinxAtStartPar
If it is required that a common set of settings member be used for both W2H\sphinxhyphen{}DIR and CLI\sphinxhyphen{}DIR than a small modification to the W2H\sphinxhyphen{}DIR transaction, W2H\sphinxhyphen{}03P, is required. Changing the location directory from W2H\sphinxhyphen{}DIR to CLI\sphinxhyphen{}DIR will ensure that settings will be derived from the CLI\sphinxhyphen{}DIR W2HPARM.JS member. For example if you modify a key settings, which keyboard key reflects they “ENTER” key for example, than this change can used by both W2H and CLI transactions. An example of a modified W2H\sphinxhyphen{}03P transaction is displayed below:

\sphinxAtStartPar
\sphinxincludegraphics{{image56}.png}

\sphinxAtStartPar
\sphinxstyleemphasis{Example of W2H\sphinxhyphen{}03P pointing at the CLI\sphinxhyphen{}DIR}

\sphinxAtStartPar
The following JCL may also be used to modify the W2H\sphinxhyphen{}03P transaction.

\begin{sphinxadmonition}{note}{Note:}
\sphinxAtStartPar
Virtel must not be running when using the VIRCONF program to update transactions:
\end{sphinxadmonition}

\begin{sphinxVerbatim}[commandchars=\\\{\}]
\PYG{o}{/}\PYG{o}{/} \PYG{n}{SET} \PYG{n}{LOAD}\PYG{o}{=}\PYG{n}{HLQ}\PYG{o}{.}\PYG{n}{VIRTEL}\PYG{o}{.}\PYG{n}{LOADLIB}
\PYG{o}{/}\PYG{o}{/} \PYG{n}{SET} \PYG{n}{ARBO}\PYG{o}{=}\PYG{n}{HLQ}\PYG{o}{.}\PYG{n}{VIRTEL}\PYG{o}{.}\PYG{n}{ARBO}
\PYG{o}{/}\PYG{o}{/}\PYG{o}{*}
\PYG{o}{/}\PYG{o}{/}\PYG{n}{CONFIG}  \PYG{n}{EXEC} \PYG{n}{PGM}\PYG{o}{=}\PYG{n}{VIRCONF}\PYG{p}{,}\PYG{n}{PARM}\PYG{o}{=}\PYG{l+s+s1}{\PYGZsq{}}\PYG{l+s+s1}{LOAD,REPL}\PYG{l+s+s1}{\PYGZsq{}}\PYG{p}{,}\PYG{n}{REGION}\PYG{o}{=}\PYG{l+m+mi}{2}\PYG{n}{M}
\PYG{o}{/}\PYG{o}{/}\PYG{n}{STEPLIB}  \PYG{n}{DD}  \PYG{n}{DSN}\PYG{o}{=}\PYG{o}{\PYGZam{}}\PYG{n}{LOAD}\PYG{p}{,}\PYG{n}{DISP}\PYG{o}{=}\PYG{n}{SHR}
\PYG{o}{/}\PYG{o}{/}\PYG{n}{SYSPRINT} \PYG{n}{DD}  \PYG{n}{SYSOUT}\PYG{o}{=}\PYG{o}{*}
\PYG{o}{/}\PYG{o}{/}\PYG{n}{SYSUDUMP} \PYG{n}{DD}  \PYG{n}{SYSOUT}\PYG{o}{=}\PYG{o}{*}
\PYG{o}{/}\PYG{o}{/}\PYG{n}{VIRARBO}  \PYG{n}{DD}  \PYG{n}{DSN}\PYG{o}{=}\PYG{o}{\PYGZam{}}\PYG{n}{ARBO}\PYG{p}{,}\PYG{n}{DISP}\PYG{o}{=}\PYG{n}{SHR}
\PYG{o}{/}\PYG{o}{/}\PYG{n}{SYSIN}      \PYG{n}{DD} \PYG{o}{*}
     \PYG{n}{TRANSACT} \PYG{n}{ID}\PYG{o}{=}\PYG{n}{W2H}\PYG{o}{\PYGZhy{}}\PYG{l+m+mi}{03}\PYG{n}{P}\PYG{p}{,}
     \PYG{n}{NAME}\PYG{o}{=}\PYG{l+s+s1}{\PYGZsq{}}\PYG{l+s+s1}{w2h}\PYG{l+s+s1}{\PYGZsq{}}\PYG{p}{,}
     \PYG{n}{DESC}\PYG{o}{=}\PYG{l+s+s1}{\PYGZsq{}}\PYG{l+s+s1}{Directory for w2hparm.js}\PYG{l+s+s1}{\PYGZsq{}}\PYG{p}{,}
     \PYG{n}{APPL}\PYG{o}{=}\PYG{n}{CLI}\PYG{o}{\PYGZhy{}}\PYG{n}{DIR}\PYG{p}{,}
     \PYG{n}{TYPE}\PYG{o}{=}\PYG{l+m+mi}{4}\PYG{p}{,}
     \PYG{n}{TERMINAL}\PYG{o}{=}\PYG{n}{DELOC}\PYG{p}{,}
     \PYG{n}{STARTUP}\PYG{o}{=}\PYG{l+m+mi}{2}\PYG{p}{,}
     \PYG{n}{SECURITY}\PYG{o}{=}\PYG{l+m+mi}{0}\PYG{p}{,}
     \PYG{n}{LOGMSG}\PYG{o}{=}\PYG{l+s+s1}{\PYGZsq{}}\PYG{l+s+s1}{/w2h/w2hparm\PYGZhy{}js}\PYG{l+s+s1}{\PYGZsq{}}
\end{sphinxVerbatim}

\index{Customization@\spxentry{Customization}!Global Level Options@\spxentry{Global Level Options}}\ignorespaces 

\section{Global Level Options}
\label{\detokenize{Customization:global-level-options}}\label{\detokenize{Customization:index-4}}
\sphinxAtStartPar
Virtel implements a “global” option customized level whereby all customized elements apply to all transactions of a particular Entry Point. Each Entry Point may have its own “global” options. To implement global option customization, the W2HPARM.JS settings member is modified to contain a global settings object whose attributes point to members located in the options directory OPT\sphinxhyphen{}DIR.

\begin{sphinxVerbatim}[commandchars=\\\{\}]
\PYG{n}{var} \PYG{n}{w2hparm} \PYG{o}{=} \PYG{p}{\PYGZob{}}
    \PYG{l+s+s2}{\PYGZdq{}}\PYG{l+s+s2}{global\PYGZhy{}settings}\PYG{l+s+s2}{\PYGZdq{}}\PYG{p}{:}\PYG{p}{\PYGZob{}}
        \PYG{l+s+s2}{\PYGZdq{}}\PYG{l+s+s2}{pathToJsCustom}\PYG{l+s+s2}{\PYGZdq{}}\PYG{p}{:}\PYG{l+s+s2}{\PYGZdq{}}\PYG{l+s+s2}{../option/custJS.global.js}\PYG{l+s+s2}{\PYGZdq{}}\PYG{p}{,}
        \PYG{l+s+s2}{\PYGZdq{}}\PYG{l+s+s2}{pathToCssCustom}\PYG{l+s+s2}{\PYGZdq{}}\PYG{p}{:} \PYG{l+s+s2}{\PYGZdq{}}\PYG{l+s+s2}{../option/custCSS.global.css}\PYG{l+s+s2}{\PYGZdq{}}\PYG{p}{,}
        \PYG{l+s+s2}{\PYGZdq{}}\PYG{l+s+s2}{pathToW2hparm}\PYG{l+s+s2}{\PYGZdq{}}\PYG{p}{:}\PYG{l+s+s2}{\PYGZdq{}}\PYG{l+s+s2}{../option/w2hparm.global.js}\PYG{l+s+s2}{\PYGZdq{}}\PYG{p}{,}
        \PYG{l+s+s2}{\PYGZdq{}}\PYG{l+s+s2}{pathToFavicon}\PYG{l+s+s2}{\PYGZdq{}}\PYG{p}{:} \PYG{l+s+s2}{\PYGZdq{}}\PYG{l+s+s2}{../option/cust\PYGZhy{}favicon.ico}\PYG{l+s+s2}{\PYGZdq{}}
        \PYG{l+s+s2}{\PYGZdq{}}\PYG{l+s+s2}{pathToHelp}\PYG{l+s+s2}{\PYGZdq{}}\PYG{p}{:} \PYG{l+s+s2}{\PYGZdq{}}\PYG{l+s+s2}{../option/myHelp.html}\PYG{l+s+s2}{\PYGZdq{}}
    \PYG{p}{\PYGZcb{}}
\PYG{p}{\PYGZcb{}}
\end{sphinxVerbatim}

\sphinxAtStartPar
The “option” pathname is normally associated with the OPT\sphinxhyphen{}DIR. The directory is pointed to by the OPTION transaction CLI\sphinxhyphen{}03CO and by default is the CLI\sphinxhyphen{}DIR:

\sphinxAtStartPar
\sphinxincludegraphics{{image33}.png}

\sphinxAtStartPar
\sphinxstyleemphasis{Example of option transaction}

\sphinxAtStartPar
Each property of the global\sphinxhyphen{}settings object uses a “pathTo” property which has the following pattern:\sphinxhyphen{} “pathTo.key.id.type” where:

\begin{sphinxVerbatim}[commandchars=\\\{\}]
\PYG{n}{key}     \PYG{o}{=} \PYG{n}{W2hparm} \PYG{o}{|} \PYG{n}{CssCustom} \PYG{o}{|} \PYG{n}{JSCustom} \PYG{o}{|} \PYG{n}{PrintCss} \PYG{o}{|} \PYG{n}{Help} \PYG{o}{|} \PYG{n}{Favicon}  \PYG{n}{Case} \PYG{n}{sensitive}\PYG{o}{.}
\PYG{n+nb}{id}      \PYG{o}{=} \PYG{l+s+s2}{\PYGZdq{}}\PYG{l+s+s2}{global}\PYG{l+s+s2}{\PYGZdq{}} \PYG{o+ow}{or} \PYG{n}{a} \PYG{n}{transaction} \PYG{n}{option} \PYG{n}{identifier}       \PYG{n}{Option} \PYG{l+s+s2}{\PYGZdq{}}\PYG{l+s+s2}{id}\PYG{l+s+s2}{\PYGZdq{}}
\PYG{n+nb}{type}    \PYG{o}{=} \PYG{n}{css} \PYG{o}{|} \PYG{n}{html} \PYG{o}{|} \PYG{n}{js}                                   \PYG{n}{Type}
\end{sphinxVerbatim}

\sphinxAtStartPar
Global\sphinxhyphen{}settings object can contain the following pathTo properties:

\begin{sphinxVerbatim}[commandchars=\\\{\}]
\PYG{l+s+s2}{\PYGZdq{}}\PYG{l+s+s2}{pathToJsCustom}\PYG{l+s+s2}{\PYGZdq{}}  \PYG{o}{\PYGZhy{}} \PYG{n}{Customized} \PYG{n}{JavaSCript} \PYG{n}{elements}
\PYG{l+s+s2}{\PYGZdq{}}\PYG{l+s+s2}{pathToW2hparm}\PYG{l+s+s2}{\PYGZdq{}}   \PYG{o}{\PYGZhy{}} \PYG{n}{Customized} \PYG{n}{w2hparm} \PYG{n}{parameters}
\PYG{l+s+s2}{\PYGZdq{}}\PYG{l+s+s2}{pathToCssCustom}\PYG{l+s+s2}{\PYGZdq{}} \PYG{o}{\PYGZhy{}} \PYG{n}{Customized} \PYG{n}{CSS} \PYG{n}{stylesheets}
\PYG{l+s+s2}{\PYGZdq{}}\PYG{l+s+s2}{pathToPrintCss}\PYG{l+s+s2}{\PYGZdq{}}  \PYG{o}{\PYGZhy{}} \PYG{n}{Customized} \PYG{n}{Print} \PYG{n}{CSS} \PYG{n}{Style} \PYG{n}{sheets}
\PYG{l+s+s2}{\PYGZdq{}}\PYG{l+s+s2}{pathToHelp}\PYG{l+s+s2}{\PYGZdq{}}      \PYG{o}{\PYGZhy{}} \PYG{n}{Customized} \PYG{n}{Help} \PYG{n}{pages}
\PYG{l+s+s2}{\PYGZdq{}}\PYG{l+s+s2}{pathToFavicon}\PYG{l+s+s2}{\PYGZdq{}}   \PYG{o}{\PYGZhy{}} \PYG{n}{Customized} \PYG{n}{Favicon} \PYG{n}{image}
\end{sphinxVerbatim}

\sphinxAtStartPar
Restart Virtel after uploading the W2HPARM.JS and other global members. Refresh the cache then check that the correct settings are being used using the following URLs to list w2hparm.js and w2hparm.global.js:

\begin{sphinxVerbatim}[commandchars=\\\{\}]
\PYG{n}{http}\PYG{p}{:}\PYG{o}{/}\PYG{o}{/}\PYG{l+m+mf}{192.168}\PYG{l+m+mf}{.170}\PYG{l+m+mf}{.48}\PYG{p}{:}\PYG{l+m+mi}{41002}\PYG{o}{/}\PYG{n}{w2h}\PYG{o}{/}\PYG{n}{w2hparm}\PYG{o}{\PYGZhy{}}\PYG{n}{js}\PYG{o}{/}\PYG{n}{w2hparm}\PYG{o}{.}\PYG{n}{js}

\PYG{n}{var} \PYG{n}{w2hparm} \PYG{o}{=} \PYG{p}{\PYGZob{}}
    \PYG{l+s+s2}{\PYGZdq{}}\PYG{l+s+s2}{global\PYGZhy{}settings}\PYG{l+s+s2}{\PYGZdq{}}\PYG{p}{:}\PYG{p}{\PYGZob{}}
        \PYG{l+s+s2}{\PYGZdq{}}\PYG{l+s+s2}{pathToW2hparm}\PYG{l+s+s2}{\PYGZdq{}}\PYG{p}{:}\PYG{l+s+s2}{\PYGZdq{}}\PYG{l+s+s2}{../option/w2hparm.global.js}\PYG{l+s+s2}{\PYGZdq{}}
    \PYG{p}{\PYGZcb{}}
\PYG{p}{\PYGZcb{}}\PYG{p}{;}
\end{sphinxVerbatim}

\sphinxAtStartPar
\sphinxstyleemphasis{Example of w2hparm.js pointing to a global parm member}

\begin{sphinxVerbatim}[commandchars=\\\{\}]
\PYG{n}{http}\PYG{p}{:}\PYG{o}{/}\PYG{o}{/}\PYG{l+m+mf}{192.168}\PYG{l+m+mf}{.170}\PYG{l+m+mf}{.48}\PYG{p}{:}\PYG{l+m+mi}{41002}\PYG{o}{/}\PYG{n}{option}\PYG{o}{/}\PYG{n}{w2hparm}\PYG{o}{.}\PYG{k}{global}\PYG{o}{.}\PYG{n}{js}
\end{sphinxVerbatim}

\sphinxAtStartPar
The output from this display should show the DDI/Macros options that we used in compatibility mode.

\begin{sphinxVerbatim}[commandchars=\\\{\}]
\PYG{n}{var} \PYG{n}{w2hparm} \PYG{o}{=} \PYG{p}{\PYGZob{}}\PYG{l+s+s2}{\PYGZdq{}}\PYG{l+s+s2}{ctrl}\PYG{l+s+s2}{\PYGZdq{}}\PYG{p}{:}\PYG{l+s+s2}{\PYGZdq{}}\PYG{l+s+s2}{Newline}\PYG{l+s+s2}{\PYGZdq{}}\PYG{p}{,}
    \PYG{l+s+s2}{\PYGZdq{}}\PYG{l+s+s2}{enter}\PYG{l+s+s2}{\PYGZdq{}}\PYG{p}{:}\PYG{l+s+s2}{\PYGZdq{}}\PYG{l+s+s2}{ENTER}\PYG{l+s+s2}{\PYGZdq{}}\PYG{p}{,}
    \PYG{l+s+s2}{\PYGZdq{}}\PYG{l+s+s2}{pgup}\PYG{l+s+s2}{\PYGZdq{}}\PYG{p}{:}\PYG{l+s+s2}{\PYGZdq{}}\PYG{l+s+s2}{PA1}\PYG{l+s+s2}{\PYGZdq{}}\PYG{p}{,}
    \PYG{l+s+s2}{\PYGZdq{}}\PYG{l+s+s2}{pgdn}\PYG{l+s+s2}{\PYGZdq{}}\PYG{p}{:}\PYG{l+s+s2}{\PYGZdq{}}\PYG{l+s+s2}{PA2}\PYG{l+s+s2}{\PYGZdq{}}\PYG{p}{,}
    \PYG{l+s+s2}{\PYGZdq{}}\PYG{l+s+s2}{useVirtelMacros}\PYG{l+s+s2}{\PYGZdq{}}\PYG{p}{:}\PYG{p}{\PYGZob{}}\PYG{l+s+s2}{\PYGZdq{}}\PYG{l+s+s2}{macrosAutoRefresh}\PYG{l+s+s2}{\PYGZdq{}}\PYG{p}{:}\PYG{l+s+s2}{\PYGZdq{}}\PYG{l+s+s2}{session}\PYG{l+s+s2}{\PYGZdq{}}\PYG{p}{\PYGZcb{}}
    \PYG{p}{\PYGZcb{}}\PYG{p}{;}
\PYG{n}{w2hparm}\PYG{o}{.}\PYG{n}{keymapping}\PYG{o}{=}\PYG{n}{true}\PYG{p}{;}
\end{sphinxVerbatim}

\sphinxAtStartPar
\sphinxstyleemphasis{Example of a wh2parm.global.js member}


\subsection{Adding your Company Logo to Virtel Toolbar using global options}
\label{\detokenize{Customization:adding-your-company-logo-to-virtel-toolbar-using-global-options}}
\sphinxAtStartPar
Update the following two files:\sphinxhyphen{}
\begin{enumerate}
\sphinxsetlistlabels{\arabic}{enumi}{enumii}{}{.}%
\item {} 
\sphinxAtStartPar
custCSS.{[}key name{]}.css      Cascading style sheet customisation for company logo

\end{enumerate}

\sphinxAtStartPar
Append to you existing Cascading Style sheet the following additional style sheet definitions along with a .jpeg or .png file containing the company logo. Use the Chrome debugger tool to help determine the proper logo size.

\sphinxAtStartPar
\sphinxincludegraphics{{image83}.png}
\begin{enumerate}
\sphinxsetlistlabels{\arabic}{enumi}{enumii}{}{.}%
\setcounter{enumi}{1}
\item {} 
\sphinxAtStartPar
w2hparm.js                  Update or add a Global settings w2hparm.js file.

\end{enumerate}

\sphinxAtStartPar
The Global VWA display setting file also has 2 parameters used to apply .css customization to all transactions. Add the last two lines below to your existing w2hparm.js file.  This will apply the custCSS.CLIENT.css setting to all defined Virtel Transactions.

\sphinxAtStartPar
\sphinxincludegraphics{{image84}.png}

\sphinxAtStartPar
From the Administration panel select the VWA customized files, the custCSS,{[}key name{]},css and the w2hparm.js file with the global setting, and upload to your CLI directory using the Virtel Drap and drop function.

\sphinxAtStartPar
\sphinxincludegraphics{{image85}.png}

\begin{sphinxadmonition}{note}{Note:}
\sphinxAtStartPar
Refresh your browser cache and Reconnect to the Application Menu to see the results.
\end{sphinxadmonition}

\index{Customization@\spxentry{Customization}!Transaction Level Options@\spxentry{Transaction Level Options}}\ignorespaces 

\section{Transaction Level Options}
\label{\detokenize{Customization:transaction-level-options}}\label{\detokenize{Customization:index-5}}
\sphinxAtStartPar
Some transactions may require a different set of options or settings than those defined at the Entry Point or Global level. For example, certain transactions may have a modified Tool Bar containing a Printer Icon. Transactions which require such specific customization should have a unique option identifier or “Id” associated with the transaction definition. This “id” will pull in a different set of options “key”:”values” settings and will override the “global” settings. The option “id” will be used to associate customized elements against specific transaction(s).

\sphinxAtStartPar
In the following example the value “myOptions” has been chosen as an “id” for a particular transaction. A transaction level “core option file”, using the format “option.id.js”, will be used to locate they customized elements. In this example we are using a customized JavaScript member to add a “Printer Icon” to the Virtel Toolbar. Using the option “id” of “myOptions” a javascript file will be created called “option.myOptions.js”. This will be uploaded to CLI\sphinxhyphen{}DIR. Customized settings for transactions are held in a JavaScript object called “oCustom”. Here is an example of the myOptions JavaScript file:

\begin{sphinxVerbatim}[commandchars=\\\{\}]
\PYG{o}{/}\PYG{o}{/}\PYG{n}{CLI}\PYG{o}{\PYGZhy{}}\PYG{n}{DIR} \PYG{o}{\PYGZhy{}} \PYG{n}{option}\PYG{o}{.}\PYG{n}{myOptions}\PYG{o}{.}\PYG{n}{js}\PYG{o}{.} \PYG{n}{Transaction} \PYG{n}{level} \PYG{n}{core} \PYG{n}{option} \PYG{n}{file}
\PYG{n}{var} \PYG{n}{oCustom}\PYG{o}{=}\PYG{p}{\PYGZob{}}
    \PYG{l+s+s2}{\PYGZdq{}}\PYG{l+s+s2}{pathToJsCustom}\PYG{l+s+s2}{\PYGZdq{}}\PYG{p}{:}\PYG{l+s+s2}{\PYGZdq{}}\PYG{l+s+s2}{../option/custJS.myOptions.js}\PYG{l+s+s2}{\PYGZdq{}}
\PYG{p}{\PYGZcb{}}
\end{sphinxVerbatim}

\sphinxAtStartPar
\sphinxstyleemphasis{Example of a core option file for customized JavaScript}

\sphinxAtStartPar
This “”core option file” points to our customized JavaScript member for this transaction. This member contains the Virtel JavaScript function “after\_standardInit” which will be called after initialization to modify the standard Virtel Tool Bar:

\begin{sphinxVerbatim}[commandchars=\\\{\}]
\PYG{o}{/}\PYG{o}{/}\PYG{n}{CLI}\PYG{o}{\PYGZhy{}}\PYG{n}{DIR} \PYG{o}{\PYGZhy{}} \PYG{n}{custJS}\PYG{o}{.}\PYG{n}{myOptions}\PYG{o}{.}\PYG{n}{js}
\PYG{o}{/}\PYG{o}{/}\PYG{n}{Add} \PYG{n}{Print} \PYG{n}{Button} \PYG{n}{To} \PYG{n}{Toolbar}
\PYG{n}{function} \PYG{n}{after\PYGZus{}standardInit}\PYG{p}{(}\PYG{p}{)} \PYG{p}{\PYGZob{}}
    \PYG{n}{addTool} \PYG{n}{Barbutton}\PYG{p}{(}\PYG{l+m+mi}{000}\PYG{p}{,}\PYG{l+s+s2}{\PYGZdq{}}\PYG{l+s+s2}{../print.ico}\PYG{l+s+s2}{\PYGZdq{}}\PYG{p}{,}\PYG{l+s+s2}{\PYGZdq{}}\PYG{l+s+s2}{Print Screen}\PYG{l+s+s2}{\PYGZdq{}}\PYG{p}{,}\PYG{n}{do\PYGZus{}print}\PYG{p}{)}\PYG{p}{;}
\PYG{p}{\PYGZcb{}}

\PYG{n}{function} \PYG{n}{do\PYGZus{}print}\PYG{p}{(}\PYG{p}{)} \PYG{p}{\PYGZob{}}
\PYG{o}{/}\PYG{o}{/} \PYG{n}{do\PYGZus{}print} \PYG{n}{function}
\PYG{p}{\PYGZcb{}}
\end{sphinxVerbatim}

\sphinxAtStartPar
\sphinxstyleemphasis{Example of customized JavaScript element associated with a transaction}

\sphinxAtStartPar
To support this customized transaction the following two files are uploaded to the OPT\sphinxhyphen{}DIR or options directory:
\begin{description}
\sphinxlineitem{::}
\sphinxAtStartPar
option.myOptions.js  \sphinxhyphen{} core option file for identifier “myOptions”. Points to the custom javascript file for transactions which have option=myOptions set.
custjJS.myOptions.js \sphinxhyphen{} customized java script file for “myOptions”.

\end{description}

\begin{sphinxadmonition}{note}{Note:}
\sphinxAtStartPar
More than one transaction within an Entry Point can point to the same “id” option files.
\end{sphinxadmonition}

\index{Customization@\spxentry{Customization}!Transaction Level Options@\spxentry{Transaction Level Options}}\index{Transaction Level Options@\spxentry{Transaction Level Options}!Customization@\spxentry{Customization}}\ignorespaces 

\section{Defining Transaction Options}
\label{\detokenize{Customization:defining-transaction-options}}\label{\detokenize{Customization:index-6}}
\sphinxAtStartPar
The Virtel User Guide contains further information on how to use GUI based controls to create the necessary customization members and associated files for transaction settings customization. See the section “Defining Transaction Options” in the Customization section for further details.

\index{Customization@\spxentry{Customization}!JavaScript User Exits@\spxentry{JavaScript User Exits}}\index{JavaScript User Exits@\spxentry{JavaScript User Exits}!Customization@\spxentry{Customization}}\ignorespaces 

\section{Customizable JavaScript User Exits}
\label{\detokenize{Customization:customizable-javascript-user-exits}}\label{\detokenize{Customization:index-7}}
\sphinxAtStartPar
Virtel provides the following JavaScript functions that can be customized to give bespoke behavior. The custom.js file is load when a session started. This function, which is normally empty, can contain bespoke javascript functions which are called at strategic exits points during the life of a Virtel transaction. The functions are:

\index{JavaScript User Exits@\spxentry{JavaScript User Exits}!before\_submitForm@\spxentry{before\_submitForm}}\index{before\_submitForm@\spxentry{before\_submitForm}!JavaScript User Exits@\spxentry{JavaScript User Exits}}\ignorespaces \begin{itemize}
\item {} 
\sphinxAtStartPar
\sphinxstylestrong{before\_submitForm(pfKey,oFrom)} \sphinxhyphen{} Called before submitting a Virtel request.

\begin{sphinxVerbatim}[commandchars=\\\{\}]
\PYG{n}{Arguments}   \PYG{o}{*}\PYG{n}{pfKey}\PYG{o}{*} \PYG{n}{The} \PYG{n}{Name} \PYG{n}{of} \PYG{n}{the} \PYG{n}{key} \PYG{n}{pressed}\PYG{o}{.} \PYG{l+s+s2}{\PYGZdq{}}\PYG{l+s+s2}{ENTER}\PYG{l+s+s2}{\PYGZdq{}}\PYG{p}{,}\PYG{l+s+s2}{\PYGZdq{}}\PYG{l+s+s2}{PF1}\PYG{l+s+s2}{\PYGZdq{}}\PYG{p}{,}\PYG{n}{etc}\PYG{o}{.}
            \PYG{o}{*}\PYG{n}{oForm}\PYG{o}{*} \PYG{n}{Form} \PYG{n}{DOM} \PYG{n+nb}{object} \PYG{n}{to} \PYG{n}{be} \PYG{n}{submitted}\PYG{o}{.} \PYG{p}{(}\PYG{n}{usually} \PYG{n}{document}\PYG{o}{.}\PYG{n}{virtelForm}\PYG{p}{)}
\end{sphinxVerbatim}

\end{itemize}

\index{JavaScript User Exits@\spxentry{JavaScript User Exits}!after\_standardInit@\spxentry{after\_standardInit}}\index{after\_standardInit@\spxentry{after\_standardInit}!JavaScript User Exits@\spxentry{JavaScript User Exits}}\ignorespaces \begin{itemize}
\item {} 
\sphinxAtStartPar
\sphinxstylestrong{after\_standardInit()} \sphinxhyphen{} Called after a session with the host application has started.

\end{itemize}

\index{JavaScript User Exits@\spxentry{JavaScript User Exits}!after\_responseHandle@\spxentry{after\_responseHandle}}\index{after\_responseHandle@\spxentry{after\_responseHandle}!JavaScript User Exits@\spxentry{JavaScript User Exits}}\ignorespaces \begin{itemize}
\item {} 
\sphinxAtStartPar
\sphinxstylestrong{after\_responseHandle(o,url,xtim)} \sphinxhyphen{} Called after receiving a response from the VIRTEL server.

\begin{sphinxVerbatim}[commandchars=\\\{\}]
\PYG{n}{Arguments}   \PYG{o}{*}\PYG{n}{o}\PYG{o}{*}     \PYG{n}{XMLHttpRequest} \PYG{n+nb}{object} \PYG{p}{(}\PYG{n}{status} \PYG{o+ow}{and} \PYG{n}{responseText}\PYG{p}{)}
            \PYG{o}{*}\PYG{n}{url}\PYG{o}{*}   \PYG{n}{The} \PYG{n}{URL} \PYG{n}{which} \PYG{n}{was} \PYG{n}{used} \PYG{o+ow}{in} \PYG{n}{the} \PYG{n}{request}
            \PYG{o}{*}\PYG{n}{xtim}\PYG{o}{*}  \PYG{n}{JavaScript} \PYG{n}{Date} \PYG{n+nb}{object} \PYG{n}{of} \PYG{n}{time} \PYG{n}{request} \PYG{n}{was} \PYG{n}{sent} \PYG{n}{to} \PYG{n}{the} \PYG{n}{server}
\end{sphinxVerbatim}

\end{itemize}

\index{JavaScript User Exits@\spxentry{JavaScript User Exits}!modify\_settings@\spxentry{modify\_settings}}\index{modify\_settings@\spxentry{modify\_settings}!JavaScript User Exits@\spxentry{JavaScript User Exits}}\ignorespaces \begin{itemize}
\item {} 
\sphinxAtStartPar
\sphinxstylestrong{modify\_settingsValues(name,values)} \sphinxhyphen{}  Allows modification of settings before parameters are displayed in the Settings Menu. It allows the list of values to be modified.

\begin{sphinxVerbatim}[commandchars=\\\{\}]
\PYG{n}{Arguments}   \PYG{o}{*}\PYG{n}{name}\PYG{o}{*}  \PYG{n}{The} \PYG{n}{parameter} \PYG{n}{name}
            \PYG{o}{*}\PYG{n}{values}\PYG{o}{*} \PYG{n}{The} \PYG{n+nb}{list} \PYG{n}{of} \PYG{n}{possible} \PYG{n}{values}\PYG{o}{.}

\PYG{n}{The} \PYG{k}{return} \PYG{n}{value} \PYG{o+ow}{is} \PYG{n}{treated} \PYG{k}{as} \PYG{n}{the} \PYG{n}{new} \PYG{n}{value}\PYG{o}{.} \PYG{n}{If} \PYG{n}{the} \PYG{n}{function} \PYG{n}{returns}
\PYG{n}{null} \PYG{o+ow}{or} \PYG{n}{undefined}\PYG{p}{,} \PYG{n}{the} \PYG{n+nb}{list} \PYG{n}{remains} \PYG{n}{unchanged}\PYG{o}{.}
\end{sphinxVerbatim}

\end{itemize}

\index{JavaScript User Exits@\spxentry{JavaScript User Exits}!when\_init()@\spxentry{when\_init()}}\index{when\_init()@\spxentry{when\_init()}!JavaScript User Exits@\spxentry{JavaScript User Exits}}\ignorespaces \begin{itemize}
\item {} 
\sphinxAtStartPar
\sphinxstylestrong{when\_init()} \sphinxhyphen{} Called for each sub\sphinxhyphen{}page after vir3270 initialization.

\end{itemize}

\index{JavaScript User Exits@\spxentry{JavaScript User Exits}!when\_focusGained@\spxentry{when\_focusGained}}\index{when\_focusGained@\spxentry{when\_focusGained}!JavaScript User Exits@\spxentry{JavaScript User Exits}}\ignorespaces \begin{itemize}
\item {} 
\sphinxAtStartPar
\sphinxstylestrong{when\_focusGained()} \sphinxhyphen{} Called whenever the 3270 window gains focus.

\end{itemize}

\index{JavaScript User Exits@\spxentry{JavaScript User Exits}!when\_focusLost@\spxentry{when\_focusLost}}\index{when\_focusLost@\spxentry{when\_focusLost}!JavaScript User Exits@\spxentry{JavaScript User Exits}}\ignorespaces \begin{itemize}
\item {} 
\sphinxAtStartPar
\sphinxstylestrong{when\_focusLost()} \sphinxhyphen{} Called whenever the 3270 window loses focus.

\end{itemize}


\chapter{Virtel Web Access Interface}
\label{\detokenize{Customization:virtel-web-access-interface}}
\index{Virtel Web Access@\spxentry{Virtel Web Access}!VWA User Interface@\spxentry{VWA User Interface}}\ignorespaces 

\section{VWA User Interface}
\label{\detokenize{Customization:vwa-user-interface}}\label{\detokenize{Customization:index-15}}
\sphinxAtStartPar
The VIRTEL Web Access user interface is in the form of a conventional 3270 screen divided into three areas:
\begin{itemize}
\item {} 
\sphinxAtStartPar
The Tool Bar located in the upper of the screen contains the icons of functions (1), the information environment (2),the language selection (3).

\item {} 
\sphinxAtStartPar
The status line at the bottom includes the monitoring zone (4), particulars of the terminals associated with the session(5), mode and cursor position (6).

\item {} 
\sphinxAtStartPar
The area between the Tool Bar and the status bar is used to display the contents of 3270 screens, it can be of variable size between 24x80 and 62x132.

\end{itemize}

\sphinxAtStartPar
\sphinxincludegraphics{{image1}.png}
\sphinxstyleemphasis{The VWA 3270 screen areas}


\subsection{Title}
\label{\detokenize{Customization:title}}
\index{Customization@\spxentry{Customization}!Add User Name to Title Tab@\spxentry{Add User Name to Title Tab}}\ignorespaces 

\subsubsection{Add User Name to Title Tab}
\label{\detokenize{Customization:add-user-name-to-title-tab}}\label{\detokenize{Customization:index-16}}
\sphinxAtStartPar
You can add the User Name into the title tab by using the following function in a custom.js file:

\begin{sphinxVerbatim}[commandchars=\\\{\}]
function after\PYGZus{}responseHandle() \PYGZob{}
    document.title =document.body.className + “\PYGZhy{}“ + getUserId();
\PYGZcb{}
\end{sphinxVerbatim}

\sphinxAtStartPar
\sphinxstyleemphasis{Example to add User Name to Title Tab.}


\subsection{Tool bar}
\label{\detokenize{Customization:tool-bar}}
\sphinxAtStartPar
The Tool Bar is located at the top of the 3270 screen. It contains, icons, language selection tool and environment
information. Some of those component can be removed, added or modified.

\index{Customization@\spxentry{Customization}!Hide Tool Bar {[}Settings{]}@\spxentry{Hide Tool Bar {[}Settings{]}}}\index{Tool Bar@\spxentry{Tool Bar}!Hide@\spxentry{Hide}}\ignorespaces 

\subsubsection{Hiding the Tool Bar}
\label{\detokenize{Customization:hiding-the-tool-bar}}\label{\detokenize{Customization:index-17}}
\sphinxAtStartPar
The administrator may prevent access to features like copy/paste, print, and settings by removing the corresponding Tool Bar icons, or by hiding the Tool Bar altogether. This example shows how to hide the Tool Bar using a custom.css file:

\begin{sphinxVerbatim}[commandchars=\\\{\}]
\PYG{o}{/}\PYG{o}{*} \PYG{n}{VIRTEL} \PYG{n}{Web} \PYG{n}{Access} \PYG{n}{style} \PYG{n}{sheet} \PYG{k}{for} \PYG{n}{site} \PYG{n}{customization}
\PYG{o}{*} \PYG{p}{(}\PYG{n}{c}\PYG{p}{)}\PYG{n}{Copyright} \PYG{n}{SysperTec} \PYG{n}{Communication} \PYG{l+m+mi}{2007}\PYG{p}{,}\PYG{l+m+mi}{2010} \PYG{n}{All} \PYG{n}{Rights} \PYG{n}{Reserved}
\PYG{o}{*}\PYG{o}{/}
 \PYG{c+c1}{\PYGZsh{}Tool Bar \PYGZob{}display:none;\PYGZcb{}}
\end{sphinxVerbatim}

\sphinxAtStartPar
\sphinxstyleemphasis{Example custom.css for hiding the Tool Bar}

\sphinxAtStartPar
You can also use custom.js to remove individual icons from the Tool Bar, see “Removing unwanted Tool Bar icons” on page 11.

\index{Customization@\spxentry{Customization}!Enable/Disable the Tool Bar display@\spxentry{Enable/Disable the Tool Bar display}}\index{Tool Bar@\spxentry{Tool Bar}!Enable and Disable Display@\spxentry{Enable and Disable Display}}\ignorespaces 

\subsubsection{Enable/Disable the Tool Bar display}
\label{\detokenize{Customization:enable-disable-the-tool-bar-display}}\label{\detokenize{Customization:index-18}}
\sphinxAtStartPar
The administrator can provide users with controls that enable or disable the display of the Tool Bar.

\begin{sphinxVerbatim}[commandchars=\\\{\}]
\PYG{o}{/}\PYG{o}{*}
 \PYG{o}{*} \PYG{n}{Configuration} \PYG{n}{to} \PYG{n}{allow} \PYG{n}{the} \PYG{n}{user} \PYG{n}{to} \PYG{n}{manage} \PYG{n}{show}\PYG{o}{/}\PYG{n}{hide} \PYG{n}{Tool} \PYG{n}{Bar} \PYG{n}{by} \PYG{n}{himself}\PYG{o}{.}
 \PYG{o}{*}\PYG{o}{/}
\PYG{n}{w2hparm}\PYG{o}{.}\PYG{n}{switchToolBarDisplay} \PYG{o}{=} \PYG{n}{true}\PYG{p}{;}
\end{sphinxVerbatim}

\sphinxAtStartPar
\sphinxstyleemphasis{Example w2hparm.js to allow users to enable/disable the Tool Bar display}

\sphinxAtStartPar
If this option is enabled, the “Alt + Insert” and “Alt + Home” keys in the Key Mappings tab control the “Tool bar” display. The user can these key assignments to control the Tool Bar display.

\sphinxAtStartPar
\sphinxincludegraphics[scale=0.75]{{image2}.png}

\sphinxAtStartPar
\sphinxstyleemphasis{Key Mappings Setting Tab}

\index{Customization@\spxentry{Customization}!Tool Bar key assignment@\spxentry{Tool Bar key assignment}}\ignorespaces 
\sphinxAtStartPar
These two assignments can also be predefined in the w2hparm.js file by using the commands:

\begin{sphinxVerbatim}[commandchars=\\\{\}]
\PYG{o}{/}\PYG{o}{*}
\PYG{o}{*} \PYG{n}{Configuration} \PYG{n}{to} \PYG{n}{assign} \PYG{n}{Alt}\PYG{o}{+}\PYG{n}{Insert} \PYG{o+ow}{and}\PYG{o}{/}\PYG{o+ow}{or} \PYG{n}{Alt}\PYG{o}{+}\PYG{n}{Home} \PYG{n}{key} \PYG{n}{to} \PYG{n}{show}\PYG{o}{/}\PYG{n}{hide} \PYG{n}{Tool} \PYG{n}{Bar}\PYG{o}{.}
\PYG{o}{*}\PYG{o}{/}
\PYG{l+s+s2}{\PYGZdq{}}\PYG{l+s+s2}{altins}\PYG{l+s+s2}{\PYGZdq{}}\PYG{p}{:}\PYG{l+s+s2}{\PYGZdq{}}\PYG{l+s+s2}{ToolBar}\PYG{l+s+s2}{\PYGZdq{}} \PYG{o+ow}{and}\PYG{o}{/}\PYG{o+ow}{or}
\PYG{l+s+s2}{\PYGZdq{}}\PYG{l+s+s2}{althome}\PYG{l+s+s2}{\PYGZdq{}}\PYG{p}{:}\PYG{l+s+s2}{\PYGZdq{}}\PYG{l+s+s2}{ToolBar}\PYG{l+s+s2}{\PYGZdq{}}
\end{sphinxVerbatim}

\sphinxAtStartPar
\sphinxstyleemphasis{Example w2hparm.js to assign Alt+Insert and/or Alt+Home key to show/hide Tool Bar.}

\index{Customization@\spxentry{Customization}!Hide Tool Bar {[}User{]}@\spxentry{Hide Tool Bar {[}User{]}}}\index{Tool Bar@\spxentry{Tool Bar}!Enable and Disable Display {[}User{]}@\spxentry{Enable and Disable Display {[}User{]}}}\ignorespaces 
\sphinxAtStartPar
The Tool Bar can be automatically disabled at each new session opening by using the following command in w2hparm.js

\begin{sphinxVerbatim}[commandchars=\\\{\}]
\PYG{o}{/}\PYG{o}{*}
\PYG{o}{*} \PYG{n}{Configuration} \PYG{n}{to} \PYG{n}{automatically} \PYG{n}{hide} \PYG{n}{the} \PYG{n}{Tool} \PYG{n}{Bar} \PYG{n}{when} \PYG{n}{opening} \PYG{n}{a} \PYG{n}{new} \PYG{n}{session}\PYG{o}{.}
\PYG{o}{*}\PYG{o}{/}
\PYG{l+s+s2}{\PYGZdq{}}\PYG{l+s+s2}{hideTool Bar}\PYG{l+s+s2}{\PYGZdq{}}\PYG{p}{:}\PYG{l+s+s2}{\PYGZdq{}}\PYG{l+s+s2}{true}\PYG{l+s+s2}{\PYGZdq{}}
\end{sphinxVerbatim}

\sphinxAtStartPar
\sphinxstyleemphasis{Example w2hparm.js to automatically hide the Tool Bar when opening a new session.}

\sphinxAtStartPar
Under Virtel 4.57, those features require update \#5555 or above to be applied.

\index{Customization@\spxentry{Customization}!Hide Tool Bar Maintenance Level@\spxentry{Hide Tool Bar Maintenance Level}}\index{Tool Bar@\spxentry{Tool Bar}!Hide Maintenance Level@\spxentry{Hide Maintenance Level}}\ignorespaces 

\subsubsection{Enable / Disable Maintenance Level}
\label{\detokenize{Customization:enable-disable-maintenance-level}}\label{\detokenize{Customization:index-21}}
\sphinxAtStartPar
On the right side of the Tool Bar, the running version and the level of maintenance of VIRTEL is shown. This information is important in diagnosing problems. It can be hidden if the \sphinxstyleemphasis{w2hparm.switchToolBarDisplay = true;} option is set or the user selects to disable to Tool Bar.


\subsubsection{Changing background color of the Tool Bar buttons}
\label{\detokenize{Customization:changing-background-color-of-the-tool-bar-buttons}}
\sphinxAtStartPar
This example shows how to change the background color of the Tool Bar buttons by adding CSS orders in the custom.css file:

\begin{sphinxVerbatim}[commandchars=\\\{\}]
\PYG{o}{/}\PYG{o}{*}
\PYG{o}{*} \PYG{n}{VIRTEL} \PYG{n}{Web} \PYG{n}{Access} \PYG{n}{style} \PYG{n}{sheet} \PYG{n}{customization} \PYG{n}{the} \PYG{n}{background} \PYG{n}{of} \PYG{n}{the} \PYG{n}{Tool} \PYG{n}{Bar}
\PYG{o}{*} \PYG{n}{buttons}\PYG{p}{(}\PYG{n}{c}\PYG{p}{)}\PYG{n}{Copyright} \PYG{n}{SysperTec} \PYG{n}{Communication} \PYG{l+m+mi}{2014} \PYG{n}{All} \PYG{n}{Rights} \PYG{n}{Reserved}
\PYG{o}{*}\PYG{o}{/}
\PYG{o}{|}\PYG{o}{\PYGZhy{}} \PYG{n}{transparent} \PYG{l+s+s2}{\PYGZdq{}}\PYG{l+s+s2}{at rest}\PYG{l+s+s2}{\PYGZdq{}}
\PYG{o}{|}\PYG{o}{\PYGZhy{}} \PYG{n}{white} \PYG{n}{when} \PYG{n}{cursor} \PYG{n}{moves} \PYG{n}{on}
\PYG{o}{|}\PYG{o}{\PYGZhy{}} \PYG{n}{yellow} \PYG{n}{when} \PYG{n}{button} \PYG{o+ow}{is} \PYG{n}{clicked}
\PYG{c+c1}{\PYGZsh{}Tool Bar td .tbButton \PYGZob{}}
\PYG{n}{background}\PYG{o}{\PYGZhy{}}\PYG{n}{color}\PYG{p}{:} \PYG{n}{inherit}\PYG{p}{;}
\PYG{p}{\PYGZcb{}}
\PYG{c+c1}{\PYGZsh{}Tool Bar td .tbButton:hover \PYGZob{}}
\PYG{n}{background}\PYG{o}{\PYGZhy{}}\PYG{n}{color}\PYG{p}{:} \PYG{n}{white}\PYG{p}{;}
\PYG{p}{\PYGZcb{}}
\PYG{c+c1}{\PYGZsh{}Tool Bar td .tbButton:active \PYGZob{}}
\PYG{n}{background}\PYG{o}{\PYGZhy{}}\PYG{n}{color}\PYG{p}{:} \PYG{n}{yellow}\PYG{p}{;}
\PYG{p}{\PYGZcb{}}
    \PYG{o}{|} \PYG{n}{To} \PYG{n}{remove} \PYG{n}{the} \PYG{n}{background} \PYG{n}{color} \PYG{o+ow}{and} \PYG{n}{the} \PYG{n}{border} \PYG{n}{of} \PYG{n}{buttons} \PYG{l+s+s2}{\PYGZdq{}}\PYG{l+s+s2}{at rest}\PYG{l+s+s2}{\PYGZdq{}}\PYG{p}{:}
\PYG{c+c1}{\PYGZsh{}Tool Bar td .tbButton \PYGZob{}}
\PYG{n}{background}\PYG{o}{\PYGZhy{}}\PYG{n}{color}\PYG{p}{:} \PYG{n}{inherit}\PYG{p}{;}
\PYG{n}{border}\PYG{p}{:} \PYG{l+m+mi}{1}\PYG{n}{px} \PYG{n}{solid} \PYG{n}{transparent}\PYG{p}{;}
\PYG{p}{\PYGZcb{}}
\end{sphinxVerbatim}

\sphinxAtStartPar
\sphinxstyleemphasis{Example custom.css managing the background color of the Tool Bar buttons}

\index{Customization@\spxentry{Customization}!Customize the Tool Bar by application@\spxentry{Customize the Tool Bar by application}}\index{Tool Bar@\spxentry{Tool Bar}!Customize by color@\spxentry{Customize by color}}\ignorespaces 

\subsubsection{Customize the Tool Bar color by application}
\label{\detokenize{Customization:customize-the-tool-bar-color-by-application}}\label{\detokenize{Customization:index-22}}
\sphinxAtStartPar
It is sometimes useful for the user to have a clear visual indication of which system they are logged on to. This example shows how to set the color of the Tool Bar to yellow for SPCICSP and pink for SPCICSQ. Note that these names are the APPLID names that the transactions will connect to, they are not the transaction names.

\begin{sphinxVerbatim}[commandchars=\\\{\}]
\PYG{o}{/}\PYG{o}{*} \PYG{n}{VIRTEL} \PYG{n}{Web} \PYG{n}{Access} \PYG{n}{style} \PYG{n}{sheet} \PYG{k}{for} \PYG{n}{site} \PYG{n}{customization}
\PYG{o}{*} \PYG{p}{(}\PYG{n}{c}\PYG{p}{)}\PYG{n}{Copyright} \PYG{n}{SysperTec} \PYG{n}{Communication} \PYG{l+m+mi}{2007}\PYG{p}{,}\PYG{l+m+mi}{2010} \PYG{n}{All} \PYG{n}{Rights} \PYG{n}{Reserved}
\PYG{o}{*}\PYG{o}{/}

\PYG{o}{.}\PYG{n}{SPCICSP} \PYG{c+c1}{\PYGZsh{}toolbar \PYGZob{}background\PYGZhy{}color:yellow;\PYGZcb{}}
\PYG{o}{.}\PYG{n}{SPCICSQ} \PYG{c+c1}{\PYGZsh{}toolbar \PYGZob{}background\PYGZhy{}color:pink;\PYGZcb{}}
\end{sphinxVerbatim}

\sphinxAtStartPar
\sphinxstyleemphasis{Example custom.css for coloring the Tool Bar according to CICS region}

\sphinxAtStartPar
\sphinxincludegraphics{{image3}.png}
\sphinxstyleemphasis{Web Access screen with yellow Tool Bar for SPCICSP}

\sphinxAtStartPar
\sphinxincludegraphics{{image4}.png}
\sphinxstyleemphasis{Web Access screen with pink Tool Bar for SPCICSQ}

\index{Customization@\spxentry{Customization}!Add a Web Link in the Tool Bar@\spxentry{Add a Web Link in the Tool Bar}}\index{Tool Bar@\spxentry{Tool Bar}!Add web link@\spxentry{Add web link}}\ignorespaces 

\subsubsection{Add a Web Link in the Tool Bar}
\label{\detokenize{Customization:add-a-web-link-in-the-tool-bar}}\label{\detokenize{Customization:index-23}}
\sphinxAtStartPar
You can add a web link in the Tool Bar by using the following order included in a custom.js file:

\begin{sphinxVerbatim}[commandchars=\\\{\}]
\PYG{n}{function} \PYG{n}{after\PYGZus{}standardInit}\PYG{p}{(}\PYG{p}{)} \PYG{p}{\PYGZob{}}
    \PYG{o}{*} \PYG{n}{Adds} \PYG{n}{a} \PYG{n}{button} \PYG{n}{to} \PYG{n}{the} \PYG{n}{Tool} \PYG{n}{Bar} \PYG{n}{which} \PYG{n}{performs} \PYG{n}{a} \PYG{n}{Google} \PYG{n}{search}
    \PYG{n}{addTool} \PYG{n}{Barbutton}\PYG{p}{(}\PYG{n}{position}\PYG{p}{,} \PYG{l+s+s2}{\PYGZdq{}}\PYG{l+s+s2}{http://www.yourtargetsit.com/favicon.ico}\PYG{l+s+s2}{\PYGZdq{}}\PYG{p}{,}
    \PYG{l+s+s2}{\PYGZdq{}}\PYG{l+s+s2}{Title}\PYG{l+s+s2}{\PYGZdq{}}\PYG{p}{,} \PYG{n}{linked\PYGZus{}function}\PYG{p}{)}\PYG{p}{;}
\PYG{p}{\PYGZcb{}}
\end{sphinxVerbatim}

\sphinxAtStartPar
\sphinxstyleemphasis{Example to add a web link in the Tool Bar.}


\subsubsection{Example : Add a Google Search link into the Tool Bar}
\label{\detokenize{Customization:example-add-a-google-search-link-into-the-tool-bar}}
\begin{sphinxVerbatim}[commandchars=\\\{\}]
\PYG{n}{function} \PYG{n}{after\PYGZus{}standardInit}\PYG{p}{(}\PYG{p}{)} \PYG{p}{\PYGZob{}}
\PYG{o}{/}\PYG{o}{*}
\PYG{o}{*} \PYG{n}{Adds} \PYG{n}{a} \PYG{n}{button} \PYG{n}{to} \PYG{n}{the} \PYG{n}{Tool} \PYG{n}{Bar} \PYG{n}{which} \PYG{n}{performs} \PYG{n}{a} \PYG{n}{Google} \PYG{n}{search} \PYG{k}{for}
\PYG{o}{*} \PYG{n}{the} \PYG{n}{text} \PYG{n}{selected} \PYG{o+ow}{in} \PYG{n}{the} \PYG{n}{red} \PYG{n}{box} \PYG{o+ow}{in} \PYG{n}{the} \PYG{l+m+mi}{3270} \PYG{n}{screen}\PYG{p}{,} \PYG{o+ow}{or} \PYG{k}{for} \PYG{n}{the}
\PYG{o}{*} \PYG{n}{word} \PYG{n}{at} \PYG{n}{the} \PYG{n}{cursor} \PYG{k}{if} \PYG{n}{no} \PYG{n}{box} \PYG{o+ow}{is} \PYG{n}{drawn}
\PYG{o}{*}\PYG{o}{/}
\PYG{n}{addTool} \PYG{n}{Barbutton}\PYG{p}{(}\PYG{l+m+mi}{999}\PYG{p}{,} \PYG{l+s+s2}{\PYGZdq{}}\PYG{l+s+s2}{http://www.google.com/favicon.ico}\PYG{l+s+s2}{\PYGZdq{}}\PYG{p}{,}
\PYG{l+s+s2}{\PYGZdq{}}\PYG{l+s+s2}{Search engine query}\PYG{l+s+s2}{\PYGZdq{}}\PYG{p}{,} \PYG{n}{do\PYGZus{}search}\PYG{p}{)}\PYG{p}{;}
\PYG{p}{\PYGZcb{}}
\PYG{n}{function} \PYG{n}{do\PYGZus{}search}\PYG{p}{(}\PYG{p}{)} \PYG{p}{\PYGZob{}}
    \PYG{n}{var} \PYG{n}{searcharg} \PYG{o}{=} \PYG{n}{VIR3270}\PYG{o}{.}\PYG{n}{getBoxedText}\PYG{p}{(}\PYG{p}{)} \PYG{o}{|}\PYG{o}{|} \PYG{n}{VIR3270}\PYG{o}{.}\PYG{n}{getWordAtCursor}\PYG{p}{(}\PYG{p}{)}\PYG{p}{;}
    \PYG{n}{var} \PYG{n}{windowname} \PYG{o}{=} \PYG{l+s+s2}{\PYGZdq{}}\PYG{l+s+s2}{search}\PYG{l+s+s2}{\PYGZdq{}}\PYG{p}{;}
    \PYG{n}{var} \PYG{n}{searchURL} \PYG{o}{=} \PYG{l+s+s2}{\PYGZdq{}}\PYG{l+s+s2}{http://www.google.com}\PYG{l+s+s2}{\PYGZdq{}}\PYG{p}{;}
    \PYG{k}{if} \PYG{p}{(}\PYG{n}{searcharg}\PYG{p}{)} \PYG{n}{searchURL} \PYG{o}{+}\PYG{o}{=} \PYG{l+s+s2}{\PYGZdq{}}\PYG{l+s+s2}{/search?q=}\PYG{l+s+s2}{\PYGZdq{}} \PYG{o}{+}
        \PYG{n}{encodeURIComponent}\PYG{p}{(}\PYG{n}{searcharg}\PYG{o}{.}\PYG{n}{replace}\PYG{p}{(}\PYG{o}{/}\PYGZbs{}\PYG{n}{s}\PYG{o}{+}\PYG{o}{/}\PYG{n}{g}\PYG{p}{,} \PYG{l+s+s2}{\PYGZdq{}}\PYG{l+s+s2}{\PYGZdq{}}\PYG{p}{)}\PYG{p}{)}\PYG{p}{;}
    \PYG{n}{var} \PYG{n}{windowopts} \PYG{o}{=} \PYG{l+s+s2}{\PYGZdq{}}\PYG{l+s+s2}{location=yes,status=yes,resizable=yes,}\PYG{l+s+s2}{\PYGZdq{}} \PYG{o}{+}
    \PYG{l+s+s2}{\PYGZdq{}}\PYG{l+s+s2}{scrollbars=yes,Tool Bar=yes,menubar=yes,width=640,height=480}\PYG{l+s+s2}{\PYGZdq{}}\PYG{p}{;}
    \PYG{n}{var} \PYG{n}{searchwin} \PYG{o}{=} \PYG{n}{window}\PYG{o}{.}\PYG{n}{open}\PYG{p}{(}\PYG{n}{searchURL}\PYG{p}{,} \PYG{n}{windowname}\PYG{p}{,} \PYG{n}{windowopts}\PYG{p}{)}\PYG{p}{;}
    \PYG{k}{if} \PYG{p}{(}\PYG{n}{searchwin}\PYG{p}{)} \PYG{n}{searchwin}\PYG{o}{.}\PYG{n}{focus}\PYG{p}{(}\PYG{p}{)}\PYG{p}{;}
\PYG{p}{\PYGZcb{}}
\end{sphinxVerbatim}

\index{Customization@\spxentry{Customization}!Add a Company Logo into the Tool Bar@\spxentry{Add a Company Logo into the Tool Bar}}\index{Tool Bar@\spxentry{Tool Bar}!Add Company Logo@\spxentry{Add Company Logo}}\ignorespaces 

\subsubsection{Adding a Company Logo}
\label{\detokenize{Customization:adding-a-company-logo}}\label{\detokenize{Customization:index-24}}
\sphinxAtStartPar
This example shows how to display an icon (for example, a company logo) at the left of the Tool Bar:

\begin{sphinxVerbatim}[commandchars=\\\{\}]
\PYG{o}{/}\PYG{o}{*}
\PYG{o}{*} \PYG{n}{VIRTEL} \PYG{n}{Web} \PYG{n}{Access} \PYG{n}{style} \PYG{n}{sheet} \PYG{n}{customisation} \PYG{k}{for} \PYG{n}{company} \PYG{n}{logo}
\PYG{o}{*} \PYG{p}{(}\PYG{n}{c}\PYG{p}{)}\PYG{n}{Copyright} \PYG{n}{SysperTec} \PYG{n}{Communication} \PYG{l+m+mi}{2012} \PYG{n}{All} \PYG{n}{Rights} \PYG{n}{Reserved}
\PYG{o}{*}\PYG{o}{/}
\PYG{c+c1}{\PYGZsh{}Tool Bar td\PYGZsh{}companyIcon \PYGZob{}}
\PYG{n}{height}\PYG{p}{:}\PYG{l+m+mi}{30}\PYG{n}{px}\PYG{p}{;}
\PYG{n}{display}\PYG{p}{:}\PYG{n}{table}\PYG{o}{\PYGZhy{}}\PYG{n}{cell}\PYG{p}{;}
\PYG{p}{\PYGZcb{}}

\PYG{c+c1}{\PYGZsh{}companyIcon div \PYGZob{}}
    \PYG{n}{background}\PYG{o}{\PYGZhy{}}\PYG{n}{image}\PYG{p}{:}\PYG{n}{url}\PYG{p}{(}\PYG{l+s+s2}{\PYGZdq{}}\PYG{l+s+s2}{/w2h/virtblue.jpg}\PYG{l+s+s2}{\PYGZdq{}}\PYG{p}{)}\PYG{p}{;}
    \PYG{n}{background}\PYG{o}{\PYGZhy{}}\PYG{n}{position}\PYG{p}{:}\PYG{l+m+mi}{0}\PYG{n}{px} \PYG{o}{\PYGZhy{}}\PYG{l+m+mi}{4}\PYG{n}{px}\PYG{p}{;}
    \PYG{n}{background}\PYG{o}{\PYGZhy{}}\PYG{n}{repeat}\PYG{p}{:}\PYG{n}{no}\PYG{o}{\PYGZhy{}}\PYG{n}{repeat}\PYG{p}{;}
    \PYG{n}{height}\PYG{p}{:}\PYG{l+m+mi}{26}\PYG{n}{px}\PYG{p}{;}
    \PYG{n}{width}\PYG{p}{:}\PYG{l+m+mi}{145}\PYG{n}{px}\PYG{p}{;}
\PYG{p}{\PYGZcb{}}
\end{sphinxVerbatim}

\sphinxAtStartPar
\sphinxstyleemphasis{Example custom.css for displaying company logo in the Tool Bar}

\index{Customization@\spxentry{Customization}!Replace the Virtel Logo in the Tool Bar@\spxentry{Replace the Virtel Logo in the Tool Bar}}\index{Tool Bar@\spxentry{Tool Bar}!Replace Virtel Logo@\spxentry{Replace Virtel Logo}}\ignorespaces 
\sphinxAtStartPar
This example shows how to replace the Virtel logo in the VIRTEL Web Access menu and the Application menu by your company logo:

\begin{sphinxVerbatim}[commandchars=\\\{\}]
/*
* VIRTEL Web Access style sheet for site customization
* (c)Copyright SysperTec Communication 2013 All Rights Reserved
* \PYGZdl{}Id\PYGZdl{}
*/
\PYGZsh{}appmenulogo \PYGZob{}
    background\PYGZhy{}image: url(\PYGZdq{}mycompany.gif\PYGZdq{});
    height: 65px;
    width: 266px;
\PYGZcb{}
\end{sphinxVerbatim}

\sphinxAtStartPar
\sphinxstyleemphasis{Example custom.css for replacing the Virtel logo by a company logo}

\begin{sphinxadmonition}{note}{Note:}
\sphinxAtStartPar
If no explicit path is given, the company logo will be loaded from the same directory as the custom.css file.
\end{sphinxadmonition}


\subsection{Tool Bar Icons}
\label{\detokenize{Customization:tool-bar-icons}}
\index{Customization@\spxentry{Customization}!Adding Tool Bar Icons@\spxentry{Adding Tool Bar Icons}}\index{Tool Bar@\spxentry{Tool Bar}!Add Icons@\spxentry{Add Icons}}\ignorespaces 

\subsubsection{Adding a Tool Bar icon}
\label{\detokenize{Customization:adding-a-tool-bar-icon}}\label{\detokenize{Customization:index-26}}
\sphinxAtStartPar
This example uses the after\_standardInit function to insert additional icons into the Tool Bar when the session is started. Icons may subsequently be added or removed from the Tool Bar after each screen by means of the after\_responseHandle function.

\begin{sphinxVerbatim}[commandchars=\\\{\}]
\PYG{o}{/}\PYG{o}{*}
\PYG{o}{*} \PYG{p}{(}\PYG{n}{c}\PYG{p}{)}\PYG{n}{Copyright} \PYG{n}{SysperTec} \PYG{n}{Communication} \PYG{l+m+mi}{2012} \PYG{n}{All} \PYG{n}{Rights} \PYG{n}{Reserved}
\PYG{o}{*} \PYG{n}{VIRTEL} \PYG{n}{Web} \PYG{n}{Access} \PYG{n}{customer}\PYG{o}{\PYGZhy{}}\PYG{n}{specific} \PYG{n}{javascript} \PYG{n}{functions}
\PYG{o}{*}\PYG{o}{/}
\PYG{o}{/}\PYG{o}{*}
\PYG{o}{*} \PYG{n}{Adds} \PYG{n}{a} \PYG{n}{button} \PYG{n}{to} \PYG{n}{the} \PYG{n}{Tool} \PYG{n}{Bar} \PYG{n}{which} \PYG{n}{performs} \PYG{n}{a} \PYG{n}{Google} \PYG{n}{search} \PYG{k}{for}
\PYG{o}{*} \PYG{n}{the} \PYG{n}{text} \PYG{n}{selected} \PYG{o+ow}{in} \PYG{n}{the} \PYG{n}{red} \PYG{n}{box} \PYG{o+ow}{in} \PYG{n}{the} \PYG{l+m+mi}{3270} \PYG{n}{screen}\PYG{p}{,} \PYG{o+ow}{or} \PYG{k}{for} \PYG{n}{the}
\PYG{o}{*} \PYG{n}{word} \PYG{n}{at} \PYG{n}{the} \PYG{n}{cursor} \PYG{k}{if} \PYG{n}{no} \PYG{n}{box} \PYG{o+ow}{is} \PYG{n}{drawn}
\PYG{o}{*} \PYG{n}{after\PYGZus{}standardInit}\PYG{p}{(}\PYG{p}{)} \PYG{n}{function} \PYG{n}{must} \PYG{n}{be} \PYG{n}{implemented} \PYG{o+ow}{in} \PYG{n}{a} \PYG{n}{custom}\PYG{o}{.}\PYG{n}{js} \PYG{n}{file}
\PYG{o}{*}\PYG{o}{/}

\PYG{n}{function} \PYG{n}{after\PYGZus{}standardInit}\PYG{p}{(}\PYG{p}{)} \PYG{p}{\PYGZob{}}
    \PYG{n}{addTool} \PYG{n}{Barbutton}\PYG{p}{(}\PYG{l+m+mi}{999}\PYG{p}{,} \PYG{l+s+s2}{\PYGZdq{}}\PYG{l+s+s2}{http://www.google.com/favicon.ico}\PYG{l+s+s2}{\PYGZdq{}}\PYG{p}{,}
    \PYG{l+s+s2}{\PYGZdq{}}\PYG{l+s+s2}{Search engine query}\PYG{l+s+s2}{\PYGZdq{}}\PYG{p}{,} \PYG{n}{do\PYGZus{}search}\PYG{p}{)}\PYG{p}{;}
\PYG{p}{\PYGZcb{}}

\PYG{n}{function} \PYG{n}{do\PYGZus{}search}\PYG{p}{(}\PYG{p}{)} \PYG{p}{\PYGZob{}}
    \PYG{n}{var} \PYG{n}{searcharg} \PYG{o}{=} \PYG{n}{VIR3270}\PYG{o}{.}\PYG{n}{getBoxedText}\PYG{p}{(}\PYG{p}{)} \PYG{o}{|}\PYG{o}{|} \PYG{n}{VIR3270}\PYG{o}{.}\PYG{n}{getWordAtCursor}\PYG{p}{(}\PYG{p}{)}\PYG{p}{;}
    \PYG{n}{var} \PYG{n}{windowname} \PYG{o}{=} \PYG{l+s+s2}{\PYGZdq{}}\PYG{l+s+s2}{search}\PYG{l+s+s2}{\PYGZdq{}}\PYG{p}{;}
    \PYG{n}{var} \PYG{n}{searchURL} \PYG{o}{=} \PYG{l+s+s2}{\PYGZdq{}}\PYG{l+s+s2}{http://www.google.com}\PYG{l+s+s2}{\PYGZdq{}}\PYG{p}{;}
    \PYG{k}{if} \PYG{p}{(}\PYG{n}{searcharg}\PYG{p}{)} \PYG{n}{searchURL} \PYG{o}{+}\PYG{o}{=} \PYG{l+s+s2}{\PYGZdq{}}\PYG{l+s+s2}{/search?q=}\PYG{l+s+s2}{\PYGZdq{}} \PYG{o}{+}
        \PYG{n}{encodeURIComponent}\PYG{p}{(}\PYG{n}{searcharg}\PYG{o}{.}\PYG{n}{replace}\PYG{p}{(}\PYG{o}{/}\PYGZbs{}\PYG{n}{s}\PYG{o}{+}\PYG{o}{/}\PYG{n}{g}\PYG{p}{,}\PYG{l+s+s2}{\PYGZdq{}}\PYG{l+s+s2}{ }\PYG{l+s+s2}{\PYGZdq{}}\PYG{p}{)}\PYG{p}{)}\PYG{p}{;}
    \PYG{n}{var} \PYG{n}{windowopts} \PYG{o}{=} \PYG{l+s+s2}{\PYGZdq{}}\PYG{l+s+s2}{location=yes,status=yes,resizable=yes,}\PYG{l+s+s2}{\PYGZdq{}}\PYG{o}{+}
    \PYG{l+s+s2}{\PYGZdq{}}\PYG{l+s+s2}{scrollbars=yes,Tool Bar=yes,menubar=yes,width=640,height=480}\PYG{l+s+s2}{\PYGZdq{}}\PYG{p}{;}
    \PYG{n}{var} \PYG{n}{searchwin} \PYG{o}{=} \PYG{n}{window}\PYG{o}{.}\PYG{n}{open}\PYG{p}{(}\PYG{n}{searchURL}\PYG{p}{,} \PYG{n}{windowname}\PYG{p}{,} \PYG{n}{windowopts}\PYG{p}{)}\PYG{p}{;}
    \PYG{k}{if} \PYG{p}{(}\PYG{n}{searchwin}\PYG{p}{)} \PYG{n}{searchwin}\PYG{o}{.}\PYG{n}{focus}\PYG{p}{(}\PYG{p}{)}\PYG{p}{;}
\PYG{p}{\PYGZcb{}}
\end{sphinxVerbatim}

\sphinxAtStartPar
\sphinxstyleemphasis{Example custom.js to customize the Tool Bar icons}

\index{Customization@\spxentry{Customization}!Removing Tool Bar Icons@\spxentry{Removing Tool Bar Icons}}\index{Tool Bar@\spxentry{Tool Bar}!Remove Icons@\spxentry{Remove Icons}}\ignorespaces 

\subsubsection{Removing unwanted Tool Bar icons}
\label{\detokenize{Customization:removing-unwanted-tool-bar-icons}}\label{\detokenize{Customization:index-27}}
\sphinxAtStartPar
This example uses the after\_standardInit function to disable macro functions by removing the corresponding icons from the Tool Bar.

\begin{sphinxVerbatim}[commandchars=\\\{\}]
\PYG{o}{/}\PYG{o}{*}
\PYG{o}{*} \PYG{p}{(}\PYG{n}{c}\PYG{p}{)}\PYG{n}{Copyright} \PYG{n}{SysperTec} \PYG{n}{Communication} \PYG{l+m+mi}{2017} \PYG{n}{All} \PYG{n}{Rights} \PYG{n}{Reserved}
\PYG{o}{*} \PYG{n}{VIRTEL} \PYG{n}{Web} \PYG{n}{Access} \PYG{n}{customer}\PYG{o}{\PYGZhy{}}\PYG{n}{specific} \PYG{n}{javascript} \PYG{n}{functions}
\PYG{o}{*} \PYG{n}{after\PYGZus{}standardInit}\PYG{p}{(}\PYG{p}{)} \PYG{n}{function} \PYG{n}{must} \PYG{n}{be} \PYG{n}{implemented} \PYG{o+ow}{in} \PYG{n}{a} \PYG{n}{custom}\PYG{o}{.}\PYG{n}{js} \PYG{n}{file}
\PYG{o}{*}\PYG{o}{/}
\PYG{n}{function} \PYG{n}{after\PYGZus{}standardInit}\PYG{p}{(}\PYG{p}{)} \PYG{p}{\PYGZob{}}
    \PYG{o}{/}\PYG{o}{*} \PYG{n}{Remove} \PYG{n}{macro} \PYG{n}{buttons} \PYG{k+kn}{from} \PYG{n+nn}{the} \PYG{n}{Tool} \PYG{n}{Bar} \PYG{o}{*}\PYG{o}{/}
    \PYG{n}{removetoolbarbutton}\PYG{p}{(}\PYG{l+s+s2}{\PYGZdq{}}\PYG{l+s+s2}{startrecording}\PYG{l+s+s2}{\PYGZdq{}}\PYG{p}{)}\PYG{p}{;}
    \PYG{n}{removetoolbarbutton}\PYG{p}{(}\PYG{l+s+s2}{\PYGZdq{}}\PYG{l+s+s2}{playback}\PYG{l+s+s2}{\PYGZdq{}}\PYG{p}{)}\PYG{p}{;}
\PYG{p}{\PYGZcb{}}
\end{sphinxVerbatim}

\sphinxAtStartPar
\sphinxstyleemphasis{Example custom.js to remove selected Tool Bar icons}

\sphinxAtStartPar
The names of the other Tool Bar icons which can be removed in this way are:
\begin{itemize}
\item {} 
\sphinxAtStartPar
capture,

\item {} 
\sphinxAtStartPar
disconnect,

\item {} 
\sphinxAtStartPar
document\sphinxhyphen{}print\sphinxhyphen{}preview,

\item {} 
\sphinxAtStartPar
edit\sphinxhyphen{}copy,

\item {} 
\sphinxAtStartPar
edit\sphinxhyphen{}cut,

\item {} 
\sphinxAtStartPar
edit\sphinxhyphen{}paste,

\item {} 
\sphinxAtStartPar
emptybuf,

\item {} 
\sphinxAtStartPar
file\sphinxhyphen{}rcv,

\item {} 
\sphinxAtStartPar
file\sphinxhyphen{}send,

\item {} 
\sphinxAtStartPar
help,

\item {} 
\sphinxAtStartPar
keypad,

\item {} 
\sphinxAtStartPar
playback,

\item {} 
\sphinxAtStartPar
printer,

\item {} 
\sphinxAtStartPar
settings,

\item {} 
\sphinxAtStartPar
settingsV2,

\item {} 
\sphinxAtStartPar
startrecording.

\end{itemize}

\begin{sphinxadmonition}{note}{Note:}
\sphinxAtStartPar
(settingsV2 is présent only if w2hparms.js contains “settingsGUI”:\{“version”:”V2”\}, or “settingsGUI”:\{“version”:”V1+V2”\},)
\end{sphinxadmonition}

\index{Removing 3D/Hover effects from Tool Bar buttons@\spxentry{Removing 3D/Hover effects from Tool Bar buttons}}\index{Tool Bar@\spxentry{Tool Bar}!Remove Button effects@\spxentry{Remove Button effects}}\ignorespaces 
\index{Customization@\spxentry{Customization}!File Transfer Icons@\spxentry{File Transfer Icons}}\index{Tool Bar@\spxentry{Tool Bar}!File Transfer Icons@\spxentry{File Transfer Icons}}\ignorespaces 

\subsection{File Transfer icons}
\label{\detokenize{Customization:file-transfer-icons}}\label{\detokenize{Customization:index-29}}
\sphinxAtStartPar
The File Transfer icons are present in the Tool Bar only for access to TSO and if the transaction includes an INPUT and/or OUTPUT scenario with a call to the INDSCEN\$ macro (See “TSO File Transfer” in “Virtel Web Access User Guide”).

\index{Customization@\spxentry{Customization}!Removing File Transfer Icons@\spxentry{Removing File Transfer Icons}}\index{Tool Bar@\spxentry{Tool Bar}!Remove File Transfer Icons@\spxentry{Remove File Transfer Icons}}\ignorespaces 

\subsubsection{Removing File Transfer Icons}
\label{\detokenize{Customization:removing-file-transfer-icons}}\label{\detokenize{Customization:index-30}}
\sphinxAtStartPar
VIRTEL Web Access supports transfer of files between the browser and a TSO session using the IND\$FILE protocol. The function is activated displayed only for the transaction defined with the SCENINDT scenario entered in Input and Output scenario fields. For some specific users, it may be necessary to remove the file transfer icons from the Tool Bar.
This can be done by using the removetoolbarbutton function using the appropriate variable name.(see “Removing unwanted Tool Bar icons”, page 11.)

\index{Customization@\spxentry{Customization}!Macro Icons@\spxentry{Macro Icons}}\index{Tool Bar@\spxentry{Tool Bar}!Macro Icons@\spxentry{Macro Icons}}\ignorespaces 

\subsection{Macro Icons}
\label{\detokenize{Customization:macro-icons}}\label{\detokenize{Customization:index-31}}
\index{Customization@\spxentry{Customization}!Removing Macro Icons@\spxentry{Removing Macro Icons}}\index{Tool Bar@\spxentry{Tool Bar}!Remove Macro Icons@\spxentry{Remove Macro Icons}}\ignorespaces 

\subsubsection{Removing Macro Icons}
\label{\detokenize{Customization:removing-macro-icons}}\label{\detokenize{Customization:index-32}}
\sphinxAtStartPar
See “Removing unwanted Tool Bar icons”, page 11.

\sphinxAtStartPar
To hide the Tool Bar completely, see “Hiding the Tool Bar” on page 6. To hide only the Virtel Application name, see “Showing / Hiding server information”, on page 15.


\subsubsection{Positioning Tool Bar Icons}
\label{\detokenize{Customization:positioning-tool-bar-icons}}
\sphinxAtStartPar
In certain circumstances, the default position of an icon may not be at the user’s convenience. Is possible to change an icon’s position based on the position of another icon.

\begin{sphinxVerbatim}[commandchars=\\\{\}]
\PYG{o}{/}\PYG{o}{*}
\PYG{o}{*} \PYG{n}{Customize} \PYG{n}{the} \PYG{n}{location} \PYG{n}{of} \PYG{n}{dynamic} \PYG{n}{Tool} \PYG{n}{Bar} \PYG{n}{buttons}\PYG{o}{.}
\PYG{o}{*} \PYG{n}{The} \PYG{n}{calls} \PYG{n}{to} \PYG{n}{this} \PYG{n}{function} \PYG{n}{are} \PYG{n}{ignored} \PYG{n}{when} \PYG{n}{they}
\PYG{o}{*} \PYG{k}{return} \PYG{n}{nothing}\PYG{p}{,} \PYG{o+ow}{or} \PYG{n}{an} \PYG{n}{integer} \PYG{o+ow}{not} \PYG{n}{greater} \PYG{n}{than} \PYG{l+m+mf}{0.}
\PYG{o}{*}
\PYG{o}{*} \PYG{n}{Customizable} \PYG{n}{buttons} \PYG{n}{IDs} \PYG{p}{:}
\PYG{o}{*} \PYG{o}{\PYGZgt{}} \PYG{l+s+s1}{\PYGZsq{}}\PYG{l+s+s1}{3278T}\PYG{l+s+s1}{\PYGZsq{}}
\PYG{o}{*} \PYG{o}{\PYGZgt{}} \PYG{l+s+s1}{\PYGZsq{}}\PYG{l+s+s1}{document\PYGZhy{}print\PYGZhy{}preview}\PYG{l+s+s1}{\PYGZsq{}}
\PYG{o}{*} \PYG{o}{\PYGZgt{}} \PYG{l+s+s1}{\PYGZsq{}}\PYG{l+s+s1}{file\PYGZhy{}send}\PYG{l+s+s1}{\PYGZsq{}}
\PYG{o}{*} \PYG{o}{\PYGZgt{}} \PYG{l+s+s1}{\PYGZsq{}}\PYG{l+s+s1}{file\PYGZhy{}recv}\PYG{l+s+s1}{\PYGZsq{}}
\PYG{o}{*} \PYG{o}{\PYGZgt{}} \PYG{l+s+s1}{\PYGZsq{}}\PYG{l+s+s1}{printer}\PYG{l+s+s1}{\PYGZsq{}}
\PYG{o}{*}\PYG{o}{/}
\PYG{n}{function} \PYG{n}{customize\PYGZus{}Tool} \PYG{n}{BarButtonIndex}\PYG{p}{(}\PYG{n+nb}{id}\PYG{p}{)} \PYG{p}{\PYGZob{}}
    \PYG{k}{if} \PYG{p}{(}\PYG{n+nb}{id}\PYG{o}{==}\PYG{o}{=}\PYG{l+s+s1}{\PYGZsq{}}\PYG{l+s+s1}{printer}\PYG{l+s+s1}{\PYGZsq{}} \PYG{o}{|}\PYG{o}{|} \PYG{n+nb}{id}\PYG{o}{==}\PYG{o}{=}\PYG{l+s+s1}{\PYGZsq{}}\PYG{l+s+s1}{document\PYGZhy{}print\PYGZhy{}preview}\PYG{l+s+s1}{\PYGZsq{}}\PYG{p}{)} \PYG{p}{\PYGZob{}}
    \PYG{k}{return} \PYG{n}{getToolbarButtonIndex}\PYG{p}{(}\PYG{l+s+s1}{\PYGZsq{}}\PYG{l+s+s1}{disconnect}\PYG{l+s+s1}{\PYGZsq{}}\PYG{p}{)} \PYG{o}{+} \PYG{l+m+mi}{1}\PYG{p}{;}
    \PYG{p}{\PYGZcb{}}
\PYG{p}{\PYGZcb{}}
\end{sphinxVerbatim}

\sphinxAtStartPar
\sphinxstyleemphasis{Example custom.js to select a position for printer icon}

\index{Customization@\spxentry{Customization}!Center User Icons@\spxentry{Center User Icons}}\index{Tool Bar@\spxentry{Tool Bar}!Center User Icons@\spxentry{Center User Icons}}\ignorespaces 

\subsubsection{Centring User Icons}
\label{\detokenize{Customization:centring-user-icons}}\label{\detokenize{Customization:index-33}}
\sphinxAtStartPar
The recommended size for an icon is 32x32 pixels. For bigger or smaller icons, it possible to offer better center rendering by modifying the content of the CSS class attribute passed within the “addTool Barbutton” function.

\begin{sphinxVerbatim}[commandchars=\\\{\}]
\PYG{o}{/}\PYG{o}{*}
\PYG{o}{*} \PYG{p}{(}\PYG{n}{c}\PYG{p}{)}\PYG{n}{Copyright} \PYG{n}{SysperTec} \PYG{n}{Communication} \PYG{l+m+mi}{2014} \PYG{n}{All} \PYG{n}{Rights} \PYG{n}{Reserved}
\PYG{o}{*} \PYG{n}{VIRTEL} \PYG{n}{Web} \PYG{n}{Access} \PYG{n}{customer}\PYG{o}{\PYGZhy{}}\PYG{n}{specific} \PYG{n}{javascript} \PYG{n}{functions}
\PYG{o}{*} \PYG{n}{Resizing} \PYG{n}{a} \PYG{n}{too} \PYG{n}{small} \PYG{o+ow}{or} \PYG{n}{too} \PYG{n}{big} \PYG{n}{Tool} \PYG{n}{Bar} \PYG{n}{icon}\PYG{o}{.}
\PYG{o}{*} \PYG{n}{For} \PYG{n}{example} \PYG{n}{toosmall\PYGZus{}pic}\PYG{o}{.}\PYG{n}{png}\PYG{o}{=}\PYG{l+m+mi}{22}\PYG{n}{x22} \PYG{o+ow}{and} \PYG{n}{toobig\PYGZus{}pic}\PYG{o}{.}\PYG{n}{jpg}\PYG{o}{=}\PYG{l+m+mi}{145}\PYG{n}{x30}
\PYG{o}{*}\PYG{o}{/}
\PYG{n}{addTool} \PYG{n}{Barbutton}\PYG{p}{(}\PYG{l+m+mi}{999}\PYG{p}{,} \PYG{l+s+s2}{\PYGZdq{}}\PYG{l+s+s2}{/w2h/toosmall\PYGZus{}pic.png}\PYG{l+s+s2}{\PYGZdq{}}\PYG{p}{,} \PYG{l+s+s2}{\PYGZdq{}}\PYG{l+s+s2}{Custom button \PYGZsh{}1 tooltip}\PYG{l+s+s2}{\PYGZdq{}}\PYG{p}{,}
    \PYG{n}{do\PYGZus{}search}\PYG{p}{,} \PYG{l+s+s2}{\PYGZdq{}}\PYG{l+s+s2}{tbButton size22x22}\PYG{l+s+s2}{\PYGZdq{}}\PYG{p}{)}\PYG{p}{;}
\PYG{n}{addTool} \PYG{n}{Barbutton}\PYG{p}{(}\PYG{l+m+mi}{999}\PYG{p}{,} \PYG{l+s+s2}{\PYGZdq{}}\PYG{l+s+s2}{/w2h/toobig\PYGZus{}pic.jpg}\PYG{l+s+s2}{\PYGZdq{}}\PYG{p}{,} \PYG{l+s+s2}{\PYGZdq{}}\PYG{l+s+s2}{Custom button \PYGZsh{}2 tooltip}\PYG{l+s+s2}{\PYGZdq{}}\PYG{p}{,}
    \PYG{n}{do\PYGZus{}search}\PYG{p}{,} \PYG{l+s+s2}{\PYGZdq{}}\PYG{l+s+s2}{tbButton size145x30}\PYG{l+s+s2}{\PYGZdq{}}\PYG{p}{)}\PYG{p}{;}
\end{sphinxVerbatim}

\sphinxAtStartPar
\sphinxstyleemphasis{Example custom.js to specify the Tool Bar icon size}

\begin{sphinxVerbatim}[commandchars=\\\{\}]
\PYG{o}{/}\PYG{o}{*}
\PYG{c+c1}{\PYGZsh{}Tool Bar td img.tbButton.size22x22 \PYGZob{}}
    \PYG{n}{width}\PYG{p}{:} \PYG{l+m+mi}{22}\PYG{n}{px}\PYG{p}{;}
    \PYG{n}{height}\PYG{p}{:} \PYG{l+m+mi}{22}\PYG{n}{px}\PYG{p}{;}
    \PYG{n}{padding}\PYG{p}{:} \PYG{l+m+mi}{5}\PYG{n}{px}\PYG{p}{;} \PYG{o}{/}\PYG{o}{*} \PYG{n}{padding} \PYG{o+ow}{is} \PYG{n}{calculated} \PYG{n}{to} \PYG{n}{center} \PYG{n}{the} \PYG{n}{picture} \PYG{n}{horizontaly}
    \PYG{o+ow}{and} \PYG{n}{verticaly} \PYG{o+ow}{in} \PYG{n}{the} \PYG{l+m+mi}{32}\PYG{n}{x32} \PYG{n}{allocated} \PYG{n}{area}\PYG{o}{.} \PYG{p}{(}\PYG{l+m+mi}{5}\PYG{o}{+}\PYG{l+m+mi}{22}\PYG{o}{+}\PYG{l+m+mi}{5} \PYG{o}{=} \PYG{l+m+mi}{32}\PYG{p}{)} \PYG{o}{*}\PYG{o}{/}
\PYG{p}{\PYGZcb{}}

\PYG{c+c1}{\PYGZsh{}Tool Bar td img.tbButton.size145x30 \PYGZob{}}
    \PYG{n}{width}\PYG{p}{:} \PYG{l+m+mi}{145}\PYG{n}{px}\PYG{p}{;}
    \PYG{n}{height}\PYG{p}{:} \PYG{l+m+mi}{30}\PYG{n}{px}\PYG{p}{;}
    \PYG{n}{padding}\PYG{p}{:} \PYG{l+m+mi}{1}\PYG{n}{px} \PYG{l+m+mi}{0}\PYG{p}{;} \PYG{o}{/}\PYG{o}{*} \PYG{n}{padding} \PYG{o+ow}{is} \PYG{n}{calculated} \PYG{n}{to} \PYG{n}{center} \PYG{n}{the} \PYG{n}{picture} \PYG{n}{verticaly}
    \PYG{o+ow}{in} \PYG{n}{the} \PYG{l+m+mi}{32}\PYG{n}{x32} \PYG{n}{allocated} \PYG{n}{area} \PYG{p}{(}\PYG{l+m+mi}{1}\PYG{o}{+}\PYG{l+m+mi}{30}\PYG{o}{+}\PYG{l+m+mi}{1} \PYG{o}{=} \PYG{l+m+mi}{32}\PYG{p}{)} \PYG{n}{without} \PYG{n+nb}{any} \PYG{n}{horizontaly} \PYG{n}{padding} \PYG{o}{*}\PYG{o}{/}
\PYG{p}{\PYGZcb{}}
\end{sphinxVerbatim}

\sphinxAtStartPar
\sphinxstyleemphasis{Example custom.css to manage a Tool Bar icon with a non standard size}

\index{Customization@\spxentry{Customization}!Removing 3D/Hover effects from Tool Bar buttons@\spxentry{Removing 3D/Hover effects from Tool Bar buttons}}\index{Tool Bar@\spxentry{Tool Bar}!Customize effects@\spxentry{Customize effects}}\ignorespaces 

\subsubsection{Removing 3D/hover effects on the Tool Bar buttons}
\label{\detokenize{Customization:removing-3d-hover-effects-on-the-tool-bar-buttons}}\label{\detokenize{Customization:index-34}}
\sphinxAtStartPar
This example shows how to remove the 3D/hover effects on Tool Bar buttons by adding orders in a Custom.CSS file:

\begin{sphinxVerbatim}[commandchars=\\\{\}]
\PYG{o}{/}\PYG{o}{*}
\PYG{o}{*} \PYG{n}{VIRTEL} \PYG{n}{Web} \PYG{n}{Access} \PYG{n}{style} \PYG{n}{sheet} \PYG{n}{customisation} \PYG{k}{for} \PYG{n}{removing} \PYG{l+m+mi}{3}\PYG{n}{D}\PYG{o}{/}\PYG{n}{hover} \PYG{n}{effects}
\PYG{o}{*} \PYG{p}{(}\PYG{n}{c}\PYG{p}{)}\PYG{n}{Copyright} \PYG{n}{SysperTec} \PYG{n}{Communication} \PYG{l+m+mi}{2014} \PYG{n}{All} \PYG{n}{Rights} \PYG{n}{Reserved}
\PYG{o}{*}\PYG{o}{/}
\PYG{c+c1}{\PYGZsh{}Tool Bar td .tbButton,}
\PYG{c+c1}{\PYGZsh{}Tool Bar td .tbButton:hover,}
\PYG{c+c1}{\PYGZsh{}Tool Bar td .tbButton:active \PYGZob{}}
    \PYG{n}{background}\PYG{o}{\PYGZhy{}}\PYG{n}{color}\PYG{p}{:} \PYG{n}{inherit}\PYG{p}{;}
    \PYG{n}{border}\PYG{p}{:} \PYG{n}{inherit}\PYG{p}{;}
    \PYG{n}{box}\PYG{o}{\PYGZhy{}}\PYG{n}{shadow}\PYG{p}{:} \PYG{n}{inherit}\PYG{p}{;}
\PYG{p}{\PYGZcb{}}
\end{sphinxVerbatim}

\sphinxAtStartPar
\sphinxstyleemphasis{Example custom.css for removing 3D/hover effects on buttons}

\index{Customization@\spxentry{Customization}!Icon display troubleshooting@\spxentry{Icon display troubleshooting}}\index{Tool Bar@\spxentry{Tool Bar}!Troubleshooting@\spxentry{Troubleshooting}}\ignorespaces 

\subsection{Icon display troubleshooting}
\label{\detokenize{Customization:icon-display-troubleshooting}}\label{\detokenize{Customization:index-35}}
\sphinxAtStartPar
If Icons on the Tool Bar are not rendered correctly, check that the browser is not in a zoom mode greater than 100\%.


\subsection{Background information}
\label{\detokenize{Customization:background-information}}
\index{Customization@\spxentry{Customization}!Adding Application customized text to the Tool Bar@\spxentry{Adding Application customized text to the Tool Bar}}\index{Tool Bar@\spxentry{Tool Bar}!Adding customized text@\spxentry{Adding customized text}}\ignorespaces 

\subsubsection{Adding Application customized text to the Tool Bar}
\label{\detokenize{Customization:adding-application-customized-text-to-the-tool-bar}}\label{\detokenize{Customization:index-36}}
\sphinxAtStartPar
Another way of providing a clear visual indication of which application the user is logged on to is to add a text label to the Tool Bar. In this example the text “MVS1” is displayed when logged on to application TSO1A, and “MVS2” is displayed for application TSO2A.

\begin{sphinxVerbatim}[commandchars=\\\{\}]
\PYG{o}{/}\PYG{o}{*}
\PYG{o}{*} \PYG{n}{VIRTEL} \PYG{n}{Web} \PYG{n}{Access} \PYG{n}{style} \PYG{n}{sheet} \PYG{k}{for} \PYG{n}{site} \PYG{n}{customisation}
\PYG{o}{*} \PYG{p}{(}\PYG{n}{c}\PYG{p}{)}\PYG{n}{Copyright} \PYG{n}{SysperTec} \PYG{n}{Communication} \PYG{l+m+mi}{2007}\PYG{p}{,}\PYG{l+m+mi}{2010} \PYG{n}{All} \PYG{n}{Rights} \PYG{n}{Reserved}
\PYG{o}{*}\PYG{o}{/}
\PYG{o}{.}\PYG{n}{toolbarLast}\PYG{p}{\PYGZob{}}
    \PYG{n}{text}\PYG{o}{\PYGZhy{}}\PYG{n}{align}\PYG{p}{:} \PYG{n}{right}\PYG{p}{;} \PYG{o}{/}\PYG{o}{*} \PYG{n}{Text} \PYG{n}{alignment} \PYG{o}{*}\PYG{o}{/}
\PYG{p}{\PYGZcb{}}
\PYG{o}{.}\PYG{n}{TSO1A} \PYG{o}{.}\PYG{n}{toolbarLast}\PYG{p}{:}\PYG{n}{before} \PYG{p}{\PYGZob{}}
    \PYG{n}{content}\PYG{p}{:} \PYG{l+s+s2}{\PYGZdq{}}\PYG{l+s+s2}{MVS1}\PYG{l+s+s2}{\PYGZdq{}}\PYG{p}{;}
    \PYG{n}{opacity}\PYG{p}{:} \PYG{l+m+mf}{0.25}\PYG{p}{;}
    \PYG{n}{font}\PYG{o}{\PYGZhy{}}\PYG{n}{size}\PYG{p}{:} \PYG{l+m+mi}{30}\PYG{n}{px}\PYG{p}{;}
    \PYG{n}{width}\PYG{p}{:} \PYG{l+m+mi}{100}\PYG{o}{\PYGZpc{}}\PYG{p}{;}
    \PYG{n}{z}\PYG{o}{\PYGZhy{}}\PYG{n}{index}\PYG{p}{:} \PYG{l+m+mi}{1000}\PYG{p}{;}
    \PYG{o}{\PYGZhy{}}\PYG{n}{webkit}\PYG{o}{\PYGZhy{}}\PYG{n}{text}\PYG{o}{\PYGZhy{}}\PYG{n}{stroke}\PYG{p}{:} \PYG{l+m+mi}{1}\PYG{n}{px} \PYG{n}{blue}\PYG{p}{;} \PYG{o}{/}\PYG{o}{*} \PYG{n}{Select} \PYG{n}{color} \PYG{o}{*}\PYG{o}{/}
    \PYG{n}{padding}\PYG{o}{\PYGZhy{}}\PYG{n}{right}\PYG{p}{:} \PYG{l+m+mi}{5}\PYG{n}{px}\PYG{p}{;} \PYG{o}{/}\PYG{o}{*} \PYG{n}{To} \PYG{n}{separate} \PYG{n}{cells} \PYG{o}{*}\PYG{o}{/}
\PYG{p}{\PYGZcb{}}

\PYG{o}{.}\PYG{n}{TSO2A} \PYG{o}{.}\PYG{n}{toolbarLast}\PYG{p}{:}\PYG{n}{before} \PYG{p}{\PYGZob{}}
    \PYG{n}{content}\PYG{p}{:} \PYG{l+s+s2}{\PYGZdq{}}\PYG{l+s+s2}{MVS2}\PYG{l+s+s2}{\PYGZdq{}}\PYG{p}{;}
    \PYG{n}{opacity}\PYG{p}{:} \PYG{l+m+mf}{0.25}\PYG{p}{;}
    \PYG{n}{font}\PYG{o}{\PYGZhy{}}\PYG{n}{size}\PYG{p}{:} \PYG{l+m+mi}{30}\PYG{n}{px}\PYG{p}{;}
    \PYG{n}{width}\PYG{p}{:} \PYG{l+m+mi}{100}\PYG{o}{\PYGZpc{}}\PYG{p}{;}
    \PYG{n}{z}\PYG{o}{\PYGZhy{}}\PYG{n}{index}\PYG{p}{:} \PYG{l+m+mi}{1000}\PYG{p}{;}
    \PYG{o}{\PYGZhy{}}\PYG{n}{webkit}\PYG{o}{\PYGZhy{}}\PYG{n}{text}\PYG{o}{\PYGZhy{}}\PYG{n}{stroke}\PYG{p}{:} \PYG{l+m+mi}{1}\PYG{n}{px} \PYG{n}{red}\PYG{p}{;} \PYG{o}{/}\PYG{o}{*} \PYG{n}{Select} \PYG{n}{color} \PYG{o}{*}\PYG{o}{/}
    \PYG{n}{padding}\PYG{o}{\PYGZhy{}}\PYG{n}{right}\PYG{p}{:} \PYG{l+m+mi}{5}\PYG{n}{px}\PYG{p}{;} \PYG{o}{/}\PYG{o}{*} \PYG{n}{To} \PYG{n}{separate} \PYG{n}{cells} \PYG{o}{*}\PYG{o}{/}
\PYG{p}{\PYGZcb{}}
\end{sphinxVerbatim}

\sphinxAtStartPar
\sphinxstyleemphasis{Example custom.css for adding custom text to the Tool Bar}

\sphinxAtStartPar
\sphinxincludegraphics{{image5}.png}
\sphinxstyleemphasis{Web Access screen with custom text in the Tool Bar}

\index{Customization@\spxentry{Customization}!Adding Transaction Application Name to the Tool Bar@\spxentry{Adding Transaction Application Name to the Tool Bar}}\index{Tool Bar@\spxentry{Tool Bar}!Adding Transaction Application Name@\spxentry{Adding Transaction Application Name}}\ignorespaces 

\subsubsection{Adding Transaction Application name to the Tool Bar}
\label{\detokenize{Customization:adding-transaction-application-name-to-the-tool-bar}}\label{\detokenize{Customization:index-37}}
\sphinxAtStartPar
Adding Application Name from the transaction definition to the Tool Bar. In this example the text “TSO” is displayed when logged on to application TSO, and “SPCICSH” is displayed for application SPCICSH. Again, this is another way to provide feedback information to the user as to which system they are connected to.

\begin{sphinxVerbatim}[commandchars=\\\{\}]
\PYG{o}{/}\PYG{o}{*}
\PYG{o}{*} \PYG{n}{VIRTEL} \PYG{n}{Web} \PYG{n}{Access} \PYG{n}{style} \PYG{n}{sheet} \PYG{k}{for} \PYG{n}{site} \PYG{n}{customisation}
\PYG{o}{*} \PYG{p}{(}\PYG{n}{c}\PYG{p}{)}\PYG{n}{Copyright} \PYG{n}{SysperTec} \PYG{n}{Communication} \PYG{l+m+mi}{2007}\PYG{p}{,}\PYG{l+m+mi}{2010} \PYG{n}{All} \PYG{n}{Rights} \PYG{n}{Reserved}
\PYG{o}{*}\PYG{o}{/}
\PYG{o}{.}\PYG{n}{toolbarLast}\PYG{p}{\PYGZob{}}
    \PYG{n}{text}\PYG{o}{\PYGZhy{}}\PYG{n}{align}\PYG{p}{:} \PYG{n}{right}\PYG{p}{;} \PYG{o}{/}\PYG{o}{*} \PYG{n}{Text} \PYG{n}{alignment} \PYG{o}{*}\PYG{o}{/}
\PYG{p}{\PYGZcb{}}
\PYG{o}{.}\PYG{n}{TSO} \PYG{o}{.}\PYG{n}{toolbarLast}\PYG{p}{:}\PYG{n}{before} \PYG{p}{\PYGZob{}}
    \PYG{n}{content}\PYG{p}{:} \PYG{l+s+s2}{\PYGZdq{}}\PYG{l+s+s2}{TSO}\PYG{l+s+s2}{\PYGZdq{}}\PYG{p}{;}
    \PYG{n}{opacity}\PYG{p}{:} \PYG{l+m+mf}{0.25}\PYG{p}{;}
    \PYG{n}{font}\PYG{o}{\PYGZhy{}}\PYG{n}{size}\PYG{p}{:} \PYG{l+m+mi}{30}\PYG{n}{px}\PYG{p}{;}
    \PYG{n}{width}\PYG{p}{:} \PYG{l+m+mi}{100}\PYG{o}{\PYGZpc{}}\PYG{p}{;}
    \PYG{n}{z}\PYG{o}{\PYGZhy{}}\PYG{n}{index}\PYG{p}{:} \PYG{l+m+mi}{1000}\PYG{p}{;}
    \PYG{o}{\PYGZhy{}}\PYG{n}{webkit}\PYG{o}{\PYGZhy{}}\PYG{n}{text}\PYG{o}{\PYGZhy{}}\PYG{n}{stroke}\PYG{p}{:} \PYG{l+m+mi}{1}\PYG{n}{px} \PYG{n}{red}\PYG{p}{;} \PYG{o}{/}\PYG{o}{*} \PYG{n}{Select} \PYG{n}{color} \PYG{o}{*}\PYG{o}{/}
    \PYG{n}{padding}\PYG{o}{\PYGZhy{}}\PYG{n}{right}\PYG{p}{:} \PYG{l+m+mi}{5}\PYG{n}{px}\PYG{p}{;} \PYG{o}{/}\PYG{o}{*} \PYG{n}{To} \PYG{n}{separate} \PYG{n}{cells} \PYG{o}{*}\PYG{o}{/}
\PYG{p}{\PYGZcb{}}
\PYG{o}{.}\PYG{n}{SPCICSH} \PYG{o}{.}\PYG{n}{toolbarLast}\PYG{p}{:}\PYG{n}{before} \PYG{p}{\PYGZob{}}
    \PYG{n}{content}\PYG{p}{:} \PYG{l+s+s2}{\PYGZdq{}}\PYG{l+s+s2}{SPCICSH}\PYG{l+s+s2}{\PYGZdq{}}\PYG{p}{;}
    \PYG{n}{opacity}\PYG{p}{:} \PYG{l+m+mf}{0.25}\PYG{p}{;}
    \PYG{n}{font}\PYG{o}{\PYGZhy{}}\PYG{n}{size}\PYG{p}{:} \PYG{l+m+mi}{30}\PYG{n}{px}\PYG{p}{;}
    \PYG{n}{width}\PYG{p}{:} \PYG{l+m+mi}{100}\PYG{o}{\PYGZpc{}}\PYG{p}{;}
    \PYG{n}{z}\PYG{o}{\PYGZhy{}}\PYG{n}{index}\PYG{p}{:} \PYG{l+m+mi}{1000}\PYG{p}{;}
    \PYG{o}{\PYGZhy{}}\PYG{n}{webkit}\PYG{o}{\PYGZhy{}}\PYG{n}{text}\PYG{o}{\PYGZhy{}}\PYG{n}{stroke}\PYG{p}{:} \PYG{l+m+mi}{1}\PYG{n}{px} \PYG{n}{blue}\PYG{p}{;} \PYG{o}{/}\PYG{o}{*} \PYG{n}{Select} \PYG{n}{color} \PYG{o}{*}\PYG{o}{/}
    \PYG{n}{padding}\PYG{o}{\PYGZhy{}}\PYG{n}{right}\PYG{p}{:} \PYG{l+m+mi}{5}\PYG{n}{px}\PYG{p}{;} \PYG{o}{/}\PYG{o}{*} \PYG{n}{To} \PYG{n}{separate} \PYG{n}{cells} \PYG{o}{*}\PYG{o}{/}
\PYG{p}{\PYGZcb{}}
\end{sphinxVerbatim}

\sphinxAtStartPar
\sphinxstyleemphasis{Example custom.css for adding custom text to the Tool Bar}

\sphinxAtStartPar
\sphinxincludegraphics{{image6}.png}
\sphinxstyleemphasis{Web Access screen with custom application name in the Tool Bar}

\index{Customization@\spxentry{Customization}!Showing/Hiding Server Information@\spxentry{Showing/Hiding Server Information}}\index{Tool Bar@\spxentry{Tool Bar}!Server Information@\spxentry{Server Information}}\ignorespaces 

\subsubsection{Showing / Hiding server information}
\label{\detokenize{Customization:showing-hiding-server-information}}\label{\detokenize{Customization:index-38}}
\sphinxAtStartPar
It is sometimes useful to have a clear visual indication of which server a user is logged on to, its version and the maintenance level applied on the system. By default, the value specified into the APPLID parameter of the VIRTCT is displayed at the top\sphinxhyphen{}right of the Tool Bar as shown below. This information is followed by the running version number and the Virtel Web access level of maintenance used. This last information is enclosed in parentheses.

\sphinxAtStartPar
\sphinxincludegraphics{{image7}.png}

\sphinxAtStartPar
The running version and the level of maintenance cannot be hidden, only the server name can be \sphinxstylestrong{permanently} removed by modifying the w2hparm.hideinfo property in a customized w2hparms.js file:

\begin{sphinxVerbatim}[commandchars=\\\{\}]
\PYG{o}{/}\PYG{o}{*}
\PYG{o}{*} \PYG{n}{Configuration} \PYG{n}{of} \PYG{n}{the} \PYG{n}{server} \PYG{n}{name} \PYG{n}{connected} \PYG{n}{to}\PYG{o}{.}
\PYG{o}{*}\PYG{o}{/}
\PYG{n}{w2hparm}\PYG{o}{.}\PYG{n}{hideinfo} \PYG{o}{=} \PYG{n}{true}\PYG{p}{;}
\end{sphinxVerbatim}

\sphinxAtStartPar
\sphinxstyleemphasis{Example w2hparm.js for hiding the mainframe application name on which a user is connected to.}

\sphinxAtStartPar
If the default value is preserved, the user can hide this information for his own usage by checking “Hide Virtel information in Tool Bar” in the Display tab of the settings panel. On the right side of the Tool Bar, the running version and the level of maintenance of VIRTEL is shown. As this information is important and very helpful in case of troubleshooting, those information cannot be hidden.


\subsection{Language}
\label{\detokenize{Customization:language}}
\index{Customization@\spxentry{Customization}!Hiding the Language Icon@\spxentry{Hiding the Language Icon}}\index{Tool Bar@\spxentry{Tool Bar}!Hiding Language Icon@\spxentry{Hiding Language Icon}}\ignorespaces 

\subsubsection{Hide the Language Icon}
\label{\detokenize{Customization:hide-the-language-icon}}\label{\detokenize{Customization:index-39}}
\sphinxAtStartPar
You can hide the language icon by using the following CSS orders included in a customized custom.css file:

\begin{sphinxVerbatim}[commandchars=\\\{\}]
\PYG{c+c1}{\PYGZsh{}Tool Bar td\PYGZsh{}toolbar\PYGZhy{}lang \PYGZob{} width: 1px; \PYGZcb{}}
\PYG{c+c1}{\PYGZsh{}Tool Bar td\PYGZsh{}toolbar\PYGZhy{}lang a \PYGZob{} display: none; \PYGZcb{}}
\end{sphinxVerbatim}

\sphinxAtStartPar
\sphinxstyleemphasis{Example to hide the language icon}

\index{Customization@\spxentry{Customization}!Assign Default Language {[}With User Option{]}@\spxentry{Assign Default Language {[}With User Option{]}}}\index{Language@\spxentry{Language}!Default Language {[}With User Option{]}@\spxentry{Default Language {[}With User Option{]}}}\ignorespaces 

\subsubsection{Assign a default language}
\label{\detokenize{Customization:assign-a-default-language}}\label{\detokenize{Customization:index-40}}
\sphinxAtStartPar
You can set a default language by using the following CSS order included in a customized custom.js file. The ability to select other languages is through a selection drop down GUI widget. :

\begin{sphinxVerbatim}[commandchars=\\\{\}]
\PYG{n}{function} \PYG{n}{after\PYGZus{}standardInit}\PYG{p}{(}\PYG{p}{)} \PYG{p}{\PYGZob{}}
    \PYG{o}{/}\PYG{o}{*} \PYG{n}{Will} \PYG{n}{force} \PYG{n}{default} \PYG{n}{language} \PYG{n}{to} \PYG{n}{Croatian} \PYG{o+ow}{and} \PYG{n}{will} \PYG{n}{leave} \PYG{n}{to} \PYG{n}{the} \PYG{n}{customer} \PYG{n}{the} \PYG{n}{possibility} \PYG{n}{to} \PYG{n}{select} \PYG{n}{another} \PYG{n}{one} \PYG{k}{if} \PYG{n}{necessary} \PYG{o}{*}\PYG{o}{/}
    \PYG{n}{oVWAmsg}\PYG{o}{.}\PYG{n}{changeLang}\PYG{p}{(}\PYG{l+s+s2}{\PYGZdq{}}\PYG{l+s+s2}{hr}\PYG{l+s+s2}{\PYGZdq{}}\PYG{p}{)}\PYG{p}{;}
\PYG{p}{\PYGZcb{}}
\end{sphinxVerbatim}

\sphinxAtStartPar
\sphinxstyleemphasis{Example to assign a default langage}

\sphinxAtStartPar
Possible values for the language code are:
\begin{itemize}
\item {} 
\sphinxAtStartPar
DE for Deutsch

\item {} 
\sphinxAtStartPar
EN for English

\item {} 
\sphinxAtStartPar
ES for Spanish

\item {} 
\sphinxAtStartPar
FR for French

\item {} 
\sphinxAtStartPar
HR for Croatian

\item {} 
\sphinxAtStartPar
IT for Italian

\end{itemize}

\sphinxAtStartPar
They must be entered in lower case.

\index{Customization@\spxentry{Customization}!Assign Default Language {[}Without User Option{]}@\spxentry{Assign Default Language {[}Without User Option{]}}}\index{Language@\spxentry{Language}!Default Language {[}Without User Option{]}@\spxentry{Default Language {[}Without User Option{]}}}\ignorespaces 

\subsubsection{Assign a permanent default language}
\label{\detokenize{Customization:assign-a-permanent-default-language}}\label{\detokenize{Customization:index-41}}
\sphinxAtStartPar
You can set a permanent language by including the following CSS order in a customized custom.js file. In this case, the user has no ptions to override the default language.

\begin{sphinxVerbatim}[commandchars=\\\{\}]
\PYG{n}{function} \PYG{n}{after\PYGZus{}standardInit}\PYG{p}{(}\PYG{p}{)} \PYG{p}{\PYGZob{}}
    \PYG{o}{/}\PYG{o}{*} \PYG{n}{Will} \PYG{n}{force} \PYG{n}{default} \PYG{n}{language} \PYG{n}{to} \PYG{n}{Croatian} \PYG{o+ow}{and} \PYG{n}{will} \PYG{o+ow}{not} \PYG{n}{allow} \PYG{n}{the} \PYG{n}{customer} \PYG{n}{to} \PYG{n}{select} \PYG{n}{another} \PYG{n}{one} \PYG{k}{if} \PYG{n}{necessary} \PYG{o}{*}\PYG{o}{/}
    \PYG{n}{oVWAmsgVWAmsg}\PYG{o}{.}\PYG{n}{restrictLanguages}\PYG{p}{(}\PYG{l+s+s2}{\PYGZdq{}}\PYG{l+s+s2}{hr}\PYG{l+s+s2}{\PYGZdq{}}\PYG{p}{)}\PYG{p}{;}
\PYG{p}{\PYGZcb{}}
\end{sphinxVerbatim}

\sphinxAtStartPar
\sphinxstyleemphasis{Example to assign a permanent default language}

\sphinxAtStartPar
Possible values for the language code are:
\begin{itemize}
\item {} 
\sphinxAtStartPar
DE for Deutsch

\item {} 
\sphinxAtStartPar
EN for English

\item {} 
\sphinxAtStartPar
ES for Spanish

\item {} 
\sphinxAtStartPar
FR for French

\item {} 
\sphinxAtStartPar
HR for Croatian

\item {} 
\sphinxAtStartPar
IT for Italian

\end{itemize}

\sphinxAtStartPar
They must be entered in lower case.

\index{Status Bar@\spxentry{Status Bar}}\ignorespaces 

\section{Status bar}
\label{\detokenize{Customization:status-bar}}\label{\detokenize{Customization:index-42}}
\sphinxAtStartPar
The status bar of the VWA user interface is located at the bottom and includes the monitoring zone (4), particulars of the terminals associated with the session (5), mode and cursor position (6).

\sphinxAtStartPar
\sphinxincludegraphics{{image7}.png}
\sphinxstyleemphasis{The VWA 3270 screen’s areas}

\index{Status Bar@\spxentry{Status Bar}!Deactivate@\spxentry{Deactivate}}\ignorespaces 

\subsection{How to deactivate the Virtel status bar}
\label{\detokenize{Customization:how-to-deactivate-the-virtel-status-bar}}\label{\detokenize{Customization:index-43}}
\sphinxAtStartPar
To deactivate the status bar create a ccs rule and add it to a customized custom.css file. For example, to deactivate it for a transactions associated with the the options “ID” “myOptions” create a “myoptions” core file called option.myOptions.css.

\begin{sphinxVerbatim}[commandchars=\\\{\}]
\PYG{o}{/}\PYG{o}{/}\PYG{n}{CLI}\PYG{o}{\PYGZhy{}}\PYG{n}{DIR} \PYG{o}{\PYGZhy{}} \PYG{n}{option}\PYG{o}{.}\PYG{n}{myOptions}\PYG{o}{.}\PYG{n}{css}\PYG{o}{.} \PYG{n}{Transaction} \PYG{n}{level} \PYG{n}{core} \PYG{n}{option} \PYG{n}{file}
\PYG{n}{var} \PYG{n}{oCustom}\PYG{o}{=}\PYG{p}{\PYGZob{}}
    \PYG{l+s+s2}{\PYGZdq{}}\PYG{l+s+s2}{pathToCssCustom}\PYG{l+s+s2}{\PYGZdq{}}\PYG{p}{:}\PYG{l+s+s2}{\PYGZdq{}}\PYG{l+s+s2}{../option/custCSS.myOptions.css}\PYG{l+s+s2}{\PYGZdq{}}
\PYG{p}{\PYGZcb{}}
\end{sphinxVerbatim}

\sphinxAtStartPar
\sphinxstyleemphasis{Example of a core option file for customized CSS file}

\sphinxAtStartPar
This “core option file” points to a customized CSS file called “custCSS.myOptions.css”. This is used by transactions which have the option=”myOptions” defined. This file contains a CSS style statement:

\begin{sphinxVerbatim}[commandchars=\\\{\}]
\PYG{n}{div}\PYG{c+c1}{\PYGZsh{}statusbar \PYGZob{}display:none!important\PYGZcb{}}
\end{sphinxVerbatim}

\sphinxAtStartPar
\sphinxstyleemphasis{Example of customized CSS element}

\index{Relay and Printer Names@\spxentry{Relay and Printer Names}}\ignorespaces 

\subsection{Relay and Printer Name}
\label{\detokenize{Customization:relay-and-printer-name}}\label{\detokenize{Customization:index-44}}
\index{Customization@\spxentry{Customization}!Relay and Printer@\spxentry{Relay and Printer}}\index{Relay and Printer Name@\spxentry{Relay and Printer Name}!Functions@\spxentry{Functions}}\ignorespaces 

\subsubsection{Managing Relay and Printer name area}
\label{\detokenize{Customization:managing-relay-and-printer-name-area}}\label{\detokenize{Customization:index-45}}
\sphinxAtStartPar
The area (5) of the VWA user interface contains information about the terminals used during the session. The name of the 3270 relay terminal is shown in the leftmost portion of the area while the name of the virtual printer terminal is shown in the rightmost portion of the area. The presence of a printer device is optional and depends on the terminal
definition itself. The information in this area can be managed using the following functions:
\begin{itemize}
\item {} 
\sphinxAtStartPar
Editing functions
\begin{itemize}
\item {} 
\sphinxAtStartPar
vwaStatusBar.setRelay(“some txt”) to customize the 3270 relay area

\item {} 
\sphinxAtStartPar
vwaStatusBar.setPrintRelay(“some txt”) to customize the print relay area

\end{itemize}

\item {} 
\sphinxAtStartPar
Retrieval functions
\begin{itemize}
\item {} 
\sphinxAtStartPar
vwaStatusBar.getRelay(P1) to retrieve the content of the 3270 relay area

\item {} 
\sphinxAtStartPar
vwaStatusBar.getPrintRelay(P1) to retrieve the content of the print relay area

\end{itemize}

\end{itemize}

\sphinxAtStartPar
For the retrieval function, if the value of the parameter P1 is “true” (without the double quotes), the information returned is the value of “Relay or Printer” as valid at the time the command executes. If the P1 is undefined or if its value is different from “true” the information returned is the value of “Relay or Printer” as existed at the time the page
was sent by VWA to the browser.


\subsubsection{Relay name area}
\label{\detokenize{Customization:relay-name-area}}
\sphinxAtStartPar
You can manage the content of the relay name area by adding calling related functions from within the after\_responseHandle function.

\begin{sphinxVerbatim}[commandchars=\\\{\}]
\PYG{o}{/}\PYG{o}{*}
\PYG{o}{*} \PYG{p}{(}\PYG{n}{c}\PYG{p}{)}\PYG{n}{Copyright} \PYG{n}{SysperTec} \PYG{n}{Communication} \PYG{l+m+mi}{2012} \PYG{n}{All} \PYG{n}{Rights} \PYG{n}{Reserved}
\PYG{o}{*} \PYG{n}{VIRTEL} \PYG{n}{Web} \PYG{n}{Access} \PYG{n}{customer}\PYG{o}{\PYGZhy{}}\PYG{n}{specific} \PYG{n}{javascript} \PYG{n}{functions}
\PYG{o}{*}\PYG{o}{/}
\PYG{n}{function} \PYG{n}{after\PYGZus{}responseHandle}\PYG{p}{(}\PYG{n}{httpXmlObj}\PYG{p}{,} \PYG{n}{url}\PYG{p}{,} \PYG{n}{xmitTimestamp}\PYG{p}{)} \PYG{p}{\PYGZob{}}
    \PYG{n}{vwaStatusBar}\PYG{o}{.}\PYG{n}{setRelay}\PYG{p}{(}\PYG{p}{)}\PYG{p}{;} \PYG{o}{/}\PYG{o}{/} \PYG{n}{Clears} \PYG{n}{relay} \PYG{n}{field}
    \PYG{n}{vwaStatusBar}\PYG{o}{.}\PYG{n}{setRelay}\PYG{p}{(}\PYG{l+s+s2}{\PYGZdq{}}\PYG{l+s+s2}{Relay: }\PYG{l+s+s2}{\PYGZdq{}} \PYG{o}{+} \PYG{o}{/}\PYG{o}{/} \PYG{n}{Adds} \PYG{n}{some} \PYG{n}{text}
    \PYG{n}{vwaStatusBar}\PYG{o}{.}\PYG{n}{getRelay}\PYG{p}{(}\PYG{p}{)}\PYG{p}{)}\PYG{p}{;} \PYG{o}{/}\PYG{o}{/} \PYG{n}{Get} \PYG{n}{the} \PYG{n}{relay} \PYG{n}{name}
\PYG{p}{\PYGZcb{}}
\end{sphinxVerbatim}

\sphinxAtStartPar
\sphinxstyleemphasis{Example custom.js to customize the content of the relay name area}

\index{Customization@\spxentry{Customization}!Hiding the Relay Name@\spxentry{Hiding the Relay Name}}\index{Relay and Printer Name@\spxentry{Relay and Printer Name}!Hiding the Relay Name@\spxentry{Hiding the Relay Name}}\ignorespaces 

\subsubsection{Hiding the Relay Name}
\label{\detokenize{Customization:hiding-the-relay-name}}\label{\detokenize{Customization:index-46}}
\sphinxAtStartPar
You can hide the the relay name by replacing its content by spaces.

\begin{sphinxVerbatim}[commandchars=\\\{\}]
\PYG{o}{/}\PYG{o}{*}
\PYG{o}{*} \PYG{p}{(}\PYG{n}{c}\PYG{p}{)}\PYG{n}{Copyright} \PYG{n}{SysperTec} \PYG{n}{Communication} \PYG{l+m+mi}{2012} \PYG{n}{All} \PYG{n}{Rights} \PYG{n}{Reserved}
\PYG{o}{*} \PYG{n}{VIRTEL} \PYG{n}{Web} \PYG{n}{Access} \PYG{n}{customer}\PYG{o}{\PYGZhy{}}\PYG{n}{specific} \PYG{n}{javascript} \PYG{n}{functions}
\PYG{o}{*}\PYG{o}{/}
\PYG{n}{function} \PYG{n}{after\PYGZus{}responseHandle}\PYG{p}{(}\PYG{n}{httpXmlObj}\PYG{p}{,} \PYG{n}{url}\PYG{p}{,} \PYG{n}{xmitTimestamp}\PYG{p}{)} \PYG{p}{\PYGZob{}}
    \PYG{n}{vwaStatusBar}\PYG{o}{.}\PYG{n}{setRelay}\PYG{p}{(}\PYG{p}{)}\PYG{p}{;} \PYG{o}{/}\PYG{o}{/} \PYG{n}{Clears} \PYG{n}{relay} \PYG{n}{field}
\PYG{p}{\PYGZcb{}}
\end{sphinxVerbatim}

\sphinxAtStartPar
\sphinxstyleemphasis{Example custom.js to hide the content of the relay name area}


\subsubsection{Printer name area}
\label{\detokenize{Customization:printer-name-area}}
\sphinxAtStartPar
You can manage the content of the printer name area by adding appropriate orders in after\_responseHandle function.

\begin{sphinxVerbatim}[commandchars=\\\{\}]
\PYG{o}{/}\PYG{o}{*}
\PYG{o}{*} \PYG{p}{(}\PYG{n}{c}\PYG{p}{)}\PYG{n}{Copyright} \PYG{n}{SysperTec} \PYG{n}{Communication} \PYG{l+m+mi}{2012} \PYG{n}{All} \PYG{n}{Rights} \PYG{n}{Reserved}
\PYG{o}{*} \PYG{n}{VIRTEL} \PYG{n}{Web} \PYG{n}{Access} \PYG{n}{customer}\PYG{o}{\PYGZhy{}}\PYG{n}{specific} \PYG{n}{javascript} \PYG{n}{functions}
\PYG{o}{*}\PYG{o}{/}
\PYG{n}{function} \PYG{n}{after\PYGZus{}responseHandle}\PYG{p}{(}\PYG{n}{httpXmlObj}\PYG{p}{,} \PYG{n}{url}\PYG{p}{,} \PYG{n}{xmitTimestamp}\PYG{p}{)} \PYG{p}{\PYGZob{}}
    \PYG{n}{vwaStatusBar}\PYG{o}{.}\PYG{n}{setPrintRelay}\PYG{p}{(}\PYG{p}{)}\PYG{p}{;} \PYG{o}{/}\PYG{o}{/} \PYG{n}{Clears} \PYG{n}{printer} \PYG{n}{field}
    \PYG{n}{vwaStatusBar}\PYG{o}{.}\PYG{n}{setPrintRelay}\PYG{p}{(}\PYG{l+s+s2}{\PYGZdq{}}\PYG{l+s+s2}{CICS printer: }\PYG{l+s+s2}{\PYGZdq{}} \PYG{o}{+} \PYG{o}{/}\PYG{o}{/} \PYG{n}{Adds} \PYG{n}{some} \PYG{n}{text}
    \PYG{n}{vwaStatusBar}\PYG{o}{.}\PYG{n}{getPrintRelay}\PYG{p}{(}\PYG{p}{)}\PYG{p}{)}\PYG{p}{;} \PYG{o}{/}\PYG{o}{/} \PYG{n}{Get} \PYG{n}{the} \PYG{n}{printer} \PYG{n}{name}
\PYG{p}{\PYGZcb{}}
\end{sphinxVerbatim}

\sphinxAtStartPar
\sphinxstyleemphasis{Example custom.js to customize the content of the printer name area}

\index{Customization@\spxentry{Customization}!Hiding the Printer Name@\spxentry{Hiding the Printer Name}}\index{Relay and Printer Name@\spxentry{Relay and Printer Name}!Hiding the Printer Name@\spxentry{Hiding the Printer Name}}\ignorespaces 

\subsubsection{Hide Printer Name}
\label{\detokenize{Customization:hide-printer-name}}\label{\detokenize{Customization:index-47}}
\sphinxAtStartPar
You can hide the content of the printer name by replacing its content by spaces.

\begin{sphinxVerbatim}[commandchars=\\\{\}]
\PYG{o}{/}\PYG{o}{*}
\PYG{o}{*} \PYG{p}{(}\PYG{n}{c}\PYG{p}{)}\PYG{n}{Copyright} \PYG{n}{SysperTec} \PYG{n}{Communication} \PYG{l+m+mi}{2012} \PYG{n}{All} \PYG{n}{Rights} \PYG{n}{Reserved}
\PYG{o}{*} \PYG{n}{VIRTEL} \PYG{n}{Web} \PYG{n}{Access} \PYG{n}{customer}\PYG{o}{\PYGZhy{}}\PYG{n}{specific} \PYG{n}{javascript} \PYG{n}{functions}
\PYG{o}{*}\PYG{o}{/}

\PYG{n}{function} \PYG{n}{after\PYGZus{}responseHandle}\PYG{p}{(}\PYG{n}{httpXmlObj}\PYG{p}{,} \PYG{n}{url}\PYG{p}{,} \PYG{n}{xmitTimestamp}\PYG{p}{)} \PYG{p}{\PYGZob{}}
    \PYG{n}{vwaStatusBar}\PYG{o}{.}\PYG{n}{setPrintRelay}\PYG{p}{(}\PYG{p}{)}\PYG{p}{;} \PYG{o}{/}\PYG{o}{/} \PYG{n}{Clears} \PYG{n}{printer} \PYG{n}{field}
\PYG{p}{\PYGZcb{}}
\end{sphinxVerbatim}

\sphinxAtStartPar
\sphinxstyleemphasis{Example custom.js to hide the content of the printer name area}

\index{Dynamic Directory Interface@\spxentry{Dynamic Directory Interface}}\ignorespaces 

\chapter{Dynamic Directory Interface (DDI)}
\label{\detokenize{Customization:dynamic-directory-interface-ddi}}\label{\detokenize{Customization:index-48}}
\sphinxAtStartPar
The Dynamic Directory Interface is intended for use by Virtel Administrators in managing dynamic directories dedicated to supporting centralized macros associated with users and user groups. The DDI interface is accessible from the Administration Portal, normally served by port 40001 and the associated W\sphinxhyphen{}HTTP line.

\sphinxAtStartPar
\sphinxincludegraphics{{image26}.png}

\sphinxAtStartPar
\sphinxstyleemphasis{Accessing the Dynamic Directory Interface}

\index{Virtel Lines@\spxentry{Virtel Lines}}\ignorespaces 

\chapter{Virtel Lines}
\label{\detokenize{Customization:virtel-lines}}\label{\detokenize{Customization:index-49}}
\index{Virtel Lines@\spxentry{Virtel Lines}!Controlling@\spxentry{Controlling}}\ignorespaces 

\section{Starting and stopping a line}
\label{\detokenize{Customization:starting-and-stopping-a-line}}\label{\detokenize{Customization:index-50}}
\sphinxAtStartPar
By default, a line is automatically initialized at Virtel startup, and is terminated when Virtel stops. In some cases, the handling of line initialization/termination needs to be managed differently:

\index{Virtel Lines@\spxentry{Virtel Lines}!Controlling initial state with ""Possible Calls"" parameter@\spxentry{Controlling initial state with ""Possible Calls"" parameter}}\ignorespaces 
\sphinxAtStartPar
Determining the initial state of a line is determined by the “Possible Calls” parameter of the line definition.

\sphinxAtStartPar
\sphinxincludegraphics{{image27}.png}

\sphinxAtStartPar
\sphinxstyleemphasis{Line definition \sphinxhyphen{} Possible Calls Parameter}

\sphinxAtStartPar
\sphinxstylestrong{Initialization through “Possible Calls” definition}

\sphinxAtStartPar
A “Possible Calls” determines the possible direction of communication and can have one of four values:\sphinxhyphen{}
\begin{itemize}
\item {} 
\sphinxAtStartPar
0 \sphinxhyphen{} No Calls

\item {} 
\sphinxAtStartPar
1 \sphinxhyphen{} Inbound Calls.

\item {} 
\sphinxAtStartPar
2 \sphinxhyphen{} Outbound Calls.

\item {} 
\sphinxAtStartPar
3 \sphinxhyphen{} Inbound and Outbound Calls.

\end{itemize}

\sphinxAtStartPar
If the value is 0 then the line is not initialized at Virtel start\sphinxhyphen{}up. It must be started manually. For values 1, 2, 3 the line is initialized automatically at Virtel start\sphinxhyphen{}up except for the cases below.

\sphinxAtStartPar
\sphinxstylestrong{Initialization permanently completely inhibited}

\index{Virtel Lines@\spxentry{Virtel Lines}!Controlling state with TCT IGNLU keyword@\spxentry{Controlling state with TCT IGNLU keyword}}\ignorespaces 
\sphinxAtStartPar
If the VIRTCT contains a parameter IGNLU that references the line, then this line shall not be initialized automatically at Virtel start\sphinxhyphen{}up. In this case it will no longer be possible to start this line manually after Virtel has started.

\index{Virtel Lines@\spxentry{Virtel Lines}!Controlling line state conditionally@\spxentry{Controlling line state conditionally}}\ignorespaces 
\sphinxAtStartPar
\sphinxstylestrong{Conditional initialization of a line}

\sphinxAtStartPar
It is possible to condition the initialization of a line to that of another line. This can be necessary for example when an application communicates with Virtel via MQ/Series, with one line that accepts inbound messages and another line that handles outbound messages. In this case, it is useful to wait until the communication with the partner has been
established before accepting inbound messages.

\begin{sphinxVerbatim}[commandchars=\\\{\}]
\PYG{n}{LINE} \PYG{n}{ID}\PYG{o}{=}\PYG{n}{M}\PYG{o}{\PYGZhy{}}\PYG{n}{MQ1}\PYG{p}{,} \PYG{o}{\PYGZhy{}}
    \PYG{n}{NAME}\PYG{o}{=}\PYG{n}{MQ}\PYG{o}{\PYGZhy{}}\PYG{n}{IN}\PYG{p}{,} \PYG{o}{\PYGZhy{}}
    \PYG{n}{LOCADDR}\PYG{o}{=}\PYG{n}{REQ}\PYG{o}{.}\PYG{n}{INPUT}\PYG{o}{.}\PYG{n}{QUEUE}\PYG{p}{,} \PYG{o}{\PYGZhy{}}
    \PYG{n}{DESC}\PYG{o}{=}\PYG{l+s+s1}{\PYGZsq{}}\PYG{l+s+s1}{MQ \PYGZhy{} REQUEST}\PYG{l+s+s1}{\PYGZsq{}}\PYG{p}{,} \PYG{o}{\PYGZhy{}}
    \PYG{n}{TERMINAL}\PYG{o}{=}\PYG{n}{MQINT}\PYG{p}{,} \PYG{o}{\PYGZhy{}}
    \PYG{n}{ENTRY}\PYG{o}{=}\PYG{n}{MQINEP}\PYG{p}{,} \PYG{o}{\PYGZhy{}}
    \PYG{n}{TYPE}\PYG{o}{=}\PYG{n}{MQ1}\PYG{p}{,} \PYG{o}{\PYGZhy{}}
    \PYG{n}{INOUT}\PYG{o}{=}\PYG{l+m+mi}{1}\PYG{p}{,} \PYG{o}{\PYGZhy{}}
    \PYG{n}{COND}\PYG{o}{=}\PYG{l+s+s1}{\PYGZsq{}}\PYG{l+s+s1}{MIMIC\PYGZhy{}LINE(M\PYGZhy{}MQO)}\PYG{l+s+s1}{\PYGZsq{}}\PYG{p}{,} \PYG{o}{\PYGZhy{}}
    \PYG{n}{PROTOCOL}\PYG{o}{=}\PYG{n}{PREFIXED}\PYG{p}{,} \PYG{o}{\PYGZhy{}}
    \PYG{n}{RULESET}\PYG{o}{=}\PYG{n}{M}\PYG{o}{\PYGZhy{}}\PYG{n}{MQ1}
    \PYG{n}{LINE} \PYG{n}{ID}\PYG{o}{=}\PYG{n}{M}\PYG{o}{\PYGZhy{}}\PYG{n}{MQ2}\PYG{p}{,} \PYG{o}{\PYGZhy{}}
    \PYG{n}{NAME}\PYG{o}{=}\PYG{n}{MQ}\PYG{o}{\PYGZhy{}}\PYG{n}{OUT}\PYG{p}{,} \PYG{o}{\PYGZhy{}}
    \PYG{n}{LOCADDR}\PYG{o}{=}\PYG{n}{REQ}\PYG{o}{.}\PYG{n}{OUTPUT}\PYG{o}{.}\PYG{n}{QUEUE}\PYG{p}{,} \PYG{o}{\PYGZhy{}}
    \PYG{n}{DESC}\PYG{o}{=}\PYG{l+s+s1}{\PYGZsq{}}\PYG{l+s+s1}{MQ \PYGZhy{} OUTPUT REQUETE}\PYG{l+s+s1}{\PYGZsq{}}\PYG{p}{,} \PYG{o}{\PYGZhy{}}
    \PYG{n}{TERMINAL}\PYG{o}{=}\PYG{n}{MQOUT}\PYG{p}{,} \PYG{o}{\PYGZhy{}}
    \PYG{n}{TYPE}\PYG{o}{=}\PYG{n}{MQ1}\PYG{p}{,} \PYG{o}{\PYGZhy{}}
    \PYG{n}{INOUT}\PYG{o}{=}\PYG{l+m+mi}{2}\PYG{p}{,} \PYG{o}{\PYGZhy{}}
    \PYG{n}{PROTOCOL}\PYG{o}{=}\PYG{n}{PREFIXED}\PYG{p}{,} \PYG{o}{\PYGZhy{}}
    \PYG{n}{RULESET}\PYG{o}{=}\PYG{n}{M}\PYG{o}{\PYGZhy{}}\PYG{n}{MQ2}
\end{sphinxVerbatim}

\index{Virtel Lines@\spxentry{Virtel Lines}!Deferring initialization@\spxentry{Deferring initialization}}\ignorespaces 
\sphinxAtStartPar
\sphinxstylestrong{Deferred initialization}

\sphinxAtStartPar
The “Possible calls” field must be set to 0. The line may subsequently be started by a START command. It must not be referenced by an IGNLU parameter in the VIRTCT. Automation could be used to start the line when a particular event occurs and can be trapped by some automation software.

\index{Virtel Lines@\spxentry{Virtel Lines}!Start and Stop Commands@\spxentry{Start and Stop Commands}}\index{Commands@\spxentry{Commands}!Start and Stop lines@\spxentry{Start and Stop lines}}\ignorespaces 
\sphinxAtStartPar
\sphinxstylestrong{Virtel Line Commands}

\sphinxAtStartPar
Using commands at the system console Lines can be used to start or stop a line by entering the appropriate command at the console. For further information on how to issue Virtel commands, see the Virtel “Audit, Operation And Performance” manuel.

\begin{sphinxVerbatim}[commandchars=\\\{\}]
\PYG{n}{LINE}\PYG{o}{=}\PYG{n}{linename}\PYG{p}{,}\PYG{n}{START} \PYG{p}{(}\PYG{o+ow}{or} \PYG{n}{L}\PYG{o}{=}\PYG{n}{linename}\PYG{p}{,}\PYG{n}{S}\PYG{p}{)}
\PYG{n}{LINE}\PYG{o}{=}\PYG{n}{linename}\PYG{p}{,}\PYG{n}{STOP} \PYG{p}{(}\PYG{o+ow}{or} \PYG{n}{L}\PYG{o}{=}\PYG{n}{linename}\PYG{p}{,}\PYG{n}{P}\PYG{p}{)}

\PYG{o}{*}\PYG{o}{*}\PYG{n}{linename}\PYG{o}{*}\PYG{o}{*} \PYG{o}{\PYGZhy{}} \PYG{n}{internal} \PYG{o+ow}{or} \PYG{n}{external} \PYG{n}{name} \PYG{n}{of} \PYG{n}{the} \PYG{n}{line}
\end{sphinxVerbatim}

\sphinxAtStartPar
The LINE START and LINE STOP commands perform the same function as using the “S” and “P” commands on the “Status of lines” application panel. These commands may only be issued for line types AntiGATE, AntiPCNE, AntiFASTC, and TCP/IP.

\index{Virtel Transactions@\spxentry{Virtel Transactions}}\ignorespaces 

\chapter{Virtel Transactions}
\label{\detokenize{Customization:virtel-transactions}}\label{\detokenize{Customization:index-56}}
\index{Virtel Transactions@\spxentry{Virtel Transactions}!Multi\sphinxhyphen{}Session Feature@\spxentry{Multi\sphinxhyphen{}Session Feature}}\ignorespaces 

\section{Virtel Multi\sphinxhyphen{}session Within Virtel Web Access}
\label{\detokenize{Customization:virtel-multi-session-within-virtel-web-access}}\label{\detokenize{Customization:index-57}}
\sphinxAtStartPar
In some situations, it is necessary to allow a group of users to simultaneously access several distinct 3270 applications. This can be solved by using the “appmenu.htm” associated with the “applist” transaction, or by using the 3270 Multi\sphinxhyphen{}session function of Virtel. In the first situation, for a same original calling terminal, Virtel uses as many relays as VTAM open sessions, in the second configuration, a single relay is used for all sessions, Virtel ensuring the swapping between the opened session. To access the Multi\sphinxhyphen{}session function of Virtel, you must define a transaction whose application name is the same as the main Virtel ACB name. This transaction is accessible in the same way as any other VTAM application. As the first screen shown is a signature screen, the transaction does not necessarily need to be secured. The lists of applications presented to the user depends on the selected parameter setting (See the Virtel Multi\sphinxhyphen{}session documented section for further information on this subject). Use of the multi\sphinxhyphen{}session Virtel module, even in a VWA context, requires an appropriate license agreement.

\sphinxAtStartPar
\sphinxincludegraphics{{image29}.png}

\sphinxAtStartPar
\sphinxstyleemphasis{Transaction definition for the Virtel Multi\sphinxhyphen{}session module}


\section{Directly access host Applications by URL}
\label{\detokenize{Customization:directly-access-host-applications-by-url}}
\index{Virtel Transactions@\spxentry{Virtel Transactions}!Application access by URL@\spxentry{Application access by URL}}\index{URL Formats@\spxentry{URL Formats}!URL specifying target application@\spxentry{URL specifying target application}}\ignorespaces 
\sphinxAtStartPar
It is not always necessary to pass via an application selection menu to connect to a host application. A host application may be accessed directly by opening the URL containing the complete path to the application. This URL may result in the display of the host signon screen, the first application screen, or possibly (if a script or scenario is used), a
subsequent screen sent by the application. For more information about how VIRTEL can be used to automate the process of connection to a host application, see Virtel URL formats in the Virtel Web Access Reference manual, and “Connection/Disconnection Scripts” in the VIRTEL Connectivity Reference manual.


\subsection{Full path URL}
\label{\detokenize{Customization:full-path-url}}
\sphinxAtStartPar
For example, you can access the VIRTEL transaction whose external name is “CICS” by pointing the browser at a URL of the following format:

\begin{sphinxVerbatim}[commandchars=\\\{\}]
\PYG{n}{http}\PYG{p}{:}\PYG{o}{/}\PYG{o}{/}\PYG{l+m+mf}{10.20}\PYG{l+m+mf}{.170}\PYG{l+m+mf}{.71}\PYG{p}{:}\PYG{l+m+mi}{41001}\PYG{o}{/}\PYG{n}{w2h}\PYG{o}{/}\PYG{n}{WEB2AJAX}\PYG{o}{.}\PYG{n}{htm}\PYG{o}{+}\PYG{n}{Cics}
\end{sphinxVerbatim}

\sphinxAtStartPar
At the end of the session with the host application, VIRTEL examines the “Last page” field (see previous section) to decide whether to return to the desktop or to redisplay the application selection menu.

\index{Virtel Transactions@\spxentry{Virtel Transactions}!Application access by Entry Point@\spxentry{Application access by Entry Point}}\index{URL Formats@\spxentry{URL Formats}!Default Entry Point URL@\spxentry{Default Entry Point URL}}\ignorespaces 

\subsection{Default URL for the entry point}
\label{\detokenize{Customization:default-url-for-the-entry-point}}\label{\detokenize{Customization:index-59}}
\sphinxAtStartPar
An application URL may be coded in the “TIOA at logon” field of the default transaction for the entry point (the default transaction is the transaction whose external name is the same as the entry point name). This allows the user to go directly to the host application simply by entering a URL of the format:

\begin{sphinxVerbatim}[commandchars=\\\{\}]
\PYG{n}{http}\PYG{p}{:}\PYG{o}{/}\PYG{o}{/}\PYG{l+m+mf}{10.20}\PYG{l+m+mf}{.170}\PYG{l+m+mf}{.71}\PYG{p}{:}\PYG{l+m+mi}{41001}
\end{sphinxVerbatim}

\sphinxAtStartPar
The example below shows the default transaction for the WEB2HOST entry point set up to go directly to the transaction whose external name is “CICS”:

\sphinxAtStartPar
\sphinxincludegraphics{{image30}.png}

\sphinxAtStartPar
\sphinxstyleemphasis{Example of default URL}

\sphinxAtStartPar
For more information see Virtel URL formats in the Virtel Web Access Users Guide.

\index{Virtel Transactions@\spxentry{Virtel Transactions}!Specifying screen size and attributes@\spxentry{Specifying screen size and attributes}}\index{Application customization@\spxentry{Application customization}!Specifying screen size and attributes@\spxentry{Specifying screen size and attributes}}\ignorespaces 

\subsection{How To Use Different Screen Sizes}
\label{\detokenize{Customization:how-to-use-different-screen-sizes}}\label{\detokenize{Customization:index-60}}
\sphinxAtStartPar
Although the standard 3270 screen size is 24 rows by 80 columns, certain applications benefit from the use of terminals with larger screen sizes. The screen size is determined by the LOGMODE used for the session between VIRTEL and the host application. VTAM offers logmodes for the following standard screen sizes:
\begin{itemize}
\item {} 
\sphinxAtStartPar
model 2 : 24x80 (logmode SNX32702)

\item {} 
\sphinxAtStartPar
model 3 : 32x80 (logmode SNX32703)

\item {} 
\sphinxAtStartPar
model 4 : 43x80 (logmode SNX32704)

\item {} 
\sphinxAtStartPar
model 5 : 27x132 (logmode SNX32705)

\end{itemize}

\sphinxAtStartPar
There are two different ways that the VIRTEL administrator can set up the configuration to allow the VIRTEL Web Access user to select the desired logmode:
\begin{itemize}
\item {} 
\sphinxAtStartPar
Define a separate VIRTEL transaction for each screen size, and allow the user to select the appropriate transaction.

\item {} 
\sphinxAtStartPar
Group the VTAM relay LUs into pools, each pool having a different logmode, and allow the user to select the pool by coding an appropriate parameter on the URL.

\end{itemize}

\index{Virtel Transactions@\spxentry{Virtel Transactions}!Specifying default logmode@\spxentry{Specifying default logmode}}\index{Application customization@\spxentry{Application customization}!Specifying default logmode@\spxentry{Specifying default logmode}}\ignorespaces 

\subsection{LOGMODE defined by the transaction}
\label{\detokenize{Customization:logmode-defined-by-the-transaction}}\label{\detokenize{Customization:index-61}}
\sphinxAtStartPar
With this method, the administrator defines multiple VIRTEL transactions for a single application, each transaction specifying a different logmode. For example, transactions Tso2 and Tso5 delivered in the sample configuration both define TSO as the target application, but specify different logmodes SNX32702 and SNX32705 respectively. The user selects the desired transaction from the applist menu displayed by the “Other applications” link in the VIRTEL Web Access menu.

\sphinxAtStartPar
The figure below shows the definition of the Tso5 transaction defined under the WEB2HOST entry point:

\sphinxAtStartPar
\sphinxincludegraphics{{image31}.png}

\sphinxAtStartPar
\sphinxstyleemphasis{Example of TSO transaction TSO specifying logmode SNX32705}

\sphinxAtStartPar
The URL to access this transaction could be of the format:

\begin{sphinxVerbatim}[commandchars=\\\{\}]
\PYG{n}{http}\PYG{p}{:}\PYG{o}{/}\PYG{o}{/}\PYG{l+m+mf}{10.20}\PYG{l+m+mf}{.170}\PYG{l+m+mf}{.71}\PYG{p}{:}\PYG{l+m+mi}{41001}\PYG{o}{/}\PYG{n}{w2h}\PYG{o}{/}\PYG{n}{WEB3270}\PYG{o}{.}\PYG{n}{htm}\PYG{o}{+}\PYG{n}{Tso5}
\end{sphinxVerbatim}

\index{Virtel Transactions@\spxentry{Virtel Transactions}!Specifying logmode in the URL@\spxentry{Specifying logmode in the URL}}\index{Application customization@\spxentry{Application customization}!Specifying logmode in the URL@\spxentry{Specifying logmode in the URL}}\index{URL Formats@\spxentry{URL Formats}!Specifying the logmode@\spxentry{Specifying the logmode}}\ignorespaces 

\subsection{Assigning a LOGMODE by URL parameter}
\label{\detokenize{Customization:assigning-a-logmode-by-url-parameter}}\label{\detokenize{Customization:index-62}}
\sphinxAtStartPar
The URL which allows the browser to connect to a host application via VIRTEL may contain a parameter, such as “model5” as shown in this example:

\begin{sphinxVerbatim}[commandchars=\\\{\}]
\PYG{n}{http}\PYG{p}{:}\PYG{o}{/}\PYG{o}{/}\PYG{l+m+mf}{10.20}\PYG{l+m+mf}{.170}\PYG{l+m+mf}{.71}\PYG{p}{:}\PYG{l+m+mi}{41001}\PYG{o}{/}\PYG{n}{w2h}\PYG{o}{/}\PYG{n}{WEB3270}\PYG{o}{.}\PYG{n}{htm}\PYG{o}{+}\PYG{n}{Tso}\PYG{o}{+}\PYG{n}{model5}
\end{sphinxVerbatim}

\sphinxAtStartPar
This form of a VIRTEL URL is described in the section “Dynamic URL with user data”, page 1. This form of URL is processed by VIRTEL with reference to the “rule set” associated with the HTTP line. VIRTEL looks for a rule whose “User Data” field matches the value of the parameter (model5). The “Parameter” field of the selected rule assigns a relay LU name from the pool defined with logmode SNX32705. The VTAM definition of the relay pool is shown in the example below. In this example, LU names in the range RHTVT5nn are defined to have the
model5 logmode SNX32705 :\sphinxhyphen{}

\begin{sphinxVerbatim}[commandchars=\\\{\}]
VIRTAPPL VBUILD TYPE=APPL
* \PYGZhy{}\PYGZhy{}\PYGZhy{}\PYGZhy{}\PYGZhy{}\PYGZhy{}\PYGZhy{}\PYGZhy{}\PYGZhy{}\PYGZhy{}\PYGZhy{}\PYGZhy{}\PYGZhy{}\PYGZhy{}\PYGZhy{}\PYGZhy{}\PYGZhy{}\PYGZhy{}\PYGZhy{}\PYGZhy{}\PYGZhy{}\PYGZhy{}\PYGZhy{}\PYGZhy{}\PYGZhy{}\PYGZhy{}\PYGZhy{}\PYGZhy{}\PYGZhy{}\PYGZhy{}\PYGZhy{}\PYGZhy{}\PYGZhy{}\PYGZhy{}\PYGZhy{}\PYGZhy{}\PYGZhy{}\PYGZhy{}\PYGZhy{}\PYGZhy{}\PYGZhy{}\PYGZhy{}\PYGZhy{}\PYGZhy{}\PYGZhy{}\PYGZhy{}\PYGZhy{}\PYGZhy{}\PYGZhy{}\PYGZhy{}\PYGZhy{}\PYGZhy{}\PYGZhy{}\PYGZhy{}\PYGZhy{}\PYGZhy{}\PYGZhy{}\PYGZhy{}\PYGZhy{}\PYGZhy{}\PYGZhy{}\PYGZhy{}\PYGZhy{}\PYGZhy{}\PYGZhy{}\PYGZhy{} *
* RHTVTxxx : Relay for VTAM applications acceded by WEB to HOST *
* \PYGZhy{}\PYGZhy{}\PYGZhy{}\PYGZhy{}\PYGZhy{}\PYGZhy{}\PYGZhy{}\PYGZhy{}\PYGZhy{}\PYGZhy{}\PYGZhy{}\PYGZhy{}\PYGZhy{}\PYGZhy{}\PYGZhy{}\PYGZhy{}\PYGZhy{}\PYGZhy{}\PYGZhy{}\PYGZhy{}\PYGZhy{}\PYGZhy{}\PYGZhy{}\PYGZhy{}\PYGZhy{}\PYGZhy{}\PYGZhy{}\PYGZhy{}\PYGZhy{}\PYGZhy{}\PYGZhy{}\PYGZhy{}\PYGZhy{}\PYGZhy{}\PYGZhy{}\PYGZhy{}\PYGZhy{}\PYGZhy{}\PYGZhy{}\PYGZhy{}\PYGZhy{}\PYGZhy{}\PYGZhy{}\PYGZhy{}\PYGZhy{}\PYGZhy{}\PYGZhy{}\PYGZhy{}\PYGZhy{}\PYGZhy{}\PYGZhy{}\PYGZhy{}\PYGZhy{}\PYGZhy{}\PYGZhy{}\PYGZhy{}\PYGZhy{}\PYGZhy{}\PYGZhy{}\PYGZhy{}\PYGZhy{}\PYGZhy{}\PYGZhy{}\PYGZhy{}\PYGZhy{}\PYGZhy{} *
* 3270 model 2 terminals
RHTVT0?? APPL AUTH=(ACQ,PASS),MODETAB=ISTINCLM,DLOGMOD=SNX32702,EAS=1
* 3270 model 5 terminals
RHTVT5?? APPL AUTH=(ACQ,PASS),MODETAB=ISTINCLM,DLOGMOD=SNX32705,EAS=1
VTAM definition of terminal groups
\end{sphinxVerbatim}

\sphinxAtStartPar
The screen below shows an example rule which assigns a relay LU from the range RHTVT5nn when the URL contains the parameter model5:

\begin{sphinxVerbatim}[commandchars=\\\{\}]
\PYG{n}{DETAIL} \PYG{n}{of} \PYG{n}{RULE} \PYG{k+kn}{from} \PYG{n+nn}{RULE} \PYG{n}{SET}\PYG{p}{:} \PYG{n}{W}\PYG{o}{\PYGZhy{}}\PYG{n}{HTTP} \PYG{o}{\PYGZhy{}}\PYG{o}{\PYGZhy{}}\PYG{o}{\PYGZhy{}}\PYG{o}{\PYGZhy{}}\PYG{o}{\PYGZhy{}}\PYG{o}{\PYGZhy{}}\PYG{o}{\PYGZhy{}}\PYG{o}{\PYGZhy{}}\PYG{o}{\PYGZhy{}}\PYG{o}{\PYGZhy{}}\PYG{o}{\PYGZhy{}}\PYG{o}{\PYGZhy{}}\PYG{o}{\PYGZhy{}} \PYG{n}{Applid}\PYG{p}{:} \PYG{n}{VIRTEL} \PYG{l+m+mi}{17}\PYG{p}{:}\PYG{l+m+mi}{15}\PYG{p}{:}\PYG{l+m+mi}{15}
\PYG{n}{Name} \PYG{o}{==}\PYG{o}{=}\PYG{o}{\PYGZgt{}} \PYG{n}{WHT00150}                      \PYG{n}{Rule} \PYG{n}{priority} \PYG{o+ow}{is} \PYG{n}{per} \PYG{n}{name}
\PYG{n}{Status} \PYG{o}{==}\PYG{o}{=}\PYG{o}{\PYGZgt{}} \PYG{n}{INACTIVE}                    \PYG{n}{Mon}\PYG{p}{,} \PYG{l+m+mi}{24} \PYG{n}{Sep} \PYG{l+m+mi}{2001} \PYG{l+m+mi}{14}\PYG{p}{:}\PYG{l+m+mi}{19}\PYG{p}{:}\PYG{l+m+mi}{14}
\PYG{n}{Description} \PYG{o}{==}\PYG{o}{=}\PYG{o}{\PYGZgt{}} \PYG{n}{HTTP} \PYG{n}{access}    \PYG{p}{(}\PYG{k}{with} \PYG{n}{model5} \PYG{n}{URL} \PYG{n}{parameter}\PYG{p}{)}
\PYG{n}{Entry} \PYG{n}{point} \PYG{o}{==}\PYG{o}{=}\PYG{o}{\PYGZgt{}} \PYG{n}{WEB2HOST}               \PYG{n}{Target} \PYG{n}{Entry} \PYG{n}{Point}
\PYG{n}{Parameter} \PYG{o}{==}\PYG{o}{=}\PYG{o}{\PYGZgt{}} \PYG{n}{RHTVT5}\PYG{o}{*}                  \PYG{o}{\PYGZam{}}\PYG{l+m+mi}{1} \PYG{n}{value} \PYG{o+ow}{or} \PYG{n}{LUNAME}
\PYG{n}{Trace} \PYG{o}{==}\PYG{o}{=}\PYG{o}{\PYGZgt{}}                              \PYG{l+m+mi}{1}\PYG{o}{=}\PYG{n}{commands} \PYG{l+m+mi}{2}\PYG{o}{=}\PYG{n}{data} \PYG{l+m+mi}{3}\PYG{o}{=}\PYG{n}{partner}
\PYG{n}{C} \PYG{p}{:} \PYG{l+m+mi}{0}\PYG{o}{=}\PYG{n}{IGNORE} \PYG{l+m+mi}{1}\PYG{o}{=}\PYG{n}{IS} \PYG{l+m+mi}{2}\PYG{o}{=}\PYG{n}{IS} \PYG{n}{NOT} \PYG{l+m+mi}{3}\PYG{o}{=}\PYG{n}{STARTS} \PYG{n}{WITH} \PYG{l+m+mi}{4}\PYG{o}{=}\PYG{n}{DOES} \PYG{n}{NOT} \PYG{l+m+mi}{5}\PYG{o}{=}\PYG{n}{ENDS} \PYG{n}{WITH} \PYG{l+m+mi}{6}\PYG{o}{=}\PYG{n}{DOES} \PYG{n}{NOT}
\PYG{l+m+mi}{0} \PYG{n}{IP} \PYG{n}{Subnet} \PYG{o}{==}\PYG{o}{=}\PYG{o}{\PYGZgt{}} \PYG{n}{Mask} \PYG{o}{==}\PYG{o}{=}\PYG{o}{\PYGZgt{}}
\PYG{l+m+mi}{0} \PYG{n}{Host} \PYG{o}{==}\PYG{o}{=}\PYG{o}{\PYGZgt{}}
\PYG{l+m+mi}{0} \PYG{n}{eMail} \PYG{o}{==}\PYG{o}{=}\PYG{o}{\PYGZgt{}}
\PYG{l+m+mi}{0} \PYG{n}{Calling} \PYG{n}{DTE} \PYG{o}{==}\PYG{o}{=}\PYG{o}{\PYGZgt{}}                      \PYG{n}{Calling} \PYG{n}{DTE} \PYG{n}{address} \PYG{o+ow}{or} \PYG{n}{proxy}
\PYG{l+m+mi}{0} \PYG{n}{Called} \PYG{o}{==}\PYG{o}{=}\PYG{o}{\PYGZgt{}}                           \PYG{n}{Called} \PYG{n}{DTE} \PYG{n}{address}
\PYG{l+m+mi}{0} \PYG{n}{CUD0} \PYG{p}{(}\PYG{n}{Hex}\PYG{p}{)} \PYG{o}{==}\PYG{o}{=}\PYG{o}{\PYGZgt{}}                       \PYG{n}{First} \PYG{l+m+mi}{4} \PYG{n+nb}{bytes} \PYG{n}{of} \PYG{n}{CUD} \PYG{p}{(}\PYG{n}{X25} \PYG{n}{protocol}\PYG{p}{)}
\PYG{l+m+mi}{1} \PYG{n}{User} \PYG{n}{Data} \PYG{o}{==}\PYG{o}{=}\PYG{o}{\PYGZgt{}} \PYG{n}{model5}
\PYG{l+m+mi}{0} \PYG{n}{Days}       \PYG{o}{==}\PYG{o}{=}\PYG{o}{\PYGZgt{}} \PYG{n}{M}\PYG{p}{:}  \PYG{n}{T}\PYG{p}{:}  \PYG{n}{W}\PYG{p}{:}  \PYG{n}{T}\PYG{p}{:}  \PYG{n}{F}\PYG{p}{:}  \PYG{n}{S}\PYG{p}{:}  \PYG{n}{S}\PYG{p}{:}
\PYG{l+m+mi}{0} \PYG{n}{Start} \PYG{n}{time} \PYG{o}{==}\PYG{o}{=}\PYG{o}{\PYGZgt{}} \PYG{n}{H}\PYG{p}{:}  \PYG{n}{M}\PYG{p}{:}  \PYG{n}{S}\PYG{p}{:} \PYG{n}{End} \PYG{n}{time} \PYG{o}{==}\PYG{o}{=}\PYG{o}{\PYGZgt{}} \PYG{n}{H}\PYG{p}{:}   \PYG{n}{M}\PYG{p}{:}  \PYG{n}{S}\PYG{p}{:}

\PYG{n}{P1}\PYG{o}{=}\PYG{n}{Update}                           \PYG{n}{P3}\PYG{o}{=}\PYG{n}{Return}                       \PYG{n}{Enter}\PYG{o}{=}\PYG{n}{Add}
\PYG{n}{P4}\PYG{o}{=}\PYG{n}{Activate}                         \PYG{n}{P5}\PYG{o}{=}\PYG{n}{Inactivate}                   \PYG{n}{P12}\PYG{o}{=}\PYG{n}{Entry} \PYG{n}{P}\PYG{o}{.}
\end{sphinxVerbatim}

\sphinxAtStartPar
\sphinxstyleemphasis{Example rule for selection of logmode by URL}

\newpage

\sphinxAtStartPar
The LU name (RHTVT5nn) assigned by the rule must belong to the LU pool shared assigned to the HTTP line, as shown in the example below :\sphinxhyphen{}

\begin{sphinxVerbatim}[commandchars=\\\{\}]
TERMINAL DETAIL DEFINITION \PYGZhy{}\PYGZhy{}\PYGZhy{}\PYGZhy{}\PYGZhy{}\PYGZhy{}\PYGZhy{}\PYGZhy{}\PYGZhy{}\PYGZhy{}\PYGZhy{}\PYGZhy{}\PYGZhy{}\PYGZhy{}\PYGZhy{}\PYGZhy{}\PYGZhy{}\PYGZhy{}\PYGZhy{}\PYGZhy{}\PYGZhy{}\PYGZhy{}\PYGZhy{}\PYGZhy{}\PYGZhy{} Applid: VIRTEL 13:32:28
Terminal ===\PYGZgt{} W2HTP500      ?wxyZZZZ for dynamic allocation
                            w : Sna or Non\PYGZhy{}sna or * (category)
                            x : 1, 2, 3, 4, 5 or * (model)
                            y : Colour, Monochrome or *
                            Z : any characters
Relay ===\PYGZgt{} RHTVT500         Name seen by VTAM applications
                            = : copied from the terminal name
*Pool name ===\PYGZgt{} *W2HPOOL    Pool where to put this terminal
Description ===\PYGZgt{} Relay pool for HTTP (3270 model 5)

Entry Point ===\PYGZgt{}            Enforced Entry Point
2nd relay ===\PYGZgt{} RHTIM500     Possible 2nd relay (Printer)
Terminal type ===\PYGZgt{} 3        1=LU1 2=3270 3=FC P=Printer S=Scs
Compression ===\PYGZgt{} 2          0, 1, 2 or 3 : compression type
Possible Calls ===\PYGZgt{} 3       0=None 1=Inbound 2=Outbound 3=Both
Write Stats to ===\PYGZgt{} 12      1,4=VIRSTAT 2=VIRLOG

Repeat ===\PYGZgt{} 0020            Number of generated terminals

P1=Update               P3=Return                           Enter=Add
                                                            P12=Server
\end{sphinxVerbatim}

\sphinxAtStartPar
\sphinxstyleemphasis{Definition of model 5 terminals in the W2HPOOL pool}

\index{Virtel Transactions@\spxentry{Virtel Transactions}!Overriding a default scenario logmode@\spxentry{Overriding a default scenario logmode}}\index{Application customization@\spxentry{Application customization}!Overriding a default scenario logmode@\spxentry{Overriding a default scenario logmode}}\index{URL Formats@\spxentry{URL Formats}!Overriding a default logmode@\spxentry{Overriding a default logmode}}\index{Scenarios@\spxentry{Scenarios}!SCENLOGM \sphinxhyphen{} Logmode defaults@\spxentry{SCENLOGM \sphinxhyphen{} Logmode defaults}}\ignorespaces 

\subsection{User\sphinxhyphen{}specified LOGMODE}
\label{\detokenize{Customization:user-specified-logmode}}\label{\detokenize{Customization:index-63}}
\sphinxAtStartPar
When the entry point definition specifies SCENLOGM in the “Identification scenario” field, the user may override the default logmode by appending an additional parameter LOGMODE=modename to the URL, as shown in this example:

\begin{sphinxVerbatim}[commandchars=\\\{\}]
http://10.20.170.71:41001/w2h/WEB3270.htm+Tso?logmode=SNX32705
\end{sphinxVerbatim}

\sphinxAtStartPar
The source code for the SCENLOGM scenario is supplied in the VIRTEL SAMPLIB.

\begin{sphinxadmonition}{note}{Note:}
\sphinxAtStartPar
To activate this functionality, SCENLOGM must be specified in the “Identification scenario” field of the ENTRY POINT (not the transaction definition).
\end{sphinxadmonition}

\index{Virtel Transactions@\spxentry{Virtel Transactions}!Overriding the default codepage@\spxentry{Overriding the default codepage}}\index{Application customization@\spxentry{Application customization}!Overriding a default codepage@\spxentry{Overriding a default codepage}}\index{URL Formats@\spxentry{URL Formats}!Overriding a default codepage@\spxentry{Overriding a default codepage}}\index{Codepage@\spxentry{Codepage}}\ignorespaces 

\subsection{User\sphinxhyphen{}specified CODEPAGE}
\label{\detokenize{Customization:user-specified-codepage}}\label{\detokenize{Customization:index-64}}
\sphinxAtStartPar
Users can override the default code page by specifying a different codepage in the URL. For example the following URL overrides the default TCT codepage with the codepage IBM0037.

\begin{sphinxVerbatim}[commandchars=\\\{\}]
http://nn.nn.nn.nn:41002/w2h/WEB2AJAX.htm+Tsop?codepage=ibm0037\PYGZam{}logmode=D4A32XX3\PYGZam{}rows=54\PYGZam{}cols=160
\end{sphinxVerbatim}

\sphinxAtStartPar
The entry point must refer to a scenario allowing to process the contents of the URL parameter CODEPAGE. By default the SCENLOGM scenario can be used. If another identification scenario is implemented, it must contain the following lines:

\begin{sphinxVerbatim}[commandchars=\\\{\}]
COPY\PYGZdl{} INPUT\PYGZhy{}TO\PYGZhy{}VARIABLE,FIELD=\PYGZsq{}CODEPAGE\PYGZsq{}, *
        VAR=\PYGZsq{}CODEPAGE\PYGZsq{}
IF\PYGZdl{} NOT\PYGZhy{}FOUND,THEN=NOCODEPG
SET\PYGZdl{} ENCODING,UTF\PYGZhy{}8,\PYGZsq{}*CODEPAGE\PYGZsq{}
\end{sphinxVerbatim}

\index{Virtel Transactions@\spxentry{Virtel Transactions}!Setting dynamic logmode and user screen sizes@\spxentry{Setting dynamic logmode and user screen sizes}}\index{Application customization@\spxentry{Application customization}!Setting dynamic logmode and user screen sizes@\spxentry{Setting dynamic logmode and user screen sizes}}\index{URL Formats@\spxentry{URL Formats}!Setting dynamic logmode and user screen sizes@\spxentry{Setting dynamic logmode and user screen sizes}}\ignorespaces 

\subsection{Dynamic logmode with user\sphinxhyphen{}specified screen size}
\label{\detokenize{Customization:dynamic-logmode-with-user-specified-screen-size}}\label{\detokenize{Customization:index-65}}
\sphinxAtStartPar
VIRTEL Web Access also supports the use of “dynamic” logmodes, such as D4A32XX3, which allow the user to specify a non\sphinxhyphen{}standard alternate screen size. When the entry point definition specifies SCENLOGM in the “Identification scenario” field, the user may also append ROWS and COLS parameters to the URL, as shown in this example:

\begin{sphinxVerbatim}[commandchars=\\\{\}]
http://10.20.170.71:41001/w2h/WEB3270.htm+Tso?logmode=D4A32XX3\PYGZam{}rows=54\PYGZam{}cols=132
\end{sphinxVerbatim}

\sphinxAtStartPar
VIRTEL allows a maximum screen size of 62 rows by 160 columns. The host application must also support the use of non\sphinxhyphen{}standard screen sizes.

\index{Virtel Transactions@\spxentry{Virtel Transactions}!The Application Menu List@\spxentry{The Application Menu List}}\index{Session Management@\spxentry{Session Management}!The Application Menu list@\spxentry{The Application Menu list}}\ignorespaces 

\subsection{Session Management In The Application Menu List}
\label{\detokenize{Customization:session-management-in-the-application-menu-list}}\label{\detokenize{Customization:index-66}}
\sphinxAtStartPar
By default, when a user selects an application from the “Application Selection Menu” the target application will replace the APPMENU application list display. Different behaviour may be required. The following sections look at some options.

\index{Virtel Transactions@\spxentry{Virtel Transactions}!Open a session in a separate tab@\spxentry{Open a session in a separate tab}}\index{Session Management@\spxentry{Session Management}!Open a session in a separate tab@\spxentry{Open a session in a separate tab}}\ignorespaces 

\subsection{Open session in separate tab}
\label{\detokenize{Customization:open-session-in-separate-tab}}\label{\detokenize{Customization:index-67}}
\sphinxAtStartPar
To open each session in separate tabs and keep the application menu available, add the following code in a customized Javascript “custom.js” file:

\begin{sphinxVerbatim}[commandchars=\\\{\}]
\PYG{o}{/}\PYG{o}{*} \PYG{n}{To} \PYG{n+nb}{open} \PYG{n}{an} \PYG{n}{application} \PYG{p}{(}\PYG{n}{issued} \PYG{k+kn}{from} \PYG{n+nn}{applist} \PYG{n}{transaction}\PYG{p}{)} \PYG{o+ow}{in} \PYG{n}{a} \PYG{n}{new} \PYG{n}{TAB} \PYG{n}{instead} \PYG{n}{of} \PYG{n}{the} \PYG{n}{same} \PYG{n}{window} \PYG{o}{*}\PYG{o}{/}
\PYG{n}{function} \PYG{n}{before\PYGZus{}launchApplink}\PYG{p}{(}\PYG{n}{href}\PYG{p}{)} \PYG{p}{\PYGZob{}}
\PYG{k}{return} \PYG{p}{\PYGZob{}}
    \PYG{n}{url}\PYG{p}{:} \PYG{n}{href}\PYG{p}{,} \PYG{o}{/}\PYG{o}{/} \PYG{n}{Return} \PYG{n}{received} \PYG{n}{URL}
    \PYG{n}{target}\PYG{p}{:} \PYG{l+s+s1}{\PYGZsq{}}\PYG{l+s+s1}{\PYGZus{}blank}\PYG{l+s+s1}{\PYGZsq{}} \PYG{o}{/}\PYG{o}{/} \PYG{n}{Target} \PYG{o+ow}{is} \PYG{n}{a} \PYG{n}{new} \PYG{n}{TAB}
    \PYG{p}{\PYGZcb{}}\PYG{p}{;}
\PYG{p}{\PYGZcb{}}
\end{sphinxVerbatim}

\sphinxAtStartPar
\sphinxstyleemphasis{Example of JavaScript code to open different sessions in separate tabs}


\subsubsection{Restrictions}
\label{\detokenize{Customization:restrictions}}
\sphinxAtStartPar
Opening simultaneous sessions in different tabs imposes certain restrictions:
\begin{itemize}
\item {} 
\sphinxAtStartPar
Browsers deliberately limit the opening of multiple simultaneous HTTP sessions on the same domain. This number varies depending on the browser itself and the version used. A detailed census is available on the BrowserScope website.

\item {} 
\sphinxAtStartPar
Each new session gives rise to the opening of a specific IP socket, and therefore the use of a separate relay terminal for each session. The LU Nailing is therefore not always possible or easy to implement in such situation.

\end{itemize}

\index{Virtel Transactions@\spxentry{Virtel Transactions}!Handling Session termination@\spxentry{Handling Session termination}}\index{Session Management@\spxentry{Session Management}!Handling Session termination@\spxentry{Handling Session termination}}\ignorespaces 

\section{How To Handle Host Session Termination}
\label{\detokenize{Customization:how-to-handle-host-session-termination}}\label{\detokenize{Customization:index-68}}
\sphinxAtStartPar
When the user terminates the application session by pressing the “Disconnect” button in the browser, various options are available: \sphinxhyphen{}
\begin{itemize}
\item {} 
\sphinxAtStartPar
Return to the application selection menu

\item {} 
\sphinxAtStartPar
Display a specific HTML page

\item {} 
\sphinxAtStartPar
Close the browser window and return to the desktop

\end{itemize}

\sphinxAtStartPar
Remember that it is always best to exit cleanly from the host application by pressing the “Disconnect” button, rather than closing the browser window. If the browser window is closed abruptly, the host session resources may not be freed until the expiry of the time\sphinxhyphen{}out period specified in the entry point definition.

\index{Virtel Transactions@\spxentry{Virtel Transactions}!Return to the Application Selection Menu@\spxentry{Return to the Application Selection Menu}}\index{Session Management@\spxentry{Session Management}!Return to the Application Selection Menu@\spxentry{Return to the Application Selection Menu}}\ignorespaces 

\subsection{Return to the application selection menu}
\label{\detokenize{Customization:return-to-the-application-selection-menu}}\label{\detokenize{Customization:index-69}}
\sphinxAtStartPar
When a “Disconnect” request is received, VIRTEL returns to the root URL and displays the default page for the line, which will normally be an application selection menu. For detailed information, see “Virtel URL formats”, page 1. The user can then choose to connect to the same or a different application by clicking on the appropriate link in the
application selection menu.

\index{Virtel Transactions@\spxentry{Virtel Transactions}!Display a specific page on Disconnection@\spxentry{Display a specific page on Disconnection}}\index{Session Management@\spxentry{Session Management}!Display a specific page on Disconnection@\spxentry{Display a specific page on Disconnection}}\ignorespaces 

\subsection{Displaying a specific page on disconnection}
\label{\detokenize{Customization:displaying-a-specific-page-on-disconnection}}\label{\detokenize{Customization:index-70}}
\sphinxAtStartPar
Those sites wishing to display a specific page at the end of a session may use the “Last page” field in the definition of the entry point associated with the HTTP line or the entry point selected by the rules of the line. The “Last page” field indicates the name of the page to be displayed following disconnection from the host application. The indicated file
must be uploaded to the same directory as specified in the URL for the host application (for example CLI\sphinxhyphen{}DIR if the URL specifies /w2h/web2ajax.htm). The “Last page” may contain instructions to the user and may include system information provided by VIRTEL (such as the application and terminal name, date and time, etc.).

\index{Virtel Transactions@\spxentry{Virtel Transactions}!Closing the browser on Disconnection@\spxentry{Closing the browser on Disconnection}}\index{Session Management@\spxentry{Session Management}!Closing the browser on Disconnection@\spxentry{Closing the browser on Disconnection}}\ignorespaces 

\subsection{Closing the browser window automatically}
\label{\detokenize{Customization:closing-the-browser-window-automatically}}\label{\detokenize{Customization:index-71}}
\sphinxAtStartPar
Sites who wish to close the browser window and return to the desktop when the user disconnects from the host application may specify close.htm in the “Last page” field of the entry point definition. This page contains JavaScript code which will attempt to close the current browser window. Depending on the browser version and security settings, the window may close, a prompt may be issued, or the window may remain open. The close.htm page is delivered as standard in the W2H\sphinxhyphen{}DIR directory but may be copied to another directory if required, for example CLI\sphinxhyphen{}DIR.

\sphinxAtStartPar
The figure below shows an example of an entry point definition with close.htm specified as the “Last page”:

\sphinxAtStartPar
\sphinxincludegraphics{{image32}.png}

\sphinxAtStartPar
\sphinxstyleemphasis{Example of entry point with last page}

\index{Virtel Macros@\spxentry{Virtel Macros}}\ignorespaces 

\chapter{Virtel Macros}
\label{\detokenize{Customization:virtel-macros}}\label{\detokenize{Customization:index-72}}

\section{Introduction}
\label{\detokenize{Customization:introduction}}
\sphinxAtStartPar
Virtel macros capture keystroke operations which can subsequently be used to automate 3270 functions. These user captured macros are stored within a file called MACROS.JSON. This file is a JavaScript array of JSON objects, with each object representing a user macro. Here is an example:

\begin{sphinxVerbatim}[commandchars=\\\{\}]
\PYG{p}{\PYGZob{}}\PYG{l+s+s2}{\PYGZdq{}}\PYG{l+s+s2}{macros}\PYG{l+s+s2}{\PYGZdq{}}\PYG{p}{:}\PYG{p}{[}
    \PYG{p}{\PYGZob{}}\PYG{l+s+s2}{\PYGZdq{}}\PYG{l+s+s2}{name}\PYG{l+s+s2}{\PYGZdq{}}\PYG{p}{:}\PYG{l+s+s2}{\PYGZdq{}}\PYG{l+s+s2}{mylogon}\PYG{l+s+s2}{\PYGZdq{}}\PYG{p}{,}\PYG{l+s+s2}{\PYGZdq{}}\PYG{l+s+s2}{rev}\PYG{l+s+s2}{\PYGZdq{}}\PYG{p}{:}\PYG{l+m+mi}{2}\PYG{p}{,}\PYG{l+s+s2}{\PYGZdq{}}\PYG{l+s+s2}{def}\PYG{l+s+s2}{\PYGZdq{}}\PYG{p}{:}\PYG{p}{[}
            \PYG{p}{\PYGZob{}}\PYG{l+s+s2}{\PYGZdq{}}\PYG{l+s+s2}{txt}\PYG{l+s+s2}{\PYGZdq{}}\PYG{p}{:}\PYG{l+s+s2}{\PYGZdq{}}\PYG{l+s+s2}{HLQ}\PYG{l+s+s2}{\PYGZdq{}}\PYG{p}{\PYGZcb{}}\PYG{p}{,}\PYG{l+s+s2}{\PYGZdq{}}\PYG{l+s+s2}{ENTER}\PYG{l+s+s2}{\PYGZdq{}}\PYG{p}{,}
            \PYG{p}{\PYGZob{}}\PYG{l+s+s2}{\PYGZdq{}}\PYG{l+s+s2}{txt}\PYG{l+s+s2}{\PYGZdq{}}\PYG{p}{:}\PYG{l+s+s2}{\PYGZdq{}}\PYG{l+s+s2}{myPassword}\PYG{l+s+s2}{\PYGZdq{}}\PYG{p}{\PYGZcb{}}\PYG{p}{,}\PYG{l+s+s2}{\PYGZdq{}}\PYG{l+s+s2}{ENTER}\PYG{l+s+s2}{\PYGZdq{}}\PYG{p}{,}\PYG{l+s+s2}{\PYGZdq{}}\PYG{l+s+s2}{ENTER}\PYG{l+s+s2}{\PYGZdq{}}\PYG{p}{,}\PYG{l+s+s2}{\PYGZdq{}}\PYG{l+s+s2}{ENTER}\PYG{l+s+s2}{\PYGZdq{}}
        \PYG{p}{]}\PYG{p}{,}
        \PYG{l+s+s2}{\PYGZdq{}}\PYG{l+s+s2}{mapping}\PYG{l+s+s2}{\PYGZdq{}}\PYG{p}{:}\PYG{p}{\PYGZob{}}\PYG{l+s+s2}{\PYGZdq{}}\PYG{l+s+s2}{key}\PYG{l+s+s2}{\PYGZdq{}}\PYG{p}{:}\PYG{l+s+s2}{\PYGZdq{}}\PYG{l+s+s2}{ctrl}\PYG{l+s+s2}{\PYGZdq{}}\PYG{p}{,}\PYG{l+s+s2}{\PYGZdq{}}\PYG{l+s+s2}{keycode}\PYG{l+s+s2}{\PYGZdq{}}\PYG{p}{:}\PYG{l+m+mi}{76}\PYG{p}{\PYGZcb{}}
    \PYG{p}{\PYGZcb{}}\PYG{p}{,}
    \PYG{p}{\PYGZob{}}\PYG{l+s+s2}{\PYGZdq{}}\PYG{l+s+s2}{name}\PYG{l+s+s2}{\PYGZdq{}}\PYG{p}{:}\PYG{l+s+s2}{\PYGZdq{}}\PYG{l+s+s2}{logoff}\PYG{l+s+s2}{\PYGZdq{}}\PYG{p}{,}\PYG{l+s+s2}{\PYGZdq{}}\PYG{l+s+s2}{rev}\PYG{l+s+s2}{\PYGZdq{}}\PYG{p}{:}\PYG{l+m+mi}{1}\PYG{p}{,}\PYG{l+s+s2}{\PYGZdq{}}\PYG{l+s+s2}{def}\PYG{l+s+s2}{\PYGZdq{}}\PYG{p}{:}\PYG{p}{[}
        \PYG{p}{\PYGZob{}}\PYG{l+s+s2}{\PYGZdq{}}\PYG{l+s+s2}{txt}\PYG{l+s+s2}{\PYGZdq{}}\PYG{p}{:}\PYG{l+s+s2}{\PYGZdq{}}\PYG{l+s+s2}{=x}\PYG{l+s+s2}{\PYGZdq{}}\PYG{p}{\PYGZcb{}}\PYG{p}{,}\PYG{l+s+s2}{\PYGZdq{}}\PYG{l+s+s2}{ENTER}\PYG{l+s+s2}{\PYGZdq{}}\PYG{p}{,}
        \PYG{p}{\PYGZob{}}\PYG{l+s+s2}{\PYGZdq{}}\PYG{l+s+s2}{txt}\PYG{l+s+s2}{\PYGZdq{}}\PYG{p}{:}\PYG{l+s+s2}{\PYGZdq{}}\PYG{l+s+s2}{logoff}\PYG{l+s+s2}{\PYGZdq{}}\PYG{p}{\PYGZcb{}}\PYG{p}{,}\PYG{l+s+s2}{\PYGZdq{}}\PYG{l+s+s2}{ENTER}\PYG{l+s+s2}{\PYGZdq{}}
        \PYG{p}{]}\PYG{p}{,}
        \PYG{l+s+s2}{\PYGZdq{}}\PYG{l+s+s2}{mapping}\PYG{l+s+s2}{\PYGZdq{}}\PYG{p}{:}\PYG{p}{\PYGZob{}}\PYG{l+s+s2}{\PYGZdq{}}\PYG{l+s+s2}{key}\PYG{l+s+s2}{\PYGZdq{}}\PYG{p}{:}\PYG{l+s+s2}{\PYGZdq{}}\PYG{l+s+s2}{ctrl}\PYG{l+s+s2}{\PYGZdq{}}\PYG{p}{,}\PYG{l+s+s2}{\PYGZdq{}}\PYG{l+s+s2}{keycode}\PYG{l+s+s2}{\PYGZdq{}}\PYG{p}{:}\PYG{l+m+mi}{79}\PYG{p}{\PYGZcb{}}
        \PYG{p}{\PYGZcb{}}\PYG{p}{,}
    \PYG{p}{\PYGZob{}}\PYG{l+s+s2}{\PYGZdq{}}\PYG{l+s+s2}{name}\PYG{l+s+s2}{\PYGZdq{}}\PYG{p}{:}\PYG{l+s+s2}{\PYGZdq{}}\PYG{l+s+s2}{logon}\PYG{l+s+s2}{\PYGZdq{}}\PYG{p}{,}\PYG{l+s+s2}{\PYGZdq{}}\PYG{l+s+s2}{rev}\PYG{l+s+s2}{\PYGZdq{}}\PYG{p}{:}\PYG{l+m+mi}{2}\PYG{p}{,}\PYG{l+s+s2}{\PYGZdq{}}\PYG{l+s+s2}{def}\PYG{l+s+s2}{\PYGZdq{}}\PYG{p}{:}\PYG{p}{[}
        \PYG{l+s+s2}{\PYGZdq{}}\PYG{l+s+s2}{Tab}\PYG{l+s+s2}{\PYGZdq{}}\PYG{p}{,}\PYG{l+s+s2}{\PYGZdq{}}\PYG{l+s+s2}{Down}\PYG{l+s+s2}{\PYGZdq{}}\PYG{p}{,}
        \PYG{p}{\PYGZob{}}\PYG{l+s+s2}{\PYGZdq{}}\PYG{l+s+s2}{txt}\PYG{l+s+s2}{\PYGZdq{}}\PYG{p}{:}\PYG{l+s+s2}{\PYGZdq{}}\PYG{l+s+s2}{sptholx}\PYG{l+s+s2}{\PYGZdq{}}\PYG{p}{\PYGZcb{}}\PYG{p}{,}\PYG{l+s+s2}{\PYGZdq{}}\PYG{l+s+s2}{ENTER}\PYG{l+s+s2}{\PYGZdq{}}\PYG{p}{,}
        \PYG{p}{\PYGZob{}}\PYG{l+s+s2}{\PYGZdq{}}\PYG{l+s+s2}{txt}\PYG{l+s+s2}{\PYGZdq{}}\PYG{p}{:}\PYG{l+s+s2}{\PYGZdq{}}\PYG{l+s+s2}{password}\PYG{l+s+s2}{\PYGZdq{}}\PYG{p}{\PYGZcb{}}\PYG{p}{,}\PYG{l+s+s2}{\PYGZdq{}}\PYG{l+s+s2}{ENTER}\PYG{l+s+s2}{\PYGZdq{}}\PYG{p}{,}\PYG{l+s+s2}{\PYGZdq{}}\PYG{l+s+s2}{ENTER}\PYG{l+s+s2}{\PYGZdq{}}\PYG{p}{,}\PYG{l+s+s2}{\PYGZdq{}}\PYG{l+s+s2}{ENTER}\PYG{l+s+s2}{\PYGZdq{}}
        \PYG{p}{]}\PYG{p}{,}
        \PYG{l+s+s2}{\PYGZdq{}}\PYG{l+s+s2}{mapping}\PYG{l+s+s2}{\PYGZdq{}}\PYG{p}{:}\PYG{p}{\PYGZob{}}\PYG{l+s+s2}{\PYGZdq{}}\PYG{l+s+s2}{key}\PYG{l+s+s2}{\PYGZdq{}}\PYG{p}{:}\PYG{l+s+s2}{\PYGZdq{}}\PYG{l+s+s2}{alt}\PYG{l+s+s2}{\PYGZdq{}}\PYG{p}{,}\PYG{l+s+s2}{\PYGZdq{}}\PYG{l+s+s2}{keycode}\PYG{l+s+s2}{\PYGZdq{}}\PYG{p}{:}\PYG{l+m+mi}{76}\PYG{p}{\PYGZcb{}}
        \PYG{p}{\PYGZcb{}}
\PYG{p}{]}\PYG{p}{,}\PYG{l+s+s2}{\PYGZdq{}}\PYG{l+s+s2}{fmt}\PYG{l+s+s2}{\PYGZdq{}}\PYG{p}{:}\PYG{l+m+mi}{2}\PYG{p}{\PYGZcb{}}
\end{sphinxVerbatim}

\index{Virtel Macros@\spxentry{Virtel Macros}!Local Macro mode@\spxentry{Local Macro mode}}\index{Local Macro mode@\spxentry{Local Macro mode}!User Interface@\spxentry{User Interface}}\ignorespaces 

\section{Macro modes}
\label{\detokenize{Customization:macro-modes}}\label{\detokenize{Customization:index-73}}

\subsection{Local Macro mode}
\label{\detokenize{Customization:local-macro-mode}}
\sphinxAtStartPar
All user macros are objects within a file called MACROS.JSON and with suitable knowledge this file can be maintained locally, known as local mode. The MACROS.JSON file is maintained through either the local macro interface, as launched by the user from the Virtel Tool Bar, or can be imported from a flat file where the MACROS.JSON file can be edited outside of these interfaces but this is not recommended unless you understand the macro structure and are familiar with JavaScript. By default Virtel maintains macros in local mode with the contents of the MACROS.JSON file being stored in the browsers local storage.


\subsubsection{User Interface}
\label{\detokenize{Customization:user-interface}}
\index{Virtel Macros@\spxentry{Virtel Macros}!Local Macro mode@\spxentry{Local Macro mode}}\index{Local Macro mode@\spxentry{Local Macro mode}!Creating a new macro@\spxentry{Creating a new macro}}\ignorespaces 
\sphinxAtStartPar
\sphinxstylestrong{Macro functions}

\sphinxAtStartPar
You can capture and list macros by using the two macro ICON functions displayed in the Virtel Tool Bar. These are the red record/stop button and the green triangular play/display button. Green indicates local mode and will display the macros held in local storage.

\sphinxAtStartPar
\sphinxincludegraphics{{image34}.png}

\sphinxAtStartPar
\sphinxstyleemphasis{The local storage record and play/display macro buttons}

\sphinxAtStartPar
The record function is an on/off button that will record key strokes. When recording, the ICON will flash until it is clicked at which point it will stop recording and save the key strokes. A save panel will be displayed asking for the name of the macro entry. Note that an ENTER/PFK key must be pressed at least once, i.e. data must be sent by some key operation in order to create and save an entry within the macro.JSON file.

\sphinxAtStartPar
\sphinxincludegraphics{{image35}.png}

\sphinxAtStartPar
\sphinxstyleemphasis{Creating a new macro}

\index{Virtel Macros@\spxentry{Virtel Macros}!Local Macro mode@\spxentry{Local Macro mode}}\index{Local Macro mode@\spxentry{Local Macro mode}!Keyboard mapping@\spxentry{Keyboard mapping}}\ignorespaces 
\sphinxAtStartPar
\sphinxstylestrong{Keyboard mapping}

\sphinxAtStartPar
When saving the macro you have the option of assigning a “hot key” or shortcut to the macro through keyboard mapping. Keyboard mapping can be a combination of ALT or CTRL keys and another keyboard key (F1 thru F12, A thru to Z, 1 thru 9). Beware that some keyboard combinations may be reserved for the operating system or Virtel functions. For example, CTRL\sphinxhyphen{}R is a browser refresh option. Allocating this combination as a hotkey will only invoke the refresh option and not the Virtel macro. Keyboard mapping is a feature that is turned on through a parameter in the w2hparm.js file. By default, keyboard mapping is set to false. To turn on keyboard mapping specify the following in the w2hparm.js member:

\begin{sphinxVerbatim}[commandchars=\\\{\}]
\PYG{n}{w2hparm}\PYG{o}{.}\PYG{n}{keymapping}\PYG{o}{=}\PYG{n}{true}
\end{sphinxVerbatim}

\sphinxAtStartPar
With keyboard mapping enabled the macro interface will display the associated key mapping against the macro.

\sphinxAtStartPar
\sphinxincludegraphics{{image36}.png}

\sphinxAtStartPar
\sphinxstyleemphasis{Saving a macro with keyboard mapping}

\index{Virtel Macros@\spxentry{Virtel Macros}!Local Macro mode@\spxentry{Local Macro mode}}\index{Local Macro mode@\spxentry{Local Macro mode}!Keyboard Mapping Option@\spxentry{Keyboard Mapping Option}}\ignorespaces 
\sphinxAtStartPar
\sphinxstylestrong{Settings for macros}

\sphinxAtStartPar
Although local macros will work “out of the box” most users would probably want to have the key mapping option set. To use this function the global options parameters must be customized to add the w2hparm.keymapping=true to the w2hparm options file. The following instructions can be used as an example as to how to set up global w2hparm customization for the CLI port 41002. Create the following two JavaScript members and upload them to the CLI\sphinxhyphen{}DIR directory:

\begin{sphinxVerbatim}[commandchars=\\\{\}]
\PYG{n}{var} \PYG{n}{w2hparm} \PYG{o}{=} \PYG{p}{\PYGZob{}}
\PYG{l+s+s2}{\PYGZdq{}}\PYG{l+s+s2}{global\PYGZhy{}settings}\PYG{l+s+s2}{\PYGZdq{}}\PYG{p}{:}\PYG{p}{\PYGZob{}}
\PYG{l+s+s2}{\PYGZdq{}}\PYG{l+s+s2}{pathToW2hparm}\PYG{l+s+s2}{\PYGZdq{}}\PYG{p}{:}\PYG{l+s+s2}{\PYGZdq{}}\PYG{l+s+s2}{../option/w2hparm.global.js}\PYG{l+s+s2}{\PYGZdq{}}\PYG{p}{,}
\PYG{p}{\PYGZcb{}}
\PYG{p}{\PYGZcb{}}\PYG{p}{;}
\end{sphinxVerbatim}

\sphinxAtStartPar
\sphinxstyleemphasis{w2hparm.js}
\begin{description}
\sphinxlineitem{::}
\sphinxAtStartPar
w2hparm.keymapping=true;

\end{description}

\sphinxAtStartPar
\sphinxstyleemphasis{w2hparm.global.js}

\sphinxAtStartPar
These setup will use the customized w2hparms for CLI from the member w2hparms.global.js. The keymapping property has been set to true.

\index{Virtel Macros@\spxentry{Virtel Macros}!Local Macro mode@\spxentry{Local Macro mode}}\index{Local Macro mode@\spxentry{Local Macro mode}!Record/Play Buttons@\spxentry{Record/Play Buttons}}\ignorespaces 
\sphinxAtStartPar
\sphinxstylestrong{The Display Record/Play Buttons}

\sphinxAtStartPar
The green triangular button will display the local storage macros. From here a context menu can be opened against each macro using the mouse right click. This will provide delete, edit, save as and run functions.

\sphinxAtStartPar
\sphinxincludegraphics{{image37}.png}

\sphinxAtStartPar
\sphinxstyleemphasis{Displaying the local macros}

\index{Virtel Macros@\spxentry{Virtel Macros}!Local Macro mode@\spxentry{Local Macro mode}}\index{Local Macro mode@\spxentry{Local Macro mode}!Export/Import Options@\spxentry{Export/Import Options}}\ignorespaces 
\sphinxAtStartPar
\sphinxstylestrong{Export and Import Options}

\sphinxAtStartPar
Macros can be exported or imported using the Export and Import buttons. On export, the MACROS.JSON file will be created. If you plan to migrate to using the DDI option you will need to export the macros and then upload the relevant MACROS.JSON file through the DDI interface.

\index{Virtel Macros@\spxentry{Virtel Macros}!centralized Macro mode@\spxentry{centralized Macro mode}}\ignorespaces 

\subsection{centralized Macros \sphinxhyphen{} DDI mode}
\label{\detokenize{Customization:centralized-macros-ddi-mode}}\label{\detokenize{Customization:index-79}}
\sphinxAtStartPar
The MACROS.JSON file can be automatically downloaded from a centralized repository on the host through the Virtel Dynamic Directory Interface (DDI). This is known as remote or DDI mode. The browser’s local storage is synchronized, via a date stamp, with the centralized VSAM repository. Management of the centralized repository is through the DDI GUI interface accessed and managed by the Virtel Administrator within the Administration Portal, normally located via port 41001. In DDI mode the macro definitions are initially initialized through an imported MACROS.JSON file. The actual centralized repository normally resides within the HTMLTRSF VSAM file made up of user, group and global directories.

\sphinxAtStartPar
The advantage of maintaining macros in a central repository is that the administrator has control over the business logic defined by the macros and can also control who has access to them through Group, Global and User profiles. As part of their Virtel interface a user can now only access site controlled macros. Each user has access to three distinct levels of a macro \sphinxhyphen{} User, Group and global. A user’s user and group level are assigned based upon their corresponding security subsystem security profiles whereas all macros are available at the global level. A user can maintain macros at their user level.

\begin{sphinxadmonition}{note}{Note:}
\sphinxAtStartPar
To use centralized DDI mode users have a userid and group defined within a security subsystem such as RACF.
\end{sphinxadmonition}

\index{Virtel Macros@\spxentry{Virtel Macros}!centralized Macro mode@\spxentry{centralized Macro mode}}\index{centralized Macro mode@\spxentry{centralized Macro mode}!Implementation@\spxentry{Implementation}}\ignorespaces 

\subsubsection{Implementation}
\label{\detokenize{Customization:implementation}}\label{\detokenize{Customization:index-80}}
\sphinxAtStartPar
To use Centralized Macros users and Administrators must have “READ” access to the relevant DDI security resources. All transactions that use DDI must be defined with at least Security=1 (Basic Security) in order that the security context can be established for the user.  Stop Virtel and run the following JCL to create these resources:

\begin{sphinxVerbatim}[commandchars=\\\{\}]
\PYG{o}{/}\PYG{o}{/}\PYG{n}{STEP0}   \PYG{n}{EXEC} \PYG{n}{PGM}\PYG{o}{=}\PYG{n}{IKJEFT01}\PYG{p}{,}\PYG{n}{DYNAMNBR}\PYG{o}{=}\PYG{l+m+mi}{20} \PYG{n}{COND}\PYG{o}{=}\PYG{n}{ONLY}
\PYG{o}{/}\PYG{o}{/}\PYG{n}{SYSTSPRT} \PYG{n}{DD}  \PYG{n}{SYSOUT}\PYG{o}{=}\PYG{o}{*}
\PYG{o}{/}\PYG{o}{/}\PYG{n}{SYSTSIN}  \PYG{n}{DD}  \PYG{o}{*}
\PYG{o}{/}\PYG{o}{*}\PYG{o}{\PYGZhy{}}\PYG{o}{\PYGZhy{}}\PYG{o}{\PYGZhy{}}\PYG{o}{\PYGZhy{}}\PYG{o}{\PYGZhy{}}\PYG{o}{\PYGZhy{}}\PYG{o}{\PYGZhy{}}\PYG{o}{\PYGZhy{}}\PYG{o}{\PYGZhy{}}\PYG{o}{\PYGZhy{}}\PYG{o}{\PYGZhy{}}\PYG{o}{\PYGZhy{}}\PYG{o}{\PYGZhy{}}\PYG{o}{\PYGZhy{}}\PYG{o}{\PYGZhy{}}\PYG{o}{\PYGZhy{}}\PYG{o}{\PYGZhy{}}\PYG{o}{\PYGZhy{}}\PYG{o}{\PYGZhy{}}\PYG{o}{\PYGZhy{}}\PYG{o}{\PYGZhy{}}\PYG{o}{\PYGZhy{}}\PYG{o}{\PYGZhy{}}\PYG{o}{\PYGZhy{}}\PYG{o}{\PYGZhy{}}\PYG{o}{\PYGZhy{}}\PYG{o}{\PYGZhy{}}\PYG{o}{\PYGZhy{}}\PYG{o}{\PYGZhy{}}\PYG{o}{\PYGZhy{}}\PYG{o}{\PYGZhy{}}\PYG{o}{\PYGZhy{}}\PYG{o}{\PYGZhy{}}\PYG{o}{\PYGZhy{}}\PYG{o}{\PYGZhy{}}\PYG{o}{\PYGZhy{}}\PYG{o}{\PYGZhy{}}\PYG{o}{\PYGZhy{}}\PYG{o}{\PYGZhy{}}\PYG{o}{\PYGZhy{}}\PYG{o}{\PYGZhy{}}\PYG{o}{\PYGZhy{}}\PYG{o}{\PYGZhy{}}\PYG{o}{\PYGZhy{}}\PYG{o}{\PYGZhy{}}\PYG{o}{\PYGZhy{}}\PYG{o}{\PYGZhy{}}\PYG{o}{\PYGZhy{}}\PYG{o}{\PYGZhy{}}\PYG{o}{\PYGZhy{}}\PYG{o}{\PYGZhy{}}\PYG{o}{\PYGZhy{}}\PYG{o}{\PYGZhy{}}\PYG{o}{\PYGZhy{}}\PYG{o}{\PYGZhy{}}\PYG{o}{*}\PYG{o}{/}
\PYG{o}{/}\PYG{o}{*} \PYG{n}{Directory} \PYG{n}{Access}                                      \PYG{o}{*}\PYG{o}{/}
\PYG{o}{/}\PYG{o}{*}\PYG{o}{\PYGZhy{}}\PYG{o}{\PYGZhy{}}\PYG{o}{\PYGZhy{}}\PYG{o}{\PYGZhy{}}\PYG{o}{\PYGZhy{}}\PYG{o}{\PYGZhy{}}\PYG{o}{\PYGZhy{}}\PYG{o}{\PYGZhy{}}\PYG{o}{\PYGZhy{}}\PYG{o}{\PYGZhy{}}\PYG{o}{\PYGZhy{}}\PYG{o}{\PYGZhy{}}\PYG{o}{\PYGZhy{}}\PYG{o}{\PYGZhy{}}\PYG{o}{\PYGZhy{}}\PYG{o}{\PYGZhy{}}\PYG{o}{\PYGZhy{}}\PYG{o}{\PYGZhy{}}\PYG{o}{\PYGZhy{}}\PYG{o}{\PYGZhy{}}\PYG{o}{\PYGZhy{}}\PYG{o}{\PYGZhy{}}\PYG{o}{\PYGZhy{}}\PYG{o}{\PYGZhy{}}\PYG{o}{\PYGZhy{}}\PYG{o}{\PYGZhy{}}\PYG{o}{\PYGZhy{}}\PYG{o}{\PYGZhy{}}\PYG{o}{\PYGZhy{}}\PYG{o}{\PYGZhy{}}\PYG{o}{\PYGZhy{}}\PYG{o}{\PYGZhy{}}\PYG{o}{\PYGZhy{}}\PYG{o}{\PYGZhy{}}\PYG{o}{\PYGZhy{}}\PYG{o}{\PYGZhy{}}\PYG{o}{\PYGZhy{}}\PYG{o}{\PYGZhy{}}\PYG{o}{\PYGZhy{}}\PYG{o}{\PYGZhy{}}\PYG{o}{\PYGZhy{}}\PYG{o}{\PYGZhy{}}\PYG{o}{\PYGZhy{}}\PYG{o}{\PYGZhy{}}\PYG{o}{\PYGZhy{}}\PYG{o}{\PYGZhy{}}\PYG{o}{\PYGZhy{}}\PYG{o}{\PYGZhy{}}\PYG{o}{\PYGZhy{}}\PYG{o}{\PYGZhy{}}\PYG{o}{\PYGZhy{}}\PYG{o}{\PYGZhy{}}\PYG{o}{\PYGZhy{}}\PYG{o}{\PYGZhy{}}\PYG{o}{\PYGZhy{}}\PYG{o}{*}\PYG{o}{/}
\PYG{n}{RDEF} \PYG{n}{FACILITY} \PYG{n}{SPVIREH}\PYG{o}{.}\PYG{n}{GLB}\PYG{o}{\PYGZhy{}}\PYG{n}{DIR} \PYG{n}{UACC}\PYG{p}{(}\PYG{n}{NONE}\PYG{p}{)} \PYG{o}{/}\PYG{o}{*} \PYG{n}{Global} \PYG{n}{Dir}\PYG{o}{.} \PYG{o}{*}\PYG{o}{/}
\PYG{n}{PE} \PYG{n}{SPVIREH}\PYG{o}{.}\PYG{n}{GLB}\PYG{o}{\PYGZhy{}}\PYG{n}{DIR} \PYG{n}{CL}\PYG{p}{(}\PYG{n}{FACILITY}\PYG{p}{)} \PYG{n}{RESET}
\PYG{n}{PE} \PYG{n}{SPVIREH}\PYG{o}{.}\PYG{n}{GLB}\PYG{o}{\PYGZhy{}}\PYG{n}{DIR} \PYG{n}{CL}\PYG{p}{(}\PYG{n}{FACILITY}\PYG{p}{)} \PYG{n}{ACC}\PYG{p}{(}\PYG{n}{READ}\PYG{p}{)} \PYG{n}{ID}\PYG{p}{(}\PYG{n}{SPGPTECH}\PYG{p}{)}
\PYG{n}{RDEF} \PYG{n}{FACILITY} \PYG{n}{SPVIREH}\PYG{o}{.}\PYG{n}{GRP}\PYG{o}{\PYGZhy{}}\PYG{n}{DIR} \PYG{n}{UACC}\PYG{p}{(}\PYG{n}{NONE}\PYG{p}{)} \PYG{o}{/}\PYG{o}{*} \PYG{n}{Group} \PYG{n}{Dir}\PYG{o}{.}  \PYG{o}{*}\PYG{o}{/}
\PYG{n}{PE} \PYG{n}{SPVIREH}\PYG{o}{.}\PYG{n}{GRP}\PYG{o}{\PYGZhy{}}\PYG{n}{DIR} \PYG{n}{CL}\PYG{p}{(}\PYG{n}{FACILITY}\PYG{p}{)} \PYG{n}{RESET}
\PYG{n}{PE} \PYG{n}{SPVIREH}\PYG{o}{.}\PYG{n}{GRP}\PYG{o}{\PYGZhy{}}\PYG{n}{DIR} \PYG{n}{CL}\PYG{p}{(}\PYG{n}{FACILITY}\PYG{p}{)} \PYG{n}{ACC}\PYG{p}{(}\PYG{n}{READ}\PYG{p}{)} \PYG{n}{ID}\PYG{p}{(}\PYG{n}{SPGPTECH}\PYG{p}{)}
\PYG{n}{RDEF} \PYG{n}{FACILITY} \PYG{n}{SPVIREH}\PYG{o}{.}\PYG{n}{USR}\PYG{o}{\PYGZhy{}}\PYG{n}{DIR} \PYG{n}{UACC}\PYG{p}{(}\PYG{n}{NONE}\PYG{p}{)} \PYG{o}{/}\PYG{o}{*} \PYG{n}{Global} \PYG{n}{Dir}\PYG{o}{.} \PYG{o}{*}\PYG{o}{/}
\PYG{n}{PE} \PYG{n}{SPVIREH}\PYG{o}{.}\PYG{n}{USR}\PYG{o}{\PYGZhy{}}\PYG{n}{DIR} \PYG{n}{CL}\PYG{p}{(}\PYG{n}{FACILITY}\PYG{p}{)} \PYG{n}{RESET}
\PYG{n}{PE} \PYG{n}{SPVIREH}\PYG{o}{.}\PYG{n}{USR}\PYG{o}{\PYGZhy{}}\PYG{n}{DIR} \PYG{n}{CL}\PYG{p}{(}\PYG{n}{FACILITY}\PYG{p}{)} \PYG{n}{ACC}\PYG{p}{(}\PYG{n}{READ}\PYG{p}{)} \PYG{n}{ID}\PYG{p}{(}\PYG{n}{SPGPTECH}\PYG{p}{)}
\PYG{o}{/}\PYG{o}{*}\PYG{o}{\PYGZhy{}}\PYG{o}{\PYGZhy{}}\PYG{o}{\PYGZhy{}}\PYG{o}{\PYGZhy{}}\PYG{o}{\PYGZhy{}}\PYG{o}{\PYGZhy{}}\PYG{o}{\PYGZhy{}}\PYG{o}{\PYGZhy{}}\PYG{o}{\PYGZhy{}}\PYG{o}{\PYGZhy{}}\PYG{o}{\PYGZhy{}}\PYG{o}{\PYGZhy{}}\PYG{o}{\PYGZhy{}}\PYG{o}{\PYGZhy{}}\PYG{o}{\PYGZhy{}}\PYG{o}{\PYGZhy{}}\PYG{o}{\PYGZhy{}}\PYG{o}{\PYGZhy{}}\PYG{o}{\PYGZhy{}}\PYG{o}{\PYGZhy{}}\PYG{o}{\PYGZhy{}}\PYG{o}{\PYGZhy{}}\PYG{o}{\PYGZhy{}}\PYG{o}{\PYGZhy{}}\PYG{o}{\PYGZhy{}}\PYG{o}{\PYGZhy{}}\PYG{o}{\PYGZhy{}}\PYG{o}{\PYGZhy{}}\PYG{o}{\PYGZhy{}}\PYG{o}{\PYGZhy{}}\PYG{o}{\PYGZhy{}}\PYG{o}{\PYGZhy{}}\PYG{o}{\PYGZhy{}}\PYG{o}{\PYGZhy{}}\PYG{o}{\PYGZhy{}}\PYG{o}{\PYGZhy{}}\PYG{o}{\PYGZhy{}}\PYG{o}{\PYGZhy{}}\PYG{o}{\PYGZhy{}}\PYG{o}{\PYGZhy{}}\PYG{o}{\PYGZhy{}}\PYG{o}{\PYGZhy{}}\PYG{o}{\PYGZhy{}}\PYG{o}{\PYGZhy{}}\PYG{o}{\PYGZhy{}}\PYG{o}{\PYGZhy{}}\PYG{o}{\PYGZhy{}}\PYG{o}{\PYGZhy{}}\PYG{o}{\PYGZhy{}}\PYG{o}{\PYGZhy{}}\PYG{o}{\PYGZhy{}}\PYG{o}{\PYGZhy{}}\PYG{o}{\PYGZhy{}}\PYG{o}{\PYGZhy{}}\PYG{o}{\PYGZhy{}}\PYG{o}{*}\PYG{o}{/}
\PYG{o}{/}\PYG{o}{*} \PYG{n}{DDI} \PYG{o}{/} \PYG{n}{Capability}                                      \PYG{o}{*}\PYG{o}{/}
\PYG{o}{/}\PYG{o}{*}\PYG{o}{\PYGZhy{}}\PYG{o}{\PYGZhy{}}\PYG{o}{\PYGZhy{}}\PYG{o}{\PYGZhy{}}\PYG{o}{\PYGZhy{}}\PYG{o}{\PYGZhy{}}\PYG{o}{\PYGZhy{}}\PYG{o}{\PYGZhy{}}\PYG{o}{\PYGZhy{}}\PYG{o}{\PYGZhy{}}\PYG{o}{\PYGZhy{}}\PYG{o}{\PYGZhy{}}\PYG{o}{\PYGZhy{}}\PYG{o}{\PYGZhy{}}\PYG{o}{\PYGZhy{}}\PYG{o}{\PYGZhy{}}\PYG{o}{\PYGZhy{}}\PYG{o}{\PYGZhy{}}\PYG{o}{\PYGZhy{}}\PYG{o}{\PYGZhy{}}\PYG{o}{\PYGZhy{}}\PYG{o}{\PYGZhy{}}\PYG{o}{\PYGZhy{}}\PYG{o}{\PYGZhy{}}\PYG{o}{\PYGZhy{}}\PYG{o}{\PYGZhy{}}\PYG{o}{\PYGZhy{}}\PYG{o}{\PYGZhy{}}\PYG{o}{\PYGZhy{}}\PYG{o}{\PYGZhy{}}\PYG{o}{\PYGZhy{}}\PYG{o}{\PYGZhy{}}\PYG{o}{\PYGZhy{}}\PYG{o}{\PYGZhy{}}\PYG{o}{\PYGZhy{}}\PYG{o}{\PYGZhy{}}\PYG{o}{\PYGZhy{}}\PYG{o}{\PYGZhy{}}\PYG{o}{\PYGZhy{}}\PYG{o}{\PYGZhy{}}\PYG{o}{\PYGZhy{}}\PYG{o}{\PYGZhy{}}\PYG{o}{\PYGZhy{}}\PYG{o}{\PYGZhy{}}\PYG{o}{\PYGZhy{}}\PYG{o}{\PYGZhy{}}\PYG{o}{\PYGZhy{}}\PYG{o}{\PYGZhy{}}\PYG{o}{\PYGZhy{}}\PYG{o}{\PYGZhy{}}\PYG{o}{\PYGZhy{}}\PYG{o}{\PYGZhy{}}\PYG{o}{\PYGZhy{}}\PYG{o}{\PYGZhy{}}\PYG{o}{\PYGZhy{}}\PYG{o}{*}\PYG{o}{/}
\PYG{n}{RDEF} \PYG{n}{FACILITY} \PYG{n}{SPVIREH}\PYG{o}{.}\PYG{n}{W2H}\PYG{o}{\PYGZhy{}}\PYG{l+m+mi}{07}  \PYG{n}{UACC}\PYG{p}{(}\PYG{n}{NONE}\PYG{p}{)}  \PYG{o}{/}\PYG{o}{*} \PYG{n}{DDI}         \PYG{o}{*}\PYG{o}{/}
\PYG{n}{PE} \PYG{n}{SPVIREH}\PYG{o}{.}\PYG{n}{W2H}\PYG{o}{\PYGZhy{}}\PYG{l+m+mi}{07} \PYG{n}{CL}\PYG{p}{(}\PYG{n}{FACILITY}\PYG{p}{)} \PYG{n}{RESET}
\PYG{n}{PE} \PYG{n}{SPVIREH}\PYG{o}{.}\PYG{n}{W2H}\PYG{o}{\PYGZhy{}}\PYG{l+m+mi}{07} \PYG{n}{CL}\PYG{p}{(}\PYG{n}{FACILITY}\PYG{p}{)} \PYG{n}{ACC}\PYG{p}{(}\PYG{n}{READ}\PYG{p}{)} \PYG{n}{ID}\PYG{p}{(}\PYG{n}{SPGPTECH}\PYG{p}{)}
\PYG{n}{RDEF} \PYG{n}{FACILITY} \PYG{n}{SPVIREH}\PYG{o}{.}\PYG{n}{W2H}\PYG{o}{\PYGZhy{}}\PYG{l+m+mi}{66}  \PYG{n}{UACC}\PYG{p}{(}\PYG{n}{NONE}\PYG{p}{)}  \PYG{o}{/}\PYG{o}{*} \PYG{n}{DDI}         \PYG{o}{*}\PYG{o}{/}
\PYG{n}{PE} \PYG{n}{SPVIREH}\PYG{o}{.}\PYG{n}{W2H}\PYG{o}{\PYGZhy{}}\PYG{l+m+mi}{66} \PYG{n}{CL}\PYG{p}{(}\PYG{n}{FACILITY}\PYG{p}{)} \PYG{n}{RESET}
\PYG{n}{PE} \PYG{n}{SPVIREH}\PYG{o}{.}\PYG{n}{W2H}\PYG{o}{\PYGZhy{}}\PYG{l+m+mi}{66} \PYG{n}{CL}\PYG{p}{(}\PYG{n}{FACILITY}\PYG{p}{)} \PYG{n}{ACC}\PYG{p}{(}\PYG{n}{READ}\PYG{p}{)} \PYG{n}{ID}\PYG{p}{(}\PYG{n}{SPGPTECH}\PYG{p}{)}
\PYG{o}{/}\PYG{o}{*}\PYG{o}{\PYGZhy{}}\PYG{o}{\PYGZhy{}}\PYG{o}{\PYGZhy{}}\PYG{o}{\PYGZhy{}}\PYG{o}{\PYGZhy{}}\PYG{o}{\PYGZhy{}}\PYG{o}{\PYGZhy{}}\PYG{o}{\PYGZhy{}}\PYG{o}{\PYGZhy{}}\PYG{o}{\PYGZhy{}}\PYG{o}{\PYGZhy{}}\PYG{o}{\PYGZhy{}}\PYG{o}{\PYGZhy{}}\PYG{o}{\PYGZhy{}}\PYG{o}{\PYGZhy{}}\PYG{o}{\PYGZhy{}}\PYG{o}{\PYGZhy{}}\PYG{o}{\PYGZhy{}}\PYG{o}{\PYGZhy{}}\PYG{o}{\PYGZhy{}}\PYG{o}{\PYGZhy{}}\PYG{o}{\PYGZhy{}}\PYG{o}{\PYGZhy{}}\PYG{o}{\PYGZhy{}}\PYG{o}{\PYGZhy{}}\PYG{o}{\PYGZhy{}}\PYG{o}{\PYGZhy{}}\PYG{o}{\PYGZhy{}}\PYG{o}{\PYGZhy{}}\PYG{o}{\PYGZhy{}}\PYG{o}{\PYGZhy{}}\PYG{o}{\PYGZhy{}}\PYG{o}{\PYGZhy{}}\PYG{o}{\PYGZhy{}}\PYG{o}{\PYGZhy{}}\PYG{o}{\PYGZhy{}}\PYG{o}{\PYGZhy{}}\PYG{o}{\PYGZhy{}}\PYG{o}{\PYGZhy{}}\PYG{o}{\PYGZhy{}}\PYG{o}{\PYGZhy{}}\PYG{o}{\PYGZhy{}}\PYG{o}{\PYGZhy{}}\PYG{o}{\PYGZhy{}}\PYG{o}{\PYGZhy{}}\PYG{o}{\PYGZhy{}}\PYG{o}{\PYGZhy{}}\PYG{o}{\PYGZhy{}}\PYG{o}{\PYGZhy{}}\PYG{o}{\PYGZhy{}}\PYG{o}{\PYGZhy{}}\PYG{o}{\PYGZhy{}}\PYG{o}{\PYGZhy{}}\PYG{o}{\PYGZhy{}}\PYG{o}{\PYGZhy{}}\PYG{o}{*}\PYG{o}{/}
\PYG{o}{/}\PYG{o}{*} \PYG{n}{Administrators} \PYG{n}{Upload}                                 \PYG{o}{*}\PYG{o}{/}
\PYG{o}{/}\PYG{o}{*}\PYG{o}{\PYGZhy{}}\PYG{o}{\PYGZhy{}}\PYG{o}{\PYGZhy{}}\PYG{o}{\PYGZhy{}}\PYG{o}{\PYGZhy{}}\PYG{o}{\PYGZhy{}}\PYG{o}{\PYGZhy{}}\PYG{o}{\PYGZhy{}}\PYG{o}{\PYGZhy{}}\PYG{o}{\PYGZhy{}}\PYG{o}{\PYGZhy{}}\PYG{o}{\PYGZhy{}}\PYG{o}{\PYGZhy{}}\PYG{o}{\PYGZhy{}}\PYG{o}{\PYGZhy{}}\PYG{o}{\PYGZhy{}}\PYG{o}{\PYGZhy{}}\PYG{o}{\PYGZhy{}}\PYG{o}{\PYGZhy{}}\PYG{o}{\PYGZhy{}}\PYG{o}{\PYGZhy{}}\PYG{o}{\PYGZhy{}}\PYG{o}{\PYGZhy{}}\PYG{o}{\PYGZhy{}}\PYG{o}{\PYGZhy{}}\PYG{o}{\PYGZhy{}}\PYG{o}{\PYGZhy{}}\PYG{o}{\PYGZhy{}}\PYG{o}{\PYGZhy{}}\PYG{o}{\PYGZhy{}}\PYG{o}{\PYGZhy{}}\PYG{o}{\PYGZhy{}}\PYG{o}{\PYGZhy{}}\PYG{o}{\PYGZhy{}}\PYG{o}{\PYGZhy{}}\PYG{o}{\PYGZhy{}}\PYG{o}{\PYGZhy{}}\PYG{o}{\PYGZhy{}}\PYG{o}{\PYGZhy{}}\PYG{o}{\PYGZhy{}}\PYG{o}{\PYGZhy{}}\PYG{o}{\PYGZhy{}}\PYG{o}{\PYGZhy{}}\PYG{o}{\PYGZhy{}}\PYG{o}{\PYGZhy{}}\PYG{o}{\PYGZhy{}}\PYG{o}{\PYGZhy{}}\PYG{o}{\PYGZhy{}}\PYG{o}{\PYGZhy{}}\PYG{o}{\PYGZhy{}}\PYG{o}{\PYGZhy{}}\PYG{o}{\PYGZhy{}}\PYG{o}{\PYGZhy{}}\PYG{o}{\PYGZhy{}}\PYG{o}{\PYGZhy{}}\PYG{o}{*}\PYG{o}{/}
\PYG{n}{RDEF} \PYG{n}{FACILITY} \PYG{n}{SPVIREH}\PYG{o}{.}\PYG{n}{W2H}\PYG{o}{\PYGZhy{}}\PYG{l+m+mi}{80}\PYG{n}{A} \PYG{n}{UACC}\PYG{p}{(}\PYG{n}{NONE}\PYG{p}{)} \PYG{o}{/}\PYG{o}{*} \PYG{n}{Global} \PYG{n}{Dir}\PYG{o}{.} \PYG{o}{*}\PYG{o}{/}
\PYG{n}{PE} \PYG{n}{SPVIREH}\PYG{o}{.}\PYG{n}{W2H}\PYG{o}{\PYGZhy{}}\PYG{l+m+mi}{80}\PYG{n}{A} \PYG{n}{CL}\PYG{p}{(}\PYG{n}{FACILITY}\PYG{p}{)} \PYG{n}{RESET}
\PYG{n}{PE} \PYG{n}{SPVIREH}\PYG{o}{.}\PYG{n}{W2H}\PYG{o}{\PYGZhy{}}\PYG{l+m+mi}{80}\PYG{n}{A} \PYG{n}{CL}\PYG{p}{(}\PYG{n}{FACILITY}\PYG{p}{)} \PYG{n}{ACC}\PYG{p}{(}\PYG{n}{READ}\PYG{p}{)} \PYG{n}{ID}\PYG{p}{(}\PYG{n}{SPGPTECH}\PYG{p}{)}
\PYG{n}{RDEF} \PYG{n}{FACILITY} \PYG{n}{SPVIREH}\PYG{o}{.}\PYG{n}{W2H}\PYG{o}{\PYGZhy{}}\PYG{l+m+mi}{80}\PYG{n}{G} \PYG{n}{UACC}\PYG{p}{(}\PYG{n}{NONE}\PYG{p}{)} \PYG{o}{/}\PYG{o}{*} \PYG{n}{Global} \PYG{n}{Dir}\PYG{o}{.} \PYG{o}{*}\PYG{o}{/}
\PYG{n}{PE} \PYG{n}{SPVIREH}\PYG{o}{.}\PYG{n}{W2H}\PYG{o}{\PYGZhy{}}\PYG{l+m+mi}{80}\PYG{n}{G} \PYG{n}{CL}\PYG{p}{(}\PYG{n}{FACILITY}\PYG{p}{)} \PYG{n}{RESET}
\PYG{n}{PE} \PYG{n}{SPVIREH}\PYG{o}{.}\PYG{n}{W2H}\PYG{o}{\PYGZhy{}}\PYG{l+m+mi}{80}\PYG{n}{G} \PYG{n}{CL}\PYG{p}{(}\PYG{n}{FACILITY}\PYG{p}{)} \PYG{n}{ACC}\PYG{p}{(}\PYG{n}{READ}\PYG{p}{)} \PYG{n}{ID}\PYG{p}{(}\PYG{n}{SPGPTECH}\PYG{p}{)}
\PYG{n}{RDEF} \PYG{n}{FACILITY} \PYG{n}{SPVIREH}\PYG{o}{.}\PYG{n}{W2H}\PYG{o}{\PYGZhy{}}\PYG{l+m+mi}{80}\PYG{n}{U} \PYG{n}{UACC}\PYG{p}{(}\PYG{n}{NONE}\PYG{p}{)} \PYG{o}{/}\PYG{o}{*} \PYG{n}{Global} \PYG{n}{Dir}\PYG{o}{.} \PYG{o}{*}\PYG{o}{/}
\PYG{n}{PE} \PYG{n}{SPVIREH}\PYG{o}{.}\PYG{n}{W2H}\PYG{o}{\PYGZhy{}}\PYG{l+m+mi}{80}\PYG{n}{U} \PYG{n}{CL}\PYG{p}{(}\PYG{n}{FACILITY}\PYG{p}{)} \PYG{n}{RESET}
\PYG{n}{PE} \PYG{n}{SPVIREH}\PYG{o}{.}\PYG{n}{W2H}\PYG{o}{\PYGZhy{}}\PYG{l+m+mi}{80}\PYG{n}{U} \PYG{n}{CL}\PYG{p}{(}\PYG{n}{FACILITY}\PYG{p}{)} \PYG{n}{ACC}\PYG{p}{(}\PYG{n}{READ}\PYG{p}{)} \PYG{n}{ID}\PYG{p}{(}\PYG{n}{SPGPTECH}\PYG{p}{)}
\end{sphinxVerbatim}

\sphinxAtStartPar
\sphinxstyleemphasis{Security Resources required for DDI}

\index{Virtel Macros@\spxentry{Virtel Macros}!Centralized Macro mode@\spxentry{Centralized Macro mode}}\index{Centralized Macro mode@\spxentry{Centralized Macro mode}!Directories@\spxentry{Directories}}\ignorespaces 
\sphinxAtStartPar
Additional Virtel transactions and directories are required to support DDI. These can be implemented by running the ARBOLOAD job, located in the Virtel CNTL library, with VMACROS=YES. After executing this job start Virtel and access the Drag and Drop GUI from the Administration Portal on 41001. You should now see additional GRP\sphinxhyphen{}DIR, GLB\sphinxhyphen{}DIR and USR\sphinxhyphen{}DIR directories.

\sphinxAtStartPar
\sphinxincludegraphics{{image38}.png}

\sphinxAtStartPar
\sphinxstyleemphasis{DDI Directories}

\index{Virtel Macros@\spxentry{Virtel Macros}!Centralized Macro mode@\spxentry{Centralized Macro mode}}\index{Centralized Macro mode@\spxentry{Centralized Macro mode}!Setting options@\spxentry{Setting options}}\ignorespaces 
\sphinxAtStartPar
Once the DDI directories are set up will can configure the necessary settings to invoke DDI. A w2hParm.global.js object is uploaded to the CLI\_DIR and will contain the the DDI settings. We also need to ensure that our global\sphinxhyphen{}settings object is defined in the wh2parm.js member. The w2hparm.js in CLI\sphinxhyphen{}DIR looks like this:

\begin{sphinxVerbatim}[commandchars=\\\{\}]
\PYG{o}{/}\PYG{o}{/} \PYG{n}{NOTE}\PYG{p}{:} \PYG{n}{this} \PYG{n}{file} \PYG{o+ow}{is} \PYG{n}{a} \PYG{n}{sample} \PYG{o+ow}{and} \PYG{n}{will} \PYG{n}{be} \PYG{n}{replaced} \PYG{n}{after} \PYG{n}{a} \PYG{n}{new} \PYG{n}{install}\PYG{o}{.}
\PYG{o}{/}\PYG{o}{/} \PYG{n}{please} \PYG{n}{see} \PYG{n}{the} \PYG{n}{documentation} \PYG{n}{on} \PYG{n}{possible} \PYG{n}{parameter} \PYG{n}{values}
\PYG{o}{/}\PYG{o}{/} \PYG{o+ow}{and} \PYG{n}{on} \PYG{n}{how} \PYG{n}{to} \PYG{n+nb}{set} \PYG{n}{this} \PYG{n}{file} \PYG{n}{up} \PYG{o+ow}{in} \PYG{n}{CLI}\PYG{o}{\PYGZhy{}}\PYG{n}{DIR} \PYG{n}{instead} \PYG{n}{of} \PYG{n}{W2H}\PYG{o}{\PYGZhy{}}\PYG{n}{DIR}
\PYG{o}{/}\PYG{o}{/}
\PYG{o}{/}\PYG{o}{/}\PYG{n}{var} \PYG{n}{w2hparm} \PYG{o}{=} \PYG{p}{\PYGZob{}}\PYG{p}{\PYGZcb{}}\PYG{p}{;}
\PYG{n}{var} \PYG{n}{w2hparm} \PYG{o}{=} \PYG{p}{\PYGZob{}}
     \PYG{l+s+s2}{\PYGZdq{}}\PYG{l+s+s2}{global\PYGZhy{}settings}\PYG{l+s+s2}{\PYGZdq{}}\PYG{p}{:}\PYG{p}{\PYGZob{}}
        \PYG{l+s+s2}{\PYGZdq{}}\PYG{l+s+s2}{pathToW2hparm}\PYG{l+s+s2}{\PYGZdq{}}\PYG{p}{:}\PYG{l+s+s2}{\PYGZdq{}}\PYG{l+s+s2}{../option/w2hparm.global.js}\PYG{l+s+s2}{\PYGZdq{}}
    \PYG{p}{\PYGZcb{}}
\PYG{p}{\PYGZcb{}}\PYG{p}{;}
\end{sphinxVerbatim}

\sphinxAtStartPar
\sphinxstyleemphasis{Sample w2hparm.js setting}

\sphinxAtStartPar
Our w2hparm.global.js member looks like the following example. This too is also uploaded to the CLI\sphinxhyphen{}DIR:

\begin{sphinxVerbatim}[commandchars=\\\{\}]
\PYG{o}{/}\PYG{o}{/} \PYG{n}{w2hparm}\PYG{o}{.}\PYG{k}{global}\PYG{o}{.}\PYG{n}{js} \PYG{p}{(}\PYG{n}{CLI}\PYG{o}{\PYGZhy{}}\PYG{n}{DIR}\PYG{p}{)}
\PYG{o}{/}\PYG{o}{/} \PYG{n}{w2h} \PYG{n}{parameters} \PYG{n}{to} \PYG{n}{override} \PYG{n}{defaults} \PYG{o+ow}{and} \PYG{n}{add} \PYG{n}{DDI} \PYG{n}{macro} \PYG{n}{support}\PYG{o}{.}
\PYG{n}{var} \PYG{n}{w2hparm} \PYG{o}{=} \PYG{p}{\PYGZob{}}\PYG{l+s+s2}{\PYGZdq{}}\PYG{l+s+s2}{ctrl}\PYG{l+s+s2}{\PYGZdq{}}\PYG{p}{:}\PYG{l+s+s2}{\PYGZdq{}}\PYG{l+s+s2}{Newline}\PYG{l+s+s2}{\PYGZdq{}}\PYG{p}{,}
    \PYG{l+s+s2}{\PYGZdq{}}\PYG{l+s+s2}{enter}\PYG{l+s+s2}{\PYGZdq{}}\PYG{p}{:}\PYG{l+s+s2}{\PYGZdq{}}\PYG{l+s+s2}{ENTER}\PYG{l+s+s2}{\PYGZdq{}}\PYG{p}{,}
    \PYG{l+s+s2}{\PYGZdq{}}\PYG{l+s+s2}{pgup}\PYG{l+s+s2}{\PYGZdq{}}\PYG{p}{:}\PYG{l+s+s2}{\PYGZdq{}}\PYG{l+s+s2}{PA1}\PYG{l+s+s2}{\PYGZdq{}}\PYG{p}{,}
    \PYG{l+s+s2}{\PYGZdq{}}\PYG{l+s+s2}{pgdn}\PYG{l+s+s2}{\PYGZdq{}}\PYG{p}{:}\PYG{l+s+s2}{\PYGZdq{}}\PYG{l+s+s2}{PA2}\PYG{l+s+s2}{\PYGZdq{}}\PYG{p}{,}
    \PYG{l+s+s2}{\PYGZdq{}}\PYG{l+s+s2}{useVirtelMacros}\PYG{l+s+s2}{\PYGZdq{}}\PYG{p}{:}\PYG{p}{\PYGZob{}}\PYG{l+s+s2}{\PYGZdq{}}\PYG{l+s+s2}{macrosAutoRefresh}\PYG{l+s+s2}{\PYGZdq{}}\PYG{p}{:}\PYG{l+s+s2}{\PYGZdq{}}\PYG{l+s+s2}{session}\PYG{l+s+s2}{\PYGZdq{}}\PYG{p}{\PYGZcb{}}
\PYG{p}{\PYGZcb{}}\PYG{p}{;}
\PYG{n}{w2hparm}\PYG{o}{.}\PYG{n}{keymapping}\PYG{o}{=}\PYG{n}{true}\PYG{p}{;}
\end{sphinxVerbatim}

\sphinxAtStartPar
The “useVirtelMacros” is the key property required for DDI. The various values for this key name control the synchronisation between the user’s local macro storage and the DDI central repository. Specify one of the following options:

\begin{sphinxVerbatim}[commandchars=\\\{\}]
\PYGZdq{}w2hparm.useVirtelMacros\PYGZdq{}:\PYGZob{}“macrosAutoRefresh”: “never” | “once” | “daily” | “session” \PYGZcb{}
\end{sphinxVerbatim}

\index{Virtel Macros@\spxentry{Virtel Macros}!centralized Macro mode@\spxentry{centralized Macro mode}}\index{centralized Macro mode@\spxentry{centralized Macro mode}!Setting options {[}useVirtelMacros{]}@\spxentry{Setting options {[}useVirtelMacros{]}}}\ignorespaces 
\sphinxAtStartPar
\sphinxstylestrong{Macro Options}


\begin{savenotes}\sphinxattablestart
\sphinxthistablewithglobalstyle
\centering
\begin{tabulary}{\linewidth}[t]{TT}
\sphinxtoprule
\sphinxstyletheadfamily 
\sphinxAtStartPar
\sphinxstylestrong{Option}
&\sphinxstyletheadfamily 
\sphinxAtStartPar
\sphinxstylestrong{Description}
\\
\sphinxmidrule
\sphinxtableatstartofbodyhook
\sphinxAtStartPar
Never
&
\sphinxAtStartPar
Do not synchronize with DDI unless the user presses the refresh button. Default
\\
\sphinxhline
\sphinxAtStartPar
Once
&
\sphinxAtStartPar
Synchronize with DDI only when local storage hasn’t been initialized
\\
\sphinxhline
\sphinxAtStartPar
Daily
&
\sphinxAtStartPar
Once a Day
\\
\sphinxhline
\sphinxAtStartPar
Session
&
\sphinxAtStartPar
Every time a new browser session is opened
\\
\sphinxbottomrule
\end{tabulary}
\sphinxtableafterendhook\par
\sphinxattableend\end{savenotes}

\sphinxAtStartPar
Two other settings that can be used with DDI are: \sphinxhyphen{}
\begin{itemize}
\item {} 
\sphinxAtStartPar
\sphinxstylestrong{w2hparm.keymapping=true | false}                        Turn on keymapping
\begin{quote}

\sphinxAtStartPar
This option turns on the ability to assign a macro with a shortcut or “hot key” made up of a combination of keys; such as ALT+F1, or CTRL+A.  Beware that some keyboard combinations may be reserved for the operating system or other Virtel functions. For example, CTRL\sphinxhyphen{}R is a browser refresh option.
\end{quote}

\item {} 
\sphinxAtStartPar
\sphinxstylestrong{w2hparm.synchronizeVirtelMacros=true | false}           Synchronize all centralized Repositories
\begin{quote}

\sphinxAtStartPar
If you are running multiple images of Virtel, say in a SYSPLEX arrangement using separate centralized repositories then the “synchronizeVirtelMacros” option should be set to true. This ensures that macro changes are reflected in all DDI repositories and associated local storage.
\end{quote}

\end{itemize}

\index{Virtel Macros@\spxentry{Virtel Macros}!centralized Macro mode@\spxentry{centralized Macro mode}}\index{centralized Macro mode@\spxentry{centralized Macro mode}!DDI Validation@\spxentry{DDI Validation}}\ignorespaces 

\subsection{centralized DDI validation}
\label{\detokenize{Customization:centralized-ddi-validation}}\label{\detokenize{Customization:index-84}}
\sphinxAtStartPar
After uploading the settings objects stop and restart Virtel and then open a TSO or CICS session through the client port 41002.

\begin{sphinxVerbatim}[commandchars=\\\{\}]
\PYG{n}{http}\PYG{p}{:}\PYG{o}{/}\PYG{o}{/}\PYG{l+m+mf}{192.168}\PYG{l+m+mf}{.170}\PYG{l+m+mf}{.48}\PYG{p}{:}\PYG{l+m+mi}{41002}\PYG{o}{/}\PYG{n}{w2h}\PYG{o}{/}\PYG{n}{WEB2AJAX}\PYG{o}{.}\PYG{n}{htm}\PYG{o}{+}\PYG{n}{TSO}
\end{sphinxVerbatim}

\sphinxAtStartPar
The macro ICON should now be blue, indicating centralized DDI has been correctly configured. In local mode this ICON is green.

\sphinxAtStartPar
\sphinxincludegraphics{{image39}.png}

\sphinxAtStartPar
\sphinxstyleemphasis{Centralized DDI is active \sphinxhyphen{} ICON is blue}

\sphinxAtStartPar
Pressing the Blue ICON should display an empty macro list:

\sphinxAtStartPar
\sphinxincludegraphics{{image40}.png}

\sphinxAtStartPar
\sphinxstyleemphasis{Empty Macro list}

\index{Virtel Macros@\spxentry{Virtel Macros}!centralized Macro mode@\spxentry{centralized Macro mode}}\index{centralized Macro mode@\spxentry{centralized Macro mode}!Configuring the centralized Repository@\spxentry{Configuring the centralized Repository}}\ignorespaces 
\sphinxAtStartPar
Configuring the centralized Repository

\sphinxAtStartPar
To configure the centralized DDI repository we need to create an “initial” MACROS.JSON file. If we load up a TSO or CICS transaction on the W2H line (41001) we can see that the Macro ICON is green, its running in local mode, which is what we would also expect as centralized macros are not active on this port. We will use this line to create some initial macros to load up into our centralized repository on the mainframe. Remember that in centralized DDI mode, macros are downloaded from the repository. Users do not have any capability of “writing” or exporting their own macros up to the repository. This is administered by the Virtel administrator.

\begin{sphinxVerbatim}[commandchars=\\\{\}]
\PYG{n}{http}\PYG{p}{:}\PYG{o}{/}\PYG{o}{/}\PYG{l+m+mf}{192.168}\PYG{l+m+mf}{.170}\PYG{l+m+mf}{.48}\PYG{p}{:}\PYG{l+m+mi}{41001}\PYG{o}{/}\PYG{n}{w2h}\PYG{o}{/}\PYG{n}{WEB2AJAX}\PYG{o}{.}\PYG{n}{htm}\PYG{o}{+}\PYG{n}{Tso}
\end{sphinxVerbatim}

\sphinxAtStartPar
\sphinxincludegraphics{{image41}.png}

\sphinxAtStartPar
\sphinxstyleemphasis{Accessing a local macro facility \sphinxhyphen{}  Green ICON}

\index{Virtel Macros@\spxentry{Virtel Macros}!centralized Macro mode@\spxentry{centralized Macro mode}}\index{centralized Macro mode@\spxentry{centralized Macro mode}!Creating an initial macro@\spxentry{Creating an initial macro}}\ignorespaces 

\subsection{Creating an initial macro}
\label{\detokenize{Customization:creating-an-initial-macro}}\label{\detokenize{Customization:index-86}}

\subsubsection{SDSF test macro}
\label{\detokenize{Customization:sdsf-test-macro}}
\sphinxAtStartPar
In the next section we will create a macro, which performs a basic SDSF task in TSO, export it and then import it into the USR\sphinxhyphen{}DIR directory through the centralized DDI import facility. We will use the non\sphinxhyphen{}DDI macro feature on port 41001 to create a test macro using the macro record button. The macro will be called SDSF. The following key sequences where entered from the ISPF primary option menu after logging on to TSO on line 41001:

\begin{sphinxVerbatim}[commandchars=\\\{\}]
\PYG{n}{Logon} \PYG{n}{to} \PYG{n}{TSO} \PYG{o+ow}{and} \PYG{n}{bring} \PYG{n}{up} \PYG{n}{the} \PYG{n}{ISPF} \PYG{n}{primary} \PYG{n}{menu} \PYG{n}{panel}\PYG{o}{.}
\PYG{n}{Press} \PYG{n}{Virtel} \PYG{n}{macro} \PYG{n}{record} \PYG{n}{button}    \PYG{p}{[}\PYG{n}{To} \PYG{n}{the} \PYG{n}{left} \PYG{n}{of} \PYG{n}{the} \PYG{n}{green} \PYG{n}{ICON}\PYG{p}{]}
\PYG{o}{=}\PYG{n}{S}                                                  \PYG{n}{Short} \PYG{n}{cut} \PYG{n}{to} \PYG{n}{SDSF} \PYG{p}{(}\PYG{n}{This} \PYG{n}{will} \PYG{n}{be} \PYG{n}{site} \PYG{n}{dependant}\PYG{p}{]}
\PYG{n}{LOG}                                                 \PYG{n}{SDSF} \PYG{n}{Log} \PYG{n}{command}
\PYG{n}{Press} \PYG{n}{Virtel} \PYG{n}{macro} \PYG{n}{record} \PYG{n}{button} \PYG{n}{again} \PYG{n}{to} \PYG{n}{cease} \PYG{n}{recording}\PYG{o}{.}
\end{sphinxVerbatim}

\sphinxAtStartPar
The following panel should now pop up:

\sphinxAtStartPar
\sphinxincludegraphics{{image42}.png}

\sphinxAtStartPar
\sphinxstyleemphasis{Creating a MACROS.JSON file}

\sphinxAtStartPar
Saved the macro with the name SDSF. Now, press the Green ICON. The macro should appear in the local list and would have been saved to local browser storage. To run the macro press the green arrow to the left of the macro name SDSF.

\sphinxAtStartPar
\sphinxincludegraphics{{image43}.png}

\sphinxAtStartPar
Next, we need to create a “MACROS.JSON” file by exporting the SDSF macro we have just created.

\index{Virtel Macros@\spxentry{Virtel Macros}!centralized Macro mode@\spxentry{centralized Macro mode}}\index{centralized Macro mode@\spxentry{centralized Macro mode}!Uploading macros through the DDI interface@\spxentry{Uploading macros through the DDI interface}}\ignorespaces 

\subsubsection{Uploading macros to the centralized repository}
\label{\detokenize{Customization:uploading-macros-to-the-centralized-repository}}\label{\detokenize{Customization:index-87}}
\sphinxAtStartPar
To upload any MACROS.JSON file to the centralized repository we must first create a “local” export version. In the “Local Macros” popup export the SDSF macro using the export option. This will create a local “macros.json” file. The “MACROS.JSON” file will be exported to your default “download” directory on your PC.

\sphinxAtStartPar
If you view the file the following will be displayed:

\begin{sphinxVerbatim}[commandchars=\\\{\}]
\PYG{p}{\PYGZob{}}\PYG{l+s+s2}{\PYGZdq{}}\PYG{l+s+s2}{macros}\PYG{l+s+s2}{\PYGZdq{}}\PYG{p}{:}\PYG{p}{[}
    \PYG{p}{\PYGZob{}}\PYG{l+s+s2}{\PYGZdq{}}\PYG{l+s+s2}{name}\PYG{l+s+s2}{\PYGZdq{}}\PYG{p}{:}\PYG{l+s+s2}{\PYGZdq{}}\PYG{l+s+s2}{SDSF}\PYG{l+s+s2}{\PYGZdq{}}\PYG{p}{,}\PYG{l+s+s2}{\PYGZdq{}}\PYG{l+s+s2}{rev}\PYG{l+s+s2}{\PYGZdq{}}\PYG{p}{:}\PYG{l+m+mi}{1}\PYG{p}{,}\PYG{l+s+s2}{\PYGZdq{}}\PYG{l+s+s2}{def}\PYG{l+s+s2}{\PYGZdq{}}\PYG{p}{:}\PYG{p}{[}\PYG{l+m+mi}{61}\PYG{p}{,}\PYG{l+m+mi}{115}\PYG{p}{,}\PYG{l+s+s2}{\PYGZdq{}}\PYG{l+s+s2}{ENTER}\PYG{l+s+s2}{\PYGZdq{}}\PYG{p}{,}\PYG{l+m+mi}{108}\PYG{p}{,}\PYG{l+m+mi}{111}\PYG{p}{,}\PYG{l+m+mi}{103}\PYG{p}{,}\PYG{l+s+s2}{\PYGZdq{}}\PYG{l+s+s2}{ENTER}\PYG{l+s+s2}{\PYGZdq{}}\PYG{p}{]}\PYG{p}{\PYGZcb{}}
    \PYG{p}{]}\PYG{p}{,}
\PYG{l+s+s2}{\PYGZdq{}}\PYG{l+s+s2}{fmt}\PYG{l+s+s2}{\PYGZdq{}}\PYG{p}{:}\PYG{l+m+mi}{2}\PYG{p}{\PYGZcb{}}
\end{sphinxVerbatim}

\sphinxAtStartPar
We now have a macro which we can import to the centralized repository. From the Administration portal, select the Dynamic Directory Interface option:

\sphinxAtStartPar
\sphinxincludegraphics{{image44}.png}

\sphinxAtStartPar
This will open the Dynamic directories main panel. Select the Global tab. The Global Drag and Drop area will be displayed. Drag the MACROS.JSON file from the “downloads” folder on to the Global D\&D upload zone delineated by a dotted line. The MACROS.JSON file will be uploaded to the GLOBAL\sphinxhyphen{}DIR. The upload results will be shown on the D\&D zone area.

\sphinxAtStartPar
\sphinxincludegraphics{{image45}.png}

\sphinxAtStartPar
\sphinxstyleemphasis{Global DDI Upload}

\sphinxAtStartPar
Refresh the GLOBAL directory by clicking the broken circled arrow in the top left of the screen. The MACROS.JSON file will be displayed. We can validate the Global macro upload by opening a TSO or CICS session with the DDI configured line (41002). Press the “Blue” ICON to display all available macros. It should now show that the SDSF macro has indeed been imported to the GLOBAL area.

\sphinxAtStartPar
\sphinxincludegraphics{{image46}.png}

\sphinxAtStartPar
\sphinxstyleemphasis{Uploaded Global Macro}

\sphinxAtStartPar
To run the macro, enter the ISPF primary menu and then press the Green “ARROW” next to the SDSF name. The SDSF log display should appear.

\index{Virtel Macros@\spxentry{Virtel Macros}!centralized Macro mode@\spxentry{centralized Macro mode}}\index{centralized Macro mode@\spxentry{centralized Macro mode}!Creating user macros@\spxentry{Creating user macros}}\ignorespaces 

\subsection{Creating a user macro}
\label{\detokenize{Customization:creating-a-user-macro}}\label{\detokenize{Customization:index-88}}
\sphinxAtStartPar
With centralized macros, users can create “local” macros which will by synchronized with the centralized repository. They will not have the ability to import/export any macro.
Creating a “user” macro follows the same process as before. The user clicks the “record” button, records their macro, stops recording by pressing the record button again. As before, a popup window will open allowing the user to name the macro and associate with a “keyboard mapping” if that option has been set. The macro will then appear in the users list of macros.

\sphinxAtStartPar
Here is an example of a user macro called “CUSTINQ” which drives a sequence of CICS keyboard sequences to display a customer Inquiry panel within a CICS transaction. The macro has been associated with key mapping “ALT+6”.

\sphinxAtStartPar
\sphinxincludegraphics{{image47}.png}

\index{Virtel Macros@\spxentry{Virtel Macros}!centralized Macro mode@\spxentry{centralized Macro mode}}\index{centralized Macro mode@\spxentry{centralized Macro mode}!Macro panel options@\spxentry{Macro panel options}}\ignorespaces 

\subsection{Macro Options}
\label{\detokenize{Customization:macro-options}}\label{\detokenize{Customization:index-89}}
\sphinxAtStartPar
Pressing the “right click” button against the macro name will reveal a small popup which provides the user with some options.

\sphinxAtStartPar
\sphinxincludegraphics{{image48}.png}

\sphinxAtStartPar
The user has the ability to “DELETE”, RENAME (“Save As”), EDIT or RUN the macro. Whatever changes are made will be synchronized with the central repository.

\begin{sphinxadmonition}{note}{Note:}
\sphinxAtStartPar
The “Edit” feature is for viewing only.
\end{sphinxadmonition}

\index{Virtel Macros@\spxentry{Virtel Macros}!centralized Macro mode@\spxentry{centralized Macro mode}}\index{centralized Macro mode@\spxentry{centralized Macro mode}!Directory and Macro Administration@\spxentry{Directory and Macro Administration}}\ignorespaces 

\subsection{Directory and Macro Administration}
\label{\detokenize{Customization:directory-and-macro-administration}}\label{\detokenize{Customization:index-90}}
\sphinxAtStartPar
Users do not have the ability to create macros. This function is performed by the Virtel Administrator. The Administrator would develop macros using a “local” facility and export these up into the central repository to make them available to users at a Global, Group and User level.

\index{Virtel Macros@\spxentry{Virtel Macros}!centralized Macro mode@\spxentry{centralized Macro mode}}\index{centralized Macro mode@\spxentry{centralized Macro mode}!Making macros available@\spxentry{Making macros available}}\ignorespaces 

\subsubsection{Making a user’s DDI macro available to a DDI Group}
\label{\detokenize{Customization:making-a-user-s-ddi-macro-available-to-a-ddi-group}}\label{\detokenize{Customization:index-91}}
\sphinxAtStartPar
The following procedure describes how to move a user’s DDI macro to a DDI GROUP level. Start by opening the DDI interface from the main Administration Portal and select the USER tab;specify the userid in the user name field. Press the Green tick. The macros associated with the userid will be listed.

\sphinxAtStartPar
\sphinxincludegraphics{{image49}.png}

\sphinxAtStartPar
\sphinxstyleemphasis{Moving macros to a Group}

\sphinxAtStartPar
Select the MACROS.JSON file. This file contains all the macros associated with this userid. Right click to reveal the options available \sphinxhyphen{} Valid options are EDIT, DOWNLOAD or DELETE.
Note, the “COPY” option cannot be used to move MACRO.JSON files to another directory, such as GRP\sphinxhyphen{}DIR. To move the macro to GRP\sphinxhyphen{}DIR the MACRO.JSON must be downloaded and then uploaded, through the DDI upload interface, to the target group DDI zone.

\sphinxAtStartPar
Select the “DOWNLOAD” tab. The MACROS.JSON will be downloaded to the default download directory.

\sphinxAtStartPar
Now, select the GROUP TAB and specify the target RACF GROUP name. Press the GREEN tick. The GROUP directory for the group name will be displayed in a delineated DDI zone area. Select the MACROS.JSON file from the default download directory, or another directory where you might have a customised MACROS.JSON file, and drag it into the zone area, delineated by a row of dots. The macro will be uploaded to the selected GROUP.

\sphinxAtStartPar
If you now go to the TSO session under port 41002, the centralized DDI line, and list the macros by pressing the Blue ICON, you should see that the Group Macros now contains a list of macros. In our example it should be the macro CUSTINQ.

\sphinxAtStartPar
\sphinxincludegraphics{{image50}.png}

\sphinxAtStartPar
\sphinxstyleemphasis{Populated Group macro}

\sphinxAtStartPar
From this display you can see that the User Macro CUSTINQ has been moved to the Group area. The CUSTINQ macro can now be delete the user’s MACROS.JSON list using the DDI interface. To do this, enter the Administration Portal DDI interface, select the USERTAB, enter the userid into the username, press the Green tick to list the user’s MACROS.JSON file. Select the MACROS.JSON file, right click and press the DELETE tab. Confirm the deletion. The user’s MACRO.JSON fill will be deleted from the DDI USR\sphinxhyphen{}DIR directory and will no longer appear in the User macros list when listing the macros in the TSO session.

\begin{sphinxadmonition}{note}{Note:}
\sphinxAtStartPar
Press the Refresh button after “deleting” the user macro.
\end{sphinxadmonition}

\sphinxAtStartPar
\sphinxincludegraphics{{image51}.png}
\sphinxstyleemphasis{Deleting macros from user’s list}

\sphinxAtStartPar
The process of moving macros between the various DDI levels, USER, GROUP and GLOBAL, can be applied using the procedure described above. You must “download” and then “upload” using the DDI drag and drop interface to move macros between levels. Also, remember that all macros are associated with a JSON file \textendash{} MACROS.JSON. One such file will exist for each USER,  GROUP and there will be only one global MARCOS.JSON file available to all users.

\sphinxAtStartPar
It is not recommended that you EDIT a downloaded MACROS.JSON file unless you have an understanding of JavaScript and JSON objects. It is better the use the online “EDIT” functions associated with the DDI interface or that of the Macro display list. Right click either against the MACROS.JSON file or the individual Macros listed in the TSO session will bring up the Editor function.


\subsubsection{Dummy Macro File}
\label{\detokenize{Customization:dummy-macro-file}}
\sphinxAtStartPar
It is not necessary to export a “dummy” macro from a “local mode” setup to get started. You can start with a dummy MACROS.JSON file and upload it to User, Group and Global levels. From there you can build up your MACROS.JSON files and upload them to the relevant User, Group and Global zones. A dummy macro file looks like:

\begin{sphinxVerbatim}[commandchars=\\\{\}]
\PYG{p}{\PYGZob{}}\PYG{l+s+s2}{\PYGZdq{}}\PYG{l+s+s2}{macros}\PYG{l+s+s2}{\PYGZdq{}}\PYG{p}{:}\PYG{p}{[}\PYG{p}{]}\PYG{p}{,}\PYG{l+s+s2}{\PYGZdq{}}\PYG{l+s+s2}{fmt}\PYG{l+s+s2}{\PYGZdq{}}\PYG{p}{:}\PYG{l+m+mi}{2}\PYG{p}{\PYGZcb{}}
\end{sphinxVerbatim}

\sphinxAtStartPar
\sphinxstyleemphasis{Dummy MACROS.JSON file}

\index{Virtel Macros@\spxentry{Virtel Macros}!centralized Macro mode@\spxentry{centralized Macro mode}}\index{centralized Macro mode@\spxentry{centralized Macro mode}!Troubleshooting@\spxentry{Troubleshooting}}\ignorespaces 

\subsection{Centralized DDI Troubleshooting}
\label{\detokenize{Customization:centralized-ddi-troubleshooting}}\label{\detokenize{Customization:index-92}}

\subsubsection{Out of Sync condition or no transaction security}
\label{\detokenize{Customization:out-of-sync-condition-or-no-transaction-security}}
\sphinxAtStartPar
If the Virtel cache is out of sync with the DDI/macro settings or you are trying to access DDI macros using a non\sphinxhyphen{}secure transaction i.e. TRANSACTION SECURITY=0  then following page can appear when pressing the macro ICON:\sphinxhyphen{}

\sphinxAtStartPar
\sphinxincludegraphics{{image52}.png}

\sphinxAtStartPar
\sphinxstyleemphasis{Out of Sync. Problem}

\sphinxAtStartPar
\sphinxstylestrong{Solution}
\begin{itemize}
\item {} 
\sphinxAtStartPar
Make sure you are using a transaction with security set.

\item {} 
\sphinxAtStartPar
Refresh the cache with the following commands.

\sphinxAtStartPar
CTRL\sphinxhyphen{}SHIFT\sphinxhyphen{}DEL
CTRL R

\end{itemize}

\index{Virtel Macros@\spxentry{Virtel Macros}!centralized Macro mode@\spxentry{centralized Macro mode}}\index{centralized Macro mode@\spxentry{centralized Macro mode}!Macro Formats@\spxentry{Macro Formats}}\ignorespaces 

\subsection{Macro formats and Commands}
\label{\detokenize{Customization:macro-formats-and-commands}}\label{\detokenize{Customization:index-93}}
\sphinxAtStartPar
The format of the MACROS.JSON file is an embedded JSON structure. Each name structure represents a keystroke macro identified by the “name” keyword.
\begin{itemize}
\item {} 
\sphinxAtStartPar
Name : The name of the macro entry.

\item {} 
\sphinxAtStartPar
Rev : The “rev” is a user revision keyword.

\item {} 
\sphinxAtStartPar
Def : The “def” keyword identifies the commands and entry values.

\end{itemize}

\sphinxAtStartPar
The macro editor supports the following commands:\sphinxhyphen{}
*   “any string of characters to input into 3270 screen”
*   move(pos)
*   copy(startRow,startCol,endRow,endCol)
*   paste(pos)
*   paste(pos,nbRows,nbCols)
*   key(keyIdentifier)

\sphinxAtStartPar
\sphinxstylestrong{Macro example}

\begin{sphinxVerbatim}[commandchars=\\\{\}]
\PYG{p}{\PYGZob{}}\PYG{l+s+s2}{\PYGZdq{}}\PYG{l+s+s2}{macros}\PYG{l+s+s2}{\PYGZdq{}}\PYG{p}{:}\PYG{p}{[}
    \PYG{p}{\PYGZob{}}
        \PYG{l+s+s2}{\PYGZdq{}}\PYG{l+s+s2}{name}\PYG{l+s+s2}{\PYGZdq{}}\PYG{p}{:}\PYG{l+s+s2}{\PYGZdq{}}\PYG{l+s+s2}{SDSFLOG}\PYG{l+s+s2}{\PYGZdq{}}\PYG{p}{,}
        \PYG{l+s+s2}{\PYGZdq{}}\PYG{l+s+s2}{rev}\PYG{l+s+s2}{\PYGZdq{}}\PYG{p}{:}\PYG{l+m+mi}{2}\PYG{p}{,}
        \PYG{l+s+s2}{\PYGZdq{}}\PYG{l+s+s2}{def}\PYG{l+s+s2}{\PYGZdq{}}\PYG{p}{:}\PYG{p}{[}\PYG{l+s+s2}{\PYGZdq{}}\PYG{l+s+s2}{move(435)}\PYG{l+s+s2}{\PYGZdq{}}\PYG{p}{,}\PYG{l+s+s2}{\PYGZdq{}}\PYG{l+s+s2}{ENTER}\PYG{l+s+s2}{\PYGZdq{}}\PYG{p}{,}\PYG{p}{\PYGZob{}}\PYG{l+s+s2}{\PYGZdq{}}\PYG{l+s+s2}{txt}\PYG{l+s+s2}{\PYGZdq{}}\PYG{p}{:}\PYG{l+s+s2}{\PYGZdq{}}\PYG{l+s+s2}{=M}\PYG{l+s+s2}{\PYGZdq{}}\PYG{p}{\PYGZcb{}}\PYG{p}{,}\PYG{l+s+s2}{\PYGZdq{}}\PYG{l+s+s2}{ErEof}\PYG{l+s+s2}{\PYGZdq{}}\PYG{p}{,}\PYG{l+s+s2}{\PYGZdq{}}\PYG{l+s+s2}{ENTER}\PYG{l+s+s2}{\PYGZdq{}}\PYG{p}{,}\PYG{p}{\PYGZob{}}\PYG{l+s+s2}{\PYGZdq{}}\PYG{l+s+s2}{txt}\PYG{l+s+s2}{\PYGZdq{}}\PYG{p}{:}\PYG{l+s+s2}{\PYGZdq{}}\PYG{l+s+s2}{6}\PYG{l+s+s2}{\PYGZdq{}}\PYG{p}{\PYGZcb{}}\PYG{p}{,}\PYG{l+s+s2}{\PYGZdq{}}\PYG{l+s+s2}{ENTER}\PYG{l+s+s2}{\PYGZdq{}}\PYG{p}{]}
    \PYG{p}{\PYGZcb{}}\PYG{p}{,}
    \PYG{p}{\PYGZob{}}
        \PYG{l+s+s2}{\PYGZdq{}}\PYG{l+s+s2}{name}\PYG{l+s+s2}{\PYGZdq{}}\PYG{p}{:}\PYG{l+s+s2}{\PYGZdq{}}\PYG{l+s+s2}{SDSFDA}\PYG{l+s+s2}{\PYGZdq{}}\PYG{p}{,}
        \PYG{l+s+s2}{\PYGZdq{}}\PYG{l+s+s2}{rev}\PYG{l+s+s2}{\PYGZdq{}}\PYG{p}{:}\PYG{l+m+mi}{1}\PYG{p}{,}
        \PYG{l+s+s2}{\PYGZdq{}}\PYG{l+s+s2}{def}\PYG{l+s+s2}{\PYGZdq{}}\PYG{p}{:}\PYG{p}{[}\PYG{l+m+mi}{77}\PYG{p}{,}\PYG{l+s+s2}{\PYGZdq{}}\PYG{l+s+s2}{ENTER}\PYG{l+s+s2}{\PYGZdq{}}\PYG{p}{,}\PYG{l+m+mi}{53}\PYG{p}{,}\PYG{l+s+s2}{\PYGZdq{}}\PYG{l+s+s2}{ENTER}\PYG{l+s+s2}{\PYGZdq{}}\PYG{p}{,}\PYG{l+m+mi}{100}\PYG{p}{,}\PYG{l+m+mi}{97}\PYG{p}{,}\PYG{l+s+s2}{\PYGZdq{}}\PYG{l+s+s2}{ENTER}\PYG{l+s+s2}{\PYGZdq{}}\PYG{p}{]}
    \PYG{p}{\PYGZcb{}}
\PYG{p}{]}\PYG{p}{,}\PYG{l+s+s2}{\PYGZdq{}}\PYG{l+s+s2}{fmt}\PYG{l+s+s2}{\PYGZdq{}}\PYG{p}{:}\PYG{l+m+mi}{1}\PYG{p}{\PYGZcb{}}
\end{sphinxVerbatim}

\index{Macros@\spxentry{Macros}!Quick Reference Guide@\spxentry{Quick Reference Guide}}\ignorespaces 
\sphinxAtStartPar
See Appendix A \sphinxhyphen{} Virtel MAcro Quick Reference Sheet for further details on macro formats, commands, identifies and built in functions.

\index{Userparms@\spxentry{Userparms}!Centralized USERPARM mode@\spxentry{Centralized USERPARM mode}}\index{Centralized USERPARM mode@\spxentry{Centralized USERPARM mode}!Setup@\spxentry{Setup}}\ignorespaces 

\chapter{centralized User Parameters}
\label{\detokenize{Customization:centralized-user-parameters}}\label{\detokenize{Customization:index-95}}
\sphinxAtStartPar
Centralized USERPARM provides the ability to save user settings in a centralized VSAM file on the mainframe. The user settings with be synchronized with the browser whenever a user logs on, regardless of the device they are using. Users must have a RACF or equivalent userid.


\section{USERPARM Setup}
\label{\detokenize{Customization:userparm-setup}}
\sphinxAtStartPar
1   Stop Virtel. Update the TCT with the following definitions and re\sphinxhyphen{}assemble using the ASMTCT JCL found in the CNTL library.
\begin{quote}

\begin{sphinxVerbatim}[commandchars=\\\{\}]
\PYG{n}{UPARMS}\PYG{o}{=}\PYG{p}{(}\PYG{n}{USERPARM}\PYG{p}{)}\PYG{p}{,}                                              \PYG{o}{*}
\PYG{n}{UFILE3}\PYG{o}{=}\PYG{p}{(}\PYG{n}{USERTRSF}\PYG{p}{,}\PYG{n}{ACBH3}\PYG{p}{,}\PYG{l+m+mi}{0}\PYG{p}{,}\PYG{l+m+mi}{10}\PYG{p}{,}\PYG{l+m+mi}{01}\PYG{p}{)}\PYG{p}{,}                                \PYG{o}{*}
\PYG{n}{ACBH3}    \PYG{n}{ACB}   \PYG{n}{AM}\PYG{o}{=}\PYG{n}{VSAM}\PYG{p}{,}\PYG{n}{DDNAME}\PYG{o}{=}\PYG{n}{USERTRSF}\PYG{p}{,}\PYG{n}{MACRF}\PYG{o}{=}\PYG{p}{(}\PYG{n}{SEQ}\PYG{p}{,}\PYG{n}{DIR}\PYG{p}{,}\PYG{n}{OUT}\PYG{p}{)}\PYG{p}{,}     \PYG{o}{*}
       \PYG{n}{STRNO}\PYG{o}{=}\PYG{l+m+mi}{3}
\end{sphinxVerbatim}
\end{quote}

\sphinxAtStartPar
2   Run HLQ.VIRTxxx.SAMPLIB(USERPARM)
\begin{quote}

\sphinxAtStartPar
You will need to modify the JCL before running the Job. This job will allocate the USERTRSF VSAM file which will be the repository for user parameters.
\end{quote}

\sphinxAtStartPar
3   Update RACF to support the USERPARM transactions
\begin{quote}

\sphinxAtStartPar
The following security definitions will be needed to support the USERPARM feature. For example, for RACF the following FACILITY profile definitions need to be added and users given READ access.

\begin{sphinxVerbatim}[commandchars=\\\{\}]
\PYG{o}{/}\PYG{o}{/}\PYG{n}{STEP1}   \PYG{n}{EXEC} \PYG{n}{PGM}\PYG{o}{=}\PYG{n}{IKJEFT01}\PYG{p}{,}\PYG{n}{DYNAMNBR}\PYG{o}{=}\PYG{l+m+mi}{20}
\PYG{o}{/}\PYG{o}{/}\PYG{n}{SYSTSPRT} \PYG{n}{DD}  \PYG{n}{SYSOUT}\PYG{o}{=}\PYG{o}{*}
\PYG{o}{/}\PYG{o}{/}\PYG{n}{SYSTSIN}  \PYG{n}{DD}  \PYG{o}{*}
\PYG{n}{RDEF} \PYG{n}{FACILITY} \PYG{n}{VIRTEL}\PYG{o}{.}\PYG{n}{USERPARM} \PYG{n}{UACC}\PYG{p}{(}\PYG{n}{NONE}\PYG{p}{)}     \PYG{o}{/}\PYG{o}{*} \PYG{n}{DIRECTORY} \PYG{o}{*}\PYG{o}{/}
\PYG{n}{PE} \PYG{n}{VIRTEL}\PYG{o}{.}\PYG{n}{USERPARM} \PYG{n}{CL}\PYG{p}{(}\PYG{n}{FACILITY}\PYG{p}{)} \PYG{n}{RESET}
\PYG{n}{PE} \PYG{n}{VIRTEL}\PYG{o}{.}\PYG{n}{USERPARM} \PYG{n}{CL}\PYG{p}{(}\PYG{n}{FACILITY}\PYG{p}{)} \PYG{n}{ACC}\PYG{p}{(}\PYG{n}{READ}\PYG{p}{)} \PYG{n}{ID}\PYG{p}{(}\PYG{n}{SPGPTECH}\PYG{p}{)}
\PYG{n}{RDEF} \PYG{n}{FACILITY} \PYG{n}{VIRTEL}\PYG{o}{.}\PYG{n}{W2H}\PYG{o}{\PYGZhy{}}\PYG{l+m+mi}{74} \PYG{n}{UACC}\PYG{p}{(}\PYG{n}{NONE}\PYG{p}{)}       \PYG{o}{/}\PYG{o}{*} \PYG{n}{UPLOAD} \PYG{o}{*}\PYG{o}{/}
\PYG{n}{PE} \PYG{n}{VIRTEL}\PYG{o}{.}\PYG{n}{W2H}\PYG{o}{\PYGZhy{}}\PYG{l+m+mi}{74} \PYG{n}{CL}\PYG{p}{(}\PYG{n}{FACILITY}\PYG{p}{)} \PYG{n}{RESET}
\PYG{n}{PE} \PYG{n}{VIRTEL}\PYG{o}{.}\PYG{n}{W2H}\PYG{o}{\PYGZhy{}}\PYG{l+m+mi}{74} \PYG{n}{CL}\PYG{p}{(}\PYG{n}{FACILITY}\PYG{p}{)} \PYG{n}{ACC}\PYG{p}{(}\PYG{n}{READ}\PYG{p}{)} \PYG{n}{ID}\PYG{p}{(}\PYG{n}{SPGPTECH}\PYG{p}{)}
\PYG{n}{SETR} \PYG{n}{REFRESH} \PYG{n}{RACLIST}\PYG{p}{(}\PYG{n}{FACILITY}\PYG{p}{)}
\end{sphinxVerbatim}
\end{quote}

\sphinxAtStartPar
4   Update the VIRTEL procedure
\begin{quote}

\sphinxAtStartPar
Add the USERTRSF JCL statement to the Virtel procedure. For example :

\begin{sphinxVerbatim}[commandchars=\\\{\}]
\PYG{o}{/}\PYG{o}{/}\PYG{n}{USERTRSF} \PYG{n}{DD}  \PYG{n}{DSN}\PYG{o}{=}\PYG{o}{\PYGZam{}}\PYG{n}{QUAL}\PYG{o}{.}\PYG{o}{.}\PYG{n}{USER}\PYG{o}{.}\PYG{n}{TRSF}\PYG{p}{,}\PYG{n}{DISP}\PYG{o}{=}\PYG{n}{SHR}
\end{sphinxVerbatim}

\sphinxAtStartPar
Restart Virtel
\end{quote}

\sphinxAtStartPar
5   Perform a USERPARM sanity check
\begin{quote}

\sphinxAtStartPar
A ‘sanity check’ template is provided to validate the USERPARM environment at the transaction level. For example:\sphinxhyphen{}

\begin{sphinxVerbatim}[commandchars=\\\{\}]
\PYG{n}{http}\PYG{p}{:}\PYG{o}{/}\PYG{o}{/}\PYG{l+m+mf}{192.168}\PYG{l+m+mf}{.170}\PYG{l+m+mf}{.48}\PYG{p}{:}\PYG{l+m+mi}{41002}\PYG{o}{/}\PYG{n}{w2h}\PYG{o}{/}\PYG{n}{checkCentralizedSettings}\PYG{o}{.}\PYG{n}{htm}\PYG{o}{+}\PYG{n}{TSO}
\end{sphinxVerbatim}

\sphinxAtStartPar
If every has been configured correctly you should get a “GREEN” light.
\end{quote}

\sphinxAtStartPar
\sphinxincludegraphics{{image53}.png}

\sphinxAtStartPar
\sphinxstyleemphasis{Example of a Sanity Check after setting up USERPARM}

\index{Userparms@\spxentry{Userparms}!Centralized USERPARM mode@\spxentry{Centralized USERPARM mode}}\index{Centralized USERPARM mode@\spxentry{Centralized USERPARM mode}!Validation@\spxentry{Validation}}\ignorespaces 

\subsection{Testing the USERPARM feature}
\label{\detokenize{Customization:testing-the-userparm-feature}}\label{\detokenize{Customization:index-96}}
\sphinxAtStartPar
To test the USERPARM feature follow the procedure below :\sphinxhyphen{}

\sphinxAtStartPar
1   Logon to a 3270 application, say TSO
2   Open the settings dialog
3   Change the “Display Style” to Grey and save. The background colour should change to grey.
4   Disconnect from Virtel using the Red button in the Tool Bar.
5   Reconnect to Virtel, using a different browser and open a 3270 session.
6   The background colour should still be set to grey.
7   Open the settings dialogue and confirm that Grey button is still highlighted.

\sphinxAtStartPar
In the Virtel log you should see the following “upload” message:\sphinxhyphen{}

\begin{sphinxVerbatim}[commandchars=\\\{\}]
\PYG{n}{VIRU122I} \PYG{n}{CLVTA079} \PYG{n}{FILE} \PYG{n}{UPLOAD} \PYG{p}{:} \PYG{n}{ENTRY} \PYG{n}{POINT} \PYG{l+s+s1}{\PYGZsq{}}\PYG{l+s+s1}{CLIWHOST}\PYG{l+s+s1}{\PYGZsq{}} \PYG{n}{DIRECTORY} \PYG{l+s+s1}{\PYGZsq{}}\PYG{l+s+s1}{USERP  882}
\PYG{n}{ARM}\PYG{l+s+s1}{\PYGZsq{}}\PYG{l+s+s1}{ USER }\PYG{l+s+s1}{\PYGZsq{}}\PYG{n}{HLQ}             \PYG{l+s+s1}{\PYGZsq{}}
        \PYG{n}{FILE} \PYG{p}{:} \PYG{l+s+s1}{\PYGZsq{}}\PYG{l+s+s1}{8393DD4A16F0E1C9723F9D9CFA2C39F1}\PYG{l+s+s1}{\PYGZsq{}}
\end{sphinxVerbatim}

\index{Userparms@\spxentry{Userparms}!Centralized USERPARM mode@\spxentry{Centralized USERPARM mode}}\index{Centralized USERPARM mode@\spxentry{Centralized USERPARM mode}!Removing a users settings@\spxentry{Removing a users settings}}\ignorespaces 

\subsection{Removing a user’s settings}
\label{\detokenize{Customization:removing-a-user-s-settings}}\label{\detokenize{Customization:index-97}}
\sphinxAtStartPar
This can be done by deleting the key associated with the user from the USERPARM directory. For example, the key associated with HLQ is C6D24022819C599803A86BB3C42298B6. If we look at USERPARM directory in the Administration portal on line 41001 we can see it listed:

\sphinxAtStartPar
\sphinxincludegraphics{{image55}.png}

\sphinxAtStartPar
\sphinxstyleemphasis{Listing the USERPARM directory}

\sphinxAtStartPar
We can remove the entry by using the delete option. This will remove any user settings. The user will then have to “restore defaults” to correctly pickup up the default user parameters as defined by the Administrator.

\index{Userparms@\spxentry{Userparms}!Centralized USERPARM mode@\spxentry{Centralized USERPARM mode}}\index{Centralized USERPARM mode@\spxentry{Centralized USERPARM mode}!Troubleshooting@\spxentry{Troubleshooting}}\ignorespaces 

\subsection{USERPARM Problems}
\label{\detokenize{Customization:userparm-problems}}\label{\detokenize{Customization:index-98}}
\sphinxAtStartPar
When I try to save my settings, I get the following dialog box and the following messages in the Virtel log.

\begin{sphinxVerbatim}[commandchars=\\\{\}]
\PYG{n}{VIRU122I} \PYG{n}{CLVTA079} \PYG{n}{FILE} \PYG{n}{UPLOAD} \PYG{p}{:} \PYG{n}{ENTRY} \PYG{n}{POINT} \PYG{l+s+s1}{\PYGZsq{}}\PYG{l+s+s1}{CLIWHOST}\PYG{l+s+s1}{\PYGZsq{}} \PYG{n}{DIRECTORY} \PYG{l+s+s1}{\PYGZsq{}}\PYG{l+s+s1}{USERP}
\PYG{l+m+mi}{330}
\PYG{n}{ARM}\PYG{l+s+s1}{\PYGZsq{}}\PYG{l+s+s1}{ USER }\PYG{l+s+s1}{\PYGZsq{}}\PYG{n}{HLQ}             \PYG{l+s+s1}{\PYGZsq{}}
    \PYG{n}{FILE} \PYG{p}{:} \PYG{l+s+s1}{\PYGZsq{}}\PYG{l+s+s1}{9346CCEF695D8FF2D7C1B2DA6C81CFC8}\PYG{l+s+s1}{\PYGZsq{}}
\PYG{n}{VIRC121E} \PYG{n}{PAGE} \PYG{n}{NOT} \PYG{n}{FOUND} \PYG{n}{FOR} \PYG{n}{CLVTA079} \PYG{n}{ENTRY} \PYG{n}{POINT} \PYG{l+s+s1}{\PYGZsq{}}\PYG{l+s+s1}{CLIWHOST}\PYG{l+s+s1}{\PYGZsq{}} \PYG{n}{DIRECTORY} \PYG{l+s+s1}{\PYGZsq{}}
\PYG{l+m+mi}{331}
\PYG{n}{W2H}     \PYG{l+s+s1}{\PYGZsq{}}\PYG{l+s+s1}{(W2H\PYGZhy{}DIR  W2H\PYGZhy{}KEY )}
    \PYG{n}{PAGE} \PYG{p}{:} \PYG{l+s+s1}{\PYGZsq{}}\PYG{l+s+s1}{VPLEX=UNDEFINEDPARAMETERS\PYGZhy{}CODE.TXT}\PYG{l+s+s1}{\PYGZsq{}} \PYG{n}{URL} \PYG{p}{:} \PYG{l+s+s1}{\PYGZsq{}}\PYG{l+s+s1}{/w2h/VPLEX=un}
\PYG{n}{definedPARAMETERS}\PYG{o}{\PYGZhy{}}\PYG{n}{CODE}\PYG{o}{.}\PYG{n}{txt}\PYG{o}{+}\PYG{o}{+}\PYG{n}{AjaxSession}\PYG{o}{=}\PYG{n}{ADvzWAAAAAAeu}
\end{sphinxVerbatim}

\sphinxAtStartPar
\sphinxincludegraphics{{image54}.png}

\sphinxAtStartPar
\sphinxstyleemphasis{Userparm Error}

\sphinxAtStartPar
\sphinxstylestrong{Solution}

\sphinxAtStartPar
Apply the latest maintenance. You must be at UPDT5708 or higher.

\sphinxAtStartPar
\sphinxstylestrong{Problem}

\sphinxAtStartPar
I save my user parameter changes but when I use a different PC or browser I am reverted back to my old or default settings.

\sphinxAtStartPar
\sphinxstylestrong{Solution}

\sphinxAtStartPar
Check the Virtel log and look for any RACF violations that may be preventing a user from uploading their user parameters definitions to the mainframe repository. In the console log you may well see RACF error messages:

\begin{sphinxVerbatim}[commandchars=\\\{\}]
\PYG{n}{ICH408I} \PYG{n}{USER}\PYG{p}{(}\PYG{n}{CTKHOLT} \PYG{p}{)} \PYG{n}{GROUP}\PYG{p}{(}\PYG{n}{CTKGROUP}\PYG{p}{)} \PYG{n}{NAME}\PYG{p}{(}\PYG{n}{ED} \PYG{n}{HOLT}             \PYG{p}{)} \PYG{l+m+mi}{451}
\PYG{n}{SPVIREH}\PYG{o}{.}\PYG{n}{USERPARM} \PYG{n}{CL}\PYG{p}{(}\PYG{n}{FACILITY}\PYG{p}{)}
\PYG{n}{INSUFFICIENT} \PYG{n}{ACCESS} \PYG{n}{AUTHORITY}
\PYG{n}{ACCESS} \PYG{n}{INTENT}\PYG{p}{(}\PYG{n}{READ}   \PYG{p}{)}  \PYG{n}{ACCESS} \PYG{n}{ALLOWED}\PYG{p}{(}\PYG{n}{NONE}   \PYG{p}{)}
\PYG{n}{VIRU121E} \PYG{n}{CLVTA079} \PYG{n}{FILE} \PYG{n}{UPLOAD} \PYG{n}{FAILED} \PYG{p}{:} \PYG{n}{ENTRY} \PYG{n}{POINT} \PYG{l+s+s1}{\PYGZsq{}}\PYG{l+s+s1}{CLIWHOST}\PYG{l+s+s1}{\PYGZsq{}} \PYG{n}{DIRECTORY}
\PYG{l+m+mi}{452}
\PYG{l+s+s1}{\PYGZsq{}}\PYG{l+s+s1}{USERPARM}\PYG{l+s+s1}{\PYGZsq{}} \PYG{n}{USER} \PYG{l+s+s1}{\PYGZsq{}}\PYG{l+s+s1}{HLQ             }\PYG{l+s+s1}{\PYGZsq{}}
    \PYG{n}{FILE} \PYG{p}{:} \PYG{l+s+s1}{\PYGZsq{}}\PYG{l+s+s1}{C6D24022819C599803A86BB3C42298B6}\PYG{l+s+s1}{\PYGZsq{}}
\end{sphinxVerbatim}

\sphinxAtStartPar
\sphinxstylestrong{Solution}

\sphinxAtStartPar
Check the correct RACF security is in place.

\sphinxAtStartPar
\sphinxstylestrong{Problem}

\sphinxAtStartPar
Changed defaults not updated in another browser session.

\sphinxAtStartPar
\sphinxstylestrong{Solution}

\sphinxAtStartPar
This is normally related to configuration. Try the “sanity” URL test to make sure that the setup is correct for the line in use.

\index{Security@\spxentry{Security}!AT\sphinxhyphen{}TLS@\spxentry{AT\sphinxhyphen{}TLS}}\index{AT\sphinxhyphen{}TLS@\spxentry{AT\sphinxhyphen{}TLS}!How to activate@\spxentry{How to activate}}\ignorespaces 

\chapter{Security}
\label{\detokenize{Customization:security}}\label{\detokenize{Customization:index-99}}

\section{How To Activate SSL AT\sphinxhyphen{}TLS}
\label{\detokenize{Customization:how-to-activate-ssl-at-tls}}
\sphinxAtStartPar
To provide secure HTTP (https) sessions to client browsers, VIRTEL uses the Application Transparent Transport Layer Security (AT\sphinxhyphen{}TLS) feature of z/OS Communication Server. AT\sphinxhyphen{}TLS is included with z/OS V1R7 and later releases. AT\sphinxhyphen{}TLS allows socket applications to access encrypted sessions by invoking system SSL within the transport layer of the TCP/IP stack. The Policy Agent decides which connections are to use AT\sphinxhyphen{}TLS, and provides system SSL configuration for those connections. The application continues to send and receive clear text over the socket, but data sent over the network is protected by system SSL. The supported protocols are TLS, SSLv3, and SSLv2.

\index{Security@\spxentry{Security}!AT\sphinxhyphen{}TLS@\spxentry{AT\sphinxhyphen{}TLS}}\index{AT\sphinxhyphen{}TLS@\spxentry{AT\sphinxhyphen{}TLS}!Installation@\spxentry{Installation}}\ignorespaces 

\subsection{Installation steps}
\label{\detokenize{Customization:installation-steps}}\label{\detokenize{Customization:index-100}}

\subsubsection{Install Policy Agent procedure}
\label{\detokenize{Customization:install-policy-agent-procedure}}
\sphinxAtStartPar
If you do not already have the Communications Server Policy Agent (PAGENT) active in your z/OS system, copy the cataloged procedure EZAPAGSP from TCPIP.SEZAINST into your proclib, renaming it as PAGENT.


\subsubsection{Create the Policy Agent configuration file}
\label{\detokenize{Customization:create-the-policy-agent-configuration-file}}
\sphinxAtStartPar
If you do not already run the Policy Agent, you will need to create a configuration file /etc/pagent.conf using z/OS Unix System Services. If you already run Policy Agent, you will need to find the existing configuration file and add the TTLS definitions to it. Step PCONFIG in the SSLSETUP sample job contains a starter configuration. The following changes should be made:
\begin{itemize}
\item {} 
\sphinxAtStartPar
Replace \%virtjob\% by the name of your VIRTEL started task (SSLSETUP line 70)

\item {} 
\sphinxAtStartPar
Replace 41000\sphinxhyphen{}41002 by 41002 in the LocalPortRange parameter (SSLSETUP line 71) to activate AT\sphinxhyphen{}TLS for VIRTEL line C\sphinxhyphen{}HTTP

\item {} 
\sphinxAtStartPar
Replace ServerWithClientAuth by Server in the HandshakeRole parameter (SSLSETUP line 82) as we will not be using Client Certificates.

\end{itemize}


\subsubsection{Allow the Policy Agent to run during TCP/IP initialization}
\label{\detokenize{Customization:allow-the-policy-agent-to-run-during-tcp-ip-initialization}}
\sphinxAtStartPar
The Policy Agent must be given READ access to the resource EZB.INITSTACK.* in RACF class SERVAUTH. See step EZBAUTH in the SSLSETUP sample job (delivered in VIRTEL SAMPLIB).

\index{Security@\spxentry{Security}!AT\sphinxhyphen{}TLS@\spxentry{AT\sphinxhyphen{}TLS}}\index{AT\sphinxhyphen{}TLS@\spxentry{AT\sphinxhyphen{}TLS}!Certificates@\spxentry{Certificates}}\ignorespaces 

\subsubsection{Create the server certificate}
\label{\detokenize{Customization:create-the-server-certificate}}\label{\detokenize{Customization:index-101}}
\sphinxAtStartPar
A server certificate for VIRTEL must be created, signed by a certificate authority, and stored in the RACF database. In the SSLSETUP sample job we create a signing certificate and use RACF itself as the certificate authority. Alternatively, you may use an external certificate authority such as Verisign to create and sign the certificate, then import it into
RACF. At SSLSETUP line 228, replace \%virtssl\% by the DNS name assigned to the VIRTEL host (for example, virtssl.syspertec.com)


\subsubsection{Add the certificate to the keyring}
\label{\detokenize{Customization:add-the-certificate-to-the-keyring}}
\sphinxAtStartPar
The server certificate must be added to the VIRTRING keyring. See step CCERTIF in the SSLSETUP sample job.


\subsubsection{Allow VIRTEL to access its own certificate}
\label{\detokenize{Customization:allow-virtel-to-access-its-own-certificate}}
\sphinxAtStartPar
To allow VIRTEL to access its own keyring and server certificate, the VIRTEL started task must have READ access to the resource IRR.DIGTCERT.LISTRING in the RACF class FACILITY. See step IRRAUTH in the SSLSETUP sample job.

\index{Security@\spxentry{Security}!AT\sphinxhyphen{}TLS@\spxentry{AT\sphinxhyphen{}TLS}}\index{AT\sphinxhyphen{}TLS@\spxentry{AT\sphinxhyphen{}TLS}!TCPIP definitions@\spxentry{TCPIP definitions}}\ignorespaces 

\subsubsection{TCPIP Definitions for AT\sphinxhyphen{}TLS}
\label{\detokenize{Customization:tcpip-definitions-for-at-tls}}\label{\detokenize{Customization:index-102}}
\sphinxAtStartPar
To activate AT\sphinxhyphen{}TLS, add the following statements to TCPIP PROFILE:

\begin{sphinxVerbatim}[commandchars=\\\{\}]
\PYG{n}{TCPCONFIG} \PYG{n}{TTLS}
\PYG{n}{AUTOLOG} \PYG{l+m+mi}{5} \PYG{n}{PAGENT} \PYG{n}{ENDAUTOLOG}
\end{sphinxVerbatim}

\sphinxAtStartPar
Stop and restart TCP/IP to activate the TCPCONFIG TTLS profile statement. The AUTOLOG statement will cause the PAGENT procedure to be started automatically during TCP/IP initialization.

\index{Security@\spxentry{Security}!AT\sphinxhyphen{}TLS@\spxentry{AT\sphinxhyphen{}TLS}}\index{AT\sphinxhyphen{}TLS@\spxentry{AT\sphinxhyphen{}TLS}!Operations@\spxentry{Operations}}\ignorespaces 

\subsection{Operations}
\label{\detokenize{Customization:operations}}\label{\detokenize{Customization:index-103}}

\subsubsection{Starting the Policy Agent}
\label{\detokenize{Customization:starting-the-policy-agent}}
\sphinxAtStartPar
The AUTOLOG statement in the TCP/IP profile will start the PAGENT procedure automatically at TCP/IP initialization. Alternatively you can issue the MVS command S PAGENT.
Note: if this is the first time you have activated the SERVAUTH class, you are likely to see RACF failure messages during TCP/IP initialization indicating that other applications are unable to access the resource EZB.INITSTACK. This is normal, because Communications Server uses this mechanism to prevent applications from accessing TCP/IP before the Policy
Agent is started. Do not be tempted to authorize applications to use this RACF resource. Either ignore the messages (they will go away once PAGENT has started), or ensure that PAGENT starts before all other applications.


\subsubsection{Altering the Policy Agent configuration}
\label{\detokenize{Customization:altering-the-policy-agent-configuration}}
\sphinxAtStartPar
To make changes to the Policy Agent configuration file, either edit and resubmit the PCONFIG step of the SSLSETUP sample job, or use the TSO ISHELL command to edit the file /etc/pagent.conf directly from ISPF. After you make changes to the Policy Agent configuration, use the MVS command F PAGENT,REFRESH to force PAGENT to reread the file.


\subsubsection{Logon to VIRTEL using secure session}
\label{\detokenize{Customization:logon-to-virtel-using-secure-session}}
\sphinxAtStartPar
To access VIRTEL line C\sphinxhyphen{}HTTP you must now use URL \sphinxurl{https://n.n.n.n:41002} instead of \sphinxurl{http://n.n.n.n:41002} (where n.n.n.n is the IP address of the z/OS host running VIRTEL).


\subsection{Problem determination}
\label{\detokenize{Customization:problem-determination}}

\subsubsection{Policy Agent log file}
\label{\detokenize{Customization:policy-agent-log-file}}
\sphinxAtStartPar
Policy Agent startup messages are written to the /tmp/pagent.log file of z/OS Unix System Services. You can use the TSO ISHELL command to browse this file from ISPF.

\index{Security@\spxentry{Security}!AT\sphinxhyphen{}TLS@\spxentry{AT\sphinxhyphen{}TLS}}\index{AT\sphinxhyphen{}TLS@\spxentry{AT\sphinxhyphen{}TLS}!Troubleshootings@\spxentry{Troubleshootings}}\ignorespaces 

\subsubsection{Common error messages}
\label{\detokenize{Customization:common-error-messages}}\label{\detokenize{Customization:index-104}}
\sphinxAtStartPar
Error messages relating to session setup are written to the MVS SYSLOG. The most common error message is:

\begin{sphinxVerbatim}[commandchars=\\\{\}]
\PYG{n}{EZD1287I} \PYG{n}{TTLS} \PYG{n}{Error} \PYG{n}{RC}\PYG{p}{:} \PYG{n}{nnn} \PYG{n}{event}
\end{sphinxVerbatim}

\sphinxAtStartPar
where nnn represents a return code. Return codes under 5000 are generated by System SSL and are defined in the System SSL Programming manual. Return codes over 5000 are generated by AT\sphinxhyphen{}TLS and are defined in the IP Diagnosis Guide. Some commonly encountered return codes are:
\begin{itemize}
\item {} 
\sphinxAtStartPar
7 No certificate

\item {} 
\sphinxAtStartPar
8 Certificate not trusted

\item {} 
\sphinxAtStartPar
109 No certification authority certificates

\item {} 
\sphinxAtStartPar
202 Keyring does not exist

\item {} 
\sphinxAtStartPar
401 Certificate expired or not yet valid

\item {} 
\sphinxAtStartPar
402 or 412 Client and server cannot agree on cipher suite

\item {} 
\sphinxAtStartPar
416 VIRTEL does not have permission to list the keyring

\item {} 
\sphinxAtStartPar
431 Certificate is revoked

\item {} 
\sphinxAtStartPar
434 Certificate key not compatible with cipher suite

\item {} 
\sphinxAtStartPar
435 Certificate authority unknown

\item {} 
\sphinxAtStartPar
5003 Browser sent clear text (http instead of https)

\end{itemize}


\subsubsection{Cipher suite}
\label{\detokenize{Customization:cipher-suite}}
\sphinxAtStartPar
The client and server cipher specifications must contain at least one value in common. The TTLSEnvironmentAdvancedParms parameter of the Policy Agent configuration file allows you to turn on or off the SSLv2, SSLv3, and TLSv1 protocols at the server end. The list of supported cipher suites for each protocol is in the TTLSCipherParms parameter. Check the /tmp/pagent.log file to determine whether any cipher suites were discarded at startup time. In Microsoft Internet Explorer, follow the menu Tools \textendash{} Internet Options \textendash{} Advanced. Under the security heading there are three options which allow you to enable or disable the SSL 2.0, SSL 3.0, and TLS 1.0 protocols. You cannot enable or disable individual cipher suites.
In Firefox the cipher specifications are accessed by typing \sphinxurl{about:config} in the address bar and typing security in the filter box. By default, ssl2 is disabled, and ssl3 and tls are enabled. By default, all weak encryption cipher suites are disabled, and 128\sphinxhyphen{}bit or higher cipher suites are enabled.


\subsubsection{Bibliography}
\label{\detokenize{Customization:bibliography}}\begin{itemize}
\item {} 
\sphinxAtStartPar
SA22\sphinxhyphen{}7683\sphinxhyphen{}07 z/OS V1R7 Security Server: RACF Security Administrator’s Guide Chapter 21. RACF and Digital Certificates

\item {} 
\sphinxAtStartPar
SC24\sphinxhyphen{}5901\sphinxhyphen{}04 z/OS V1R6 Cryptographic Services: System SSL Programming Chapter 12. Messages and Codes

\item {} 
\sphinxAtStartPar
SC31\sphinxhyphen{}8775\sphinxhyphen{}07 z/OS V1R7 Communications Server: IP Configuration Guide Chapter 14. Policy\sphinxhyphen{}based networking Chapter 18. Application Transparent Transport Layer Security (AT\sphinxhyphen{}TLS) data protection

\item {} 
\sphinxAtStartPar
SC31\sphinxhyphen{}8776\sphinxhyphen{}08 z/OS V1R7 Communications Server: IP Configuration Reference Chapter 21. Policy Agent and policy applications

\item {} 
\sphinxAtStartPar
GC31\sphinxhyphen{}8782\sphinxhyphen{}06 z/OS V1R7 Communications Server: IP Diagnosis Guide Chapter 28. Diagnosing Application Transparent Transport Layer Security (AT\sphinxhyphen{}TLS)

\item {} 
\sphinxAtStartPar
SC31\sphinxhyphen{}8784\sphinxhyphen{}05 z/OS V1R7 Communications Server: IP Messages: Volume 2 (EZB, EZD) Chapter 10. EZD1xxxx messages

\end{itemize}

\index{Security@\spxentry{Security}!AT\sphinxhyphen{}TLS@\spxentry{AT\sphinxhyphen{}TLS}}\index{AT\sphinxhyphen{}TLS@\spxentry{AT\sphinxhyphen{}TLS}!Using Server and Client certificates@\spxentry{Using Server and Client certificates}}\ignorespaces 

\subsection{SSL \sphinxhyphen{} Signing On Using Server And Client Certificates}
\label{\detokenize{Customization:ssl-signing-on-using-server-and-client-certificates}}\label{\detokenize{Customization:index-105}}
\sphinxAtStartPar
In this section we look at setting up Virtel to work with client and user certificates and to effectively remove the need for a user to provide a user id and password. This is equivalent to the Express Logon Feature (ELF) provided by Host on Demand and other Telnet clients. First, let’s review what is going on behind the scenes with certificate authentication and X.509 certificate validation within TLS/SSL. The guiding principle here is that Public Key Infrastructure (PKI) requires that data encrypted with a
public key can only be decrypted with a private key and data encrypted with a private key can only be decrypted with a public key. The secure session (https) that runs between the browser and Virtel uses the Application Transparent Transport Layer Security feature of z/OS Communications Server, also known as AT\sphinxhyphen{}TLS. AT\sphinxhyphen{}TLS allows socket applications to access encrypted sessions by invoking Secure Socket Layer (SSL) within the transport layer of the TCP/IP stack. A policy agent (PAGENT) is used to configure AT\sphinxhyphen{}TLS using parameter statements which will determine which sessions are to use AT\sphinxhyphen{}TLS. AT\_TLS inserts itself in the connection between the application and browser. This means that the application will send and receive clear text over the socket interface, but data over the network is encrypted by system SSL. System SSL has three supported protocol levels:
\begin{itemize}
\item {} 
\sphinxAtStartPar
TLSv1

\item {} 
\sphinxAtStartPar
SSLv2

\item {} 
\sphinxAtStartPar
SSLv3

\end{itemize}

\sphinxAtStartPar
In this configuration we will be using SSLv3. The server / client process, which in Virtel’s case is the Virtel started task (server) and the browser (client), implements
the following SSL protocol or handshake during the “hello” phase of establishing a secure session:
\begin{enumerate}
\sphinxsetlistlabels{\arabic}{enumi}{enumii}{}{.}%
\item {} 
\sphinxAtStartPar
The Client contacts the Server ;

\item {} 
\sphinxAtStartPar
The Server sends a certificate;

\item {} 
\sphinxAtStartPar
Server authentication is performed by the Client ;

\item {} 
\sphinxAtStartPar
Client sends the certificate;

\item {} 
\sphinxAtStartPar
Client authentication is performed by the Server ;

\item {} 
\sphinxAtStartPar
An encryption algorithm and single key is chosen to encrypt / decrypt data The purpose of the authentication is to ensure that the server/client are in fact who they say they are. This is to ensure that they server/client private and public keys haven’t been stolen and are purporting to be an entity that they aren’t and thereby compromising security. Authentication uses X.509 digital certificates. Further details of this handshake and the certificate exchange can be found in Appendix B. TLS/SSL Security z/OS Communications Server: IP Configuration Guide.

\end{enumerate}

\index{Security@\spxentry{Security}!AT\sphinxhyphen{}TLS@\spxentry{AT\sphinxhyphen{}TLS}}\index{AT\sphinxhyphen{}TLS@\spxentry{AT\sphinxhyphen{}TLS}!What is a X.509 certificate@\spxentry{What is a X.509 certificate}}\ignorespaces 

\subsection{What is a Certificates}
\label{\detokenize{Customization:what-is-a-certificates}}\label{\detokenize{Customization:index-106}}

\subsubsection{X.509 certificate?}
\label{\detokenize{Customization:x-509-certificate}}
\sphinxAtStartPar
Amongst other things it includes the Distinguished Name of the Server (DNS), the public key of the Server, Distinguished Name of the Server organization issuing the certificate and the issuer’s signature. If we look at a certificate held with RACF we can see this information. Certificates are identified by a combination of LABEL, USERID or Certification Authority (CA).

\begin{sphinxVerbatim}[commandchars=\\\{\}]
\PYG{n}{READY}
\PYG{n}{RACDCERT} \PYG{n}{ID}\PYG{p}{(}\PYG{n}{SPVIRSTC}\PYG{p}{)} \PYG{n}{LIST}\PYG{p}{(}\PYG{n}{LABEL}\PYG{p}{(}\PYG{l+s+s1}{\PYGZsq{}}\PYG{l+s+s1}{VIRTEL SSL DEMO}\PYG{l+s+s1}{\PYGZsq{}}\PYG{p}{)}\PYG{p}{)}
\PYG{n}{Digital} \PYG{n}{certificate} \PYG{n}{information} \PYG{k}{for} \PYG{n}{user} \PYG{n}{SPVIRSTC}\PYG{p}{:}
\PYG{n}{Label}\PYG{p}{:} \PYG{n}{VIRTEL} \PYG{n}{SSL} \PYG{n}{DEMO}
\PYG{n}{Certificate} \PYG{n}{ID}\PYG{p}{:} \PYG{l+m+mi}{2}\PYG{n}{Qji1}\PYG{o}{+}\PYG{n}{XJ2eLjw}\PYG{o}{+}\PYG{n}{XJ2ePF00Di4tNAxMXU1kBA}
\PYG{n}{Status}\PYG{p}{:} \PYG{n}{TRUST}
\PYG{n}{Start} \PYG{n}{Date}\PYG{p}{:} \PYG{l+m+mi}{2014}\PYG{o}{/}\PYG{l+m+mi}{07}\PYG{o}{/}\PYG{l+m+mi}{08} \PYG{l+m+mi}{00}\PYG{p}{:}\PYG{l+m+mi}{00}\PYG{p}{:}\PYG{l+m+mi}{00}
\PYG{n}{End} \PYG{n}{Date}\PYG{p}{:} \PYG{l+m+mi}{2015}\PYG{o}{/}\PYG{l+m+mi}{07}\PYG{o}{/}\PYG{l+m+mi}{08} \PYG{l+m+mi}{23}\PYG{p}{:}\PYG{l+m+mi}{59}\PYG{p}{:}\PYG{l+m+mi}{59}
\PYG{n}{Serial} \PYG{n}{Number}\PYG{p}{:}
    \PYG{o}{\PYGZgt{}}\PYG{l+m+mi}{05}\PYG{o}{\PYGZlt{}}
\PYG{n}{Issuer}\PYG{l+s+s1}{\PYGZsq{}}\PYG{l+s+s1}{s Name:}
    \PYG{o}{\PYGZgt{}}\PYG{n}{CN}\PYG{o}{=}\PYG{n}{z}\PYG{o}{/}\PYG{n}{OS} \PYG{n}{Security} \PYG{n}{Server}\PYG{o}{.}\PYG{n}{O}\PYG{o}{=}\PYG{n}{SYSPERTEC}\PYG{o}{.}\PYG{n}{C}\PYG{o}{=}\PYG{n}{FR}\PYG{o}{\PYGZlt{}}
\PYG{n}{Subject}\PYG{l+s+s1}{\PYGZsq{}}\PYG{l+s+s1}{s Name:}
    \PYG{o}{\PYGZgt{}}\PYG{n}{CN}\PYG{o}{=}\PYG{n}{RECETTE} \PYG{n}{VIRTEL}\PYG{o}{.}\PYG{n}{T}\PYG{o}{=}\PYG{n}{VIRTEL} \PYG{n}{Web} \PYG{n}{Access}\PYG{o}{.}\PYG{n}{O}\PYG{o}{=}\PYG{n}{SYSPERTEC}\PYG{o}{.}\PYG{n}{C}\PYG{o}{=}\PYG{n}{FR}\PYG{o}{\PYGZlt{}}
\PYG{n}{Key} \PYG{n}{Usage}\PYG{p}{:} \PYG{n}{HANDSHAKE}\PYG{p}{,} \PYG{n}{DATAENCRYPT}
\PYG{n}{Key} \PYG{n}{Type}\PYG{p}{:} \PYG{n}{RSA}
\PYG{n}{Key} \PYG{n}{Size}\PYG{p}{:} \PYG{l+m+mi}{1024}
\PYG{n}{Private} \PYG{n}{Key}\PYG{p}{:} \PYG{n}{YES}
\PYG{n}{Ring} \PYG{n}{Associations}\PYG{p}{:}
\PYG{n}{Ring} \PYG{n}{Owner}\PYG{p}{:} \PYG{n}{SPVIRSTC}
\PYG{n}{Ring}\PYG{p}{:}
    \PYG{o}{\PYGZgt{}}\PYG{n}{VIRTRING}\PYG{o}{\PYGZlt{}}
\end{sphinxVerbatim}

\sphinxAtStartPar
Similar details can be found in the browser settings. For example here is what Chrome displays in the HTTPS/SSL certificate database.

\sphinxAtStartPar
\sphinxincludegraphics{{image18}.png}


\subsubsection{Types of certificates}
\label{\detokenize{Customization:types-of-certificates}}\begin{itemize}
\item {} 
\sphinxAtStartPar
Client certificate

\item {} 
\sphinxAtStartPar
Server certificate

\item {} 
\sphinxAtStartPar
Well\sphinxhyphen{}known Certificate Authority (CA) Signing certificate

\item {} 
\sphinxAtStartPar
RACF signing certificate

\end{itemize}

\sphinxAtStartPar
In this configuration we will be using self\sphinxhyphen{}signed server and client certificates. In most installation you would use server and client certificates signed by a well\sphinxhyphen{}known CA. These well\sphinxhyphen{}known CA certificates are normally available in the RACF and browser key data bases.

\index{Security@\spxentry{Security}!AT\sphinxhyphen{}TLS@\spxentry{AT\sphinxhyphen{}TLS}}\index{AT\sphinxhyphen{}TLS@\spxentry{AT\sphinxhyphen{}TLS}!Configuring certificates@\spxentry{Configuring certificates}}\index{RACF@\spxentry{RACF}!Configuring a X.509 certificate@\spxentry{Configuring a X.509 certificate}}\ignorespaces 

\subsubsection{Configuring the certificates}
\label{\detokenize{Customization:configuring-the-certificates}}\label{\detokenize{Customization:index-107}}
\sphinxAtStartPar
The first step is to create the necessary certificates. We require a server certificate, a RACF signing certificate and a user certificate. In the Virtel SAMPLIB there is a member called SSLSETUP. This will initialize the SSL environment and create the RACF signing certificate. Some of the steps may or may not be relevant so you will need to customize SSLSETUP accordingly. For example, you might already be running the PAGENT started task and have RACF definitions in place to support the required SSL access. The following is the certificate generation statement for the RACF signing certificate.

\begin{sphinxVerbatim}[commandchars=\\\{\}]
\PYG{o}{/}\PYG{o}{/}\PYG{n}{DCERTCA} \PYG{n}{EXEC} \PYG{n}{PGM}\PYG{o}{=}\PYG{n}{IKJEFT01}
\PYG{o}{/}\PYG{o}{/}\PYG{n}{SYSTSPRT} \PYG{n}{DD} \PYG{n}{SYSOUT}\PYG{o}{=}\PYG{o}{*}
\PYG{o}{/}\PYG{o}{/}\PYG{n}{SYSTSIN} \PYG{n}{DD} \PYG{o}{*}
\PYG{o}{/}\PYG{o}{*}\PYG{o}{\PYGZhy{}}\PYG{o}{\PYGZhy{}}\PYG{o}{\PYGZhy{}}\PYG{o}{\PYGZhy{}}\PYG{o}{\PYGZhy{}}\PYG{o}{\PYGZhy{}}\PYG{o}{\PYGZhy{}}\PYG{o}{\PYGZhy{}}\PYG{o}{\PYGZhy{}}\PYG{o}{\PYGZhy{}}\PYG{o}{\PYGZhy{}}\PYG{o}{\PYGZhy{}}\PYG{o}{\PYGZhy{}}\PYG{o}{\PYGZhy{}}\PYG{o}{\PYGZhy{}}\PYG{o}{\PYGZhy{}}\PYG{o}{\PYGZhy{}}\PYG{o}{\PYGZhy{}}\PYG{o}{\PYGZhy{}}\PYG{o}{\PYGZhy{}}\PYG{o}{\PYGZhy{}}\PYG{o}{\PYGZhy{}}\PYG{o}{\PYGZhy{}}\PYG{o}{\PYGZhy{}}\PYG{o}{\PYGZhy{}}\PYG{o}{\PYGZhy{}}\PYG{o}{\PYGZhy{}}\PYG{o}{\PYGZhy{}}\PYG{o}{\PYGZhy{}}\PYG{o}{\PYGZhy{}}\PYG{o}{\PYGZhy{}}\PYG{o}{\PYGZhy{}}\PYG{o}{\PYGZhy{}}\PYG{o}{\PYGZhy{}}\PYG{o}{\PYGZhy{}}\PYG{o}{\PYGZhy{}}\PYG{o}{\PYGZhy{}}\PYG{o}{\PYGZhy{}}\PYG{o}{\PYGZhy{}}\PYG{o}{\PYGZhy{}}\PYG{o}{\PYGZhy{}}\PYG{o}{\PYGZhy{}}\PYG{o}{\PYGZhy{}}\PYG{o}{\PYGZhy{}}\PYG{o}{\PYGZhy{}}\PYG{o}{\PYGZhy{}}\PYG{o}{\PYGZhy{}}\PYG{o}{\PYGZhy{}}\PYG{o}{\PYGZhy{}}\PYG{o}{\PYGZhy{}}\PYG{o}{\PYGZhy{}}\PYG{o}{\PYGZhy{}}\PYG{o}{\PYGZhy{}}\PYG{o}{\PYGZhy{}}\PYG{o}{\PYGZhy{}}\PYG{o}{\PYGZhy{}}\PYG{o}{\PYGZhy{}}\PYG{o}{\PYGZhy{}}\PYG{o}{\PYGZhy{}}\PYG{o}{\PYGZhy{}}\PYG{o}{\PYGZhy{}}\PYG{o}{\PYGZhy{}}\PYG{o}{\PYGZhy{}}\PYG{o}{\PYGZhy{}}\PYG{o}{\PYGZhy{}}\PYG{o}{*}\PYG{o}{/}
\PYG{o}{/}\PYG{o}{*} \PYG{n}{Delete} \PYG{n}{previous} \PYG{n}{signing} \PYG{n}{certificate} \PYG{o}{*}\PYG{o}{/}
\PYG{o}{/}\PYG{o}{*}\PYG{o}{\PYGZhy{}}\PYG{o}{\PYGZhy{}}\PYG{o}{\PYGZhy{}}\PYG{o}{\PYGZhy{}}\PYG{o}{\PYGZhy{}}\PYG{o}{\PYGZhy{}}\PYG{o}{\PYGZhy{}}\PYG{o}{\PYGZhy{}}\PYG{o}{\PYGZhy{}}\PYG{o}{\PYGZhy{}}\PYG{o}{\PYGZhy{}}\PYG{o}{\PYGZhy{}}\PYG{o}{\PYGZhy{}}\PYG{o}{\PYGZhy{}}\PYG{o}{\PYGZhy{}}\PYG{o}{\PYGZhy{}}\PYG{o}{\PYGZhy{}}\PYG{o}{\PYGZhy{}}\PYG{o}{\PYGZhy{}}\PYG{o}{\PYGZhy{}}\PYG{o}{\PYGZhy{}}\PYG{o}{\PYGZhy{}}\PYG{o}{\PYGZhy{}}\PYG{o}{\PYGZhy{}}\PYG{o}{\PYGZhy{}}\PYG{o}{\PYGZhy{}}\PYG{o}{\PYGZhy{}}\PYG{o}{\PYGZhy{}}\PYG{o}{\PYGZhy{}}\PYG{o}{\PYGZhy{}}\PYG{o}{\PYGZhy{}}\PYG{o}{\PYGZhy{}}\PYG{o}{\PYGZhy{}}\PYG{o}{\PYGZhy{}}\PYG{o}{\PYGZhy{}}\PYG{o}{\PYGZhy{}}\PYG{o}{\PYGZhy{}}\PYG{o}{\PYGZhy{}}\PYG{o}{\PYGZhy{}}\PYG{o}{\PYGZhy{}}\PYG{o}{\PYGZhy{}}\PYG{o}{\PYGZhy{}}\PYG{o}{\PYGZhy{}}\PYG{o}{\PYGZhy{}}\PYG{o}{\PYGZhy{}}\PYG{o}{\PYGZhy{}}\PYG{o}{\PYGZhy{}}\PYG{o}{\PYGZhy{}}\PYG{o}{\PYGZhy{}}\PYG{o}{\PYGZhy{}}\PYG{o}{\PYGZhy{}}\PYG{o}{\PYGZhy{}}\PYG{o}{\PYGZhy{}}\PYG{o}{\PYGZhy{}}\PYG{o}{\PYGZhy{}}\PYG{o}{\PYGZhy{}}\PYG{o}{\PYGZhy{}}\PYG{o}{\PYGZhy{}}\PYG{o}{\PYGZhy{}}\PYG{o}{\PYGZhy{}}\PYG{o}{\PYGZhy{}}\PYG{o}{\PYGZhy{}}\PYG{o}{\PYGZhy{}}\PYG{o}{\PYGZhy{}}\PYG{o}{\PYGZhy{}}\PYG{o}{*}\PYG{o}{/}
\PYG{n}{RACDCERT} \PYG{n}{CERTAUTH} \PYG{o}{+}
\PYG{n}{DELETE}\PYG{p}{(}\PYG{n}{LABEL}\PYG{p}{(}\PYG{l+s+s1}{\PYGZsq{}}\PYG{l+s+s1}{z/OS signing certificate}\PYG{l+s+s1}{\PYGZsq{}}\PYG{p}{)}\PYG{p}{)}
\PYG{o}{/}\PYG{o}{/}\PYG{o}{*}\PYG{o}{\PYGZhy{}}\PYG{o}{\PYGZhy{}}\PYG{o}{\PYGZhy{}}\PYG{o}{\PYGZhy{}}\PYG{o}{\PYGZhy{}}\PYG{o}{\PYGZhy{}}\PYG{o}{\PYGZhy{}}\PYG{o}{\PYGZhy{}}\PYG{o}{\PYGZhy{}}\PYG{o}{\PYGZhy{}}\PYG{o}{\PYGZhy{}}\PYG{o}{\PYGZhy{}}\PYG{o}{\PYGZhy{}}\PYG{o}{\PYGZhy{}}\PYG{o}{\PYGZhy{}}\PYG{o}{\PYGZhy{}}\PYG{o}{\PYGZhy{}}\PYG{o}{\PYGZhy{}}\PYG{o}{\PYGZhy{}}\PYG{o}{\PYGZhy{}}\PYG{o}{\PYGZhy{}}\PYG{o}{\PYGZhy{}}\PYG{o}{\PYGZhy{}}\PYG{o}{\PYGZhy{}}\PYG{o}{\PYGZhy{}}\PYG{o}{\PYGZhy{}}\PYG{o}{\PYGZhy{}}\PYG{o}{\PYGZhy{}}\PYG{o}{\PYGZhy{}}\PYG{o}{\PYGZhy{}}\PYG{o}{\PYGZhy{}}\PYG{o}{\PYGZhy{}}\PYG{o}{\PYGZhy{}}\PYG{o}{\PYGZhy{}}\PYG{o}{\PYGZhy{}}\PYG{o}{\PYGZhy{}}\PYG{o}{\PYGZhy{}}\PYG{o}{\PYGZhy{}}\PYG{o}{\PYGZhy{}}\PYG{o}{\PYGZhy{}}\PYG{o}{\PYGZhy{}}\PYG{o}{\PYGZhy{}}\PYG{o}{\PYGZhy{}}\PYG{o}{\PYGZhy{}}\PYG{o}{\PYGZhy{}}\PYG{o}{\PYGZhy{}}\PYG{o}{\PYGZhy{}}\PYG{o}{\PYGZhy{}}\PYG{o}{\PYGZhy{}}\PYG{o}{\PYGZhy{}}\PYG{o}{\PYGZhy{}}\PYG{o}{\PYGZhy{}}\PYG{o}{\PYGZhy{}}\PYG{o}{\PYGZhy{}}\PYG{o}{\PYGZhy{}}\PYG{o}{\PYGZhy{}}\PYG{o}{\PYGZhy{}}\PYG{o}{\PYGZhy{}}\PYG{o}{\PYGZhy{}}\PYG{o}{\PYGZhy{}}\PYG{o}{\PYGZhy{}}\PYG{o}{\PYGZhy{}}\PYG{o}{\PYGZhy{}}\PYG{o}{\PYGZhy{}}\PYG{o}{*}
\PYG{o}{/}\PYG{o}{/}\PYG{o}{*} \PYG{n}{CCERTCA} \PYG{p}{:} \PYG{n}{CREATE} \PYG{n}{SIGNING} \PYG{n}{CERTIFICATE} \PYG{o}{*}
\PYG{o}{/}\PYG{o}{/}\PYG{o}{*}\PYG{o}{\PYGZhy{}}\PYG{o}{\PYGZhy{}}\PYG{o}{\PYGZhy{}}\PYG{o}{\PYGZhy{}}\PYG{o}{\PYGZhy{}}\PYG{o}{\PYGZhy{}}\PYG{o}{\PYGZhy{}}\PYG{o}{\PYGZhy{}}\PYG{o}{\PYGZhy{}}\PYG{o}{\PYGZhy{}}\PYG{o}{\PYGZhy{}}\PYG{o}{\PYGZhy{}}\PYG{o}{\PYGZhy{}}\PYG{o}{\PYGZhy{}}\PYG{o}{\PYGZhy{}}\PYG{o}{\PYGZhy{}}\PYG{o}{\PYGZhy{}}\PYG{o}{\PYGZhy{}}\PYG{o}{\PYGZhy{}}\PYG{o}{\PYGZhy{}}\PYG{o}{\PYGZhy{}}\PYG{o}{\PYGZhy{}}\PYG{o}{\PYGZhy{}}\PYG{o}{\PYGZhy{}}\PYG{o}{\PYGZhy{}}\PYG{o}{\PYGZhy{}}\PYG{o}{\PYGZhy{}}\PYG{o}{\PYGZhy{}}\PYG{o}{\PYGZhy{}}\PYG{o}{\PYGZhy{}}\PYG{o}{\PYGZhy{}}\PYG{o}{\PYGZhy{}}\PYG{o}{\PYGZhy{}}\PYG{o}{\PYGZhy{}}\PYG{o}{\PYGZhy{}}\PYG{o}{\PYGZhy{}}\PYG{o}{\PYGZhy{}}\PYG{o}{\PYGZhy{}}\PYG{o}{\PYGZhy{}}\PYG{o}{\PYGZhy{}}\PYG{o}{\PYGZhy{}}\PYG{o}{\PYGZhy{}}\PYG{o}{\PYGZhy{}}\PYG{o}{\PYGZhy{}}\PYG{o}{\PYGZhy{}}\PYG{o}{\PYGZhy{}}\PYG{o}{\PYGZhy{}}\PYG{o}{\PYGZhy{}}\PYG{o}{\PYGZhy{}}\PYG{o}{\PYGZhy{}}\PYG{o}{\PYGZhy{}}\PYG{o}{\PYGZhy{}}\PYG{o}{\PYGZhy{}}\PYG{o}{\PYGZhy{}}\PYG{o}{\PYGZhy{}}\PYG{o}{\PYGZhy{}}\PYG{o}{\PYGZhy{}}\PYG{o}{\PYGZhy{}}\PYG{o}{\PYGZhy{}}\PYG{o}{\PYGZhy{}}\PYG{o}{\PYGZhy{}}\PYG{o}{\PYGZhy{}}\PYG{o}{\PYGZhy{}}\PYG{o}{\PYGZhy{}}\PYG{o}{*}
\PYG{o}{/}\PYG{o}{/}\PYG{n}{CCERTCA} \PYG{n}{EXEC} \PYG{n}{PGM}\PYG{o}{=}\PYG{n}{IKJEFT1A}
\PYG{o}{/}\PYG{o}{/}\PYG{n}{SYSTSPRT} \PYG{n}{DD} \PYG{n}{SYSOUT}\PYG{o}{=}\PYG{o}{*}
\PYG{o}{/}\PYG{o}{/}\PYG{n}{SYSTSIN} \PYG{n}{DD} \PYG{o}{*}
\PYG{o}{/}\PYG{o}{*}\PYG{o}{\PYGZhy{}}\PYG{o}{\PYGZhy{}}\PYG{o}{\PYGZhy{}}\PYG{o}{\PYGZhy{}}\PYG{o}{\PYGZhy{}}\PYG{o}{\PYGZhy{}}\PYG{o}{\PYGZhy{}}\PYG{o}{\PYGZhy{}}\PYG{o}{\PYGZhy{}}\PYG{o}{\PYGZhy{}}\PYG{o}{\PYGZhy{}}\PYG{o}{\PYGZhy{}}\PYG{o}{\PYGZhy{}}\PYG{o}{\PYGZhy{}}\PYG{o}{\PYGZhy{}}\PYG{o}{\PYGZhy{}}\PYG{o}{\PYGZhy{}}\PYG{o}{\PYGZhy{}}\PYG{o}{\PYGZhy{}}\PYG{o}{\PYGZhy{}}\PYG{o}{\PYGZhy{}}\PYG{o}{\PYGZhy{}}\PYG{o}{\PYGZhy{}}\PYG{o}{\PYGZhy{}}\PYG{o}{\PYGZhy{}}\PYG{o}{\PYGZhy{}}\PYG{o}{\PYGZhy{}}\PYG{o}{\PYGZhy{}}\PYG{o}{\PYGZhy{}}\PYG{o}{\PYGZhy{}}\PYG{o}{\PYGZhy{}}\PYG{o}{\PYGZhy{}}\PYG{o}{\PYGZhy{}}\PYG{o}{\PYGZhy{}}\PYG{o}{\PYGZhy{}}\PYG{o}{\PYGZhy{}}\PYG{o}{\PYGZhy{}}\PYG{o}{\PYGZhy{}}\PYG{o}{\PYGZhy{}}\PYG{o}{\PYGZhy{}}\PYG{o}{\PYGZhy{}}\PYG{o}{\PYGZhy{}}\PYG{o}{\PYGZhy{}}\PYG{o}{\PYGZhy{}}\PYG{o}{\PYGZhy{}}\PYG{o}{\PYGZhy{}}\PYG{o}{\PYGZhy{}}\PYG{o}{\PYGZhy{}}\PYG{o}{\PYGZhy{}}\PYG{o}{\PYGZhy{}}\PYG{o}{\PYGZhy{}}\PYG{o}{\PYGZhy{}}\PYG{o}{\PYGZhy{}}\PYG{o}{\PYGZhy{}}\PYG{o}{\PYGZhy{}}\PYG{o}{\PYGZhy{}}\PYG{o}{\PYGZhy{}}\PYG{o}{\PYGZhy{}}\PYG{o}{\PYGZhy{}}\PYG{o}{\PYGZhy{}}\PYG{o}{\PYGZhy{}}\PYG{o}{\PYGZhy{}}\PYG{o}{\PYGZhy{}}\PYG{o}{\PYGZhy{}}\PYG{o}{\PYGZhy{}}\PYG{o}{*}\PYG{o}{/}
\PYG{o}{/}\PYG{o}{*} \PYG{n}{Create} \PYG{n}{a} \PYG{n}{signing} \PYG{n}{certificate} \PYG{o}{*}\PYG{o}{/}
\PYG{o}{/}\PYG{o}{*}\PYG{o}{\PYGZhy{}}\PYG{o}{\PYGZhy{}}\PYG{o}{\PYGZhy{}}\PYG{o}{\PYGZhy{}}\PYG{o}{\PYGZhy{}}\PYG{o}{\PYGZhy{}}\PYG{o}{\PYGZhy{}}\PYG{o}{\PYGZhy{}}\PYG{o}{\PYGZhy{}}\PYG{o}{\PYGZhy{}}\PYG{o}{\PYGZhy{}}\PYG{o}{\PYGZhy{}}\PYG{o}{\PYGZhy{}}\PYG{o}{\PYGZhy{}}\PYG{o}{\PYGZhy{}}\PYG{o}{\PYGZhy{}}\PYG{o}{\PYGZhy{}}\PYG{o}{\PYGZhy{}}\PYG{o}{\PYGZhy{}}\PYG{o}{\PYGZhy{}}\PYG{o}{\PYGZhy{}}\PYG{o}{\PYGZhy{}}\PYG{o}{\PYGZhy{}}\PYG{o}{\PYGZhy{}}\PYG{o}{\PYGZhy{}}\PYG{o}{\PYGZhy{}}\PYG{o}{\PYGZhy{}}\PYG{o}{\PYGZhy{}}\PYG{o}{\PYGZhy{}}\PYG{o}{\PYGZhy{}}\PYG{o}{\PYGZhy{}}\PYG{o}{\PYGZhy{}}\PYG{o}{\PYGZhy{}}\PYG{o}{\PYGZhy{}}\PYG{o}{\PYGZhy{}}\PYG{o}{\PYGZhy{}}\PYG{o}{\PYGZhy{}}\PYG{o}{\PYGZhy{}}\PYG{o}{\PYGZhy{}}\PYG{o}{\PYGZhy{}}\PYG{o}{\PYGZhy{}}\PYG{o}{\PYGZhy{}}\PYG{o}{\PYGZhy{}}\PYG{o}{\PYGZhy{}}\PYG{o}{\PYGZhy{}}\PYG{o}{\PYGZhy{}}\PYG{o}{\PYGZhy{}}\PYG{o}{\PYGZhy{}}\PYG{o}{\PYGZhy{}}\PYG{o}{\PYGZhy{}}\PYG{o}{\PYGZhy{}}\PYG{o}{\PYGZhy{}}\PYG{o}{\PYGZhy{}}\PYG{o}{\PYGZhy{}}\PYG{o}{\PYGZhy{}}\PYG{o}{\PYGZhy{}}\PYG{o}{\PYGZhy{}}\PYG{o}{\PYGZhy{}}\PYG{o}{\PYGZhy{}}\PYG{o}{\PYGZhy{}}\PYG{o}{\PYGZhy{}}\PYG{o}{\PYGZhy{}}\PYG{o}{\PYGZhy{}}\PYG{o}{\PYGZhy{}}\PYG{o}{\PYGZhy{}}\PYG{o}{*}\PYG{o}{/}
\PYG{n}{RACDCERT} \PYG{n}{CERTAUTH} \PYG{o}{+}
\PYG{n}{GENCERT} \PYG{o}{+}
\PYG{n}{WITHLABEL}\PYG{p}{(}\PYG{l+s+s1}{\PYGZsq{}}\PYG{l+s+s1}{z/OS signing certificate}\PYG{l+s+s1}{\PYGZsq{}}\PYG{p}{)} \PYG{o}{+}
\PYG{n}{SUBJECTSDN}\PYG{p}{(} \PYG{o}{+}
\PYG{n}{CN}\PYG{p}{(}\PYG{l+s+s1}{\PYGZsq{}}\PYG{l+s+s1}{z/OS Security Server}\PYG{l+s+s1}{\PYGZsq{}}\PYG{p}{)} \PYG{o}{+}
\PYG{n}{O}\PYG{p}{(}\PYG{l+s+s1}{\PYGZsq{}}\PYG{l+s+s1}{SYSPERTEC}\PYG{l+s+s1}{\PYGZsq{}}\PYG{p}{)} \PYG{o}{+}
\PYG{n}{C}\PYG{p}{(}\PYG{l+s+s1}{\PYGZsq{}}\PYG{l+s+s1}{FR}\PYG{l+s+s1}{\PYGZsq{}}\PYG{p}{)}\PYG{p}{)} \PYG{o}{+}
\PYG{n}{KEYUSAGE}\PYG{p}{(}\PYG{n}{CERTSIGN}\PYG{p}{)} \PYG{n}{SIZE}\PYG{p}{(}\PYG{l+m+mi}{1024}\PYG{p}{)} \PYG{o}{+}
\PYG{n}{NOTAFTER}\PYG{p}{(}\PYG{n}{DATE}\PYG{p}{(}\PYG{l+m+mi}{2026}\PYG{o}{\PYGZhy{}}\PYG{l+m+mi}{06}\PYG{o}{\PYGZhy{}}\PYG{l+m+mi}{30}\PYG{p}{)}\PYG{p}{)}
\end{sphinxVerbatim}

\sphinxAtStartPar
If we list the certificate after we have created it will get the following:

\begin{sphinxVerbatim}[commandchars=\\\{\}]
\PYG{n}{READY}
\PYG{n}{RACDCERT} \PYG{n}{CERTAUTH} \PYG{n}{LIST}\PYG{p}{(}\PYG{n}{LABEL}\PYG{p}{(}\PYG{l+s+s1}{\PYGZsq{}}\PYG{l+s+s1}{z/OS signing certificate}\PYG{l+s+s1}{\PYGZsq{}}\PYG{p}{)}\PYG{p}{)}
\PYG{n}{Digital} \PYG{n}{certificate} \PYG{n}{information} \PYG{k}{for} \PYG{n}{CERTAUTH}\PYG{p}{:}
\PYG{n}{Label}\PYG{p}{:} \PYG{n}{z}\PYG{o}{/}\PYG{n}{OS} \PYG{n}{signing} \PYG{n}{certificate}
\PYG{n}{Certificate} \PYG{n}{ID}\PYG{p}{:} \PYG{l+m+mi}{2}\PYG{n}{QiJmZmDhZmjgalh1uJAoomHlYmVh0CDhZmjiYaJg4GjhUBA}
\PYG{n}{Status}\PYG{p}{:} \PYG{n}{TRUST}
\PYG{n}{Start} \PYG{n}{Date}\PYG{p}{:} \PYG{l+m+mi}{2013}\PYG{o}{/}\PYG{l+m+mi}{07}\PYG{o}{/}\PYG{l+m+mi}{03} \PYG{l+m+mi}{00}\PYG{p}{:}\PYG{l+m+mi}{00}\PYG{p}{:}\PYG{l+m+mi}{00}
\PYG{n}{End} \PYG{n}{Date}\PYG{p}{:} \PYG{l+m+mi}{2026}\PYG{o}{/}\PYG{l+m+mi}{06}\PYG{o}{/}\PYG{l+m+mi}{30} \PYG{l+m+mi}{23}\PYG{p}{:}\PYG{l+m+mi}{59}\PYG{p}{:}\PYG{l+m+mi}{59}
\PYG{n}{Serial} \PYG{n}{Number}\PYG{p}{:}
    \PYG{o}{\PYGZgt{}}\PYG{l+m+mi}{00}\PYG{o}{\PYGZlt{}}
\PYG{n}{Issuer}\PYG{l+s+s1}{\PYGZsq{}}\PYG{l+s+s1}{s Name:}
    \PYG{o}{\PYGZgt{}}\PYG{n}{CN}\PYG{o}{=}\PYG{n}{z}\PYG{o}{/}\PYG{n}{OS} \PYG{n}{Security} \PYG{n}{Server}\PYG{o}{.}\PYG{n}{O}\PYG{o}{=}\PYG{n}{SYSPERTEC}\PYG{o}{.}\PYG{n}{C}\PYG{o}{=}\PYG{n}{FR}\PYG{o}{\PYGZlt{}}
\PYG{n}{Subject}\PYG{l+s+s1}{\PYGZsq{}}\PYG{l+s+s1}{s Name:}
    \PYG{o}{\PYGZgt{}}\PYG{n}{CN}\PYG{o}{=}\PYG{n}{z}\PYG{o}{/}\PYG{n}{OS} \PYG{n}{Security} \PYG{n}{Server}\PYG{o}{.}\PYG{n}{O}\PYG{o}{=}\PYG{n}{SYSPERTEC}\PYG{o}{.}\PYG{n}{C}\PYG{o}{=}\PYG{n}{FR}\PYG{o}{\PYGZlt{}}
\PYG{n}{Key} \PYG{n}{Usage}\PYG{p}{:} \PYG{n}{CERTSIGN}    \PYG{o}{\PYGZlt{}\PYGZlt{}}\PYG{o}{\PYGZlt{}}
\PYG{n}{Key} \PYG{n}{Type}\PYG{p}{:} \PYG{n}{RSA}
\PYG{n}{Key} \PYG{n}{Size}\PYG{p}{:} \PYG{l+m+mi}{1024}
\PYG{n}{Private} \PYG{n}{Key}\PYG{p}{:} \PYG{n}{YES}
\PYG{n}{Ring} \PYG{n}{Associations}\PYG{p}{:}
\PYG{n}{Ring} \PYG{n}{Owner}\PYG{p}{:} \PYG{n}{SPVIRSTC}
\PYG{n}{Ring}\PYG{p}{:}
    \PYG{o}{\PYGZgt{}}\PYG{n}{VIRTRING}\PYG{o}{\PYGZlt{}}
\end{sphinxVerbatim}

\sphinxAtStartPar
The key usage identifies this certificate as a signing certificate. This certificate will be used to sign other certificates that we generate. Next is the server certificate. Again we use RACF to generate the certificate and use the RACF signing certificate to “sign” it. The following extract is from the Virtel SAMPLIB member SSLUCERT.

\begin{sphinxVerbatim}[commandchars=\\\{\}]
\PYG{o}{/}\PYG{o}{/}\PYG{n}{CCERTIF} \PYG{n}{EXEC} \PYG{n}{PGM}\PYG{o}{=}\PYG{n}{IKJEFT1A}
\PYG{o}{/}\PYG{o}{/}\PYG{n}{SYSTSPRT} \PYG{n}{DD} \PYG{n}{SYSOUT}\PYG{o}{=}\PYG{o}{*}
\PYG{o}{/}\PYG{o}{/}\PYG{n}{SYSTSIN} \PYG{n}{DD} \PYG{o}{*}
\PYG{o}{/}\PYG{o}{*}\PYG{o}{\PYGZhy{}}\PYG{o}{\PYGZhy{}}\PYG{o}{\PYGZhy{}}\PYG{o}{\PYGZhy{}}\PYG{o}{\PYGZhy{}}\PYG{o}{\PYGZhy{}}\PYG{o}{\PYGZhy{}}\PYG{o}{\PYGZhy{}}\PYG{o}{\PYGZhy{}}\PYG{o}{\PYGZhy{}}\PYG{o}{\PYGZhy{}}\PYG{o}{\PYGZhy{}}\PYG{o}{\PYGZhy{}}\PYG{o}{\PYGZhy{}}\PYG{o}{\PYGZhy{}}\PYG{o}{\PYGZhy{}}\PYG{o}{\PYGZhy{}}\PYG{o}{\PYGZhy{}}\PYG{o}{\PYGZhy{}}\PYG{o}{\PYGZhy{}}\PYG{o}{\PYGZhy{}}\PYG{o}{\PYGZhy{}}\PYG{o}{\PYGZhy{}}\PYG{o}{\PYGZhy{}}\PYG{o}{\PYGZhy{}}\PYG{o}{\PYGZhy{}}\PYG{o}{\PYGZhy{}}\PYG{o}{\PYGZhy{}}\PYG{o}{\PYGZhy{}}\PYG{o}{\PYGZhy{}}\PYG{o}{\PYGZhy{}}\PYG{o}{\PYGZhy{}}\PYG{o}{\PYGZhy{}}\PYG{o}{\PYGZhy{}}\PYG{o}{\PYGZhy{}}\PYG{o}{\PYGZhy{}}\PYG{o}{\PYGZhy{}}\PYG{o}{\PYGZhy{}}\PYG{o}{\PYGZhy{}}\PYG{o}{\PYGZhy{}}\PYG{o}{\PYGZhy{}}\PYG{o}{\PYGZhy{}}\PYG{o}{\PYGZhy{}}\PYG{o}{\PYGZhy{}}\PYG{o}{\PYGZhy{}}\PYG{o}{\PYGZhy{}}\PYG{o}{\PYGZhy{}}\PYG{o}{\PYGZhy{}}\PYG{o}{\PYGZhy{}}\PYG{o}{\PYGZhy{}}\PYG{o}{\PYGZhy{}}\PYG{o}{\PYGZhy{}}\PYG{o}{\PYGZhy{}}\PYG{o}{\PYGZhy{}}\PYG{o}{\PYGZhy{}}\PYG{o}{\PYGZhy{}}\PYG{o}{\PYGZhy{}}\PYG{o}{\PYGZhy{}}\PYG{o}{\PYGZhy{}}\PYG{o}{\PYGZhy{}}\PYG{o}{\PYGZhy{}}\PYG{o}{\PYGZhy{}}\PYG{o}{\PYGZhy{}}\PYG{o}{\PYGZhy{}}\PYG{o}{\PYGZhy{}}\PYG{o}{*}\PYG{o}{/}
\PYG{o}{/}\PYG{o}{*} \PYG{n}{Create} \PYG{n}{a} \PYG{n}{digital} \PYG{n}{certificate} \PYG{o}{*}\PYG{o}{/}
\PYG{o}{/}\PYG{o}{*}\PYG{o}{\PYGZhy{}}\PYG{o}{\PYGZhy{}}\PYG{o}{\PYGZhy{}}\PYG{o}{\PYGZhy{}}\PYG{o}{\PYGZhy{}}\PYG{o}{\PYGZhy{}}\PYG{o}{\PYGZhy{}}\PYG{o}{\PYGZhy{}}\PYG{o}{\PYGZhy{}}\PYG{o}{\PYGZhy{}}\PYG{o}{\PYGZhy{}}\PYG{o}{\PYGZhy{}}\PYG{o}{\PYGZhy{}}\PYG{o}{\PYGZhy{}}\PYG{o}{\PYGZhy{}}\PYG{o}{\PYGZhy{}}\PYG{o}{\PYGZhy{}}\PYG{o}{\PYGZhy{}}\PYG{o}{\PYGZhy{}}\PYG{o}{\PYGZhy{}}\PYG{o}{\PYGZhy{}}\PYG{o}{\PYGZhy{}}\PYG{o}{\PYGZhy{}}\PYG{o}{\PYGZhy{}}\PYG{o}{\PYGZhy{}}\PYG{o}{\PYGZhy{}}\PYG{o}{\PYGZhy{}}\PYG{o}{\PYGZhy{}}\PYG{o}{\PYGZhy{}}\PYG{o}{\PYGZhy{}}\PYG{o}{\PYGZhy{}}\PYG{o}{\PYGZhy{}}\PYG{o}{\PYGZhy{}}\PYG{o}{\PYGZhy{}}\PYG{o}{\PYGZhy{}}\PYG{o}{\PYGZhy{}}\PYG{o}{\PYGZhy{}}\PYG{o}{\PYGZhy{}}\PYG{o}{\PYGZhy{}}\PYG{o}{\PYGZhy{}}\PYG{o}{\PYGZhy{}}\PYG{o}{\PYGZhy{}}\PYG{o}{\PYGZhy{}}\PYG{o}{\PYGZhy{}}\PYG{o}{\PYGZhy{}}\PYG{o}{\PYGZhy{}}\PYG{o}{\PYGZhy{}}\PYG{o}{\PYGZhy{}}\PYG{o}{\PYGZhy{}}\PYG{o}{\PYGZhy{}}\PYG{o}{\PYGZhy{}}\PYG{o}{\PYGZhy{}}\PYG{o}{\PYGZhy{}}\PYG{o}{\PYGZhy{}}\PYG{o}{\PYGZhy{}}\PYG{o}{\PYGZhy{}}\PYG{o}{\PYGZhy{}}\PYG{o}{\PYGZhy{}}\PYG{o}{\PYGZhy{}}\PYG{o}{\PYGZhy{}}\PYG{o}{\PYGZhy{}}\PYG{o}{\PYGZhy{}}\PYG{o}{\PYGZhy{}}\PYG{o}{\PYGZhy{}}\PYG{o}{\PYGZhy{}}\PYG{o}{*}\PYG{o}{/}
\PYG{n}{RACDCERT} \PYG{n}{ID}\PYG{p}{(}\PYG{n}{SPVIRSTC}\PYG{p}{)} \PYG{o}{/}\PYG{o}{*} \PYG{n}{VIRTEL} \PYG{n}{userid} \PYG{o}{*}\PYG{o}{/} \PYG{o}{+}
\PYG{n}{GENCERT} \PYG{o}{+}
\PYG{n}{WITHLABEL}\PYG{p}{(}\PYG{l+s+s1}{\PYGZsq{}}\PYG{l+s+s1}{VIRTEL SSL DEMO}\PYG{l+s+s1}{\PYGZsq{}}\PYG{p}{)} \PYG{o}{+}
\PYG{n}{SIGNWITH}\PYG{p}{(}\PYG{n}{CERTAUTH} \PYG{n}{LABEL}\PYG{p}{(}\PYG{l+s+s1}{\PYGZsq{}}\PYG{l+s+s1}{z/OS signing certificate}\PYG{l+s+s1}{\PYGZsq{}}\PYG{p}{)}\PYG{p}{)} \PYG{o}{+}
\PYG{n}{SUBJECTSDN}\PYG{p}{(} \PYG{o}{+}
\PYG{n}{CN}\PYG{p}{(}\PYG{l+s+s1}{\PYGZsq{}}\PYG{l+s+s1}{RECETTE VIRTEL}\PYG{l+s+s1}{\PYGZsq{}}\PYG{p}{)} \PYG{o}{+}
\PYG{n}{T}\PYG{p}{(}\PYG{l+s+s1}{\PYGZsq{}}\PYG{l+s+s1}{VIRTEL Web Access}\PYG{l+s+s1}{\PYGZsq{}}\PYG{p}{)} \PYG{o}{+}
\PYG{n}{O}\PYG{p}{(}\PYG{l+s+s1}{\PYGZsq{}}\PYG{l+s+s1}{SYSPERTEC}\PYG{l+s+s1}{\PYGZsq{}}\PYG{p}{)} \PYG{o}{+}
\PYG{n}{C}\PYG{p}{(}\PYG{l+s+s1}{\PYGZsq{}}\PYG{l+s+s1}{FR}\PYG{l+s+s1}{\PYGZsq{}}\PYG{p}{)}\PYG{p}{)} \PYG{o}{+}
\PYG{n}{KEYUSAGE}\PYG{p}{(}\PYG{n}{HANDSHAKE} \PYG{n}{DATAENCRYPT}\PYG{p}{)} \PYG{n}{SIZE}\PYG{p}{(}\PYG{l+m+mi}{1024}\PYG{p}{)}
\end{sphinxVerbatim}

\sphinxAtStartPar
Note how we identify the signing certificate with the SIGNWITH parameter using the same label information that we used when defining the RACF signing certificate.

\index{Security@\spxentry{Security}!AT\sphinxhyphen{}TLS@\spxentry{AT\sphinxhyphen{}TLS}}\index{AT\sphinxhyphen{}TLS@\spxentry{AT\sphinxhyphen{}TLS}!Configuring key rings@\spxentry{Configuring key rings}}\index{RACF@\spxentry{RACF}!Configuring key rings@\spxentry{Configuring key rings}}\ignorespaces 

\subsection{Certificates and Key rings}
\label{\detokenize{Customization:certificates-and-key-rings}}\label{\detokenize{Customization:index-108}}
\sphinxAtStartPar
Having generated two of our certificates we now need a place to keep them. We place the certificates on a key ring and associate the key ring with the VIRTEL server RACF user id (in our case SPVIRSTC). The member SSLSETUP has some RACF commands to perform the key ring generation. Here is an extract:

\begin{sphinxVerbatim}[commandchars=\\\{\}]
\PYG{o}{/}\PYG{o}{*}\PYG{o}{\PYGZhy{}}\PYG{o}{\PYGZhy{}}\PYG{o}{\PYGZhy{}}\PYG{o}{\PYGZhy{}}\PYG{o}{\PYGZhy{}}\PYG{o}{\PYGZhy{}}\PYG{o}{\PYGZhy{}}\PYG{o}{\PYGZhy{}}\PYG{o}{\PYGZhy{}}\PYG{o}{\PYGZhy{}}\PYG{o}{\PYGZhy{}}\PYG{o}{\PYGZhy{}}\PYG{o}{\PYGZhy{}}\PYG{o}{\PYGZhy{}}\PYG{o}{\PYGZhy{}}\PYG{o}{\PYGZhy{}}\PYG{o}{\PYGZhy{}}\PYG{o}{\PYGZhy{}}\PYG{o}{\PYGZhy{}}\PYG{o}{\PYGZhy{}}\PYG{o}{\PYGZhy{}}\PYG{o}{\PYGZhy{}}\PYG{o}{\PYGZhy{}}\PYG{o}{\PYGZhy{}}\PYG{o}{\PYGZhy{}}\PYG{o}{\PYGZhy{}}\PYG{o}{\PYGZhy{}}\PYG{o}{\PYGZhy{}}\PYG{o}{\PYGZhy{}}\PYG{o}{\PYGZhy{}}\PYG{o}{\PYGZhy{}}\PYG{o}{\PYGZhy{}}\PYG{o}{\PYGZhy{}}\PYG{o}{\PYGZhy{}}\PYG{o}{\PYGZhy{}}\PYG{o}{\PYGZhy{}}\PYG{o}{\PYGZhy{}}\PYG{o}{\PYGZhy{}}\PYG{o}{\PYGZhy{}}\PYG{o}{\PYGZhy{}}\PYG{o}{\PYGZhy{}}\PYG{o}{\PYGZhy{}}\PYG{o}{\PYGZhy{}}\PYG{o}{\PYGZhy{}}\PYG{o}{\PYGZhy{}}\PYG{o}{\PYGZhy{}}\PYG{o}{\PYGZhy{}}\PYG{o}{\PYGZhy{}}\PYG{o}{\PYGZhy{}}\PYG{o}{\PYGZhy{}}\PYG{o}{\PYGZhy{}}\PYG{o}{\PYGZhy{}}\PYG{o}{\PYGZhy{}}\PYG{o}{\PYGZhy{}}\PYG{o}{\PYGZhy{}}\PYG{o}{\PYGZhy{}}\PYG{o}{\PYGZhy{}}\PYG{o}{\PYGZhy{}}\PYG{o}{\PYGZhy{}}\PYG{o}{\PYGZhy{}}\PYG{o}{\PYGZhy{}}\PYG{o}{\PYGZhy{}}\PYG{o}{\PYGZhy{}}\PYG{o}{\PYGZhy{}}\PYG{o}{\PYGZhy{}}\PYG{o}{*}\PYG{o}{/}
\PYG{o}{/}\PYG{o}{*} \PYG{n}{Create} \PYG{n}{a} \PYG{n}{keyring} \PYG{o}{*}\PYG{o}{/}
\PYG{o}{/}\PYG{o}{*}\PYG{o}{\PYGZhy{}}\PYG{o}{\PYGZhy{}}\PYG{o}{\PYGZhy{}}\PYG{o}{\PYGZhy{}}\PYG{o}{\PYGZhy{}}\PYG{o}{\PYGZhy{}}\PYG{o}{\PYGZhy{}}\PYG{o}{\PYGZhy{}}\PYG{o}{\PYGZhy{}}\PYG{o}{\PYGZhy{}}\PYG{o}{\PYGZhy{}}\PYG{o}{\PYGZhy{}}\PYG{o}{\PYGZhy{}}\PYG{o}{\PYGZhy{}}\PYG{o}{\PYGZhy{}}\PYG{o}{\PYGZhy{}}\PYG{o}{\PYGZhy{}}\PYG{o}{\PYGZhy{}}\PYG{o}{\PYGZhy{}}\PYG{o}{\PYGZhy{}}\PYG{o}{\PYGZhy{}}\PYG{o}{\PYGZhy{}}\PYG{o}{\PYGZhy{}}\PYG{o}{\PYGZhy{}}\PYG{o}{\PYGZhy{}}\PYG{o}{\PYGZhy{}}\PYG{o}{\PYGZhy{}}\PYG{o}{\PYGZhy{}}\PYG{o}{\PYGZhy{}}\PYG{o}{\PYGZhy{}}\PYG{o}{\PYGZhy{}}\PYG{o}{\PYGZhy{}}\PYG{o}{\PYGZhy{}}\PYG{o}{\PYGZhy{}}\PYG{o}{\PYGZhy{}}\PYG{o}{\PYGZhy{}}\PYG{o}{\PYGZhy{}}\PYG{o}{\PYGZhy{}}\PYG{o}{\PYGZhy{}}\PYG{o}{\PYGZhy{}}\PYG{o}{\PYGZhy{}}\PYG{o}{\PYGZhy{}}\PYG{o}{\PYGZhy{}}\PYG{o}{\PYGZhy{}}\PYG{o}{\PYGZhy{}}\PYG{o}{\PYGZhy{}}\PYG{o}{\PYGZhy{}}\PYG{o}{\PYGZhy{}}\PYG{o}{\PYGZhy{}}\PYG{o}{\PYGZhy{}}\PYG{o}{\PYGZhy{}}\PYG{o}{\PYGZhy{}}\PYG{o}{\PYGZhy{}}\PYG{o}{\PYGZhy{}}\PYG{o}{\PYGZhy{}}\PYG{o}{\PYGZhy{}}\PYG{o}{\PYGZhy{}}\PYG{o}{\PYGZhy{}}\PYG{o}{\PYGZhy{}}\PYG{o}{\PYGZhy{}}\PYG{o}{\PYGZhy{}}\PYG{o}{\PYGZhy{}}\PYG{o}{\PYGZhy{}}\PYG{o}{\PYGZhy{}}\PYG{o}{\PYGZhy{}}\PYG{o}{*}\PYG{o}{/}
\PYG{n}{RACDCERT} \PYG{n}{ID}\PYG{p}{(}\PYG{n}{SPVIRSTC}\PYG{p}{)} \PYG{o}{/}\PYG{o}{*} \PYG{n}{VIRTEL} \PYG{n}{userid} \PYG{o}{*}\PYG{o}{/} \PYG{o}{+}
\PYG{n}{ADDRING}\PYG{p}{(}\PYG{n}{VIRTRING}\PYG{p}{)}
\PYG{o}{/}\PYG{o}{*}\PYG{o}{\PYGZhy{}}\PYG{o}{\PYGZhy{}}\PYG{o}{\PYGZhy{}}\PYG{o}{\PYGZhy{}}\PYG{o}{\PYGZhy{}}\PYG{o}{\PYGZhy{}}\PYG{o}{\PYGZhy{}}\PYG{o}{\PYGZhy{}}\PYG{o}{\PYGZhy{}}\PYG{o}{\PYGZhy{}}\PYG{o}{\PYGZhy{}}\PYG{o}{\PYGZhy{}}\PYG{o}{\PYGZhy{}}\PYG{o}{\PYGZhy{}}\PYG{o}{\PYGZhy{}}\PYG{o}{\PYGZhy{}}\PYG{o}{\PYGZhy{}}\PYG{o}{\PYGZhy{}}\PYG{o}{\PYGZhy{}}\PYG{o}{\PYGZhy{}}\PYG{o}{\PYGZhy{}}\PYG{o}{\PYGZhy{}}\PYG{o}{\PYGZhy{}}\PYG{o}{\PYGZhy{}}\PYG{o}{\PYGZhy{}}\PYG{o}{\PYGZhy{}}\PYG{o}{\PYGZhy{}}\PYG{o}{\PYGZhy{}}\PYG{o}{\PYGZhy{}}\PYG{o}{\PYGZhy{}}\PYG{o}{\PYGZhy{}}\PYG{o}{\PYGZhy{}}\PYG{o}{\PYGZhy{}}\PYG{o}{\PYGZhy{}}\PYG{o}{\PYGZhy{}}\PYG{o}{\PYGZhy{}}\PYG{o}{\PYGZhy{}}\PYG{o}{\PYGZhy{}}\PYG{o}{\PYGZhy{}}\PYG{o}{\PYGZhy{}}\PYG{o}{\PYGZhy{}}\PYG{o}{\PYGZhy{}}\PYG{o}{\PYGZhy{}}\PYG{o}{\PYGZhy{}}\PYG{o}{\PYGZhy{}}\PYG{o}{\PYGZhy{}}\PYG{o}{\PYGZhy{}}\PYG{o}{\PYGZhy{}}\PYG{o}{\PYGZhy{}}\PYG{o}{\PYGZhy{}}\PYG{o}{\PYGZhy{}}\PYG{o}{\PYGZhy{}}\PYG{o}{\PYGZhy{}}\PYG{o}{\PYGZhy{}}\PYG{o}{\PYGZhy{}}\PYG{o}{\PYGZhy{}}\PYG{o}{\PYGZhy{}}\PYG{o}{\PYGZhy{}}\PYG{o}{\PYGZhy{}}\PYG{o}{\PYGZhy{}}\PYG{o}{\PYGZhy{}}\PYG{o}{\PYGZhy{}}\PYG{o}{\PYGZhy{}}\PYG{o}{\PYGZhy{}}\PYG{o}{\PYGZhy{}}\PYG{o}{*}\PYG{o}{/}
\PYG{o}{/}\PYG{o}{*} \PYG{n}{Add} \PYG{n}{the} \PYG{n}{certificate} \PYG{n}{to} \PYG{n}{the} \PYG{n}{keyring} \PYG{o}{*}\PYG{o}{/}
\PYG{o}{/}\PYG{o}{*}\PYG{o}{\PYGZhy{}}\PYG{o}{\PYGZhy{}}\PYG{o}{\PYGZhy{}}\PYG{o}{\PYGZhy{}}\PYG{o}{\PYGZhy{}}\PYG{o}{\PYGZhy{}}\PYG{o}{\PYGZhy{}}\PYG{o}{\PYGZhy{}}\PYG{o}{\PYGZhy{}}\PYG{o}{\PYGZhy{}}\PYG{o}{\PYGZhy{}}\PYG{o}{\PYGZhy{}}\PYG{o}{\PYGZhy{}}\PYG{o}{\PYGZhy{}}\PYG{o}{\PYGZhy{}}\PYG{o}{\PYGZhy{}}\PYG{o}{\PYGZhy{}}\PYG{o}{\PYGZhy{}}\PYG{o}{\PYGZhy{}}\PYG{o}{\PYGZhy{}}\PYG{o}{\PYGZhy{}}\PYG{o}{\PYGZhy{}}\PYG{o}{\PYGZhy{}}\PYG{o}{\PYGZhy{}}\PYG{o}{\PYGZhy{}}\PYG{o}{\PYGZhy{}}\PYG{o}{\PYGZhy{}}\PYG{o}{\PYGZhy{}}\PYG{o}{\PYGZhy{}}\PYG{o}{\PYGZhy{}}\PYG{o}{\PYGZhy{}}\PYG{o}{\PYGZhy{}}\PYG{o}{\PYGZhy{}}\PYG{o}{\PYGZhy{}}\PYG{o}{\PYGZhy{}}\PYG{o}{\PYGZhy{}}\PYG{o}{\PYGZhy{}}\PYG{o}{\PYGZhy{}}\PYG{o}{\PYGZhy{}}\PYG{o}{\PYGZhy{}}\PYG{o}{\PYGZhy{}}\PYG{o}{\PYGZhy{}}\PYG{o}{\PYGZhy{}}\PYG{o}{\PYGZhy{}}\PYG{o}{\PYGZhy{}}\PYG{o}{\PYGZhy{}}\PYG{o}{\PYGZhy{}}\PYG{o}{\PYGZhy{}}\PYG{o}{\PYGZhy{}}\PYG{o}{\PYGZhy{}}\PYG{o}{\PYGZhy{}}\PYG{o}{\PYGZhy{}}\PYG{o}{\PYGZhy{}}\PYG{o}{\PYGZhy{}}\PYG{o}{\PYGZhy{}}\PYG{o}{\PYGZhy{}}\PYG{o}{\PYGZhy{}}\PYG{o}{\PYGZhy{}}\PYG{o}{\PYGZhy{}}\PYG{o}{\PYGZhy{}}\PYG{o}{\PYGZhy{}}\PYG{o}{\PYGZhy{}}\PYG{o}{\PYGZhy{}}\PYG{o}{\PYGZhy{}}\PYG{o}{\PYGZhy{}}\PYG{o}{*}\PYG{o}{/}
\PYG{n}{RACDCERT} \PYG{n}{ID}\PYG{p}{(}\PYG{n}{SPVIRSTC}\PYG{p}{)} \PYG{o}{/}\PYG{o}{*} \PYG{n}{VIRTEL} \PYG{n}{userid} \PYG{o}{*}\PYG{o}{/} \PYG{o}{+}
\PYG{n}{CONNECT}\PYG{p}{(} \PYG{o}{+}
\PYG{n}{ID}\PYG{p}{(}\PYG{n}{SPVIRSTC}\PYG{p}{)} \PYG{o}{+}
\PYG{n}{LABEL}\PYG{p}{(}\PYG{l+s+s1}{\PYGZsq{}}\PYG{l+s+s1}{VIRTEL SSL DEMO}\PYG{l+s+s1}{\PYGZsq{}}\PYG{p}{)} \PYG{o}{+}
\PYG{n}{RING}\PYG{p}{(}\PYG{n}{VIRTRING}\PYG{p}{)} \PYG{o}{+}
\PYG{n}{DEFAULT}\PYG{p}{)}
\end{sphinxVerbatim}

\sphinxAtStartPar
Again it is the label that identifies the key(certificate) that we want to add to the key ring owned by user SPVIRSTC.

\index{Security@\spxentry{Security}!AT\sphinxhyphen{}TLS@\spxentry{AT\sphinxhyphen{}TLS}}\index{AT\sphinxhyphen{}TLS@\spxentry{AT\sphinxhyphen{}TLS}!User certificates@\spxentry{User certificates}}\index{RACF@\spxentry{RACF}!User certificates@\spxentry{User certificates}}\ignorespaces 

\subsubsection{User Certificate}
\label{\detokenize{Customization:user-certificate}}\label{\detokenize{Customization:index-109}}
\sphinxAtStartPar
The next step is to create a user certificate which we will export and import into our browser’s key data base. In the Virtel SAMPLIB member SSLUCERT performs the task of creating the user certificate and creating an “exportable” file.

\begin{sphinxVerbatim}[commandchars=\\\{\}]
\PYG{o}{/}\PYG{o}{/}\PYG{o}{*}\PYG{o}{\PYGZhy{}}\PYG{o}{\PYGZhy{}}\PYG{o}{\PYGZhy{}}\PYG{o}{\PYGZhy{}}\PYG{o}{\PYGZhy{}}\PYG{o}{\PYGZhy{}}\PYG{o}{\PYGZhy{}}\PYG{o}{\PYGZhy{}}\PYG{o}{\PYGZhy{}}\PYG{o}{\PYGZhy{}}\PYG{o}{\PYGZhy{}}\PYG{o}{\PYGZhy{}}\PYG{o}{\PYGZhy{}}\PYG{o}{\PYGZhy{}}\PYG{o}{\PYGZhy{}}\PYG{o}{\PYGZhy{}}\PYG{o}{\PYGZhy{}}\PYG{o}{\PYGZhy{}}\PYG{o}{\PYGZhy{}}\PYG{o}{\PYGZhy{}}\PYG{o}{\PYGZhy{}}\PYG{o}{\PYGZhy{}}\PYG{o}{\PYGZhy{}}\PYG{o}{\PYGZhy{}}\PYG{o}{\PYGZhy{}}\PYG{o}{\PYGZhy{}}\PYG{o}{\PYGZhy{}}\PYG{o}{\PYGZhy{}}\PYG{o}{\PYGZhy{}}\PYG{o}{\PYGZhy{}}\PYG{o}{\PYGZhy{}}\PYG{o}{\PYGZhy{}}\PYG{o}{\PYGZhy{}}\PYG{o}{\PYGZhy{}}\PYG{o}{\PYGZhy{}}\PYG{o}{\PYGZhy{}}\PYG{o}{\PYGZhy{}}\PYG{o}{\PYGZhy{}}\PYG{o}{\PYGZhy{}}\PYG{o}{\PYGZhy{}}\PYG{o}{\PYGZhy{}}\PYG{o}{\PYGZhy{}}\PYG{o}{\PYGZhy{}}\PYG{o}{\PYGZhy{}}\PYG{o}{\PYGZhy{}}\PYG{o}{\PYGZhy{}}\PYG{o}{\PYGZhy{}}\PYG{o}{\PYGZhy{}}\PYG{o}{\PYGZhy{}}\PYG{o}{\PYGZhy{}}\PYG{o}{\PYGZhy{}}\PYG{o}{\PYGZhy{}}\PYG{o}{\PYGZhy{}}\PYG{o}{\PYGZhy{}}\PYG{o}{\PYGZhy{}}\PYG{o}{\PYGZhy{}}\PYG{o}{\PYGZhy{}}\PYG{o}{\PYGZhy{}}\PYG{o}{\PYGZhy{}}\PYG{o}{\PYGZhy{}}\PYG{o}{\PYGZhy{}}\PYG{o}{\PYGZhy{}}\PYG{o}{\PYGZhy{}}\PYG{o}{\PYGZhy{}}\PYG{o}{*}
\PYG{o}{/}\PYG{o}{/}\PYG{o}{*} \PYG{n}{Associate} \PYG{n}{certificate} \PYG{k}{with} \PYG{n}{user} \PYG{n+nb}{id} \PYG{o}{*}
\PYG{o}{/}\PYG{o}{/}\PYG{o}{*}\PYG{o}{\PYGZhy{}}\PYG{o}{\PYGZhy{}}\PYG{o}{\PYGZhy{}}\PYG{o}{\PYGZhy{}}\PYG{o}{\PYGZhy{}}\PYG{o}{\PYGZhy{}}\PYG{o}{\PYGZhy{}}\PYG{o}{\PYGZhy{}}\PYG{o}{\PYGZhy{}}\PYG{o}{\PYGZhy{}}\PYG{o}{\PYGZhy{}}\PYG{o}{\PYGZhy{}}\PYG{o}{\PYGZhy{}}\PYG{o}{\PYGZhy{}}\PYG{o}{\PYGZhy{}}\PYG{o}{\PYGZhy{}}\PYG{o}{\PYGZhy{}}\PYG{o}{\PYGZhy{}}\PYG{o}{\PYGZhy{}}\PYG{o}{\PYGZhy{}}\PYG{o}{\PYGZhy{}}\PYG{o}{\PYGZhy{}}\PYG{o}{\PYGZhy{}}\PYG{o}{\PYGZhy{}}\PYG{o}{\PYGZhy{}}\PYG{o}{\PYGZhy{}}\PYG{o}{\PYGZhy{}}\PYG{o}{\PYGZhy{}}\PYG{o}{\PYGZhy{}}\PYG{o}{\PYGZhy{}}\PYG{o}{\PYGZhy{}}\PYG{o}{\PYGZhy{}}\PYG{o}{\PYGZhy{}}\PYG{o}{\PYGZhy{}}\PYG{o}{\PYGZhy{}}\PYG{o}{\PYGZhy{}}\PYG{o}{\PYGZhy{}}\PYG{o}{\PYGZhy{}}\PYG{o}{\PYGZhy{}}\PYG{o}{\PYGZhy{}}\PYG{o}{\PYGZhy{}}\PYG{o}{\PYGZhy{}}\PYG{o}{\PYGZhy{}}\PYG{o}{\PYGZhy{}}\PYG{o}{\PYGZhy{}}\PYG{o}{\PYGZhy{}}\PYG{o}{\PYGZhy{}}\PYG{o}{\PYGZhy{}}\PYG{o}{\PYGZhy{}}\PYG{o}{\PYGZhy{}}\PYG{o}{\PYGZhy{}}\PYG{o}{\PYGZhy{}}\PYG{o}{\PYGZhy{}}\PYG{o}{\PYGZhy{}}\PYG{o}{\PYGZhy{}}\PYG{o}{\PYGZhy{}}\PYG{o}{\PYGZhy{}}\PYG{o}{\PYGZhy{}}\PYG{o}{\PYGZhy{}}\PYG{o}{\PYGZhy{}}\PYG{o}{\PYGZhy{}}\PYG{o}{\PYGZhy{}}\PYG{o}{\PYGZhy{}}\PYG{o}{\PYGZhy{}}\PYG{o}{*}
\PYG{o}{/}\PYG{o}{/}\PYG{n}{UCERTIF} \PYG{n}{EXEC} \PYG{n}{PGM}\PYG{o}{=}\PYG{n}{IKJEFT1A}
\PYG{o}{/}\PYG{o}{/}\PYG{n}{SYSTSPRT} \PYG{n}{DD} \PYG{n}{SYSOUT}\PYG{o}{=}\PYG{o}{*}
\PYG{o}{/}\PYG{o}{/}\PYG{n}{SYSTSIN} \PYG{n}{DD} \PYG{o}{*}
\PYG{o}{/}\PYG{o}{*}\PYG{o}{\PYGZhy{}}\PYG{o}{\PYGZhy{}}\PYG{o}{\PYGZhy{}}\PYG{o}{\PYGZhy{}}\PYG{o}{\PYGZhy{}}\PYG{o}{\PYGZhy{}}\PYG{o}{\PYGZhy{}}\PYG{o}{\PYGZhy{}}\PYG{o}{\PYGZhy{}}\PYG{o}{\PYGZhy{}}\PYG{o}{\PYGZhy{}}\PYG{o}{\PYGZhy{}}\PYG{o}{\PYGZhy{}}\PYG{o}{\PYGZhy{}}\PYG{o}{\PYGZhy{}}\PYG{o}{\PYGZhy{}}\PYG{o}{\PYGZhy{}}\PYG{o}{\PYGZhy{}}\PYG{o}{\PYGZhy{}}\PYG{o}{\PYGZhy{}}\PYG{o}{\PYGZhy{}}\PYG{o}{\PYGZhy{}}\PYG{o}{\PYGZhy{}}\PYG{o}{\PYGZhy{}}\PYG{o}{\PYGZhy{}}\PYG{o}{\PYGZhy{}}\PYG{o}{\PYGZhy{}}\PYG{o}{\PYGZhy{}}\PYG{o}{\PYGZhy{}}\PYG{o}{\PYGZhy{}}\PYG{o}{\PYGZhy{}}\PYG{o}{\PYGZhy{}}\PYG{o}{\PYGZhy{}}\PYG{o}{\PYGZhy{}}\PYG{o}{\PYGZhy{}}\PYG{o}{\PYGZhy{}}\PYG{o}{\PYGZhy{}}\PYG{o}{\PYGZhy{}}\PYG{o}{\PYGZhy{}}\PYG{o}{\PYGZhy{}}\PYG{o}{\PYGZhy{}}\PYG{o}{\PYGZhy{}}\PYG{o}{\PYGZhy{}}\PYG{o}{\PYGZhy{}}\PYG{o}{\PYGZhy{}}\PYG{o}{\PYGZhy{}}\PYG{o}{\PYGZhy{}}\PYG{o}{\PYGZhy{}}\PYG{o}{\PYGZhy{}}\PYG{o}{\PYGZhy{}}\PYG{o}{\PYGZhy{}}\PYG{o}{\PYGZhy{}}\PYG{o}{\PYGZhy{}}\PYG{o}{\PYGZhy{}}\PYG{o}{\PYGZhy{}}\PYG{o}{\PYGZhy{}}\PYG{o}{\PYGZhy{}}\PYG{o}{\PYGZhy{}}\PYG{o}{\PYGZhy{}}\PYG{o}{\PYGZhy{}}\PYG{o}{\PYGZhy{}}\PYG{o}{\PYGZhy{}}\PYG{o}{\PYGZhy{}}\PYG{o}{\PYGZhy{}}\PYG{o}{\PYGZhy{}}\PYG{o}{*}\PYG{o}{/}
\PYG{o}{/}\PYG{o}{*} \PYG{n}{Add} \PYG{n}{certificate} \PYG{n}{to} \PYG{n}{Server} \PYG{n}{ring} \PYG{o}{*}\PYG{o}{/}
\PYG{o}{/}\PYG{o}{*}\PYG{o}{\PYGZhy{}}\PYG{o}{\PYGZhy{}}\PYG{o}{\PYGZhy{}}\PYG{o}{\PYGZhy{}}\PYG{o}{\PYGZhy{}}\PYG{o}{\PYGZhy{}}\PYG{o}{\PYGZhy{}}\PYG{o}{\PYGZhy{}}\PYG{o}{\PYGZhy{}}\PYG{o}{\PYGZhy{}}\PYG{o}{\PYGZhy{}}\PYG{o}{\PYGZhy{}}\PYG{o}{\PYGZhy{}}\PYG{o}{\PYGZhy{}}\PYG{o}{\PYGZhy{}}\PYG{o}{\PYGZhy{}}\PYG{o}{\PYGZhy{}}\PYG{o}{\PYGZhy{}}\PYG{o}{\PYGZhy{}}\PYG{o}{\PYGZhy{}}\PYG{o}{\PYGZhy{}}\PYG{o}{\PYGZhy{}}\PYG{o}{\PYGZhy{}}\PYG{o}{\PYGZhy{}}\PYG{o}{\PYGZhy{}}\PYG{o}{\PYGZhy{}}\PYG{o}{\PYGZhy{}}\PYG{o}{\PYGZhy{}}\PYG{o}{\PYGZhy{}}\PYG{o}{\PYGZhy{}}\PYG{o}{\PYGZhy{}}\PYG{o}{\PYGZhy{}}\PYG{o}{\PYGZhy{}}\PYG{o}{\PYGZhy{}}\PYG{o}{\PYGZhy{}}\PYG{o}{\PYGZhy{}}\PYG{o}{\PYGZhy{}}\PYG{o}{\PYGZhy{}}\PYG{o}{\PYGZhy{}}\PYG{o}{\PYGZhy{}}\PYG{o}{\PYGZhy{}}\PYG{o}{\PYGZhy{}}\PYG{o}{\PYGZhy{}}\PYG{o}{\PYGZhy{}}\PYG{o}{\PYGZhy{}}\PYG{o}{\PYGZhy{}}\PYG{o}{\PYGZhy{}}\PYG{o}{\PYGZhy{}}\PYG{o}{\PYGZhy{}}\PYG{o}{\PYGZhy{}}\PYG{o}{\PYGZhy{}}\PYG{o}{\PYGZhy{}}\PYG{o}{\PYGZhy{}}\PYG{o}{\PYGZhy{}}\PYG{o}{\PYGZhy{}}\PYG{o}{\PYGZhy{}}\PYG{o}{\PYGZhy{}}\PYG{o}{\PYGZhy{}}\PYG{o}{\PYGZhy{}}\PYG{o}{\PYGZhy{}}\PYG{o}{\PYGZhy{}}\PYG{o}{\PYGZhy{}}\PYG{o}{\PYGZhy{}}\PYG{o}{\PYGZhy{}}\PYG{o}{\PYGZhy{}}\PYG{o}{*}\PYG{o}{/}
\PYG{n}{RACDCERT} \PYG{n}{ID}\PYG{p}{(}\PYG{n}{SPVIRSTC}\PYG{p}{)} \PYG{o}{/}\PYG{o}{*} \PYG{n}{client} \PYG{n}{userid} \PYG{o}{*}\PYG{o}{/} \PYG{o}{+}
\PYG{n}{CONNECT} \PYG{p}{(}\PYG{n}{CERTAUTH} \PYG{o}{+}
\PYG{n}{LABEL}\PYG{p}{(}\PYG{l+s+s1}{\PYGZsq{}}\PYG{l+s+s1}{z/OS signing certificate}\PYG{l+s+s1}{\PYGZsq{}}\PYG{p}{)} \PYG{o}{+}
\PYG{n}{RING}\PYG{p}{(}\PYG{n}{VIRTRING}\PYG{p}{)} \PYG{o}{+}
\PYG{n}{USAGE}\PYG{p}{(}\PYG{n}{CERTAUTH}\PYG{p}{)}\PYG{p}{)}
\PYG{o}{/}\PYG{o}{*}\PYG{o}{\PYGZhy{}}\PYG{o}{\PYGZhy{}}\PYG{o}{\PYGZhy{}}\PYG{o}{\PYGZhy{}}\PYG{o}{\PYGZhy{}}\PYG{o}{\PYGZhy{}}\PYG{o}{\PYGZhy{}}\PYG{o}{\PYGZhy{}}\PYG{o}{\PYGZhy{}}\PYG{o}{\PYGZhy{}}\PYG{o}{\PYGZhy{}}\PYG{o}{\PYGZhy{}}\PYG{o}{\PYGZhy{}}\PYG{o}{\PYGZhy{}}\PYG{o}{\PYGZhy{}}\PYG{o}{\PYGZhy{}}\PYG{o}{\PYGZhy{}}\PYG{o}{\PYGZhy{}}\PYG{o}{\PYGZhy{}}\PYG{o}{\PYGZhy{}}\PYG{o}{\PYGZhy{}}\PYG{o}{\PYGZhy{}}\PYG{o}{\PYGZhy{}}\PYG{o}{\PYGZhy{}}\PYG{o}{\PYGZhy{}}\PYG{o}{\PYGZhy{}}\PYG{o}{\PYGZhy{}}\PYG{o}{\PYGZhy{}}\PYG{o}{\PYGZhy{}}\PYG{o}{\PYGZhy{}}\PYG{o}{\PYGZhy{}}\PYG{o}{\PYGZhy{}}\PYG{o}{\PYGZhy{}}\PYG{o}{\PYGZhy{}}\PYG{o}{\PYGZhy{}}\PYG{o}{\PYGZhy{}}\PYG{o}{\PYGZhy{}}\PYG{o}{\PYGZhy{}}\PYG{o}{\PYGZhy{}}\PYG{o}{\PYGZhy{}}\PYG{o}{\PYGZhy{}}\PYG{o}{\PYGZhy{}}\PYG{o}{\PYGZhy{}}\PYG{o}{\PYGZhy{}}\PYG{o}{\PYGZhy{}}\PYG{o}{\PYGZhy{}}\PYG{o}{\PYGZhy{}}\PYG{o}{\PYGZhy{}}\PYG{o}{\PYGZhy{}}\PYG{o}{\PYGZhy{}}\PYG{o}{\PYGZhy{}}\PYG{o}{\PYGZhy{}}\PYG{o}{\PYGZhy{}}\PYG{o}{\PYGZhy{}}\PYG{o}{\PYGZhy{}}\PYG{o}{\PYGZhy{}}\PYG{o}{\PYGZhy{}}\PYG{o}{\PYGZhy{}}\PYG{o}{\PYGZhy{}}\PYG{o}{\PYGZhy{}}\PYG{o}{\PYGZhy{}}\PYG{o}{\PYGZhy{}}\PYG{o}{\PYGZhy{}}\PYG{o}{\PYGZhy{}}\PYG{o}{\PYGZhy{}}\PYG{o}{*}\PYG{o}{/}
\PYG{o}{/}\PYG{o}{*} \PYG{n}{Add} \PYG{n}{certificate} \PYG{n}{to} \PYG{n}{Server} \PYG{n}{ring} \PYG{o}{*}\PYG{o}{/}
\PYG{o}{/}\PYG{o}{*}\PYG{o}{\PYGZhy{}}\PYG{o}{\PYGZhy{}}\PYG{o}{\PYGZhy{}}\PYG{o}{\PYGZhy{}}\PYG{o}{\PYGZhy{}}\PYG{o}{\PYGZhy{}}\PYG{o}{\PYGZhy{}}\PYG{o}{\PYGZhy{}}\PYG{o}{\PYGZhy{}}\PYG{o}{\PYGZhy{}}\PYG{o}{\PYGZhy{}}\PYG{o}{\PYGZhy{}}\PYG{o}{\PYGZhy{}}\PYG{o}{\PYGZhy{}}\PYG{o}{\PYGZhy{}}\PYG{o}{\PYGZhy{}}\PYG{o}{\PYGZhy{}}\PYG{o}{\PYGZhy{}}\PYG{o}{\PYGZhy{}}\PYG{o}{\PYGZhy{}}\PYG{o}{\PYGZhy{}}\PYG{o}{\PYGZhy{}}\PYG{o}{\PYGZhy{}}\PYG{o}{\PYGZhy{}}\PYG{o}{\PYGZhy{}}\PYG{o}{\PYGZhy{}}\PYG{o}{\PYGZhy{}}\PYG{o}{\PYGZhy{}}\PYG{o}{\PYGZhy{}}\PYG{o}{\PYGZhy{}}\PYG{o}{\PYGZhy{}}\PYG{o}{\PYGZhy{}}\PYG{o}{\PYGZhy{}}\PYG{o}{\PYGZhy{}}\PYG{o}{\PYGZhy{}}\PYG{o}{\PYGZhy{}}\PYG{o}{\PYGZhy{}}\PYG{o}{\PYGZhy{}}\PYG{o}{\PYGZhy{}}\PYG{o}{\PYGZhy{}}\PYG{o}{\PYGZhy{}}\PYG{o}{\PYGZhy{}}\PYG{o}{\PYGZhy{}}\PYG{o}{\PYGZhy{}}\PYG{o}{\PYGZhy{}}\PYG{o}{\PYGZhy{}}\PYG{o}{\PYGZhy{}}\PYG{o}{\PYGZhy{}}\PYG{o}{\PYGZhy{}}\PYG{o}{\PYGZhy{}}\PYG{o}{\PYGZhy{}}\PYG{o}{\PYGZhy{}}\PYG{o}{\PYGZhy{}}\PYG{o}{\PYGZhy{}}\PYG{o}{\PYGZhy{}}\PYG{o}{\PYGZhy{}}\PYG{o}{\PYGZhy{}}\PYG{o}{\PYGZhy{}}\PYG{o}{\PYGZhy{}}\PYG{o}{\PYGZhy{}}\PYG{o}{\PYGZhy{}}\PYG{o}{\PYGZhy{}}\PYG{o}{\PYGZhy{}}\PYG{o}{\PYGZhy{}}\PYG{o}{\PYGZhy{}}\PYG{o}{*}\PYG{o}{/}
\PYG{n}{RACDCERT} \PYG{n}{ID}\PYG{p}{(}\PYG{n}{SPVIRSTC}\PYG{p}{)} \PYG{o}{/}\PYG{o}{*} \PYG{n}{client} \PYG{n}{userid} \PYG{o}{*}\PYG{o}{/} \PYG{o}{+}
\PYG{n}{CONNECT} \PYG{p}{(}\PYG{n}{ID}\PYG{p}{(}\PYG{n}{HLQ}\PYG{p}{)} \PYG{o}{+}
\PYG{n}{LABEL}\PYG{p}{(}\PYG{l+s+s1}{\PYGZsq{}}\PYG{l+s+s1}{SSL client certificate}\PYG{l+s+s1}{\PYGZsq{}}\PYG{p}{)} \PYG{o}{+}
\PYG{n}{RING}\PYG{p}{(}\PYG{n}{VIRTRING}\PYG{p}{)} \PYG{o}{+}
\PYG{n}{USAGE}\PYG{p}{(}\PYG{n}{CERTAUTH}\PYG{p}{)}\PYG{p}{)}
\PYG{o}{/}\PYG{o}{*}\PYG{o}{\PYGZhy{}}\PYG{o}{\PYGZhy{}}\PYG{o}{\PYGZhy{}}\PYG{o}{\PYGZhy{}}\PYG{o}{\PYGZhy{}}\PYG{o}{\PYGZhy{}}\PYG{o}{\PYGZhy{}}\PYG{o}{\PYGZhy{}}\PYG{o}{\PYGZhy{}}\PYG{o}{\PYGZhy{}}\PYG{o}{\PYGZhy{}}\PYG{o}{\PYGZhy{}}\PYG{o}{\PYGZhy{}}\PYG{o}{\PYGZhy{}}\PYG{o}{\PYGZhy{}}\PYG{o}{\PYGZhy{}}\PYG{o}{\PYGZhy{}}\PYG{o}{\PYGZhy{}}\PYG{o}{\PYGZhy{}}\PYG{o}{\PYGZhy{}}\PYG{o}{\PYGZhy{}}\PYG{o}{\PYGZhy{}}\PYG{o}{\PYGZhy{}}\PYG{o}{\PYGZhy{}}\PYG{o}{\PYGZhy{}}\PYG{o}{\PYGZhy{}}\PYG{o}{\PYGZhy{}}\PYG{o}{\PYGZhy{}}\PYG{o}{\PYGZhy{}}\PYG{o}{\PYGZhy{}}\PYG{o}{\PYGZhy{}}\PYG{o}{\PYGZhy{}}\PYG{o}{\PYGZhy{}}\PYG{o}{\PYGZhy{}}\PYG{o}{\PYGZhy{}}\PYG{o}{\PYGZhy{}}\PYG{o}{\PYGZhy{}}\PYG{o}{\PYGZhy{}}\PYG{o}{\PYGZhy{}}\PYG{o}{\PYGZhy{}}\PYG{o}{\PYGZhy{}}\PYG{o}{\PYGZhy{}}\PYG{o}{\PYGZhy{}}\PYG{o}{\PYGZhy{}}\PYG{o}{\PYGZhy{}}\PYG{o}{\PYGZhy{}}\PYG{o}{\PYGZhy{}}\PYG{o}{\PYGZhy{}}\PYG{o}{\PYGZhy{}}\PYG{o}{\PYGZhy{}}\PYG{o}{\PYGZhy{}}\PYG{o}{\PYGZhy{}}\PYG{o}{\PYGZhy{}}\PYG{o}{\PYGZhy{}}\PYG{o}{\PYGZhy{}}\PYG{o}{\PYGZhy{}}\PYG{o}{\PYGZhy{}}\PYG{o}{\PYGZhy{}}\PYG{o}{\PYGZhy{}}\PYG{o}{\PYGZhy{}}\PYG{o}{\PYGZhy{}}\PYG{o}{\PYGZhy{}}\PYG{o}{\PYGZhy{}}\PYG{o}{\PYGZhy{}}\PYG{o}{\PYGZhy{}}\PYG{o}{*}\PYG{o}{/}
\PYG{o}{/}\PYG{o}{*} \PYG{n}{Refresh} \PYG{n}{the} \PYG{n}{RACF} \PYG{n}{profiles} \PYG{o}{*}\PYG{o}{/}
\PYG{o}{/}\PYG{o}{*}\PYG{o}{\PYGZhy{}}\PYG{o}{\PYGZhy{}}\PYG{o}{\PYGZhy{}}\PYG{o}{\PYGZhy{}}\PYG{o}{\PYGZhy{}}\PYG{o}{\PYGZhy{}}\PYG{o}{\PYGZhy{}}\PYG{o}{\PYGZhy{}}\PYG{o}{\PYGZhy{}}\PYG{o}{\PYGZhy{}}\PYG{o}{\PYGZhy{}}\PYG{o}{\PYGZhy{}}\PYG{o}{\PYGZhy{}}\PYG{o}{\PYGZhy{}}\PYG{o}{\PYGZhy{}}\PYG{o}{\PYGZhy{}}\PYG{o}{\PYGZhy{}}\PYG{o}{\PYGZhy{}}\PYG{o}{\PYGZhy{}}\PYG{o}{\PYGZhy{}}\PYG{o}{\PYGZhy{}}\PYG{o}{\PYGZhy{}}\PYG{o}{\PYGZhy{}}\PYG{o}{\PYGZhy{}}\PYG{o}{\PYGZhy{}}\PYG{o}{\PYGZhy{}}\PYG{o}{\PYGZhy{}}\PYG{o}{\PYGZhy{}}\PYG{o}{\PYGZhy{}}\PYG{o}{\PYGZhy{}}\PYG{o}{\PYGZhy{}}\PYG{o}{\PYGZhy{}}\PYG{o}{\PYGZhy{}}\PYG{o}{\PYGZhy{}}\PYG{o}{\PYGZhy{}}\PYG{o}{\PYGZhy{}}\PYG{o}{\PYGZhy{}}\PYG{o}{\PYGZhy{}}\PYG{o}{\PYGZhy{}}\PYG{o}{\PYGZhy{}}\PYG{o}{\PYGZhy{}}\PYG{o}{\PYGZhy{}}\PYG{o}{\PYGZhy{}}\PYG{o}{\PYGZhy{}}\PYG{o}{\PYGZhy{}}\PYG{o}{\PYGZhy{}}\PYG{o}{\PYGZhy{}}\PYG{o}{\PYGZhy{}}\PYG{o}{\PYGZhy{}}\PYG{o}{\PYGZhy{}}\PYG{o}{\PYGZhy{}}\PYG{o}{\PYGZhy{}}\PYG{o}{\PYGZhy{}}\PYG{o}{\PYGZhy{}}\PYG{o}{\PYGZhy{}}\PYG{o}{\PYGZhy{}}\PYG{o}{\PYGZhy{}}\PYG{o}{\PYGZhy{}}\PYG{o}{\PYGZhy{}}\PYG{o}{\PYGZhy{}}\PYG{o}{\PYGZhy{}}\PYG{o}{\PYGZhy{}}\PYG{o}{\PYGZhy{}}\PYG{o}{\PYGZhy{}}\PYG{o}{\PYGZhy{}}\PYG{o}{*}\PYG{o}{/}
\PYG{n}{SETROPTS} \PYG{n}{RACLIST}\PYG{p}{(}\PYG{n}{DIGTRING}\PYG{p}{)} \PYG{n}{REFRESH}
\PYG{n}{SETROPTS} \PYG{n}{RACLIST}\PYG{p}{(}\PYG{n}{DIGTCERT}\PYG{p}{)} \PYG{n}{REFRESH}
\end{sphinxVerbatim}

\sphinxAtStartPar
The “CONNECT CERTAUTH” tells RACF that this is a signing CA certificate and the “CONNECT ID(HLQ) indicates that the certificate labelled ‘SSL client certificate’ is associated with USERID HLQ. This is how Virtel obtains the USERID. Also, note that we refresh the RACF profiles related to certificates and key rings. If we list our key ring for user SPVIRSTC we should have three certificates.

\begin{sphinxVerbatim}[commandchars=\\\{\}]
\PYG{n}{READY}
\PYG{n}{RACDCERT} \PYG{n}{ID}\PYG{p}{(}\PYG{n}{SPVIRSTC}\PYG{p}{)} \PYG{n}{LISTRING}\PYG{p}{(}\PYG{n}{VIRTRING}\PYG{p}{)}
\PYG{n}{Digital} \PYG{n}{ring} \PYG{n}{information} \PYG{k}{for} \PYG{n}{user} \PYG{n}{SPVIRSTC}\PYG{p}{:}
\PYG{n}{Ring}\PYG{p}{:}
    \PYG{o}{\PYGZgt{}}\PYG{n}{VIRTRING}\PYG{o}{\PYGZlt{}}
\PYG{n}{Certificate}         \PYG{n}{Label}   \PYG{n}{Name}    \PYG{n}{Cert} \PYG{n}{Owner}     \PYG{n}{USAGE}        \PYG{n}{DEFAULT}
\PYG{o}{\PYGZhy{}}\PYG{o}{\PYGZhy{}}\PYG{o}{\PYGZhy{}}\PYG{o}{\PYGZhy{}}\PYG{o}{\PYGZhy{}}\PYG{o}{\PYGZhy{}}\PYG{o}{\PYGZhy{}}\PYG{o}{\PYGZhy{}}\PYG{o}{\PYGZhy{}}\PYG{o}{\PYGZhy{}}\PYG{o}{\PYGZhy{}}\PYG{o}{\PYGZhy{}}\PYG{o}{\PYGZhy{}}\PYG{o}{\PYGZhy{}}\PYG{o}{\PYGZhy{}}\PYG{o}{\PYGZhy{}}\PYG{o}{\PYGZhy{}}\PYG{o}{\PYGZhy{}}\PYG{o}{\PYGZhy{}}\PYG{o}{\PYGZhy{}}\PYG{o}{\PYGZhy{}}\PYG{o}{\PYGZhy{}}\PYG{o}{\PYGZhy{}}\PYG{o}{\PYGZhy{}}\PYG{o}{\PYGZhy{}}\PYG{o}{\PYGZhy{}}\PYG{o}{\PYGZhy{}}\PYG{o}{\PYGZhy{}}\PYG{o}{\PYGZhy{}}\PYG{o}{\PYGZhy{}}\PYG{o}{\PYGZhy{}}\PYG{o}{\PYGZhy{}}    \PYG{o}{\PYGZhy{}}\PYG{o}{\PYGZhy{}}\PYG{o}{\PYGZhy{}}\PYG{o}{\PYGZhy{}}\PYG{o}{\PYGZhy{}}\PYG{o}{\PYGZhy{}}\PYG{o}{\PYGZhy{}}\PYG{o}{\PYGZhy{}}\PYG{o}{\PYGZhy{}}\PYG{o}{\PYGZhy{}}\PYG{o}{\PYGZhy{}}\PYG{o}{\PYGZhy{}}   \PYG{o}{\PYGZhy{}}\PYG{o}{\PYGZhy{}}\PYG{o}{\PYGZhy{}}\PYG{o}{\PYGZhy{}}\PYG{o}{\PYGZhy{}}\PYG{o}{\PYGZhy{}}\PYG{o}{\PYGZhy{}}\PYG{o}{\PYGZhy{}}     \PYG{o}{\PYGZhy{}}\PYG{o}{\PYGZhy{}}\PYG{o}{\PYGZhy{}}\PYG{o}{\PYGZhy{}}\PYG{o}{\PYGZhy{}}\PYG{o}{\PYGZhy{}}\PYG{o}{\PYGZhy{}}
\PYG{n}{VIRTEL} \PYG{n}{SSL}          \PYG{n}{DEMO}            \PYG{n}{ID}\PYG{p}{(}\PYG{n}{SPVIRSTC}\PYG{p}{)}   \PYG{n}{PERSONAL}     \PYG{n}{YES}
\PYG{n}{z}\PYG{o}{/}\PYG{n}{OS} \PYG{n}{signing} \PYG{n}{certificate}            \PYG{n}{CERTAUTH}       \PYG{n}{CERTAUTH}     \PYG{n}{NO}
\PYG{n}{SSL} \PYG{n}{client} \PYG{n}{certificate}              \PYG{n}{ID}\PYG{p}{(}\PYG{n}{HLQ}\PYG{p}{)}    \PYG{n}{CERTAUTH}     \PYG{n}{NO}
\end{sphinxVerbatim}

\index{Security@\spxentry{Security}!AT\sphinxhyphen{}TLS@\spxentry{AT\sphinxhyphen{}TLS}}\index{AT\sphinxhyphen{}TLS@\spxentry{AT\sphinxhyphen{}TLS}!Importing certificates@\spxentry{Importing certificates}}\index{RACF@\spxentry{RACF}!Importing certificates@\spxentry{Importing certificates}}\ignorespaces 

\subsubsection{Importing the certificate on the client work station.}
\label{\detokenize{Customization:importing-the-certificate-on-the-client-work-station}}\label{\detokenize{Customization:index-110}}
\sphinxAtStartPar
To import the user certificate into the client workstation the P12 file must be downloaded in binary and then the
certificate import wizard is run to import the certificate.

\sphinxAtStartPar
\sphinxincludegraphics{{image19}.png}

\sphinxAtStartPar
After importing the following panel is displayed:

\sphinxAtStartPar
\sphinxincludegraphics{{image20}.png}

\sphinxAtStartPar
At this stage we have completed our certificate generation. Through the use of the certificates we will be able to initiate a secure session (https) with an application and obtain a user id.

\index{Security@\spxentry{Security}!PassTicket Support@\spxentry{PassTicket Support}}\index{PassTicket Support@\spxentry{PassTicket Support}!Setting up RACF for PassTicket@\spxentry{Setting up RACF for PassTicket}}\ignorespaces 

\subsection{PassTicket support}
\label{\detokenize{Customization:passticket-support}}\label{\detokenize{Customization:index-111}}
\sphinxAtStartPar
The next step is to obtain a pass ticket in place of a password so that Virtel can log on to the target application and present a user id and password combination on behalf of the user. The following job will enable PassTicket support for our target application SPCICSH and using user id SPVIRSTC, out Virtel server user id. This job will have to be customized
accordingly:

\begin{sphinxVerbatim}[commandchars=\\\{\}]
\PYG{o}{/}\PYG{o}{/}\PYG{n}{STEP1} \PYG{n}{EXEC} \PYG{n}{PGM}\PYG{o}{=}\PYG{n}{IKJEFT1A}\PYG{p}{,}\PYG{n}{DYNAMNBR}\PYG{o}{=}\PYG{l+m+mi}{20}
\PYG{o}{/}\PYG{o}{/}\PYG{o}{*} \PYG{n}{RDEFINE} \PYG{n}{FACILITY} \PYG{n}{IRR}\PYG{o}{.}\PYG{n}{RTICKETSERV}
\PYG{o}{/}\PYG{o}{/}\PYG{n}{SYSTSPRT} \PYG{n}{DD} \PYG{n}{SYSOUT}\PYG{o}{=}\PYG{o}{*}
\PYG{o}{/}\PYG{o}{/}\PYG{n}{SYSTSIN} \PYG{n}{DD} \PYG{o}{*}
 \PYG{n}{SETROPTS} \PYG{n}{CLASSACT}\PYG{p}{(}\PYG{n}{APPL}\PYG{p}{)}
 \PYG{n}{SETROPTS} \PYG{n}{CLASSACT}\PYG{p}{(}\PYG{n}{PTKTDATA}\PYG{p}{)}
 \PYG{n}{SETROPTS} \PYG{n}{RACLIST}\PYG{p}{(}\PYG{n}{PTKTDATA}\PYG{p}{)}
 \PYG{n}{SETROPTS} \PYG{n}{GENERIC}\PYG{p}{(}\PYG{n}{PTKTDATA}\PYG{p}{)}
 \PYG{n}{RDELETE} \PYG{n}{PTKTDATA} \PYG{n}{SPCICSH}
 \PYG{n}{RDELETE} \PYG{n}{PTKTDATA} \PYG{n}{IRRPTAUTH}\PYG{o}{.}\PYG{n}{SPCICSH}\PYG{o}{.}\PYG{o}{*}
 \PYG{n}{RDEFINE} \PYG{n}{PTKTDATA} \PYG{n}{IRRPTAUTH}\PYG{o}{.}\PYG{n}{SPCICSH}\PYG{o}{.}\PYG{o}{*} \PYG{n}{UACC}\PYG{p}{(}\PYG{n}{NONE}\PYG{p}{)}
 \PYG{n}{RDEFINE} \PYG{n}{PTKTDATA} \PYG{n}{SPCICSH} \PYG{n}{SSIGNON}\PYG{p}{(}\PYG{n}{KEYMASKED}\PYG{p}{(}\PYG{l+m+mi}{998}\PYG{n}{A654FEBCDA123}\PYG{p}{)}\PYG{p}{)} \PYG{o}{+}    \PYG{o}{\PYGZlt{}\PYGZlt{}}\PYG{o}{\PYGZlt{}}
 \PYG{n}{UACC}\PYG{p}{(}\PYG{n}{NONE}\PYG{p}{)}
\PYG{o}{/}\PYG{o}{/}\PYG{n}{STEP1} \PYG{n}{EXEC} \PYG{n}{PGM}\PYG{o}{=}\PYG{n}{IKJEFT1A}\PYG{p}{,}\PYG{n}{DYNAMNBR}\PYG{o}{=}\PYG{l+m+mi}{20}
\PYG{o}{/}\PYG{o}{/}\PYG{n}{SYSTSPRT} \PYG{n}{DD} \PYG{n}{SYSOUT}\PYG{o}{=}\PYG{o}{*}
\PYG{o}{/}\PYG{o}{/}\PYG{n}{SYSTSIN} \PYG{n}{DD} \PYG{o}{*}
 \PYG{n}{PERMIT} \PYG{n}{IRR}\PYG{o}{.}\PYG{n}{RTICKETSERV} \PYG{n}{CL}\PYG{p}{(}\PYG{n}{FACILITY}\PYG{p}{)} \PYG{n}{ID}\PYG{p}{(}\PYG{n}{SPVIRSTC}\PYG{p}{)} \PYG{n}{ACC}\PYG{p}{(}\PYG{n}{READ}\PYG{p}{)}            \PYG{o}{\PYGZlt{}\PYGZlt{}}\PYG{o}{\PYGZlt{}}
 \PYG{n}{PERMIT} \PYG{n}{IRRPTAUTH}\PYG{o}{.}\PYG{n}{SPCICSH}\PYG{o}{.}\PYG{o}{*} \PYG{n}{CL}\PYG{p}{(}\PYG{n}{PTKTDATA}\PYG{p}{)} \PYG{n}{ID}\PYG{p}{(}\PYG{n}{SPVIRSTC}\PYG{p}{)} \PYG{n}{ACC}\PYG{p}{(}\PYG{n}{UPDATE}\PYG{p}{)}
 \PYG{n}{SETROPTS} \PYG{n}{REFRESH} \PYG{n}{RACLIST}\PYG{p}{(}\PYG{n}{PTKTDATA}\PYG{p}{)}
 \PYG{n}{SETROPTS} \PYG{n}{REFRESH} \PYG{n}{RACLIST}\PYG{p}{(}\PYG{n}{FACILITY}\PYG{p}{)}
\end{sphinxVerbatim}

\sphinxAtStartPar
In order for Virtel to generate PassTickets, you must also modify your VIRTCT to include the parameter PASSTCK=YES and then reassemble the VIRTCT. See chapter 6 of the Virtel Installation Guide for more details on the Virtel VIRTCT.

\index{Security@\spxentry{Security}!PassTicket Support@\spxentry{PassTicket Support}}\index{PassTicket Support@\spxentry{PassTicket Support}!Pagent Configuration@\spxentry{Pagent Configuration}}\ignorespaces 

\subsection{PAGENT Configuration}
\label{\detokenize{Customization:pagent-configuration}}\label{\detokenize{Customization:index-112}}
\sphinxAtStartPar
To enable system SSL sessions to take place between the browser and the application we have to tell AT\sphinxhyphen{}TLS and SSL which sockets to intercept. This is configured in the pagent configuration file which can be found in /etc/pagent.conf. The two areas that we are interested in are the TTLSEnvironmentAction section and the TTLSRule section.

\begin{sphinxVerbatim}[commandchars=\\\{\}]
\PYG{n}{TTLSEnvironmentAction} \PYG{n}{VIRTELenvir\PYGZus{}inSec}
\PYG{p}{\PYGZob{}}
    \PYG{n}{HandshakeRole} \PYG{n}{ServerWithClientAuth}
    \PYG{n}{Trace} \PYG{l+m+mi}{7}
    \PYG{n}{TTLSKeyringParms}
    \PYG{p}{\PYGZob{}}
        \PYG{n}{Keyring} \PYG{n}{VIRTRING}
    \PYG{p}{\PYGZcb{}}
    \PYG{n}{TTLSEnvironmentAdvancedParms}
    \PYG{p}{\PYGZob{}}
        \PYG{n}{SSLv2} \PYG{n}{On}
        \PYG{n}{SSLv3} \PYG{n}{On}
        \PYG{n}{TLSv1} \PYG{n}{On}
        \PYG{n}{ClientAuthType} \PYG{n}{SAFCheck}
    \PYG{p}{\PYGZcb{}}
    \PYG{n}{TTLSCipherParmsRef} \PYG{n}{VIRTELcipher}
\PYG{p}{\PYGZcb{}}

\PYG{o}{.}\PYG{o}{.}\PYG{o}{.}\PYG{o}{.}

\PYG{n}{TTLSRule} \PYG{n}{VIRTELrule\PYGZus{}in\PYGZus{}eh}
\PYG{p}{\PYGZob{}}
    \PYG{n}{Jobname} \PYG{n}{SPVIREH}
    \PYG{n}{LocalPortRange} \PYG{l+m+mi}{41002}
    \PYG{n}{Direction} \PYG{n}{Inbound}
    \PYG{n}{TTLSGroupActionRef} \PYG{n}{VIRTELgroup}
    \PYG{n}{TTLSEnvironmentActionRef} \PYG{n}{VIRTELenvir\PYGZus{}inSec}
\PYG{p}{\PYGZcb{}}
\end{sphinxVerbatim}

\sphinxAtStartPar
The TTLS Rule identifies Virtel Started task name via the Jobname parameter and also the port number that can support secured sessions \_ https. In this case it is port 41002.

\sphinxAtStartPar
The rules section also identifies the environmental section. In this case we have selected an environment section called VIRTELenvir\_insec.

\sphinxAtStartPar
In VIRTELenvir\_insec we identify that we want to use both server and client certificates:

\begin{sphinxVerbatim}[commandchars=\\\{\}]
\PYG{n}{HandShakeRole}   \PYG{n}{ServerWithClientAuth}
\end{sphinxVerbatim}

\sphinxAtStartPar
That the user certificate must be associated with a valid RACF userid:

\begin{sphinxVerbatim}[commandchars=\\\{\}]
\PYG{n}{ClientAuthType}  \PYG{n}{SAFCheck}
\end{sphinxVerbatim}

\sphinxAtStartPar
The name of the keyring that holds the keys(certificates):

\begin{sphinxVerbatim}[commandchars=\\\{\}]
\PYG{n}{Keyring}     \PYG{n}{VIRTRING}
\end{sphinxVerbatim}

\sphinxAtStartPar
A default pagent.conf is shipped with the SAMPLIB member SSLSETUP which you can use to modify accordingly to define the above SSL sections. To refresh a pagent.conf profile after
you have made changes you can issue the following z/OS command:

\begin{sphinxVerbatim}[commandchars=\\\{\}]
\PYG{n}{F} \PYG{n}{PAGENT}\PYG{p}{,}\PYG{n}{REFRESH}
\end{sphinxVerbatim}

\index{Security@\spxentry{Security}!AT\sphinxhyphen{}TLS@\spxentry{AT\sphinxhyphen{}TLS}}\index{AT\sphinxhyphen{}TLS@\spxentry{AT\sphinxhyphen{}TLS}!Example of Passtick support with CICS@\spxentry{Example of Passtick support with CICS}}\ignorespaces 

\subsection{Configuring Virtel \sphinxhyphen{} CICS Example}
\label{\detokenize{Customization:configuring-virtel-cics-example}}\label{\detokenize{Customization:index-113}}
\sphinxAtStartPar
The final part in our configuration is to configure Virtel to use SSL to obtain the user id and PassTicket support to create a password. We configure Virtel in the transaction associated with our target application, in this case the CICS application called SPCICSH.

\sphinxAtStartPar
\sphinxincludegraphics{{image21}.png}

\sphinxAtStartPar
\sphinxstyleemphasis{Note that PassTicket is set to 2.} This will enable Virtel to generate a temporary password. Security is set to 3. This indicates that Virtel will receive a USERID based upon the user certificate used in the authentication process. The TIOA at logon is a string that will logon to the CICS application using the user id and password values that Virtel has
obtained.

\sphinxAtStartPar
With this configuration we can logon to our CICS application without the user presenting any user id or password. This is very much like the Express Logon Facility implemented in our Telnet clients.


\subsubsection{Logon Example}
\label{\detokenize{Customization:logon-example}}
\sphinxAtStartPar
In the following screen shots we demonstrate logging into a CICS application via a secure session (https) without specifying any user id/password. Our initial URL is \sphinxurl{https://192.168.170.30:41002/w2h/WEB2AJAX.htm+CICS}; you will replace the IP address with your own installation IP address or domain name.

\sphinxAtStartPar
\sphinxincludegraphics{{image22}.png}

\sphinxAtStartPar
We are presented with a « Select a certificate » window from the browser requesting the certificate we wish to user for authenication purposes. We select the certificate we downloaded. The next panel is a warning panel which identifies that the certificate we are using has not been authenticated by a well\sphinxhyphen{}known CA authority. We are of course aware of this as we are using a RACF self signed certificate.

\sphinxAtStartPar
\sphinxincludegraphics{{image23}.png}

\sphinxAtStartPar
We select Advanced and are then presented with information about the certificate.

\sphinxAtStartPar
\sphinxincludegraphics{{image24}.png}

\sphinxAtStartPar
We select the “Proceed” link.

\sphinxAtStartPar
\sphinxincludegraphics{{image25}.png}

\sphinxAtStartPar
We are signed into CICS without having to specify any user id or password.

\index{Security@\spxentry{Security}!AT\sphinxhyphen{}TLS@\spxentry{AT\sphinxhyphen{}TLS}}\index{AT\sphinxhyphen{}TLS@\spxentry{AT\sphinxhyphen{}TLS}!Troubleshooting@\spxentry{Troubleshooting}}\ignorespaces 

\subsection{AT\sphinxhyphen{}TLS Problems}
\label{\detokenize{Customization:at-tls-problems}}\label{\detokenize{Customization:index-114}}
\sphinxAtStartPar
It is easy to miss something when configuring user certificate sign on. Here are some general guidelines that should help in debugging configuration errors.
\begin{enumerate}
\sphinxsetlistlabels{\arabic}{enumi}{enumii}{}{.}%
\item {} 
\sphinxAtStartPar
Is AT\sphinxhyphen{}TLS active.

\end{enumerate}

\begin{sphinxVerbatim}[commandchars=\\\{\}]
Issue the following z/OS command \textendash{} D TCPIP,,N,TTLS
The response should be :
    EZD0101I NETSTAT CS V1R13 TCPIP 706
    TTLSGRPACTION GROUP ID CONNS
    VIRTELGROUP 00000002 3
    1 OF 1 RECORDS DISPLAYED
    END OF THE REPORT
\end{sphinxVerbatim}
\begin{description}
\sphinxlineitem{2   PAGENT return codes.}
\sphinxAtStartPar
Common session startup/handshake errors are reported through messqge EZD1287I. In the example below we can see that the handshake has failed with a return code of 5003. Return codes under 5000 are generated by System SSL and are defined in the System SSL Programming manual. Return codes over 5000 are generated by AT\sphinxhyphen{}TLS and are defined in the IP Diagnosis Guide. In the following the 5013 suggests that the browser has sent clear text; in other words, http was used instead of https in the URL.

\end{description}

\begin{sphinxVerbatim}[commandchars=\\\{\}]
\PYG{n}{BPXF024I} \PYG{p}{(}\PYG{n}{TCPIP}\PYG{p}{)} \PYG{n}{Oct} \PYG{l+m+mi}{7} \PYG{l+m+mi}{13}\PYG{p}{:}\PYG{l+m+mi}{33}\PYG{p}{:}\PYG{l+m+mi}{08} \PYG{n}{TTLS} \PYG{l+m+mi}{83951769} \PYG{p}{:} \PYG{l+m+mi}{15}\PYG{p}{:}\PYG{l+m+mi}{33}\PYG{p}{:}\PYG{l+m+mi}{08} \PYG{n}{TCPIP} \PYG{l+m+mi}{367}
\PYG{n}{EZD1281I} \PYG{n}{TTLS} \PYG{n}{Map} \PYG{n}{CONNID}\PYG{p}{:} \PYG{l+m+mi}{000006}\PYG{n}{A2} \PYG{n}{LOCAL}\PYG{p}{:} \PYG{l+m+mf}{192.168}\PYG{l+m+mf}{.170}\PYG{l+m+mf}{.30}\PYG{o}{.}\PYG{l+m+mf}{.41002}
\PYG{n}{REMOTE}\PYG{p}{:} \PYG{l+m+mf}{192.168}\PYG{l+m+mf}{.92}\PYG{l+m+mf}{.62}\PYG{o}{.}\PYG{l+m+mf}{.57545} \PYG{n}{JOBNAME}\PYG{p}{:} \PYG{n}{SPVIREH} \PYG{n}{USERID}\PYG{p}{:} \PYG{n}{SPVIRSTC} \PYG{n}{TYPE}\PYG{p}{:}
\PYG{n}{InBound} \PYG{n}{STATUS}\PYG{p}{:} \PYG{n}{Enabled} \PYG{n}{RULE}\PYG{p}{:} \PYG{n}{VIRTELrule\PYGZus{}in\PYGZus{}eh} \PYG{n}{ACTIONS}\PYG{p}{:} \PYG{n}{VIRTELgroup}
\PYG{n}{VIRTELenvir\PYGZus{}inSec} \PYG{o}{*}\PYG{o}{*}\PYG{n}{N}\PYG{o}{/}\PYG{n}{A}\PYG{o}{*}\PYG{o}{*}
\PYG{n}{BPXF024I} \PYG{p}{(}\PYG{n}{TCPIP}\PYG{p}{)} \PYG{n}{Oct} \PYG{l+m+mi}{7} \PYG{l+m+mi}{13}\PYG{p}{:}\PYG{l+m+mi}{33}\PYG{p}{:}\PYG{l+m+mi}{08} \PYG{n}{TTLS} \PYG{l+m+mi}{83951769} \PYG{p}{:} \PYG{l+m+mi}{15}\PYG{p}{:}\PYG{l+m+mi}{33}\PYG{p}{:}\PYG{l+m+mi}{08} \PYG{n}{TCPIP} \PYG{l+m+mi}{368}
\PYG{n}{EZD1286I} \PYG{n}{TTLS} \PYG{n}{Error} \PYG{n}{GRPID}\PYG{p}{:} \PYG{l+m+mi}{00000002} \PYG{n}{ENVID}\PYG{p}{:} \PYG{l+m+mi}{00000000} \PYG{n}{CONNID}\PYG{p}{:} \PYG{l+m+mi}{000006}\PYG{n}{A2}
\PYG{n}{LOCAL}\PYG{p}{:} \PYG{l+m+mf}{192.168}\PYG{l+m+mf}{.170}\PYG{l+m+mf}{.30}\PYG{o}{.}\PYG{l+m+mf}{.41002}
\PYG{n}{REMOTE}\PYG{p}{:} \PYG{l+m+mf}{192.168}\PYG{l+m+mf}{.92}\PYG{l+m+mf}{.62}\PYG{o}{.}\PYG{l+m+mf}{.57545}
\PYG{n}{JOBNAME}\PYG{p}{:} \PYG{n}{SPVIREH} \PYG{n}{USERID}\PYG{p}{:} \PYG{n}{SPVIRSTC}
\PYG{n}{RULE}\PYG{p}{:} \PYG{n}{VIRTELrule\PYGZus{}in\PYGZus{}eh}
\PYG{n}{RC}\PYG{p}{:} \PYG{l+m+mi}{5003} \PYG{n}{Data} \PYG{n}{Decryption} \PYG{n}{EZD1287I} \PYG{n}{TTLS} \PYG{n}{Error} \PYG{n}{RC}\PYG{p}{:} \PYG{l+m+mi}{5003} \PYG{n}{Data} \PYG{n}{Decryption} \PYG{l+m+mi}{369}   \PYG{o}{\PYGZlt{}\PYGZlt{}}\PYG{o}{\PYGZlt{}\PYGZlt{}}
\PYG{n}{LOCAL}\PYG{p}{:} \PYG{l+m+mf}{192.168}\PYG{l+m+mf}{.170}\PYG{l+m+mf}{.30}\PYG{o}{.}\PYG{l+m+mf}{.41002}
\PYG{n}{REMOTE}\PYG{p}{:} \PYG{l+m+mf}{192.168}\PYG{l+m+mf}{.92}\PYG{l+m+mf}{.62}\PYG{o}{.}\PYG{l+m+mf}{.57545}
\PYG{n}{JOBNAME}\PYG{p}{:} \PYG{n}{SPVIREH} \PYG{n}{RULE}\PYG{p}{:}
\PYG{n}{VIRTELrule\PYGZus{}in\PYGZus{}eh} \PYG{n}{USERID}\PYG{p}{:}
\PYG{n}{SPVIRSTC} \PYG{n}{GRPID}\PYG{p}{:} \PYG{l+m+mi}{00000002}
\PYG{n}{ENVID}\PYG{p}{:} \PYG{l+m+mi}{00000000}
\PYG{n}{CONNID}\PYG{p}{:} \PYG{l+m+mi}{000006}\PYG{n}{A2}
\end{sphinxVerbatim}
\begin{description}
\sphinxlineitem{Common PAGENT return codes:}\begin{itemize}
\item {} 
\sphinxAtStartPar
7 No certificate

\item {} 
\sphinxAtStartPar
8 Certificate not trusted

\item {} 
\sphinxAtStartPar
109 No certification authority certificates

\item {} 
\sphinxAtStartPar
202 Keyring does not exist

\item {} 
\sphinxAtStartPar
401 Certificate expired or not yet valid

\item {} 
\sphinxAtStartPar
402 or 412 Client and server cannot agree on cipher suite

\item {} 
\sphinxAtStartPar
416 VIRTEL does not have permission to list the keyring

\item {} 
\sphinxAtStartPar
431 Certificate is revoked

\item {} 
\sphinxAtStartPar
434 Certificate key not compatible with cipher suite

\item {} 
\sphinxAtStartPar
435 Certificate authority unknown

\item {} 
\sphinxAtStartPar
5003 Browser sent clear text (http instead of https)

\item {} 
\sphinxAtStartPar
5006 SSL failed to initialize. Check RACF SSLSETUP job.

\end{itemize}

\end{description}

\sphinxAtStartPar
For 5003, make sure your url is HTTPS and not HTTP.

\sphinxAtStartPar
For 5006, list the Virtel key ring and ensure all the relevant keys are attached. There should be a client certificate (if using client certificates), a server certificate and an associated signing certificate.

\sphinxAtStartPar
3   Virtel messages
\begin{quote}

\sphinxAtStartPar
\sphinxstylestrong{VIRHT57E LINE IS NOT SET UP FOR HTTPS}

\sphinxAtStartPar
This means that the browser has sent encrypted text (https) but that AT\sphinxhyphen{}TLS has not decrypted it before sending it to VIRTEL. The PAGENT rules haven’t correctly identified this port as a SSL jobname/port. Check the /etc/pagent.conf member. The message is a bit misleading as there is no line setup required by Virtel.

\sphinxAtStartPar
Normally AT\sphinxhyphen{}TLS is transparent to VIRTEL. AT\sphinxhyphen{}TLS performs the decryption and transforms the https request into an http request before passing it to VIRTEL. The only case where VIRTEL is AT\sphinxhyphen{}TLS aware is when the VIRTEL transaction definition specifies SECURITY=3 (TLS) and in this case VIRTEL will check that the session has been processed by AT\sphinxhyphen{}TLS and will issue an IOCTL to obtain the userid associated with the certificate.

\sphinxAtStartPar
In the normal case, you should specify HandshakeRole Server, ClientAuthType Full, and ApplicationControlled Off in the AT\sphinxhyphen{}TLS rules, as in the example in VIRT447.SAMPLIB(SSLSETUP). VIRTEL does not issue an IOCTL to turn decryption on and off, so if you specified ApplicationControlled On then you would get VIRHT57E because AT\sphinxhyphen{}TLS has not been instructed to start decryption.

\sphinxAtStartPar
If you still get an error when you have ApplicationControlled Off then we will need to see the SYSLOG (for the EZD TTLS messages), the JESMSGLG from the VIRTEL started task, and the SYSPRINT resulting from a z/OS command F VIRTEL,SNAP immediately after the error occurs. We would also like to see the exact URL which was entered at the browser, as well as the AT\sphinxhyphen{}TLS pagent.conf file.
\end{quote}


\subsection{References}
\label{\detokenize{Customization:references}}

\subsubsection{z/OS IBM References}
\label{\detokenize{Customization:z-os-ibm-references}}\begin{itemize}
\item {} 
\sphinxAtStartPar
SA22\sphinxhyphen{}7683\sphinxhyphen{}07 z/OS V1R7 Security Server: RACF Security Administrator’s Guide Chapter 21. RACF and Digital Certificates

\item {} 
\sphinxAtStartPar
SC24\sphinxhyphen{}5901\sphinxhyphen{}04 z/OS V1R6 Cryptographic Services: System SSL Programming Chapter 12. Messages and Codes

\item {} 
\sphinxAtStartPar
SC31\sphinxhyphen{}8775\sphinxhyphen{}07 z/OS V1R7 Communications Server: IP Configuration Guide Chapter 14. Policy\sphinxhyphen{}based networking Chapter 18. Application Transparent Transport Layer Security (AT\sphinxhyphen{}TLS) data protection

\item {} 
\sphinxAtStartPar
SC31\sphinxhyphen{}8776\sphinxhyphen{}08 z/OS V1R7 Communications Server: IP Configuration Reference Chapter 21. Policy Agent and policy applications

\item {} 
\sphinxAtStartPar
GC31\sphinxhyphen{}8782\sphinxhyphen{}06 z/OS V1R7 Communications Server: IP Diagnosis Guide Chapter 28. Diagnosing Application Transparent Transport Layer Security (AT\sphinxhyphen{}TLS)

\item {} 
\sphinxAtStartPar
SC31\sphinxhyphen{}8784\sphinxhyphen{}05 z/OS V1R7 Communications Server: IP Messages: Volume 2 (EZB, EZD) Chapter 10. EZD1xxxx messages

\end{itemize}


\subsubsection{Virtel References}
\label{\detokenize{Customization:virtel-references}}\begin{itemize}
\item {} 
\sphinxAtStartPar
VIRTEL Installation Guide PASSTCK parameter

\item {} 
\sphinxAtStartPar
VIRTEL Connectivity Reference Transactions \textendash{} PassTicket Parameter / Transactions \textendash{} Security Parameter

\item {} 
\sphinxAtStartPar
VIRTEL Web Access Guide Security \textendash{} Data encryption by SSL

\end{itemize}

\index{Security@\spxentry{Security}!NTLM Support@\spxentry{NTLM Support}}\ignorespaces 

\section{NTLM Support}
\label{\detokenize{Customization:ntlm-support}}\label{\detokenize{Customization:index-115}}
\sphinxAtStartPar
Windows Integrated Authentication allows a users’ Active Directory credentials to pass through their browser to a web server. The NTLM support is automatically activated whenever the Virtel transaction invoked is defined with type 2 security (NTLM). The NTLM support allows to retrieve the Windows userid but NOT the associated password. It could be therefore appropriate to complete the signon process by obtaining a password using a PASSTICKET request. Windows Integrated Authentication is enabled by default for Internet Explorer but not necessarily for Google Chrome or Mozilla Firefox. Users who use the non\sphinxhyphen{}Microsoft browsers will receive a pop\sphinxhyphen{}up box to enter their Mainframe credentials before continuing to the Virtel Web Access website. In an effort to make this process as easy as possible for end\sphinxhyphen{}users and if the Mainframe and Active Directory userid are the same it could be efficient to enable Windows Integrated Authentication for the browser. This can be done with Chrome and Firefox with a few additional steps.


\subsection{Enable Windows Integrated Authentication for Internet Explorer}
\label{\detokenize{Customization:enable-windows-integrated-authentication-for-internet-explorer}}

\subsubsection{For Internet Explorer 11}
\label{\detokenize{Customization:for-internet-explorer-11}}\begin{itemize}
\item {} 
\sphinxAtStartPar
Open Internet Explorer and select “Tools” Pull\sphinxhyphen{}down menu,

\item {} 
\sphinxAtStartPar
Select the “Internet Option” line,

\item {} 
\sphinxAtStartPar
In the opened Pop Up window, select the “Advanced” tab,

\item {} 
\sphinxAtStartPar
Scroll down to the “Security” section until you see “Enable Integrated Windows Authentication”. Select the box next to this field to enable,

\item {} 
\sphinxAtStartPar
Select the “Security” tab,

\item {} 
\sphinxAtStartPar
Select the “Local Intranet Area”,

\item {} 
\sphinxAtStartPar
Add the web server address (DNS or IP address) to the list.

\end{itemize}


\subsection{Enable Windows Integrated Authentication for Google Chrome}
\label{\detokenize{Customization:enable-windows-integrated-authentication-for-google-chrome}}
\sphinxAtStartPar
The latest version of Chrome uses existing Internet Explorer settings, but older version require additional configurations (see below). You can use three methods to enable Chrome to use Windows Integrated Authentication.Your options are the command line, editing the registry, or using ADMX templates through group policy. If you choose to use the command line or edit the registry, you could use Group Policy Preferences to distribute those changes on a broader scale. Each of these three methods achieve the same results for configuring Google Chrome for Windows Integrated Authentication. Below are the steps for the three methods:

\sphinxAtStartPar
\sphinxstylestrong{To use the command line to configure Google Chrome.}

\sphinxAtStartPar
Start Chrome with the following command:

\begin{sphinxVerbatim}[commandchars=\\\{\}]
Chrome.exe  \textendash{}auth\PYGZhy{}server\PYGZhy{}whitelist=”MyVirtel.Domain.Name”
            \textendash{}auth\PYGZhy{}negotiate\PYGZhy{}delegatewhitelist=”MyVirtel.Domain.Name”
            \textendash{}auth\PYGZhy{}schemes=”digest,ntlm,negotiate”
\end{sphinxVerbatim}


\subsubsection{To modify the registry to configure Google Chrome}
\label{\detokenize{Customization:to-modify-the-registry-to-configure-google-chrome}}
\sphinxAtStartPar
Configure the following registry settings with the corresponding values:

\sphinxAtStartPar
Registry AuthSchemes
\begin{itemize}
\item {} 
\sphinxAtStartPar
Data type: String (REG\_SZ)

\item {} 
\sphinxAtStartPar
Windows registry location: SoftwarePoliciesGoogleChromeAuthSchemes

\item {} 
\sphinxAtStartPar
Mac/Linux preference name: AuthSchemes

\item {} 
\sphinxAtStartPar
Supported on: Google Chrome (Linux, Mac, Windows) since version 9

\item {} 
\sphinxAtStartPar
Supported features: Dynamic Policy Refresh: No, Per Profile: No

\item {} 
\sphinxAtStartPar
Description: Specifies which HTTP Authentication schemes are supported by Google Chrome. Possible values are ‘basic’, ‘digest’, ‘ntlm’ and ‘negotiate’. Separate multiple values with commas. If this policy is left not set, all four schemes will be used.

\item {} 
\sphinxAtStartPar
Value: “basic,digest,ntlm,negotiate”

\end{itemize}

\sphinxAtStartPar
AuthServerWhitelist
\begin{itemize}
\item {} 
\sphinxAtStartPar
Data type: String (REG\_SZ)

\item {} 
\sphinxAtStartPar
Windows registry location: SoftwarePoliciesGoogleChromeAuthServerWhitelist

\item {} 
\sphinxAtStartPar
Mac/Linux preference name: AuthServerWhitelist

\item {} 
\sphinxAtStartPar
Supported on: Google Chrome (Linux, Mac, Windows) since version 9

\item {} 
\sphinxAtStartPar
Supported features: Dynamic Policy Refresh: No, Per Profile: No

\item {} 
\sphinxAtStartPar
Description: Specifies which servers should be whitelisted for integrated authentication. Integrated authentication is only enabled when Google Chrome receives an authentication challenge from a proxy or from a server which is in this permitted list. Separate multiple server names with commas. Wildcards (*) are allowed. If you leave this policy not set Chrome will try to detect if a server is on the Intranet and only then will it respond to IWA requests. If a server is detected as Internet then IWA requests from it will be ignored by Chrome.

\item {} 
\sphinxAtStartPar
Value: “MyVirtel.Domain.Name”

\end{itemize}

\sphinxAtStartPar
AuthNegotiateDelegateWhitelist
\begin{itemize}
\item {} 
\sphinxAtStartPar
Data type: String (REG\_SZ)

\item {} 
\sphinxAtStartPar
Windows registry location: SoftwarePoliciesGoogleChromeAuthNegotiateDelegateWhitelist

\item {} 
\sphinxAtStartPar
Mac/Linux preference name: AuthServerWhitelist

\item {} 
\sphinxAtStartPar
Supported on: Google Chrome (Linux, Mac, Windows) since version 9

\item {} 
\sphinxAtStartPar
Supported features: Dynamic Policy Refresh: No, Per Profile: No

\item {} 
\sphinxAtStartPar
Description: Servers that Google Chrome may delegate to. Separate multiple server names with commas. Wildcards (*) are allowed. If you leave this policy not set Chrome will not delegate user credentials even if a server is detected as Intranet.

\item {} 
\sphinxAtStartPar
Value: “MyVirtel.Domain.Name”

\end{itemize}

\sphinxAtStartPar
To use ADM/ADMX templates through Group Policy to configure Google Chrome:
\begin{itemize}
\item {} 
\sphinxAtStartPar
Download Zip file of ADM/ADMX templates and documentation from: \sphinxurl{http://www.chromium.org/administrators/} policy\sphinxhyphen{}templates,

\item {} 
\sphinxAtStartPar
Add the ADMX template to your central store, if you are using a central store,

\item {} 
\sphinxAtStartPar
Configure a GPO with your application server DNS host name with Kerberos Delegation Server Whitelist and Authentication Server Whitelist enabled.

\end{itemize}


\subsection{Enable Windows Integrated Authentication for Mozilla Firefox}
\label{\detokenize{Customization:enable-windows-integrated-authentication-for-mozilla-firefox}}\begin{itemize}
\item {} 
\sphinxAtStartPar
Open Firefox,

\item {} 
\sphinxAtStartPar
In the address bar type \sphinxurl{about:config}

\item {} 
\sphinxAtStartPar
In the opened Pop Up window, select the “Advanced” tab,

\item {} 
\sphinxAtStartPar
You will receive a security warning. To continue, click “I’ll be careful, I promise”,

\item {} 
\sphinxAtStartPar
You will see a list of preferences listed. Find the settings below by browsing through the list or searching for them in the search box. Once you have located each setting, update the value to the following:

\end{itemize}

\begin{sphinxVerbatim}[commandchars=\\\{\}]
\PYG{n}{Setting} \PYG{n}{Value}

    \PYG{n}{network}\PYG{o}{.}\PYG{n}{automatic}\PYG{o}{\PYGZhy{}}\PYG{n}{ntlm}\PYG{o}{\PYGZhy{}}\PYG{n}{auth}\PYG{o}{.}\PYG{n}{trusted}\PYG{o}{\PYGZhy{}}\PYG{n}{uris} \PYG{n}{MyVirtel}\PYG{o}{.}\PYG{n}{Domain}\PYG{o}{.}\PYG{n}{Name}

    \PYG{n}{network}\PYG{o}{.}\PYG{n}{automatic}\PYG{o}{\PYGZhy{}}\PYG{n}{ntlm}\PYG{o}{\PYGZhy{}}\PYG{n}{auth}\PYG{o}{.}\PYG{n}{allow}\PYG{o}{\PYGZhy{}}\PYG{n}{proxies} \PYG{k+kc}{True}

    \PYG{n}{network}\PYG{o}{.}\PYG{n}{negotiate}\PYG{o}{\PYGZhy{}}\PYG{n}{auth}\PYG{o}{.}\PYG{n}{allow}\PYG{o}{\PYGZhy{}}\PYG{n}{proxies} \PYG{k+kc}{True}
\end{sphinxVerbatim}

\index{Security@\spxentry{Security}!NTLM Support@\spxentry{NTLM Support}}\index{NTLM Support@\spxentry{NTLM Support}!Troubleshooting@\spxentry{Troubleshooting}}\ignorespaces 

\subsection{NTLM Troubleshooting}
\label{\detokenize{Customization:ntlm-troubleshooting}}\label{\detokenize{Customization:index-116}}

\subsubsection{Still receiving login prompt}
\label{\detokenize{Customization:still-receiving-login-prompt}}
\sphinxAtStartPar
If you still continue to receive logon prompt, please check your configuration. Most common error are:
\begin{itemize}
\item {} 
\sphinxAtStartPar
The accessed server is not listed in the NTLM trusted uris section (Local Intranet Area for Internet Explorer).

\item {} 
\sphinxAtStartPar
The invoked Virtel transaction is not defined with security = 2.

\end{itemize}

\index{System Management@\spxentry{System Management}}\ignorespaces 

\chapter{System Management}
\label{\detokenize{Customization:system-management}}\label{\detokenize{Customization:index-117}}
\index{System Management@\spxentry{System Management}!Message controls@\spxentry{Message controls}}\index{Message controls@\spxentry{Message controls}!Messages to SYSOUT@\spxentry{Messages to SYSOUT}}\ignorespaces 

\section{Virtel Message Controls}
\label{\detokenize{Customization:virtel-message-controls}}\label{\detokenize{Customization:index-118}}
\sphinxAtStartPar
By default, the system messages prefixed by VIR issued by VIRTEL and the messages issued from the scenarios are written both in the STC SYSOUT or the JOB as well as on the system console. If certain messages prefixed by VIR can be deleted using the parameter (or command line) SILENCE of the VIRTCT, it may sometimes be necessary to stop the
message issuance to SYSLOG or restrict their diffusion to another direction. From Virtel 4.54 (not available until 2015/Q1) the LOG feature enables the VIRTEL log to be spun off to JES2. Setting up VIRTEL to use the LOG sysout facility requires a change to the TCT definition to direct WTOs to a SYSOUT dataset. In the TCT code the following statement:

\begin{sphinxVerbatim}[commandchars=\\\{\}]
\PYG{n}{LOG}\PYG{o}{=}\PYG{p}{(}\PYG{n}{SYSOUT}\PYG{p}{,}\PYG{n}{class}\PYG{p}{,}\PYG{n}{destination}\PYG{p}{)}
\end{sphinxVerbatim}

\sphinxAtStartPar
For example, \sphinxstyleemphasis{LOG=(SYSOUT,A,EDSPRT)} directs all WTOs to a SYSOUT dataset rather than the system console log (SYSLOG). If you want WTO messages going to both the system console and a SYSOUT dataset then issue the following VIRTEL command:

\begin{sphinxVerbatim}[commandchars=\\\{\}]
\PYG{n}{F} \PYG{n}{VIRTEL}\PYG{p}{,}\PYG{n}{LOG}\PYG{o}{=}\PYG{n}{BOTH}
\end{sphinxVerbatim}

\index{System Management@\spxentry{System Management}!Message controls@\spxentry{Message controls}}\index{Message controls@\spxentry{Message controls}!Messages control options@\spxentry{Messages control options}}\ignorespaces 
\sphinxAtStartPar
The LOG feature in Virtel 4.54 is controlled by the TCT parameters and a new LOG command. The new LOG command has the following format:

\begin{sphinxVerbatim}[commandchars=\\\{\}]
\PYG{n}{F} \PYG{n}{VIRTEL}\PYG{p}{,}\PYG{n}{LOG}\PYG{o}{=}\PYG{n}{CONSOLE} \PYG{o}{|} \PYG{n}{SYSOUT} \PYG{o}{|} \PYG{n}{BOTH} \PYG{o}{|} \PYG{n}{SPIN} \PYG{n}{Default}\PYG{o}{=}\PYG{n}{CONSOLE}
\end{sphinxVerbatim}

\sphinxAtStartPar
\sphinxstylestrong{LOG}

\sphinxAtStartPar
The following parameters are available:
\begin{itemize}
\item {} 
\sphinxAtStartPar
CONSOLE: Routes all WTO message to the system console. This is the default action.

\item {} 
\sphinxAtStartPar
SYSOUT : Routes all WTO messages to a VIRTEL SYSOUT dataset which is dynamically allocated.

\item {} 
\sphinxAtStartPar
BOTH : Routes all WTO messages to both the system console and a dynamically allocated SYSOUT dataset.

\item {} 
\sphinxAtStartPar
SPIN : Forces Virtel to release the current SYSOUT dataset for further processing and allocates a new SYSOUT dataset.

\end{itemize}

\sphinxAtStartPar
Virtel will direct messages to an allocated SYSOUT data set and | or the console (default). If SYSOUT processing is being used you will see a dynamically allocated dataset in the SDSF display panel named SYSnnnnn. You can spin this dataset off using the Virtel command LOG=SPIN . VIRTEL will release the SYSOUT dataset from JES2 and allocate a new one.

\sphinxAtStartPar
Once the LOG command activated, only the following messages will be displayed at the SYSLOG during startup process :

\begin{sphinxVerbatim}[commandchars=\\\{\}]
\PYG{l+m+mf}{15.20}\PYG{l+m+mf}{.18} \PYG{n}{STC07592} \PYG{n}{VIR0096I} \PYG{n}{VIRTEL} \PYG{n}{IS} \PYG{n}{USING} \PYG{n}{VIRTCT} \PYG{l+s+s1}{\PYGZsq{}}\PYG{l+s+s1}{VIRTCTDO}\PYG{l+s+s1}{\PYGZsq{}}
\PYG{l+m+mf}{15.20}\PYG{l+m+mf}{.18} \PYG{n}{STC07592} \PYG{n}{VIR0000I} \PYG{n}{STARTING} \PYG{n}{LICENCE} \PYG{n}{SPVIRDOC}\PYG{o}{\PYGZhy{}}\PYG{n}{VIRTEL}\PYG{o}{\PYGZhy{}}\PYG{n}{A20191231} \PYG{p}{(}\PYG{l+m+mi}{2019} \PYG{o}{\PYGZhy{}} \PYG{l+m+mi}{12} \PYG{o}{\PYGZhy{}} \PYG{l+m+mi}{31}\PYG{p}{)}
\PYG{l+m+mf}{15.20}\PYG{l+m+mf}{.18} \PYG{n}{STC07592} \PYG{n}{VIR0019I} \PYG{n}{VIRTEL} \PYG{l+m+mf}{4.57} \PYG{n}{HAS} \PYG{n}{NO} \PYG{n}{PTFS} \PYG{n}{APPLIED}
\PYG{l+m+mf}{15.20}\PYG{l+m+mf}{.18} \PYG{n}{STC07592} \PYG{n}{VIR0089I} \PYG{n}{VIRTEL} \PYG{n}{RUNNING} \PYG{n}{FROM} \PYG{n}{AN} \PYG{n}{AUTHORIZED} \PYG{n}{LIBRARY}
\PYG{l+m+mf}{15.20}\PYG{l+m+mf}{.18} \PYG{n}{STC07592} \PYG{n}{VIR0860I} \PYG{n}{VIRTEL} \PYG{n}{IS} \PYG{n}{USING} \PYG{n}{RACROUTE} \PYG{n}{SECURITY}
\PYG{l+m+mf}{15.20}\PYG{l+m+mf}{.18} \PYG{n}{STC07592} \PYG{n}{VIR0861I} \PYG{n}{MIXED}\PYG{o}{\PYGZhy{}}\PYG{n}{CASE} \PYG{n}{PASSWORD} \PYG{n}{SUPPORT} \PYG{n}{IS} \PYG{n}{ACTIVE}
\PYG{l+m+mf}{15.20}\PYG{l+m+mf}{.18} \PYG{n}{STC07592} \PYG{n}{VSV0207I} \PYG{n}{VIRSV} \PYG{n}{V3R3} \PYG{n}{STARTED}
\PYG{l+m+mf}{15.20}\PYG{l+m+mf}{.19} \PYG{n}{STC07592} \PYG{n}{VIR0093I} \PYG{n}{VTAM} \PYG{n}{GENERIC} \PYG{n}{RESOURCE} \PYG{n}{NAME} \PYG{n}{IS} \PYG{n}{ABOVE}
\PYG{l+m+mf}{15.20}\PYG{l+m+mf}{.19} \PYG{n}{STC07592} \PYG{n}{VIR0024I} \PYG{n}{OPENING} \PYG{n}{FILE} \PYG{n}{VIRARBO}
\PYG{l+m+mf}{15.20}\PYG{l+m+mf}{.19} \PYG{n}{STC07592} \PYG{n}{VIR0024I} \PYG{n}{OPENING} \PYG{n}{FILE} \PYG{n}{VIRSWAP}
\PYG{l+m+mf}{15.20}\PYG{l+m+mf}{.19} \PYG{n}{STC07592} \PYG{n}{VIR0024I} \PYG{n}{OPENING} \PYG{n}{FILE} \PYG{n}{VIRHTML}
\PYG{l+m+mf}{15.20}\PYG{l+m+mf}{.20} \PYG{n}{STC07592} \PYG{n}{VIR0024I} \PYG{n}{OPENING} \PYG{n}{FILE} \PYG{n}{HTMLTRSF}
\PYG{l+m+mf}{15.20}\PYG{l+m+mf}{.20} \PYG{n}{STC07592} \PYG{n}{VIR0024I} \PYG{n}{OPENING} \PYG{n}{FILE} \PYG{n}{SAMPTRSF}
\PYG{l+m+mf}{15.20}\PYG{l+m+mf}{.20} \PYG{n}{STC07592} \PYG{n}{VIR0024I} \PYG{n}{OPENING} \PYG{n}{FILE} \PYG{n}{VIRHTML}
\PYG{l+m+mf}{15.20}\PYG{l+m+mf}{.20} \PYG{n}{STC07592} \PYG{n}{VIR0024I} \PYG{n}{OPENING} \PYG{n}{FILE} \PYG{n}{CUSTRSF}
\PYG{l+m+mf}{15.20}\PYG{l+m+mf}{.20} \PYG{n}{STC07592} \PYG{n}{VIR0024I} \PYG{n}{ATTACHING} \PYG{n}{SUBTASKS}
\PYG{l+m+mf}{15.20}\PYG{l+m+mf}{.21} \PYG{n}{STC07592} \PYG{n}{VIR0000I} \PYG{n}{THIS} \PYG{n}{COPY} \PYG{n}{OF} \PYG{n}{VIRTEL} \PYG{n}{IS} \PYG{n}{FOR} \PYG{n}{THE} \PYG{n}{EXCLUSIVE} \PYG{n}{USE} \PYG{n}{OF}\PYG{p}{:}
\PYG{l+m+mf}{15.20}\PYG{l+m+mf}{.21} \PYG{n}{STC07592} \PYG{n}{VIR0000I} \PYG{n}{SYSPERTEC} \PYG{n}{COMMUNICATION}
\PYG{l+m+mf}{15.20}\PYG{l+m+mf}{.21} \PYG{n}{STC07592} \PYG{n}{VIR0000I} \PYG{l+m+mi}{196} \PYG{n}{BUREAUX} \PYG{n}{DE} \PYG{n}{LA} \PYG{n}{COLLINE}
\PYG{l+m+mf}{15.20}\PYG{l+m+mf}{.21} \PYG{n}{STC07592} \PYG{n}{VIR0000I} \PYG{l+m+mi}{92213} \PYG{n}{SAINT} \PYG{n}{CLOUD} \PYG{n}{CEDEX}
\PYG{l+m+mf}{15.20}\PYG{l+m+mf}{.21} \PYG{n}{STC07592} \PYG{n}{VIR0000I} \PYG{n}{HTTP} \PYG{n}{Date}\PYG{p}{:} \PYG{n}{Thu}\PYG{p}{,} \PYG{l+m+mi}{10} \PYG{n}{Aug} \PYG{l+m+mi}{2017} \PYG{l+m+mi}{13}\PYG{p}{:}\PYG{l+m+mi}{20}\PYG{p}{:}\PYG{l+m+mi}{21} \PYG{n}{GMT}
\PYG{l+m+mf}{15.20}\PYG{l+m+mf}{.21} \PYG{n}{STC07592} \PYG{n}{VIR0000I} \PYG{n}{SMTP} \PYG{n}{Date}\PYG{p}{:} \PYG{n}{Thu}\PYG{p}{,} \PYG{l+m+mi}{10} \PYG{n}{Aug} \PYG{l+m+mi}{2017} \PYG{l+m+mi}{14}\PYG{p}{:}\PYG{l+m+mi}{20}\PYG{p}{:}\PYG{l+m+mi}{21} \PYG{o}{+}\PYG{l+m+mi}{0100}
\PYG{l+m+mf}{15.20}\PYG{l+m+mf}{.21} \PYG{n}{STC07592} \PYG{n}{VIR0900I} \PYG{n}{LICENCE} \PYG{n}{SPVIRDOC}\PYG{o}{\PYGZhy{}}\PYG{n}{VIRTEL}\PYG{o}{\PYGZhy{}}\PYG{n}{A20191231} \PYG{p}{(}\PYG{l+m+mi}{2019} \PYG{o}{\PYGZhy{}} \PYG{l+m+mi}{12} \PYG{o}{\PYGZhy{}} \PYG{l+m+mi}{31}\PYG{p}{)}
\PYG{l+m+mf}{15.20}\PYG{l+m+mf}{.21} \PYG{n}{STC07592} \PYG{n}{VIR0099I} \PYG{n}{SPVIRDOC} \PYG{n}{STARTED} \PYG{n}{AT} \PYG{l+m+mi}{10}\PYG{o}{/}\PYG{l+m+mi}{08}\PYG{o}{/}\PYG{l+m+mi}{17} \PYG{l+m+mi}{14}\PYG{p}{:}\PYG{l+m+mi}{20}\PYG{p}{:}\PYG{l+m+mi}{21} \PYG{p}{,} \PYG{n}{VERSION} \PYG{l+m+mf}{4.57}
\end{sphinxVerbatim}

\sphinxAtStartPar
The display of message VIR0099I requires the PTF 5530 be applied. Similarly the following messages will be displayed at the SYSLOG during shutdown process:

\begin{sphinxVerbatim}[commandchars=\\\{\}]
\PYG{l+m+mf}{15.21}\PYG{l+m+mf}{.48} \PYG{n}{STC07592} \PYG{n}{VIR0009I} \PYG{n}{SPVIRDOC} \PYG{p}{:} \PYG{n}{SHUT} \PYG{n}{DOWN} \PYG{n}{IN} \PYG{n}{PROGRESS}
\PYG{l+m+mf}{15.21}\PYG{l+m+mf}{.48} \PYG{n}{STC07592} \PYG{n}{VIR0024I} \PYG{n}{CLOSING} \PYG{n}{FILE} \PYG{n}{VIRARBO}
\PYG{l+m+mf}{15.21}\PYG{l+m+mf}{.49} \PYG{n}{STC07592} \PYG{n}{VIR0024I} \PYG{n}{CLOSING} \PYG{n}{FILE} \PYG{n}{VIRSWAP}
\PYG{l+m+mf}{15.21}\PYG{l+m+mf}{.49} \PYG{n}{STC07592} \PYG{n}{VIR0024I} \PYG{n}{CLOSING} \PYG{n}{FILE} \PYG{n}{VIRHTML}
\PYG{l+m+mf}{15.21}\PYG{l+m+mf}{.49} \PYG{n}{STC07592} \PYG{n}{VIR0024I} \PYG{n}{CLOSING} \PYG{n}{FILE} \PYG{n}{HTMLTRSF}
\PYG{l+m+mf}{15.21}\PYG{l+m+mf}{.49} \PYG{n}{STC07592} \PYG{n}{VIR0024I} \PYG{n}{CLOSING} \PYG{n}{FILE} \PYG{n}{SAMPTRSF}
\PYG{l+m+mf}{15.21}\PYG{l+m+mf}{.49} \PYG{n}{STC07592} \PYG{n}{VIR0024I} \PYG{n}{CLOSING} \PYG{n}{FILE} \PYG{n}{VIRHTML}
\PYG{l+m+mf}{15.21}\PYG{l+m+mf}{.49} \PYG{n}{STC07592} \PYG{n}{VIR0024I} \PYG{n}{CLOSING} \PYG{n}{FILE} \PYG{n}{CUSTRSF}
\PYG{l+m+mf}{15.21}\PYG{l+m+mf}{.49} \PYG{n}{STC07592} \PYG{n}{VIR0026W} \PYG{n}{HTTP}\PYG{o}{\PYGZhy{}}\PYG{n}{CLI} \PYG{n}{OPEN} \PYG{n}{SESSION} \PYG{l+m+mi}{00000000} \PYG{o}{\PYGZhy{}} \PYG{l+m+mf}{192.168}\PYG{l+m+mf}{.170}\PYG{l+m+mf}{.046}\PYG{p}{:}\PYG{l+m+mi}{41002}
\PYG{l+m+mf}{15.21}\PYG{l+m+mf}{.49} \PYG{n}{STC07592} \PYG{n}{VIRT922W} \PYG{n}{HTTP}\PYG{o}{\PYGZhy{}}\PYG{n}{CLI} \PYG{n}{SOCKET} \PYG{l+m+mi}{00000000} \PYG{n}{ENDED} \PYG{n}{FOR} \PYG{l+m+mf}{192.168}\PYG{l+m+mf}{.170}\PYG{l+m+mf}{.046}\PYG{p}{:}\PYG{l+m+mi}{41002}
\PYG{l+m+mf}{15.21}\PYG{l+m+mf}{.49} \PYG{n}{STC07592} \PYG{n}{VIR0026W} \PYG{n}{HTTP}\PYG{o}{\PYGZhy{}}\PYG{n}{W2H} \PYG{n}{OPEN} \PYG{n}{SESSION} \PYG{l+m+mi}{00000000} \PYG{o}{\PYGZhy{}} \PYG{l+m+mf}{192.168}\PYG{l+m+mf}{.170}\PYG{l+m+mf}{.046}\PYG{p}{:}\PYG{l+m+mi}{41001}
\PYG{l+m+mf}{15.21}\PYG{l+m+mf}{.49} \PYG{n}{STC07592} \PYG{n}{VIRT922W} \PYG{n}{HTTP}\PYG{o}{\PYGZhy{}}\PYG{n}{W2H} \PYG{n}{SOCKET} \PYG{l+m+mi}{00000000} \PYG{n}{ENDED} \PYG{n}{FOR} \PYG{l+m+mf}{192.168}\PYG{l+m+mf}{.170}\PYG{l+m+mf}{.046}\PYG{p}{:}\PYG{l+m+mi}{41001}
\PYG{l+m+mf}{15.21}\PYG{l+m+mf}{.49} \PYG{n}{STC07592} \PYG{n}{VIR0006I} \PYG{n}{CLOSING} \PYG{n}{ACBS}
\PYG{l+m+mf}{15.21}\PYG{l+m+mf}{.49} \PYG{n}{STC07592} \PYG{n}{VIR0006I} \PYG{n}{DETACHING} \PYG{n}{CONSOLE} \PYG{n}{SUBTASK}
\PYG{l+m+mf}{15.21}\PYG{l+m+mf}{.49} \PYG{n}{STC07592} \PYG{n}{VIR0233I} \PYG{n}{VIRTEL} \PYG{n}{SYSOUT} \PYG{n}{DATASET} \PYG{n}{HAS} \PYG{n}{BEEN} \PYG{n}{SPUN} \PYG{n}{OFF}
\PYG{l+m+mf}{15.21}\PYG{l+m+mf}{.49} \PYG{n}{STC07592} \PYG{n}{VIR0006I} \PYG{n}{DETACHING} \PYG{n}{LOADER} \PYG{n}{SUBTASK}
\PYG{l+m+mf}{15.21}\PYG{l+m+mf}{.49} \PYG{n}{STC07592} \PYG{n}{VIR0006I} \PYG{n}{DETACHING} \PYG{n}{VSAM} \PYG{n}{SUBTASK}
\PYG{l+m+mf}{15.21}\PYG{l+m+mf}{.49} \PYG{n}{STC07592} \PYG{n}{VIR0006I} \PYG{n}{DETACHING} \PYG{n}{STATISTICS} \PYG{n}{SUBTASK}
\PYG{l+m+mf}{15.21}\PYG{l+m+mf}{.49} \PYG{n}{STC07592} \PYG{n}{VIR0006I} \PYG{n}{CLOSING} \PYG{n}{VIRSV}
\PYG{l+m+mf}{15.21}\PYG{l+m+mf}{.50} \PYG{n}{STC07592} \PYG{n}{VIR0003I} \PYG{n}{SPVIRDOC} \PYG{n}{ENDED}
\end{sphinxVerbatim}

\index{System Management@\spxentry{System Management}!VIRCONF Utility@\spxentry{VIRCONF Utility}}\index{VIRCONF Utility@\spxentry{VIRCONF Utility}!Running@\spxentry{Running}}\ignorespaces 

\section{VIRCONF Usage}
\label{\detokenize{Customization:virconf-usage}}\label{\detokenize{Customization:index-120}}
\sphinxAtStartPar
VIRCONF is a batch facilty you can use to add, update, delete VIRTEL definitions stored in the ARBO file. There is also an unload facility which can be invoked either in batch or dynamically through the Virtel UNLOAD command. See below.

\sphinxAtStartPar
\sphinxstylestrong{VIRCONF does not open the ARBO for output.}

\sphinxAtStartPar
If you run VIRCONF when Virtel is active you can get IEC161I error messages when performing add, update or replace options. The reason for this error is that the VIRCONF program requires exclusive control of the ARBO file. Following the performance improvements, through the introduction of VSAM LSR, VIRCONF is no longer able to access the ARBO VSAM file when VIRTEL is running. The unload option is not affected by this serialization.


\subsection{Unloading the ARBO dataset}
\label{\detokenize{Customization:unloading-the-arbo-dataset}}
\index{System Management@\spxentry{System Management}!VIRCONF Utility@\spxentry{VIRCONF Utility}}\index{VIRCONF Utility@\spxentry{VIRCONF Utility}!Virtel Unload JCL@\spxentry{Virtel Unload JCL}}\ignorespaces 
\sphinxAtStartPar
The following JCL is an example of the VIRCONF program.

\begin{sphinxVerbatim}[commandchars=\\\{\}]
\PYG{o}{/}\PYG{o}{/}\PYG{o}{*}
\PYG{o}{/}\PYG{o}{/}\PYG{o}{*} \PYG{n}{THIS} \PYG{n}{JOB} \PYG{n}{UNLOADS} \PYG{n}{AN} \PYG{n}{ARBO} \PYG{n}{FILE}
\PYG{o}{/}\PYG{o}{/}\PYG{o}{*}
\PYG{o}{/}\PYG{o}{/} \PYG{n}{SET} \PYG{n}{LOAD}\PYG{o}{=}\PYG{n}{SP000}\PYG{o}{.}\PYG{n}{VIRTEL}\PYG{o}{.}\PYG{n}{LOADLIB}
\PYG{o}{/}\PYG{o}{/} \PYG{n}{SET} \PYG{n}{ARBO}\PYG{o}{=}\PYG{n}{SP000}\PYG{o}{.}\PYG{n}{VIRTEL}\PYG{o}{.}\PYG{n}{ARBO}
\PYG{o}{/}\PYG{o}{/}\PYG{o}{*}
\PYG{o}{/}\PYG{o}{/}\PYG{n}{DEL} \PYG{n}{EXEC} \PYG{n}{PGM}\PYG{o}{=}\PYG{n}{IEFBR14}
\PYG{o}{/}\PYG{o}{/}\PYG{n}{DDA} \PYG{n}{DD}  \PYG{n}{DSN}\PYG{o}{=}\PYG{o}{\PYGZam{}}\PYG{n}{SYSUID}\PYG{o}{.}\PYG{o}{.}\PYG{n}{VIRCONF}\PYG{o}{.}\PYG{n}{TEST}\PYG{o}{.}\PYG{n}{SYSIN}\PYG{p}{,}\PYG{n}{DISP}\PYG{o}{=}\PYG{p}{(}\PYG{n}{MOD}\PYG{p}{,}\PYG{n}{DELETE}\PYG{p}{)}\PYG{p}{,}
\PYG{o}{/}\PYG{o}{/} \PYG{n}{UNIT}\PYG{o}{=}\PYG{n}{SYSDA}\PYG{p}{,}\PYG{n}{SPACE}\PYG{o}{=}\PYG{p}{(}\PYG{n}{TRK}\PYG{p}{,}\PYG{l+m+mi}{0}\PYG{p}{)}
\PYG{o}{/}\PYG{o}{/}\PYG{o}{*}
\PYG{o}{/}\PYG{o}{/}\PYG{n}{UNLOAD}  \PYG{n}{EXEC} \PYG{n}{PGM}\PYG{o}{=}\PYG{n}{VIRCONF}\PYG{p}{,}\PYG{n}{PARM}\PYG{o}{=}\PYG{n}{UNLOAD}
\PYG{o}{/}\PYG{o}{/}\PYG{n}{STEPLIB}  \PYG{n}{DD}  \PYG{n}{DSN}\PYG{o}{=}\PYG{o}{\PYGZam{}}\PYG{n}{LOAD}\PYG{p}{,}\PYG{n}{DISP}\PYG{o}{=}\PYG{n}{SHR}
\PYG{o}{/}\PYG{o}{/}\PYG{n}{SYSPRINT} \PYG{n}{DD}  \PYG{n}{SYSOUT}\PYG{o}{=}\PYG{o}{*}
\PYG{o}{/}\PYG{o}{/}\PYG{n}{VIRARBO}  \PYG{n}{DD}  \PYG{n}{DSN}\PYG{o}{=}\PYG{o}{\PYGZam{}}\PYG{n}{ARBO}\PYG{p}{,}\PYG{n}{DISP}\PYG{o}{=}\PYG{n}{SHR}\PYG{p}{,}\PYG{n}{AMP}\PYG{o}{=}\PYG{p}{(}\PYG{l+s+s1}{\PYGZsq{}}\PYG{l+s+s1}{RMODE31=NONE}\PYG{l+s+s1}{\PYGZsq{}}\PYG{p}{)}
\PYG{o}{/}\PYG{o}{/}\PYG{n}{SYSPUNCH} \PYG{n}{DD}  \PYG{n}{DSN}\PYG{o}{=}\PYG{o}{\PYGZam{}}\PYG{n}{SYSUID}\PYG{o}{.}\PYG{o}{.}\PYG{n}{VIRCONF}\PYG{o}{.}\PYG{n}{TEST}\PYG{o}{.}\PYG{n}{SYSIN}\PYG{p}{,}\PYG{n}{DISP}\PYG{o}{=}\PYG{p}{(}\PYG{p}{,}\PYG{n}{CATLG}\PYG{p}{)}\PYG{p}{,}
\PYG{o}{/}\PYG{o}{/}             \PYG{n}{UNIT}\PYG{o}{=}\PYG{n}{SYSDA}\PYG{p}{,}\PYG{n}{VOL}\PYG{o}{=}\PYG{n}{SER}\PYG{o}{=}\PYG{n}{SPT30C}\PYG{p}{,}\PYG{n}{SPACE}\PYG{o}{=}\PYG{p}{(}\PYG{n}{CYL}\PYG{p}{,}\PYG{p}{(}\PYG{l+m+mi}{5}\PYG{p}{,}\PYG{l+m+mi}{1}\PYG{p}{)}\PYG{p}{)}\PYG{p}{,}
\PYG{o}{/}\PYG{o}{/}             \PYG{n}{DCB}\PYG{o}{=}\PYG{p}{(}\PYG{n}{RECFM}\PYG{o}{=}\PYG{n}{FB}\PYG{p}{,}\PYG{n}{LRECL}\PYG{o}{=}\PYG{l+m+mi}{80}\PYG{p}{,}\PYG{n}{BLKSIZE}\PYG{o}{=}\PYG{l+m+mi}{6080}\PYG{p}{)}
\end{sphinxVerbatim}

\index{System Management@\spxentry{System Management}!VIRCONF Utility@\spxentry{VIRCONF Utility}}\index{VIRCONF Utility@\spxentry{VIRCONF Utility}!Virtel Unload Command@\spxentry{Virtel Unload Command}}\ignorespaces 
\sphinxAtStartPar
A dynamic UNLOAD can be run through the following Virtel command:\sphinxhyphen{}

\begin{sphinxVerbatim}[commandchars=\\\{\}]
\PYG{n}{F} \PYG{n}{virtel}\PYG{p}{,}\PYG{n}{unload}
\end{sphinxVerbatim}

\sphinxAtStartPar
This command invokes the VIRCONF program which unloads the ARBO file. The following DD statement can be added to the Virtel procedure to direct the output to a dataset:

\begin{sphinxVerbatim}[commandchars=\\\{\}]
\PYG{o}{/}\PYG{o}{/}\PYG{n}{SYSPUNCH} \PYG{n}{DD} \PYG{n}{DSN}\PYG{o}{=}\PYG{n}{SP000}\PYG{o}{.}\PYG{n}{VIRTELxx}\PYG{o}{.}\PYG{n}{VIRCONF}\PYG{o}{.}\PYG{n}{SYSIN}\PYG{p}{,}\PYG{n}{DISP}\PYG{o}{=}\PYG{p}{(}\PYG{p}{,}\PYG{n}{CATLG}\PYG{p}{)}\PYG{p}{,}
\PYG{o}{/}\PYG{o}{/} \PYG{n}{UNIT}\PYG{o}{=}\PYG{n}{SYSDA}\PYG{p}{,}\PYG{n}{VOL}\PYG{o}{=}\PYG{n}{SER}\PYG{o}{=}\PYG{n}{VVVVVV}\PYG{p}{,}\PYG{n}{SPACE}\PYG{o}{=}\PYG{p}{(}\PYG{n}{TRK}\PYG{p}{,}\PYG{p}{(}\PYG{l+m+mi}{5}\PYG{p}{,}\PYG{l+m+mi}{1}\PYG{p}{)}\PYG{p}{)}\PYG{p}{,}
\PYG{o}{/}\PYG{o}{/} \PYG{n}{DCB}\PYG{o}{=}\PYG{p}{(}\PYG{n}{RECFM}\PYG{o}{=}\PYG{n}{FB}\PYG{p}{,}\PYG{n}{LRECL}\PYG{o}{=}\PYG{l+m+mi}{80}\PYG{p}{,}\PYG{n}{BLKSIZE}\PYG{o}{=}\PYG{l+m+mi}{6080}\PYG{p}{)}
\end{sphinxVerbatim}

\sphinxAtStartPar
Or alternatively, if no SYSPUNCH is found, dynamic allocation will be invoke to write the ARBO statements to JES SYSOUT=B: The following mesages will appear in the log:\sphinxhyphen{}

\begin{sphinxVerbatim}[commandchars=\\\{\}]
F VIRTEL,UNLOAD
VIR0200I UNLOAD
\PYGZdl{}HASP375 VIRTEL ESTIMATED  CARDS EXCEEDED
VIR0223I UNLOAD ENDED
\end{sphinxVerbatim}

\index{Virtel MSG10 USSTAB@\spxentry{Virtel MSG10 USSTAB}!VIR0021W@\spxentry{VIR0021W}}\ignorespaces 

\chapter{Virtel MSG10 USSTAB}
\label{\detokenize{Customization:virtel-msg10-usstab}}\label{\detokenize{Customization:index-123}}
\sphinxAtStartPar
This section discusses how to build and implement a VTAM USSMSG10 screen using the VIRTEL VIR0021W USSTAB menu program. Customers can take there existing VTAM USSMSG10 code and pace the assembled module in the VIRTEL DFHRPL load library. The VIR0021W program will interrogate the customers USSTAB module and create an equivalent VIRTEL 3270 MAP. The 3270 MAP will be passed to the module VIR0010 where it will be converted into a HTML template and served to the browser. The generated template will provide similar functionality to that of the VTAM USSMSG10, that being a presentation screen and support for USSCMD and USSPARM entries. This allows customers to maintain their USSTAB MSG10 presentation for both VTAM and VIRTEL users without modification. The customers assembled USSTAB module, normally found in USER.VTAMLIB or an equivalent library, must be made available to VIRTEL. This can be done by either copying the module to a VIRTEL DFHRPL library or concatenating the USER.VTAMLIB library into the VIRTEL started procedure.


\section{Program restrictions}
\label{\detokenize{Customization:program-restrictions}}
\sphinxAtStartPar
In most cases no changes are required but that only applies if certain caveats are met. Those being
\begin{itemize}
\item {} 
\sphinxAtStartPar
There is only one unprotected field with an IC instruction (x’13’). The maximum length of 50 characters. This area is for command entry.

\item {} 
\sphinxAtStartPar
Data areas are broken up with Start Field (x’1D’) or extended Start Field (x’29’) instructions. Each area cannot be greater than 240 characters.

\item {} 
\sphinxAtStartPar
Displacement addresses, such as AL2(((24\sphinxhyphen{}1)*80)+(80\sphinxhyphen{}1)), are not supported. Only use 3270 Buffer addresses such as x’C1C1’, representing Columun 1 Row 1.

\item {} 
\sphinxAtStartPar
The following PARM fields are supported \sphinxhyphen{} APPLID, P1 and P12.

\item {} 
\sphinxAtStartPar
The SCAN parameter must be included on the Buffer statement. For example:\sphinxhyphen{} MSG10 USSMSG MSG=10,BUFFER=(MSG10BUF,SCAN)

\end{itemize}


\section{Setting up to use the VIRTEL’s USSMSG support}
\label{\detokenize{Customization:setting-up-to-use-the-virtel-s-ussmsg-support}}

\subsection{Build a Virtel transaction}
\label{\detokenize{Customization:build-a-virtel-transaction}}
\sphinxAtStartPar
Start by defining a Virtel transaction to support the VIR0021W program and the USSMSG display.

\begin{sphinxVerbatim}[commandchars=\\\{\}]
\PYG{n}{TRANSACT} \PYG{n}{ID}\PYG{o}{=}\PYG{n}{CLI}\PYG{o}{\PYGZhy{}}\PYG{l+m+mi}{16}\PYG{n}{A}\PYG{p}{,}                                             \PYG{o}{\PYGZhy{}}
        \PYG{n}{NAME}\PYG{o}{=}\PYG{n}{VTAMUSS}\PYG{p}{,}                                           \PYG{o}{\PYGZhy{}}
        \PYG{n}{DESC}\PYG{o}{=}\PYG{l+s+s1}{\PYGZsq{}}\PYG{l+s+s1}{Logon through USSTAB}\PYG{l+s+s1}{\PYGZsq{}}\PYG{p}{,}                            \PYG{o}{\PYGZhy{}}
        \PYG{n}{APPL}\PYG{o}{=}\PYG{n}{VIR0021W}\PYG{p}{,}                                          \PYG{o}{\PYGZhy{}}
        \PYG{n}{TYPE}\PYG{o}{=}\PYG{l+m+mi}{2}\PYG{p}{,}                                                 \PYG{o}{\PYGZhy{}}
        \PYG{n}{TERMINAL}\PYG{o}{=}\PYG{n}{CLVTA}\PYG{p}{,}                                         \PYG{o}{\PYGZhy{}}
        \PYG{n}{STARTUP}\PYG{o}{=}\PYG{l+m+mi}{1}\PYG{p}{,}                                              \PYG{o}{\PYGZhy{}}
        \PYG{n}{SECURITY}\PYG{o}{=}\PYG{l+m+mi}{1}\PYG{p}{,}                                             \PYG{o}{\PYGZhy{}}
        \PYG{n}{LOGMSG}\PYG{o}{=}\PYG{l+s+s1}{\PYGZsq{}}\PYG{l+s+s1}{usstab=myusstab,errmsg=y}\PYG{l+s+s1}{\PYGZsq{}}
\end{sphinxVerbatim}

\sphinxAtStartPar
\sphinxincludegraphics{{image57}.png}

\sphinxAtStartPar
\sphinxstyleemphasis{Example of a USSMSG transaction}


\subsection{Create your USSTAB load module}
\label{\detokenize{Customization:create-your-usstab-load-module}}
\sphinxAtStartPar
Check your existing USSTAB code, in particular the USSMSG10 section, and check that it meets the caveats as outlinded above. Re\sphinxhyphen{}assemble the code and place in the DFHRPL loadlib.


\subsection{Test the USSMSG10 display}
\label{\detokenize{Customization:test-the-ussmsg10-display}}
\sphinxAtStartPar
To display the USSMSG10 buffer you will need to use a VIRTEL URL that invokes the Virtel transaction. For example, \sphinxurl{http://192.168.170.48:41002/w2h/WEB2AJAX.htm+VTAMUSS}. Here is the default example that is shipped with Virtel. The source can be found in the SAMPLIB library as member USSVIRT.

\sphinxAtStartPar
\sphinxincludegraphics{{image58}.png}

\sphinxAtStartPar
\sphinxstyleemphasis{Default USSMSG display}

\sphinxAtStartPar
You can, of course, make this transaction the default transaction for the Entry Point, so that the USSMSG display is shown whenever a user enters just the line and port. For example, \sphinxurl{http://192.168.170.48:41002}. You would specify the USSMSG transaction in the TIOA at logon field in the Entry Point definition \sphinxhyphen{} w2h/WEB2AJAX.HTM+VTAMUSS


\subsection{The errmsg= keyword}
\label{\detokenize{Customization:the-errmsg-keyword}}
\sphinxAtStartPar
The VIR0021W program supports the errmsg=Y|N keyword. If errmsg=Y is specified in the LOGMSG it instructs the VIR0021W program to use line 24 in your display as a feedback area. Meaningful messages will be reported back to the user should the VIR0021W fail to initiate a session with the target 3270 application. This area must not be utilized by your USSMSG10 definitions i.e. your should have no data defined in line 24.


\subsection{Dyanmic substitution}
\label{\detokenize{Customization:dyanmic-substitution}}
\sphinxAtStartPar
VIR0021W supports symbolic values within the USSMSG10 code. Any values defined in the IEASYM00 zOS parmlib member may be substituted in the USSMSG buffer. For example:\sphinxhyphen{}

\begin{sphinxVerbatim}[commandchars=\\\{\}]
\PYG{n}{DC} \PYG{n}{CL10}\PYG{l+s+s1}{\PYGZsq{}}\PYG{l+s+s1}{\PYGZam{}\PYGZam{}SYSNAME.}\PYG{l+s+s1}{\PYGZsq{}}
\end{sphinxVerbatim}

\sphinxAtStartPar
The \&\&SYSNAME would be replaced bu the systems SYSNAME. The template length must be sufficient to support the value being substituted. Apart from the values defined in the IEASYM00 parmlib member the following substition values can also be used:\sphinxhyphen{}

\begin{sphinxVerbatim}[commandchars=\\\{\}]
\PYG{o}{@}\PYG{o}{@}\PYG{o}{@}\PYG{o}{@}\PYG{p}{:}\PYG{o}{@}\PYG{o}{@}\PYG{o}{@}\PYG{o}{@}\PYG{p}{:}\PYG{o}{@}\PYG{o}{@}\PYG{o}{@}\PYG{o}{@}\PYG{p}{:}\PYG{o}{@}\PYG{o}{@}\PYG{o}{@}\PYG{o}{@}\PYG{p}{:}\PYG{o}{@}\PYG{o}{@}\PYG{o}{@}\PYG{o}{@}\PYG{p}{:}\PYG{o}{@}\PYG{o}{@}\PYG{o}{@}\PYG{o}{@}\PYG{p}{:}\PYG{o}{@}\PYG{o}{@}\PYG{n+nd}{@IPADDR} \PYG{n}{Callers} \PYG{n}{IPV6} \PYG{o+ow}{or} \PYG{n}{IPV4} \PYG{n}{address}
\PYG{o}{@}\PYG{o}{@}\PYG{o}{@}\PYG{o}{@}\PYG{o}{@}\PYG{o}{@}\PYG{o}{@}\PYG{o}{@}\PYG{n+nd}{@IPADDR}                         \PYG{n}{Callers} \PYG{n}{IPV4} \PYG{n}{Address}
\PYG{o}{@}\PYG{n+nd}{@PORT}                                  \PYG{n}{PORT}
\PYG{o}{@}\PYG{n+nd}{@LUNAME}                                \PYG{n}{Virtel} \PYG{n}{LU} \PYG{n}{Name}
\PYG{o}{@}\PYG{o}{@}\PYG{o}{@}\PYG{n+nd}{@DATE}                                \PYG{n}{Date}
\PYG{o}{@}\PYG{o}{@}\PYG{o}{@}\PYG{n+nd}{@TIME}                                \PYG{n}{Time}
\end{sphinxVerbatim}


\section{Problems}
\label{\detokenize{Customization:problems}}
\sphinxAtStartPar
Should your user USSATAB fail to load , or breaks one of the caveats listed above the the default USSTAB USSVIRT will be loaded. This default module can be overridden by specifying USSTAB=name in the TCT. You could provide a USSMSG10 help page directing users to a contact number, for example. If the default page is shown, check the VIRTEL log for error messages. Normally when the user USSTAB module fails an error message is written to the VIRTEL log.

\sphinxAtStartPar
Msg VIR0060W MAPFAIL WAS DETECTED ON TERMINAL TTTTTTTT

\sphinxAtStartPar
The USSTAB input area conflicts with the Virtel MAP that supports the USSTAB application processing. Check that none of the constraints listed above above have been broken. In particular the defined input area is no larger than 50 characters, that there is only one input area, and that data within protected fields does not exceed 240 characters.

\index{Virtel Tier Menu@\spxentry{Virtel Tier Menu}!cAppMenu Customization@\spxentry{cAppMenu Customization}}\ignorespaces 

\chapter{cAppMenu Customization}
\label{\detokenize{Customization:cappmenu-customization}}\label{\detokenize{Customization:index-124}}

\section{Tier Menu Design Overview}
\label{\detokenize{Customization:tier-menu-design-overview}}
\sphinxAtStartPar
The Default Application Menu can be customized to the NEW enhanced feature!  The Virtel Administrator can now choose to emulate any 3270 Session Manager; making Virtel a viable replacement for any Session Manager product.


\section{Default Virtel Application Menu design}
\label{\detokenize{Customization:default-virtel-application-menu-design}}
\sphinxAtStartPar
\sphinxincludegraphics{{image70}.png}

\sphinxAtStartPar
\sphinxstyleemphasis{Default Virtel Application Menu}


\section{The New Virtel Tier Menu Style}
\label{\detokenize{Customization:the-new-virtel-tier-menu-style}}
\sphinxAtStartPar
The application user list displayed after logon is dynamically linked to their security profile, i.e.; RACF, ACF2, or TSS authorizations. Virtel administrators can organize the application list in collapsible (\textgreater{}) expandable (+) sections. All levels are displayed in expanded form, at first logon. Users can collapse levels containing less used applications, therefore highlighting focus on frequently used applications.

\sphinxAtStartPar
\sphinxincludegraphics{{image71}.png}

\sphinxAtStartPar
\sphinxstyleemphasis{cAppMenu Application Tier Menu}


\section{Required files for Tier Menu Support}
\label{\detokenize{Customization:required-files-for-tier-menu-support}}
\sphinxAtStartPar
For the purpose of this document, the value {[}key name{]} can be replaced with your company name. In this document, we use the key name equal to “level3”.

\sphinxAtStartPar
\sphinxincludegraphics{{image72}.png}
\sphinxstyleemphasis{Example of required files using a key name of level3}

\begin{sphinxVerbatim}[commandchars=\\\{\}]
\PYG{n}{appmenu}\PYG{o}{.}\PYG{p}{[}\PYG{n}{key} \PYG{n}{name}\PYG{p}{]}\PYG{o}{.}\PYG{n}{js}       \PYG{n}{Custom} \PYG{n}{parameter} \PYG{n}{values} \PYG{k}{for} \PYG{n}{the} \PYG{n}{cAppMenu}\PYG{o}{.}\PYG{n}{htm} \PYG{n}{page}
\PYG{n}{custCSS}\PYG{o}{.}\PYG{p}{[}\PYG{n}{key} \PYG{n}{name}\PYG{p}{]}\PYG{o}{.}\PYG{n}{js}       \PYG{n}{CSS} \PYG{n}{customization} \PYG{k}{for} \PYG{n}{company} \PYG{n}{logo}
\PYG{n}{option}\PYG{o}{.}\PYG{p}{[}\PYG{n}{key} \PYG{n}{name}\PYG{p}{]}\PYG{o}{.}\PYG{n}{js}        \PYG{n}{Option} \PYG{n}{file} \PYG{n}{to} \PYG{n}{define} \PYG{n}{appmenu} \PYG{o+ow}{and} \PYG{n}{custCSS} \PYG{n}{files}
\PYG{n}{syspertec}\PYG{o}{.}\PYG{n}{png}               \PYG{n}{Company} \PYG{n}{logo}
\end{sphinxVerbatim}


\section{Customizing the appmenu.{[}key name{]}.js JavaScript}
\label{\detokenize{Customization:customizing-the-appmenu-key-name-js-javascript}}

\subsection{Key Values Criteria Types}
\label{\detokenize{Customization:key-values-criteria-types}}
\sphinxAtStartPar
The Key values for the Tier Level display are triggered by any of the following criteria, as documented in the appmenu.client.js script:
\begin{itemize}
\item {} 
\sphinxAtStartPar
status;

\item {} 
\sphinxAtStartPar
tran;

\item {} 
\sphinxAtStartPar
application;

\item {} 
\sphinxAtStartPar
description

\end{itemize}

\sphinxAtStartPar
The application contains 4 columes of data :\sphinxhyphen{}

\begin{sphinxVerbatim}[commandchars=\\\{\}]
\PYG{l+m+mi}{1}   \PYG{n}{Key}\PYG{o}{=}\PYG{l+s+s2}{\PYGZdq{}}\PYG{l+s+s2}{status}\PYG{l+s+s2}{\PYGZdq{}}
\PYG{l+m+mi}{2}   \PYG{n}{Key}\PYG{o}{=}\PYG{l+s+s2}{\PYGZdq{}}\PYG{l+s+s2}{tran}\PYG{l+s+s2}{\PYGZdq{}}
\PYG{l+m+mi}{3}   \PYG{n}{Key}\PYG{o}{=}\PYG{l+s+s2}{\PYGZdq{}}\PYG{l+s+s2}{application}\PYG{l+s+s2}{\PYGZdq{}}
\PYG{l+m+mi}{4}   \PYG{n}{Key}\PYG{o}{=}\PYG{l+s+s2}{\PYGZdq{}}\PYG{l+s+s2}{description}\PYG{l+s+s2}{\PYGZdq{}}
\end{sphinxVerbatim}

\sphinxAtStartPar
Each key can be taken as a criteria for a level selection


\subsection{Virtel cAppMenu Tier Menu description}
\label{\detokenize{Customization:virtel-cappmenu-tier-menu-description}}
\sphinxAtStartPar
\sphinxincludegraphics{{image73}.png}


\subsubsection{Modify the \sphinxstyleemphasis{Main} and \sphinxstyleemphasis{Sub} titles}
\label{\detokenize{Customization:modify-the-main-and-sub-titles}}
\begin{sphinxVerbatim}[commandchars=\\\{\}]
\PYG{n}{var} \PYG{n}{cAppMenuOptions} \PYG{o}{=} \PYG{p}{\PYGZob{}}
\PYG{l+s+s2}{\PYGZdq{}}\PYG{l+s+s2}{close\PYGZus{}VWA\PYGZus{}when\PYGZus{}disconnects}\PYG{l+s+s2}{\PYGZdq{}}\PYG{p}{:}\PYG{n}{true}
\PYG{p}{\PYGZcb{}}

\PYG{n}{cAppMenuOptions}\PYG{p}{[}\PYG{l+s+s2}{\PYGZdq{}}\PYG{l+s+s2}{texts}\PYG{l+s+s2}{\PYGZdq{}}\PYG{p}{]} \PYG{o}{=} \PYG{p}{\PYGZob{}}
        \PYG{l+s+s2}{\PYGZdq{}}\PYG{l+s+s2}{main\PYGZhy{}title}\PYG{l+s+s2}{\PYGZdq{}}\PYG{p}{:}\PYG{l+s+s2}{\PYGZdq{}}\PYG{l+s+s2}{Virtel Application Tier Menu}\PYG{l+s+s2}{\PYGZdq{}}\PYG{p}{,}
        \PYG{l+s+s2}{\PYGZdq{}}\PYG{l+s+s2}{sub\PYGZhy{}title}\PYG{l+s+s2}{\PYGZdq{}}\PYG{p}{:}\PYG{l+s+s2}{\PYGZdq{}}\PYG{l+s+s2}{Syspertec Virtel V4.61 Application Menu}\PYG{l+s+s2}{\PYGZdq{}}\PYG{p}{,}
        \PYG{p}{\PYGZcb{}}\PYG{p}{;}
\end{sphinxVerbatim}


\subsubsection{Setup Level 0, first level of the Tier menu \sphinxhyphen{} \sphinxstylestrong{CICS Production Regions}}
\label{\detokenize{Customization:setup-level-0-first-level-of-the-tier-menu-cics-production-regions}}\begin{quote}

\sphinxAtStartPar
Level 0 is setup as CICS Productions Regions. Using KEY \sphinxstyleemphasis{tran} so all defined transactions with an External name of cicsp anywhere in the transaction name; the /i means any case.
\end{quote}

\sphinxAtStartPar
\sphinxincludegraphics{{image74}.png}

\begin{sphinxVerbatim}[commandchars=\\\{\}]
cAppMenuOptions[\PYGZdq{}levels\PYGZdq{}] = []

// First group \PYGZhy{} CICS Production \PYGZhy{} criteria: the VIRTEL transaction name that contains “CICSP\PYGZdq{}
cAppMenuOptions[\PYGZdq{}levels\PYGZdq{}][0] =  \PYGZob{}
        \PYGZdq{}title\PYGZdq{}:\PYGZdq{}CICS Production Regions\PYGZdq{},
        \PYGZdq{}criteria\PYGZdq{} : \PYGZdq{}tran\PYGZdq{},
        \PYGZdq{}regexp\PYGZdq{} : /cicsp/i
            \PYGZcb{};
\end{sphinxVerbatim}


\subsubsection{Setup Level 1 of the Tier menu \sphinxhyphen{} \sphinxstylestrong{CICS Development Regions}}
\label{\detokenize{Customization:setup-level-1-of-the-tier-menu-cics-development-regions}}
\sphinxAtStartPar
Level  1 defined as CICS Development Regions.  Using KEY \sphinxstyleemphasis{tran} so all defined transactions with an External name of cicsd anywhere in the transaction name, the /i means any case

\sphinxAtStartPar
\sphinxincludegraphics{{image75}.png}

\begin{sphinxVerbatim}[commandchars=\\\{\}]
\PYG{o}{/}\PYG{o}{/} \PYG{n}{Second} \PYG{n}{group} \PYG{o}{\PYGZhy{}} \PYG{n}{CICS} \PYG{n}{development}  \PYG{o}{\PYGZhy{}} \PYG{n}{criteria} \PYG{p}{:} \PYG{n}{the} \PYG{n}{VIRTEL} \PYG{n}{transaction} \PYG{n}{name} \PYG{n}{starts} \PYG{k}{with} \PYG{l+s+s2}{\PYGZdq{}}\PYG{l+s+s2}{CICSD}\PYG{l+s+s2}{\PYGZdq{}}
\PYG{n}{cAppMenuOptions}\PYG{p}{[}\PYG{l+s+s2}{\PYGZdq{}}\PYG{l+s+s2}{levels}\PYG{l+s+s2}{\PYGZdq{}}\PYG{p}{]}\PYG{p}{[}\PYG{l+m+mi}{1}\PYG{p}{]} \PYG{o}{=}  \PYG{p}{\PYGZob{}}
        \PYG{l+s+s2}{\PYGZdq{}}\PYG{l+s+s2}{title}\PYG{l+s+s2}{\PYGZdq{}}\PYG{p}{:}\PYG{l+s+s2}{\PYGZdq{}}\PYG{l+s+s2}{CICS Development Regions}\PYG{l+s+s2}{\PYGZdq{}}\PYG{p}{,}
        \PYG{l+s+s2}{\PYGZdq{}}\PYG{l+s+s2}{criteria}\PYG{l+s+s2}{\PYGZdq{}} \PYG{p}{:} \PYG{l+s+s2}{\PYGZdq{}}\PYG{l+s+s2}{tran}\PYG{l+s+s2}{\PYGZdq{}}\PYG{p}{,}
        \PYG{l+s+s2}{\PYGZdq{}}\PYG{l+s+s2}{regexp}\PYG{l+s+s2}{\PYGZdq{}} \PYG{p}{:} \PYG{o}{/}\PYG{n}{cicsd}\PYG{o}{/}\PYG{n}{i}
        \PYG{p}{\PYGZcb{}}\PYG{p}{;}
\end{sphinxVerbatim}

\begin{sphinxadmonition}{note}{Note:}
\sphinxAtStartPar
Repeat the above code for each required level. See the sample appmenu.sample.js at the end of this section.
\end{sphinxadmonition}


\subsubsection{Setup Final Level as Other Applications if Required}
\label{\detokenize{Customization:setup-final-level-as-other-applications-if-required}}
\sphinxAtStartPar
The final Level, level 6,  would be a catch all, called Other Applications.  In this Tier any application that did not match the previous criteria will be displayed here. This level can also be used simply as a list of Generic Applications, or merely as a temporary level to show remaining applications that need to be added to a previously defined level. Note that there is no “criteria” or “regexp” properties.

\sphinxAtStartPar
\sphinxincludegraphics{{image76}.png}

\begin{sphinxVerbatim}[commandchars=\\\{\}]
\PYG{o}{/}\PYG{o}{/} \PYG{n}{Other} \PYG{n}{Applications}  \PYG{o}{\PYGZhy{}} \PYG{n}{criteria} \PYG{p}{:} \PYG{n}{the} \PYG{n}{VIRTEL} \PYG{n}{transaction} \PYG{n}{names} \PYG{o+ow}{not} \PYG{o+ow}{in} \PYG{n+nb}{any} \PYG{n}{previous} \PYG{n}{levels}\PYG{l+s+s2}{\PYGZdq{}}
\PYG{n}{cAppMenuOptions}\PYG{p}{[}\PYG{l+s+s2}{\PYGZdq{}}\PYG{l+s+s2}{levels}\PYG{l+s+s2}{\PYGZdq{}}\PYG{p}{]}\PYG{p}{[}\PYG{l+m+mi}{6}\PYG{p}{]} \PYG{o}{=}  \PYG{p}{\PYGZob{}}
        \PYG{l+s+s2}{\PYGZdq{}}\PYG{l+s+s2}{title}\PYG{l+s+s2}{\PYGZdq{}}\PYG{p}{:}\PYG{l+s+s2}{\PYGZdq{}}\PYG{l+s+s2}{Other Sessions}\PYG{l+s+s2}{\PYGZdq{}}

        \PYG{p}{\PYGZcb{}}\PYG{p}{;}
\end{sphinxVerbatim}


\section{Customizing the CSS custCSS.{[}key name{]}.css style sheet}
\label{\detokenize{Customization:customizing-the-css-custcss-key-name-css-style-sheet}}
\sphinxAtStartPar
Add company logo to the Virtel cAppMenu Tier Menu Syspertec group

\sphinxAtStartPar
\sphinxincludegraphics{{image77}.png}

\sphinxAtStartPar
Create a Cascading Style sheet using the following code with a .jpeg or .png file containing the company logo. Use the Chrome debugger tool to help determine the proper logo size. This example also adds the Company Logo to the Toolbar.

\begin{sphinxVerbatim}[commandchars=\\\{\}]
\PYG{o}{*}
\PYG{o}{*} \PYG{n}{VIRTEL} \PYG{n}{Web} \PYG{n}{Access} \PYG{n}{style} \PYG{n}{sheet} \PYG{n}{customisation} \PYG{k}{for} \PYG{n}{company} \PYG{n}{logo}
\PYG{o}{*} \PYG{p}{(}\PYG{n}{c}\PYG{p}{)}\PYG{n}{Copyright} \PYG{n}{SysperTec} \PYG{n}{Communication} \PYG{l+m+mi}{2012} \PYG{n}{All} \PYG{n}{Rights} \PYG{n}{Reserved}
\PYG{o}{*}
\PYG{o}{*}\PYG{o}{/}
\PYG{n}{div}\PYG{o}{.}\PYG{n}{App}\PYG{o}{\PYGZhy{}}\PYG{n}{logo} \PYG{p}{\PYGZob{}}
\PYG{n}{width}\PYG{p}{:} \PYG{l+m+mi}{326}\PYG{n}{px}\PYG{p}{;}
\PYG{n}{height}\PYG{p}{:} \PYG{l+m+mi}{57}\PYG{n}{px}\PYG{p}{;}
\PYG{n}{background}\PYG{o}{\PYGZhy{}}\PYG{n}{image}\PYG{p}{:} \PYG{n}{url}\PYG{p}{(}\PYG{l+s+s2}{\PYGZdq{}}\PYG{l+s+s2}{syspertec.png}\PYG{l+s+s2}{\PYGZdq{}}\PYG{p}{)}\PYG{p}{;}
\PYG{n}{background}\PYG{o}{\PYGZhy{}}\PYG{n}{size}\PYG{p}{:} \PYG{l+m+mi}{200}\PYG{n}{px} \PYG{l+m+mi}{50}\PYG{n}{px}\PYG{p}{;}
\PYG{n}{height}\PYG{p}{:}\PYG{l+m+mi}{50}\PYG{n}{px}\PYG{p}{;}
\PYG{n}{width}\PYG{p}{:}\PYG{l+m+mi}{200}\PYG{n}{px}\PYG{p}{;}
\PYG{n}{background}\PYG{o}{\PYGZhy{}}\PYG{n}{size}\PYG{p}{:} \PYG{l+m+mi}{100}\PYG{o}{\PYGZpc{}}\PYG{p}{;}
\PYG{p}{\PYGZcb{}}
\end{sphinxVerbatim}


\section{Create an option.{[}key name{]}.js file}
\label{\detokenize{Customization:create-an-option-key-name-js-file}}
\sphinxAtStartPar
Modify the following file, change {[}key name{]} to your company name.  This value {[}key name{]}  will be used in step “Update the Required Virtel Transactions”

\begin{sphinxVerbatim}[commandchars=\\\{\}]
// \PYGZdl{}Id: option.level3.js 4809 2018\PYGZhy{}10\PYGZhy{}10 14:38:13Z riou \PYGZdl{}
var oCustom=\PYGZob{}
    \PYGZdq{}pathToJsCustom\PYGZdq{}:\PYGZdq{}../../option/appmenu.[key name].js\PYGZdq{},
    \PYGZdq{}pathToCssCustom\PYGZdq{} : \PYGZdq{}../option/custCSS.[key name].css\PYGZdq{}
    \PYGZcb{}
\end{sphinxVerbatim}


\section{Create a Company Logo file}
\label{\detokenize{Customization:create-a-company-logo-file}}
\sphinxAtStartPar
Download a company logo that will look best on the cAppMenu application Menu, save the image as a .JPEG or .PNG.

\sphinxAtStartPar
\sphinxincludegraphics{{image78}.png}


\section{Update the required Virtel Transactions}
\label{\detokenize{Customization:update-the-required-virtel-transactions}}

\subsection{CLI\sphinxhyphen{}90}
\label{\detokenize{Customization:cli-90}}
\sphinxAtStartPar
Update the CLI\sphinxhyphen{}90 Applist Transaction to invoke the customization by adding an option field with your {[}key name{]}, for example level3. In the example below the value CLIENT has been used as the option.

\sphinxAtStartPar
\sphinxincludegraphics{{image79}.png}


\subsection{CLI\sphinxhyphen{}00}
\label{\detokenize{Customization:cli-00}}
\sphinxAtStartPar
Modify the default Entry Point transaction CLI\sphinxhyphen{}00 and update the TIOA at logon fiedl to call cAppMenu.htm

\sphinxAtStartPar
\sphinxincludegraphics{{image80}.png}


\section{Update the CLI\sphinxhyphen{}DIR directoy}
\label{\detokenize{Customization:update-the-cli-dir-directoy}}
\sphinxAtStartPar
Using the Virtel Drag and Drop feature upload the customized files to the CLI\sphinxhyphen{}DIR.

\sphinxAtStartPar
\sphinxincludegraphics{{image81}.png}

\sphinxAtStartPar
From the Administration Panel, select the customized files and Drop \& Drag them into the CLI\sphinxhyphen{}DIR.

\sphinxAtStartPar
\sphinxincludegraphics{{image82}.png}

\begin{sphinxadmonition}{note}{Note:}
\sphinxAtStartPar
Refresh the browser cache after uploading and reconnect to the Application Menu to see the results.
\end{sphinxadmonition}


\section{Example of appmenu.{[}keyname{]}.js}
\label{\detokenize{Customization:example-of-appmenu-keyname-js}}
\begin{sphinxVerbatim}[commandchars=\\\{\}]
\PYG{n}{var} \PYG{n}{cAppMenuOptions} \PYG{o}{=} \PYG{p}{\PYGZob{}}
\PYG{l+s+s2}{\PYGZdq{}}\PYG{l+s+s2}{close\PYGZus{}VWA\PYGZus{}when\PYGZus{}disconnects}\PYG{l+s+s2}{\PYGZdq{}}\PYG{p}{:}\PYG{n}{true}
\PYG{p}{\PYGZcb{}}

\PYG{o}{/}\PYG{o}{/} \PYG{n}{Titles}
\PYG{n}{cAppMenuOptions}\PYG{p}{[}\PYG{l+s+s2}{\PYGZdq{}}\PYG{l+s+s2}{texts}\PYG{l+s+s2}{\PYGZdq{}}\PYG{p}{]}\PYG{o}{=} \PYG{p}{\PYGZob{}}
\PYG{l+s+s2}{\PYGZdq{}}\PYG{l+s+s2}{main\PYGZhy{}title}\PYG{l+s+s2}{\PYGZdq{}}\PYG{p}{:}\PYG{l+s+s2}{\PYGZdq{}}\PYG{l+s+s2}{Virtel Sample Application Tier Menu}\PYG{l+s+s2}{\PYGZdq{}}\PYG{p}{,}
\PYG{l+s+s2}{\PYGZdq{}}\PYG{l+s+s2}{sub\PYGZhy{}title}\PYG{l+s+s2}{\PYGZdq{}}\PYG{p}{:} \PYG{l+s+s2}{\PYGZdq{}}\PYG{l+s+s2}{Syspertec Virtel V4.61 Application Menu}\PYG{l+s+s2}{\PYGZdq{}}\PYG{p}{,}
\PYG{p}{\PYGZcb{}}\PYG{p}{;}


\PYG{n}{cAppMenuOptions}\PYG{p}{[}\PYG{l+s+s2}{\PYGZdq{}}\PYG{l+s+s2}{levels}\PYG{l+s+s2}{\PYGZdq{}}\PYG{p}{]} \PYG{o}{=} \PYG{p}{[}\PYG{p}{]}\PYG{p}{;}

\PYG{o}{/}\PYG{o}{/} \PYG{n}{First} \PYG{n}{Group} \PYG{p}{(}\PYG{n}{Level} \PYG{l+m+mi}{0}\PYG{p}{)}\PYG{o}{.} \PYG{n}{Any} \PYG{n}{transaction} \PYG{n}{name} \PYG{k}{with} \PYG{n}{cicsp} \PYG{n}{becomes} \PYG{n}{part} \PYG{n}{of} \PYG{n}{this} \PYG{n}{level}\PYG{o}{.}
\PYG{o}{/}\PYG{o}{/} \PYG{n}{Regular} \PYG{n}{expression} \PYG{o+ow}{is} \PYG{n}{used} \PYG{n}{to} \PYG{n+nb}{filter} \PYG{n}{out} \PYG{n}{qualifying} \PYG{n}{applications}
\PYG{n}{cAppMenuOptions}\PYG{p}{[}\PYG{l+s+s2}{\PYGZdq{}}\PYG{l+s+s2}{levels}\PYG{l+s+s2}{\PYGZdq{}}\PYG{p}{]}\PYG{p}{[}\PYG{l+m+mi}{0}\PYG{p}{]} \PYG{o}{=} \PYG{p}{\PYGZob{}}
\PYG{l+s+s2}{\PYGZdq{}}\PYG{l+s+s2}{title}\PYG{l+s+s2}{\PYGZdq{}}\PYG{p}{:}\PYG{l+s+s2}{\PYGZdq{}}\PYG{l+s+s2}{CICS Production Regions}\PYG{l+s+s2}{\PYGZdq{}}\PYG{p}{,}
\PYG{l+s+s2}{\PYGZdq{}}\PYG{l+s+s2}{criteria}\PYG{l+s+s2}{\PYGZdq{}}\PYG{p}{:}\PYG{l+s+s2}{\PYGZdq{}}\PYG{l+s+s2}{tran}\PYG{l+s+s2}{\PYGZdq{}}\PYG{p}{,}
\PYG{l+s+s2}{\PYGZdq{}}\PYG{l+s+s2}{regexp}\PYG{l+s+s2}{\PYGZdq{}}\PYG{p}{:} \PYG{o}{/}\PYG{n}{cicsp}\PYG{o}{/}\PYG{n}{i}
\PYG{p}{\PYGZcb{}}\PYG{p}{;}

\PYG{o}{/}\PYG{o}{/} \PYG{n}{Second} \PYG{n}{Group} \PYG{p}{(}\PYG{n}{Level} \PYG{l+m+mi}{1}\PYG{p}{)}\PYG{o}{.} \PYG{n}{Any} \PYG{n}{transaction} \PYG{n}{name} \PYG{k}{with} \PYG{n}{cicsd} \PYG{n}{becomes} \PYG{n}{part} \PYG{n}{of} \PYG{n}{this} \PYG{n}{level}\PYG{o}{.}
\PYG{o}{/}\PYG{o}{/} \PYG{n}{Regular} \PYG{n}{expression} \PYG{o+ow}{is} \PYG{n}{used} \PYG{n}{to} \PYG{n+nb}{filter} \PYG{n}{out} \PYG{n}{qualifying} \PYG{n}{applications}
\PYG{n}{cAppMenuOptions}\PYG{p}{[}\PYG{l+s+s2}{\PYGZdq{}}\PYG{l+s+s2}{levels}\PYG{l+s+s2}{\PYGZdq{}}\PYG{p}{]}\PYG{p}{[}\PYG{l+m+mi}{1}\PYG{p}{]} \PYG{o}{=} \PYG{p}{\PYGZob{}}
\PYG{l+s+s2}{\PYGZdq{}}\PYG{l+s+s2}{title}\PYG{l+s+s2}{\PYGZdq{}}\PYG{p}{:}\PYG{l+s+s2}{\PYGZdq{}}\PYG{l+s+s2}{CICS Development Regions}\PYG{l+s+s2}{\PYGZdq{}}\PYG{p}{,}
\PYG{l+s+s2}{\PYGZdq{}}\PYG{l+s+s2}{criteria}\PYG{l+s+s2}{\PYGZdq{}}\PYG{p}{:}\PYG{l+s+s2}{\PYGZdq{}}\PYG{l+s+s2}{tran}\PYG{l+s+s2}{\PYGZdq{}}\PYG{p}{,}
\PYG{l+s+s2}{\PYGZdq{}}\PYG{l+s+s2}{regexp}\PYG{l+s+s2}{\PYGZdq{}}\PYG{p}{:} \PYG{o}{/}\PYG{n}{cicsd}\PYG{o}{/}\PYG{n}{i}
\PYG{p}{\PYGZcb{}}\PYG{p}{;}

\PYG{o}{/}\PYG{o}{/} \PYG{n}{Third} \PYG{n}{Group} \PYG{p}{(}\PYG{n}{Level} \PYG{l+m+mi}{2}\PYG{p}{)}\PYG{o}{.} \PYG{n}{Any} \PYG{n}{transaction} \PYG{n}{name} \PYG{k}{with} \PYG{n}{cicst} \PYG{n}{becomes} \PYG{n}{part} \PYG{n}{of} \PYG{n}{this} \PYG{n}{level}\PYG{o}{.}
\PYG{o}{/}\PYG{o}{/} \PYG{n}{Regular} \PYG{n}{expression} \PYG{o+ow}{is} \PYG{n}{used} \PYG{n}{to} \PYG{n+nb}{filter} \PYG{n}{out} \PYG{n}{qualifying} \PYG{n}{applications}
\PYG{n}{cAppMenuOptions}\PYG{p}{[}\PYG{l+s+s2}{\PYGZdq{}}\PYG{l+s+s2}{levels}\PYG{l+s+s2}{\PYGZdq{}}\PYG{p}{]}\PYG{p}{[}\PYG{l+m+mi}{2}\PYG{p}{]} \PYG{o}{=} \PYG{p}{\PYGZob{}}
\PYG{l+s+s2}{\PYGZdq{}}\PYG{l+s+s2}{title}\PYG{l+s+s2}{\PYGZdq{}}\PYG{p}{:}\PYG{l+s+s2}{\PYGZdq{}}\PYG{l+s+s2}{CICS Test Regions}\PYG{l+s+s2}{\PYGZdq{}}\PYG{p}{,}
\PYG{l+s+s2}{\PYGZdq{}}\PYG{l+s+s2}{criteria}\PYG{l+s+s2}{\PYGZdq{}}\PYG{p}{:}\PYG{l+s+s2}{\PYGZdq{}}\PYG{l+s+s2}{tran}\PYG{l+s+s2}{\PYGZdq{}}\PYG{p}{,}
\PYG{l+s+s2}{\PYGZdq{}}\PYG{l+s+s2}{regexp}\PYG{l+s+s2}{\PYGZdq{}}\PYG{p}{:} \PYG{o}{/}\PYG{n}{cicst}\PYG{o}{/}\PYG{n}{i}
\PYG{p}{\PYGZcb{}}\PYG{p}{;}

\PYG{o}{/}\PYG{o}{/} \PYG{n}{Fourth} \PYG{n}{Group} \PYG{p}{(}\PYG{n}{Level} \PYG{l+m+mi}{3}\PYG{p}{)}\PYG{o}{.} \PYG{n}{Any} \PYG{n}{transaction} \PYG{n}{name} \PYG{k}{with} \PYG{n}{tsop} \PYG{n}{becomes} \PYG{n}{part} \PYG{n}{of} \PYG{n}{this} \PYG{n}{level}\PYG{o}{.}
\PYG{o}{/}\PYG{o}{/} \PYG{n}{Regular} \PYG{n}{expression} \PYG{o+ow}{is} \PYG{n}{used} \PYG{n}{to} \PYG{n+nb}{filter} \PYG{n}{out} \PYG{n}{qualifying} \PYG{n}{applications}
\PYG{n}{cAppMenuOptions}\PYG{p}{[}\PYG{l+s+s2}{\PYGZdq{}}\PYG{l+s+s2}{levels}\PYG{l+s+s2}{\PYGZdq{}}\PYG{p}{]}\PYG{p}{[}\PYG{l+m+mi}{3}\PYG{p}{]} \PYG{o}{=} \PYG{p}{\PYGZob{}}
\PYG{l+s+s2}{\PYGZdq{}}\PYG{l+s+s2}{title}\PYG{l+s+s2}{\PYGZdq{}}\PYG{p}{:}\PYG{l+s+s2}{\PYGZdq{}}\PYG{l+s+s2}{TSO Production Systems}\PYG{l+s+s2}{\PYGZdq{}}\PYG{p}{,}
\PYG{l+s+s2}{\PYGZdq{}}\PYG{l+s+s2}{criteria}\PYG{l+s+s2}{\PYGZdq{}}\PYG{p}{:}\PYG{l+s+s2}{\PYGZdq{}}\PYG{l+s+s2}{tran}\PYG{l+s+s2}{\PYGZdq{}}\PYG{p}{,}
\PYG{l+s+s2}{\PYGZdq{}}\PYG{l+s+s2}{regexp}\PYG{l+s+s2}{\PYGZdq{}}\PYG{p}{:} \PYG{o}{/}\PYG{n}{tsop}\PYG{o}{/}\PYG{n}{i}
\PYG{p}{\PYGZcb{}}\PYG{p}{;}

\PYG{o}{/}\PYG{o}{/} \PYG{n}{Fifth} \PYG{n}{Group} \PYG{p}{(}\PYG{n}{Level} \PYG{l+m+mi}{4}\PYG{p}{)}\PYG{o}{.} \PYG{n}{Any} \PYG{n}{transaction} \PYG{n}{name} \PYG{k}{with} \PYG{n}{tsod} \PYG{n}{becomes} \PYG{n}{part} \PYG{n}{of} \PYG{n}{this} \PYG{n}{level}\PYG{o}{.}
\PYG{o}{/}\PYG{o}{/} \PYG{n}{Regular} \PYG{n}{expression} \PYG{o+ow}{is} \PYG{n}{used} \PYG{n}{to} \PYG{n+nb}{filter} \PYG{n}{out} \PYG{n}{qualifying} \PYG{n}{applications}
\PYG{n}{cAppMenuOptions}\PYG{p}{[}\PYG{l+s+s2}{\PYGZdq{}}\PYG{l+s+s2}{levels}\PYG{l+s+s2}{\PYGZdq{}}\PYG{p}{]}\PYG{p}{[}\PYG{l+m+mi}{4}\PYG{p}{]} \PYG{o}{=} \PYG{p}{\PYGZob{}}
\PYG{l+s+s2}{\PYGZdq{}}\PYG{l+s+s2}{title}\PYG{l+s+s2}{\PYGZdq{}}\PYG{p}{:}\PYG{l+s+s2}{\PYGZdq{}}\PYG{l+s+s2}{TSO Development Systems}\PYG{l+s+s2}{\PYGZdq{}}\PYG{p}{,}
\PYG{l+s+s2}{\PYGZdq{}}\PYG{l+s+s2}{criteria}\PYG{l+s+s2}{\PYGZdq{}}\PYG{p}{:}\PYG{l+s+s2}{\PYGZdq{}}\PYG{l+s+s2}{tran}\PYG{l+s+s2}{\PYGZdq{}}\PYG{p}{,}
\PYG{l+s+s2}{\PYGZdq{}}\PYG{l+s+s2}{regexp}\PYG{l+s+s2}{\PYGZdq{}}\PYG{p}{:} \PYG{o}{/}\PYG{n}{tsod}\PYG{o}{/}\PYG{n}{i}
\PYG{p}{\PYGZcb{}}\PYG{p}{;}

\PYG{o}{/}\PYG{o}{/} \PYG{n}{Sixth} \PYG{n}{Group} \PYG{p}{(}\PYG{n}{Level} \PYG{l+m+mi}{5}\PYG{p}{)}\PYG{o}{.} \PYG{n}{Any} \PYG{n}{transaction} \PYG{n}{name} \PYG{k}{with} \PYG{n}{tsod} \PYG{n}{becomes} \PYG{n}{part} \PYG{n}{of} \PYG{n}{this} \PYG{n}{level}\PYG{o}{.}
\PYG{o}{/}\PYG{o}{/} \PYG{n}{Regular} \PYG{n}{expression} \PYG{o+ow}{is} \PYG{n}{used} \PYG{n}{to} \PYG{n+nb}{filter} \PYG{n}{out} \PYG{n}{qualifying} \PYG{n}{applications}
\PYG{n}{cAppMenuOptions}\PYG{p}{[}\PYG{l+s+s2}{\PYGZdq{}}\PYG{l+s+s2}{levels}\PYG{l+s+s2}{\PYGZdq{}}\PYG{p}{]}\PYG{p}{[}\PYG{l+m+mi}{5}\PYG{p}{]} \PYG{o}{=} \PYG{p}{\PYGZob{}}
\PYG{l+s+s2}{\PYGZdq{}}\PYG{l+s+s2}{title}\PYG{l+s+s2}{\PYGZdq{}}\PYG{p}{:}\PYG{l+s+s2}{\PYGZdq{}}\PYG{l+s+s2}{VTAM USSTAB Sessions}\PYG{l+s+s2}{\PYGZdq{}}\PYG{p}{,}
\PYG{l+s+s2}{\PYGZdq{}}\PYG{l+s+s2}{criteria}\PYG{l+s+s2}{\PYGZdq{}}\PYG{p}{:}\PYG{l+s+s2}{\PYGZdq{}}\PYG{l+s+s2}{tran}\PYG{l+s+s2}{\PYGZdq{}}\PYG{p}{,}
\PYG{l+s+s2}{\PYGZdq{}}\PYG{l+s+s2}{regexp}\PYG{l+s+s2}{\PYGZdq{}}\PYG{p}{:} \PYG{o}{/}\PYG{n}{vtam}\PYG{o}{/}\PYG{n}{i}
\PYG{p}{\PYGZcb{}}\PYG{p}{;}

\PYG{o}{/}\PYG{o}{/} \PYG{n}{Final} \PYG{n}{Group} \PYG{p}{(}\PYG{n}{Level} \PYG{l+m+mi}{6}\PYG{p}{)}\PYG{o}{.} \PYG{n}{Any} \PYG{n}{transaction} \PYG{o+ow}{not} \PYG{n}{caught} \PYG{n}{by} \PYG{n}{the} \PYG{n}{previous} \PYG{n}{levels}\PYG{o}{.}
\PYG{o}{/}\PYG{o}{/} \PYG{n}{Note} \PYG{n}{that} \PYG{n}{the} \PYG{n}{Criteria} \PYG{o+ow}{and} \PYG{n}{Regular} \PYG{n}{expression} \PYG{n}{properties} \PYG{n}{are} \PYG{o+ow}{not} \PYG{n}{used}\PYG{o}{.}
\PYG{n}{cAppMenuOptions}\PYG{p}{[}\PYG{l+s+s2}{\PYGZdq{}}\PYG{l+s+s2}{levels}\PYG{l+s+s2}{\PYGZdq{}}\PYG{p}{]}\PYG{p}{[}\PYG{l+m+mi}{6}\PYG{p}{]} \PYG{o}{=} \PYG{p}{\PYGZob{}}
\PYG{l+s+s2}{\PYGZdq{}}\PYG{l+s+s2}{title}\PYG{l+s+s2}{\PYGZdq{}}\PYG{p}{:}\PYG{l+s+s2}{\PYGZdq{}}\PYG{l+s+s2}{Other Sessions}\PYG{l+s+s2}{\PYGZdq{}}\PYG{p}{,}
\PYG{p}{\PYGZcb{}}\PYG{p}{;}
\end{sphinxVerbatim}


\section{PFK key assignment}
\label{\detokenize{Customization:pfk-key-assignment}}
\sphinxAtStartPar
PFK’s can be assigned to transactions within the CappMenu display as shortcuts.

\sphinxAtStartPar
\sphinxincludegraphics{{image87}.png}

\sphinxAtStartPar
The pf2tran array needs to be added to the cAppmenOptions array. The following is an example.

\begin{sphinxVerbatim}[commandchars=\\\{\}]
\PYG{o}{/}\PYG{o}{/} \PYG{n}{Titles} \PYG{k}{with} \PYG{n}{PFKs}
\PYG{n}{cAppMenuOptions} \PYG{o}{=} \PYG{p}{\PYGZob{}}
    \PYG{l+s+s2}{\PYGZdq{}}\PYG{l+s+s2}{texts}\PYG{l+s+s2}{\PYGZdq{}}\PYG{p}{:} \PYG{p}{\PYGZob{}}
    \PYG{l+s+s2}{\PYGZdq{}}\PYG{l+s+s2}{main\PYGZhy{}title}\PYG{l+s+s2}{\PYGZdq{}}\PYG{p}{:} \PYG{l+s+s2}{\PYGZdq{}}\PYG{l+s+s2}{Virtel Demo Application Tier Menu Session Manager}\PYG{l+s+s2}{\PYGZdq{}}\PYG{p}{,}
    \PYG{l+s+s2}{\PYGZdq{}}\PYG{l+s+s2}{sub\PYGZhy{}title}\PYG{l+s+s2}{\PYGZdq{}}\PYG{p}{:} \PYG{l+s+s2}{\PYGZdq{}}\PYG{l+s+s2}{SYSPERTEC Virtel V4.61 Application Menu}\PYG{l+s+s2}{\PYGZdq{}}
\PYG{p}{\PYGZcb{}}\PYG{p}{,}
\PYG{o}{/}\PYG{o}{/} \PYG{n}{mapping} \PYG{n}{of} \PYG{n}{the} \PYG{n}{pf} \PYG{n}{keys} \PYG{n}{connected} \PYG{n}{to} \PYG{n}{VIRTEL} \PYG{n}{transactions}
\PYG{l+s+s2}{\PYGZdq{}}\PYG{l+s+s2}{pf2tran}\PYG{l+s+s2}{\PYGZdq{}}\PYG{p}{:} \PYG{p}{[}
   \PYG{p}{\PYGZob{}} \PYG{l+s+s2}{\PYGZdq{}}\PYG{l+s+s2}{pf}\PYG{l+s+s2}{\PYGZdq{}}  \PYG{p}{:} \PYG{l+s+s2}{\PYGZdq{}}\PYG{l+s+s2}{PF 2}\PYG{l+s+s2}{\PYGZdq{}}     \PYG{p}{,}  \PYG{l+s+s2}{\PYGZdq{}}\PYG{l+s+s2}{tran}\PYG{l+s+s2}{\PYGZdq{}} \PYG{p}{:} \PYG{l+s+s2}{\PYGZdq{}}\PYG{l+s+s2}{CICSP1}\PYG{l+s+s2}{\PYGZdq{}}         \PYG{p}{\PYGZcb{}}\PYG{p}{,}
   \PYG{p}{\PYGZob{}} \PYG{l+s+s2}{\PYGZdq{}}\PYG{l+s+s2}{pf}\PYG{l+s+s2}{\PYGZdq{}}  \PYG{p}{:} \PYG{l+s+s2}{\PYGZdq{}}\PYG{l+s+s2}{PF 10}\PYG{l+s+s2}{\PYGZdq{}}    \PYG{p}{,}  \PYG{l+s+s2}{\PYGZdq{}}\PYG{l+s+s2}{tran}\PYG{l+s+s2}{\PYGZdq{}} \PYG{p}{:} \PYG{l+s+s2}{\PYGZdq{}}\PYG{l+s+s2}{CICST1}\PYG{l+s+s2}{\PYGZdq{}}         \PYG{p}{\PYGZcb{}}\PYG{p}{,}
   \PYG{p}{\PYGZob{}} \PYG{l+s+s2}{\PYGZdq{}}\PYG{l+s+s2}{pf}\PYG{l+s+s2}{\PYGZdq{}}  \PYG{p}{:} \PYG{l+s+s2}{\PYGZdq{}}\PYG{l+s+s2}{PF 12}\PYG{l+s+s2}{\PYGZdq{}}    \PYG{p}{,}  \PYG{l+s+s2}{\PYGZdq{}}\PYG{l+s+s2}{tran}\PYG{l+s+s2}{\PYGZdq{}} \PYG{p}{:} \PYG{l+s+s2}{\PYGZdq{}}\PYG{l+s+s2}{TSOP1}\PYG{l+s+s2}{\PYGZdq{}}          \PYG{p}{\PYGZcb{}}\PYG{p}{,}
   \PYG{p}{\PYGZob{}} \PYG{l+s+s2}{\PYGZdq{}}\PYG{l+s+s2}{pf}\PYG{l+s+s2}{\PYGZdq{}}  \PYG{p}{:} \PYG{l+s+s2}{\PYGZdq{}}\PYG{l+s+s2}{PF 9}\PYG{l+s+s2}{\PYGZdq{}}     \PYG{p}{,}  \PYG{l+s+s2}{\PYGZdq{}}\PYG{l+s+s2}{tran}\PYG{l+s+s2}{\PYGZdq{}} \PYG{p}{:} \PYG{l+s+s2}{\PYGZdq{}}\PYG{l+s+s2}{TSOT1}\PYG{l+s+s2}{\PYGZdq{}}          \PYG{p}{\PYGZcb{}}\PYG{p}{,}
   \PYG{p}{\PYGZob{}} \PYG{l+s+s2}{\PYGZdq{}}\PYG{l+s+s2}{pf}\PYG{l+s+s2}{\PYGZdq{}}  \PYG{p}{:} \PYG{l+s+s2}{\PYGZdq{}}\PYG{l+s+s2}{PF 4}\PYG{l+s+s2}{\PYGZdq{}}     \PYG{p}{,}  \PYG{l+s+s2}{\PYGZdq{}}\PYG{l+s+s2}{tran}\PYG{l+s+s2}{\PYGZdq{}} \PYG{p}{:} \PYG{l+s+s2}{\PYGZdq{}}\PYG{l+s+s2}{TSOT2}\PYG{l+s+s2}{\PYGZdq{}}          \PYG{p}{\PYGZcb{}}
    \PYG{p}{]}
\PYG{p}{\PYGZcb{}}
\end{sphinxVerbatim}


\section{PassThru Security}
\label{\detokenize{Customization:passthru-security}}
\sphinxAtStartPar
The CappMenu has a feature which allows security information to be passed through to the application rather than presenting the user with the transaction Virtel HTML sigon panel. This passthru feature is triggered by specifying the “passOnIdentification” option.

\begin{sphinxVerbatim}[commandchars=\\\{\}]
\PYG{n}{var} \PYG{n}{cAppMenuOptions} \PYG{o}{=} \PYG{p}{\PYGZob{}}
  \PYG{l+s+s2}{\PYGZdq{}}\PYG{l+s+s2}{passOnIdentification}\PYG{l+s+s2}{\PYGZdq{}}\PYG{p}{:} \PYG{n}{true}\PYG{p}{,}
  \PYG{l+s+s2}{\PYGZdq{}}\PYG{l+s+s2}{close\PYGZus{}VWA\PYGZus{}when\PYGZus{}disconnects}\PYG{l+s+s2}{\PYGZdq{}}\PYG{p}{:} \PYG{n}{true}
\PYG{p}{\PYGZcb{}}
\end{sphinxVerbatim}

\newpage


\section{Full reference document for cAppMenuOptions object}
\label{\detokenize{Customization:full-reference-document-for-cappmenuoptions-object}}
\begin{sphinxVerbatim}[commandchars=\\\{\}]
// v 02/04/2023
// Customer version
// customer directory.

// This file is related to the option of the application menu list VIRTEL transaction definition.


// This file builds the javascript object cAppMenuOptions,
// that defines the customizations.

// Create the javascript object that contains the options.
var cAppMenuOptions = \PYGZob{}\PYGZcb{};

// ====================
// Customize some texts
// ====================
// Remove the // comment mark on the next line, to create the level for texts customization:
// cAppMenuOptions.texts = \PYGZob{}\PYGZcb{};
//
// customize main title
// \PYGZhy{}\PYGZhy{}\PYGZhy{}\PYGZhy{}\PYGZhy{}\PYGZhy{}\PYGZhy{}\PYGZhy{}\PYGZhy{}\PYGZhy{}\PYGZhy{}\PYGZhy{}\PYGZhy{}\PYGZhy{}\PYGZhy{}\PYGZhy{}\PYGZhy{}\PYGZhy{}\PYGZhy{}\PYGZhy{}
// Replace xxx by the text for your main title
// remove the // comment mark on the next line, to create the main title customization:
// cAppMenuOptions.texts[\PYGZdq{}main\PYGZhy{}title\PYGZdq{}] = \PYGZdq{}xxx\PYGZdq{};
//
// customize subtitle
// \PYGZhy{}\PYGZhy{}\PYGZhy{}\PYGZhy{}\PYGZhy{}\PYGZhy{}\PYGZhy{}\PYGZhy{}\PYGZhy{}\PYGZhy{}\PYGZhy{}\PYGZhy{}\PYGZhy{}\PYGZhy{}\PYGZhy{}\PYGZhy{}\PYGZhy{}\PYGZhy{}
// Replace xxx by the text for your sub title
// remove the // comment mark on the next line, to create the sub title customization:
// cAppMenuOptions.texts[\PYGZdq{}sub\PYGZhy{}title\PYGZdq{}] = \PYGZdq{}xxx\PYGZdq{};
//
// customize texts of signon form
// \PYGZhy{}\PYGZhy{}\PYGZhy{}\PYGZhy{}\PYGZhy{}\PYGZhy{}\PYGZhy{}\PYGZhy{}\PYGZhy{}\PYGZhy{}\PYGZhy{}\PYGZhy{}\PYGZhy{}\PYGZhy{}\PYGZhy{}\PYGZhy{}\PYGZhy{}\PYGZhy{}\PYGZhy{}\PYGZhy{}\PYGZhy{}\PYGZhy{}\PYGZhy{}\PYGZhy{}\PYGZhy{}\PYGZhy{}\PYGZhy{}\PYGZhy{}\PYGZhy{}\PYGZhy{}
// Remove the // comment mark on the next line, to create the level for signon texts customization:
// cAppMenuOptions.texts.signon = \PYGZob{}\PYGZcb{};

// Replace the English standard text on the right part of the assignments,
// remove the leading // comment mark on the modified lines
//  cAppMenuOptions.texts.signon[\PYGZdq{}signon\PYGZhy{}invite\PYGZdq{}]=\PYGZdq{}Enter your username and password\PYGZdq{};
//  cAppMenuOptions.texts.signon[\PYGZdq{}user\PYGZhy{}label\PYGZdq{}]=\PYGZdq{}Username\PYGZdq{};
//  cAppMenuOptions.texts.signon[\PYGZdq{}password\PYGZhy{}label\PYGZdq{}]=\PYGZdq{}Password\PYGZdq{};
//  cAppMenuOptions.texts.signon[\PYGZdq{}newpassword\PYGZhy{}label\PYGZdq{}]=\PYGZdq{}New Password\PYGZdq{};
//  cAppMenuOptions.texts.signon[\PYGZdq{}confirmpassword\PYGZhy{}label\PYGZdq{}]=\PYGZdq{}Confirm password\PYGZdq{};
//  cAppMenuOptions.texts.signon[\PYGZdq{}login\PYGZhy{}button\PYGZdq{}]=\PYGZdq{}Validate\PYGZdq{};
//  cAppMenuOptions.texts.signon[\PYGZdq{}cancel\PYGZhy{}button\PYGZdq{}]=\PYGZdq{}Cancel\PYGZdq{};
//  cAppMenuOptions.texts.signon[\PYGZdq{}user\PYGZhy{}invite\PYGZdq{}]=\PYGZdq{}ENTER YOUR USERNAME\PYGZdq{};
//  cAppMenuOptions.texts.signon[\PYGZdq{}password\PYGZhy{}invite\PYGZdq{}]=\PYGZdq{}ENTER YOUR PASSWORD\PYGZdq{};
//  cAppMenuOptions.texts.signon[\PYGZdq{}newpassword\PYGZhy{}invite\PYGZdq{}]=\PYGZdq{}Enter and confirm your new password\PYGZdq{};
//  cAppMenuOptions.texts.signon[\PYGZdq{}newpassword\PYGZhy{}mismatch\PYGZdq{}]=\PYGZdq{}NEW PASSWORDS DO NOT MATCH\PYGZdq{};
//  cAppMenuOptions.texts.signon[\PYGZdq{}user\PYGZhy{}unknown\PYGZdq{}]=\PYGZdq{}USER NAME UNKNOWN\PYGZdq{};
//  cAppMenuOptions.texts.signon[\PYGZdq{}password\PYGZhy{}incorrect\PYGZdq{}]=\PYGZdq{}INCORRECT PASSWORD\PYGZdq{};
//  cAppMenuOptions.texts.signon[\PYGZdq{}password\PYGZhy{}expired\PYGZdq{}]=\PYGZdq{}Password expired\PYGZdq{};
//  cAppMenuOptions.texts.signon[\PYGZdq{}newpassword\PYGZhy{}invalid\PYGZdq{}]=\PYGZdq{}Newpassword is invalid.\PYGZdq{};

// From VIRTE 4.61 onwards:
// \PYGZhy{}\PYGZhy{}\PYGZhy{}\PYGZhy{}\PYGZhy{}\PYGZhy{}\PYGZhy{}\PYGZhy{}\PYGZhy{}\PYGZhy{}\PYGZhy{}\PYGZhy{}\PYGZhy{}\PYGZhy{}\PYGZhy{}\PYGZhy{}\PYGZhy{}\PYGZhy{}\PYGZhy{}\PYGZhy{}\PYGZhy{}\PYGZhy{}\PYGZhy{}
// customize credentials transmission logic
// in case the VIRTEL is set up with the screen lock feature.
// By default, you need to signon again when opening the session for the transaction selectged by the user on th application menu.
// This can be overriden by removing the leading // comment mark on the next line to activate it:
// cAppMenuOptions.passOnIdentification = true;


// =====================================
// Customize transaction opening options
// =====================================
//
// open to a new window
// \PYGZhy{}\PYGZhy{}\PYGZhy{}\PYGZhy{}\PYGZhy{}\PYGZhy{}\PYGZhy{}\PYGZhy{}\PYGZhy{}\PYGZhy{}\PYGZhy{}\PYGZhy{}\PYGZhy{}\PYGZhy{}\PYGZhy{}\PYGZhy{}\PYGZhy{}\PYGZhy{}\PYGZhy{}\PYGZhy{}
// Use: Each time you click a line of the application menu, you will open a new window for the session of the selected transaction.
// remove the // comment mark on the next line to set this option:
// cAppMenuOptions.open\PYGZus{}name = \PYGZdq{}\PYGZus{}blank\PYGZdq{};

//
// open to the current window
// \PYGZhy{}\PYGZhy{}\PYGZhy{}\PYGZhy{}\PYGZhy{}\PYGZhy{}\PYGZhy{}\PYGZhy{}\PYGZhy{}\PYGZhy{}\PYGZhy{}\PYGZhy{}\PYGZhy{}\PYGZhy{}\PYGZhy{}\PYGZhy{}\PYGZhy{}\PYGZhy{}\PYGZhy{}\PYGZhy{}\PYGZhy{}\PYGZhy{}\PYGZhy{}\PYGZhy{}\PYGZhy{}\PYGZhy{}
// Use: Clicking a application menu item will replace the application menu page with the session of the selected transaction.
// remove the // comment mark on the next line to set this option:
// cAppMenuOptions.open\PYGZus{}name = \PYGZdq{}\PYGZus{}self\PYGZdq{};

//
// open to a dedicated window
// \PYGZhy{}\PYGZhy{}\PYGZhy{}\PYGZhy{}\PYGZhy{}\PYGZhy{}\PYGZhy{}\PYGZhy{}\PYGZhy{}\PYGZhy{}\PYGZhy{}\PYGZhy{}\PYGZhy{}\PYGZhy{}\PYGZhy{}\PYGZhy{}\PYGZhy{}\PYGZhy{}\PYGZhy{}\PYGZhy{}\PYGZhy{}\PYGZhy{}\PYGZhy{}\PYGZhy{}\PYGZhy{}\PYGZhy{}
// Use: Clicking a application menu item will open the session of the selected transaction in a dedicated window, replacing the previous session if any.
// This feature is based on a name for this window.
// Below, you can keep the name \PYGZdq{}VWA\PYGZdq{}, or replace it with the name you want;
// remove the // comment mark on the next line to set this option:
// cAppMenuOptions.open\PYGZus{}name = \PYGZdq{}VWA\PYGZdq{};

//
// open specs
// \PYGZhy{}\PYGZhy{}\PYGZhy{}\PYGZhy{}\PYGZhy{}\PYGZhy{}\PYGZhy{}\PYGZhy{}\PYGZhy{}\PYGZhy{}
// To add HTML standard specs of window opening,
// replace specs by your specs (e.g.\PYGZdq{}location=no\PYGZdq{}),
// remove the // comment mark on the next line to set this option:
// cAppMenuOptions.open\PYGZus{}specs = specs;



//
// skip the application menu in case of a single transaction
// \PYGZhy{}\PYGZhy{}\PYGZhy{}\PYGZhy{}\PYGZhy{}\PYGZhy{}\PYGZhy{}\PYGZhy{}\PYGZhy{}\PYGZhy{}\PYGZhy{}\PYGZhy{}\PYGZhy{}\PYGZhy{}\PYGZhy{}\PYGZhy{}\PYGZhy{}\PYGZhy{}\PYGZhy{}\PYGZhy{}\PYGZhy{}\PYGZhy{}\PYGZhy{}\PYGZhy{}\PYGZhy{}\PYGZhy{}\PYGZhy{}\PYGZhy{}\PYGZhy{}\PYGZhy{}\PYGZhy{}\PYGZhy{}\PYGZhy{}\PYGZhy{}\PYGZhy{}\PYGZhy{}\PYGZhy{}\PYGZhy{}\PYGZhy{}\PYGZhy{}\PYGZhy{}\PYGZhy{}\PYGZhy{}\PYGZhy{}\PYGZhy{}\PYGZhy{}\PYGZhy{}\PYGZhy{}\PYGZhy{}\PYGZhy{}\PYGZhy{}\PYGZhy{}\PYGZhy{}\PYGZhy{}\PYGZhy{}\PYGZhy{}\PYGZhy{}\PYGZhy{}
// Use: If the list contains one transaction only (this might happen after filtering the transactions),
// the menu display will be skiped
// and the window will be opened of the VIRTEL transaction session.
// Remove the // comment mark on the next line to set this option:
// cAppMenuOptions.skip\PYGZus{}menu\PYGZus{}if\PYGZus{}single = true;



//
// close the VWA window after disconnection
// \PYGZhy{}\PYGZhy{}\PYGZhy{}\PYGZhy{}\PYGZhy{}\PYGZhy{}\PYGZhy{}\PYGZhy{}\PYGZhy{}\PYGZhy{}\PYGZhy{}\PYGZhy{}\PYGZhy{}\PYGZhy{}\PYGZhy{}\PYGZhy{}\PYGZhy{}\PYGZhy{}\PYGZhy{}\PYGZhy{}\PYGZhy{}\PYGZhy{}\PYGZhy{}\PYGZhy{}\PYGZhy{}\PYGZhy{}\PYGZhy{}\PYGZhy{}\PYGZhy{}\PYGZhy{}\PYGZhy{}\PYGZhy{}\PYGZhy{}\PYGZhy{}\PYGZhy{}\PYGZhy{}\PYGZhy{}\PYGZhy{}\PYGZhy{}\PYGZhy{}
// Remove the // comment mark on the next line to set this option:
// cAppMenuOptions.close\PYGZus{}VWA\PYGZus{}when\PYGZus{}disconnects = true;






// =====================================
// Customize user selectable information
// =====================================
//
// Prepare the user selectable design
// \PYGZhy{}\PYGZhy{}\PYGZhy{}\PYGZhy{}\PYGZhy{}\PYGZhy{}\PYGZhy{}\PYGZhy{}\PYGZhy{}\PYGZhy{}\PYGZhy{}\PYGZhy{}\PYGZhy{}\PYGZhy{}\PYGZhy{}\PYGZhy{}\PYGZhy{}\PYGZhy{}\PYGZhy{}\PYGZhy{}\PYGZhy{}\PYGZhy{}\PYGZhy{}\PYGZhy{}\PYGZhy{}\PYGZhy{}\PYGZhy{}\PYGZhy{}\PYGZhy{}\PYGZhy{}\PYGZhy{}\PYGZhy{}\PYGZhy{}\PYGZhy{}
// Use: Each time you click a line of the application menu, an url is build for starting the VIRTEL transaction session in its window.
// Some user selectable information can be taken into account to build it.
// Remove the // comment mark on the next line to prepare the user selectable design:
// cAppMenuOptions.userSelectable = \PYGZob{}\PYGZcb{};

//
// Prepare query parameters definition
// \PYGZhy{}\PYGZhy{}\PYGZhy{}\PYGZhy{}\PYGZhy{}\PYGZhy{}\PYGZhy{}\PYGZhy{}\PYGZhy{}\PYGZhy{}\PYGZhy{}\PYGZhy{}\PYGZhy{}\PYGZhy{}\PYGZhy{}\PYGZhy{}\PYGZhy{}\PYGZhy{}\PYGZhy{}\PYGZhy{}\PYGZhy{}\PYGZhy{}\PYGZhy{}\PYGZhy{}\PYGZhy{}\PYGZhy{}\PYGZhy{}\PYGZhy{}\PYGZhy{}\PYGZhy{}\PYGZhy{}\PYGZhy{}\PYGZhy{}\PYGZhy{}\PYGZhy{}
// Use: Each time you click a line of the application menu, an url is build for starting the VIRTEL transaction session in its window.
// Some query parameters might be taken into account, e.g. (after the question mark):
// http://my\PYGZhy{}VIRTEL\PYGZhy{}line\PYGZhy{}address/w2h/WEB2AJAX.htm+Tso?logmode=D4A32XX3\PYGZam{}rows=43\PYGZam{}cols=132
// Remove the // comment mark on the next line to prepare the definition of query parameters:
// cAppMenuOptions.userSelectable.queryParameter = \PYGZob{}\PYGZcb{};


//
// Define query parameters
// \PYGZhy{}\PYGZhy{}\PYGZhy{}\PYGZhy{}\PYGZhy{}\PYGZhy{}\PYGZhy{}\PYGZhy{}\PYGZhy{}\PYGZhy{}\PYGZhy{}\PYGZhy{}\PYGZhy{}\PYGZhy{}\PYGZhy{}\PYGZhy{}\PYGZhy{}\PYGZhy{}\PYGZhy{}\PYGZhy{}\PYGZhy{}\PYGZhy{}\PYGZhy{}
// Use: Each time you click a line of the application menu, an url is build for starting the VIRTEL transaction session in its window.
// Some query parameters might be taken into account, e.g. (after the question mark):
// http://my\PYGZhy{}VIRTEL\PYGZhy{}line\PYGZhy{}address/w2h/WEB2AJAX.htm+Tso?logmode=D4A32XX3\PYGZam{}rows=43\PYGZam{}cols=132

// \PYGZhy{} Define the name of the query parameter
// Replace xxx by the name of your query parameter, making sure that your VIRTEL is set up to take it into account.
// Remove the // comment mark on the next line to Start the definition for this query parameter:
// cAppMenuOptions.userSelectable.queryParameter[\PYGZdq{}xxx\PYGZdq{}] = \PYGZob{}\PYGZdq{}label\PYGZdq{}:\PYGZdq{}\PYGZdq{}, \PYGZdq{}values\PYGZdq{}:[]\PYGZcb{};

// \PYGZhy{} Define the label to be displayed on the application menu for the query parameter
// Replace xxx by the name of your query parameter, making sure that your VIRTEL is set up to take it into account.
// Replace yyy by the label.
// Remove the // comment mark on the next line to Start the definition for this query parameter:
// cAppMenuOptions.userSelectable.queryParameter[\PYGZdq{}xxx\PYGZdq{}][\PYGZdq{}label\PYGZdq{}] = \PYGZdq{}yyy\PYGZdq{};

// \PYGZhy{} Define the values of the query parameter
// Duplicate this last line as much as the number of values you want to take into account.
// For each of those lines
// \PYGZhy{} replace xxx by the value of your query parameter,
// \PYGZhy{} replace aaa by the label for the current value,
// \PYGZhy{} replace bbb by the current value.
// Remove the // comment mark on all of those lines to take these values into account:
// cAppMenuOptions.userSelectable.queryParameter[\PYGZdq{}xxx\PYGZdq{}][\PYGZdq{}items\PYGZdq{}].push(\PYGZob{}\PYGZdq{}label\PYGZdq{}:\PYGZdq{}aaaa\PYGZdq{}, \PYGZdq{}value\PYGZdq{}:\PYGZdq{}bbbb\PYGZdq{}\PYGZcb{});


// You can define an additional query parameter in the same way.
// You can define as much query parameters as you want,
// provided that your VIRTEL is set up to take them into account.



//
// Define templates
// \PYGZhy{}\PYGZhy{}\PYGZhy{}\PYGZhy{}\PYGZhy{}\PYGZhy{}\PYGZhy{}\PYGZhy{}\PYGZhy{}\PYGZhy{}\PYGZhy{}\PYGZhy{}\PYGZhy{}\PYGZhy{}\PYGZhy{}\PYGZhy{}
// Use: Each time you click a line of the application menu, an url is build for starting the VIRTEL transaction session in its window.  This url contains a template name.
// e.g. \PYGZdq{}WEB2AJAX.htm\PYGZdq{} in http://my\PYGZhy{}VIRTEL\PYGZhy{}line\PYGZhy{}address/w2h/WEB2AJAX.htm+Tso
// e Various templates, even customer templates in case of specific projets, can be made available for the user selection.

// \PYGZhy{} Start the templates definition
// Replace \PYGZdq{}xxx\PYGZdq{} by your label for templates.
// Remove the // comment mark on the next lines to Start the templates definition:
// cAppMenuOptions.userSelectable.template =  \PYGZob{}\PYGZdq{}label\PYGZdq{}:\PYGZdq{}\PYGZdq{}, \PYGZdq{}items\PYGZdq{}:[]\PYGZcb{};


// \PYGZhy{} Define the information for each template:
// Duplicate the last lines of this block ( with push(\PYGZdq{}xxx\PYGZdq{})) as much as the number of templates you want to take into account.
// For each of those lines, replace
// \PYGZhy{} \PYGZdq{}xxx\PYGZdq{} by the label for this template (e.g. \PYGZdq{}Standard VWA\PYGZdq{})
// \PYGZhy{} \PYGZdq{}yyy\PYGZdq{} by the path to the template (e.g. \PYGZdq{}../w2h/web2ajax.htm\PYGZdq{}), making sure that your VIRTEL is set up to take it into account.
// Remove the // comment mark on all of those lines to take these values into account:
// cAppMenuOptions.userSelectable.template.items.push(\PYGZob{}\PYGZdq{}label\PYGZdq{}:\PYGZdq{}xxx\PYGZdq{}, \PYGZdq{}value\PYGZdq{}:\PYGZdq{}yyy\PYGZdq{}\PYGZcb{});





// ==================
// Subdivide the list
// ==================
// Remove the // comment mark on the next line, to prepare a the first level of subdivision.
// cAppMenuOptions.levels = [];
//
// Design a selection definition
// \PYGZhy{}\PYGZhy{}\PYGZhy{}\PYGZhy{}\PYGZhy{}\PYGZhy{}\PYGZhy{}\PYGZhy{}\PYGZhy{}\PYGZhy{}\PYGZhy{}\PYGZhy{}\PYGZhy{}\PYGZhy{}\PYGZhy{}\PYGZhy{}\PYGZhy{}\PYGZhy{}\PYGZhy{}\PYGZhy{}\PYGZhy{}\PYGZhy{}\PYGZhy{}\PYGZhy{}\PYGZhy{}\PYGZhy{}\PYGZhy{}\PYGZhy{}\PYGZhy{}
// Remove the // comment mark on the next line, to prepare the selection definition number one ([0]), (to be transposed to [1] for the second, [2] for the third one and so on)
// cAppMenuOptions.levels[0] = \PYGZob{}\PYGZcb{};
//
// \PYGZhy{} Define the title of the selection:
// Replace xxx by the text for your selection title
// remove the // comment mark on the next line to define this title.
// cAppMenuOptions.levels[0].title = \PYGZdq{}xxx\PYGZdq{};
//
// \PYGZhy{} Define the criteria of the selection:
// Remove the // comment mark leading the line with the criteria you choose for your selection
// cAppMenuOptions.levels[0].criteria = \PYGZdq{}status\PYGZdq{};       // status of the transaction
// cAppMenuOptions.levels[0].criteria = \PYGZdq{}tran\PYGZdq{};         // external name of the VIRTEL transaction
// cAppMenuOptions.levels[0].criteria = \PYGZdq{}application\PYGZdq{};  // application of the VIRTEL transaction
// cAppMenuOptions.levels[0].criteria = \PYGZdq{}description\PYGZdq{};  // description of the VIRTEL transaction
//
// \PYGZhy{} Define the regular expression of the selection:
// Replace xxx by the regular expression for your selection
// Examples:
//   /cics/      // to select the transactions where the criteria contains \PYGZdq{}cics\PYGZdq{} (case sensitive)
//   /cics/i     // to select the transactions where the criteria contains \PYGZdq{}cics\PYGZdq{} (case insensitive)
//   /\PYGZca{}cics/     // to select the transactions where the criteria starts with \PYGZdq{}cics\PYGZdq{}
//   /cics\PYGZdl{}/     // to select the transactions where the criteria ends with \PYGZdq{}cics\PYGZdq{}

// remove the // comment mark on the next line to define this regular expression
// cAppMenuOptions.levels[0].regexp = /xxx/;

// \PYGZhy{} Define an additional level of subdivision:
// Advanced!
// You could divide further this selection, by adding a level, and defining it following the same instructions at the next level, and add recursively further levels.
// remove the // comment mark on the next line to start this next level definition:
// cAppMenuOptions.levels[0].levels = [];



// You can add a second ([1]) definition following the same instructions.
// If an additional definition does not contain a criteria,
// all the remaining list transactions will be kept.



// ==================
// Menu with pf keys
// ==================
// Remove the // comment mark on the next line, to prepare a the table connecting pf key values to transaction names:
// cAppMenuOptions.pf2tran = [];

// Remove the // comment mark on the next line and replace \PYGZdq{}xxx\PYGZdq{} by the target pf key value, and \PYGZdq{}yyy\PYGZdq{} by the external name of target VIRTEL transaction to be accessed by pressing the pf key.
// cAppMenuOptions.pf2tran.push(\PYGZob{} \PYGZdq{}pf\PYGZdq{}  : \PYGZdq{}xxx\PYGZdq{} , \PYGZdq{}tran\PYGZdq{} : \PYGZdq{}yyy\PYGZdq{} \PYGZcb{})
// Duplicate the step above for as much connexion pf / tran as needed.
\end{sphinxVerbatim}

\index{Language Support@\spxentry{Language Support}!Chinese@\spxentry{Chinese}}\ignorespaces 

\chapter{Virtel Language Support}
\label{\detokenize{Customization:virtel-language-support}}\label{\detokenize{Customization:index-125}}

\section{Introduction}
\label{\detokenize{Customization:id1}}
\sphinxAtStartPar
VIRTEL Web Access v4.59 and above supports a variety of SBDC and DBCS code pages. To support DBCS the Entry Point must have EXTCOLOR=X set.


\subsection{Chinese}
\label{\detokenize{Customization:chinese}}
\sphinxAtStartPar
Add IBM1388 into the CHARSET parameter list in the VIRTCT and the use the WEB2AJAC.HTM page instead of the (default) WEB2AJAX.HTM in the URL. See the Technical newsletter 2015/17 Virtel Chinese Character Language Support. TCT example follows: \sphinxhyphen{}
\begin{quote}

\sphinxAtStartPar
DEFUTF8=IBM1388,        DEFAULT OUTPUT ENCODING         *
CHARSET=(IBM1388,       CHINESE SIMPLIFIED              *
IBM1047,                US OPEN SYSTEMS                 *
IBM933A,                KOREAN                          *
IBM0037,                US EBCDIC                       *
IBM1390,IBM1399),       JAPANESE                        *
\end{quote}


\subsection{Thai}
\label{\detokenize{Customization:thai}}
\sphinxAtStartPar
The Thai character set is implement using the following TCT options : \sphinxhyphen{}

\sphinxAtStartPar
A W2H parm option also needs to be set. “w2hparm.leftalignpage” should be set to true. This value can also be set in the Miscellaneous options tab. This will allow left alignment of the W2H page. This may be useful when no monospace font is available (e.g. for Thailand).


\subsection{Arabic}
\label{\detokenize{Customization:arabic}}
\sphinxAtStartPar
The Alt\sphinxhyphen{}ENTER key is used to globally switch the screen display from ‘left to right’ to ‘right to left’ and return. In this way Arabic 3270 is supported. Alternate.Enter Mirror mode option needs to be set in the w2hparm defaults and Bi\sphinxhyphen{}Direction screen write.

\begin{sphinxadmonition}{note}{Note:}\begin{itemize}
\item {} 
\sphinxAtStartPar
the Alt\sphinxhyphen{}ENTER is active only if the ‘Bidirectional data’ Display option is set.

\item {} 
\sphinxAtStartPar
the Alt\sphinxhyphen{}ENTER key may be unset by going to w2h key mappings options.

\item {} 
\sphinxAtStartPar
w2hparm.mirrorMode may be set in a w2hparm.js file to preset the option (e.g. for user screens defaulting to arabic) use w2hparm.mirrorMode = true

\end{itemize}
\end{sphinxadmonition}

\sphinxAtStartPar
When the ‘Bidirectional Data’ option is active, the Alt+Shift key combination toggles data entry to/from Right\sphinxhyphen{}to\sphinxhyphen{}Left. This may be disturbing in contexts other than Hebrew, e.g. in Tunisia, where this key combination is used to switch keyboards. This option can be disabled by setting the w2hparm.altshift = no.  See the Key Mapping option in the Alt modifiers section.
\begin{quote}

\sphinxAtStartPar
altshift = “RTLinput” is the default, altshift = “no” may be specified.
\end{quote}

\sphinxAtStartPar
Contact Sysypertec support for more information on setting the correct w2hparm options to support Arabic.

\index{Virtel FTP Client@\spxentry{Virtel FTP Client}!FTP@\spxentry{FTP}}\ignorespaces 

\chapter{Virtel FTP Client}
\label{\detokenize{Customization:virtel-ftp-client}}\label{\detokenize{Customization:index-126}}

\section{Introduction}
\label{\detokenize{Customization:id2}}
\sphinxAtStartPar
VIRTEL Web Access v4.59 and above supports a FTP file transfer interface through the use of FTP command templates and the local FTP client. The Virtel FTP package consists of the following elements:
\begin{itemize}
\item {} 
\sphinxAtStartPar
One or more templates that Virtel uses to generate FTP parameters. These can be configured by the Virtel administrator.

\item {} 
\sphinxAtStartPar
A Virtel scenario that generates the FTP parameter file, when the function is called via a button on the toolbar of from an option on the Virtel administration menu.

\item {} 
\sphinxAtStartPar
A Windows desktop script that invokes the local FTP client and passes through the FTP commands generated by Virtel.

\end{itemize}


\section{How does it work?}
\label{\detokenize{Customization:how-does-it-work}}
\sphinxAtStartPar
Virtel runs in a browser and, for security reasons, cannot directly call the local FTP client program. Other than through the use of IND\$FILE, Virtel does not provide direct FTP connection to the mainframe. The Virtel FTP package centralizes the FTP file transfer parameters, where they can be configured by the Virtel Administrator. When an end\sphinxhyphen{}user requests for an FTP transfer to be initiated, an FTP parameter file is downloaded to the browser default download directory. The file transfer can then be executed by calling a batch script on the user’s desktop, that will use the parameter file as an input.


\subsection{Pre\sphinxhyphen{}requisites}
\label{\detokenize{Customization:pre-requisites}}\begin{itemize}
\item {} 
\sphinxAtStartPar
The user workstation must be able to run Microsoft’s ftp.exe program.

\item {} 
\sphinxAtStartPar
The user must be authorized to run batch (BAT) and Power shell scripts

\item {} 
\sphinxAtStartPar
The Virtel Entry point(s) must be defined to load scenarios from the SCE\sphinxhyphen{}DIR directory. This is normally the default.

\item {} 
\sphinxAtStartPar
A directory to keep the ancillary programs should be created on the users PC. This could be something like c:virtelftp. See Appendix A for me details on the ancillary post processing programs.

\end{itemize}


\section{Installing and configuring the FTP function}
\label{\detokenize{Customization:installing-and-configuring-the-ftp-function}}
\sphinxAtStartPar
Download and apply the Virtel 4.59 UPDT5799 update from the Virtel FTP website. This should apply they latest maintenance to Virtel which includes the FTP feature.
\begin{itemize}
\item {} 
\sphinxAtStartPar
Stop Virtel.

\item {} 
\sphinxAtStartPar
Add the following Virtel transactions: \sphinxhyphen{}

\end{itemize}

\begin{sphinxVerbatim}[commandchars=\\\{\}]
//INIARBO EXEC PGM=VIRCONF,PARM=LOAD
//STEPLIB  DD  DISP=SHR,DSN=VIRTEL.LOADLIB
//SYSPRINT DD  SYSOUT=*
//VIRARBO  DD  DISP=SHR,DSN=VIRTEL.ARBO
//SYSIN    DD  *
    TRANSACT ID=W2H\PYGZhy{}04E,                                             \PYGZhy{}
            NAME=FTPPARM,                                           \PYGZhy{}
            DESC=\PYGZsq{}FTPPARM   \PYGZsq{},                                      \PYGZhy{}
            APPL=\PYGZdl{}NONE\PYGZdl{},                                            \PYGZhy{}
            TYPE=2,                                                 \PYGZhy{}
            TERMINAL=DELOC,                                         \PYGZhy{}
            STARTUP=2,                                              \PYGZhy{}
            SECURITY=0,                                             \PYGZhy{}
            TIOASTA=\PYGZam{}/S,                                            \PYGZhy{}
            EXITSTA=FTPPARM
    TRANSACT ID=CLI\PYGZhy{}04E,                                             \PYGZhy{}
            NAME=FTPPARM,                                           \PYGZhy{}
            DESC=\PYGZsq{}FTPPARM   \PYGZsq{},                                      \PYGZhy{}
            APPL=\PYGZdl{}NONE\PYGZdl{},                                            \PYGZhy{}
            TYPE=2,                                                 \PYGZhy{}
            TERMINAL=CLLOC,                                         \PYGZhy{}
            STARTUP=2,                                              \PYGZhy{}
            SECURITY=0,                                             \PYGZhy{}
            TIOASTA=\PYGZam{}/S,                                            \PYGZhy{}
            EXITSTA=FTPPARM
\end{sphinxVerbatim}

\sphinxAtStartPar
The new transactions will provide the support for the FTP feature. The external/internal must be called FTPPARM. The transactions update both the administration entry point (W2H) and the client entry point (CLI).

\sphinxAtStartPar
\sphinxincludegraphics{{image59}.png}

\sphinxAtStartPar
\sphinxstyleemphasis{FTP transaction definition for FTP file transfer}


\subsection{FTP Scenario}
\label{\detokenize{Customization:ftp-scenario}}
\sphinxAtStartPar
The FTPPARM scenario, FTPPARM.390, can be found in the SCE directory. By default, the W2H entry point WEB2HOST, uses SCE\sphinxhyphen{}DIR as its directory for scenarios. However, for backward compatibility, the default for the client entry point CLI, CLIWHOST, doesn’t use SCE\sphinxhyphen{}DIR by default. It still loads scenarios from the Virtel LOADLIB. The scenario source FTPPARM can be found in the SAMPLIB datatset. Users can assemble it and place it in the VIRTEL LOADLIB  update the CLI entry point must be update to locate scenarios through the SCE\sphinxhyphen{}DIR. Alternatively, you could use a RULE | Entry Point combination or define anothe VIRTEL LINE for FTP support, pointing the associated entry point at SCE\sphinxhyphen{}DIR.


\section{Defining the FTP templates}
\label{\detokenize{Customization:defining-the-ftp-templates}}
\sphinxAtStartPar
Templates reside in the related directory, for eample W2H\sphinxhyphen{}DIR, CLI\sphinxhyphen{}DIR, or a user directory. The FTP package provides default templates for Upload and Download FTP requests. These can be customized to your site specifications and then re\sphinxhyphen{}uploaded to the W2H\sphinxhyphen{}DIR directory.

\sphinxAtStartPar
For example: sendipnw.txt. (Send new file to target with IP address)

\begin{sphinxVerbatim}[commandchars=\\\{\}]
\PYGZlt{}!\PYGZhy{}\PYGZhy{}VIRTEL start=\PYGZdq{}\PYGZob{}\PYGZob{}\PYGZob{}\PYGZdq{} end=\PYGZdq{}\PYGZcb{}\PYGZcb{}\PYGZcb{}\PYGZdq{} \PYGZhy{}\PYGZhy{}\PYGZgt{}OPEN \PYGZob{}\PYGZob{}\PYGZob{}CURRENT\PYGZhy{}VALUE\PYGZhy{}OF \PYGZdq{}IPDSN\PYGZdq{}\PYGZcb{}\PYGZcb{}\PYGZcb{}
USER \PYGZob{}\PYGZob{}\PYGZob{}CURRENT\PYGZhy{}VALUE\PYGZhy{}OF \PYGZdq{}USERFTP\PYGZdq{}\PYGZcb{}\PYGZcb{}\PYGZcb{}
?                                                                           \PYGZlt{}==== Optional Password Trigger!
\PYGZob{}\PYGZob{}\PYGZob{}CURRENT\PYGZhy{}VALUE\PYGZhy{}OF \PYGZdq{}TYPFTP\PYGZdq{}\PYGZcb{}\PYGZcb{}\PYGZcb{}
quote site blksize=\PYGZob{}\PYGZob{}\PYGZob{}CURRENT\PYGZhy{}VALUE\PYGZhy{}OF \PYGZdq{}BSZFTP\PYGZdq{}\PYGZcb{}\PYGZcb{}\PYGZcb{} lrecl=\PYGZob{}\PYGZob{}\PYGZob{}CURRENT\PYGZhy{}VALUE\PYGZhy{}OF \PYGZdq{}LRLFTP\PYGZdq{}\PYGZcb{}\PYGZcb{}\PYGZcb{} recfm=\PYGZob{}\PYGZob{}\PYGZob{}CURRENT\PYGZhy{}VALUE\PYGZhy{}OF \PYGZdq{}RFMFTP\PYGZdq{}\PYGZcb{}\PYGZcb{}\PYGZcb{}
\PYGZob{}\PYGZob{}\PYGZob{}CURRENT\PYGZhy{}VALUE\PYGZhy{}OF \PYGZdq{}SPCFTP\PYGZdq{}\PYGZcb{}\PYGZcb{}\PYGZcb{} pri=\PYGZob{}\PYGZob{}\PYGZob{}CURRENT\PYGZhy{}VALUE\PYGZhy{}OF \PYGZdq{}PRIFTP\PYGZdq{}\PYGZcb{}\PYGZcb{}\PYGZcb{} sec=\PYGZob{}\PYGZob{}\PYGZob{}CURRENT\PYGZhy{}VALUE\PYGZhy{}OF \PYGZdq{}SECFTP\PYGZdq{}\PYGZcb{}\PYGZcb{}\PYGZcb{}
PUT \PYGZob{}\PYGZob{}\PYGZob{}CURRENT\PYGZhy{}VALUE\PYGZhy{}OF \PYGZdq{}LOCALFILEFTP\PYGZdq{}\PYGZcb{}\PYGZcb{}\PYGZcb{} \PYGZob{}\PYGZob{}\PYGZob{}CURRENT\PYGZhy{}VALUE\PYGZhy{}OF \PYGZdq{}REMOTEFILEFTP\PYGZdq{}\PYGZcb{}\PYGZcb{}\PYGZcb{}
QUIT
\end{sphinxVerbatim}

\sphinxAtStartPar
The third line of the template defines the optional user password character. If it is set to “?”, then the user will be prompted for a password when running the background FTP script PSVirtelFtp.ps1.


\subsection{Template Variables}
\label{\detokenize{Customization:template-variables}}
\sphinxAtStartPar
The following variables can be used in the FTP templates:

\begin{sphinxVerbatim}[commandchars=\\\{\}]
\PYGZob{}\PYGZob{}\PYGZob{}CURRENT\PYGZhy{}VALUE\PYGZhy{}OF \PYGZdq{}USERFTP\PYGZdq{}\PYGZcb{}\PYGZcb{}\PYGZcb{}    = Screen field « Userid »
\PYGZob{}\PYGZob{}\PYGZob{}CURRENT\PYGZhy{}VALUE\PYGZhy{}OF \PYGZdq{}TYPFTP\PYGZdq{}\PYGZcb{}\PYGZcb{}\PYGZcb{}     = Screen field « text or Binary », value « ASCII » or « TYPE I »
\PYGZob{}\PYGZob{}\PYGZob{}CURRENT\PYGZhy{}VALUE\PYGZhy{}OF \PYGZdq{}TYPCHS\PYGZdq{}\PYGZcb{}\PYGZcb{}\PYGZcb{}     = Screen field « Charset »
\PYGZob{}\PYGZob{}\PYGZob{}CURRENT\PYGZhy{}VALUE\PYGZhy{}OF \PYGZdq{}REMOTEFILEFTP\PYGZdq{}\PYGZcb{}\PYGZcb{}\PYGZcb{}      = Grouped Screen field of  «Path: » and « File : » of Section « Host File »
\PYGZob{}\PYGZob{}\PYGZob{}CURRENT\PYGZhy{}VALUE\PYGZhy{}OF \PYGZdq{}LOCALFILEFTP\PYGZdq{}\PYGZcb{}\PYGZcb{}\PYGZcb{}       = Grouped Screen field «Path: » and « File : » of Section « PC File »
\PYGZob{}\PYGZob{}\PYGZob{}CURRENT\PYGZhy{}VALUE\PYGZhy{}OF \PYGZdq{}IPDSN\PYGZdq{}\PYGZcb{}\PYGZcb{}\PYGZcb{}      = Screen field « IUP/DNS »
\PYGZob{}\PYGZob{}\PYGZob{}CURRENT\PYGZhy{}VALUE\PYGZhy{}OF \PYGZdq{}BSZFTP\PYGZdq{}\PYGZcb{}\PYGZcb{}\PYGZcb{}     = Screen field « Blksize»
\PYGZob{}\PYGZob{}\PYGZob{}CURRENT\PYGZhy{}VALUE\PYGZhy{}OF \PYGZdq{}LRLFTP\PYGZdq{}\PYGZcb{}\PYGZcb{}\PYGZcb{}     = Screen field « Lrecl»
\PYGZob{}\PYGZob{}\PYGZob{}CURRENT\PYGZhy{}VALUE\PYGZhy{}OF \PYGZdq{}RFMFTP\PYGZdq{}\PYGZcb{}\PYGZcb{}\PYGZcb{}     = Screen field « Recfm»
\PYGZob{}\PYGZob{}\PYGZob{}CURRENT\PYGZhy{}VALUE\PYGZhy{}OF \PYGZdq{}SPCFTP\PYGZdq{}\PYGZcb{}\PYGZcb{}\PYGZcb{}     = Screen field « Space»
\PYGZob{}\PYGZob{}\PYGZob{}CURRENT\PYGZhy{}VALUE\PYGZhy{}OF \PYGZdq{}PRIFTP\PYGZdq{}\PYGZcb{}\PYGZcb{}\PYGZcb{}     = Screen field « Pri»
\PYGZob{}\PYGZob{}\PYGZob{}CURRENT\PYGZhy{}VALUE\PYGZhy{}OF \PYGZdq{}SECFTP\PYGZdq{}\PYGZcb{}\PYGZcb{}\PYGZcb{}     = Screen field « Sec»
\end{sphinxVerbatim}


\subsection{Defining the FTP templates for the action listbox}
\label{\detokenize{Customization:defining-the-ftp-templates-for-the-action-listbox}}
\sphinxAtStartPar
Modify the supplied JavaScript file “dataftplist.js” source to populate the “Action” list box with your personal templates. Upload to your Virtel directory.

\begin{sphinxVerbatim}[commandchars=\\\{\}]
// \PYGZdl{}Id: dataftplist.js ???? 2019\PYGZhy{}10\PYGZhy{}15 09:01:46 lepi \PYGZdl{}
var FTPLIST = new Array();

var alink = new Array();
alink[\PYGZdq{}ftpvalue\PYGZdq{}] = \PYGZdq{}send\PYGZdq{};
alink[\PYGZdq{}host\PYGZdq{}] = \PYGZdq{}T\PYGZdq{};
alink[\PYGZdq{}ftpmsg\PYGZdq{}] = \PYGZdq{}Send ip 192.168.92.161\PYGZdq{};
FTPLIST[0] = alink;

var alink = new Array();
alink[\PYGZdq{}ftpvalue\PYGZdq{}] = \PYGZdq{}sendip\PYGZdq{};
alink[\PYGZdq{}host\PYGZdq{}] = \PYGZdq{}T\PYGZdq{};
alink[\PYGZdq{}ftpmsg\PYGZdq{}] = \PYGZdq{}Send ip Target\PYGZdq{};
FTPLIST[1] = alink;
\end{sphinxVerbatim}


\subsection{Activate the Toolbar icons for file transfer}
\label{\detokenize{Customization:activate-the-toolbar-icons-for-file-transfer}}
\sphinxAtStartPar
A new icon for FTP package will displayed on the VIRTEL Web Access toolbar via a setting in the user parameter settings dialog. When the settings Option “Show the toggle button for VirtelFtp” is checked, a blue FTP icon will appear, as shown below:

\sphinxAtStartPar
\sphinxincludegraphics{{image60}.png}

\sphinxAtStartPar
\sphinxstyleemphasis{VIRTEL Web Access toolbar for FTP file transfer}


\subsection{Installing the bat files, shortcut and ancillary programs}
\label{\detokenize{Customization:installing-the-bat-files-shortcut-and-ancillary-programs}}
\sphinxAtStartPar
Within the SAMPTRSF directory the will be a zip file virtelFTP.zip. Download and unzip this to a directory on your workstation, for example c:VirtelFtp. To download enter the following URL

\begin{sphinxVerbatim}[commandchars=\\\{\}]
\PYG{l+m+mf}{192.168}\PYG{l+m+mf}{.170}\PYG{l+m+mf}{.48}\PYG{p}{:}\PYG{l+m+mi}{41001}\PYG{o}{/}\PYG{n}{PUBLIC}\PYG{o}{/}\PYG{n}{virtelftp}\PYG{o}{.}\PYG{n}{zip}
\end{sphinxVerbatim}

\sphinxAtStartPar
A “Save As” dialog box will open. Select a directory where you can save the file. Once unziped you will find a virtelFTP directory which contains the following:\sphinxhyphen{}
\begin{itemize}
\item {} 
\sphinxAtStartPar
documentation

\item {} 
\sphinxAtStartPar
A software directory containing four files

\end{itemize}

\sphinxAtStartPar
These four files are used to initiate the FTP file transfer and maintain the FTP profiles in local browser storage.

\begin{sphinxVerbatim}[commandchars=\\\{\}]
1.  Downloadftp.bat

    rem Test Batch FTP script.
    :: \PYGZdl{}Id: downloadftp.bat 5228 2019\PYGZhy{}12\PYGZhy{}03 13:57:05Z riou \PYGZdl{}
    ::Script Batch de transfert ftp.exe avec fichier FTPPARAM.TXT
    @echo off
    set dateLog=\PYGZpc{}date:\PYGZti{}6,4\PYGZpc{}\PYGZpc{}date:\PYGZti{}3,2\PYGZpc{}\PYGZpc{}date:\PYGZti{}0,2\PYGZpc{}
    set h=\PYGZpc{}TIME:\PYGZti{}0,2\PYGZpc{}
    set m=\PYGZpc{}TIME:\PYGZti{}3,2\PYGZpc{}
    set s=\PYGZpc{}TIME:\PYGZti{}6,2\PYGZpc{}
    set timeLog=\PYGZpc{}h\PYGZpc{}\PYGZus{}\PYGZpc{}m\PYGZpc{}\PYGZus{}\PYGZpc{}s\PYGZpc{}
    REM Launch FTP and pass it the  script
    ftp \PYGZhy{}n \PYGZhy{}s:\PYGZpc{}HOMEDRIVE\PYGZpc{}\PYGZpc{}HomePath\PYGZpc{}\PYGZbs{}Downloads\PYGZbs{}FTPPARAM.txt \PYGZgt{}\PYGZpc{}HOMEDRIVE\PYGZpc{}\PYGZpc{}HomePath\PYGZpc{}\PYGZbs{}Downloads\PYGZbs{}Log\PYGZus{}Ftp\PYGZus{}\PYGZpc{}dateLog\PYGZpc{}\PYGZus{}\PYGZpc{}timeLog\PYGZpc{}.log
    REM Delete params
    rem del \PYGZpc{}HOMEDRIVE\PYGZpc{}\PYGZpc{}HomePath\PYGZpc{}\PYGZbs{}Downloads\PYGZbs{}FTPPARAM.txt

2.  VirtelFtp.ps1

    Ancillary Power Shell script.

3.  VirtelFtp.vbs

    Ancillary Visual Basic program.

4.  PSVirtelFtp.bat

    Initial Bat file which triggers the FTP process. It should be defined as a short\PYGZhy{}cut on the desktop. It will process the FTPPARM.TXT file that has been created in the users “Download” directory. This file contains all the details of the FTP request built from the Virtel FTP screens. The file is used by the ancillary programs to initiate and perform the FTP operation.

    PSVirtelFtp.bat kicks off a power shell script VirtelFtp.ps1.

        echo off
        move \PYGZpc{}USERPROFILE\PYGZpc{}\PYGZbs{}Downloads\PYGZbs{}ftpparam.txt c:\PYGZbs{}virtel\PYGZbs{}ftp
        start /W /B powershell \PYGZhy{}file VirtelFtp.ps1 \PYGZpc{}1
        rem del \PYGZpc{}USERPROFILE\PYGZpc{}\PYGZbs{}Downloads\PYGZbs{}ftpparam.txt
\end{sphinxVerbatim}


\subsection{Set up a shortcut on your desktop}
\label{\detokenize{Customization:set-up-a-shortcut-on-your-desktop}}
\sphinxAtStartPar
For ease of operation set up a shortcut on your directory, pointing to the .bat file in c:virtelftp. Once the FTP parameters have been built by Virtel FTP, clicking this shortcut will invoke the FTP file transfer.


\section{How to use the Virtel FTP function}
\label{\detokenize{Customization:how-to-use-the-virtel-ftp-function}}

\subsection{Sending and Receiving a file}
\label{\detokenize{Customization:sending-and-receiving-a-file}}
\sphinxAtStartPar
To send a file from your workstation to a target FTP site, click the FTP icon on the Virtel toolbar. The “VIRTEL FTP” dialog will open:

\sphinxAtStartPar
\sphinxincludegraphics{{image61}.png}

\sphinxAtStartPar
\sphinxstyleemphasis{VIRTEL FTP Dialog Part 1}

\begin{sphinxVerbatim}[commandchars=\\\{\}]
*   Select « Send ip Target » in the action list box.
*   Specify the Username to be used for the FTP connection.
*   In Section “PC File”
*   Specify the Local Path if necessary.
*   Specify the PC file name. You can also specify the pathname here or \PYGZhy{}
*   Drag And Drop a file onto the “Drag files from your directory path to here! » zone. This will populate the filename field.
*   Specify the Charset to be used for the FTP transfer. The default is Windows CP\PYGZhy{}1252
*   In Section “Host File”
*   Specify the Path (equal TSO dataset name or TSO PDS), surrounded by quotes if necessary. Dataset names without quotes will be prefixed by your TSO prefix (usually your userid).
*   Specify the File (equal TSO dataset name).
*   In sub\PYGZhy{}Section “FTP Options” in the “Host File” section
*   Choose either “Text” or “Binary”. “Text” translates the file from EBCDIC to ASCII and inserts carriage return line feed sequences (x’0D0A’) at the end of each record. “Binary” performs no translation.
*   In sub\PYGZhy{}Section “Host” in “Host File”
*   Select the target host type (TSO/UNIX/JES2) associated with the \PYGZdq{}Transfer Type\PYGZdq{} action. This will modify the actions to the selected type.
*   Finally click “Build” to generate the FTP parameter file.
\end{sphinxVerbatim}

\sphinxAtStartPar
When the Generate Parameter is complete, the browser’s “VIRTEL FTP Save Parameter” dialog appears:

\sphinxAtStartPar
\sphinxincludegraphics{{image62}.png}

\sphinxAtStartPar
\sphinxstyleemphasis{VIRTEL FTP Dialog Part 2}

\sphinxAtStartPar
Click “Save”. The “Save As” dialog will open to allow you to specify the name and location of the destination file on your workstation:

\sphinxAtStartPar
\sphinxincludegraphics{{image63}.png}

\sphinxAtStartPar
\sphinxstyleemphasis{VIRTEL FTP Dialog part 3}

\sphinxAtStartPar
Now select the destination file and click “Save”. If the file already exists, you will be prompted for permission to overwrite it. The “Download Complete” dialog appears when the file has been saved. Once the file has downloaded, run the PSVirtelFtp.bat procedure. This should be setup as a shortcut on your desktop.

\sphinxAtStartPar
If the password was defined by a”?” in the template, the following dialog will appear.

\sphinxAtStartPar
\sphinxincludegraphics{{image64}.png}

\sphinxAtStartPar
\sphinxstyleemphasis{VIRTEL FTP Entering your password}

\sphinxAtStartPar
After entering the password, the transfer will be executed, and the following console log should be displayed.

\sphinxAtStartPar
\sphinxincludegraphics{{image65}.png}

\sphinxAtStartPar
\sphinxstyleemphasis{VIRTEL FTP Console Log}

\sphinxAtStartPar
.The last line of the execution report in the DOS window provides information on the execution status.


\subsection{Selecting a PC Codepage for FTP file transfer}
\label{\detokenize{Customization:selecting-a-pc-codepage-for-ftp-file-transfer}}
\sphinxAtStartPar
Users can select a PC Codepage for FTP file transfer. Possible values are: \sphinxhyphen{}
\begin{itemize}
\item {} 
\sphinxAtStartPar
Windows (CP\sphinxhyphen{}1252) : CP\sphinxhyphen{}1252 ASCII\sphinxhyphen{}EBCDIC translation table.

\item {} 
\sphinxAtStartPar
MS\sphinxhyphen{}DOS (CP\sphinxhyphen{}850) : CP\sphinxhyphen{}850 ASCII\sphinxhyphen{}EBCDIC translation table. This depends on the value of the COUNTRY parameter specified in the VIRTCT. If this value is “FR”, “DE” or “BE”, system will use corresponding table FR\sphinxhyphen{}850, DE\sphinxhyphen{}850 or BE\sphinxhyphen{}850. If the specified country value is different, by default the BE\sphinxhyphen{}850 table will be used to support CECP 500 international EBCDIC.

\end{itemize}


\subsection{Saving and reusing FTP file transfer parameters}
\label{\detokenize{Customization:saving-and-reusing-ftp-file-transfer-parameters}}
\sphinxAtStartPar
Users who frequently use the same FTP file transfers can save the parameters for later reuse. To save a FTP file transfer, enter the dataset name and the type of transfer, and click the “Save” button: \sphinxhyphen{}

\sphinxAtStartPar
\sphinxincludegraphics{{image66}.png}

\sphinxAtStartPar
\sphinxstyleemphasis{VIRTEL FTP Receive dialog saved transfers}

\sphinxAtStartPar
The user can then choose a name for the saved transfer and click “OK” to save the parameters. At the next transfer, the user clicks the name of the saved transfer to retrieve the parameters, then clicks “Generate Parameter” to start the generate.

\sphinxAtStartPar
\sphinxincludegraphics{{image67}.png}

\sphinxAtStartPar
\sphinxstyleemphasis{VIRTEL FTP Saving the FTP file transfer parameters}

\sphinxAtStartPar
Users can save transfer parameters for “Generate Parameter”. The parameters are saved in browser local storage. The number of sets of parameters which can be saved is limited only by the amount of local storage available.


\subsection{Showing / Hiding FTP File Transfer}
\label{\detokenize{Customization:showing-hiding-ftp-file-transfer}}
\sphinxAtStartPar
The administrator may wish to prevent users from accessing features like FTP File Transfer. This example shows how to hide the toolbar option by setting a parameter in the w2hparm.

\begin{sphinxVerbatim}[commandchars=\\\{\}]
/*
* set the value of the “Show toggle button for VirtelFtp”.
*/
w2hparm.showftp = true;
\end{sphinxVerbatim}

\sphinxAtStartPar
\sphinxstyleemphasis{Example w2hparm.js for hiding FTP File Transfer}

\sphinxAtStartPar
Users may also use the Virtel Settings Display options and deselect the VirtelFTP toggle.

\sphinxAtStartPar
\sphinxincludegraphics{{image68}.png}

\sphinxAtStartPar
\sphinxstyleemphasis{VIRTEL FTP Toggle the FTP button in the settings panel}

\sphinxAtStartPar
If the default value is used, users can show this function for their own usage by checking “Show the toggle button for VirtelFtp” in the Display tab of the settings panel.


\section{Debugging VirtelFTP}
\label{\detokenize{Customization:debugging-virtelftp}}

\subsection{Power Shell authority}
\label{\detokenize{Customization:power-shell-authority}}
\sphinxAtStartPar
Users may find that the an error message appears when invoking the Power Shell Script. This can be corrected by issuing the Windows command:

\begin{sphinxVerbatim}[commandchars=\\\{\}]
\PYG{n}{Set}\PYG{o}{\PYGZhy{}}\PYG{n}{ExecutionPolicy} \PYG{n}{Bypass}
\end{sphinxVerbatim}

\index{FAQ Questions@\spxentry{FAQ Questions}!Square Brackets and codepages@\spxentry{Square Brackets and codepages}}\index{Codepage in the URL@\spxentry{Codepage in the URL}}\ignorespaces 

\chapter{FAQs}
\label{\detokenize{Customization:faqs}}\label{\detokenize{Customization:index-127}}

\section{How to specify a codepage in a URL}
\label{\detokenize{Customization:how-to-specify-a-codepage-in-a-url}}
\sphinxAtStartPar
Some users want to see square brackets displayed in their VWA presentation, for example “C” programmers. The default UTF8 / MVS codepage translation doesn’t transalate square brackets correctly. Users must define the correct codepage in the URL. In this instance for square brackets it is codepage IBM1047.

\begin{sphinxVerbatim}[commandchars=\\\{\}]
http://10.20.170.71:41001/w2h/WEB2AJAX.htm+Tso?codepage=IBM1047
\end{sphinxVerbatim}

\index{FAQ Questions@\spxentry{FAQ Questions}!OMVS panel@\spxentry{OMVS panel}}\ignorespaces 

\section{How to use the TSO OMVS panel command with VWA}
\label{\detokenize{Customization:how-to-use-the-tso-omvs-panel-command-with-vwa}}\label{\detokenize{Customization:index-128}}
\sphinxAtStartPar
TSO OMVS writes to the screen after 20 seconds changing the status from RUNNING to INPUT. This asynchronous nature forces Virtel to generate a HTML page which updates the whole screen, erasing any partially entered input field. Once the screen is in INPUT mode OMVS will not write to the screen in this ASYNC fashion anymore. The work around for this is to wait for the status change before entering any input. Alternatively you can put the screen into INPUT mode immediately by pressing F12=Retrieve, and then pressing Home and End to clear the input area.

\sphinxAtStartPar
Another solution is to use PuTTY to connect to the OMVS shell via rlogin to port 513. This solution avoids many of the other usability problems which are inherent in the OMVS shell.

\index{FAQ Questions@\spxentry{FAQ Questions}!z/OS Consoles@\spxentry{z/OS Consoles}}\ignorespaces 

\section{Can VIRTEL WEB ACCESS emulate z/OS Operator consoles}
\label{\detokenize{Customization:can-virtel-web-access-emulate-z-os-operator-consoles}}\label{\detokenize{Customization:index-129}}
\sphinxAtStartPar
Technically VIRTEL does not support the OSA\sphinxhyphen{}ICC console server because it is based on the tn3270 protocol which VIRTEL does not use. But in any case it probably would not make sense to put consoles under VIRTEL control because the consoles need to be operational during IPL before TCP/IP and VIRTEL are started. So we would expect customers to continue to run tn3270 emulators on those PCs which are used as z/OS operator consoles.

\index{FAQ Questions@\spxentry{FAQ Questions}!Hotspot recognition with regular expressions@\spxentry{Hotspot recognition with regular expressions}}\ignorespaces 

\section{Can I customize hotspot recognition}
\label{\detokenize{Customization:can-i-customize-hotspot-recognition}}\label{\detokenize{Customization:index-130}}
\sphinxAtStartPar
The regular expressions which control hotspot recognition may be overridden by setting VIR3270.customPfKeysHotspotRegex and/or VIR3270.customUrlHotspotsRegex in a customized javascript member. For example, an application requires strings of the format “PFnn\sphinxhyphen{}caption” to be recognized as PF hotspots. The customized code in the javascript member would look like would be:

\begin{sphinxVerbatim}[commandchars=\\\{\}]
function after\PYGZus{}responseHandle(o, url, xtim) \PYGZob{}
VIR3270.customPfKeysHotspotRegex = /(P?F\PYGZbs{}d\PYGZob{}1,2\PYGZcb{}|PA[1\PYGZhy{}3]|ENTER|CLEAR)((?:\PYGZbs{}/P?F\PYGZbs{}d\PYGZob{}1,2\PYGZcb{})?\PYGZbs{}s*[=:\PYGZhy{}])/;
\PYGZcb{}
\end{sphinxVerbatim}


\chapter{Appendix}
\label{\detokenize{Customization:appendix}}

\section{Appendix A \sphinxhyphen{} Virtel Macro Quick reference Sheet}
\label{\detokenize{Customization:appendix-a-virtel-macro-quick-reference-sheet}}
\sphinxAtStartPar
Virtel macros capture keystroke operations which can subsequently be used to automate 3270 functions.
These user captured macros are stored within a file called macros.json. This file is a JavaScript array of JSON objects, with each object representing a user
macro.


\subsection{Example of a Macro. JSON file}
\label{\detokenize{Customization:example-of-a-macro-json-file}}
\begin{sphinxVerbatim}[commandchars=\\\{\}]
\PYGZob{}«macros»:[
\PYGZob{}«name»:»mylogon»,»rev»:2,»def»:[\PYGZob{}«txt»:»sptholt»\PYGZcb{},»ENTER»,\PYGZob{}«txt»:»password»\PYGZcb{},»ENTER»,»ENTER»,»ENTER»],»mapping»:\PYGZob{}«key»:»ctrl»,»keycode»:76\PYGZcb{}\PYGZcb{},
\PYGZob{}«name»:»logoff»,»rev»:1,»def»:[\PYGZob{}«txt»:»=x»\PYGZcb{},»ENTER»,\PYGZob{}«txt»:»logoff»\PYGZcb{},»ENTER»],»mapping»:\PYGZob{}«key»:»ctrl»,»keycode»:79\PYGZcb{}\PYGZcb{},
\PYGZob{}«name»:»logon»,»rev»:2,»def»:[«Tab»,»Down»,\PYGZob{}«txt»:»sptholx»\PYGZcb{},»ENTER»,\PYGZob{}«txt»:»PASSWORD»\PYGZcb{},»ENTER»,»ENTER»,»ENTER»],»mapping»:\PYGZob{}«key»:»alt»,»keycode»:76\PYGZcb{}\PYGZcb{}
],»fmt»:2\PYGZcb{}
\end{sphinxVerbatim}


\subsection{Macro Formats and Commands}
\label{\detokenize{Customization:id3}}
\sphinxAtStartPar
The format of the MACROS.JSON file is an embedded JSON structure. Each name structure represents a keystroke macro identified by the “name” keyword.
* Name: The name of the macro entry.
* Rev: The «rev» is a user revision keyword.
* Def: The «def» keyword identifies the commands and entry values


\subsection{Key Identifiers}
\label{\detokenize{Customization:key-identifiers}}
\begin{sphinxVerbatim}[commandchars=\\\{\}]
\PYG{n}{key}\PYG{p}{(}\PYG{n}{ENTER}\PYG{p}{)}   \PYG{n}{key}\PYG{p}{(}\PYG{n}{PF1}\PYG{o}{\PYGZhy{}}\PYG{l+m+mi}{24}\PYG{p}{)}    \PYG{n}{key}\PYG{p}{(}\PYG{n}{PA1}\PYG{o}{\PYGZhy{}}\PYG{l+m+mi}{3}\PYG{p}{)}       \PYG{n}{key}\PYG{p}{(}\PYG{n}{Down}\PYG{p}{)}       \PYG{n}{key}\PYG{p}{(}\PYG{n}{Up}\PYG{p}{)}         \PYG{n}{key}\PYG{p}{(}\PYG{n}{Left}\PYG{p}{)}       \PYG{n}{key}\PYG{p}{(}\PYG{n}{Right}\PYG{p}{)}          \PYG{n}{key}\PYG{p}{(}\PYG{n}{Newline}\PYG{p}{)}       \PYG{n}{Key}\PYG{p}{(}\PYG{n}{Tab}\PYG{p}{)}         \PYG{n}{Key}\PYG{p}{(}\PYG{n}{Backtab}\PYG{p}{)}
\PYG{n}{key}\PYG{p}{(}\PYG{n}{CLEAR}\PYG{p}{)}   \PYG{n}{key}\PYG{p}{(}\PYG{n}{Home}\PYG{p}{)}          \PYG{n}{key}\PYG{p}{(}\PYG{n}{ATTN}\PYG{p}{)}    \PYG{n}{key}\PYG{p}{(}\PYG{n}{End}\PYG{p}{)}        \PYG{n}{key}\PYG{p}{(}\PYG{n}{Bksp}\PYG{p}{)}       \PYG{n}{key}\PYG{p}{(}\PYG{n}{ErEof}\PYG{p}{)}      \PYG{n}{key}\PYG{p}{(}\PYG{n}{InsToggle}\PYG{p}{)}  \PYG{n}{key}\PYG{p}{(}\PYG{n}{Del}\PYG{p}{)}       \PYG{n}{key}\PYG{p}{(}\PYG{n}{Reset}\PYG{p}{)}
\PYG{n}{key}\PYG{p}{(}\PYG{n}{FieldMark}\PYG{p}{)} \PYG{n}{key}\PYG{p}{(}\PYG{n}{Dup}\PYG{p}{)}
\end{sphinxVerbatim}


\subsection{Built in functions}
\label{\detokenize{Customization:built-in-functions}}
\begin{sphinxVerbatim}[commandchars=\\\{\}]
\PYG{n}{waitScreen}\PYG{p}{(}\PYG{p}{)}    \PYG{n}{Virtel} \PYG{n}{will} \PYG{n}{wait} \PYG{k}{for} \PYG{n}{a} \PYG{l+m+mi}{3270} \PYG{n}{response} \PYG{n}{before} \PYG{n}{proceeding}\PYG{o}{.}
\end{sphinxVerbatim}


\subsection{Built in functions:Literal values \sphinxhyphen{} key stroke in VWA editor}
\label{\detokenize{Customization:built-in-functions-literal-values-key-stroke-in-vwa-editor}}\begin{description}
\sphinxlineitem{::}
\sphinxAtStartPar
“ISPF”          Any string of characters to input into a 3270 screen, example ISPF

\end{description}


\subsection{Used for Cut/Paste in a Macro}
\label{\detokenize{Customization:used-for-cut-paste-in-a-macro}}
\begin{sphinxVerbatim}[commandchars=\\\{\}]
\PYG{n}{move}\PYG{p}{(}\PYG{n}{pos}\PYG{p}{)}
\PYG{n}{copy}\PYG{p}{(}\PYG{n}{startRow}\PYG{p}{,}\PYG{n}{startCol}\PYG{p}{,}\PYG{n}{endRow}\PYG{p}{,}\PYG{n}{endCol}\PYG{p}{)}
\PYG{n}{paste}\PYG{p}{(}\PYG{n}{pos}\PYG{p}{)}
\PYG{n}{paste}\PYG{p}{(}\PYG{n}{pos}\PYG{p}{,}\PYG{n}{nbRows}\PYG{p}{,}\PYG{n}{nbCols}\PYG{p}{)}
\end{sphinxVerbatim}


\chapter{Trademarks}
\label{\detokenize{Customization:trademarks}}
\sphinxAtStartPar
SysperTec, the SysperTec logo, syspertec.com and VIRTEL are trademarks or registered trademarks of SysperTec
Communication Group, registered in France and other countries.

\sphinxAtStartPar
IBM, VTAM, CICS, IMS, RACF, DB2, z/OS, WebSphere, MQSeries, System z are trademarks or registered trademarks of
International Business Machines Corp., registered in United States and other countries.

\sphinxAtStartPar
Adobe, Acrobat, PostScript and all Adobe\sphinxhyphen{}based trademarks are either registered trademarks or trademarks of Adobe
Systems Incorporated in the United States and other countries.

\sphinxAtStartPar
Microsoft, Windows, Windows NT, and the Windows logo are trademarks of Microsoft Corporation in the United States
and other countries.

\sphinxAtStartPar
UNIX is a registered trademark of The Open Group in the United States and other countries.
Java and all Java\sphinxhyphen{}based trademarks and logos are trademarks or registered trademarks of Oracle and/or its affiliates.

\sphinxAtStartPar
Linux is a trademark of Linus Torvalds in the United States, other countries, or both.

\sphinxAtStartPar
Other company, product, or service names may be trademarks or service names of others.



\renewcommand{\indexname}{Index}
\printindex
\end{document}