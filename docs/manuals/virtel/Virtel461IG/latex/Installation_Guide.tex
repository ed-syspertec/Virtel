%% Generated by Sphinx.
\def\sphinxdocclass{report}
\documentclass[letterpaper,10pt,english]{sphinxmanual}
\ifdefined\pdfpxdimen
   \let\sphinxpxdimen\pdfpxdimen\else\newdimen\sphinxpxdimen
\fi \sphinxpxdimen=.75bp\relax
\ifdefined\pdfimageresolution
    \pdfimageresolution= \numexpr \dimexpr1in\relax/\sphinxpxdimen\relax
\fi
%% let collapsible pdf bookmarks panel have high depth per default
\PassOptionsToPackage{bookmarksdepth=5}{hyperref}

\PassOptionsToPackage{booktabs}{sphinx}
\PassOptionsToPackage{colorrows}{sphinx}

\PassOptionsToPackage{warn}{textcomp}


\usepackage{cmap}

\usepackage{amsmath,amssymb,amstext}
\usepackage{babel}





\usepackage[Bjarne]{fncychap}
\usepackage[,numfigreset=1,mathnumfig]{sphinx}

\fvset{fontsize=auto}
\usepackage{geometry}


% Include hyperref last.
\usepackage{hyperref}
% Fix anchor placement for figures with captions.
\usepackage{hypcap}% it must be loaded after hyperref.
% Set up styles of URL: it should be placed after hyperref.
\urlstyle{same}

\addto\captionsenglish{\renewcommand{\contentsname}{Table of Contents:}}

\usepackage{sphinxmessages}
\setcounter{tocdepth}{2}


% Enable unicode and use Courier New to ensure the card suit
% characters that are part of the 'random' module examples
% appear properly in the PDF output.
\usepackage{fontspec}
\setmonofont{Courier New}


\title{Virtel Installation Guide}
\date{Jul 23, 2023}
\release{4.61}
\author{Syspertec Communications}
\newcommand{\sphinxlogo}{\vbox{}}
\renewcommand{\releasename}{Release}
\makeindex
\begin{document}

\ifdefined\shorthandoff
  \ifnum\catcode`\=\string=\active\shorthandoff{=}\fi
  \ifnum\catcode`\"=\active\shorthandoff{"}\fi
\fi

\pagestyle{empty}
\sphinxmaketitle
\pagestyle{plain}
\sphinxtableofcontents
\pagestyle{normal}
\phantomsection\label{\detokenize{Installation_Guide::doc}}


\sphinxAtStartPar
\sphinxincludegraphics[scale=0.5]{{logo_virtel_web}.png}

\sphinxAtStartPar
\sphinxstylestrong{VIRTEL Installation Guide}

\begin{sphinxadmonition}{warning}{Warning:}
\sphinxAtStartPar
This is a draft version of the document.
\end{sphinxadmonition}

\sphinxAtStartPar
Version : 4.61 Draft

\sphinxAtStartPar
Release Date : TBA. Publication Date : 10/10/2021

\sphinxAtStartPar
Syspertec Communication

\sphinxAtStartPar
196, Bureaux de la Colline 92213 Saint\sphinxhyphen{}Cloud Cedex Tél. : +33 (0) 1 46 02 60 42

\sphinxAtStartPar
\sphinxhref{http://www.syspertec.com/}{www.syspertec.com}

\begin{sphinxadmonition}{note}{Note:}
\sphinxAtStartPar
Reproduction, transfer, distribution, or storage, in any form, of all or any part of
the contents of this document, except by prior authorization of SysperTec
Communication, is prohibited.

\sphinxAtStartPar
Every possible effort has been made by SysperTec Communication to ensure that this document
is complete and relevant. In no case can SysperTec Communication be held responsible for
any damages, direct or indirect, caused by errors or omissions in this document.

\sphinxAtStartPar
As SysperTec Communication uses a continuous development methodology; the information
contained in this document may be subject to change without notice. Nothing in this
document should be construed in any manner as conferring a right to use, in whole or in
part, the products or trademarks quoted herein.

\sphinxAtStartPar
“SysperTec Communication” and “VIRTEL” are registered trademarks. Names of other products
and companies mentioned in this document may be trademarks or registered trademarks of
their respective owners.
\end{sphinxadmonition}

\newpage


\chapter{Summary of Amendments}
\label{\detokenize{Installation_Guide:summary-of-amendments}}\label{\detokenize{Installation_Guide:virtel461ig-summary-of-ammendments}}

\section{Virtel version 4.61 (10th Oct 2021)}
\label{\detokenize{Installation_Guide:virtel-version-4-61-10th-oct-2021}}
\begin{sphinxadmonition}{note}{Note:}
\sphinxAtStartPar
For further details see the Virtel Technical Newsletter TN202303: Whats new in Virtel 4.61.
\end{sphinxadmonition}


\section{Virtel version 4.60 (11th Nov 2020)}
\label{\detokenize{Installation_Guide:virtel-version-4-60-11th-nov-2020}}
\begin{sphinxadmonition}{note}{Note:}
\sphinxAtStartPar
For further details see the Virtel Technical Newsletter TN202003: Whats new in Virtel 4.60.
\end{sphinxadmonition}


\section{Virtel version 4.59 (12th Jul 2019)}
\label{\detokenize{Installation_Guide:virtel-version-4-59-12th-jul-2019}}
\begin{sphinxadmonition}{note}{Note:}
\sphinxAtStartPar
For further details see the Virtel Technical Newsletter TN201902: Whats new in Virtel 4.59.
\end{sphinxadmonition}


\section{Virtel version 4.58 (1st Oct 2018)}
\label{\detokenize{Installation_Guide:virtel-version-4-58-1st-oct-2018}}
\begin{sphinxadmonition}{note}{Note:}
\sphinxAtStartPar
For further details see the Virtel Technical Newsletter TN201803: Whats new in Virtel 4.58.
\end{sphinxadmonition}


\section{Virtel version 4.57 (1st Jul 2017)}
\label{\detokenize{Installation_Guide:virtel-version-4-57-1st-jul-2017}}
\sphinxAtStartPar
\sphinxstyleemphasis{VIRTEL Web Access:}
\begin{itemize}
\item {} 
\sphinxAtStartPar
Bidirectional presentation support.

\item {} 
\sphinxAtStartPar
Enhanced font stretch mode. Optimization of screen size.

\item {} 
\sphinxAtStartPar
Additional RACHECK support for ForceLUNAME

\item {} 
\sphinxAtStartPar
Licence warning feature

\item {} 
\sphinxAtStartPar
Enhancements to USSMSG10 support module

\item {} 
\sphinxAtStartPar
Enhancements to Virtel Web Macro interface (VWM)

\item {} 
\sphinxAtStartPar
Keyboard mapping enhancements

\item {} 
\sphinxAtStartPar
Enhancements to Virtel Dynamic Directories Interface (DDI)

\item {} 
\sphinxAtStartPar
Hotkey support for DDI macros

\item {} 
\sphinxAtStartPar
New refresh options.

\end{itemize}

\sphinxAtStartPar
\sphinxstyleemphasis{VIRTEL Web Modernisation \& Integration:}
\begin{itemize}
\item {} 
\sphinxAtStartPar
Enhancements to COPY\$ NAME\sphinxhyphen{}OF. Support for TERMINAL, GROUP and RELAY items.

\item {} 
\sphinxAtStartPar
Enhancements to DEFAULT\sphinxhyphen{}FILED\sphinxhyphen{}WITH\sphinxhyphen{}CURSOR statement.

\end{itemize}

\sphinxAtStartPar
\sphinxstyleemphasis{Miscellaneous:}
\begin{itemize}
\item {} 
\sphinxAtStartPar
Customizable HELP solution.

\item {} 
\sphinxAtStartPar
DNS access to resolve IP address or DNS name.

\item {} 
\sphinxAtStartPar
TCT option to support mixed case passwords

\item {} 
\sphinxAtStartPar
Additional language support

\item {} 
\sphinxAtStartPar
Batch export/import of RAW TRSF files.

\end{itemize}

\begin{sphinxadmonition}{note}{Note:}
\sphinxAtStartPar
For further details see the Virtel Technical Newsletter TN201706: Whats new in Virtel 4.57.
\end{sphinxadmonition}


\section{Virtel version 4.56 (21 Jun 2016)}
\label{\detokenize{Installation_Guide:virtel-version-4-56-21-jun-2016}}
\sphinxAtStartPar
\sphinxstyleemphasis{VIRTEL Web Access:}
\begin{itemize}
\item {} 
\sphinxAtStartPar
VIRPLEX VSAMTYP=WRITER support in TCT.

\item {} 
\sphinxAtStartPar
Enhancements to HTML Administration interface

\item {} 
\sphinxAtStartPar
Enhancements to Virtel Web Macro interface (VWM)

\item {} 
\sphinxAtStartPar
Keyboard mapping

\item {} 
\sphinxAtStartPar
Enhancements to Virtel Dynamic Directories Interface (DDI)

\item {} 
\sphinxAtStartPar
Synchronisation with VWM

\end{itemize}

\sphinxAtStartPar
\sphinxstyleemphasis{Miscellaneous:}
\begin{itemize}
\item {} 
\sphinxAtStartPar
LSR and WTO corrections.

\item {} 
\sphinxAtStartPar
TCT MAXSOC defaults to 1000.

\end{itemize}

\begin{sphinxadmonition}{note}{Note:}
\sphinxAtStartPar
For further details see the Virtel Technical Newsletter TN201607: Whats new in Virtel 4.56.
\end{sphinxadmonition}


\section{Virtel version 4.55 (31 Mar 2015)}
\label{\detokenize{Installation_Guide:virtel-version-4-55-31-mar-2015}}
\sphinxAtStartPar
\sphinxstyleemphasis{VIRTEL Web Access:}
\begin{itemize}
\item {} 
\sphinxAtStartPar
New toolbar button to toggle the 3278T mode for APL and TEXT conversion.

\end{itemize}

\sphinxAtStartPar
\sphinxstyleemphasis{VIRTEL Web Modernisation \& Integration:}
\begin{itemize}
\item {} 
\sphinxAtStartPar
‘QUICKLNK’ supports multiple containers.

\item {} 
\sphinxAtStartPar
VIRPLEX ‘SHRHTTP’ line type.

\item {} 
\sphinxAtStartPar
VIRPLEX communication.

\item {} 
\sphinxAtStartPar
Enhancements to RULE processing \sphinxhyphen{} \$REJECT\$ parameter.

\item {} 
\sphinxAtStartPar
Transaction SCRIPT command enhancements

\item {} 
\sphinxAtStartPar
Additional commands for scenarios:

\item {} 
\sphinxAtStartPar
COPY\$ supports column to variable with TYPE=REP

\item {} 
\sphinxAtStartPar
Screen positioning support for SET\$, CASE\$ and COPY\$

\item {} 
\sphinxAtStartPar
MAP\$ JSON support for COMMAREA

\item {} 
\sphinxAtStartPar
Allow embedded OCCURS=UNLIMITED keyword

\item {} 
\sphinxAtStartPar
MAP\$ FROM\sphinxhyphen{}INPUT TO\sphinxhyphen{}VARIABLES converts XML or JSON directlt to variables.

\item {} 
\sphinxAtStartPar
Allow re\sphinxhyphen{}execution of a MAP\$ FROM\sphinxhyphen{}INPUT.

\item {} 
\sphinxAtStartPar
CASE\$ and COPY\$ support RTRIM and LTRIM

\item {} 
\sphinxAtStartPar
OUTPUT\sphinxhyphen{}FILE\sphinxhyphen{}TO\sphinxhyphen{}VARIABLE supports a variable as input.

\item {} 
\sphinxAtStartPar
COPY\$ VARIABLE\sphinxhyphen{}TO\sphinxhyphen{}VARIABLE copies source to target variable.

\item {} 
\sphinxAtStartPar
COPY\$ CURRENT\sphinxhyphen{}LINE copies SCREEN\sphinxhyphen{}POSITION ifield values to a variable.

\item {} 
\sphinxAtStartPar
APPLICATON\sphinxhyphen{}OPTION support.

\item {} 
\sphinxAtStartPar
SCENARIO FAIL | SUCCESS parameter.

\end{itemize}

\sphinxAtStartPar
\sphinxstyleemphasis{Miscellaneous:}
\begin{itemize}
\item {} 
\sphinxAtStartPar
New UNLOAD command to unload ARBO file.

\item {} 
\sphinxAtStartPar
VIRSV maintenance.

\item {} 
\sphinxAtStartPar
MEMDISPLAY \sphinxhyphen{} Memory display diagnostic command.

\item {} 
\sphinxAtStartPar
Virtel displays customer USSTAB MSG10

\item {} 
\sphinxAtStartPar
LOGGER stream and structure name set in TCT

\item {} 
\sphinxAtStartPar
SNAPMSG command

\item {} 
\sphinxAtStartPar
LOGGER extraction utility

\item {} 
\sphinxAtStartPar
Override ARBO IP and \&SYSCLONE values in JCL PARM

\item {} 
\sphinxAtStartPar
Trace offload facility

\item {} 
\sphinxAtStartPar
DOC directory added for online help templates.

\item {} 
\sphinxAtStartPar
New Virtel logo.

\item {} 
\sphinxAtStartPar
Critical dataset VIRSWAP error will terminate VIRTEL.

\item {} 
\sphinxAtStartPar
TCT Compatibility mode. Downward compatibility support.

\end{itemize}

\sphinxAtStartPar
\sphinxstyleemphasis{MQSeries:}
\begin{itemize}
\item {} 
\sphinxAtStartPar
TCT additional MQ3 and MQ4 keywords.

\end{itemize}

\begin{sphinxadmonition}{note}{Note:}
\sphinxAtStartPar
For further details see the Virtel Technical Newsletter TN201522: Whats new in Virtel 4.55.
\end{sphinxadmonition}


\section{Virtel version 4.54 (05 Mar 2015)}
\label{\detokenize{Installation_Guide:virtel-version-4-54-05-mar-2015}}
\sphinxAtStartPar
\sphinxstyleemphasis{VIRTEL Universal Protocol:}
\begin{itemize}
\item {} 
\sphinxAtStartPar
TPIPE now supports multiple ICONNECT lines

\end{itemize}

\sphinxAtStartPar
\sphinxstyleemphasis{VIRTEL Web Access:}
\begin{itemize}
\item {} 
\sphinxAtStartPar
Support Query List variant of 3270 Read Partition command \sphinxhyphen{} Extended Color.

\item {} 
\sphinxAtStartPar
Display Virtel update level in tool bar.

\item {} 
\sphinxAtStartPar
New ICON in Copy/Past menu to show Firefox/Chrome extension status.

\item {} 
\sphinxAtStartPar
Enahancements to Virtel Web Macro support (VWM)

\item {} 
\sphinxAtStartPar
Enhancements to toolbar styling.

\item {} 
\sphinxAtStartPar
Support graphics in input fields in 3278T mode

\item {} 
\sphinxAtStartPar
Language support enhancements \sphinxhyphen{} German \& French.

\item {} 
\sphinxAtStartPar
Enhancements to Virtel Dynamic Directories (DDI)

\end{itemize}

\sphinxAtStartPar
\sphinxstyleemphasis{VIRTEL Web Modernisation \& Integration:}
\begin{itemize}
\item {} 
\sphinxAtStartPar
OPTION\$ statement now supports ‘QUICKLNK’ line type.

\item {} 
\sphinxAtStartPar
Support of VTA protocol through ‘QUICKLNK’ line type.

\item {} 
\sphinxAtStartPar
Support of VTA CONTAINERS

\item {} 
\sphinxAtStartPar
Additional commands for scenarios:

\item {} 
\sphinxAtStartPar
ACTION\$ now allows ‘Program Tab’ key to invoke scenario

\item {} 
\sphinxAtStartPar
COPY\$ SYSTEM\sphinxhyphen{}TO\sphinxhyphen{}VARIABLE now supports TYPE=REPLACE

\item {} 
\sphinxAtStartPar
MAP\$ instruction now supports positive/negative sign

\item {} 
\sphinxAtStartPar
MAP\$ instruction supports decimal point for FORMAT TYPE=S9

\item {} 
\sphinxAtStartPar
New template instruction IF\sphinxhyphen{}SOUND\sphinxhyphen{}ALARM\sphinxhyphen{}IS\sphinxhyphen{}REQUESTED

\end{itemize}

\sphinxAtStartPar
\sphinxstyleemphasis{Miscellaneous:}
\begin{itemize}
\item {} 
\sphinxAtStartPar
Support for ‘Above the bar’ 2GB storage for Virtel variables.

\item {} 
\sphinxAtStartPar
Allow Log datasets to be spooled/spun off to JES2.

\item {} 
\sphinxAtStartPar
TCT command now displays the LOG option.

\item {} 
\sphinxAtStartPar
Virtel installation now includes VTG.

\item {} 
\sphinxAtStartPar
IND\$FILE upload and download limit can be set in new TCT options.

\item {} 
\sphinxAtStartPar
VIRSV maintenance.

\end{itemize}

\sphinxAtStartPar
\sphinxstyleemphasis{MQSeries:}

\sphinxAtStartPar
‘TRAN’ parameter can be specified on MQ Line definition \sphinxhyphen{} Character translation.


\section{Virtel version 4.53 (24 Sep 2014)}
\label{\detokenize{Installation_Guide:virtel-version-4-53-24-sep-2014}}
\sphinxAtStartPar
\sphinxstyleemphasis{VIRTEL Web Access:}
\begin{itemize}
\item {} 
\sphinxAtStartPar
Allow FR\sphinxhyphen{}850 charset for IND\$FILE transfer

\item {} 
\sphinxAtStartPar
Limit IND\$FILE file upload to HTPARM(2) parameter value

\item {} 
\sphinxAtStartPar
Macro management enhancements for multi files synchronisation

\item {} 
\sphinxAtStartPar
Outsourcing of all messages for multi\sphinxhyphen{}language support

\item {} 
\sphinxAtStartPar
Support of cut and paste in I\sphinxhyphen{}Frame mode for Firefox, and Chrome

\item {} 
\sphinxAtStartPar
Additional HTML tags:

\item {} 
\sphinxAtStartPar
VALUE\sphinxhyphen{}OF URL and QUERY

\end{itemize}

\sphinxAtStartPar
\sphinxstyleemphasis{VIRTEL Web Integration:}
\begin{itemize}
\item {} 
\sphinxAtStartPar
Additional commands for scenarios:

\item {} 
\sphinxAtStartPar
ACTION\$ REFRESH\sphinxhyphen{}TERMINAL

\item {} 
\sphinxAtStartPar
HANDLE\$ ABEND

\item {} 
\sphinxAtStartPar
HANDLE\$ LOOP

\item {} 
\sphinxAtStartPar
PERFORM\$ subroutine

\item {} 
\sphinxAtStartPar
LABEL\$

\end{itemize}

\sphinxAtStartPar
\sphinxstyleemphasis{Miscellaneous:}
\begin{itemize}
\item {} 
\sphinxAtStartPar
Increased CISIZE for the data portion of the TRSF’s files

\item {} 
\sphinxAtStartPar
Modification of SHR options for ARBO file

\item {} 
\sphinxAtStartPar
SMF support for statistics records

\item {} 
\sphinxAtStartPar
Support of multi lines for WTO

\item {} 
\sphinxAtStartPar
TRACTIM=CPU CPU time in snap

\item {} 
\sphinxAtStartPar
Allows to specify a country code in the VIRTCT VIRSV1= second
sub\sphinxhyphen{}parameter

\item {} 
\sphinxAtStartPar
Allows BLDVRP hiperspace option in VIRTCT

\item {} 
\sphinxAtStartPar
Miscellaneous editorial corrections and enhancements

\item {} 
\sphinxAtStartPar
Additional system commands:
\begin{itemize}
\item {} 
\sphinxAtStartPar
SNAPMSG command to trap VIRHT6xE message

\end{itemize}

\end{itemize}


\section{Virtel version 4.52 (4 Dec 2013)}
\label{\detokenize{Installation_Guide:virtel-version-4-52-4-dec-2013}}
\sphinxAtStartPar
\sphinxstyleemphasis{VIRTEL Universal Protocol:}
\begin{itemize}
\item {} 
\sphinxAtStartPar
OTMAPRM parameter to specify security parameters for RESUME TPIPE

\item {} 
\sphinxAtStartPar
Modifiable exit name for RESUME TPIPE

\item {} 
\sphinxAtStartPar
Selection of transaction name for RESUME TPIPE

\end{itemize}

\sphinxAtStartPar
\sphinxstyleemphasis{VIRTEL Web Access:}
\begin{itemize}
\item {} 
\sphinxAtStartPar
Macro management enhancements

\item {} 
\sphinxAtStartPar
Size limit for IND\$FILE upload

\item {} 
\sphinxAtStartPar
Support for BLINK attribute in IE8+, Firefox, and Chrome

\item {} 
\sphinxAtStartPar
Additional HTML tags:

\item {} 
\sphinxAtStartPar
VALUE\sphinxhyphen{}OF URL and QUERY

\end{itemize}

\sphinxAtStartPar
\sphinxstyleemphasis{VIRTEL Web Modernisation:}
\begin{itemize}
\item {} 
\sphinxAtStartPar
Timeout specifiable for SEND\$ TO\sphinxhyphen{}LINE

\item {} 
\sphinxAtStartPar
Additional commands for scenarios:

\item {} 
\sphinxAtStartPar
COPY\$ SYSTEM\sphinxhyphen{}TO\sphinxhyphen{}VARIABLE URL and QUERY

\item {} 
\sphinxAtStartPar
SEND\$ MAXTIME

\end{itemize}

\sphinxAtStartPar
\sphinxstyleemphasis{Miscellaneous:}
\begin{itemize}
\item {} 
\sphinxAtStartPar
VIRCONF documentation in English

\item {} 
\sphinxAtStartPar
Modernised configuration interface

\item {} 
\sphinxAtStartPar
Support for pre\sphinxhyphen{}zSeries processors

\item {} 
\sphinxAtStartPar
Unique identification for VIRTEL startup message

\item {} 
\sphinxAtStartPar
Customizable VIRTEL application name for RACF (RACAPPL)

\item {} 
\sphinxAtStartPar
Miscellaneous editorial corrections and enhancements

\end{itemize}


\section{Virtel version 4.51 (18 Jul 2013)}
\label{\detokenize{Installation_Guide:virtel-version-4-51-18-jul-2013}}
\sphinxAtStartPar
\sphinxstyleemphasis{VIRTEL Universal Protocol:}
\begin{itemize}
\item {} 
\sphinxAtStartPar
Protocol ICONNECT (RESUME TPIPE) for communication with IMS

\end{itemize}

\sphinxAtStartPar
\sphinxstyleemphasis{VIRTEL Web Access:}
\begin{itemize}
\item {} 
\sphinxAtStartPar
Additional HTML tags:

\item {} 
\sphinxAtStartPar
NAME\sphinxhyphen{}OF VIRTEL\sphinxhyphen{}VERSION

\end{itemize}

\sphinxAtStartPar
\sphinxstyleemphasis{VIRTEL Web Modernisation:}
\begin{itemize}
\item {} 
\sphinxAtStartPar
Additional commands for scenarios:

\item {} 
\sphinxAtStartPar
COPY\$ SYSTEM\sphinxhyphen{}TO\sphinxhyphen{}VARIABLE VIRTEL\sphinxhyphen{}VERSION

\end{itemize}

\sphinxAtStartPar
\sphinxstyleemphasis{Miscellaneous:}
\begin{itemize}
\item {} 
\sphinxAtStartPar
Open and close printers on demand

\item {} 
\sphinxAtStartPar
Repeating terminal definitions in hexadecimal and alphanumeric

\item {} 
\sphinxAtStartPar
Display printer names by F VIRTEL,L=line,D command

\item {} 
\sphinxAtStartPar
New command F VIRTEL,RELAYS

\item {} 
\sphinxAtStartPar
Miscellaneous editorial corrections and enhancements

\end{itemize}


\section{Virtel version 4.50 (30 Jun 2013)}
\label{\detokenize{Installation_Guide:virtel-version-4-50-30-jun-2013}}
\sphinxAtStartPar
\sphinxstyleemphasis{VIRTEL Web Access:}
\begin{itemize}
\item {} 
\sphinxAtStartPar
Passticket support

\item {} 
\sphinxAtStartPar
New Web Access Settings:

\item {} 
\sphinxAtStartPar
Shift+Enter, Ctrl+Enter

\item {} 
\sphinxAtStartPar
Line spacing, Character spacing

\item {} 
\sphinxAtStartPar
Ctrl\sphinxhyphen{}A to mark whole 3270 screen for copy

\item {} 
\sphinxAtStartPar
Support for 3270 Graphic Escape characters

\item {} 
\sphinxAtStartPar
Downloadable fonts

\item {} 
\sphinxAtStartPar
Print SCS\sphinxhyphen{}to\sphinxhyphen{}PDF

\item {} 
\sphinxAtStartPar
Query support for SCS printers

\item {} 
\sphinxAtStartPar
Printer autoconnect

\item {} 
\sphinxAtStartPar
Miscellaneous ergonomic enhancements

\end{itemize}

\sphinxAtStartPar
\sphinxstyleemphasis{VIRTEL Web Modernisation:}
\begin{itemize}
\item {} 
\sphinxAtStartPar
Additional commands for scenarios:

\item {} 
\sphinxAtStartPar
CONVERT\$ EBCDIC\sphinxhyphen{}TO\sphinxhyphen{}UPPERCASE, EBCDIC\sphinxhyphen{}TO\sphinxhyphen{}LOWERCASE

\item {} 
\sphinxAtStartPar
COPY\$ PASSTICKET

\end{itemize}

\sphinxAtStartPar
\sphinxstyleemphasis{MQSeries:}
\begin{itemize}
\item {} 
\sphinxAtStartPar
Unique correlator for MQ requests

\item {} 
\sphinxAtStartPar
Message type REPLY for MQ responses

\end{itemize}

\sphinxAtStartPar
\sphinxstyleemphasis{Miscellaneous:}
\begin{itemize}
\item {} 
\sphinxAtStartPar
Miscellaneous editorial corrections and enhancements

\end{itemize}


\section{Virtel version 4.49 (30 Apr 2013)}
\label{\detokenize{Installation_Guide:virtel-version-4-49-30-apr-2013}}
\sphinxAtStartPar
\sphinxstyleemphasis{VIRTEL Web Access:}
\begin{itemize}
\item {} 
\sphinxAtStartPar
Codepages 0037 and 1047 included as standard

\item {} 
\sphinxAtStartPar
New Web Access Settings:

\item {} 
\sphinxAtStartPar
Adapt font size ratio

\item {} 
\sphinxAtStartPar
Additional keyboard remappings for Alt+Ins, Alt+Home, and Alt+F1

\item {} 
\sphinxAtStartPar
ChgCur key to change cursor shape

\item {} 
\sphinxAtStartPar
Settings page in French and German

\item {} 
\sphinxAtStartPar
Administrator can hide specific settings

\item {} 
\sphinxAtStartPar
Additional HTML tags:

\item {} 
\sphinxAtStartPar
CREATE\sphinxhyphen{}VARIABLE\sphinxhyphen{}IF TRACING\sphinxhyphen{}SCENARIO

\item {} 
\sphinxAtStartPar
SET\sphinxhyphen{}HEADER

\item {} 
\sphinxAtStartPar
Allow Virtel session code to be stored in cookie

\item {} 
\sphinxAtStartPar
Custom hotspot recognition

\item {} 
\sphinxAtStartPar
Custom logo for Web Access and Application menus

\item {} 
\sphinxAtStartPar
Drag and drop upload summary report

\item {} 
\sphinxAtStartPar
Miscellaneous ergonomic enhancements

\end{itemize}

\sphinxAtStartPar
\sphinxstyleemphasis{VIRTEL Web Modernisation:}
\begin{itemize}
\item {} 
\sphinxAtStartPar
Additional commands for scenarios:

\item {} 
\sphinxAtStartPar
COPY\$ SYSTEM\sphinxhyphen{}TO\sphinxhyphen{}VARIABLE USER\sphinxhyphen{}SECURITY\sphinxhyphen{}PROFILE

\item {} 
\sphinxAtStartPar
DEBUG\$

\item {} 
\sphinxAtStartPar
SET\$ SIGNON

\item {} 
\sphinxAtStartPar
VIRSV\$ TRANSACTION OPTION=CLOSE

\end{itemize}

\sphinxAtStartPar
\sphinxstyleemphasis{SYSPLEX support:}
\begin{itemize}
\item {} 
\sphinxAtStartPar
Allow read\sphinxhyphen{}only sharing of VSAM files

\end{itemize}

\sphinxAtStartPar
\sphinxstyleemphasis{Miscellaneous:}
\begin{itemize}
\item {} 
\sphinxAtStartPar
Miscellaneous editorial corrections and enhancements

\end{itemize}


\section{Virtel version 4.48 (27 Nov 2012)}
\label{\detokenize{Installation_Guide:virtel-version-4-48-27-nov-2012}}
\sphinxAtStartPar
\sphinxstyleemphasis{VIRTEL Universal Protocol:}
\begin{itemize}
\item {} 
\sphinxAtStartPar
Menu program VIR0021J

\end{itemize}

\sphinxAtStartPar
\sphinxstyleemphasis{VIRTEL Web Access:}
\begin{itemize}
\item {} 
\sphinxAtStartPar
Connection of non\sphinxhyphen{}predefined VTAM LU names

\item {} 
\sphinxAtStartPar
Support for expired passwords

\item {} 
\sphinxAtStartPar
Site\sphinxhyphen{}specific Javascript (custom.js)

\item {} 
\sphinxAtStartPar
REALM parameter to reduce multiple signon

\item {} 
\sphinxAtStartPar
Codepage 1047 for C programming

\item {} 
\sphinxAtStartPar
Codepage override by URL

\item {} 
\sphinxAtStartPar
New Web Access Settings:

\item {} 
\sphinxAtStartPar
End key

\item {} 
\sphinxAtStartPar
Remap caret to logical not

\item {} 
\sphinxAtStartPar
Additional HTML tags:

\item {} 
\sphinxAtStartPar
CREATE\sphinxhyphen{}VARIABLE\sphinxhyphen{}FROM (allows a rectangle to be copied from the 3270 screen)

\item {} 
\sphinxAtStartPar
Miscellaneous ergonomic enhancements

\end{itemize}

\sphinxAtStartPar
\sphinxstyleemphasis{VIRTEL Web Modernisation}
\begin{itemize}
\item {} 
\sphinxAtStartPar
Support for scenarios stored in VSAM

\item {} 
\sphinxAtStartPar
Additional commands for scenarios:

\item {} 
\sphinxAtStartPar
ACTION\$ PROCESS\sphinxhyphen{}RESPONSE

\item {} 
\sphinxAtStartPar
COPY\$ VARIABLE\sphinxhyphen{}TO\sphinxhyphen{}SYSTEM PASSWORD

\item {} 
\sphinxAtStartPar
IF\$ SCREEN\sphinxhyphen{}IS\sphinxhyphen{}BLANK

\item {} 
\sphinxAtStartPar
IF\$ SCREEN\sphinxhyphen{}IS\sphinxhyphen{}UNFORMATTED

\end{itemize}

\sphinxAtStartPar
\sphinxstyleemphasis{VIRTEL Web Integration}
\begin{itemize}
\item {} 
\sphinxAtStartPar
New programs CALL VIRSETAI, VIRSETVI for IMS SYSPLEX support:

\item {} 
\sphinxAtStartPar
SYSCLONE parameter in LU names

\item {} 
\sphinxAtStartPar
Dynamic VIRTCT overrides Miscellaneous:

\item {} 
\sphinxAtStartPar
Restart VIRSV service by console command

\item {} 
\sphinxAtStartPar
Patch application via the VIRTCT

\item {} 
\sphinxAtStartPar
Miscellaneous editorial corrections and enhancements

\end{itemize}


\section{Virtel version 4.47 (21 May 2012)}
\label{\detokenize{Installation_Guide:virtel-version-4-47-21-may-2012}}
\sphinxAtStartPar
\sphinxstyleemphasis{VIRTEL Universal Protocol}
\begin{itemize}
\item {} 
\sphinxAtStartPar
Native TCP/IP protocol

\end{itemize}

\sphinxAtStartPar
\sphinxstyleemphasis{VIRTEL Web Access}
\begin{itemize}
\item {} 
\sphinxAtStartPar
Support for 3270 FieldMark and Dup characters

\item {} 
\sphinxAtStartPar
Keystroke buffering

\item {} 
\sphinxAtStartPar
New Web Access Settings:

\item {} 
\sphinxAtStartPar
Move cursor on activate

\item {} 
\sphinxAtStartPar
Omit nulls from input

\item {} 
\sphinxAtStartPar
Select word by double\sphinxhyphen{}click

\item {} 
\sphinxAtStartPar
Highlight input fields

\item {} 
\sphinxAtStartPar
Keep keypad and macro pad open

\item {} 
\sphinxAtStartPar
Field mark and Dup

\item {} 
\sphinxAtStartPar
Additional HTML tags:

\item {} 
\sphinxAtStartPar
DELETE\sphinxhyphen{}ALL\sphinxhyphen{}VARIABLES (allows the variable pool to be reset from within page template)

\item {} 
\sphinxAtStartPar
SET\sphinxhyphen{}INITIAL\sphinxhyphen{}TIMEOUT (allows an initial timeout to be specified within a page)

\item {} 
\sphinxAtStartPar
SET\sphinxhyphen{}LOCAL\sphinxhyphen{}OPTIONS JSON\sphinxhyphen{}ESCAPES (allows generation of JSON page templates)

\item {} 
\sphinxAtStartPar
SET\sphinxhyphen{}LOCAL\sphinxhyphen{}OPTIONS TRACE\sphinxhyphen{}LINE, TRACE\sphinxhyphen{}RELAY (allows traces to be activated from a page template)

\item {} 
\sphinxAtStartPar
Upload by drag and drop (Chrome only)

\item {} 
\sphinxAtStartPar
Save and restore file transfer parameters

\item {} 
\sphinxAtStartPar
Long polling reduces load on IP network

\item {} 
\sphinxAtStartPar
Customizable user help page for Web Access

\item {} 
\sphinxAtStartPar
Site customization of colors and logo (custom.css)

\item {} 
\sphinxAtStartPar
Miscellaneous ergonomic enhancements

\end{itemize}

\sphinxAtStartPar
\sphinxstyleemphasis{VIRTEL Web Modernisation}
\begin{itemize}
\item {} 
\sphinxAtStartPar
Screen redesigner upgrade

\item {} 
\sphinxAtStartPar
Additional commands for scenarios:

\item {} 
\sphinxAtStartPar
CASE\$ VARIABLE

\item {} 
\sphinxAtStartPar
COPY\$ LIST\sphinxhyphen{}TO\sphinxhyphen{}VARIABLE

\item {} 
\sphinxAtStartPar
COPY\$ VARIABLE\sphinxhyphen{}TO\sphinxhyphen{}SYSTEM NAME\sphinxhyphen{}OF DIRECTORY

\item {} 
\sphinxAtStartPar
FILTER\$ VARIABLE\sphinxhyphen{}TO\sphinxhyphen{}VARIABLE

\item {} 
\sphinxAtStartPar
MAP\$ EXECUTE and RETURN\$

\end{itemize}

\sphinxAtStartPar
\sphinxstyleemphasis{Miscellaneous}
\begin{itemize}
\item {} 
\sphinxAtStartPar
Logon screen for VTAM applications

\item {} 
\sphinxAtStartPar
Latin\sphinxhyphen{}2 translate tables for Eastern European countries

\item {} 
\sphinxAtStartPar
Miscellaneous editorial corrections and enhancements

\end{itemize}


\chapter{Introduction}
\label{\detokenize{Installation_Guide:introduction}}\label{\detokenize{Installation_Guide:virtel461ig-introduction}}
\index{Virtel operating environments@\spxentry{Virtel operating environments}}\ignorespaces 

\section{Required Environment}
\label{\detokenize{Installation_Guide:required-environment}}\label{\detokenize{Installation_Guide:index-0}}
\sphinxAtStartPar
VIRTEL operates in the z/OS or z/VSE environments. Throughout the VIRTEL documentation, the term “z/OS” should be understood to include OS/390 and z/OS, and the term “z/VSE” should be understood to include z/VSE/ESA and z/VSE.


\subsection{z/OS environment}
\label{\detokenize{Installation_Guide:z-os-environment}}
\sphinxAtStartPar
In the z/OS environment, VIRTEL runs under the OS/390 or z/OS operating systems. If the VIRTEL MQ interface is used, then MQSeries Version 6 or later is required. Support for the cryptographic functions of VIRTEL requires ICSF Version HCR7740 or later.


\subsection{z/VSE environment}
\label{\detokenize{Installation_Guide:z-vse-environment}}
\sphinxAtStartPar
In the z/VSE environment, VIRTEL runs under the z/VSE/ESA or z/VSE operating systems. TCP/IP access (XOT, VIRTEL Web Access) requires z/VSE/ESA 2.5.1 or later, or any version of z/VSE.

\index{Supported browser environments@\spxentry{Supported browser environments}}\ignorespaces 

\subsection{Browser pre\sphinxhyphen{}requisities}
\label{\detokenize{Installation_Guide:browser-pre-requisities}}\label{\detokenize{Installation_Guide:index-1}}
\sphinxAtStartPar
VIRTEL Web Access requires a standard web browser on the user’s workstation. Supported browsers include:
\begin{itemize}
\item {} 
\sphinxAtStartPar
Microsoft Edge (For Windows 10)

\item {} 
\sphinxAtStartPar
Microsoft Internet Explorer Version 8 or above

\item {} 
\sphinxAtStartPar
Firefox Version 15 or above (for Windows 7 or Vista)

\item {} 
\sphinxAtStartPar
Firefox Version 17 or above (for Windows XP)

\item {} 
\sphinxAtStartPar
Chrome Version 23 or above

\item {} 
\sphinxAtStartPar
Opera Version 15 or above

\item {} 
\sphinxAtStartPar
Safari Version 5 or above

\end{itemize}

\sphinxAtStartPar
VIRTEL Web Access requires JavaScript to be enabled in the browser.

\phantomsection\label{\detokenize{Installation_Guide:virtel461ig-installz-os}}
\index{Installing under z/OS@\spxentry{Installing under z/OS}}\ignorespaces 

\chapter{Installing VIRTEL under z/OS}
\label{\detokenize{Installation_Guide:installing-virtel-under-z-os}}\label{\detokenize{Installation_Guide:index-2}}

\section{Installation procedure}
\label{\detokenize{Installation_Guide:installation-procedure}}
\sphinxAtStartPar
In the z/OS environment, VIRTEL is delivered as a zipped XMIT file containing the VIRTEL datasets compressed in DF/DSS dump format. The following sections provide details of the installation method. A quick “installation check\sphinxhyphen{}list” to start the VIRTEL installation procedure follows. For customers who do not have DF/DFSS, a non\sphinxhyphen{}DF/DFSS install package is available on request. Contact Syspertec Support for further details.

\index{Installing under z/OS@\spxentry{Installing under z/OS}!z/OS Check List.@\spxentry{z/OS Check List.}}\index{z/OS Check List.@\spxentry{z/OS Check List.}!Installing under z/OS@\spxentry{Installing under z/OS}}\ignorespaces 

\subsection{z/OS Installation Check\sphinxhyphen{}list}
\label{\detokenize{Installation_Guide:z-os-installation-check-list}}\label{\detokenize{Installation_Guide:index-3}}
\sphinxAtStartPar
Here is a standard “check\sphinxhyphen{}list” to start the WEB to HOST VIRTEL function:

\begin{sphinxadmonition}{warning}{Warning:}
\sphinxAtStartPar
If you are upgrading to a new version of Virtel, plan to move any user transactions or customisations out of W2H\sphinxhyphen{}DIR as this directory will be overriden as part of the installation process.
\end{sphinxadmonition}
\begin{enumerate}
\sphinxsetlistlabels{\arabic}{enumi}{enumii}{}{.}%
\item {} 
\sphinxAtStartPar
Download the following files from our FTP server \sphinxhref{http://ftp-group.syspertec.com/}{http://ftp\sphinxhyphen{}group.syspertec.com}
\begin{itemize}
\item {} 
\sphinxAtStartPar
Virtelvrrmvs.zip.

\item {} 
\sphinxAtStartPar
allptfs\sphinxhyphen{}mvsvrr.txt if available.

\item {} 
\sphinxAtStartPar
virtelvrrupdtnnnn.zip if available.

\end{itemize}

\item {} 
\sphinxAtStartPar
Run job \$ALOCDSU to create the TRANSFER.XMIT file.

\item {} 
\sphinxAtStartPar
Upload the virtelvrrmvs.xmit file to the TRANSFER.XMIT file in binary mode.

\item {} 
\sphinxAtStartPar
Edit job \$RESTDSU specifying the high\sphinxhyphen{}level qualifiers and SMS or volume serial information for the VIRTEL datasets. Run job \$RESTDSU to create the VIRTEL datasets yourqual.VIRTvrr.xxxxxx

\item {} 
\sphinxAtStartPar
Apply the PTFs in the allptfs\sphinxhyphen{}mvsvrr.txt file using job ZAPJCL in the VIRTEL CNTL library. If this file does not exist, skip this step.

\item {} 
\sphinxAtStartPar
Use the SETPROG APF command to add the VIRTEL LOADLIB to your system APF authorized program library list

\end{enumerate}

\begin{sphinxVerbatim}[commandchars=\\\{\}]
\PYG{n}{SETPROG} \PYG{n}{APF}\PYG{p}{,}\PYG{n}{ADD}\PYG{p}{,}\PYG{n}{DSN}\PYG{o}{=}\PYG{n}{yourqual}\PYG{o}{.}\PYG{n}{VIRTvrr}\PYG{o}{.}\PYG{n}{LOADLIB}\PYG{p}{,}\PYG{n}{VOL}\PYG{o}{=}\PYG{n}{volser}
\end{sphinxVerbatim}
\begin{enumerate}
\sphinxsetlistlabels{\arabic}{enumi}{enumii}{}{.}%
\setcounter{enumi}{6}
\item {} 
\sphinxAtStartPar
Edit member VIRTCT01 in the VIRTEL CNTL library:\sphinxhyphen{}
\begin{enumerate}
\sphinxsetlistlabels{\alph}{enumii}{enumiii}{}{)}%
\item {} 
\sphinxAtStartPar
Set a default APPLID= parameter to the VTAM ACBNAME. You can override this in the VIRTEL PROC using the APPLID= symbolic. Use this to log on to VIRTEL (the suggested default value is APPLID=VIRTEL)

\item {} 
\sphinxAtStartPar
The TCP1= parameter must match the jobname of your z/OS TCP/IP stack (the suggested value TCPIP is usually correct)

\item {} 
\sphinxAtStartPar
If you prefer VIRTEL to display English language panels, then set LANG=’E’

\item {} 
\sphinxAtStartPar
Set the COUNTRY and DEFUTF8 parameters according to your country.

\item {} 
\sphinxAtStartPar
Set the COMPANY ADDR1 ADDR2 LICENCE EXPIRE CODE parameters using the license key supplied to you by Syspertec.

\end{enumerate}

\item {} 
\sphinxAtStartPar
Run the job ASMTCT in the VIRTEL CNTL library to assemble VIRTCT01 into VIRTvrr.LOADLIB.

\item {} 
\sphinxAtStartPar
Edit member ARBOLOAD in the VIRTEL CNTL library:
\begin{enumerate}
\sphinxsetlistlabels{\alph}{enumii}{enumiii}{}{)}%
\item {} 
\sphinxAtStartPar
Change LANG=EN to LANG=FR if French language is desired

\item {} 
\sphinxAtStartPar
Set LOAD= the name of your VIRTEL LOADLIB

\item {} 
\sphinxAtStartPar
Set SAMP= the name of your VIRTEL SAMPLIB

\item {} 
\sphinxAtStartPar
Set ARBO= the name of your VIRTEL ARBO file

\item {} 
\sphinxAtStartPar
Set VTAMLST= the name of a your VIRTEL CNTL library. The job will create a sample VTAMLST member in this library.

\item {} 
\sphinxAtStartPar
CHANGE ALL ‘DBDCCICS’ ‘xxxxxx’ where xxxxxx is the APPLID of your CICS system.

\item {} 
\sphinxAtStartPar
If you changed the APPLID of VIRTEL in step 7 from its default value VIRTEL, then you must also change the ACBNAME= parameter in step VTAMDEF near the end of the ARBOLOAD job. The value of ACBNAME= in ARBOLOAD must match the value of APPLID= in VIRTCT01. Submit the ARBOLOAD job. This creates your VIRTEL configuration (the ARBO file) and a sample VTAMLST member VIRTAPPL.

\end{enumerate}

\end{enumerate}
\begin{quote}

\begin{sphinxadmonition}{note}{Note:}
\sphinxAtStartPar
If you need to rerun the ARBOLOAD job, you must change PARM=’LOAD,NOREPL’ to PARM=’LOAD,REPL’. If you wish to completely start over from the beginning, you can run the job ARBOBASE to delete and reinitialize the ARBO file, followed by a rerun of the ARBOLOAD job.
\end{sphinxadmonition}
\end{quote}
\begin{enumerate}
\sphinxsetlistlabels{\arabic}{enumi}{enumii}{}{.}%
\setcounter{enumi}{9}
\item {} 
\sphinxAtStartPar
If using APPC or X25, submit the job ASMMOD from the VIRTEL CNTL library. This job assembles the VIRTEL logon mode table (MODVIRT) into your SYS1.VTAMLIB dataset. You will need to set the QUAL= parameter to match the high\sphinxhyphen{}level qualifiers of your SAMPLIB dataset.

\item {} 
\sphinxAtStartPar
Copy the VIRTAPPL member (created by the ARBOLOAD job in step 8) from the VIRTEL CNTL library into your SYS1.VTAMLST dataset. Now activate the VTAMLST member using this command:

\end{enumerate}

\begin{sphinxVerbatim}[commandchars=\\\{\}]
\PYG{n}{V} \PYG{n}{NET}\PYG{p}{,}\PYG{n}{ACT}\PYG{p}{,}\PYG{n}{ID}\PYG{o}{=}\PYG{n}{VIRTAPPL}
\end{sphinxVerbatim}
\begin{enumerate}
\sphinxsetlistlabels{\arabic}{enumi}{enumii}{}{.}%
\setcounter{enumi}{11}
\item {} 
\sphinxAtStartPar
Edit the procedure VIRTEL4 in your VIRTEL CNTL library so that the high\sphinxhyphen{}level qualifiers match the names you used when you loaded the files in step 4. Copy the procedure to your system PROCLIB, renaming it as VIRTEL.

\item {} 
\sphinxAtStartPar
Ask your security administrator to create a userid for the VIRTEL started task, and to authorize this userid to access the datasets you created in step 3. This userid must also have an OMVS segment which authorizes VIRTEL to use TCP/IP. Your security administrator can use the job RACFSTC in the VIRTEL SAMPLIB as an example.

\item {} 
\sphinxAtStartPar
Start VIRTEL

\end{enumerate}

\sphinxAtStartPar
You can now logon to VIRTEL from a 3270 terminal using the APPLID specified in the VIRTCT01, and you can display the VIRTEL Web Access menu in your web browser using URL \sphinxurl{http://n.n.n.n:41001} where n.n.n.n is the IP address of your z/OS system.
\begin{enumerate}
\sphinxsetlistlabels{\arabic}{enumi}{enumii}{}{.}%
\setcounter{enumi}{14}
\item {} 
\sphinxAtStartPar
Apply any Virtel Web Access (See VWA maintenance) according the instructions in the Readme\sphinxhyphen{}updtnnnn.txt file included in the zip file. If the zip file does not exist, skip this step. If yo do apply maintenance then refresh the browser (CTRL\sphinxhyphen{}R) after updating the relevant TRSF directories. Check that the updtnnn is the correct number in the Administration Portal Screen.

\item {} 
\sphinxAtStartPar
The supplied system is configured with security disabled. If you wish, you can activate external security using RACF, ACF2, or TOP SECRET; please refer to the {\hyperref[\detokenize{Installation_Guide:vvrrig-security}]{\sphinxcrossref{\DUrole{std,std-ref}{“Security Chapter”}}}}.

\end{enumerate}

\index{Installing under z/OS@\spxentry{Installing under z/OS}!Restoring from the XMIT file@\spxentry{Restoring from the XMIT file}}\index{Restoring from the XMIT file@\spxentry{Restoring from the XMIT file}!Installing under z/OS@\spxentry{Installing under z/OS}}\ignorespaces 

\subsection{Restoring from the XMIT file}
\label{\detokenize{Installation_Guide:restoring-from-the-xmit-file}}\label{\detokenize{Installation_Guide:index-4}}
\sphinxAtStartPar
All the VSAM and non\sphinxhyphen{}VSAM datasets required for the installation of VIRTEL are contained in a zipped XMIT file which can be downloaded from the Syspertec file server. The size of the zipped file is approximately 2MB. Two JCL files (\$ALOCDSU and \$RESTDSU) are also     included in the zip file. The procedure for obtaining and uploading the file is as follows:


\subsubsection{Step 1}
\label{\detokenize{Installation_Guide:step-1}}
\sphinxAtStartPar
Login to the Syspertec file server \sphinxhref{http://ftp-group.syspertec.com/}{http://ftp\sphinxhyphen{}group.syspertec.com} using the userid and password supplied to you by Syspertec. Navigate to the Public directory “VIRTEL \sphinxhyphen{} 4.61\sphinxhyphen{} Products” and download the virtel461mvs.zip file. Unzip this file into a folder on your workstation.

\newpage


\subsubsection{Step 2}
\label{\detokenize{Installation_Guide:step-2}}
\sphinxAtStartPar
Run the job \$ALOCDSU to allocate a sequential file named userid.TRANSFER.XMIT with DCB attributes (RECFM=FB, LRECL=80):

\begin{sphinxVerbatim}[commandchars=\\\{\}]
//SPALODSU JOB 1,MSGCLASS=X,CLASS=A,NOTIFY=\PYGZam{}SYSUID
//*\PYGZhy{}\PYGZhy{}\PYGZhy{}\PYGZhy{}\PYGZhy{}\PYGZhy{}\PYGZhy{}\PYGZhy{}\PYGZhy{}\PYGZhy{}\PYGZhy{}\PYGZhy{}\PYGZhy{}\PYGZhy{}\PYGZhy{}\PYGZhy{}\PYGZhy{}\PYGZhy{}\PYGZhy{}\PYGZhy{}\PYGZhy{}\PYGZhy{}\PYGZhy{}\PYGZhy{}\PYGZhy{}\PYGZhy{}\PYGZhy{}\PYGZhy{}\PYGZhy{}\PYGZhy{}\PYGZhy{}\PYGZhy{}\PYGZhy{}\PYGZhy{}\PYGZhy{}\PYGZhy{}\PYGZhy{}\PYGZhy{}\PYGZhy{}\PYGZhy{}\PYGZhy{}\PYGZhy{}\PYGZhy{}\PYGZhy{}\PYGZhy{}\PYGZhy{}\PYGZhy{}\PYGZhy{}\PYGZhy{}\PYGZhy{}\PYGZhy{}\PYGZhy{}\PYGZhy{}\PYGZhy{}\PYGZhy{}\PYGZhy{}\PYGZhy{}\PYGZhy{}\PYGZhy{}\PYGZhy{}\PYGZhy{}\PYGZhy{}\PYGZhy{}\PYGZhy{}\PYGZhy{}\PYGZhy{}\PYGZhy{}*
//*\PYGZhy{}\PYGZhy{}* BINARY FILE TRANSFER \PYGZhy{} STEP NO.1                            *\PYGZhy{}\PYGZhy{}*
//*\PYGZhy{}\PYGZhy{}*                                                             *\PYGZhy{}\PYGZhy{}*
//*\PYGZhy{}\PYGZhy{}* Function : Allocate a sequential XMIT type file             *\PYGZhy{}\PYGZhy{}*
//*\PYGZhy{}\PYGZhy{}*                                                             *\PYGZhy{}\PYGZhy{}*
//*\PYGZhy{}\PYGZhy{}* Following step \PYGZdl{}RESTDSU                                     *\PYGZhy{}\PYGZhy{}*
//*\PYGZhy{}\PYGZhy{}*                                                             *\PYGZhy{}\PYGZhy{}*
//*\PYGZhy{}\PYGZhy{}\PYGZhy{}\PYGZhy{}\PYGZhy{}\PYGZhy{}\PYGZhy{}\PYGZhy{}\PYGZhy{}\PYGZhy{}\PYGZhy{}\PYGZhy{}\PYGZhy{}\PYGZhy{}\PYGZhy{}\PYGZhy{}\PYGZhy{}\PYGZhy{}\PYGZhy{}\PYGZhy{}\PYGZhy{}\PYGZhy{}\PYGZhy{}\PYGZhy{}\PYGZhy{}\PYGZhy{}\PYGZhy{}\PYGZhy{}\PYGZhy{}\PYGZhy{}\PYGZhy{}\PYGZhy{}\PYGZhy{}\PYGZhy{}\PYGZhy{}\PYGZhy{}\PYGZhy{}\PYGZhy{}\PYGZhy{}\PYGZhy{}\PYGZhy{}\PYGZhy{}\PYGZhy{}\PYGZhy{}\PYGZhy{}\PYGZhy{}\PYGZhy{}\PYGZhy{}\PYGZhy{}\PYGZhy{}\PYGZhy{}\PYGZhy{}\PYGZhy{}\PYGZhy{}\PYGZhy{}\PYGZhy{}\PYGZhy{}\PYGZhy{}\PYGZhy{}\PYGZhy{}\PYGZhy{}\PYGZhy{}\PYGZhy{}\PYGZhy{}\PYGZhy{}\PYGZhy{}\PYGZhy{}*
// SET TYPE=CYL                                        /* TYPE ALLOC */
// SET ALLOCPRI=30                                     /* PRIM ALLOC */
// SET ALLOCSEC=1                                      /* SECO ALLOC */
// SET VOLM=SPT001                                         /* VOLUME */
// SET UNIT=3390                                        /* DISK UNIT */
//*\PYGZhy{}\PYGZhy{}\PYGZhy{}\PYGZhy{}\PYGZhy{}\PYGZhy{}\PYGZhy{}\PYGZhy{}\PYGZhy{}\PYGZhy{}\PYGZhy{}\PYGZhy{}\PYGZhy{}\PYGZhy{}\PYGZhy{}\PYGZhy{}\PYGZhy{}\PYGZhy{}\PYGZhy{}\PYGZhy{}\PYGZhy{}\PYGZhy{}\PYGZhy{}\PYGZhy{}\PYGZhy{}\PYGZhy{}\PYGZhy{}\PYGZhy{}\PYGZhy{}\PYGZhy{}\PYGZhy{}\PYGZhy{}\PYGZhy{}\PYGZhy{}\PYGZhy{}\PYGZhy{}\PYGZhy{}\PYGZhy{}\PYGZhy{}\PYGZhy{}\PYGZhy{}\PYGZhy{}\PYGZhy{}\PYGZhy{}\PYGZhy{}\PYGZhy{}\PYGZhy{}\PYGZhy{}\PYGZhy{}\PYGZhy{}\PYGZhy{}\PYGZhy{}\PYGZhy{}\PYGZhy{}\PYGZhy{}\PYGZhy{}\PYGZhy{}\PYGZhy{}\PYGZhy{}\PYGZhy{}\PYGZhy{}\PYGZhy{}\PYGZhy{}\PYGZhy{}\PYGZhy{}\PYGZhy{}\PYGZhy{}*
//* DELETE OLD .XMIT file                                             *
//*\PYGZhy{}\PYGZhy{}\PYGZhy{}\PYGZhy{}\PYGZhy{}\PYGZhy{}\PYGZhy{}\PYGZhy{}\PYGZhy{}\PYGZhy{}\PYGZhy{}\PYGZhy{}\PYGZhy{}\PYGZhy{}\PYGZhy{}\PYGZhy{}\PYGZhy{}\PYGZhy{}\PYGZhy{}\PYGZhy{}\PYGZhy{}\PYGZhy{}\PYGZhy{}\PYGZhy{}\PYGZhy{}\PYGZhy{}\PYGZhy{}\PYGZhy{}\PYGZhy{}\PYGZhy{}\PYGZhy{}\PYGZhy{}\PYGZhy{}\PYGZhy{}\PYGZhy{}\PYGZhy{}\PYGZhy{}\PYGZhy{}\PYGZhy{}\PYGZhy{}\PYGZhy{}\PYGZhy{}\PYGZhy{}\PYGZhy{}\PYGZhy{}\PYGZhy{}\PYGZhy{}\PYGZhy{}\PYGZhy{}\PYGZhy{}\PYGZhy{}\PYGZhy{}\PYGZhy{}\PYGZhy{}\PYGZhy{}\PYGZhy{}\PYGZhy{}\PYGZhy{}\PYGZhy{}\PYGZhy{}\PYGZhy{}\PYGZhy{}\PYGZhy{}\PYGZhy{}\PYGZhy{}\PYGZhy{}\PYGZhy{}*
//STEP1 EXEC PGM=IKJEFT01,PARM=\PYGZsq{}DEL \PYGZsq{}\PYGZsq{}\PYGZam{}SYSUID..TRANSFER.XMIT\PYGZsq{}\PYGZsq{}\PYGZsq{}
//SYSTSPRT DD SYSOUT=*
//SYSOUT DD *
//SYSTSIN DD *
PROF                                            /* POUR GENERER CC=0 */
//*
//*\PYGZhy{}\PYGZhy{}\PYGZhy{}\PYGZhy{}\PYGZhy{}\PYGZhy{}\PYGZhy{}\PYGZhy{}\PYGZhy{}\PYGZhy{}\PYGZhy{}\PYGZhy{}\PYGZhy{}\PYGZhy{}\PYGZhy{}\PYGZhy{}\PYGZhy{}\PYGZhy{}\PYGZhy{}\PYGZhy{}\PYGZhy{}\PYGZhy{}\PYGZhy{}\PYGZhy{}\PYGZhy{}\PYGZhy{}\PYGZhy{}\PYGZhy{}\PYGZhy{}\PYGZhy{}\PYGZhy{}\PYGZhy{}\PYGZhy{}\PYGZhy{}\PYGZhy{}\PYGZhy{}\PYGZhy{}\PYGZhy{}\PYGZhy{}\PYGZhy{}\PYGZhy{}\PYGZhy{}\PYGZhy{}\PYGZhy{}\PYGZhy{}\PYGZhy{}\PYGZhy{}\PYGZhy{}\PYGZhy{}\PYGZhy{}\PYGZhy{}\PYGZhy{}\PYGZhy{}\PYGZhy{}\PYGZhy{}\PYGZhy{}\PYGZhy{}\PYGZhy{}\PYGZhy{}\PYGZhy{}\PYGZhy{}\PYGZhy{}\PYGZhy{}\PYGZhy{}\PYGZhy{}\PYGZhy{}\PYGZhy{}*
//* Allocate new reception .XMIT file                                 *
//*\PYGZhy{}\PYGZhy{}\PYGZhy{}\PYGZhy{}\PYGZhy{}\PYGZhy{}\PYGZhy{}\PYGZhy{}\PYGZhy{}\PYGZhy{}\PYGZhy{}\PYGZhy{}\PYGZhy{}\PYGZhy{}\PYGZhy{}\PYGZhy{}\PYGZhy{}\PYGZhy{}\PYGZhy{}\PYGZhy{}\PYGZhy{}\PYGZhy{}\PYGZhy{}\PYGZhy{}\PYGZhy{}\PYGZhy{}\PYGZhy{}\PYGZhy{}\PYGZhy{}\PYGZhy{}\PYGZhy{}\PYGZhy{}\PYGZhy{}\PYGZhy{}\PYGZhy{}\PYGZhy{}\PYGZhy{}\PYGZhy{}\PYGZhy{}\PYGZhy{}\PYGZhy{}\PYGZhy{}\PYGZhy{}\PYGZhy{}\PYGZhy{}\PYGZhy{}\PYGZhy{}\PYGZhy{}\PYGZhy{}\PYGZhy{}\PYGZhy{}\PYGZhy{}\PYGZhy{}\PYGZhy{}\PYGZhy{}\PYGZhy{}\PYGZhy{}\PYGZhy{}\PYGZhy{}\PYGZhy{}\PYGZhy{}\PYGZhy{}\PYGZhy{}\PYGZhy{}\PYGZhy{}\PYGZhy{}\PYGZhy{}*
//STEP2 EXEC PGM=IEFBR14
//SYSOUT DD *
//SYSUT2 DD DSN\PYGZam{}SYSUID..TRANSFER.XMIT,
// UNIT=\PYGZam{}UNIT,VOL=SER=\PYGZam{}VOLM,DISP=(NEW,CATLG),
// SPACE=(\PYGZam{}TYPE,(\PYGZam{}ALLOCPRI,\PYGZam{}ALLOCSEC)),
// DCB=(RECFM=FB,LRECL=80,BLKSIZE=3120,DSORG=PS)
//*
//*\PYGZhy{}\PYGZhy{}\PYGZhy{}\PYGZhy{}\PYGZhy{}\PYGZhy{}\PYGZhy{}\PYGZhy{}\PYGZhy{}\PYGZhy{}\PYGZhy{}\PYGZhy{}\PYGZhy{}\PYGZhy{}\PYGZhy{}\PYGZhy{}\PYGZhy{}\PYGZhy{}\PYGZhy{}\PYGZhy{}\PYGZhy{}\PYGZhy{}\PYGZhy{}\PYGZhy{}\PYGZhy{}\PYGZhy{}\PYGZhy{}\PYGZhy{}\PYGZhy{}\PYGZhy{}\PYGZhy{}\PYGZhy{}\PYGZhy{}\PYGZhy{}\PYGZhy{}\PYGZhy{}\PYGZhy{}\PYGZhy{}\PYGZhy{}\PYGZhy{}\PYGZhy{}\PYGZhy{}\PYGZhy{}\PYGZhy{}\PYGZhy{}\PYGZhy{}\PYGZhy{}\PYGZhy{}\PYGZhy{}\PYGZhy{}\PYGZhy{}\PYGZhy{}\PYGZhy{}\PYGZhy{}\PYGZhy{}\PYGZhy{}\PYGZhy{}\PYGZhy{}\PYGZhy{}\PYGZhy{}\PYGZhy{}\PYGZhy{}\PYGZhy{}\PYGZhy{}\PYGZhy{}\PYGZhy{}\PYGZhy{}*
//*\PYGZhy{}\PYGZhy{}* BINARY FILE TRANSFER \PYGZhy{} STEP NO.2                            *\PYGZhy{}\PYGZhy{}*
//*\PYGZhy{}\PYGZhy{}* Make a binary transfer of the given file                    *\PYGZhy{}\PYGZhy{}*
//*\PYGZhy{}\PYGZhy{}* BIN                                                         *\PYGZhy{}\PYGZhy{}*
//*\PYGZhy{}\PYGZhy{}* PUT filename.xmit TRANSFER.XMIT                             *\PYGZhy{}\PYGZhy{}*
//*\PYGZhy{}\PYGZhy{}\PYGZhy{}\PYGZhy{}\PYGZhy{}\PYGZhy{}\PYGZhy{}\PYGZhy{}\PYGZhy{}\PYGZhy{}\PYGZhy{}\PYGZhy{}\PYGZhy{}\PYGZhy{}\PYGZhy{}\PYGZhy{}\PYGZhy{}\PYGZhy{}\PYGZhy{}\PYGZhy{}\PYGZhy{}\PYGZhy{}\PYGZhy{}\PYGZhy{}\PYGZhy{}\PYGZhy{}\PYGZhy{}\PYGZhy{}\PYGZhy{}\PYGZhy{}\PYGZhy{}\PYGZhy{}\PYGZhy{}\PYGZhy{}\PYGZhy{}\PYGZhy{}\PYGZhy{}\PYGZhy{}\PYGZhy{}\PYGZhy{}\PYGZhy{}\PYGZhy{}\PYGZhy{}\PYGZhy{}\PYGZhy{}\PYGZhy{}\PYGZhy{}\PYGZhy{}\PYGZhy{}\PYGZhy{}\PYGZhy{}\PYGZhy{}\PYGZhy{}\PYGZhy{}\PYGZhy{}\PYGZhy{}\PYGZhy{}\PYGZhy{}\PYGZhy{}\PYGZhy{}\PYGZhy{}\PYGZhy{}\PYGZhy{}\PYGZhy{}\PYGZhy{}\PYGZhy{}\PYGZhy{}*
\end{sphinxVerbatim}

\sphinxAtStartPar
\sphinxstyleemphasis{JCL for allocating an XMIT file (z/OS)}

\sphinxAtStartPar
The parameters SET VOLM=SPT001 and SET UNIT=3390 at the start of this job should be changed as appropriate to match the volume on which the userid.TRANSFER.XMIT dataset is to be allocated.

\newpage


\subsubsection{Step 3}
\label{\detokenize{Installation_Guide:step-3}}
\sphinxAtStartPar
Using FTP or IND\$FILE, upload the file virtelvrrmvs.xmit to the host transfer file created in step 1. It is very important to ensure that the upload is performed in binary mode. The following is an example of an FTP session to perform the upload:

\begin{sphinxVerbatim}[commandchars=\\\{\}]
\PYG{n}{C}\PYG{p}{:}\PYGZbs{}\PYG{n}{temp}\PYG{o}{\PYGZgt{}}\PYG{n}{ftp} \PYG{l+m+mf}{192.168}\PYG{l+m+mf}{.0}\PYG{l+m+mf}{.1}
\PYG{n}{Connected} \PYG{n}{to} \PYG{l+m+mf}{192.168}\PYG{l+m+mf}{.0}\PYG{l+m+mf}{.1}\PYG{o}{.}
\PYG{l+m+mi}{220}\PYG{o}{\PYGZhy{}}\PYG{n}{FTPD1} \PYG{n}{IBM} \PYG{n}{FTP} \PYG{n}{CS} \PYG{n}{V1R4} \PYG{n}{at} \PYG{n}{ZOS1}\PYG{o}{.}\PYG{n}{COMPANY}\PYG{o}{.}\PYG{n}{COM}\PYG{p}{,} \PYG{l+m+mi}{08}\PYG{p}{:}\PYG{l+m+mi}{41}\PYG{p}{:}\PYG{l+m+mi}{36} \PYG{n}{on} \PYG{l+m+mi}{2004}\PYG{o}{\PYGZhy{}}\PYG{l+m+mi}{05}\PYG{o}{\PYGZhy{}}\PYG{l+m+mf}{24.}
\PYG{l+m+mi}{220} \PYG{n}{Connection} \PYG{n}{will} \PYG{n}{close} \PYG{k}{if} \PYG{n}{idle} \PYG{k}{for} \PYG{n}{more} \PYG{n}{than} \PYG{l+m+mi}{5} \PYG{n}{minutes}\PYG{o}{.}
\PYG{n}{User} \PYG{p}{(}\PYG{l+m+mf}{192.168}\PYG{l+m+mf}{.0}\PYG{l+m+mf}{.1}\PYG{p}{:}\PYG{p}{(}\PYG{n}{none}\PYG{p}{)}\PYG{p}{)}\PYG{p}{:} \PYG{n}{sptuser}
\PYG{l+m+mi}{331} \PYG{n}{Send} \PYG{n}{password} \PYG{n}{please}\PYG{o}{.}
\PYG{n}{Password}\PYG{p}{:}
\PYG{l+m+mi}{230} \PYG{n}{SPTUSER} \PYG{o+ow}{is} \PYG{n}{logged} \PYG{n}{on}\PYG{o}{.} \PYG{n}{Working} \PYG{n}{directory} \PYG{o+ow}{is} \PYG{l+s+s2}{\PYGZdq{}}\PYG{l+s+s2}{SPTUSER.}\PYG{l+s+s2}{\PYGZdq{}}\PYG{o}{.}
\PYG{n}{ftp}\PYG{o}{\PYGZgt{}} \PYG{n+nb}{bin}
\PYG{l+m+mi}{200} \PYG{n}{Representation} \PYG{n+nb}{type} \PYG{o+ow}{is} \PYG{n}{Image}
\PYG{n}{ftp}\PYG{o}{\PYGZgt{}} \PYG{n}{put} \PYG{n}{virtelvrrmvs}\PYG{o}{.}\PYG{n}{xmit} \PYG{n}{TRANSFER}\PYG{o}{.}\PYG{n}{XMIT}
\PYG{l+m+mi}{200} \PYG{n}{Port} \PYG{n}{request} \PYG{n}{OK}\PYG{o}{.}
\PYG{l+m+mi}{125} \PYG{n}{Storing} \PYG{n}{data} \PYG{n+nb}{set} \PYG{n}{SPTUSER}\PYG{o}{.}\PYG{n}{TRANSFER}\PYG{o}{.}\PYG{n}{XMIT}
\PYG{l+m+mi}{250} \PYG{n}{Transfer} \PYG{n}{completed} \PYG{n}{successfully}\PYG{o}{.}
\PYG{n}{ftp}\PYG{p}{:} \PYG{l+m+mi}{4067120} \PYG{n+nb}{bytes} \PYG{n}{sent} \PYG{o+ow}{in} \PYG{l+m+mi}{5}\PYG{p}{,}\PYG{l+m+mi}{59}\PYG{n}{Seconds} \PYG{l+m+mi}{727}\PYG{p}{,}\PYG{l+m+mi}{83}\PYG{n}{Kbytes}\PYG{o}{/}\PYG{n}{sec}\PYG{o}{.}
\PYG{n}{ftp}\PYG{o}{\PYGZgt{}} \PYG{n}{quit}
\PYG{l+m+mi}{221} \PYG{n}{Quit} \PYG{n}{command} \PYG{n}{received}\PYG{o}{.} \PYG{n}{Goodbye}\PYG{o}{.}
\PYG{n}{C}\PYG{p}{:}\PYGZbs{}\PYG{n}{temp}\PYG{o}{\PYGZgt{}}
\end{sphinxVerbatim}

\sphinxAtStartPar
\sphinxstyleemphasis{FTP session for uploading an XMIT file (z/OS)}


\subsubsection{Step 4}
\label{\detokenize{Installation_Guide:step-4}}
\sphinxAtStartPar
Run the job \$RESTDSU to unpack the transfer file and to restore the VIRTEL datasets by means of the ADRDSSU utility program:

\begin{sphinxVerbatim}[commandchars=\\\{\}]
//SPRESDSU JOB 1,MSGCLASS=X,CLASS=A,NOTIFY=\PYGZam{}SYSUID
//*\PYGZhy{}\PYGZhy{}\PYGZhy{}\PYGZhy{}\PYGZhy{}\PYGZhy{}\PYGZhy{}\PYGZhy{}\PYGZhy{}\PYGZhy{}\PYGZhy{}\PYGZhy{}\PYGZhy{}\PYGZhy{}\PYGZhy{}\PYGZhy{}\PYGZhy{}\PYGZhy{}\PYGZhy{}\PYGZhy{}\PYGZhy{}\PYGZhy{}\PYGZhy{}\PYGZhy{}\PYGZhy{}\PYGZhy{}\PYGZhy{}\PYGZhy{}\PYGZhy{}\PYGZhy{}\PYGZhy{}\PYGZhy{}\PYGZhy{}\PYGZhy{}\PYGZhy{}\PYGZhy{}\PYGZhy{}\PYGZhy{}\PYGZhy{}\PYGZhy{}\PYGZhy{}\PYGZhy{}\PYGZhy{}\PYGZhy{}\PYGZhy{}\PYGZhy{}\PYGZhy{}\PYGZhy{}\PYGZhy{}\PYGZhy{}\PYGZhy{}\PYGZhy{}\PYGZhy{}\PYGZhy{}\PYGZhy{}\PYGZhy{}\PYGZhy{}\PYGZhy{}\PYGZhy{}\PYGZhy{}\PYGZhy{}\PYGZhy{}\PYGZhy{}\PYGZhy{}\PYGZhy{}\PYGZhy{}\PYGZhy{}*
//*\PYGZhy{}\PYGZhy{}* Binary File Transfer \PYGZhy{} STEP No 3 *\PYGZhy{}\PYGZhy{}*
//*\PYGZhy{}\PYGZhy{}* *\PYGZhy{}\PYGZhy{}*
//*\PYGZhy{}\PYGZhy{}* Function : Reception and reload of the files *\PYGZhy{}\PYGZhy{}*
//*\PYGZhy{}\PYGZhy{}* *\PYGZhy{}\PYGZhy{}*
//*\PYGZhy{}\PYGZhy{}* Replace \PYGZsq{}??????\PYGZsq{} by target volume serial number *\PYGZhy{}\PYGZhy{}*
//*\PYGZhy{}\PYGZhy{}* Replace \PYGZsq{}yourqual\PYGZsq{} by target DSN high\PYGZhy{}level qualifier
*\PYGZhy{}\PYGZhy{}*
//*\PYGZhy{}\PYGZhy{}\PYGZhy{}\PYGZhy{}\PYGZhy{}\PYGZhy{}\PYGZhy{}\PYGZhy{}\PYGZhy{}\PYGZhy{}\PYGZhy{}\PYGZhy{}\PYGZhy{}\PYGZhy{}\PYGZhy{}\PYGZhy{}\PYGZhy{}\PYGZhy{}\PYGZhy{}\PYGZhy{}\PYGZhy{}\PYGZhy{}\PYGZhy{}\PYGZhy{}\PYGZhy{}\PYGZhy{}\PYGZhy{}\PYGZhy{}\PYGZhy{}\PYGZhy{}\PYGZhy{}\PYGZhy{}\PYGZhy{}\PYGZhy{}\PYGZhy{}\PYGZhy{}\PYGZhy{}\PYGZhy{}\PYGZhy{}\PYGZhy{}\PYGZhy{}\PYGZhy{}\PYGZhy{}\PYGZhy{}\PYGZhy{}\PYGZhy{}\PYGZhy{}\PYGZhy{}\PYGZhy{}\PYGZhy{}\PYGZhy{}\PYGZhy{}\PYGZhy{}\PYGZhy{}\PYGZhy{}\PYGZhy{}\PYGZhy{}\PYGZhy{}\PYGZhy{}\PYGZhy{}\PYGZhy{}\PYGZhy{}\PYGZhy{}\PYGZhy{}\PYGZhy{}\PYGZhy{}\PYGZhy{}*
//*\PYGZhy{}\PYGZhy{}\PYGZhy{}\PYGZhy{}\PYGZhy{}\PYGZhy{}\PYGZhy{}\PYGZhy{}\PYGZhy{}\PYGZhy{}\PYGZhy{}\PYGZhy{}\PYGZhy{}\PYGZhy{}\PYGZhy{}\PYGZhy{}\PYGZhy{}\PYGZhy{}\PYGZhy{}\PYGZhy{}\PYGZhy{}\PYGZhy{}\PYGZhy{}\PYGZhy{}\PYGZhy{}\PYGZhy{}\PYGZhy{}\PYGZhy{}\PYGZhy{}\PYGZhy{}\PYGZhy{}\PYGZhy{}\PYGZhy{}\PYGZhy{}\PYGZhy{}\PYGZhy{}\PYGZhy{}\PYGZhy{}\PYGZhy{}\PYGZhy{}\PYGZhy{}\PYGZhy{}\PYGZhy{}\PYGZhy{}\PYGZhy{}\PYGZhy{}\PYGZhy{}\PYGZhy{}\PYGZhy{}\PYGZhy{}\PYGZhy{}\PYGZhy{}\PYGZhy{}\PYGZhy{}\PYGZhy{}\PYGZhy{}\PYGZhy{}\PYGZhy{}\PYGZhy{}\PYGZhy{}\PYGZhy{}\PYGZhy{}\PYGZhy{}\PYGZhy{}\PYGZhy{}\PYGZhy{}\PYGZhy{}*
//* Reception of the .XMIT File *
//*\PYGZhy{}\PYGZhy{}\PYGZhy{}\PYGZhy{}\PYGZhy{}\PYGZhy{}\PYGZhy{}\PYGZhy{}\PYGZhy{}\PYGZhy{}\PYGZhy{}\PYGZhy{}\PYGZhy{}\PYGZhy{}\PYGZhy{}\PYGZhy{}\PYGZhy{}\PYGZhy{}\PYGZhy{}\PYGZhy{}\PYGZhy{}\PYGZhy{}\PYGZhy{}\PYGZhy{}\PYGZhy{}\PYGZhy{}\PYGZhy{}\PYGZhy{}\PYGZhy{}\PYGZhy{}\PYGZhy{}\PYGZhy{}\PYGZhy{}\PYGZhy{}\PYGZhy{}\PYGZhy{}\PYGZhy{}\PYGZhy{}\PYGZhy{}\PYGZhy{}\PYGZhy{}\PYGZhy{}\PYGZhy{}\PYGZhy{}\PYGZhy{}\PYGZhy{}\PYGZhy{}\PYGZhy{}\PYGZhy{}\PYGZhy{}\PYGZhy{}\PYGZhy{}\PYGZhy{}\PYGZhy{}\PYGZhy{}\PYGZhy{}\PYGZhy{}\PYGZhy{}\PYGZhy{}\PYGZhy{}\PYGZhy{}\PYGZhy{}\PYGZhy{}\PYGZhy{}\PYGZhy{}\PYGZhy{}\PYGZhy{}*
//BATCHTS EXEC PGM=IKJEFT1A,REGION=4M
//SYSPRINT DD SYSOUT=*
//SYSTSPRT DD SYSOUT=*
//XMITFILE DD DSN=\PYGZam{}SYSUID..TRANSFER.XMIT,DISP=OLD
//SYSTSIN DD *
RECEIVE INFILE(XMITFILE) DA(TRANSFER.DSSDUMP)
//*
//*\PYGZhy{}\PYGZhy{}\PYGZhy{}\PYGZhy{}\PYGZhy{}\PYGZhy{}\PYGZhy{}\PYGZhy{}\PYGZhy{}\PYGZhy{}\PYGZhy{}\PYGZhy{}\PYGZhy{}\PYGZhy{}\PYGZhy{}\PYGZhy{}\PYGZhy{}\PYGZhy{}\PYGZhy{}\PYGZhy{}\PYGZhy{}\PYGZhy{}\PYGZhy{}\PYGZhy{}\PYGZhy{}\PYGZhy{}\PYGZhy{}\PYGZhy{}\PYGZhy{}\PYGZhy{}\PYGZhy{}\PYGZhy{}\PYGZhy{}\PYGZhy{}\PYGZhy{}\PYGZhy{}\PYGZhy{}\PYGZhy{}\PYGZhy{}\PYGZhy{}\PYGZhy{}\PYGZhy{}\PYGZhy{}\PYGZhy{}\PYGZhy{}\PYGZhy{}\PYGZhy{}\PYGZhy{}\PYGZhy{}\PYGZhy{}\PYGZhy{}\PYGZhy{}\PYGZhy{}\PYGZhy{}\PYGZhy{}\PYGZhy{}\PYGZhy{}\PYGZhy{}\PYGZhy{}\PYGZhy{}\PYGZhy{}\PYGZhy{}\PYGZhy{}\PYGZhy{}\PYGZhy{}\PYGZhy{}\PYGZhy{}*
//* Reload of the initial files *
//*\PYGZhy{}\PYGZhy{}\PYGZhy{}\PYGZhy{}\PYGZhy{}\PYGZhy{}\PYGZhy{}\PYGZhy{}\PYGZhy{}\PYGZhy{}\PYGZhy{}\PYGZhy{}\PYGZhy{}\PYGZhy{}\PYGZhy{}\PYGZhy{}\PYGZhy{}\PYGZhy{}\PYGZhy{}\PYGZhy{}\PYGZhy{}\PYGZhy{}\PYGZhy{}\PYGZhy{}\PYGZhy{}\PYGZhy{}\PYGZhy{}\PYGZhy{}\PYGZhy{}\PYGZhy{}\PYGZhy{}\PYGZhy{}\PYGZhy{}\PYGZhy{}\PYGZhy{}\PYGZhy{}\PYGZhy{}\PYGZhy{}\PYGZhy{}\PYGZhy{}\PYGZhy{}\PYGZhy{}\PYGZhy{}\PYGZhy{}\PYGZhy{}\PYGZhy{}\PYGZhy{}\PYGZhy{}\PYGZhy{}\PYGZhy{}\PYGZhy{}\PYGZhy{}\PYGZhy{}\PYGZhy{}\PYGZhy{}\PYGZhy{}\PYGZhy{}\PYGZhy{}\PYGZhy{}\PYGZhy{}\PYGZhy{}\PYGZhy{}\PYGZhy{}\PYGZhy{}\PYGZhy{}\PYGZhy{}\PYGZhy{}*
//DSSREST EXEC PGM=ADRDSSU,REGION=6M,COND=(0,NE)
//SYSPRINT DD SYSOUT=*
//DUMPFILE DD DSN=\PYGZam{}SYSUID..TRANSFER.DSSDUMP,DISP=(OLD,DELETE)
RESTORE \PYGZhy{}
DS(INCLUDE(SPRODUIT.VIRTEL.BASE*.**)) \PYGZhy{}
OUTDYNAM(??????,3390) /* \PYGZlt{}==== VOLUME, UNIT ===== */ \PYGZhy{}
RENAMEUNC( \PYGZhy{}
    (SPRODUIT.VIRTEL.BASEvrr.LOADLIB, \PYGZhy{}
            yourqual.VIRTvrr.LOADLIB), \PYGZhy{}
    (SPRODUIT.VIRTEL.BASEvrr.MACLIB, \PYGZhy{}
            yourqual.VIRTvrr.MACLIB), \PYGZhy{}
    (SPRODUIT.VIRTEL.BASEvrr.SAMPLIB, \PYGZhy{}
            yourqual.VIRTvrr.SAMPLIB), \PYGZhy{}
    (SPRODUIT.VIRTEL.BASEvrr.SERVLIB, \PYGZhy{}
            yourqual.VIRTvrr.SERVLIB), \PYGZhy{}
    (SPRODUIT.VIRTEL.BASEvrr.DBRMLIB, \PYGZhy{}
            yourqual.VIRTvrr.DBRMLIB), \PYGZhy{}
    (SPRODUIT.VIRTEL.BASEvrr.CNTL, \PYGZhy{}
            yourqual.VIRTvrr.CNTL), \PYGZhy{}
    (SPRODUIT.VIRTEL.BASEvrr.SAMP.TRSF, \PYGZhy{}
            yourqual.VIRTvrr.SAMP.TRSF), \PYGZhy{}
    (SPRODUIT.VIRTEL.BASEvrr.CONFGEN.MACLIB, \PYGZhy{}
            yourqual.VIRTvrr.CONFGEN.MACLIB), \PYGZhy{}
    (SPRODUIT.VIRTEL.BASEvrr.FA29API.MACLIB, \PYGZhy{}
            yourqual.VIRTvrr.FA29API.MACLIB), \PYGZhy{}
    (SPRODUIT.VIRTEL.BASEvrr.SCRNAPI.MACLIB, \PYGZhy{}
            yourqual.VIRTvrr.SCRNAPI.MACLIB), \PYGZhy{}
    (SPRODUIT.VIRTEL.BASEvrr.VIRAPI.MACLIB, \PYGZhy{}
            yourqual.VIRTvrr.VIRAPI.MACLIB), \PYGZhy{}
    (SPRODUIT.VIRTEL.BASEvrr.ARBO, \PYGZhy{}
            yourqual.VIRTvrr.ARBO), \PYGZhy{}
    (SPRODUIT.VIRTEL.BASEvrr.CAPT, \PYGZhy{}
            yourqual.VIRTvrr.CAPT), \PYGZhy{}
    (SPRODUIT.VIRTEL.BASEvrr.CMP3, \PYGZhy{}
            yourqual.VIRTvrr.CMP3), \PYGZhy{}
    (SPRODUIT.VIRTEL.BASEvrr.HTML, \PYGZhy{}
            yourqual.VIRTvrr.HTML), \PYGZhy{}
    (SPRODUIT.VIRTEL.BASEvrr.HTML.TRSF, \PYGZhy{}
            yourqual.VIRTvrr.HTML.TRSF), \PYGZhy{}
    (SPRODUIT.VIRTEL.BASEvrr.PLUG.TRSF, \PYGZhy{}
            yourqual.VIRTvrr.PLUG.TRSF), \PYGZhy{}
            (SPRODUIT.VIRTEL.BASEvrr.SWAP, \PYGZhy{}
                    yourqual.VIRTvrr.SWAP), \PYGZhy{}
            (SPRODUIT.VIRTEL.BASEvrr.STAT, \PYGZhy{}
                    yourqual.VIRTvrr.STAT), \PYGZhy{}
                    ) \textendash{}
    /* NULLSTORCLAS BYPASSACS(**) */ /* \PYGZlt{}==== SMS OVERRIDE ===== */ \PYGZhy{}
    /* ADMIN TOL(ENQF) */                    /* \PYGZlt{}==== OPTIONAL ========= */ \PYGZhy{}
    /* REPLACE SHR */                                /* \PYGZlt{}==== SI EXISTE DEJA === */ \PYGZhy{}
            CATALOG INDD(DUMPFILE)
    //*
    //
\end{sphinxVerbatim}

\sphinxAtStartPar
\sphinxstyleemphasis{JCL for restoring from an XMIT file (z/OS)}

\sphinxAtStartPar
The following changes should be made to this job before submitting it:
\begin{itemize}
\item {} 
\sphinxAtStartPar
If the VIRTEL datasets are not to be managed by SMS, alter the statement OUTDYNAM(??????,3390) to specify the volume on which the datasets are to be allocated.

\item {} 
\sphinxAtStartPar
If the VIRTEL datasets are to be managed by SMS, remove the NULLSTORCLAS BYPASSACS(**) statement and replace it by STORCLAS(classname) where classname is the name of the SMS storage class on which the VIRTEL datasets are to be allocated. Do not delete the OUTDYNAM parameter, ADRDSSU requires it even though its value is ignored for SMS.

\item {} 
\sphinxAtStartPar
In the RENAMEUNC parameter, replace yourqual by the high\sphinxhyphen{}level qualifiers to be used for your VIRTEL datasets.

\item {} 
\sphinxAtStartPar
The ADMIN and TOL(ENQF) parameters may be uncommented if you are authorized to the necessary STGADMIN profiles.

\end{itemize}

\newpage

\index{Installing under z/OS@\spxentry{Installing under z/OS}!Applying maintenance@\spxentry{Applying maintenance}}\index{Applying maintenance@\spxentry{Applying maintenance}!Installing under z/OS@\spxentry{Installing under z/OS}}\ignorespaces 

\section{Applying Maintenance}
\label{\detokenize{Installation_Guide:applying-maintenance}}\label{\detokenize{Installation_Guide:index-5}}
\sphinxAtStartPar
As a general rule the application of \sphinxstylestrong{PTFs} is necessary and recommended. PTFs are maintenance files which must be applied to the VIRTEL LOADLIB to correct problems which have been discovered subsequent to the building of a VIRTEL version, or to add new function which will be included as standard in the next version. PTF files are called allptfs\sphinxhyphen{}mvsvrr.txt.

\sphinxAtStartPar
A second type of PTF, known as \sphinxstylestrong{Updates} consists of web elements or artifacts such as HTML pages, style sheets, and JavaScript files, which must be uploaded into VIRTEL internal directories contained in the SAMPTRSF VSAM file. This type of PTF is delivered with the naming convention of VirtelvrrUpdtnnnnn.zip where vrr is the Virtel release number and nnnn the update number. Updates may be distributed either by e\sphinxhyphen{}mail, or available on Syspertec FTP Server. An update is a ZIP file containing the cumulative update for a version Virtel. Once unzipped, the file content is in the form of a tree where each folder contains one or more files grouped by directory, the root contains a file named updtnnnn.txt which summarized the history of changes and any special instructions to operate.

\sphinxAtStartPar
The Virtel Administration portal is used to upload \sphinxstylestrong{Updates} to the SAMPTRSF file. An alternative batch process, using a Virtel maintenance package (VMP), can also be deployed. Updates may sometimes be supplied as a complete replacement for the SAMPTRSF file in the form of a DF/DSS dump in XMIT format. See the section “Uploading HTML Pages” from document “Virtel Web Access User Guide” for further information.


\subsection{Obtaining PTFs and Updates}
\label{\detokenize{Installation_Guide:obtaining-ptfs-and-updates}}
\sphinxAtStartPar
To download PTFs from the Syspertec file server, use your web browser to login to the file server as described 13, navigate to the Public directory “VIRTEL\sphinxhyphen{}V.461\sphinxhyphen{}PTFS” and download the ptf and update files if they exists. Virtel maintenance packages (VMP) are used to load updates via a batch process rather than the manually process required for updates. The naming convention is VirtelvrrVMPnnnn.zip. VMP’s are released periodically and are only applicable to z/OS.


\subsection{Applying PTFs}
\label{\detokenize{Installation_Guide:applying-ptfs}}
\sphinxAtStartPar
The allptfs\sphinxhyphen{}z/OSvrr.txt file should be uploaded in text format to member PTFvrrMV of the VIRTEL CNTL library.

\sphinxAtStartPar
For PTFs which contain elements to be uploaded to VIRTEL, first unzip the elements to a directory on your workstation. Then use the “Upload” link from the VIRTEL Web Access page at \sphinxurl{http://n.n.n.n:41001} to upload the elements one by one to the W2H\sphinxhyphen{}DIR directory.

\sphinxAtStartPar
In the case of a PTF containing a replacement SAMPTRSF file in DF/DSS XMIT format, use the procedure previously described (\$ALOCDSU and \$RESTDSU) to upload the file in binary and retrieve the SAMPTRSF VSAM file.


\subsection{Applying PTFs}
\label{\detokenize{Installation_Guide:id1}}
\sphinxAtStartPar
The recovered PTFs are applied to the VIRTEL LOADLIB by using AMASPZAP with the IGNIDRFULL parameter. The ZAPJCL member in the VIRTEL CNTL library (shown below) performs the apply. This job should complete with return code 0000 or 0004.:

\begin{sphinxVerbatim}[commandchars=\\\{\}]
\PYG{o}{/}\PYG{o}{/}\PYG{n}{VIRPTF} \PYG{n}{JOB} \PYG{l+m+mi}{1}\PYG{p}{,}\PYG{n}{ZAPJCL}\PYG{p}{,}\PYG{n}{CLASS}\PYG{o}{=}\PYG{n}{A}\PYG{p}{,}\PYG{n}{MSGCLASS}\PYG{o}{=}\PYG{n}{X}\PYG{p}{,}\PYG{n}{NOTIFY}\PYG{o}{=}\PYG{o}{\PYGZam{}}\PYG{n}{SYSUID}
\PYG{o}{/}\PYG{o}{/}\PYG{o}{*}
\PYG{o}{/}\PYG{o}{/}\PYG{o}{*} \PYG{n}{PTF} \PYG{n}{à} \PYG{n}{APPLIQUER}
\PYG{o}{/}\PYG{o}{/}\PYG{o}{*}
\PYG{o}{/}\PYG{o}{/} \PYG{n}{SET} \PYG{n}{LOAD}\PYG{o}{=}\PYG{n}{yourqual}\PYG{o}{.}\PYG{n}{VIRTvrr}\PYG{o}{.}\PYG{n}{LOADLIB}
\PYG{o}{/}\PYG{o}{/} \PYG{n}{SET} \PYG{n}{CNTL}\PYG{o}{=}\PYG{n}{yourqual}\PYG{o}{.}\PYG{n}{VIRTvrr}\PYG{o}{.}\PYG{n}{CNTL}
\PYG{o}{/}\PYG{o}{/} \PYG{n}{SET} \PYG{n}{MEMBER}\PYG{o}{=}\PYG{n}{PTFvrrMV}
\PYG{o}{/}\PYG{o}{/}\PYG{o}{*}
\PYG{o}{/}\PYG{o}{/}\PYG{n}{PTFZAP} \PYG{n}{EXEC} \PYG{n}{PGM}\PYG{o}{=}\PYG{n}{AMASPZAP}\PYG{p}{,}\PYG{n}{PARM}\PYG{o}{=}\PYG{l+s+s1}{\PYGZsq{}}\PYG{l+s+s1}{IGNIDRFULL}\PYG{l+s+s1}{\PYGZsq{}}
\PYG{o}{/}\PYG{o}{/}\PYG{n}{SYSPRINT} \PYG{n}{DD} \PYG{n}{SYSOUT}\PYG{o}{=}\PYG{o}{*}
\PYG{o}{/}\PYG{o}{/}\PYG{n}{SYSLIB} \PYG{n}{DD} \PYG{n}{DSN}\PYG{o}{=}\PYG{o}{\PYGZam{}}\PYG{n}{LOAD}\PYG{p}{,}\PYG{n}{DISP}\PYG{o}{=}\PYG{n}{SHR}
\PYG{o}{/}\PYG{o}{/}\PYG{n}{SYSIN} \PYG{n}{DD} \PYG{n}{DSN}\PYG{o}{=}\PYG{o}{\PYGZam{}}\PYG{n}{CNTL}\PYG{p}{(}\PYG{o}{\PYGZam{}}\PYG{n}{MEMBER}\PYG{p}{)}\PYG{p}{,}\PYG{n}{DISP}\PYG{o}{=}\PYG{n}{SHR}
\end{sphinxVerbatim}

\sphinxAtStartPar
\sphinxstyleemphasis{Member ZAPJCL for applying PTFs (z/OS)}


\subsection{Restarting VIRTEL and validation of PTF level}
\label{\detokenize{Installation_Guide:restarting-virtel-and-validation-of-ptf-level}}
\sphinxAtStartPar
VIRTEL must be stopped and restarted to allow the newly\sphinxhyphen{}applied PTFs to take effect. The list of PTFs applied is displayed near the beginning of the SYSMSGLG dataset during VIRTEL startup by message VIR0018I, as shown in the following example:

\begin{sphinxVerbatim}[commandchars=\\\{\}]
    \PYG{n}{VIR0018I} \PYG{n}{VIRTEL} \PYG{l+m+mf}{4.6}\PYG{n}{x} \PYG{n}{HAS} \PYG{n}{THE} \PYG{n}{FOLLOWING} \PYG{n}{PTF}\PYG{p}{(}\PYG{n}{S}\PYG{p}{)} \PYG{n}{APPLIED}
\PYG{n}{VIR0018I} \PYG{l+m+mi}{5887}\PYG{p}{,}\PYG{l+m+mi}{5901}\PYG{p}{,}\PYG{l+m+mi}{5903}\PYG{p}{,}\PYG{l+m+mi}{5904}\PYG{p}{,}\PYG{l+m+mi}{5912}\PYG{p}{,}\PYG{l+m+mi}{5912}\PYG{n}{A}

\PYG{o}{*}\PYG{n}{Validation} \PYG{n}{of} \PYG{n}{the} \PYG{n}{VIRTEL} \PYG{n}{PTF} \PYG{n}{level}\PYG{o}{*}
\end{sphinxVerbatim}


\subsection{Upgrading from a previous version}
\label{\detokenize{Installation_Guide:upgrading-from-a-previous-version}}

\subsubsection{Datasets to be upgraded}
\label{\detokenize{Installation_Guide:datasets-to-be-upgraded}}
\sphinxAtStartPar
If you already have a previous version of VIRTEL installed (version 4.00 or later) then you only need the datasets shown in the figure below:

\begin{sphinxVerbatim}[commandchars=\\\{\}]
\PYG{n}{yourqual}\PYG{o}{.}\PYG{n}{VIRTvrr}\PYG{o}{.}\PYG{n}{LOADLIB}
\PYG{n}{yourqual}\PYG{o}{.}\PYG{n}{VIRTvrr}\PYG{o}{.}\PYG{n}{MACLIB}
\PYG{n}{yourqual}\PYG{o}{.}\PYG{n}{VIRTvrr}\PYG{o}{.}\PYG{n}{SAMPLIB}
\PYG{n}{yourqual}\PYG{o}{.}\PYG{n}{VIRTvrr}\PYG{o}{.}\PYG{n}{SERVLIB}
\PYG{n}{yourqual}\PYG{o}{.}\PYG{n}{VIRTvrr}\PYG{o}{.}\PYG{n}{DBRMLIB}
\PYG{n}{yourqual}\PYG{o}{.}\PYG{n}{VIRTvrr}\PYG{o}{.}\PYG{n}{CNTL}
\PYG{n}{yourqual}\PYG{o}{.}\PYG{n}{VIRTvrr}\PYG{o}{.}\PYG{n}{SAMP}\PYG{o}{.}\PYG{n}{TRSF}
\PYG{n}{yourqual}\PYG{o}{.}\PYG{n}{VIRTvrr}\PYG{o}{.}\PYG{n}{CONFGEN}\PYG{o}{.}\PYG{n}{MACLIB}
\PYG{n}{yourqual}\PYG{o}{.}\PYG{n}{VIRTvrr}\PYG{o}{.}\PYG{n}{FA29API}\PYG{o}{.}\PYG{n}{MACLIB}
\PYG{n}{yourqual}\PYG{o}{.}\PYG{n}{VIRTvrr}\PYG{o}{.}\PYG{n}{SCRNAPI}\PYG{o}{.}\PYG{n}{MACLIB}
\PYG{n}{yourqual}\PYG{o}{.}\PYG{n}{VIRTvrr}\PYG{o}{.}\PYG{n}{VIRAPI}\PYG{o}{.}\PYG{n}{MACLIB}
\end{sphinxVerbatim}

\sphinxAtStartPar
\sphinxstyleemphasis{Datasets upgraded during version change}

\sphinxAtStartPar
For the remaining datasets, shown in the figure below, you should continue to use your existing datasets, as these may containing customer\sphinxhyphen{}specific configuration information which you do not want to overwrite:

\begin{sphinxVerbatim}[commandchars=\\\{\}]
\PYG{n}{yourqual}\PYG{o}{.}\PYG{n}{VIRTnnn}\PYG{o}{.}\PYG{n}{ARBO}
\PYG{n}{yourqual}\PYG{o}{.}\PYG{n}{VIRTnnn}\PYG{o}{.}\PYG{n}{CAPT}
\PYG{n}{yourqual}\PYG{o}{.}\PYG{n}{VIRTnnn}\PYG{o}{.}\PYG{n}{CMP3}
\PYG{n}{yourqual}\PYG{o}{.}\PYG{n}{VIRTnnn}\PYG{o}{.}\PYG{n}{HTML}
\PYG{n}{yourqual}\PYG{o}{.}\PYG{n}{VIRTnnn}\PYG{o}{.}\PYG{n}{HTML}\PYG{o}{.}\PYG{n}{TRSF}
\PYG{n}{yourqual}\PYG{o}{.}\PYG{n}{VIRTnnn}\PYG{o}{.}\PYG{n}{SWAP}
\PYG{n}{yourqual}\PYG{o}{.}\PYG{n}{VIRTnnn}\PYG{o}{.}\PYG{n}{STAT}
\end{sphinxVerbatim}

\sphinxAtStartPar
\sphinxstyleemphasis{Datasets to be retained from previous version}

\begin{sphinxadmonition}{note}{Note:}
\sphinxAtStartPar
It is also possible to copy your existing files into the files of the new version using IDCAMS REPRO (or by ARBOLOAD for the VIRARBO file).
\end{sphinxadmonition}

\newpage

\index{Installing under z/OS@\spxentry{Installing under z/OS}!Upgrading from a previous version@\spxentry{Upgrading from a previous version}}\index{Upgrading from a previous version@\spxentry{Upgrading from a previous version}!Installing under z/OS@\spxentry{Installing under z/OS}}\ignorespaces 

\section{Upgrading from a previous version}
\label{\detokenize{Installation_Guide:index-6}}\label{\detokenize{Installation_Guide:id2}}
\sphinxAtStartPar
The procedure for upgrading from a previous version of VIRTEL (version 4.00 or later) is as follows. Customers upgrading from earlier versions of VIRTEL should contact Syspertec for technical support. This process should not change you CLI\sphinxhyphen{}DIR or other directories which resilde in the HTML.TRSF dataset.

\begin{sphinxadmonition}{danger}{Danger:}
\sphinxAtStartPar
If you have modified default settings or added customized elements to the W2H\sphinxhyphen{}DIR directory these will be overridden when upgrading. The upgrade process installs a new SAMP.TRSF VSAM file which is the VSAM file associated with W2H\sphinxhyphen{}DIR. See note 9. You are advised not to use the Administration port 41001 for user transasctions as this port is reserved for Virtel administration and is tied to web elements distributed in W2H\sphinxhyphen{}DIR. User transaction should be associated with another port. The sample provide is port 41002 which is associated with the CLI\sphinxhyphen{}DIR directory. This directory is not affected by an upgrade.
\end{sphinxadmonition}
\begin{enumerate}
\sphinxsetlistlabels{\arabic}{enumi}{enumii}{}{.}%
\item {} 
\sphinxAtStartPar
Upload and unpack the virtelnnnmvs.xmit file as described in the previous section.

\item {} 
\sphinxAtStartPar
Apply PTFs as described in the previous section.

\item {} 
\sphinxAtStartPar
Copy your VIRTCTnn from the old VIRTnnn.CNTL library to the new VIRTnnn.CNTL

\item {} 
\sphinxAtStartPar
Reassemble your VIRTCTnn module using the ASMTCT job in VIRTnnn.CNTL

\item {} 
\sphinxAtStartPar
If you have any scenario or user exit load modules, copy them to the VIRTnnn.CNTL library and reassemble them using the ASMSCEN and ASMEXIT jobs respectively.

\item {} 
\sphinxAtStartPar
Add the new VIRTnnn.LOADLIB library to the system APF list in the z/OS PARMLIB and use the SETPROG command to authorize the VIRTnnn.LOADLIB library.

\item {} 
\sphinxAtStartPar
Edit your VIRTEL procedure in the z/OS PROCLIB, to ensure that the STEPLIB, DFHRPL, and SERVLIB DD statements reference the new VIRTnnn.LOADLIB, and that the SAMPTRSF DD statement references the new VIRTnnn.SAMP.TRSF dataset.

\item {} 
\sphinxAtStartPar
If you have modified the default values for the VIRTEL Web Access Settings (as described in the VIRTEL Web Access Guide) and these changes reside in the W2H\sphinxhyphen{}DIR then the upgrade procedure will loose these changes. You are strongly advised not to keep any user modifications in the W2H\sphinxhyphen{}DIR but instead move them to the CLI\sphinxhyphen{}DIR or any other user directory and modify transactions accordingly. User customizations, such as defaults for w2hparm settings, should be uploaded to a user directory such as the CLI\sphinxhyphen{}DIR directory. See the technical newsletter \sphinxstyleemphasis{TN201611 \sphinxhyphen{} Customising Virtel in V4.56} for further details.

\item {} 
\sphinxAtStartPar
Stop and restart VIRTEL.

\end{enumerate}

\index{Installing under z/OS@\spxentry{Installing under z/OS}!Applying Maintenance Updates@\spxentry{Applying Maintenance Updates}}\index{Applying Maintenance Updates@\spxentry{Applying Maintenance Updates}!Installing under z/OS@\spxentry{Installing under z/OS}}\ignorespaces 
\newpage

\index{Execution@\spxentry{Execution}!Executing Virtel in a z/OS Environment@\spxentry{Executing Virtel in a z/OS Environment}}\index{Executing Virtel in a z/OS Environment@\spxentry{Executing Virtel in a z/OS Environment}!Execution@\spxentry{Execution}}\ignorespaces 
\newpage


\section{Executing Virtel in a z/OS environment}
\label{\detokenize{Installation_Guide:executing-virtel-in-a-z-os-environment}}
\sphinxAtStartPar
VIRTEL can run as a JOB or as an STC. An example JCL procedure is contained in member VIRTEL4 of the VIRTEL SAMPLIB. If VIRTEL is to be run as an STC, this member must be copied into a system PROCLIB and renamed as VIRTEL.

\begin{sphinxVerbatim}[commandchars=\\\{\}]
\PYG{o}{/}\PYG{o}{/}\PYG{n}{VIRTEL} \PYG{n}{PROC} \PYG{n}{QUAL}\PYG{o}{=}\PYG{n}{yourqual}\PYG{o}{.}\PYG{n}{VIRTvrr}\PYG{p}{,}
\PYG{o}{/}\PYG{o}{/}\PYG{o}{*} \PYG{n}{QUALMQ}\PYG{o}{=}\PYG{n}{CSQ600}\PYG{p}{,}   \PYG{o}{\PYGZlt{}}\PYG{o}{\PYGZhy{}}\PYG{o}{\PYGZhy{}} \PYG{n}{MQSeries} \PYG{n}{qualifier}
\PYG{o}{/}\PYG{o}{/}  \PYG{n}{APPLID}\PYG{o}{=}\PYG{p}{,}         \PYG{o}{\PYGZlt{}}\PYG{o}{\PYGZhy{}}\PYG{o}{\PYGZhy{}} \PYG{n}{Default} \PYG{o+ow}{is} \PYG{o+ow}{in} \PYG{n}{VIRTCT}
\PYG{o}{/}\PYG{o}{/}  \PYG{n}{TCT}\PYG{o}{=}\PYG{l+m+mi}{01}           \PYG{o}{\PYGZlt{}}\PYG{o}{\PYGZhy{}}\PYG{o}{\PYGZhy{}} \PYG{n}{Suffix} \PYG{n}{of} \PYG{n}{VIRTCT}
\PYG{o}{/}\PYG{o}{/}\PYG{o}{*}\PYG{o}{\PYGZhy{}}\PYG{o}{\PYGZhy{}}\PYG{o}{\PYGZhy{}}\PYG{o}{\PYGZhy{}}\PYG{o}{\PYGZhy{}}\PYG{o}{\PYGZhy{}}\PYG{o}{\PYGZhy{}}\PYG{o}{\PYGZhy{}}\PYG{o}{\PYGZhy{}}\PYG{o}{\PYGZhy{}}\PYG{o}{\PYGZhy{}}\PYG{o}{\PYGZhy{}}\PYG{o}{\PYGZhy{}}\PYG{o}{\PYGZhy{}}\PYG{o}{\PYGZhy{}}\PYG{o}{\PYGZhy{}}\PYG{o}{\PYGZhy{}}\PYG{o}{\PYGZhy{}}\PYG{o}{\PYGZhy{}}\PYG{o}{\PYGZhy{}}\PYG{o}{\PYGZhy{}}\PYG{o}{\PYGZhy{}}\PYG{o}{\PYGZhy{}}\PYG{o}{\PYGZhy{}}\PYG{o}{\PYGZhy{}}\PYG{o}{\PYGZhy{}}\PYG{o}{\PYGZhy{}}\PYG{o}{\PYGZhy{}}\PYG{o}{\PYGZhy{}}\PYG{o}{\PYGZhy{}}\PYG{o}{\PYGZhy{}}\PYG{o}{\PYGZhy{}}\PYG{o}{\PYGZhy{}}\PYG{o}{\PYGZhy{}}\PYG{o}{\PYGZhy{}}\PYG{o}{\PYGZhy{}}\PYG{o}{\PYGZhy{}}\PYG{o}{\PYGZhy{}}\PYG{o}{\PYGZhy{}}\PYG{o}{\PYGZhy{}}\PYG{o}{\PYGZhy{}}\PYG{o}{\PYGZhy{}}\PYG{o}{\PYGZhy{}}\PYG{o}{\PYGZhy{}}\PYG{o}{\PYGZhy{}}\PYG{o}{\PYGZhy{}}\PYG{o}{\PYGZhy{}}\PYG{o}{\PYGZhy{}}\PYG{o}{\PYGZhy{}}\PYG{o}{\PYGZhy{}}\PYG{o}{\PYGZhy{}}\PYG{o}{\PYGZhy{}}\PYG{o}{\PYGZhy{}}\PYG{o}{\PYGZhy{}}\PYG{o}{\PYGZhy{}}\PYG{o}{\PYGZhy{}}\PYG{o}{\PYGZhy{}}\PYG{o}{\PYGZhy{}}\PYG{o}{\PYGZhy{}}\PYG{o}{\PYGZhy{}}\PYG{o}{\PYGZhy{}}\PYG{o}{\PYGZhy{}}\PYG{o}{\PYGZhy{}}\PYG{o}{\PYGZhy{}}\PYG{o}{\PYGZhy{}}\PYG{o}{\PYGZhy{}}\PYG{o}{\PYGZhy{}}\PYG{o}{*}
\PYG{o}{/}\PYG{o}{/}\PYG{o}{*} \PYG{n}{PROCEDURE} \PYG{n}{LANCEMENT} \PYG{n}{VIRTEL}                                        \PYG{o}{*}
\PYG{o}{/}\PYG{o}{/}\PYG{o}{*}\PYG{o}{\PYGZhy{}}\PYG{o}{\PYGZhy{}}\PYG{o}{\PYGZhy{}}\PYG{o}{\PYGZhy{}}\PYG{o}{\PYGZhy{}}\PYG{o}{\PYGZhy{}}\PYG{o}{\PYGZhy{}}\PYG{o}{\PYGZhy{}}\PYG{o}{\PYGZhy{}}\PYG{o}{\PYGZhy{}}\PYG{o}{\PYGZhy{}}\PYG{o}{\PYGZhy{}}\PYG{o}{\PYGZhy{}}\PYG{o}{\PYGZhy{}}\PYG{o}{\PYGZhy{}}\PYG{o}{\PYGZhy{}}\PYG{o}{\PYGZhy{}}\PYG{o}{\PYGZhy{}}\PYG{o}{\PYGZhy{}}\PYG{o}{\PYGZhy{}}\PYG{o}{\PYGZhy{}}\PYG{o}{\PYGZhy{}}\PYG{o}{\PYGZhy{}}\PYG{o}{\PYGZhy{}}\PYG{o}{\PYGZhy{}}\PYG{o}{\PYGZhy{}}\PYG{o}{\PYGZhy{}}\PYG{o}{\PYGZhy{}}\PYG{o}{\PYGZhy{}}\PYG{o}{\PYGZhy{}}\PYG{o}{\PYGZhy{}}\PYG{o}{\PYGZhy{}}\PYG{o}{\PYGZhy{}}\PYG{o}{\PYGZhy{}}\PYG{o}{\PYGZhy{}}\PYG{o}{\PYGZhy{}}\PYG{o}{\PYGZhy{}}\PYG{o}{\PYGZhy{}}\PYG{o}{\PYGZhy{}}\PYG{o}{\PYGZhy{}}\PYG{o}{\PYGZhy{}}\PYG{o}{\PYGZhy{}}\PYG{o}{\PYGZhy{}}\PYG{o}{\PYGZhy{}}\PYG{o}{\PYGZhy{}}\PYG{o}{\PYGZhy{}}\PYG{o}{\PYGZhy{}}\PYG{o}{\PYGZhy{}}\PYG{o}{\PYGZhy{}}\PYG{o}{\PYGZhy{}}\PYG{o}{\PYGZhy{}}\PYG{o}{\PYGZhy{}}\PYG{o}{\PYGZhy{}}\PYG{o}{\PYGZhy{}}\PYG{o}{\PYGZhy{}}\PYG{o}{\PYGZhy{}}\PYG{o}{\PYGZhy{}}\PYG{o}{\PYGZhy{}}\PYG{o}{\PYGZhy{}}\PYG{o}{\PYGZhy{}}\PYG{o}{\PYGZhy{}}\PYG{o}{\PYGZhy{}}\PYG{o}{\PYGZhy{}}\PYG{o}{\PYGZhy{}}\PYG{o}{\PYGZhy{}}\PYG{o}{\PYGZhy{}}\PYG{o}{\PYGZhy{}}\PYG{o}{*}
\PYG{o}{/}\PYG{o}{/}\PYG{n}{VIRTEL} \PYG{n}{EXEC} \PYG{n}{PGM}\PYG{o}{=}\PYG{n}{VIR6000}\PYG{p}{,}
\PYG{o}{/}\PYG{o}{/} \PYG{n}{TIME}\PYG{o}{=}\PYG{l+m+mi}{1440}\PYG{p}{,}\PYG{n}{REGION}\PYG{o}{=}\PYG{l+m+mi}{128}\PYG{n}{M}\PYG{p}{,}
\PYG{o}{/}\PYG{o}{/} \PYG{n}{PARM}\PYG{o}{=}\PYG{p}{(}\PYG{o}{\PYGZam{}}\PYG{n}{TCT}\PYG{p}{,}\PYG{o}{\PYGZam{}}\PYG{n}{APPLID}\PYG{p}{)}
\PYG{o}{/}\PYG{o}{/}\PYG{n}{STEPLIB} \PYG{n}{DD} \PYG{n}{DSN}\PYG{o}{=}\PYG{o}{\PYGZam{}}\PYG{n}{QUAL}\PYG{o}{.}\PYG{o}{.}\PYG{n}{LOADLIB}\PYG{p}{,}\PYG{n}{DISP}\PYG{o}{=}\PYG{n}{SHR}
\PYG{o}{/}\PYG{o}{/}\PYG{o}{*} \PYG{n}{DD} \PYG{n}{DSN}\PYG{o}{=}\PYG{o}{\PYGZam{}}\PYG{n}{QUALMQ}\PYG{o}{.}\PYG{o}{.}\PYG{n}{SCSQANLE}\PYG{p}{,}\PYG{n}{DISP}\PYG{o}{=}\PYG{n}{SHR}
\PYG{o}{/}\PYG{o}{/}\PYG{o}{*} \PYG{n}{DD} \PYG{n}{DSN}\PYG{o}{=}\PYG{o}{\PYGZam{}}\PYG{n}{QUALMQ}\PYG{o}{.}\PYG{o}{.}\PYG{n}{SCSQAUTH}\PYG{p}{,}\PYG{n}{DISP}\PYG{o}{=}\PYG{n}{SHR}
\PYG{o}{/}\PYG{o}{/}\PYG{n}{DFHRPL} \PYG{n}{DD} \PYG{n}{DSN}\PYG{o}{=}\PYG{o}{\PYGZam{}}\PYG{n}{QUAL}\PYG{o}{.}\PYG{o}{.}\PYG{n}{LOADLIB}\PYG{p}{,}\PYG{n}{DISP}\PYG{o}{=}\PYG{n}{SHR}
\PYG{o}{/}\PYG{o}{/}\PYG{n}{SERVLIB} \PYG{n}{DD} \PYG{n}{DSN}\PYG{o}{=}\PYG{o}{\PYGZam{}}\PYG{n}{QUAL}\PYG{o}{.}\PYG{o}{.}\PYG{n}{SERVLIB}\PYG{p}{,}\PYG{n}{DISP}\PYG{o}{=}\PYG{n}{SHR}
\PYG{o}{/}\PYG{o}{/}\PYG{n}{VIRARBO} \PYG{n}{DD} \PYG{n}{DSN}\PYG{o}{=}\PYG{o}{\PYGZam{}}\PYG{n}{QUAL}\PYG{o}{.}\PYG{o}{.}\PYG{n}{ARBO}\PYG{p}{,}\PYG{n}{DISP}\PYG{o}{=}\PYG{n}{SHR}
\PYG{o}{/}\PYG{o}{/}\PYG{n}{VIRSWAP} \PYG{n}{DD} \PYG{n}{DSN}\PYG{o}{=}\PYG{o}{\PYGZam{}}\PYG{n}{QUAL}\PYG{o}{.}\PYG{o}{.}\PYG{n}{SWAP}\PYG{p}{,}\PYG{n}{DISP}\PYG{o}{=}\PYG{n}{SHR}
\PYG{o}{/}\PYG{o}{/}\PYG{n}{VIRSTAT} \PYG{n}{DD} \PYG{n}{DSN}\PYG{o}{=}\PYG{o}{\PYGZam{}}\PYG{n}{QUAL}\PYG{o}{.}\PYG{o}{.}\PYG{n}{STAT}\PYG{p}{,}\PYG{n}{DISP}\PYG{o}{=}\PYG{n}{SHR}
\PYG{o}{/}\PYG{o}{/}\PYG{o}{*}\PYG{n}{VIRCMP3} \PYG{n}{DD} \PYG{n}{DSN}\PYG{o}{=}\PYG{o}{\PYGZam{}}\PYG{n}{QUAL}\PYG{o}{.}\PYG{o}{.}\PYG{n}{CMP3}\PYG{p}{,}\PYG{n}{DISP}\PYG{o}{=}\PYG{n}{SHR}
\PYG{o}{/}\PYG{o}{/}\PYG{o}{*}\PYG{n}{VIRCAPT} \PYG{n}{DD} \PYG{n}{DSN}\PYG{o}{=}\PYG{o}{\PYGZam{}}\PYG{n}{QUAL}\PYG{o}{.}\PYG{o}{.}\PYG{n}{CAPT}\PYG{p}{,}\PYG{n}{DISP}\PYG{o}{=}\PYG{n}{SHR}
\PYG{o}{/}\PYG{o}{/}\PYG{n}{VIRHTML} \PYG{n}{DD} \PYG{n}{DSN}\PYG{o}{=}\PYG{o}{\PYGZam{}}\PYG{n}{QUAL}\PYG{o}{.}\PYG{o}{.}\PYG{n}{HTML}\PYG{p}{,}\PYG{n}{DISP}\PYG{o}{=}\PYG{n}{SHR}
\PYG{o}{/}\PYG{o}{/}\PYG{n}{SAMPTRSF} \PYG{n}{DD} \PYG{n}{DSN}\PYG{o}{=}\PYG{o}{\PYGZam{}}\PYG{n}{QUAL}\PYG{o}{.}\PYG{o}{.}\PYG{n}{SAMP}\PYG{o}{.}\PYG{n}{TRSF}\PYG{p}{,}\PYG{n}{DISP}\PYG{o}{=}\PYG{n}{SHR}
\PYG{o}{/}\PYG{o}{/}\PYG{n}{HTMLTRSF} \PYG{n}{DD} \PYG{n}{DSN}\PYG{o}{=}\PYG{o}{\PYGZam{}}\PYG{n}{QUAL}\PYG{o}{.}\PYG{o}{.}\PYG{n}{HTML}\PYG{o}{.}\PYG{n}{TRSF}\PYG{p}{,}\PYG{n}{DISP}\PYG{o}{=}\PYG{n}{SHR}
\PYG{o}{/}\PYG{o}{/}\PYG{n}{SYSOUT} \PYG{n}{DD} \PYG{n}{SYSOUT}\PYG{o}{=}\PYG{o}{*}
\PYG{o}{/}\PYG{o}{/}\PYG{n}{VIRLOG} \PYG{n}{DD} \PYG{n}{SYSOUT}\PYG{o}{=}\PYG{o}{*}
\PYG{o}{/}\PYG{o}{/}\PYG{n}{VIRTRACE} \PYG{n}{DD} \PYG{n}{SYSOUT}\PYG{o}{=}\PYG{o}{*}
\PYG{o}{/}\PYG{o}{/}\PYG{n}{SYSPRINT} \PYG{n}{DD} \PYG{n}{SYSOUT}\PYG{o}{=}\PYG{o}{*}
\PYG{o}{/}\PYG{o}{/}\PYG{n}{SYSUDUMP} \PYG{n}{DD} \PYG{n}{SYSOUT}\PYG{o}{=}\PYG{o}{*}
\end{sphinxVerbatim}

\sphinxAtStartPar
\sphinxstyleemphasis{VIRTEL started task JCL procedure for (z/OS)}


\subsection{Required and optional files for Virtel}
\label{\detokenize{Installation_Guide:required-and-optional-files-for-virtel}}\begin{itemize}
\item {} 
\sphinxAtStartPar
Files STEPLIB, DFHRPL are always required

\item {} 
\sphinxAtStartPar
Files VIRARBO, VIRSWAP are always required

\item {} 
\sphinxAtStartPar
File SERVLIB must be present if the VIRSV1 parameter is coded in the VIRTCT or Virtel is running authorized(PGM=VIR6000)

\item {} 
\sphinxAtStartPar
File VIRSTAT must be present if the parameter STATS=YES is coded in the VIRTCT

\item {} 
\sphinxAtStartPar
File VIRCMP3 must be present if the parameter COMPR3=AUTO or COMPR3=FIXED is coded in the VIRTCT

\item {} 
\sphinxAtStartPar
File VIRCAPT must be present if the parameter FCAPT=VIRCAPT is coded in the VIRTCT

\item {} 
\sphinxAtStartPar
File VIRHTML must be present if the parameter HTVSAM=VIRHTML is coded in the VIRTCT (parameter required for clients wishing to use e\sphinxhyphen{}mail or the centralised parameters feature of VIRTEL Web Access)

\item {} 
\sphinxAtStartPar
Files SAMPTRSF, HTMLTRSF must be present if referenced by the parameters UFILEn (and their corresponding ACBs) in the VIRTCT (required for clients wishing to use VIRTEL Web Access functions)

\item {} 
\sphinxAtStartPar
File PLUGTRSF must be present if referenced by a parameter UFILEn (and its corresponding ACB) in the VIRTCT (required for clients wishing to use the Videotex Plug\sphinxhyphen{}In function)

\item {} 
\sphinxAtStartPar
Files SYSOUT, VIRLOG, VIRTRACE, SYSPRINT, SYSUDUMP are always required

\item {} 
\sphinxAtStartPar
The libraries SCSQANLE, SCSQAUTH must be concatenated to the STEPLIB unless these libraries are in the system link list or LPA list (only for clients wishing to use VIRTEL with MQSeries)

\item {} 
\sphinxAtStartPar
The CSF.SCSFMOD0 library must be concatenated to the STEPLIB or must be present in the system link list (only if the CRYPTn=(…,ICSF,…) parameter is coded in the VIRTCT)

\end{itemize}

\newpage

\index{Execution@\spxentry{Execution}!z/OS APF Authorisation, Userid and Priority@\spxentry{z/OS APF Authorisation, Userid and Priority}}\index{z/OS APF Authorisation, Userid and Priority@\spxentry{z/OS APF Authorisation, Userid and Priority}!Execution@\spxentry{Execution}}\ignorespaces 

\section{z/OS APF authorisation, userid and priority}
\label{\detokenize{Installation_Guide:z-os-apf-authorisation-userid-and-priority}}\label{\detokenize{Installation_Guide:index-9}}
\sphinxAtStartPar
VIRTEL must run from an APF\sphinxhyphen{}authorized library if either of the following is true:
\begin{itemize}
\item {} 
\sphinxAtStartPar
External security (RACF, TOP SECRET, or ACF2) is selected by means of the SECUR parameter of the VIRTCT

\item {} 
\sphinxAtStartPar
VIRTEL is made non\sphinxhyphen{}swappable by means of the DONTSWA parameter of the VIRTCT

\end{itemize}

\sphinxAtStartPar
Normally VIRTEL is started in APF\sphinxhyphen{}authorized mode via the VIR6000 module, and in this case all of the libraries specified in the STEPLIB and DFHRPL concatenations must be APF\sphinxhyphen{}authorised. For certain specialised applications (Videotex server), the DFHRPL concatenation may include screen image libraries which cannot be APF authorised. In this case it is possible to start VIRTEL via the module VIR0APF which can be isolated in an authorised library. In this way, the other libraries declared in DFHRPL do not necessarily need to be APF\sphinxhyphen{}authorized.

\sphinxAtStartPar
VIRTEL must be run under a userid which has an OMVS segment defined in its profile. If VIRTEL is started as an STC, define a profile in the RACF STARTED class (or equivalent if using another security product) to assign the VIRTEL STC to the appropriate userid.

\sphinxAtStartPar
It is necessary for VIRTEL to run at the same priority as VTAM and TCP/IP. This is usually done by assigning VIRTEL to service class SYSSTC in the workload manager. It is also recommended that VIRTEL run non swappable (DONTSWA=YES in the VIRTCT).

\sphinxAtStartPar
Virtel can run non\sphinxhyphen{}authorised by changing the PGM=VIR6000 to PGM=VIR0000. This will prevent the use of any authorized functions, for example external security.

\index{Execution@\spxentry{Execution}!Optional JCL parameters.@\spxentry{Optional JCL parameters.}}\index{Optional JCL parameters.@\spxentry{Optional JCL parameters.}!Execution@\spxentry{Execution}}\ignorespaces 

\subsection{Optional JCL parameters}
\label{\detokenize{Installation_Guide:optional-jcl-parameters}}\label{\detokenize{Installation_Guide:index-10}}
\sphinxAtStartPar
Some parameters have a value taken by VIRTEL either from the VIRTCT or from some definition contained in the
VIRARBO file. The purpose of using JCL parameters is to lower the coupling between the TCT, ARBO and instances of
VIRTEL so that there is less dependency on the parameters defined in the ARBO and TCT for any one VIRTEL instance.
If running under z/OS, the parameter list can be transmitted by using the PARM card. If under z/VSE, it can be done by
using a SYSIN card. In both cases, parameters are positionnals and coma separated as above:\sphinxhyphen{}

\begin{sphinxVerbatim}[commandchars=\\\{\}]
\PYG{n}{TCT}\PYG{p}{,}\PYG{n}{APPLID}\PYG{p}{,}\PYG{n}{VTOVER}\PYG{p}{,}\PYG{n}{IP}\PYG{p}{,}\PYG{n}{CLONE}
\end{sphinxVerbatim}

\index{JCL Parameters@\spxentry{JCL Parameters}!TCT parameter.@\spxentry{TCT parameter.}}\index{TCT parameter.@\spxentry{TCT parameter.}!JCL Parameters@\spxentry{JCL Parameters}}\ignorespaces 

\subsubsection{TCT parameter}
\label{\detokenize{Installation_Guide:tct-parameter}}\label{\detokenize{Installation_Guide:index-11}}
\sphinxAtStartPar
All the general information necessary for VIRTEL to run is contained in a table known as the VIRTCT. By default, VIRTEL
try to use the module VIRTCT01. If you want to use another specific VIRTCT module for startup you must specify its
suffix in the first position of the PARM card.

\index{JCL Parameters@\spxentry{JCL Parameters}!APPLID parameter.@\spxentry{APPLID parameter.}}\index{APPLID parameter.@\spxentry{APPLID parameter.}!JCL Parameters@\spxentry{JCL Parameters}}\ignorespaces 

\subsubsection{APPLID parameter}
\label{\detokenize{Installation_Guide:applid-parameter}}\label{\detokenize{Installation_Guide:index-12}}
\sphinxAtStartPar
The APPLID parameter of the VIRTCT specifies the label or ACBNAME parameter of the VTAM APPL for the primary
VIRTEL ACB. The value specified in the second position of the PARM card will overide this value.

\index{JCL Parameters@\spxentry{JCL Parameters}!VTOVER parameter.@\spxentry{VTOVER parameter.}}\index{VTOVER parameter.@\spxentry{VTOVER parameter.}!JCL Parameters@\spxentry{JCL Parameters}}\ignorespaces 

\subsubsection{VTOVER parameter}
\label{\detokenize{Installation_Guide:vtover-parameter}}\label{\detokenize{Installation_Guide:index-13}}
\sphinxAtStartPar
The VTOVER parameter may overrides any VIRTCT MQn parameter defined with the “\%” wildchar characters. This
feature is depending on the presence of VTOVER=VTDYNAM within the VIRTCT. For exemple:\sphinxhyphen{}

\sphinxAtStartPar
In the VIRTCT:

\begin{sphinxVerbatim}[commandchars=\\\{\}]
\PYG{n}{VIRTERM}
\PYG{o}{.}\PYG{o}{.}\PYG{o}{/}\PYG{o}{.}\PYG{o}{.}
\PYG{n}{MQ1}\PYG{o}{=}\PYG{p}{(}\PYG{n}{CSQ}\PYG{o}{\PYGZpc{}}\PYG{p}{)}\PYG{p}{,} \PYG{o}{\PYGZhy{}}\PYG{o}{\PYGZgt{}} \PYG{n}{wild} \PYG{n}{char} \PYG{o+ow}{in} \PYG{n}{MQ1} \PYG{n}{parm}
\PYG{n}{MQ2}\PYG{o}{=}\PYG{p}{(}\PYG{n}{CSQ}\PYG{o}{\PYGZpc{}}\PYG{p}{,}\PYG{l+s+s1}{\PYGZsq{}}\PYG{l+s+s1}{A}\PYG{l+s+si}{\PYGZpc{}\PYGZpc{}}\PYG{l+s+s1}{\PYGZpc{}}\PYG{l+s+s1}{\PYGZsq{}}\PYG{p}{)}\PYG{p}{,}
\PYG{n}{VTOVER}\PYG{o}{=}\PYG{n}{VTDYNAM}\PYG{p}{,} \PYG{o}{\PYGZhy{}}\PYG{o}{\PYGZgt{}} \PYG{n}{new} \PYG{n}{VIRTCT} \PYG{n}{parm}
\PYG{n}{VTDYNAM} \PYG{n}{VTOVERH} \PYG{o}{\PYGZhy{}}\PYG{o}{\PYGZgt{}} \PYG{n}{new} \PYG{n}{table} \PYG{n}{after} \PYG{n}{the} \PYG{n}{VIRTCT}
\PYG{n}{MQ1} \PYG{n}{VTOVER} \PYG{n}{PARM}\PYG{o}{=}\PYG{n}{MQ1}\PYG{p}{,} \PYG{n}{modify} \PYG{n}{MQ1}\PYG{p}{(}\PYG{l+m+mi}{1}\PYG{p}{)} \PYG{o}{*}
\PYG{n}{TARGET}\PYG{o}{=}\PYG{l+s+s1}{\PYGZsq{}}\PYG{l+s+s1}{\PYGZpc{}}\PYG{l+s+s1}{\PYGZsq{}}\PYG{p}{,} \PYG{n}{find} \PYG{o}{\PYGZpc{}} \PYG{n}{char} \PYG{o}{*}
\PYG{n}{FROM}\PYG{o}{=}\PYG{l+m+mi}{0}\PYG{p}{,} \PYG{n}{replace} \PYG{o}{\PYGZpc{}} \PYG{k}{with} \PYG{n}{VTOVER}\PYG{p}{(}\PYG{l+m+mi}{0}\PYG{p}{)} \PYG{o}{*}
\PYG{n}{ERRORC}\PYG{o}{=}\PYG{l+m+mi}{12} \PYG{n}{Virtel} \PYG{n}{RC} \PYG{k}{if} \PYG{n}{replace} \PYG{n}{failed}
\PYG{n}{MQ21} \PYG{n}{VTOVER} \PYG{n}{PARM}\PYG{o}{=}\PYG{p}{(}\PYG{n}{MQ2}\PYG{p}{,}\PYG{l+m+mi}{1}\PYG{p}{)}\PYG{p}{,}\PYG{n}{TARGET}\PYG{o}{=}\PYG{l+s+s1}{\PYGZsq{}}\PYG{l+s+s1}{\PYGZpc{}}\PYG{l+s+s1}{\PYGZsq{}}\PYG{p}{,}\PYG{n}{FROM}\PYG{o}{=}\PYG{l+m+mi}{1}
\PYG{n}{MQ22} \PYG{n}{VTOVER} \PYG{n}{PARM}\PYG{o}{=}\PYG{p}{(}\PYG{n}{MQ2}\PYG{p}{,}\PYG{l+m+mi}{2}\PYG{p}{)}\PYG{p}{,}\PYG{n}{TARGET}\PYG{o}{=}\PYG{l+s+s1}{\PYGZsq{}}\PYG{l+s+si}{\PYGZpc{}\PYGZpc{}}\PYG{l+s+s1}{\PYGZpc{}}\PYG{l+s+s1}{\PYGZsq{}}\PYG{p}{,}\PYG{n}{FROM}\PYG{o}{=}\PYG{l+m+mi}{2}
\end{sphinxVerbatim}

\sphinxAtStartPar
In the JOB:

\begin{sphinxVerbatim}[commandchars=\\\{\}]
\PYG{o}{/}\PYG{o}{/}\PYG{n}{VIR0000} \PYG{n}{EXEC} \PYG{n}{SPVIR5}\PYG{p}{,}\PYG{n}{APPLID}\PYG{o}{=}\PYG{l+s+s1}{\PYGZsq{}}\PYG{l+s+s1}{SP3VIR5}\PYG{l+s+s1}{\PYGZsq{}}\PYG{p}{,}\PYG{n}{VTOVER}\PYG{o}{=}\PYG{l+s+s1}{\PYGZsq{}}\PYG{l+s+s1}{67BCD}\PYG{l+s+s1}{\PYGZsq{}}
\end{sphinxVerbatim}

\sphinxAtStartPar
At execution time:

\begin{sphinxVerbatim}[commandchars=\\\{\}]
\PYG{n}{VIRQ903W} \PYG{n}{LINE} \PYG{n}{lin1name} \PYG{n}{HAS} \PYG{n}{A} \PYG{n}{SESSION} \PYG{n}{STARTED} \PYG{n}{WITH} \PYG{n}{MQM} \PYG{n}{CSQ7}
\PYG{n}{VIRQ923E} \PYG{n}{lin1name} \PYG{n}{REQ} \PYG{n}{MQOPEN} \PYG{n}{COMPLETION} \PYG{n}{CODE} \PYG{l+m+mi}{00000002} \PYG{n}{REASON} \PYG{n}{CODE} \PYG{l+m+mi}{00000825} \PYG{p}{(}\PYG{l+m+mi}{00002085}\PYG{p}{)} \PYG{n}{MQM} \PYG{n}{CSQ7}
\PYG{n}{VIRQ923E} \PYG{n}{lin1name} \PYG{n}{PARAM} \PYG{n}{ABCD}\PYG{o}{.}\PYG{n}{VIRTELOUT}
\PYG{n}{VIRRW01I} \PYG{n}{INITIALISATION} \PYG{n}{FOR} \PYG{n}{lin2name} \PYG{p}{(}\PYG{n}{MQI}\PYG{o}{\PYGZhy{}}\PYG{n}{XX} \PYG{p}{)}\PYG{p}{,} \PYG{n}{VERSION} \PYG{l+m+mf}{4.61}
\PYG{n}{VIRQ903W} \PYG{n}{LINE} \PYG{n}{lin2name} \PYG{n}{HAS} \PYG{n}{A} \PYG{n}{SESSION} \PYG{n}{STARTED} \PYG{n}{WITH} \PYG{n}{MQM} \PYG{n}{CSQ6}
\end{sphinxVerbatim}

\begin{sphinxadmonition}{note}{Note:}
\sphinxAtStartPar
The value specified must be placed in the third position of the PARM card.
\end{sphinxadmonition}

\index{JCL Parameters@\spxentry{JCL Parameters}!IP parameter.@\spxentry{IP parameter.}}\index{IP parameter.@\spxentry{IP parameter.}!JCL Parameters@\spxentry{JCL Parameters}}\ignorespaces 

\subsubsection{IP parameter}
\label{\detokenize{Installation_Guide:ip-parameter}}\label{\detokenize{Installation_Guide:index-14}}
\sphinxAtStartPar
Currently the IP address used by VIRTEL for a particular line can be derived from being:\sphinxhyphen{}
\begin{enumerate}
\sphinxsetlistlabels{\arabic}{enumi}{enumii}{}{.}%
\item {} 
\sphinxAtStartPar
Explicitly defined in the LINE definition in the ARBO statements

\item {} 
\sphinxAtStartPar
Defaults to the IP stack HOME address.

\end{enumerate}

\sphinxAtStartPar
The TCP/IP GETHOSTID function is used to obtain this address. This change implements the possibility to override
option (2) with the ability to specify the IP address as a keyword in the JCL PARM field. As an example:\sphinxhyphen{}

\begin{sphinxVerbatim}[commandchars=\\\{\}]
\PYG{o}{/}\PYG{o}{/}\PYG{n}{S01} \PYG{n}{EXEC} \PYG{n}{PGM}\PYG{o}{=}\PYG{n}{VIR6000}\PYG{p}{,}\PYG{n}{PARM}\PYG{o}{=}\PYG{l+s+s1}{\PYGZsq{}}\PYG{l+s+s1}{01,MYAPPL,,192.168.1.123}\PYG{l+s+s1}{\PYGZsq{}}
\end{sphinxVerbatim}

\sphinxAtStartPar
This reduces the need to specify the HOME address in the ARBO for inbound lines thereby reducing the coupling between the various VIRTEL instances that could be running within a complex and the ARBO structures. Inbound address can just define the port via the :port structure only rather than the full nnn.nnn.nnn.nnn:port specification. The IP= keyword will provide the nnn.nnn.nnn.nnn address structure for a particular instance of Virtel. So one ARBO file could provide common port addresses and the VIRTEL instance complements this with a specific IP address using the JCL IP= parameter. This also allows VIRTEL to utilize a multi TCP/IP stack environment without the need for duplicated ARBO files. This value can be placed ine the fourth position of the PARM card.

\index{JCL Parameters@\spxentry{JCL Parameters}!CLONE parameter.@\spxentry{CLONE parameter.}}\index{CLONE parameter.@\spxentry{CLONE parameter.}!JCL Parameters@\spxentry{JCL Parameters}}\ignorespaces 

\subsubsection{CLONE parameter}
\label{\detokenize{Installation_Guide:clone-parameter}}\label{\detokenize{Installation_Guide:index-15}}
\sphinxAtStartPar
Currently, VIRTEL makes use of the System Symbolic \&SYSCLONE to enable substitution of the “+” character with the two character symbolic value of the System Symbolic. This can be used with the TCT APPLID field and terminal relay names defined in the ARBO. The purpose is to facilitate the common use of an ARBO file across multiple instances of VIRTEL, however, this feature is restricted to supporting only one instance of VIRTEL per LPAR. When multiple instances are required on any one LPAR the System Symbolic \&SYSCLONE and SYSPLUS=YES feature do not provide sufficient uniqueness, consequently multiple ARBO files are required. This feature endeavours to remove the restriction by providing an override through the use of the CLONE=nn in the JCL parameter. When specified, the CLONE value will override the IBM system symbolic value and will be used to replace the “plus” character as defined in the APPLID or terminal relay names. CLONE value must be 2 characters. JCL example:\sphinxhyphen{}

\begin{sphinxVerbatim}[commandchars=\\\{\}]
\PYG{o}{/}\PYG{o}{/}\PYG{n}{S01} \PYG{n}{EXEC} \PYG{n}{PGM}\PYG{o}{=}\PYG{n}{VIR6000}\PYG{p}{,}\PYG{n}{PARM}\PYG{o}{=}\PYG{l+s+s1}{\PYGZsq{}}\PYG{l+s+s1}{EH,,,192.168.170.30,00}\PYG{l+s+s1}{\PYGZsq{}}
\end{sphinxVerbatim}

\sphinxAtStartPar
This will start Virtel with the TCT called VIRTCTEH, use a default home address of 192.168.170.30 and override and “+” character with the value “00”. The APPLID=APPLEH+ keyword, as defined in the TCT, will become APPLID=APPLEH00. The CLONE= value replaces the IBM symbolic value, consequently the SYSCLONE\sphinxhyphen{}SYMBOL within scenario statements will now represent the JCL CLONE= value in scenario statements such as:\sphinxhyphen{}

\begin{sphinxVerbatim}[commandchars=\\\{\}]
\PYG{n}{VALUE}\PYG{o}{\PYGZhy{}}\PYG{n}{OF} \PYG{p}{(}\PYG{n}{SYSCLONE}\PYG{o}{\PYGZhy{}}\PYG{n}{SYMBOL}\PYG{p}{)}
\end{sphinxVerbatim}

\sphinxAtStartPar
or

\begin{sphinxVerbatim}[commandchars=\\\{\}]
COPY\PYGZdl{} SYSTEM\PYGZhy{}TO\PYGZhy{}VARIABLE,VAR=\PYGZsq{}VAR1\PYGZsq{},                    *
  FIELD=(VALUE\PYGZhy{}OF,SYSCLONE\PYGZhy{}SYMBOL)
\end{sphinxVerbatim}

\sphinxAtStartPar
The CLONE= value will also override any \&SYSCLONE symbolic that may be specified in dataset names within the TCT. For example:\sphinxhyphen{}

\begin{sphinxVerbatim}[commandchars=\\\{\}]
\PYG{n}{STATDSN}\PYG{o}{=}\PYG{p}{(}\PYG{n}{HLQ}\PYG{o}{.}\PYG{n}{VIRTEL}\PYG{o}{.}\PYG{n}{SYS}\PYG{o}{\PYGZam{}}\PYG{o}{\PYGZam{}}\PYG{n}{SYSCLONE}\PYG{o}{.}\PYG{o}{.}\PYG{n}{STATA}\PYG{p}{,}           \PYG{n}{STATS}\PYG{o}{=}\PYG{n}{MULTI}       \PYG{o}{*}
\PYG{n}{HLQ}\PYG{o}{.}\PYG{n}{VIRTEL}\PYG{o}{.}\PYG{n}{SYS}\PYG{o}{\PYGZam{}}\PYG{o}{\PYGZam{}}\PYG{n}{SYSCLONE}\PYG{o}{.}\PYG{o}{.}\PYG{n}{STATB}\PYG{p}{)}\PYG{p}{,}                   \PYG{n}{STATS}\PYG{o}{=}\PYG{n}{MULTI}       \PYG{o}{*}
\end{sphinxVerbatim}

\sphinxAtStartPar
The STATDSN keyword as defined in the TCT will allocate and use datasets:\sphinxhyphen{}

\begin{sphinxVerbatim}[commandchars=\\\{\}]
\PYG{n}{HLQ}\PYG{o}{.}\PYG{n}{VIRTEL}\PYG{o}{.}\PYG{n}{SYS00}\PYG{o}{.}\PYG{n}{STATA} \PYG{o+ow}{and} \PYG{n}{HLQ}\PYG{o}{.}\PYG{n}{VIRTEL}\PYG{o}{.}\PYG{n}{SYS00}\PYG{o}{.}\PYG{n}{STATB}\PYG{o}{.}
\end{sphinxVerbatim}


\subsection{Executing Virtel as a Started Task}
\label{\detokenize{Installation_Guide:executing-virtel-as-a-started-task}}
\sphinxAtStartPar
VIRTEL is started by executing the command S VIRTEL from the system console. Message VIR0000I indicates that the product started properly.


\subsection{Stopping Virtel}
\label{\detokenize{Installation_Guide:stopping-virtel}}
\sphinxAtStartPar
VIRTEL may be stopped by issuing the following command:\sphinxhyphen{}

\begin{sphinxVerbatim}[commandchars=\\\{\}]
\PYG{n}{P} \PYG{n}{VIRTEL}
\end{sphinxVerbatim}

\index{Installing under z/VSE@\spxentry{Installing under z/VSE}}\ignorespaces 

\chapter{Installing under z/VSE}
\label{\detokenize{Installation_Guide:installing-under-z-vse}}\label{\detokenize{Installation_Guide:index-16}}
\index{Installing under z/VSE@\spxentry{Installing under z/VSE}!z/VSE Check list@\spxentry{z/VSE Check list}}\index{z/VSE Check list@\spxentry{z/VSE Check list}!Installing under z/VSE@\spxentry{Installing under z/VSE}}\ignorespaces 

\section{z/VSE Check List}
\label{\detokenize{Installation_Guide:z-vse-check-list}}\label{\detokenize{Installation_Guide:index-17}}
\sphinxAtStartPar
Installation of VIRTEL under z/VSE consists of the following steps. Each step is described in detail in the sections which follow.
\begin{itemize}
\item {} 
\sphinxAtStartPar
Load the installation jobs into the POWER READER QUEUE

\item {} 
\sphinxAtStartPar
Define the VIRTvrr.SUBLIB sublibrary

\item {} 
\sphinxAtStartPar
Load the CIL and SSL libraries

\item {} 
\sphinxAtStartPar
Define the files VIRARBO, VIRSWAP and VIRSTAT

\item {} 
\sphinxAtStartPar
Define the files VIRCMP3, VIRCAPT and SAMPTRF

\item {} 
\sphinxAtStartPar
Define the files HTMLTRF and VIRHTML

\item {} 
\sphinxAtStartPar
Assemble the VIRTCT

\item {} 
\sphinxAtStartPar
Assemble the VTAM mode table

\item {} 
\sphinxAtStartPar
Update the VIRARBO file (ARBOLOAD)

\item {} 
\sphinxAtStartPar
Define the VTAM application relays

\item {} 
\sphinxAtStartPar
Define the VIRTEL start procedure

\end{itemize}

\index{Installing under z/VSE@\spxentry{Installing under z/VSE}!Loading the installation job@\spxentry{Loading the installation job}}\index{Loading the installation job@\spxentry{Loading the installation job}!Installing under z/VSE@\spxentry{Installing under z/VSE}}\ignorespaces 

\subsection{Loading the installation jobs}
\label{\detokenize{Installation_Guide:loading-the-installation-jobs}}\label{\detokenize{Installation_Guide:index-18}}
\sphinxAtStartPar
The installation jobs are delivered in a file format that can be mounted on a VTS tape as a 3480 tape cartridge. To load the installation jobs into the POWER reader queue, enter the command S RDR,cuu at the z/VSE console (where cuu represents the address of the tape drive on which you have mounted the cartridge). The following jobs will be loaded into your Reader:\sphinxhyphen{}

\begin{sphinxVerbatim}[commandchars=\\\{\}]
\PYG{n}{Queue} \PYG{k}{with} \PYG{n}{DISP}\PYG{o}{=}\PYG{n}{L}\PYG{p}{,} \PYG{n}{CLASS}\PYG{o}{=}\PYG{l+m+mi}{0}\PYG{p}{:}
\end{sphinxVerbatim}


\begin{savenotes}\sphinxattablestart
\sphinxthistablewithglobalstyle
\centering
\begin{tabulary}{\linewidth}[t]{TT}
\sphinxtoprule
\sphinxstyletheadfamily 
\sphinxAtStartPar
Module
&\sphinxstyletheadfamily 
\sphinxAtStartPar
Description
\\
\sphinxmidrule
\sphinxtableatstartofbodyhook
\sphinxAtStartPar
VIRTLIB
&
\sphinxAtStartPar
define the VIRTvrr.SUBLIB sublibrary
\\
\sphinxhline
\sphinxAtStartPar
VIRTCIL
&
\sphinxAtStartPar
load executable modules into the CIL
\\
\sphinxhline
\sphinxAtStartPar
VIRTSSL
&
\sphinxAtStartPar
load source modules into the SSL
\\
\sphinxhline
\sphinxAtStartPar
VIRSAPI
&
\sphinxAtStartPar
load the VIRAPI macro library
\\
\sphinxhline
\sphinxAtStartPar
VIRFA29
&
\sphinxAtStartPar
load the FA29 macro library
\\
\sphinxhline
\sphinxAtStartPar
VIRSAPI
&
\sphinxAtStartPar
load the SCRNAPI macro library
\\
\sphinxhline\sphinxmultirow{9}{15}{%
\begin{varwidth}[t]{\sphinxcolwidth{1}{2}}
\sphinxAtStartPar
VIRTVS
\par
\vskip-\baselineskip\vbox{\hbox{\strut}}\end{varwidth}%
}%
&
\sphinxAtStartPar
\sphinxstylestrong{VIRTVS1} \sphinxhyphen{} define VIRARBO and VIRSWAP files
\\
\sphinxcline{2-2}\sphinxfixclines{2}\sphinxtablestrut{15}&
\sphinxAtStartPar
\sphinxstylestrong{VIRTVS2} \sphinxhyphen{} initialise VIRARBO file
\\
\sphinxcline{2-2}\sphinxfixclines{2}\sphinxtablestrut{15}&
\sphinxAtStartPar
\sphinxstylestrong{VIRTVS3} \sphinxhyphen{}define VIRSTAT file
\\
\sphinxcline{2-2}\sphinxfixclines{2}\sphinxtablestrut{15}&
\sphinxAtStartPar
\sphinxstylestrong{VIRTVS4} \sphinxhyphen{} define VIRCMP3 file
\\
\sphinxcline{2-2}\sphinxfixclines{2}\sphinxtablestrut{15}&
\sphinxAtStartPar
\sphinxstylestrong{VIRTVS5} \sphinxhyphen{} define VIRCAPT file
\\
\sphinxcline{2-2}\sphinxfixclines{2}\sphinxtablestrut{15}&
\sphinxAtStartPar
\sphinxstylestrong{VIRTVS6} \sphinxhyphen{} define SAMPTRF file
\\
\sphinxcline{2-2}\sphinxfixclines{2}\sphinxtablestrut{15}&
\sphinxAtStartPar
\sphinxstylestrong{VIRTVS7} \sphinxhyphen{} define HTMLTRF file
\\
\sphinxcline{2-2}\sphinxfixclines{2}\sphinxtablestrut{15}&
\sphinxAtStartPar
\sphinxstylestrong{VIRTVS8} \sphinxhyphen{} load SAMPTRF file
\\
\sphinxcline{2-2}\sphinxfixclines{2}\sphinxtablestrut{15}&
\sphinxAtStartPar
\sphinxstylestrong{VIRTVS9} \sphinxhyphen{} define VIRHTML file
\\
\sphinxhline
\sphinxAtStartPar
VIRTCT
&
\sphinxAtStartPar
VIRTEL parameter table assembly example
\\
\sphinxhline
\sphinxAtStartPar
VIRCONF
&
\sphinxAtStartPar
VIRARBO batch update (ARBOLOAD)
\\
\sphinxhline
\sphinxAtStartPar
VIRMOD
&
\sphinxAtStartPar
VTAM mode table assembly
\\
\sphinxhline
\sphinxAtStartPar
VIRTAPPL
&
\sphinxAtStartPar
VTAM application major node example
\\
\sphinxhline
\sphinxAtStartPar
VIRGROUP
&
\sphinxAtStartPar
CICS resource definition example
\\
\sphinxhline
\sphinxAtStartPar
VIRTEL
&
\sphinxAtStartPar
VIRTEL execution JCL example
\\
\sphinxbottomrule
\end{tabulary}
\sphinxtableafterendhook\par
\sphinxattableend\end{savenotes}

\begin{sphinxadmonition}{note}{Note:}
\sphinxAtStartPar
You will need to modify certain of the installation jobs before submitting them. Once the jobs have been read onto the POWER queue, you can copy them to an ICCF library (using ICCF option 3224 Operations \sphinxhyphen{} Manage Batch Queues \textendash{} Input Queue \textendash{} Copy to Primary Library) or read them into your VM machine for editing.
\end{sphinxadmonition}


\subsection{Sites installing VIRTEL for the first time}
\label{\detokenize{Installation_Guide:sites-installing-virtel-for-the-first-time}}
\sphinxAtStartPar
Jobs VIRTLIB, VIRTCIL, VIRTSSL, VIRTVS, VIRTCT, VIRMOD, and VIRTAPPL must be executed as described below.


\subsection{Sites upgrading from a previous version}
\label{\detokenize{Installation_Guide:sites-upgrading-from-a-previous-version}}
\sphinxAtStartPar
Execute jobs VIRTLIB, VIRTCIL and VIRTSSL to create a new VIRTvrr.SUBLIB. Change your VIRTEL execution JCL to reference the new sublibrary You can retain your existing VSAM files except SAMPTRSF. \sphinxstylestrong{You must install the new version of SAMPTRSF} as delivered with the new version of Virtel.


\subsection{Sites using VIRTEL Web Access}
\label{\detokenize{Installation_Guide:sites-using-virtel-web-access}}
\sphinxAtStartPar
The files required for VIRTEL Web Access base functions are loaded in steps VIRTVS6, VIRTVS7, VIRTVS8, and VIRTVS9 of job VIRTVS. If you wish to use VIRTEL Host\sphinxhyphen{}Web Services to script your 3270 applications, run job VIRSAPI also. See Host\sphinxhyphen{}Web\sphinxhyphen{}Services in the Virtel User Guide for further information.


\subsection{Sites using VIRTEL A2A}
\label{\detokenize{Installation_Guide:sites-using-virtel-a2a}}
\sphinxAtStartPar
Customers wishing to use VIRTEL Application\sphinxhyphen{}to\sphinxhyphen{}Application functions should also run jobs VIRFA29 and VIRAPI.


\subsection{Defining the library}
\label{\detokenize{Installation_Guide:defining-the-library}}
\begin{sphinxVerbatim}[commandchars=\\\{\}]
* \PYGZdl{}\PYGZdl{} JOB JNM=VIRTLIB,CLASS=0,DISP=L
* \PYGZdl{}\PYGZdl{} LST DA
// JOB VIRTLIB
* *****************************************************************
* * VIRTLIB * CREATE VIRTvrr LIBRARY                              *
* *****************************************************************
* *                                                               *
* * THIS JOB IS SUPPLIED AS AN EXAMPLE ONLY AND MUST BE MODIFIED  *
* * BEFORE EXECUTION                                              *
* *                                                               *
* *****************************************************************
// EXEC IDCAMS,SIZE=AUTO
 DELETE (VSE.VIRTvrr.LIBRARY ) \PYGZhy{}
          CLUSTER \PYGZhy{}
          PURGE \PYGZhy{}
    CATALOG (VSESP.USER.CATALOG )
 SET MAXCC=0
 DEFINE CLUSTER ( \PYGZhy{}
            NAME (VSE.VIRTvrr.LIBRARY ) \PYGZhy{}
            TRACKS (150 25) \PYGZhy{}
            SHAREOPTIONS (3) \PYGZhy{}
            RECORDFORMAT (NOCIFORMAT) \PYGZhy{}
            VOLUMES (SYSWK1) \PYGZhy{}
            NOREUSE \PYGZhy{}
            NONINDEXED \PYGZhy{}
            TO (99366)) \PYGZhy{}
            DATA (NAME (VSE.VIRTvrr.LIBRARY.DATA ) ) \PYGZhy{}
            CATALOG (VSESP.USER.CATALOG )
 IF LASTCC NE 0 THEN CANCEL JOB
/*
// OPTION STDLABEL=ADD
// DLBL VIRTvrr,\PYGZsq{}VSE.VIRTvrr.LIBRARY\PYGZsq{},,VSAM,CAT=VSESPUC
/*
// EXEC IESVCLUP,SIZE=AUTO
A VSE.VIRTvrr.LIBRARY        VIRTvrr VSESPUC OLD KEEP
/*
// EXEC LIBR,PARM=\PYGZsq{}MSHP\PYGZsq{}
              DEFINE LIB=VIRTvrr REPLACE=YES
              DEFINE SUBLIB=VIRTvrr.SUBLIB REPLACE=YES
/*
/\PYGZam{}
* \PYGZdl{}\PYGZdl{} EOJ
\end{sphinxVerbatim}

\sphinxAtStartPar
\sphinxstyleemphasis{VIRTLIB : JCL to define the sublibrary (z/VSE)}

\sphinxAtStartPar
Job VIRTLIB contains an example of JCL to define the library which will contain the VIRTEL executable modules and source books. This job is provided as an example, and may need to be modified prior to execution. The name VIRTnnn.SUBLIB indicates the VIRTEL version, for example VIRTvrr.SUBLIB for version 4.61. Parameters VOLUMES(SYSWK1), and possibly the cluster name and catalog name, may need to be modified.


\subsection{Loading the executable modules}
\label{\detokenize{Installation_Guide:loading-the-executable-modules}}
\begin{sphinxVerbatim}[commandchars=\\\{\}]
* \PYGZdl{}\PYGZdl{} JOB JNM=VIRTCIL,CLASS=0,DISP=L
* \PYGZdl{}\PYGZdl{} LST DA
// JOB VIRTCIL
* *****************************************************************
* * VIRTCIL * CATALOG PROGRAM PHASES IN CORE IMAGE LIBRARY        *
* *****************************************************************
* *                                                               *
* * AT THE PAUSE, ENTER YOUR DLBL AND LIBDEF FOR THE CIL SUBLIB   *
* *                                                               *
* * // DLBL VIRTvrr,\PYGZsq{}VSE.VIRTvrr.LIBRARY\PYGZsq{},,VSAM,CAT=VSESPUC       *
* * // LIBDEF PHASE,CATALOG=VIRTvrr.SUBLIB                        *
* *                                                               *
* *****************************************************************
// PAUSE ENTER YOUR LIBDEF PHASE STATEMENT AS IN THE ABOVE EXAMPLE
// OPTION CATAL
        INCLUDE
        (object modules)
/*
// EXEC LNKEDT,SIZE=512K
/\PYGZam{}
* \PYGZdl{}\PYGZdl{} EOJ
\end{sphinxVerbatim}

\sphinxAtStartPar
\sphinxstyleemphasis{VIRTCIL : JCL to load the executable modules (z/VSE)}

\sphinxAtStartPar
Start the job to load the executable modules by entering the POWER command

\begin{sphinxVerbatim}[commandchars=\\\{\}]
\PYG{n}{R} \PYG{n}{RDR}\PYG{p}{,}\PYG{n}{VIRTCIL}
\end{sphinxVerbatim}

\sphinxAtStartPar
When this job executes, a // PAUSE card will ask you to enter the statements to specify the name of the library into which the modules are to be loaded. Enter

\begin{sphinxVerbatim}[commandchars=\\\{\}]
\PYG{o}{/}\PYG{o}{/} \PYG{n}{DLBL} \PYG{n}{VIRTvrr}\PYG{p}{,}\PYG{l+s+s1}{\PYGZsq{}}\PYG{l+s+s1}{VSE.VIRTvrr.LIBRARY}\PYG{l+s+s1}{\PYGZsq{}}\PYG{p}{,}\PYG{p}{,}\PYG{n}{VSAM}\PYG{p}{,}\PYG{n}{CAT}\PYG{o}{=}\PYG{n}{VSESPUC}
\PYG{o}{/}\PYG{o}{/} \PYG{n}{LIBDEF} \PYG{n}{PHASE}\PYG{p}{,}\PYG{n}{CATALOG}\PYG{o}{=}\PYG{n}{xxxxx}
\end{sphinxVerbatim}

\sphinxAtStartPar
where xxxxx represents the name of the sublibrary you defined in the previous job.

\sphinxAtStartPar
As an example here is the commands entered in a VSE system:\sphinxhyphen{}

\begin{sphinxVerbatim}[commandchars=\\\{\}]
\PYG{n}{BG}\PYG{o}{\PYGZhy{}}\PYG{l+m+mi}{0000} \PYG{o}{/}\PYG{o}{/} \PYG{n}{PAUSE} \PYG{n}{ENTER} \PYG{n}{YOUR} \PYG{n}{LIBDEF} \PYG{n}{PHASE} \PYG{n}{STATEMENT} \PYG{n}{AS} \PYG{n}{IN} \PYG{n}{THE} \PYG{n}{ABOVE} \PYG{n}{EXAMPLE}
\PYG{l+m+mi}{0} \PYG{o}{/}\PYG{o}{/} \PYG{n}{DLBL} \PYG{n}{VIRTEL}\PYG{p}{,}\PYG{l+s+s1}{\PYGZsq{}}\PYG{l+s+s1}{VSE.VIRTEL.LIBRARY}\PYG{l+s+s1}{\PYGZsq{}}\PYG{p}{,}\PYG{p}{,}\PYG{n}{VSAM}\PYG{p}{,}\PYG{n}{CAT}\PYG{o}{=}\PYG{n}{SYSPUC1}           \PYG{o}{\PYGZlt{}}\PYG{n}{USER} \PYG{n}{INPUT}\PYG{o}{\PYGZgt{}}
\PYG{n}{BG}\PYG{o}{\PYGZhy{}}\PYG{l+m+mi}{0000}
\PYG{l+m+mi}{0} \PYG{o}{/}\PYG{o}{/} \PYG{n}{LIBDEF} \PYG{n}{PHASE}\PYG{p}{,}\PYG{n}{CATALOG}\PYG{o}{=}\PYG{n}{VIRTEL}\PYG{o}{.}\PYG{n}{SUBLIB}                                                         \PYG{o}{\PYGZlt{}}\PYG{n}{USER} \PYG{n}{INPUT}\PYG{o}{\PYGZgt{}}
\PYG{n}{BG}\PYG{o}{\PYGZhy{}}\PYG{l+m+mi}{0000}
\PYG{l+m+mi}{0}                                                                   \PYG{o}{\PYGZlt{}}\PYG{n}{USER} \PYG{n}{INPUT}\PYG{o}{\PYGZgt{}}
\PYG{n}{BG} \PYG{l+m+mi}{0000} \PYG{o}{/}\PYG{o}{/} \PYG{n}{OPTION} \PYG{n}{CATAL}
\PYG{n}{BG} \PYG{l+m+mi}{0000}  \PYG{n}{INCLUDE}
\PYG{n}{BG} \PYG{l+m+mi}{0000} \PYG{o}{/}\PYG{o}{/} \PYG{n}{EXEC} \PYG{n}{LNKEDT}\PYG{p}{,}\PYG{n}{SIZE}\PYG{o}{=}\PYG{l+m+mi}{512}\PYG{n}{K}
\PYG{n}{BG} \PYG{l+m+mi}{0000} \PYG{n}{EOJ} \PYG{n}{VIRTCIL}   \PYG{n}{MAX}\PYG{o}{.}\PYG{n}{RETURN} \PYG{n}{CODE}\PYG{o}{=}\PYG{l+m+mi}{0000}
\end{sphinxVerbatim}


\subsection{Loading the source modules}
\label{\detokenize{Installation_Guide:loading-the-source-modules}}
\begin{sphinxVerbatim}[commandchars=\\\{\}]
* \PYGZdl{}\PYGZdl{} JOB JNM=VIRTSSL,CLASS=0,DISP=L
* \PYGZdl{}\PYGZdl{} LST DA
// JOB VIRTSSL
* *****************************************************************
* * VIRTSSL * CATALOG SOURCE BOOKS IN SSL                         *
* *****************************************************************
* *                                                               *
* * AT THE PAUSE, ENTER THE NAME OF THE SUB\PYGZhy{}LIBRARY               *
* * FOR CATALOGING THE VIRTEL SOURCE BOOKS                        *
* *                                                               *
* * EXAMPLE: // SETPARM SUB=\PYGZsq{}VIRTvrr.SUBLIB\PYGZsq{}                      *
* *                                                               *
* *****************************************************************
// PAUSE ENTER YOUR SETPARM CARD AS SHOWN ABOVE
// EXEC PGM=LIBR,PARM=\PYGZsq{} ACCESS SUBLIB=\PYGZam{}SUB\PYGZsq{}
        (source books)
/*
/\PYGZam{}
* \PYGZdl{}\PYGZdl{} EOJ
\end{sphinxVerbatim}

\sphinxAtStartPar
\sphinxstyleemphasis{VIRTSSL : JCL to load the source modules (z/VSE)}

\sphinxAtStartPar
Start the job to load the source modules by entering the POWER commands:

\begin{sphinxVerbatim}[commandchars=\\\{\}]
\PYG{n}{R} \PYG{n}{RDR}\PYG{p}{,}\PYG{n}{VIRTSSL}
\PYG{n}{R} \PYG{n}{RDR}\PYG{p}{,}\PYG{n}{VIRFA29}
\PYG{n}{R} \PYG{n}{RDR}\PYG{p}{,}\PYG{n}{VIRAPI}
\PYG{n}{R} \PYG{n}{RDR}\PYG{p}{,}\PYG{n}{VIRSAPI}
\end{sphinxVerbatim}

\sphinxAtStartPar
When these jobs execute, a // PAUSE card will ask you to enter a SETPARM statement specifying the name of the library into which the modules are to be loaded. Enter:

\begin{sphinxVerbatim}[commandchars=\\\{\}]
\PYG{o}{/}\PYG{o}{/} \PYG{n}{SETPARM} \PYG{n}{SUB}\PYG{o}{=}\PYG{l+s+s1}{\PYGZsq{}}\PYG{l+s+s1}{xxxxxxx}\PYG{l+s+s1}{\PYGZsq{}}
\end{sphinxVerbatim}

\sphinxAtStartPar
where  xxxxxxx represents the name of the sublibrary you defined in the first job.

\begin{sphinxVerbatim}[commandchars=\\\{\}]
* \PYGZdl{}\PYGZdl{} JOB JNM=VIRFA29,CLASS=0,DISP=L
* \PYGZdl{}\PYGZdl{} LST DA
// JOB VIRFA29
* *****************************************************************
* * VIRFA29 * CATALOG SOURCE BOOKS FOR FA29 API                   *
* *****************************************************************
* *                                                               *
* * AT THE PAUSE, ENTER THE NAME OF THE SUB\PYGZhy{}LIBRARY               *
* * FOR THE FA29 MACRO SOURCE BOOKS                               *
* *                                                               *
* * EXAMPLE: // SETPARM SUB=\PYGZsq{}VIRTvrr.SUBLIB\PYGZsq{}                      *
* *                                                               *
* *****************************************************************
// PAUSE ENTER YOUR SETPARM CARD AS SHOWN ABOVE
// EXEC PGM=LIBR,PARM=\PYGZsq{} ACCESS SUBLIB=\PYGZam{}SUB\PYGZsq{}
        (FA29API source books)
/*
/\PYGZam{}
* \PYGZdl{}\PYGZdl{} EOJ
\end{sphinxVerbatim}

\sphinxAtStartPar
\sphinxstyleemphasis{VIRFA29 : JCL to load the FA29 macros (z/VSE)}

\begin{sphinxVerbatim}[commandchars=\\\{\}]
* \PYGZdl{}\PYGZdl{} JOB JNM=VIRAPI,CLASS=0,DISP=L
* \PYGZdl{}\PYGZdl{} LST DA
// JOB VIRAPI
* *****************************************************************
* * VIRAPI * CATALOG SOURCE BOOKS FOR VIRAPI                      *
* *****************************************************************
* *                                                               *
* * AT THE PAUSE, ENTER THE NAME OF THE SUB\PYGZhy{}LIBRARY               *
* * FOR THE VIRAPI MACRO SOURCE BOOKS                             *
* *                                                               *
* * EXAMPLE: // SETPARM SUB=\PYGZsq{}VIRTvrr.SUBLIB\PYGZsq{}                      *
* *                                                               *
* *****************************************************************
// PAUSE ENTER YOUR SETPARM CARD AS SHOWN ABOVE
// EXEC PGM=LIBR,PARM=\PYGZsq{} ACCESS SUBLIB=\PYGZam{}SUB\PYGZsq{}
        (VIRAPI source books)
/*
/\PYGZam{}
* \PYGZdl{}\PYGZdl{} EOJ
\end{sphinxVerbatim}

\sphinxAtStartPar
\sphinxstyleemphasis{VIRAPI : JCL to load the VIRAPI macros (z/VSE)}

\begin{sphinxVerbatim}[commandchars=\\\{\}]
* \PYGZdl{}\PYGZdl{} JOB JNM=VIRSAPI,CLASS=0,DISP=L
* \PYGZdl{}\PYGZdl{} LST DA
// JOB VIRSAPI
* *****************************************************************
* * VIRSAPI * CATALOG SOURCE BOOKS FOR SCRNAPI                    *
* *****************************************************************
* *                                                               *
* * AT THE PAUSE, ENTER THE NAME OF THE SUB\PYGZhy{}LIBRARY               *
* * FOR THE SCRNAPI MACRO SOURCE BOOKS                            *
* *                                                               *
* * EXAMPLE: // SETPARM SUB=\PYGZsq{}VIRTvrr.SUBLIB\PYGZsq{}                      *
* *                                                               *
* *****************************************************************
// PAUSE ENTER YOUR SETPARM CARD AS SHOWN ABOVE
// EXEC PGM=LIBR,PARM=\PYGZsq{} ACCESS SUBLIB=\PYGZam{}SUB\PYGZsq{}
        (SCRNAPI source books)
/*
/\PYGZam{}
* \PYGZdl{}\PYGZdl{} EOJ
\end{sphinxVerbatim}

\sphinxAtStartPar
\sphinxstyleemphasis{VIRAPI : JCL to load the SCRNAPI macros (z/VSE)}


\subsection{Defining the VIRARBO and VIRSWAP files}
\label{\detokenize{Installation_Guide:defining-the-virarbo-and-virswap-files}}
\begin{sphinxVerbatim}[commandchars=\\\{\}]
\PYG{o}{/}\PYG{o}{/} \PYG{n}{JOB} \PYG{n}{VIRTVS}
\PYG{o}{/}\PYG{o}{/} \PYG{n}{SETPARM} \PYG{n}{TAPE}\PYG{o}{=}\PYG{l+m+mi}{590}
\PYG{o}{*} \PYG{o}{*}\PYG{o}{*}\PYG{o}{*}\PYG{o}{*}\PYG{o}{*}\PYG{o}{*}\PYG{o}{*}\PYG{o}{*}\PYG{o}{*}\PYG{o}{*}\PYG{o}{*}\PYG{o}{*}\PYG{o}{*}\PYG{o}{*}\PYG{o}{*}\PYG{o}{*}\PYG{o}{*}\PYG{o}{*}\PYG{o}{*}\PYG{o}{*}\PYG{o}{*}\PYG{o}{*}\PYG{o}{*}\PYG{o}{*}\PYG{o}{*}\PYG{o}{*}\PYG{o}{*}\PYG{o}{*}\PYG{o}{*}\PYG{o}{*}\PYG{o}{*}\PYG{o}{*}\PYG{o}{*}\PYG{o}{*}\PYG{o}{*}\PYG{o}{*}\PYG{o}{*}\PYG{o}{*}\PYG{o}{*}\PYG{o}{*}\PYG{o}{*}\PYG{o}{*}\PYG{o}{*}\PYG{o}{*}\PYG{o}{*}\PYG{o}{*}\PYG{o}{*}\PYG{o}{*}\PYG{o}{*}\PYG{o}{*}\PYG{o}{*}\PYG{o}{*}\PYG{o}{*}\PYG{o}{*}\PYG{o}{*}\PYG{o}{*}\PYG{o}{*}\PYG{o}{*}\PYG{o}{*}\PYG{o}{*}\PYG{o}{*}\PYG{o}{*}\PYG{o}{*}\PYG{o}{*}\PYG{o}{*}
\PYG{o}{*} \PYG{o}{*} \PYG{n}{AT} \PYG{n}{THE} \PYG{n}{PAUSE}\PYG{p}{,} \PYG{n}{ENTER} \PYG{n}{THE} \PYG{n}{UNIT} \PYG{n}{ADDRESS} \PYG{n}{OF} \PYG{n}{THE} \PYG{n}{TAPE} \PYG{n}{DRIVE} \PYG{o}{*}
\PYG{o}{*} \PYG{o}{*} \PYG{n}{FOR} \PYG{n}{THE} \PYG{n}{VIRTEL} \PYG{n}{INSTALLATION} \PYG{n}{TAPE} \PYG{o}{*}
\PYG{o}{*} \PYG{o}{*} \PYG{o}{*}
\PYG{o}{*} \PYG{o}{*} \PYG{n}{EXAMPLE}\PYG{p}{:} \PYG{o}{/}\PYG{o}{/} \PYG{n}{SETPARM} \PYG{n}{TAPE}\PYG{o}{=}\PYG{l+m+mi}{590} \PYG{o}{*}
\PYG{o}{*} \PYG{o}{*} \PYG{o}{*}
\PYG{o}{*} \PYG{o}{*}\PYG{o}{*}\PYG{o}{*}\PYG{o}{*}\PYG{o}{*}\PYG{o}{*}\PYG{o}{*}\PYG{o}{*}\PYG{o}{*}\PYG{o}{*}\PYG{o}{*}\PYG{o}{*}\PYG{o}{*}\PYG{o}{*}\PYG{o}{*}\PYG{o}{*}\PYG{o}{*}\PYG{o}{*}\PYG{o}{*}\PYG{o}{*}\PYG{o}{*}\PYG{o}{*}\PYG{o}{*}\PYG{o}{*}\PYG{o}{*}\PYG{o}{*}\PYG{o}{*}\PYG{o}{*}\PYG{o}{*}\PYG{o}{*}\PYG{o}{*}\PYG{o}{*}\PYG{o}{*}\PYG{o}{*}\PYG{o}{*}\PYG{o}{*}\PYG{o}{*}\PYG{o}{*}\PYG{o}{*}\PYG{o}{*}\PYG{o}{*}\PYG{o}{*}\PYG{o}{*}\PYG{o}{*}\PYG{o}{*}\PYG{o}{*}\PYG{o}{*}\PYG{o}{*}\PYG{o}{*}\PYG{o}{*}\PYG{o}{*}\PYG{o}{*}\PYG{o}{*}\PYG{o}{*}\PYG{o}{*}\PYG{o}{*}\PYG{o}{*}\PYG{o}{*}\PYG{o}{*}\PYG{o}{*}\PYG{o}{*}\PYG{o}{*}\PYG{o}{*}\PYG{o}{*}\PYG{o}{*}
\PYG{o}{/}\PYG{o}{/} \PYG{n}{PAUSE} \PYG{n}{ENTER} \PYG{n}{YOUR} \PYG{n}{SETPARM} \PYG{n}{CARD} \PYG{n}{AS} \PYG{n}{SHOWN} \PYG{n}{ABOVE}
\PYG{o}{*} \PYG{o}{*}\PYG{o}{*}\PYG{o}{*}\PYG{o}{*}\PYG{o}{*}\PYG{o}{*}\PYG{o}{*}\PYG{o}{*}\PYG{o}{*}\PYG{o}{*}\PYG{o}{*}\PYG{o}{*}\PYG{o}{*}\PYG{o}{*}\PYG{o}{*}\PYG{o}{*}\PYG{o}{*}\PYG{o}{*}\PYG{o}{*}\PYG{o}{*}\PYG{o}{*}\PYG{o}{*}\PYG{o}{*}\PYG{o}{*}\PYG{o}{*}\PYG{o}{*}\PYG{o}{*}\PYG{o}{*}\PYG{o}{*}\PYG{o}{*}\PYG{o}{*}\PYG{o}{*}\PYG{o}{*}\PYG{o}{*}\PYG{o}{*}\PYG{o}{*}\PYG{o}{*}\PYG{o}{*}\PYG{o}{*}\PYG{o}{*}\PYG{o}{*}\PYG{o}{*}\PYG{o}{*}\PYG{o}{*}\PYG{o}{*}\PYG{o}{*}\PYG{o}{*}\PYG{o}{*}\PYG{o}{*}\PYG{o}{*}\PYG{o}{*}\PYG{o}{*}\PYG{o}{*}\PYG{o}{*}\PYG{o}{*}\PYG{o}{*}\PYG{o}{*}\PYG{o}{*}\PYG{o}{*}\PYG{o}{*}\PYG{o}{*}\PYG{o}{*}\PYG{o}{*}\PYG{o}{*}\PYG{o}{*}
\PYG{o}{*} \PYG{o}{*} \PYG{n}{VIRTVS1} \PYG{o}{*} \PYG{n}{DEFINITION} \PYG{n}{OF} \PYG{n}{VIRARBO} \PYG{n}{AND} \PYG{n}{VIRSWAP} \PYG{n}{FILES} \PYG{o}{*}
\PYG{o}{*} \PYG{o}{*}\PYG{o}{*}\PYG{o}{*}\PYG{o}{*}\PYG{o}{*}\PYG{o}{*}\PYG{o}{*}\PYG{o}{*}\PYG{o}{*}\PYG{o}{*}\PYG{o}{*}\PYG{o}{*}\PYG{o}{*}\PYG{o}{*}\PYG{o}{*}\PYG{o}{*}\PYG{o}{*}\PYG{o}{*}\PYG{o}{*}\PYG{o}{*}\PYG{o}{*}\PYG{o}{*}\PYG{o}{*}\PYG{o}{*}\PYG{o}{*}\PYG{o}{*}\PYG{o}{*}\PYG{o}{*}\PYG{o}{*}\PYG{o}{*}\PYG{o}{*}\PYG{o}{*}\PYG{o}{*}\PYG{o}{*}\PYG{o}{*}\PYG{o}{*}\PYG{o}{*}\PYG{o}{*}\PYG{o}{*}\PYG{o}{*}\PYG{o}{*}\PYG{o}{*}\PYG{o}{*}\PYG{o}{*}\PYG{o}{*}\PYG{o}{*}\PYG{o}{*}\PYG{o}{*}\PYG{o}{*}\PYG{o}{*}\PYG{o}{*}\PYG{o}{*}\PYG{o}{*}\PYG{o}{*}\PYG{o}{*}\PYG{o}{*}\PYG{o}{*}\PYG{o}{*}\PYG{o}{*}\PYG{o}{*}\PYG{o}{*}\PYG{o}{*}\PYG{o}{*}\PYG{o}{*}\PYG{o}{*}
\PYG{o}{/}\PYG{o}{/} \PYG{n}{DLBL} \PYG{n}{IJSYSUC}\PYG{p}{,}\PYG{l+s+s1}{\PYGZsq{}}\PYG{l+s+s1}{VSESP.USER.CATALOG}\PYG{l+s+s1}{\PYGZsq{}}\PYG{p}{,}\PYG{p}{,}\PYG{n}{VSAM}
\PYG{o}{/}\PYG{o}{/} \PYG{n}{EXEC} \PYG{n}{IDCAMS}\PYG{p}{,}\PYG{n}{SIZE}\PYG{o}{=}\PYG{n}{AUTO}
  \PYG{n}{DELETE} \PYG{p}{(}\PYG{n}{VIRTEL}\PYG{o}{.}\PYG{n}{ARBO} \PYG{p}{)} \PYG{o}{\PYGZhy{}}
    \PYG{n}{CLUSTER} \PYG{o}{\PYGZhy{}}
    \PYG{n}{PURGE} \PYG{o}{\PYGZhy{}}
    \PYG{n}{CATALOG} \PYG{p}{(}\PYG{n}{VSESP}\PYG{o}{.}\PYG{n}{USER}\PYG{o}{.}\PYG{n}{CATALOG} \PYG{p}{)}
  \PYG{n}{SET} \PYG{n}{MAXCC}\PYG{o}{=}\PYG{l+m+mi}{0}
  \PYG{n}{DEFINE} \PYG{n}{CLUSTER} \PYG{p}{(} \PYG{o}{\PYGZhy{}}
    \PYG{n}{NAME} \PYG{p}{(}\PYG{n}{VIRTEL}\PYG{o}{.}\PYG{n}{ARBO} \PYG{p}{)} \PYG{o}{\PYGZhy{}}
    \PYG{n}{RECORDS}\PYG{p}{(}\PYG{l+m+mi}{500} \PYG{l+m+mi}{100}\PYG{p}{)} \PYG{o}{\PYGZhy{}}
    \PYG{n}{SHAREOPTIONS} \PYG{p}{(}\PYG{l+m+mi}{2} \PYG{l+m+mi}{3}\PYG{p}{)} \PYG{o}{\PYGZhy{}}
    \PYG{n}{RECSZ} \PYG{p}{(}\PYG{l+m+mi}{600} \PYG{l+m+mi}{4089}\PYG{p}{)} \PYG{o}{\PYGZhy{}}
    \PYG{n}{VOLUMES} \PYG{p}{(}\PYG{n}{SYSWK1}\PYG{p}{)} \PYG{o}{\PYGZhy{}}
    \PYG{n}{KEYS} \PYG{p}{(}\PYG{l+m+mi}{9} \PYG{l+m+mi}{0}\PYG{p}{)} \PYG{o}{\PYGZhy{}}
    \PYG{n}{TO} \PYG{p}{(}\PYG{l+m+mi}{99366}\PYG{p}{)}\PYG{p}{)}\PYG{o}{\PYGZhy{}}
  \PYG{n}{DATA} \PYG{p}{(}\PYG{n}{NAME} \PYG{p}{(}\PYG{n}{VIRTEL}\PYG{o}{.}\PYG{n}{ARBO}\PYG{o}{.}\PYG{n}{DATA} \PYG{p}{)}\PYG{p}{)} \PYG{o}{\PYGZhy{}}
  \PYG{n}{INDEX} \PYG{p}{(}\PYG{n}{NAME} \PYG{p}{(}\PYG{n}{VIRTEL}\PYG{o}{.}\PYG{n}{ARBO}\PYG{o}{.}\PYG{n}{INDEX} \PYG{p}{)}\PYG{p}{)} \PYG{o}{\PYGZhy{}}
    \PYG{n}{CATALOG} \PYG{p}{(}\PYG{n}{VSESP}\PYG{o}{.}\PYG{n}{USER}\PYG{o}{.}\PYG{n}{CATALOG} \PYG{p}{)}
  \PYG{n}{IF} \PYG{n}{LASTCC} \PYG{n}{NE} \PYG{l+m+mi}{0} \PYG{n}{THEN} \PYG{n}{CANCEL} \PYG{n}{JOB}
  \PYG{n}{DELETE} \PYG{p}{(}\PYG{n}{VIRTEL}\PYG{o}{.}\PYG{n}{SWAP} \PYG{p}{)} \PYG{o}{\PYGZhy{}}
    \PYG{n}{CLUSTER} \PYG{o}{\PYGZhy{}}
    \PYG{n}{PURGE} \PYG{o}{\PYGZhy{}}
    \PYG{n}{CATALOG} \PYG{p}{(}\PYG{n}{VSESP}\PYG{o}{.}\PYG{n}{USER}\PYG{o}{.}\PYG{n}{CATALOG} \PYG{p}{)}
  \PYG{n}{SET} \PYG{n}{MAXCC}\PYG{o}{=}\PYG{l+m+mi}{0}
  \PYG{n}{DEFINE} \PYG{n}{CLUSTER} \PYG{p}{(} \PYG{o}{\PYGZhy{}}
    \PYG{n}{NAME} \PYG{p}{(}\PYG{n}{VIRTEL}\PYG{o}{.}\PYG{n}{SWAP} \PYG{p}{)} \PYG{o}{\PYGZhy{}}
    \PYG{n}{RECORDS}\PYG{p}{(}\PYG{l+m+mi}{200} \PYG{l+m+mi}{50}\PYG{p}{)} \PYG{o}{\PYGZhy{}}
    \PYG{n}{SHAREOPTIONS} \PYG{p}{(}\PYG{l+m+mi}{2} \PYG{l+m+mi}{3}\PYG{p}{)} \PYG{o}{\PYGZhy{}}
    \PYG{n}{RECSZ} \PYG{p}{(}\PYG{l+m+mi}{600} \PYG{l+m+mi}{4089}\PYG{p}{)} \PYG{o}{\PYGZhy{}}
    \PYG{n}{VOLUMES} \PYG{p}{(}\PYG{n}{SYSWK1}\PYG{p}{)} \PYG{o}{\PYGZhy{}}
    \PYG{n}{REUSE} \PYG{o}{\PYGZhy{}}
    \PYG{n}{KEYS} \PYG{p}{(}\PYG{l+m+mi}{16} \PYG{l+m+mi}{0}\PYG{p}{)} \PYG{o}{\PYGZhy{}}
    \PYG{n}{TO} \PYG{p}{(}\PYG{l+m+mi}{99366}\PYG{p}{)}\PYG{p}{)}\PYG{o}{\PYGZhy{}}
  \PYG{n}{DATA} \PYG{p}{(}\PYG{n}{NAME} \PYG{p}{(}\PYG{n}{VIRTEL}\PYG{o}{.}\PYG{n}{SWAP}\PYG{o}{.}\PYG{n}{DATA} \PYG{p}{)}\PYG{p}{)} \PYG{o}{\PYGZhy{}}
  \PYG{n}{INDEX} \PYG{p}{(}\PYG{n}{NAME} \PYG{p}{(}\PYG{n}{VIRTEL}\PYG{o}{.}\PYG{n}{SWAP}\PYG{o}{.}\PYG{n}{INDEX} \PYG{p}{)}\PYG{p}{)} \PYG{o}{\PYGZhy{}}
    \PYG{n}{CATALOG} \PYG{p}{(}\PYG{n}{VSESP}\PYG{o}{.}\PYG{n}{USER}\PYG{o}{.}\PYG{n}{CATALOG} \PYG{p}{)}
  \PYG{n}{IF} \PYG{n}{LASTCC} \PYG{n}{NE} \PYG{l+m+mi}{0} \PYG{n}{THEN} \PYG{n}{CANCEL} \PYG{n}{JOB}
\PYG{o}{/}\PYG{o}{*}
\end{sphinxVerbatim}

\sphinxAtStartPar
\sphinxstyleemphasis{VIRTVS1 : JCL to define the VIRARBO and VIRSWAP files (z/VSE)}

\sphinxAtStartPar
Step VIRTVS1 of job VIRTVS contains an example of defining the VIRARBO and VIRSWAP files. This job is provided as an example, and may need to be modified prior to execution. The parameters SETPARM TAPE=590 and VOLUMES(SYSWK1), and possible the catalog name, may need to be modified.


\subsection{Initialisation of the VIRARBO file}
\label{\detokenize{Installation_Guide:initialisation-of-the-virarbo-file}}
\begin{sphinxVerbatim}[commandchars=\\\{\}]
\PYG{o}{*} \PYG{o}{*}\PYG{o}{*}\PYG{o}{*}\PYG{o}{*}\PYG{o}{*}\PYG{o}{*}\PYG{o}{*}\PYG{o}{*}\PYG{o}{*}\PYG{o}{*}\PYG{o}{*}\PYG{o}{*}\PYG{o}{*}\PYG{o}{*}\PYG{o}{*}\PYG{o}{*}\PYG{o}{*}\PYG{o}{*}\PYG{o}{*}\PYG{o}{*}\PYG{o}{*}\PYG{o}{*}\PYG{o}{*}\PYG{o}{*}\PYG{o}{*}\PYG{o}{*}\PYG{o}{*}\PYG{o}{*}\PYG{o}{*}\PYG{o}{*}\PYG{o}{*}\PYG{o}{*}\PYG{o}{*}\PYG{o}{*}\PYG{o}{*}\PYG{o}{*}\PYG{o}{*}\PYG{o}{*}\PYG{o}{*}\PYG{o}{*}\PYG{o}{*}\PYG{o}{*}\PYG{o}{*}\PYG{o}{*}\PYG{o}{*}\PYG{o}{*}\PYG{o}{*}\PYG{o}{*}\PYG{o}{*}\PYG{o}{*}\PYG{o}{*}\PYG{o}{*}\PYG{o}{*}\PYG{o}{*}\PYG{o}{*}\PYG{o}{*}\PYG{o}{*}\PYG{o}{*}\PYG{o}{*}\PYG{o}{*}\PYG{o}{*}\PYG{o}{*}\PYG{o}{*}\PYG{o}{*}\PYG{o}{*}
\PYG{o}{*} \PYG{o}{*} \PYG{n}{VIRTVS2} \PYG{o}{*} \PYG{n}{INITIALISATION} \PYG{n}{OF} \PYG{n}{VIRARBO} \PYG{n}{FILE} \PYG{o}{*}
\PYG{o}{*} \PYG{o}{*}\PYG{o}{*}\PYG{o}{*}\PYG{o}{*}\PYG{o}{*}\PYG{o}{*}\PYG{o}{*}\PYG{o}{*}\PYG{o}{*}\PYG{o}{*}\PYG{o}{*}\PYG{o}{*}\PYG{o}{*}\PYG{o}{*}\PYG{o}{*}\PYG{o}{*}\PYG{o}{*}\PYG{o}{*}\PYG{o}{*}\PYG{o}{*}\PYG{o}{*}\PYG{o}{*}\PYG{o}{*}\PYG{o}{*}\PYG{o}{*}\PYG{o}{*}\PYG{o}{*}\PYG{o}{*}\PYG{o}{*}\PYG{o}{*}\PYG{o}{*}\PYG{o}{*}\PYG{o}{*}\PYG{o}{*}\PYG{o}{*}\PYG{o}{*}\PYG{o}{*}\PYG{o}{*}\PYG{o}{*}\PYG{o}{*}\PYG{o}{*}\PYG{o}{*}\PYG{o}{*}\PYG{o}{*}\PYG{o}{*}\PYG{o}{*}\PYG{o}{*}\PYG{o}{*}\PYG{o}{*}\PYG{o}{*}\PYG{o}{*}\PYG{o}{*}\PYG{o}{*}\PYG{o}{*}\PYG{o}{*}\PYG{o}{*}\PYG{o}{*}\PYG{o}{*}\PYG{o}{*}\PYG{o}{*}\PYG{o}{*}\PYG{o}{*}\PYG{o}{*}\PYG{o}{*}\PYG{o}{*}
\PYG{o}{/}\PYG{o}{/} \PYG{n}{DLBL} \PYG{n}{VIRARBO}\PYG{p}{,}\PYG{l+s+s1}{\PYGZsq{}}\PYG{l+s+s1}{VIRTEL.ARBO}\PYG{l+s+s1}{\PYGZsq{}}\PYG{p}{,}\PYG{p}{,}\PYG{n}{VSAM}\PYG{p}{,}\PYG{n}{CAT}\PYG{o}{=}\PYG{n}{VSESPUC}
\PYG{o}{/}\PYG{o}{/} \PYG{n}{PAUSE} \PYG{o}{*}\PYG{o}{*}\PYG{o}{*}\PYG{o}{*} \PYG{n}{VIRTEL} \PYG{o}{*}\PYG{o}{*}\PYG{o}{*}\PYG{o}{*} \PYG{n}{MOUNT} \PYG{n}{INSTALLATION} \PYG{n}{TAPE}
\PYG{o}{/}\PYG{o}{/} \PYG{n}{ASSGN} \PYG{n}{SYS004}\PYG{p}{,}\PYG{o}{\PYGZam{}}\PYG{n}{TAPE}
\PYG{o}{/}\PYG{o}{/} \PYG{n}{MTC} \PYG{n}{REW}\PYG{p}{,}\PYG{n}{SYS004}
\PYG{o}{/}\PYG{o}{/} \PYG{n}{MTC} \PYG{n}{FSF}\PYG{p}{,}\PYG{n}{SYS004}\PYG{p}{,}\PYG{l+m+mi}{2} \PYG{l+m+mi}{1}\PYG{o}{=}\PYG{n}{FRANCAIS}\PYG{p}{,}\PYG{l+m+mi}{2}\PYG{o}{=}\PYG{n}{ANGLAIS}
\PYG{o}{/}\PYG{o}{/} \PYG{n}{EXEC} \PYG{n}{IDCAMS}\PYG{p}{,}\PYG{n}{SIZE}\PYG{o}{=}\PYG{n}{AUTO}
  \PYG{n}{REPRO} \PYG{n}{IFILE}\PYG{p}{(}\PYG{n}{BANDE} \PYG{n}{ENV}\PYG{p}{(}\PYG{n}{PDEV}\PYG{p}{(}\PYG{l+m+mi}{2400}\PYG{p}{)} \PYG{n}{NOLABEL} \PYG{n}{RECFM}\PYG{p}{(}\PYG{n}{VB}\PYG{p}{)} \PYG{n}{BLKSZ}\PYG{p}{(}\PYG{l+m+mi}{32758}\PYG{p}{)}\PYG{p}{)}\PYG{p}{)} \PYG{o}{\PYGZhy{}}
  \PYG{n}{OFILE}\PYG{p}{(}\PYG{n}{VIRARBO}\PYG{p}{)}
\PYG{o}{/}\PYG{o}{*}
\end{sphinxVerbatim}

\sphinxAtStartPar
\sphinxstyleemphasis{VIRTVS2 : JCL to initialise the VIRARBO file (z/VSE)}

\sphinxAtStartPar
Step VIRTVS2 of job VIRTVS loads the base configuration definitions into the VIRARBO file. The default language is English. To load the French language version of the base configuration, change the

\begin{sphinxVerbatim}[commandchars=\\\{\}]
\PYG{o}{/}\PYG{o}{/} \PYG{n}{MTC} \PYG{n}{FSF}\PYG{p}{,}\PYG{n}{SYS004}\PYG{p}{,}\PYG{l+m+mi}{2}
\end{sphinxVerbatim}

\sphinxAtStartPar
card to

\begin{sphinxVerbatim}[commandchars=\\\{\}]
\PYG{o}{/}\PYG{o}{/} \PYG{n}{MTC} \PYG{n}{FSF}\PYG{p}{,}\PYG{n}{SYS004}\PYG{p}{,}\PYG{l+m+mi}{1}
\end{sphinxVerbatim}

\sphinxAtStartPar
before submitting this job.


\subsection{Defining the VIRSTAT file}
\label{\detokenize{Installation_Guide:defining-the-virstat-file}}
\begin{sphinxVerbatim}[commandchars=\\\{\}]
\PYG{o}{*} \PYG{o}{*}\PYG{o}{*}\PYG{o}{*}\PYG{o}{*}\PYG{o}{*}\PYG{o}{*}\PYG{o}{*}\PYG{o}{*}\PYG{o}{*}\PYG{o}{*}\PYG{o}{*}\PYG{o}{*}\PYG{o}{*}\PYG{o}{*}\PYG{o}{*}\PYG{o}{*}\PYG{o}{*}\PYG{o}{*}\PYG{o}{*}\PYG{o}{*}\PYG{o}{*}\PYG{o}{*}\PYG{o}{*}\PYG{o}{*}\PYG{o}{*}\PYG{o}{*}\PYG{o}{*}\PYG{o}{*}\PYG{o}{*}\PYG{o}{*}\PYG{o}{*}\PYG{o}{*}\PYG{o}{*}\PYG{o}{*}\PYG{o}{*}\PYG{o}{*}\PYG{o}{*}\PYG{o}{*}\PYG{o}{*}\PYG{o}{*}\PYG{o}{*}\PYG{o}{*}\PYG{o}{*}\PYG{o}{*}\PYG{o}{*}\PYG{o}{*}\PYG{o}{*}\PYG{o}{*}\PYG{o}{*}\PYG{o}{*}\PYG{o}{*}\PYG{o}{*}\PYG{o}{*}\PYG{o}{*}\PYG{o}{*}\PYG{o}{*}\PYG{o}{*}\PYG{o}{*}\PYG{o}{*}\PYG{o}{*}\PYG{o}{*}\PYG{o}{*}\PYG{o}{*}\PYG{o}{*}\PYG{o}{*}
\PYG{o}{*} \PYG{o}{*} \PYG{n}{VIRTVS3} \PYG{o}{*} \PYG{n}{DEFINITION} \PYG{n}{OF} \PYG{n}{VIRSTAT} \PYG{n}{FILE} \PYG{o}{*}
\PYG{o}{*} \PYG{o}{*}\PYG{o}{*}\PYG{o}{*}\PYG{o}{*}\PYG{o}{*}\PYG{o}{*}\PYG{o}{*}\PYG{o}{*}\PYG{o}{*}\PYG{o}{*}\PYG{o}{*}\PYG{o}{*}\PYG{o}{*}\PYG{o}{*}\PYG{o}{*}\PYG{o}{*}\PYG{o}{*}\PYG{o}{*}\PYG{o}{*}\PYG{o}{*}\PYG{o}{*}\PYG{o}{*}\PYG{o}{*}\PYG{o}{*}\PYG{o}{*}\PYG{o}{*}\PYG{o}{*}\PYG{o}{*}\PYG{o}{*}\PYG{o}{*}\PYG{o}{*}\PYG{o}{*}\PYG{o}{*}\PYG{o}{*}\PYG{o}{*}\PYG{o}{*}\PYG{o}{*}\PYG{o}{*}\PYG{o}{*}\PYG{o}{*}\PYG{o}{*}\PYG{o}{*}\PYG{o}{*}\PYG{o}{*}\PYG{o}{*}\PYG{o}{*}\PYG{o}{*}\PYG{o}{*}\PYG{o}{*}\PYG{o}{*}\PYG{o}{*}\PYG{o}{*}\PYG{o}{*}\PYG{o}{*}\PYG{o}{*}\PYG{o}{*}\PYG{o}{*}\PYG{o}{*}\PYG{o}{*}\PYG{o}{*}\PYG{o}{*}\PYG{o}{*}\PYG{o}{*}\PYG{o}{*}\PYG{o}{*}
\PYG{o}{/}\PYG{o}{/} \PYG{n}{DLBL} \PYG{n}{IJSYSUC}\PYG{p}{,}\PYG{l+s+s1}{\PYGZsq{}}\PYG{l+s+s1}{VSESP.USER.CATALOG}\PYG{l+s+s1}{\PYGZsq{}}\PYG{p}{,}\PYG{p}{,}\PYG{n}{VSAM}
\PYG{o}{/}\PYG{o}{/} \PYG{n}{EXEC} \PYG{n}{IDCAMS}\PYG{p}{,}\PYG{n}{SIZE}\PYG{o}{=}\PYG{n}{AUTO}
  \PYG{n}{DELETE} \PYG{p}{(}\PYG{n}{VIRTEL}\PYG{o}{.}\PYG{n}{STAT} \PYG{p}{)} \PYG{o}{\PYGZhy{}}
    \PYG{n}{CLUSTER} \PYG{o}{\PYGZhy{}}
    \PYG{n}{PURGE} \PYG{o}{\PYGZhy{}}
    \PYG{n}{CATALOG} \PYG{p}{(}\PYG{n}{VSESP}\PYG{o}{.}\PYG{n}{USER}\PYG{o}{.}\PYG{n}{CATALOG} \PYG{p}{)}
  \PYG{n}{SET} \PYG{n}{MAXCC}\PYG{o}{=}\PYG{l+m+mi}{0}
  \PYG{n}{DEFINE} \PYG{n}{CLUSTER} \PYG{p}{(} \PYG{o}{\PYGZhy{}}
    \PYG{n}{NAME} \PYG{p}{(}\PYG{n}{VIRTEL}\PYG{o}{.}\PYG{n}{STAT} \PYG{p}{)} \PYG{o}{\PYGZhy{}}
    \PYG{n}{RECORDS} \PYG{p}{(}\PYG{l+m+mi}{500} \PYG{l+m+mi}{100}\PYG{p}{)}\PYG{o}{\PYGZhy{}}
    \PYG{n}{SHAREOPTIONS} \PYG{p}{(}\PYG{l+m+mi}{2}\PYG{p}{)} \PYG{o}{\PYGZhy{}}
    \PYG{n}{RECSZ} \PYG{p}{(}\PYG{l+m+mi}{124} \PYG{l+m+mi}{620}\PYG{p}{)} \PYG{o}{\PYGZhy{}}
    \PYG{n}{RECORDFORMAT} \PYG{p}{(}\PYG{n}{FIXBLK} \PYG{p}{(}\PYG{l+m+mi}{124} \PYG{p}{)}\PYG{p}{)}\PYG{o}{\PYGZhy{}}
    \PYG{n}{VOLUMES} \PYG{p}{(}\PYG{n}{SYSWK1}\PYG{p}{)} \PYG{o}{\PYGZhy{}}
    \PYG{n}{NOREUSE} \PYG{o}{\PYGZhy{}}
    \PYG{n}{NONINDEXED} \PYG{o}{\PYGZhy{}}
    \PYG{n}{FREESPACE} \PYG{p}{(}\PYG{l+m+mi}{15} \PYG{l+m+mi}{7}\PYG{p}{)} \PYG{o}{\PYGZhy{}}
    \PYG{n}{TO} \PYG{p}{(}\PYG{l+m+mi}{99366}\PYG{p}{)}\PYG{p}{)}\PYG{o}{\PYGZhy{}}
  \PYG{n}{DATA} \PYG{p}{(}\PYG{n}{NAME} \PYG{p}{(}\PYG{n}{VIRTEL}\PYG{o}{.}\PYG{n}{STAT}\PYG{o}{.}\PYG{n}{DATA} \PYG{p}{)}\PYG{p}{)} \PYG{o}{\PYGZhy{}}
    \PYG{n}{CATALOG} \PYG{p}{(}\PYG{n}{VSESP}\PYG{o}{.}\PYG{n}{USER}\PYG{o}{.}\PYG{n}{CATALOG} \PYG{p}{)}
  \PYG{n}{IF} \PYG{n}{LASTCC} \PYG{n}{NE} \PYG{l+m+mi}{0} \PYG{n}{THEN} \PYG{n}{CANCEL} \PYG{n}{JOB}
 \PYG{o}{/}\PYG{o}{*}
\end{sphinxVerbatim}

\sphinxAtStartPar
\sphinxstyleemphasis{VIRTVS3 : JCL to define the VIRSTAT file (z/VSE)}

\sphinxAtStartPar
Step VIRTVS3 of job VIRTVS contains an example of defining the VIRSTAT file. This job is provided as an example, and may need to be modified prior to execution. The VIRSTAT file is required unless the STATS parameter of the VIRTCT is set to NO.


\subsection{Defining the VIRCMP3 file}
\label{\detokenize{Installation_Guide:defining-the-vircmp3-file}}
\begin{sphinxVerbatim}[commandchars=\\\{\}]
\PYG{o}{*} \PYG{o}{*}\PYG{o}{*}\PYG{o}{*}\PYG{o}{*}\PYG{o}{*}\PYG{o}{*}\PYG{o}{*}\PYG{o}{*}\PYG{o}{*}\PYG{o}{*}\PYG{o}{*}\PYG{o}{*}\PYG{o}{*}\PYG{o}{*}\PYG{o}{*}\PYG{o}{*}\PYG{o}{*}\PYG{o}{*}\PYG{o}{*}\PYG{o}{*}\PYG{o}{*}\PYG{o}{*}\PYG{o}{*}\PYG{o}{*}\PYG{o}{*}\PYG{o}{*}\PYG{o}{*}\PYG{o}{*}\PYG{o}{*}\PYG{o}{*}\PYG{o}{*}\PYG{o}{*}\PYG{o}{*}\PYG{o}{*}\PYG{o}{*}\PYG{o}{*}\PYG{o}{*}\PYG{o}{*}\PYG{o}{*}\PYG{o}{*}\PYG{o}{*}\PYG{o}{*}\PYG{o}{*}\PYG{o}{*}\PYG{o}{*}\PYG{o}{*}\PYG{o}{*}\PYG{o}{*}\PYG{o}{*}\PYG{o}{*}\PYG{o}{*}\PYG{o}{*}\PYG{o}{*}\PYG{o}{*}\PYG{o}{*}\PYG{o}{*}\PYG{o}{*}\PYG{o}{*}\PYG{o}{*}\PYG{o}{*}\PYG{o}{*}\PYG{o}{*}\PYG{o}{*}\PYG{o}{*}\PYG{o}{*}
\PYG{o}{*} \PYG{o}{*} \PYG{n}{VIRTVS4} \PYG{o}{*} \PYG{n}{DEFINITION} \PYG{n}{AND} \PYG{n}{INITIALIZATION} \PYG{n}{OF} \PYG{n}{VIRCMP3} \PYG{n}{FILE} \PYG{o}{*}
\PYG{o}{*} \PYG{o}{*}\PYG{o}{*}\PYG{o}{*}\PYG{o}{*}\PYG{o}{*}\PYG{o}{*}\PYG{o}{*}\PYG{o}{*}\PYG{o}{*}\PYG{o}{*}\PYG{o}{*}\PYG{o}{*}\PYG{o}{*}\PYG{o}{*}\PYG{o}{*}\PYG{o}{*}\PYG{o}{*}\PYG{o}{*}\PYG{o}{*}\PYG{o}{*}\PYG{o}{*}\PYG{o}{*}\PYG{o}{*}\PYG{o}{*}\PYG{o}{*}\PYG{o}{*}\PYG{o}{*}\PYG{o}{*}\PYG{o}{*}\PYG{o}{*}\PYG{o}{*}\PYG{o}{*}\PYG{o}{*}\PYG{o}{*}\PYG{o}{*}\PYG{o}{*}\PYG{o}{*}\PYG{o}{*}\PYG{o}{*}\PYG{o}{*}\PYG{o}{*}\PYG{o}{*}\PYG{o}{*}\PYG{o}{*}\PYG{o}{*}\PYG{o}{*}\PYG{o}{*}\PYG{o}{*}\PYG{o}{*}\PYG{o}{*}\PYG{o}{*}\PYG{o}{*}\PYG{o}{*}\PYG{o}{*}\PYG{o}{*}\PYG{o}{*}\PYG{o}{*}\PYG{o}{*}\PYG{o}{*}\PYG{o}{*}\PYG{o}{*}\PYG{o}{*}\PYG{o}{*}\PYG{o}{*}\PYG{o}{*}
\PYG{o}{/}\PYG{o}{/} \PYG{n}{DLBL} \PYG{n}{IJSYSUC}\PYG{p}{,}\PYG{l+s+s1}{\PYGZsq{}}\PYG{l+s+s1}{VSESP.USER.CATALOG}\PYG{l+s+s1}{\PYGZsq{}}\PYG{p}{,}\PYG{p}{,}\PYG{n}{VSAM}
\PYG{o}{/}\PYG{o}{/} \PYG{n}{EXEC} \PYG{n}{IDCAMS}\PYG{p}{,}\PYG{n}{SIZE}\PYG{o}{=}\PYG{n}{AUTO}
  \PYG{n}{DELETE} \PYG{p}{(}\PYG{n}{VIRTEL}\PYG{o}{.}\PYG{n}{CMP3} \PYG{p}{)} \PYG{o}{\PYGZhy{}}
    \PYG{n}{CLUSTER} \PYG{o}{\PYGZhy{}}
    \PYG{n}{PURGE} \PYG{o}{\PYGZhy{}}
    \PYG{n}{CATALOG} \PYG{p}{(}\PYG{n}{VSESP}\PYG{o}{.}\PYG{n}{USER}\PYG{o}{.}\PYG{n}{CATALOG} \PYG{p}{)}
  \PYG{n}{SET} \PYG{n}{MAXCC}\PYG{o}{=}\PYG{l+m+mi}{0}
  \PYG{n}{DEFINE} \PYG{n}{CLUSTER} \PYG{p}{(} \PYG{o}{\PYGZhy{}}
    \PYG{n}{NAME} \PYG{p}{(}\PYG{n}{VIRTEL}\PYG{o}{.}\PYG{n}{CMP3} \PYG{p}{)} \PYG{o}{\PYGZhy{}}
    \PYG{n}{RECORDS}\PYG{p}{(}\PYG{l+m+mi}{200} \PYG{l+m+mi}{50}\PYG{p}{)}\PYG{o}{\PYGZhy{}}
    \PYG{n}{SHAREOPTIONS} \PYG{p}{(}\PYG{l+m+mi}{2} \PYG{l+m+mi}{3}\PYG{p}{)} \PYG{o}{\PYGZhy{}}
    \PYG{n}{RECSZ} \PYG{p}{(}\PYG{l+m+mi}{600} \PYG{l+m+mi}{8185}\PYG{p}{)} \PYG{o}{\PYGZhy{}}
    \PYG{n}{VOLUMES} \PYG{p}{(}\PYG{n}{SYSWK1}\PYG{p}{)} \PYG{o}{\PYGZhy{}}
    \PYG{n}{KEYS} \PYG{p}{(}\PYG{l+m+mi}{9} \PYG{l+m+mi}{0}\PYG{p}{)} \PYG{o}{\PYGZhy{}}
    \PYG{n}{TO} \PYG{p}{(}\PYG{l+m+mi}{99366}\PYG{p}{)}\PYG{p}{)}\PYG{o}{\PYGZhy{}}
  \PYG{n}{DATA} \PYG{p}{(}\PYG{n}{NAME} \PYG{p}{(}\PYG{n}{VIRTEL}\PYG{o}{.}\PYG{n}{CMP3}\PYG{o}{.}\PYG{n}{DATA} \PYG{p}{)}\PYG{p}{)} \PYG{o}{\PYGZhy{}}
  \PYG{n}{INDEX} \PYG{p}{(}\PYG{n}{NAME} \PYG{p}{(}\PYG{n}{VIRTEL}\PYG{o}{.}\PYG{n}{CMP3}\PYG{o}{.}\PYG{n}{INDEX} \PYG{p}{)}\PYG{p}{)} \PYG{o}{\PYGZhy{}}
    \PYG{n}{CATALOG} \PYG{p}{(}\PYG{n}{VSESP}\PYG{o}{.}\PYG{n}{USER}\PYG{o}{.}\PYG{n}{CATALOG} \PYG{p}{)}
  \PYG{n}{IF} \PYG{n}{LASTCC} \PYG{n}{NE} \PYG{l+m+mi}{0} \PYG{n}{THEN} \PYG{n}{CANCEL} \PYG{n}{JOB}
\PYG{o}{/}\PYG{o}{*}
\PYG{o}{/}\PYG{o}{/} \PYG{n}{DLBL} \PYG{n}{VIRCMP3}\PYG{p}{,}\PYG{l+s+s1}{\PYGZsq{}}\PYG{l+s+s1}{VIRTEL.CMP3}\PYG{l+s+s1}{\PYGZsq{}}\PYG{p}{,}\PYG{l+m+mi}{2099}\PYG{o}{/}\PYG{l+m+mi}{365}\PYG{p}{,}\PYG{n}{VSAM}\PYG{p}{,}\PYG{n}{CAT}\PYG{o}{=}\PYG{n}{VSESPUC}
\PYG{o}{/}\PYG{o}{/} \PYG{n}{EXEC} \PYG{n}{IESVSMLD}\PYG{p}{,}\PYG{n}{SIZE}\PYG{o}{=}\PYG{n}{AUTO} \PYG{n}{LOAD} \PYG{n}{DUMMY} \PYG{n}{RECORD} \PYG{n}{INTO} \PYG{n}{VIRCMP3}
\PYG{l+m+mi}{80}\PYG{p}{,}\PYG{n}{K}\PYG{p}{,}\PYG{n}{VIRCMP3}
\PYG{n}{ZZZ}
\PYG{o}{/}\PYG{o}{*}
\end{sphinxVerbatim}

\sphinxAtStartPar
\sphinxstyleemphasis{VIRTVS4 : JCL to define the VIRCMP3 file (z/VSE)}

\sphinxAtStartPar
Step VIRTVS4 of job VIRTVS contains an example of defining the VIRCMP3 file. This job is provided as an example, and may need to be modified prior to execution. The VIRCMP3 file is used by the level 3 compression feature of VIRTEL/PC, and is required unless the COMPR3 parameter of the VIRTCT is set to NO.


\subsection{Defining the VIRCAPT file}
\label{\detokenize{Installation_Guide:defining-the-vircapt-file}}
\begin{sphinxVerbatim}[commandchars=\\\{\}]
\PYG{o}{*} \PYG{o}{*}\PYG{o}{*}\PYG{o}{*}\PYG{o}{*}\PYG{o}{*}\PYG{o}{*}\PYG{o}{*}\PYG{o}{*}\PYG{o}{*}\PYG{o}{*}\PYG{o}{*}\PYG{o}{*}\PYG{o}{*}\PYG{o}{*}\PYG{o}{*}\PYG{o}{*}\PYG{o}{*}\PYG{o}{*}\PYG{o}{*}\PYG{o}{*}\PYG{o}{*}\PYG{o}{*}\PYG{o}{*}\PYG{o}{*}\PYG{o}{*}\PYG{o}{*}\PYG{o}{*}\PYG{o}{*}\PYG{o}{*}\PYG{o}{*}\PYG{o}{*}\PYG{o}{*}\PYG{o}{*}\PYG{o}{*}\PYG{o}{*}\PYG{o}{*}\PYG{o}{*}\PYG{o}{*}\PYG{o}{*}\PYG{o}{*}\PYG{o}{*}\PYG{o}{*}\PYG{o}{*}\PYG{o}{*}\PYG{o}{*}\PYG{o}{*}\PYG{o}{*}\PYG{o}{*}\PYG{o}{*}\PYG{o}{*}\PYG{o}{*}\PYG{o}{*}\PYG{o}{*}\PYG{o}{*}\PYG{o}{*}\PYG{o}{*}\PYG{o}{*}\PYG{o}{*}\PYG{o}{*}\PYG{o}{*}\PYG{o}{*}\PYG{o}{*}\PYG{o}{*}\PYG{o}{*}\PYG{o}{*}
\PYG{o}{*} \PYG{o}{*} \PYG{n}{VIRTVS5} \PYG{o}{*} \PYG{n}{DEFINITION} \PYG{n}{AND} \PYG{n}{INITIALIZATION} \PYG{n}{OF} \PYG{n}{VIRCAPT} \PYG{n}{FILE} \PYG{o}{*}
\PYG{o}{*} \PYG{o}{*}\PYG{o}{*}\PYG{o}{*}\PYG{o}{*}\PYG{o}{*}\PYG{o}{*}\PYG{o}{*}\PYG{o}{*}\PYG{o}{*}\PYG{o}{*}\PYG{o}{*}\PYG{o}{*}\PYG{o}{*}\PYG{o}{*}\PYG{o}{*}\PYG{o}{*}\PYG{o}{*}\PYG{o}{*}\PYG{o}{*}\PYG{o}{*}\PYG{o}{*}\PYG{o}{*}\PYG{o}{*}\PYG{o}{*}\PYG{o}{*}\PYG{o}{*}\PYG{o}{*}\PYG{o}{*}\PYG{o}{*}\PYG{o}{*}\PYG{o}{*}\PYG{o}{*}\PYG{o}{*}\PYG{o}{*}\PYG{o}{*}\PYG{o}{*}\PYG{o}{*}\PYG{o}{*}\PYG{o}{*}\PYG{o}{*}\PYG{o}{*}\PYG{o}{*}\PYG{o}{*}\PYG{o}{*}\PYG{o}{*}\PYG{o}{*}\PYG{o}{*}\PYG{o}{*}\PYG{o}{*}\PYG{o}{*}\PYG{o}{*}\PYG{o}{*}\PYG{o}{*}\PYG{o}{*}\PYG{o}{*}\PYG{o}{*}\PYG{o}{*}\PYG{o}{*}\PYG{o}{*}\PYG{o}{*}\PYG{o}{*}\PYG{o}{*}\PYG{o}{*}\PYG{o}{*}\PYG{o}{*}
\PYG{o}{/}\PYG{o}{/} \PYG{n}{DLBL} \PYG{n}{IJSYSUC}\PYG{p}{,}\PYG{l+s+s1}{\PYGZsq{}}\PYG{l+s+s1}{VSESP.USER.CATALOG}\PYG{l+s+s1}{\PYGZsq{}}\PYG{p}{,}\PYG{p}{,}\PYG{n}{VSAM}
\PYG{o}{/}\PYG{o}{/} \PYG{n}{EXEC} \PYG{n}{IDCAMS}\PYG{p}{,}\PYG{n}{SIZE}\PYG{o}{=}\PYG{n}{AUTO}
  \PYG{n}{DELETE} \PYG{p}{(}\PYG{n}{VIRTEL}\PYG{o}{.}\PYG{n}{CAPT} \PYG{p}{)} \PYG{o}{\PYGZhy{}}
    \PYG{n}{CLUSTER} \PYG{o}{\PYGZhy{}}
    \PYG{n}{PURGE} \PYG{o}{\PYGZhy{}}
  \PYG{n}{CATALOG} \PYG{p}{(}\PYG{n}{VSESP}\PYG{o}{.}\PYG{n}{USER}\PYG{o}{.}\PYG{n}{CATALOG} \PYG{p}{)}
  \PYG{n}{SET} \PYG{n}{MAXCC}\PYG{o}{=}\PYG{l+m+mi}{0}
  \PYG{n}{DEFINE} \PYG{n}{CLUSTER} \PYG{p}{(} \PYG{o}{\PYGZhy{}}
    \PYG{n}{NAME} \PYG{p}{(}\PYG{n}{VIRTEL}\PYG{o}{.}\PYG{n}{CAPT} \PYG{p}{)} \PYG{o}{\PYGZhy{}}
    \PYG{n}{RECORDS}\PYG{p}{(}\PYG{l+m+mi}{200} \PYG{l+m+mi}{50}\PYG{p}{)}\PYG{o}{\PYGZhy{}}
    \PYG{n}{SHAREOPTIONS} \PYG{p}{(}\PYG{l+m+mi}{2} \PYG{l+m+mi}{3}\PYG{p}{)} \PYG{o}{\PYGZhy{}}
    \PYG{n}{RECSZ} \PYG{p}{(}\PYG{l+m+mi}{600} \PYG{l+m+mi}{8185}\PYG{p}{)} \PYG{o}{\PYGZhy{}}
    \PYG{n}{VOLUMES} \PYG{p}{(}\PYG{n}{SYSWK1}\PYG{p}{)} \PYG{o}{\PYGZhy{}}
    \PYG{n}{KEYS} \PYG{p}{(}\PYG{l+m+mi}{16} \PYG{l+m+mi}{0}\PYG{p}{)} \PYG{o}{\PYGZhy{}}
    \PYG{n}{TO} \PYG{p}{(}\PYG{l+m+mi}{99366}\PYG{p}{)}\PYG{p}{)}\PYG{o}{\PYGZhy{}}
  \PYG{n}{DATA} \PYG{p}{(}\PYG{n}{NAME} \PYG{p}{(}\PYG{n}{VIRTEL}\PYG{o}{.}\PYG{n}{CAPT}\PYG{o}{.}\PYG{n}{DATA} \PYG{p}{)}\PYG{p}{)} \PYG{o}{\PYGZhy{}}
  \PYG{n}{INDEX} \PYG{p}{(}\PYG{n}{NAME} \PYG{p}{(}\PYG{n}{VIRTEL}\PYG{o}{.}\PYG{n}{CAPT}\PYG{o}{.}\PYG{n}{INDEX} \PYG{p}{)}\PYG{p}{)} \PYG{o}{\PYGZhy{}}
    \PYG{n}{CATALOG} \PYG{p}{(}\PYG{n}{VSESP}\PYG{o}{.}\PYG{n}{USER}\PYG{o}{.}\PYG{n}{CATALOG} \PYG{p}{)}
  \PYG{n}{IF} \PYG{n}{LASTCC} \PYG{n}{NE} \PYG{l+m+mi}{0} \PYG{n}{THEN} \PYG{n}{CANCEL} \PYG{n}{JOB}
\PYG{o}{/}\PYG{o}{*}
\PYG{o}{/}\PYG{o}{/} \PYG{n}{DLBL} \PYG{n}{VIRCAPT}\PYG{p}{,}\PYG{l+s+s1}{\PYGZsq{}}\PYG{l+s+s1}{VIRTEL.CAPT}\PYG{l+s+s1}{\PYGZsq{}}\PYG{p}{,}\PYG{l+m+mi}{2099}\PYG{o}{/}\PYG{l+m+mi}{365}\PYG{p}{,}\PYG{n}{VSAM}\PYG{p}{,}\PYG{n}{CAT}\PYG{o}{=}\PYG{n}{VSESPUC}
\PYG{o}{/}\PYG{o}{/} \PYG{n}{EXEC} \PYG{n}{IESVSMLD}\PYG{p}{,}\PYG{n}{SIZE}\PYG{o}{=}\PYG{n}{AUTO} \PYG{n}{LOAD} \PYG{n}{DUMMY} \PYG{n}{RECORD} \PYG{n}{INTO} \PYG{n}{VIRCAPT}
\PYG{l+m+mi}{80}\PYG{p}{,}\PYG{n}{K}\PYG{p}{,}\PYG{n}{VIRCAPT}
\PYG{n}{ZZZ}
\PYG{o}{/}\PYG{o}{*}
\end{sphinxVerbatim}

\sphinxAtStartPar
\sphinxstyleemphasis{VIRTVS5 : JCL to define the VIRCAPT file (z/VSE)}

\sphinxAtStartPar
Step VIRTVS5 of job VIRTVS contains an example of defining the VIRCAPT file. This job is provided as an example, and may need to be modified prior to execution. The VIRCAPT file is used by the videotext page capture feature, and is referenced by the FCAPT parameter of the VIRTCT.


\subsection{Defining the SAMPTRF file}
\label{\detokenize{Installation_Guide:defining-the-samptrf-file}}
\begin{sphinxVerbatim}[commandchars=\\\{\}]
* *****************************************************************
* * VIRTVS6 * DEFINITION AND INITIALIZATION OF SAMPTRF FILE *
* *****************************************************************
// DLBL IJSYSUC,\PYGZsq{}VSESP.USER.CATALOG\PYGZsq{},,VSAM
// EXEC IDCAMS,SIZE=AUTO
  DELETE (VIRTEL.SAMP.TRSF ) \PYGZhy{}
    CLUSTER \PYGZhy{}
    PURGE \PYGZhy{}
    CATALOG (VSESP.USER.CATALOG )
  SET MAXCC=0
  DEFINE CLUSTER ( \PYGZhy{}
    NAME(VIRTEL.SAMP.TRSF ) \PYGZhy{}
    TO (99365) \PYGZhy{}
    FREESPACE (0 50) \PYGZhy{}
    SHAREOPTIONS (2) \PYGZhy{}
    INDEXED \PYGZhy{}
    KEYS (16 0) \PYGZhy{}
    RECORDSIZE (100 32758) \PYGZhy{}
    USECLASS (0) \PYGZhy{}
    VOLUMES (SYSWK1)) \PYGZhy{}
  DATA (NAME(VIRTEL.SAMP.TRSF.DATA ) \PYGZhy{}
    SPANNED \PYGZhy{}
    TRACKS(75 15) \textendash{}
    CISZ (4096)) \PYGZhy{}
  INDEX (NAME(VIRTEL.SAMP.TRSF.INDEX ) \PYGZhy{}
    TRACKS(5 1) \textendash{}
    CISZ (512)) \PYGZhy{}
  CATALOG (VSESP.USER.CATALOG )
/*
// DLBL INWFILE,\PYGZsq{}VIRTEL.SAMP.TRSF\PYGZsq{},2099/365,VSAM,CAT=VSESPUC
// EXEC IESVSMLD,SIZE=AUTO LOAD DUMMY RECORD INTO INWFILE
80,K,INWFILE
\PYGZdl{}\PYGZdl{}\PYGZdl{}\PYGZdl{}IWS.WORKREC.INW\PYGZdl{}TEMP
/*
\end{sphinxVerbatim}

\sphinxAtStartPar
\sphinxstyleemphasis{VIRTVS6 : JCL to define the SAMPTRF file (z/VSE)}

\sphinxAtStartPar
Step VIRTVS6 of job VIRTVS contains an example of defining the SAMPTRF file. This job is provided as an example, and may need to be modified prior to execution. The SAMPTRF file contains sample HTML page templates and other elements for the VIRTEL Web Access feature, and is referenced by the UFILEx parameter of the VIRTCT.


\subsection{Defining the HTMLTRF file}
\label{\detokenize{Installation_Guide:defining-the-htmltrf-file}}
\begin{sphinxVerbatim}[commandchars=\\\{\}]
* *****************************************************************
* * VIRTVS7 * DEFINITION AND INITIALIZATION OF HTMLTRF FILE *
* *****************************************************************
// DLBL IJSYSUC,\PYGZsq{}VSESP.USER.CATALOG\PYGZsq{},,VSAM
// EXEC IDCAMS,SIZE=AUTO
  DELETE (VIRTEL.HTML.TRSF ) \PYGZhy{}
    CLUSTER \PYGZhy{}
    PURGE \PYGZhy{}
    CATALOG (VSESP.USER.CATALOG )
  SET MAXCC=0
  DEFINE CLUSTER ( \PYGZhy{}
    NAME(VIRTEL.HTML.TRSF ) \PYGZhy{}
    RECORDS (2500 1000) \PYGZhy{}
    TO (99365) \PYGZhy{}
    FREESPACE (0 50) \PYGZhy{}
    SHAREOPTIONS (2) \PYGZhy{}
    INDEXED \PYGZhy{}
    KEYS (16 0) \PYGZhy{}
    RECORDSIZE (100 32758) \PYGZhy{}
    USECLASS (0) \PYGZhy{}
    VOLUMES (SYSWK1)) \PYGZhy{}
  DATA (NAME(VIRTEL.HTML.TRSF.DATA ) \PYGZhy{}
    SPANNED \PYGZhy{}
    TRACKS(75 15) \textendash{}
    CISZ (4096)) \PYGZhy{}
  INDEX (NAME(VIRTEL.HTML.TRSF.INDEX ) \PYGZhy{}
    TRACKS(5 1) \textendash{}
    CISZ (512)) \PYGZhy{}
    CATALOG (VSESP.USER.CATALOG )
/*
// DLBL HTMLTRF,\PYGZsq{}VIRTEL.HTML.TRSF\PYGZsq{},2099/365,VSAM,CAT=VSESPUC
// EXEC IESVSMLD,SIZE=AUTO LOAD DUMMY RECORD INTO HTMLTRF
80,K,HTMLTRF
\PYGZdl{}\PYGZdl{}\PYGZdl{}\PYGZdl{}IWS.WORKREC.INW\PYGZdl{}TEMP
/*
\end{sphinxVerbatim}

\sphinxAtStartPar
\sphinxstyleemphasis{VIRTVS7 : JCL to define the HTMLTRF file (z/VSE)}

\sphinxAtStartPar
Step VIRTVS7 of job VIRTVS contains an example of defining the HTMLTRF file. This job is provided as an example, and may need to be modified prior to execution. The HTMLTRF file is used by the VIRTEL Web Access feature to store HTML pages, and is referenced by the UFILEx parameter of the VIRTCT.


\subsection{Loading the SAMPTRF file}
\label{\detokenize{Installation_Guide:loading-the-samptrf-file}}
\begin{sphinxVerbatim}[commandchars=\\\{\}]
\PYG{o}{*} \PYG{o}{*}\PYG{o}{*}\PYG{o}{*}\PYG{o}{*}\PYG{o}{*}\PYG{o}{*}\PYG{o}{*}\PYG{o}{*}\PYG{o}{*}\PYG{o}{*}\PYG{o}{*}\PYG{o}{*}\PYG{o}{*}\PYG{o}{*}\PYG{o}{*}\PYG{o}{*}\PYG{o}{*}\PYG{o}{*}\PYG{o}{*}\PYG{o}{*}\PYG{o}{*}\PYG{o}{*}\PYG{o}{*}\PYG{o}{*}\PYG{o}{*}\PYG{o}{*}\PYG{o}{*}\PYG{o}{*}\PYG{o}{*}\PYG{o}{*}\PYG{o}{*}\PYG{o}{*}\PYG{o}{*}\PYG{o}{*}\PYG{o}{*}\PYG{o}{*}\PYG{o}{*}\PYG{o}{*}\PYG{o}{*}\PYG{o}{*}\PYG{o}{*}\PYG{o}{*}\PYG{o}{*}\PYG{o}{*}\PYG{o}{*}\PYG{o}{*}\PYG{o}{*}\PYG{o}{*}\PYG{o}{*}\PYG{o}{*}\PYG{o}{*}\PYG{o}{*}\PYG{o}{*}\PYG{o}{*}\PYG{o}{*}\PYG{o}{*}\PYG{o}{*}\PYG{o}{*}\PYG{o}{*}\PYG{o}{*}\PYG{o}{*}\PYG{o}{*}\PYG{o}{*}\PYG{o}{*}\PYG{o}{*}
\PYG{o}{*} \PYG{o}{*} \PYG{n}{VIRTVS8} \PYG{o}{*} \PYG{n}{LOAD} \PYG{n}{DATA} \PYG{n}{INTO} \PYG{n}{SAMPTRF} \PYG{n}{FILE} \PYG{o}{*}
\PYG{o}{*} \PYG{o}{*}\PYG{o}{*}\PYG{o}{*}\PYG{o}{*}\PYG{o}{*}\PYG{o}{*}\PYG{o}{*}\PYG{o}{*}\PYG{o}{*}\PYG{o}{*}\PYG{o}{*}\PYG{o}{*}\PYG{o}{*}\PYG{o}{*}\PYG{o}{*}\PYG{o}{*}\PYG{o}{*}\PYG{o}{*}\PYG{o}{*}\PYG{o}{*}\PYG{o}{*}\PYG{o}{*}\PYG{o}{*}\PYG{o}{*}\PYG{o}{*}\PYG{o}{*}\PYG{o}{*}\PYG{o}{*}\PYG{o}{*}\PYG{o}{*}\PYG{o}{*}\PYG{o}{*}\PYG{o}{*}\PYG{o}{*}\PYG{o}{*}\PYG{o}{*}\PYG{o}{*}\PYG{o}{*}\PYG{o}{*}\PYG{o}{*}\PYG{o}{*}\PYG{o}{*}\PYG{o}{*}\PYG{o}{*}\PYG{o}{*}\PYG{o}{*}\PYG{o}{*}\PYG{o}{*}\PYG{o}{*}\PYG{o}{*}\PYG{o}{*}\PYG{o}{*}\PYG{o}{*}\PYG{o}{*}\PYG{o}{*}\PYG{o}{*}\PYG{o}{*}\PYG{o}{*}\PYG{o}{*}\PYG{o}{*}\PYG{o}{*}\PYG{o}{*}\PYG{o}{*}\PYG{o}{*}\PYG{o}{*}
\PYG{o}{/}\PYG{o}{/} \PYG{n}{DLBL} \PYG{n}{SAMPTRF}\PYG{p}{,}\PYG{l+s+s1}{\PYGZsq{}}\PYG{l+s+s1}{VIRTEL.SAMP.TRSF}\PYG{l+s+s1}{\PYGZsq{}}\PYG{p}{,}\PYG{p}{,}\PYG{n}{VSAM}\PYG{p}{,}\PYG{n}{CAT}\PYG{o}{=}\PYG{n}{VSESPUC}
\PYG{o}{/}\PYG{o}{/} \PYG{n}{PAUSE} \PYG{o}{*}\PYG{o}{*}\PYG{o}{*}\PYG{o}{*} \PYG{n}{VIRTEL} \PYG{o}{*}\PYG{o}{*}\PYG{o}{*}\PYG{o}{*} \PYG{n}{MONTEZ} \PYG{n}{LA} \PYG{n}{BANDE} \PYG{n}{D}\PYG{l+s+s1}{\PYGZsq{}}\PYG{l+s+s1}{INSTALLATION}
\PYG{o}{/}\PYG{o}{/} \PYG{n}{ASSGN} \PYG{n}{SYS004}\PYG{p}{,}\PYG{o}{\PYGZam{}}\PYG{n}{TAPE}
\PYG{o}{/}\PYG{o}{/} \PYG{n}{MTC} \PYG{n}{REW}\PYG{p}{,}\PYG{n}{SYS004}
\PYG{o}{/}\PYG{o}{/} \PYG{n}{MTC} \PYG{n}{FSF}\PYG{p}{,}\PYG{n}{SYS004}\PYG{p}{,}\PYG{l+m+mi}{3}
\PYG{o}{/}\PYG{o}{/} \PYG{n}{EXEC} \PYG{n}{IDCAMS}\PYG{p}{,}\PYG{n}{SIZE}\PYG{o}{=}\PYG{n}{AUTO}
  \PYG{n}{REPRO} \PYG{n}{IFILE}\PYG{p}{(}\PYG{n}{BANDE} \PYG{n}{ENV}\PYG{p}{(}\PYG{n}{PDEV}\PYG{p}{(}\PYG{l+m+mi}{2400}\PYG{p}{)} \PYG{n}{NOLABEL} \PYG{n}{RECFM}\PYG{p}{(}\PYG{n}{VB}\PYG{p}{)} \PYG{n}{BLKSZ}\PYG{p}{(}\PYG{l+m+mi}{32758}\PYG{p}{)}\PYG{p}{)}\PYG{p}{)} \PYG{o}{\PYGZhy{}}
  \PYG{n}{OFILE}\PYG{p}{(}\PYG{n}{SAMPTRF}\PYG{p}{)} \PYG{n}{REPLACE}
\PYG{o}{/}\PYG{o}{*}
\end{sphinxVerbatim}

\sphinxAtStartPar
\sphinxstyleemphasis{VIRTVS8 : JCL to load the SAMPTRF file (z/VSE)}

\sphinxAtStartPar
Step VIRTVS8 of job VIRTVS contains and example of the JCL required to load the sample HTML pages into the SAMPTRF file. This job is required for sites using VIRTEL Web Access.


\subsection{Defining the VIRHTML file}
\label{\detokenize{Installation_Guide:defining-the-virhtml-file}}
\begin{sphinxVerbatim}[commandchars=\\\{\}]
\PYG{o}{*} \PYG{o}{*}\PYG{o}{*}\PYG{o}{*}\PYG{o}{*}\PYG{o}{*}\PYG{o}{*}\PYG{o}{*}\PYG{o}{*}\PYG{o}{*}\PYG{o}{*}\PYG{o}{*}\PYG{o}{*}\PYG{o}{*}\PYG{o}{*}\PYG{o}{*}\PYG{o}{*}\PYG{o}{*}\PYG{o}{*}\PYG{o}{*}\PYG{o}{*}\PYG{o}{*}\PYG{o}{*}\PYG{o}{*}\PYG{o}{*}\PYG{o}{*}\PYG{o}{*}\PYG{o}{*}\PYG{o}{*}\PYG{o}{*}\PYG{o}{*}\PYG{o}{*}\PYG{o}{*}\PYG{o}{*}\PYG{o}{*}\PYG{o}{*}\PYG{o}{*}\PYG{o}{*}\PYG{o}{*}\PYG{o}{*}\PYG{o}{*}\PYG{o}{*}\PYG{o}{*}\PYG{o}{*}\PYG{o}{*}\PYG{o}{*}\PYG{o}{*}\PYG{o}{*}\PYG{o}{*}\PYG{o}{*}\PYG{o}{*}\PYG{o}{*}\PYG{o}{*}\PYG{o}{*}\PYG{o}{*}\PYG{o}{*}\PYG{o}{*}\PYG{o}{*}\PYG{o}{*}\PYG{o}{*}\PYG{o}{*}\PYG{o}{*}\PYG{o}{*}\PYG{o}{*}\PYG{o}{*}\PYG{o}{*}
\PYG{o}{*} \PYG{o}{*} \PYG{n}{VIRTVS9} \PYG{o}{*} \PYG{n}{DEFINITION} \PYG{n}{AND} \PYG{n}{INITIALIZATION} \PYG{n}{OF} \PYG{n}{VIRHTML} \PYG{n}{FILE} \PYG{o}{*}
\PYG{o}{*} \PYG{o}{*}\PYG{o}{*}\PYG{o}{*}\PYG{o}{*}\PYG{o}{*}\PYG{o}{*}\PYG{o}{*}\PYG{o}{*}\PYG{o}{*}\PYG{o}{*}\PYG{o}{*}\PYG{o}{*}\PYG{o}{*}\PYG{o}{*}\PYG{o}{*}\PYG{o}{*}\PYG{o}{*}\PYG{o}{*}\PYG{o}{*}\PYG{o}{*}\PYG{o}{*}\PYG{o}{*}\PYG{o}{*}\PYG{o}{*}\PYG{o}{*}\PYG{o}{*}\PYG{o}{*}\PYG{o}{*}\PYG{o}{*}\PYG{o}{*}\PYG{o}{*}\PYG{o}{*}\PYG{o}{*}\PYG{o}{*}\PYG{o}{*}\PYG{o}{*}\PYG{o}{*}\PYG{o}{*}\PYG{o}{*}\PYG{o}{*}\PYG{o}{*}\PYG{o}{*}\PYG{o}{*}\PYG{o}{*}\PYG{o}{*}\PYG{o}{*}\PYG{o}{*}\PYG{o}{*}\PYG{o}{*}\PYG{o}{*}\PYG{o}{*}\PYG{o}{*}\PYG{o}{*}\PYG{o}{*}\PYG{o}{*}\PYG{o}{*}\PYG{o}{*}\PYG{o}{*}\PYG{o}{*}\PYG{o}{*}\PYG{o}{*}\PYG{o}{*}\PYG{o}{*}\PYG{o}{*}\PYG{o}{*}
\PYG{o}{/}\PYG{o}{/} \PYG{n}{DLBL} \PYG{n}{IJSYSUC}\PYG{p}{,}\PYG{l+s+s1}{\PYGZsq{}}\PYG{l+s+s1}{VSESP.USER.CATALOG}\PYG{l+s+s1}{\PYGZsq{}}\PYG{p}{,}\PYG{p}{,}\PYG{n}{VSAM}
\PYG{o}{/}\PYG{o}{/} \PYG{n}{EXEC} \PYG{n}{IDCAMS}\PYG{p}{,}\PYG{n}{SIZE}\PYG{o}{=}\PYG{n}{AUTO}
  \PYG{n}{DELETE} \PYG{p}{(}\PYG{n}{VIRTEL}\PYG{o}{.}\PYG{n}{HTML} \PYG{p}{)} \PYG{o}{\PYGZhy{}}
    \PYG{n}{CLUSTER} \PYG{o}{\PYGZhy{}}
    \PYG{n}{PURGE} \PYG{o}{\PYGZhy{}}
    \PYG{n}{CATALOG} \PYG{p}{(}\PYG{n}{VSESP}\PYG{o}{.}\PYG{n}{USER}\PYG{o}{.}\PYG{n}{CATALOG} \PYG{p}{)}
  \PYG{n}{SET} \PYG{n}{MAXCC}\PYG{o}{=}\PYG{l+m+mi}{0}
  \PYG{n}{DEFINE} \PYG{n}{CLUSTER} \PYG{p}{(} \PYG{o}{\PYGZhy{}}
    \PYG{n}{NAME}\PYG{p}{(}\PYG{n}{VIRTEL}\PYG{o}{.}\PYG{n}{HTML} \PYG{p}{)} \PYG{o}{\PYGZhy{}}
    \PYG{n}{RECORDS} \PYG{p}{(}\PYG{l+m+mi}{50} \PYG{l+m+mi}{100}\PYG{p}{)} \PYG{o}{\PYGZhy{}}
    \PYG{n}{TO} \PYG{p}{(}\PYG{l+m+mi}{99365}\PYG{p}{)} \PYG{o}{\PYGZhy{}}
    \PYG{n}{FREESPACE} \PYG{p}{(}\PYG{l+m+mi}{0} \PYG{l+m+mi}{50}\PYG{p}{)} \PYG{o}{\PYGZhy{}}
    \PYG{n}{SHAREOPTIONS} \PYG{p}{(}\PYG{l+m+mi}{2}\PYG{p}{)} \PYG{o}{\PYGZhy{}}
    \PYG{n}{INDEXED} \PYG{o}{\PYGZhy{}}
    \PYG{n}{KEYS} \PYG{p}{(}\PYG{l+m+mi}{64} \PYG{l+m+mi}{0}\PYG{p}{)} \PYG{o}{\PYGZhy{}}
    \PYG{n}{RECORDSIZE} \PYG{p}{(}\PYG{l+m+mi}{100} \PYG{l+m+mi}{32758}\PYG{p}{)} \PYG{o}{\PYGZhy{}}
    \PYG{n}{USECLASS} \PYG{p}{(}\PYG{l+m+mi}{0}\PYG{p}{)} \PYG{o}{\PYGZhy{}}
    \PYG{n}{VOLUMES} \PYG{p}{(}\PYG{n}{SYSWK1}\PYG{p}{)}\PYG{p}{)} \PYG{o}{\PYGZhy{}}
  \PYG{n}{DATA} \PYG{p}{(}\PYG{n}{NAME}\PYG{p}{(}\PYG{n}{VIRTEL}\PYG{o}{.}\PYG{n}{HTML}\PYG{o}{.}\PYG{n}{DATA} \PYG{p}{)} \PYG{o}{\PYGZhy{}}
    \PYG{n}{SPANNED} \PYG{o}{\PYGZhy{}}
    \PYG{n}{CISZ} \PYG{p}{(}\PYG{l+m+mi}{4096}\PYG{p}{)}\PYG{p}{)} \PYG{o}{\PYGZhy{}}
  \PYG{n}{INDEX} \PYG{p}{(}\PYG{n}{NAME}\PYG{p}{(}\PYG{n}{VIRTEL}\PYG{o}{.}\PYG{n}{HTML}\PYG{o}{.}\PYG{n}{INDEX} \PYG{p}{)} \PYG{o}{\PYGZhy{}}
    \PYG{n}{CISZ} \PYG{p}{(}\PYG{l+m+mi}{512}\PYG{p}{)}\PYG{p}{)} \PYG{o}{\PYGZhy{}}
  \PYG{n}{CATALOG} \PYG{p}{(}\PYG{n}{VSESP}\PYG{o}{.}\PYG{n}{USER}\PYG{o}{.}\PYG{n}{CATALOG} \PYG{p}{)}
\PYG{o}{/}\PYG{o}{*}
\PYG{o}{/}\PYG{o}{/} \PYG{n}{DLBL} \PYG{n}{VIRHTML}\PYG{p}{,}\PYG{l+s+s1}{\PYGZsq{}}\PYG{l+s+s1}{VIRTEL.HTML}\PYG{l+s+s1}{\PYGZsq{}}\PYG{p}{,}\PYG{l+m+mi}{2099}\PYG{o}{/}\PYG{l+m+mi}{365}\PYG{p}{,}\PYG{n}{VSAM}\PYG{p}{,}\PYG{n}{CAT}\PYG{o}{=}\PYG{n}{VSESPUC}
\PYG{o}{/}\PYG{o}{/} \PYG{n}{EXEC} \PYG{n}{IESVSMLD}\PYG{p}{,}\PYG{n}{SIZE}\PYG{o}{=}\PYG{n}{AUTO} \PYG{n}{LOAD} \PYG{n}{DUMMY} \PYG{n}{RECORD} \PYG{n}{INTO} \PYG{n}{VIRHTML}
\PYG{l+m+mi}{80}\PYG{p}{,}\PYG{n}{K}\PYG{p}{,}\PYG{n}{VIRHTML}
\PYG{n}{ZZZ}
\PYG{o}{/}\PYG{o}{*}
\end{sphinxVerbatim}

\sphinxAtStartPar
\sphinxstyleemphasis{VIRTVS9 : JCL to define the VIRHTML file (z/VSE)}

\sphinxAtStartPar
Step VIRTVS9 of job VIRTVS contains an example of defining the VIRHTML file. This job is provided as an example, and may need to be modified prior to execution. The VIRHTML file is used by the VIRTEL Web Access feature to store the names of E\sphinxhyphen{}mail correspondents or centralized parameter information \sphinxhyphen{} UPARM= specifed in the TCT. It is referenced by the HTVSAM parameter of the VIRTCT.


\subsection{Assembling the VIRTCT}
\label{\detokenize{Installation_Guide:assembling-the-virtct}}
\sphinxAtStartPar
Job VIRTCTUS contains an example of assembling the VIRTEL parameter table (the VIRTCT). Since the VIRTCT parameters are common across the z/VSE and z/OS environments, please refer to section {\hyperref[\detokenize{Installation_Guide:vvrrig-virtct}]{\sphinxcrossref{\DUrole{std,std-ref}{VIRTCT}}}}.

\begin{sphinxadmonition}{note}{Note:}
\sphinxAtStartPar
Users in France should use job VIRTCTFR instead of VIRTCTUS.
\end{sphinxadmonition}


\subsection{Assembling the MODVIRT mode table}
\label{\detokenize{Installation_Guide:assembling-the-modvirt-mode-table}}
\begin{sphinxVerbatim}[commandchars=\\\{\}]
* \PYGZdl{}\PYGZdl{} JOB JNM=VIRMOD,CLASS=0,DISP=L
* \PYGZdl{}\PYGZdl{} LST DA
// JOB VIRMOD
* *****************************************************************
* * VIRMOD * ASSEMBLY OF THE VTAM MODE TABLE *
* *****************************************************************
* * *
* * THIS JOB IS SUPPLIED AS AN EXAMPLE ONLY AND MUST BE MODIFIED *
* * BEFORE EXECUTION *
* * *
* *****************************************************************
// DLBL VIRTvrr,\PYGZsq{}VSE.VIRTvrr.LIBRARY\PYGZsq{},,VSAM,CAT=VSESPUC
// LIBDEF PHASE,CATALOG=PRD2.CONFIG
// LIBDEF SOURCE,SEARCH=(VIRTvrr.SUBLIB,PRD1.BASE)
// OPTION CATAL
  PHASE MODVIRT,*
// EXEC ASSEMBLY,SIZE=512K
  COPY MODVIRT
/*
// EXEC LNKEDT,SIZE=512K
/*
/\PYGZam{}
* \PYGZdl{}\PYGZdl{} EOJ
\end{sphinxVerbatim}

\sphinxAtStartPar
\sphinxstyleemphasis{VIRMOD : Assembling the MODVIRT mode table (z/VSE)}

\sphinxAtStartPar
Job VIRMOD contains an example of the JCL required to assemble the VTAM mode table (MODVIRT) supplied with VIRTEL.


\subsection{Updating the VIRARBO file (ARBOLOAD)}
\label{\detokenize{Installation_Guide:updating-the-virarbo-file-arboload}}
\begin{sphinxVerbatim}[commandchars=\\\{\}]
* \PYGZdl{}\PYGZdl{} JOB JNM=VIRCONF,CLASS=0,DISP=L
* \PYGZdl{}\PYGZdl{} LST DA
// JOB VIRCONF
* *****************************************************************
* * VIRCONF * LOAD CONFIGURATION DATA (ARBOLOAD) *
* *****************************************************************
* * *
* * THIS JOB IS SUPPLIED AS AN EXAMPLE ONLY AND MUST BE MODIFIED *
* * BEFORE EXECUTION *
* * *
* *****************************************************************
// LIBDEF *,SEARCH=(VIRTvrr.SUBLIB)
// DLBL VIRARBO,\PYGZsq{}VIRTEL.ARBO\PYGZsq{},,VSAM,CAT=VSESPUC
// SETPARM LANG=EN
// SETPARM WEB=YES
// SETPARM VMACROS=NO
// SETPARM SMTP=NO
// SETPARM IMSW=NO
// SETPARM VHOST=NO
// SETPARM PLUG=NO
// SETPARM VSR=NO
// SETPARM IPAD=NO
// SETPARM MINITEL=NO
// SETPARM PCMGMT=NO
// SETPARM NTTCP=NO
// SETPARM XOT=NO
// SETPARM NPSIFC=NO
// SETPARM NPSIGAT=NO
// SETPARM ANTIFC=NO
// SETPARM CFTGATE=NO
// SETPARM CFTPCNE=NO
// SETPARM MULTSES=NO
// SETPARM VIRSECU=NO
// IF WEB NE YES THEN
// GOTO WEB
// EXEC VIRCONF,PARM=\PYGZsq{}LOAD,LANG=\PYGZam{}LANG\PYGZsq{}
  (configuration statements for VIRTEL Web Access feature)
/*
/. WEB
// IF XOT NE YES THEN
// GOTO XOT
// EXEC VIRCONF,PARM=\PYGZsq{}LOAD,LANG=\PYGZam{}LANG\PYGZsq{}
  (configuration statements for XOT feature)
/*
/. XOT
  (etc)
/\PYGZam{}
* \PYGZdl{}\PYGZdl{} EOJ
\end{sphinxVerbatim}

\sphinxAtStartPar
\sphinxstyleemphasis{VIRCONF : ARBOLOAD job to update the VIRARBO file (z/VSE)}

\sphinxAtStartPar
Job VIRCONF contains an example of a job to load configuration elements into the VIRARBO file. This is the equivalent of the z/OS job known as ARBOLOAD. Before running this job, you will need to make the following modifications:
\begin{itemize}
\item {} 
\sphinxAtStartPar
Select the desired features (for example, WEB=YES, XOT=YES)

\item {} 
\sphinxAtStartPar
Change all ‘DBDCCICS’ to the APPLID of your CICS system.

\end{itemize}

\begin{sphinxadmonition}{note}{Note:}
\sphinxAtStartPar
Users in France may also change LANG=EN to LANG=FR to generate French language versions of the configuration elements
\end{sphinxadmonition}


\subsection{Cataloging the VTAM application book}
\label{\detokenize{Installation_Guide:cataloging-the-vtam-application-book}}
\begin{sphinxVerbatim}[commandchars=\\\{\}]
* \PYGZdl{}\PYGZdl{} JOB JNM=VIRTAPPL,CLASS=0,DISP=L
* \PYGZdl{}\PYGZdl{} LST DA
// JOB VIRTAPPL
* *****************************************************************
* * VIRTAPPL * EXAMPLE OF APPLICATION MAJOR NODE FOR VIRTEL *
* *****************************************************************
* * *
* * THIS JOB IS SUPPLIED AS AN EXAMPLE ONLY AND MUST BE MODIFIED *
* * BEFORE EXECUTION *
* * *
* *****************************************************************
// EXEC LIBR
ACCESS SUBLIB=PRD2.CONFIG
CATALOG VIRTAPPL.B REPLACE=YES
* \PYGZhy{}\PYGZhy{}\PYGZhy{}\PYGZhy{}\PYGZhy{}\PYGZhy{}\PYGZhy{}\PYGZhy{}\PYGZhy{}\PYGZhy{}\PYGZhy{}\PYGZhy{}\PYGZhy{}\PYGZhy{}\PYGZhy{}\PYGZhy{}\PYGZhy{}\PYGZhy{}\PYGZhy{}\PYGZhy{}\PYGZhy{}\PYGZhy{}\PYGZhy{}\PYGZhy{}\PYGZhy{}\PYGZhy{}\PYGZhy{}\PYGZhy{}\PYGZhy{}\PYGZhy{}\PYGZhy{}\PYGZhy{}\PYGZhy{}\PYGZhy{}\PYGZhy{}\PYGZhy{}\PYGZhy{}\PYGZhy{}\PYGZhy{}\PYGZhy{}\PYGZhy{}\PYGZhy{}\PYGZhy{}\PYGZhy{}\PYGZhy{}\PYGZhy{}\PYGZhy{}\PYGZhy{}\PYGZhy{}\PYGZhy{}\PYGZhy{}\PYGZhy{}\PYGZhy{}\PYGZhy{}\PYGZhy{}\PYGZhy{}\PYGZhy{}\PYGZhy{}\PYGZhy{}\PYGZhy{}\PYGZhy{}\PYGZhy{}\PYGZhy{}\PYGZhy{}\PYGZhy{}\PYGZhy{} *
* Product : Virtel *
* Description : Main ACB for VIRTEL application *
* \PYGZhy{}\PYGZhy{}\PYGZhy{}\PYGZhy{}\PYGZhy{}\PYGZhy{}\PYGZhy{}\PYGZhy{}\PYGZhy{}\PYGZhy{}\PYGZhy{}\PYGZhy{}\PYGZhy{}\PYGZhy{}\PYGZhy{}\PYGZhy{}\PYGZhy{}\PYGZhy{}\PYGZhy{}\PYGZhy{}\PYGZhy{}\PYGZhy{}\PYGZhy{}\PYGZhy{}\PYGZhy{}\PYGZhy{}\PYGZhy{}\PYGZhy{}\PYGZhy{}\PYGZhy{}\PYGZhy{}\PYGZhy{}\PYGZhy{}\PYGZhy{}\PYGZhy{}\PYGZhy{}\PYGZhy{}\PYGZhy{}\PYGZhy{}\PYGZhy{}\PYGZhy{}\PYGZhy{}\PYGZhy{}\PYGZhy{}\PYGZhy{}\PYGZhy{}\PYGZhy{}\PYGZhy{}\PYGZhy{}\PYGZhy{}\PYGZhy{}\PYGZhy{}\PYGZhy{}\PYGZhy{}\PYGZhy{}\PYGZhy{}\PYGZhy{}\PYGZhy{}\PYGZhy{}\PYGZhy{}\PYGZhy{}\PYGZhy{}\PYGZhy{}\PYGZhy{}\PYGZhy{}\PYGZhy{} *
VIRTEL APPL AUTH=(PASS,ACQ,SPO)
  (APPL statements for other VIRTEL relays)
/+
/*
/\PYGZam{}
* \PYGZdl{}\PYGZdl{} EOJ
\end{sphinxVerbatim}

\sphinxAtStartPar
\sphinxstyleemphasis{VIRTAPPL : Cataloging the application major node (z/VSE)}

\sphinxAtStartPar
Job VIRTAPPL contains an example of cataloging the VTAM application book. The VTAM application node VIRTAPPL must be activated before starting VIRTEL. This job is provided as an example, and may need to be modified prior to execution.


\subsection{Defining the CICS resources}
\label{\detokenize{Installation_Guide:defining-the-cics-resources}}
\begin{sphinxVerbatim}[commandchars=\\\{\}]
* \PYGZdl{}\PYGZdl{} JOB JNM=VIRGROUP,CLASS=A,DISP=D,NTFY=YES
* \PYGZdl{}\PYGZdl{} LST DA
// JOB VIRGROUP CREATION CICS CSD GROUP VIRTEL
* *****************************************************************
* * VIRGROUP * CICS RESOURCE DEFINITIONS FOR VIRTEL *
* *****************************************************************
* * *
* * THIS JOB IS SUPPLIED AS AN EXAMPLE ONLY AND MUST BE MODIFIED *
* * BEFORE EXECUTION *
* * *
* *****************************************************************
* * *
* * SEE IJSYSRS.SYSLIB/STDLABUP.PROC FOR DEFAULT DLBL DFHCSD *
* * // DLBL DFHCSD,\PYGZsq{}CICS.CSD\PYGZsq{},0,VSAM,CAT=VSESPUC *
* * *
* *****************************************************************
// EXEC DFHCSDUP,SIZE=AUTO
* VIRTEL 3270 TERMINALS FOR WEB2HOST
  DEFINE TE(T000) G(VIRTEL) TY(z/VSELU2Q) NE(RHTVT000) PRINTER(I000)
      DESC(VIRTEL WEB TO HOST TERMINAL)
  DEFINE TE(T001) G(VIRTEL) TY(z/VSELU2Q) NE(RHTVT001) PRINTER(I001)
      DESC(VIRTEL WEB TO HOST TERMINAL)
  DEFINE TE(T002) G(VIRTEL) TY(z/VSELU2Q) NE(RHTVT002) PRINTER(I002)
      DESC(VIRTEL WEB TO HOST TERMINAL)
      etc.
* VIRTEL 3284 PRINTERS FOR WEB2HOST
  DEFINE TE(I000) G(VIRTEL) TY(z/VSELU3Q) NE(RHTIM000)
      DESC(VIRTEL WEB TO HOST PRINTER)
  DEFINE TE(I001) G(VIRTEL) TY(z/VSELU3Q) NE(RHTIM001)
      DESC(VIRTEL WEB TO HOST PRINTER)
  DEFINE TE(I002) G(VIRTEL) TY(z/VSELU3Q) NE(RHTIM002)
      DESC(VIRTEL WEB TO HOST PRINTER)
    etc.
* ADD VIRTEL GROUP TO STARTUP LIST
  ADD GROUP(VIRTEL) LIST(z/VSELIST)
 /*
/\PYGZam{}
* \PYGZdl{}\PYGZdl{} EOJ
\end{sphinxVerbatim}

\sphinxAtStartPar
\sphinxstyleemphasis{VIRGROUP : Defining the CICS resources (z/VSE)}

\sphinxAtStartPar
Job VIRGROUP contains an example of defining the the CICS resources which are correspond to the relays and virtual printers used by VIRTEL Web Access. This job is provided as an example, and may need to be modified prior to execution.


\section{Executing VIRTEL In A z/VSE Environment}
\label{\detokenize{Installation_Guide:executing-virtel-in-a-z-vse-environment}}
\sphinxAtStartPar
Job VIRTEL contains an example of the z/VSE startup JCL for VIRTEL. Program VIR0000 reads a parameter card indicating the suffix of the VIRTCT to be used. This suffix must be two characters long and must start in column 1 of the parameter card. In the example supplied, the suffix is 01, indicating that parameter table VIRTCT01 is to be used. The TCT suffix may optionally be followed by a comma and the VTAM APPLID. If the APPLID is not specified then the value in the VIRTCT is used. The partition used must have a size of at least 1.5MB and must have 1MB of GETVIS. The priority of the VIRTEL partition must be immediately below that of VTAM.

\begin{sphinxVerbatim}[commandchars=\\\{\}]
* \PYGZdl{}\PYGZdl{} JOB JNM=VIRTEL,CLASS=4,DISP=L,PRI=9
* \PYGZdl{}\PYGZdl{} LST DA
// JOB VIRTEL
* *****************************************************************
* * VIRTEL * EXAMPLE JCL TO EXECUTE VIRTEL *
* *****************************************************************
* * *
* * THIS JOB IS SUPPLIED AS AN EXAMPLE ONLY AND MUST BE MODIFIED *
* * BEFORE EXECUTION *
* * *
* *****************************************************************
// OPTION SYSPARM=\PYGZsq{}00\PYGZsq{} MUST MATCH PARM ID=NN IN TCP/IP PARTITION
// LIBDEF *,SEARCH=(VIRTvrr.SUBLIB,PRD2.CONFIG,PRD1.BASE)
// DLBL VIRARBO,\PYGZsq{}VIRTEL.ARBO\PYGZsq{},,VSAM,CAT=VSESPUC
// DLBL VIRSWAP,\PYGZsq{}VIRTEL.SWAP\PYGZsq{},,VSAM,CAT=VSESPUC
// DLBL VIRCAPT,\PYGZsq{}VIRTEL.CAPT\PYGZsq{},,VSAM,CAT=VSESPUC
// DLBL VIRCMP3,\PYGZsq{}VIRTEL.CMP3\PYGZsq{},,VSAM,CAT=VSESPUC
// DLBL VIRHTML,\PYGZsq{}VIRTEL.HTML\PYGZsq{},,VSAM,CAT=VSESPUC
// DLBL SAMPTRF,\PYGZsq{}VIRTEL.SAMP.TRSF\PYGZsq{},,VSAM,CAT=VSESPUC
// DLBL HTMLTRF,\PYGZsq{}VIRTEL.HTML.TRSF\PYGZsq{},,VSAM,CAT=VSESPUC
// DLBL VIRSTAT,\PYGZsq{}VIRTEL.STAT\PYGZsq{},,VSAM,CAT=VSESPUC
* * OU BIEN // DLBL VIRSTAT,\PYGZsq{}VIRTEL.STAT\PYGZsq{},0,SD
* * // EXTENT SYS001,SYSWK1,1,0,855,15
* * // ASSGN SYS001,DISK,VOL=SYSWK1,SHR
// EXEC IESWAITT
// EXEC VIR0000,SIZE=40K,DSPACE=2M
01,VIRTEL
/*
// EXEC LISTLOG
/\PYGZam{}
* \PYGZdl{}\PYGZdl{} EOJ
\end{sphinxVerbatim}

\sphinxAtStartPar
\sphinxstyleemphasis{VIRTEL startup JCL (z/VSE)}


\subsection{Specifying the TCP/IP partition}
\label{\detokenize{Installation_Guide:specifying-the-tcp-ip-partition}}
\sphinxAtStartPar
If you have more than one TCP/IP stack, you can use the OPTION SYSPARM=’nn’ statement to specify the ID of the TCP/IP stack. VIRTEL will attempt to connect to the TCP/IP partition which has PARM=’ID=nn’ in its JCL. If OPTION is not specified, VIRTEL will attempt to connect to the default TCP/IP whose ID is 00.


\subsection{Stopping VIRTEL}
\label{\detokenize{Installation_Guide:id3}}
\sphinxAtStartPar
To stop VIRTEL, enter the command:

\begin{sphinxVerbatim}[commandchars=\\\{\}]
\PYG{n}{MSG} \PYG{n}{xx}\PYG{p}{,}\PYG{n}{DATA}\PYG{o}{=}\PYG{n}{STOP}
\end{sphinxVerbatim}

\sphinxAtStartPar
where xx is the identifier of the partition in which VIRTEL is running.

\newpage

\index{Installing under z/VSE@\spxentry{Installing under z/VSE}!Applying Maintenace@\spxentry{Applying Maintenace}}\index{Applying Maintenace@\spxentry{Applying Maintenace}!Installing under z/VSE@\spxentry{Installing under z/VSE}}\ignorespaces 

\section{Applying Maintenance}
\label{\detokenize{Installation_Guide:index-19}}\label{\detokenize{Installation_Guide:id4}}

\subsection{VWA Maintenance}
\label{\detokenize{Installation_Guide:vwa-maintenance}}
\sphinxAtStartPar
Under certain circumstances it may be necessary to apply maintenance in the form of User Interface Updates or Mainframe PTSs. These may be distributed either by e\sphinxhyphen{}mail, or available on Syspertec FTP Server. The location is the Public directory under VIRTEL V.RR/PTFs and Updates/.


\subsubsection{VWA Updates}
\label{\detokenize{Installation_Guide:vwa-updates}}
\sphinxAtStartPar
An update is available as a ZIP file containing the cumulative days update for a version. The file is represented in the form VirtelxxxUpdtnnnnn.zip where xxx is the version of Virtel to which it relates and nnnn the reference of the update itself. Once unzipped, the file content is in the form of a tree where each folder contains one or more files grouped by category, the root contains a file named updtnnnn.txt which summarized the history of changes and any special instructions to operate. Generally, the file still contains a sub directory named ” W2H ” whose content must be reloaded into the W2H\sphinxhyphen{}DIR using one of the methods described in section “Web Enitity Management” in the “Virtel Administration Guide”.


\subsubsection{Mainframe PTFs}
\label{\detokenize{Installation_Guide:mainframe-ptfs}}
\sphinxAtStartPar
Under certain circumstances it may be necessary to apply maintenance to the Virtel load library in the form of PTFs. These may be distributed through the Virtel FTP Server or by e\sphinxhyphen{}mail. The name of the PTF is in the format allptfs\sphinxhyphen{}mshpvrr.txt.

\sphinxAtStartPar
To apply the PTFs, use the following JCL:

\begin{sphinxVerbatim}[commandchars=\\\{\}]
* \PYGZdl{}\PYGZdl{} JOB JNM=PTFnnnn,CLASS=0,DISP=D,PRI=9
* \PYGZdl{}\PYGZdl{} LST DA
// JOB PTFnnnn
// EXEC MSHP
  PATCH SUBLIB=VIRTvrr.SUBLIB
  AFFECTS PHASE=modname
  ALTER xxxx vvvvvvvv:rrrrrrrr
/*
/\PYGZam{}
* \PYGZdl{}\PYGZdl{} EOJ
\end{sphinxVerbatim}

\sphinxAtStartPar
\sphinxstyleemphasis{JCL for applying PTFs (z/VSE)}

\index{VTAM Definitions@\spxentry{VTAM Definitions}}\ignorespaces 

\chapter{VTAM Definitions}
\label{\detokenize{Installation_Guide:vtam-definitions}}\label{\detokenize{Installation_Guide:index-20}}

\section{VTAM parameters}
\label{\detokenize{Installation_Guide:vtam-parameters}}
\sphinxAtStartPar
This section describes the VTAM definitions required for VIRTEL. The same definitions are used in both the z/OS and z/VSE environments.

\index{VTAM Definitions@\spxentry{VTAM Definitions}!VTAM APPL Statement@\spxentry{VTAM APPL Statement}}\index{VTAM APPL Statement@\spxentry{VTAM APPL Statement}!VTAM Definitions@\spxentry{VTAM Definitions}}\ignorespaces 

\subsection{VTAM APPL Definition \sphinxhyphen{} VIRTEL Primary ACB}
\label{\detokenize{Installation_Guide:vtam-appl-definition-virtel-primary-acb}}\label{\detokenize{Installation_Guide:index-21}}
\sphinxAtStartPar
The primary ACB is defined by means of a VTAM APPL statement:

\begin{sphinxVerbatim}[commandchars=\\\{\}]
\PYG{n}{applname} \PYG{n}{APPL} \PYG{n}{AUTH}\PYG{o}{=}\PYG{p}{(}\PYG{n}{PASS}\PYG{p}{,}\PYG{n}{ACQ}\PYG{p}{,}\PYG{n}{SPO}\PYG{p}{)}
\end{sphinxVerbatim}

\sphinxAtStartPar
\sphinxstylestrong{applname} \sphinxhyphen{} Presents the name of the ACB as it is defined in the APPLID statement of the VIRTCT.

\sphinxAtStartPar
An example of a VTAM application node is provided in member VIRTAPPL of the VIRTEL SAMPLIB dataset for z/OS, or in the VIRTAPPL installation job for VSE.

\index{VTAM Definitions@\spxentry{VTAM Definitions}!Virtel Application Relays@\spxentry{Virtel Application Relays}}\index{Virtel Application Relays@\spxentry{Virtel Application Relays}!VTAM Definitions@\spxentry{VTAM Definitions}}\ignorespaces 

\subsection{VTAM Application Relays}
\label{\detokenize{Installation_Guide:vtam-application-relays}}\label{\detokenize{Installation_Guide:index-22}}
\sphinxAtStartPar
Each terminal which logs on to a VTAM application via VIRTEL requires an application relay. An application relay is a VTAM LU, defined by means of a VTAM APPL card, which VIRTEL uses to represent the terminal when connecting to the VTAM application. These APPL cards are defined as follows:

\begin{sphinxVerbatim}[commandchars=\\\{\}]
\PYG{n}{relaynam} \PYG{n}{APPL} \PYG{n}{AUTH}\PYG{o}{=}\PYG{p}{(}\PYG{n}{PASS}\PYG{p}{,}\PYG{n}{ACQ}\PYG{p}{)}\PYG{p}{,}\PYG{n}{MODETAB}\PYG{o}{=}\PYG{n}{tablenam}\PYG{p}{,}\PYG{n}{DLOGMOD}\PYG{o}{=}\PYG{n}{modename}\PYG{p}{,}\PYG{n}{EAS}\PYG{o}{=}\PYG{l+m+mi}{1}
\end{sphinxVerbatim}

\sphinxAtStartPar
\sphinxstylestrong{relaynam} \sphinxhyphen{} Represents the name of the relay associated with the terminal. This name must match the name specified in the “Relay” field of the VIRTEL terminal definition.

\sphinxAtStartPar
\sphinxstylestrong{tablenam} \sphinxhyphen{} Is the name of the logon mode table. For VIRTEL Web Access, use the standard IBM\sphinxhyphen{}supplied table ISTINCLM. For other types of relay, use the MODVIRT table supplied by VIRTEL.

\sphinxAtStartPar
\sphinxstylestrong{modename} \sphinxhyphen{} Is the name of the LOGMODE to be used for communication with the host application. For VIRTEL Web Access, use a standard IBM\sphinxhyphen{}supplied logmode such as SNX32702.

\sphinxAtStartPar
\sphinxstylestrong{EAS=1} \sphinxhyphen{} Since each application relay only uses one session, specification of this parameter may reduce common area storage requirements.

\index{VTAM Definitions@\spxentry{VTAM Definitions}!Modetab for X25 and APPC@\spxentry{Modetab for X25 and APPC}}\index{Modetab for X25 and APPC@\spxentry{Modetab for X25 and APPC}!VTAM Definitions@\spxentry{VTAM Definitions}}\ignorespaces 

\subsection{MODETAB For X25 and APPC}
\label{\detokenize{Installation_Guide:modetab-for-x25-and-appc}}\label{\detokenize{Installation_Guide:index-23}}
\sphinxAtStartPar
If you intend to use X25, or APPC, then a mode table named MODVIRT must be assembled and link\sphinxhyphen{}edited into the library from which VTAM loads its mode tables. For z/OS, a sample job is provided in the ASMMOD member of the VIRTEL SAMPLIB. For z/VSE, sample JCL is provided in the VIRMOD installation job.

\sphinxAtStartPar
The source for the MODVIRT mode table is defined as follows:

\begin{sphinxVerbatim}[commandchars=\\\{\}]
\PYG{n}{MODVIRT} \PYG{n}{MODETAB}
\PYG{o}{*} \PYG{n}{LOGMODE} \PYG{k}{for} \PYG{n}{LUTYPE2} \PYG{n}{terminals}
\PYG{n}{DLOGREL} \PYG{n}{MODEENT} \PYG{n}{LOGMODE}\PYG{o}{=}\PYG{n}{DLOGREL}\PYG{p}{,}      \PYG{n}{X}
  \PYG{n}{FMPROF}\PYG{o}{=}\PYG{n}{X}\PYG{l+s+s1}{\PYGZsq{}}\PYG{l+s+s1}{03}\PYG{l+s+s1}{\PYGZsq{}}\PYG{p}{,}\PYG{n}{TSPROF}\PYG{o}{=}\PYG{n}{X}\PYG{l+s+s1}{\PYGZsq{}}\PYG{l+s+s1}{03}\PYG{l+s+s1}{\PYGZsq{}}\PYG{p}{,}\PYG{n}{PRIPROT}\PYG{o}{=}\PYG{n}{X}\PYG{l+s+s1}{\PYGZsq{}}\PYG{l+s+s1}{B1}\PYG{l+s+s1}{\PYGZsq{}}\PYG{p}{,} \PYG{n}{X}
  \PYG{n}{SECPROT}\PYG{o}{=}\PYG{n}{X}\PYG{l+s+s1}{\PYGZsq{}}\PYG{l+s+s1}{90}\PYG{l+s+s1}{\PYGZsq{}}\PYG{p}{,}\PYG{n}{COMPROT}\PYG{o}{=}\PYG{n}{X}\PYG{l+s+s1}{\PYGZsq{}}\PYG{l+s+s1}{3080}\PYG{l+s+s1}{\PYGZsq{}}\PYG{p}{,}\PYG{n}{RUSIZES}\PYG{o}{=}\PYG{n}{X}\PYG{l+s+s1}{\PYGZsq{}}\PYG{l+s+s1}{87F8}\PYG{l+s+s1}{\PYGZsq{}}\PYG{p}{,} \PYG{n}{X}
  \PYG{n}{PSERVIC}\PYG{o}{=}\PYG{n}{X}\PYG{l+s+s1}{\PYGZsq{}}\PYG{l+s+s1}{028000000000185000007E00}\PYG{l+s+s1}{\PYGZsq{}}
\PYG{o}{*} \PYG{n}{LOGMODE} \PYG{k}{for} \PYG{n}{LUTYPE1} \PYG{n}{terminals}
\PYG{n}{DLOGMINI} \PYG{n}{MODEENT} \PYG{n}{LOGMODE}\PYG{o}{=}\PYG{n}{DLOGMINI}\PYG{p}{,} \PYG{n}{X}
  \PYG{n}{FMPROF}\PYG{o}{=}\PYG{n}{X}\PYG{l+s+s1}{\PYGZsq{}}\PYG{l+s+s1}{03}\PYG{l+s+s1}{\PYGZsq{}}\PYG{p}{,} \PYG{n}{X}
  \PYG{n}{TSPROF}\PYG{o}{=}\PYG{n}{X}\PYG{l+s+s1}{\PYGZsq{}}\PYG{l+s+s1}{03}\PYG{l+s+s1}{\PYGZsq{}}\PYG{p}{,} \PYG{n}{X}
  \PYG{n}{PRIPROT}\PYG{o}{=}\PYG{n}{X}\PYG{l+s+s1}{\PYGZsq{}}\PYG{l+s+s1}{B1}\PYG{l+s+s1}{\PYGZsq{}}\PYG{p}{,} \PYG{n}{X}
  \PYG{n}{SECPROT}\PYG{o}{=}\PYG{n}{X}\PYG{l+s+s1}{\PYGZsq{}}\PYG{l+s+s1}{90}\PYG{l+s+s1}{\PYGZsq{}}\PYG{p}{,} \PYG{n}{X}
  \PYG{n}{COMPROT}\PYG{o}{=}\PYG{n}{X}\PYG{l+s+s1}{\PYGZsq{}}\PYG{l+s+s1}{3040}\PYG{l+s+s1}{\PYGZsq{}}\PYG{p}{,} \PYG{n}{CONTENTION} \PYG{n}{X}
  \PYG{n}{RUSIZES}\PYG{o}{=}\PYG{n}{X}\PYG{l+s+s1}{\PYGZsq{}}\PYG{l+s+s1}{8589}\PYG{l+s+s1}{\PYGZsq{}}\PYG{p}{,} \PYG{l+m+mi}{256}\PYG{o}{\PYGZhy{}}\PYG{l+m+mi}{4096} \PYG{n}{X}
  \PYG{n}{PSERVIC}\PYG{o}{=}\PYG{n}{X}\PYG{l+s+s1}{\PYGZsq{}}\PYG{l+s+s1}{010000000000000000000000}\PYG{l+s+s1}{\PYGZsq{}}
\PYG{o}{*} \PYG{n}{LOGMODE} \PYG{k}{for} \PYG{n}{inversed} \PYG{n}{GATE}
\PYG{n}{DLOGANTI} \PYG{n}{MODEENT} \PYG{n}{LOGMODE}\PYG{o}{=}\PYG{n}{DLOGANTI}\PYG{p}{,} \PYG{n}{X}
  \PYG{n}{FMPROF}\PYG{o}{=}\PYG{n}{X}\PYG{l+s+s1}{\PYGZsq{}}\PYG{l+s+s1}{03}\PYG{l+s+s1}{\PYGZsq{}}\PYG{p}{,}\PYG{n}{TSPROF}\PYG{o}{=}\PYG{n}{X}\PYG{l+s+s1}{\PYGZsq{}}\PYG{l+s+s1}{03}\PYG{l+s+s1}{\PYGZsq{}}\PYG{p}{,}\PYG{n}{PRIPROT}\PYG{o}{=}\PYG{n}{X}\PYG{l+s+s1}{\PYGZsq{}}\PYG{l+s+s1}{B1}\PYG{l+s+s1}{\PYGZsq{}}\PYG{p}{,}\PYG{n}{SECPROT}\PYG{o}{=}\PYG{n}{X}\PYG{l+s+s1}{\PYGZsq{}}\PYG{l+s+s1}{90}\PYG{l+s+s1}{\PYGZsq{}}\PYG{p}{,} \PYG{n}{X}
  \PYG{n}{COMPROT}\PYG{o}{=}\PYG{n}{X}\PYG{l+s+s1}{\PYGZsq{}}\PYG{l+s+s1}{3040}\PYG{l+s+s1}{\PYGZsq{}}\PYG{p}{,}\PYG{n}{RUSIZES}\PYG{o}{=}\PYG{n}{X}\PYG{l+s+s1}{\PYGZsq{}}\PYG{l+s+s1}{8989}\PYG{l+s+s1}{\PYGZsq{}}\PYG{p}{,} \PYG{n}{X}
  \PYG{n}{PSERVIC}\PYG{o}{=}\PYG{n}{X}\PYG{l+s+s1}{\PYGZsq{}}\PYG{l+s+s1}{010000000000000000000000}\PYG{l+s+s1}{\PYGZsq{}}
\PYG{o}{*} \PYG{n}{LOGMODE} \PYG{k}{for} \PYG{n}{inversed} \PYG{n}{PCNE} \PYG{o}{@}\PYG{l+m+mi}{416}
\PYG{n}{DLOGPCNE} \PYG{n}{MODEENT} \PYG{n}{LOGMODE}\PYG{o}{=}\PYG{n}{DLOGPCNE}\PYG{p}{,} \PYG{o}{@}\PYG{l+m+mi}{416}\PYG{n}{X}
  \PYG{n}{FMPROF}\PYG{o}{=}\PYG{n}{X}\PYG{l+s+s1}{\PYGZsq{}}\PYG{l+s+s1}{03}\PYG{l+s+s1}{\PYGZsq{}}\PYG{p}{,}\PYG{n}{TSPROF}\PYG{o}{=}\PYG{n}{X}\PYG{l+s+s1}{\PYGZsq{}}\PYG{l+s+s1}{03}\PYG{l+s+s1}{\PYGZsq{}}\PYG{p}{,} \PYG{o}{@}\PYG{l+m+mi}{416}\PYG{n}{X}
  \PYG{n}{PRIPROT}\PYG{o}{=}\PYG{n}{X}\PYG{l+s+s1}{\PYGZsq{}}\PYG{l+s+s1}{B0}\PYG{l+s+s1}{\PYGZsq{}}\PYG{p}{,}\PYG{n}{SECPROT}\PYG{o}{=}\PYG{n}{X}\PYG{l+s+s1}{\PYGZsq{}}\PYG{l+s+s1}{B0}\PYG{l+s+s1}{\PYGZsq{}}\PYG{p}{,} \PYG{o}{@}\PYG{l+m+mi}{416}\PYG{n}{X}
  \PYG{n}{COMPROT}\PYG{o}{=}\PYG{n}{X}\PYG{l+s+s1}{\PYGZsq{}}\PYG{l+s+s1}{0040}\PYG{l+s+s1}{\PYGZsq{}}\PYG{p}{,}\PYG{n}{RUSIZES}\PYG{o}{=}\PYG{n}{X}\PYG{l+s+s1}{\PYGZsq{}}\PYG{l+s+s1}{8989}\PYG{l+s+s1}{\PYGZsq{}}\PYG{p}{,} \PYG{o}{@}\PYG{l+m+mi}{416}\PYG{n}{X}
  \PYG{n}{PSERVIC}\PYG{o}{=}\PYG{n}{X}\PYG{l+s+s1}{\PYGZsq{}}\PYG{l+s+s1}{000000000000000000000000}\PYG{l+s+s1}{\PYGZsq{}} \PYG{o}{@}\PYG{l+m+mi}{416}
\PYG{o}{*} \PYG{n}{LOGMODE} \PYG{k}{for} \PYG{n}{APPC} \PYG{n}{lines} \PYG{p}{(}\PYG{n}{LU6}\PYG{l+m+mf}{.2}\PYG{p}{)}
\PYG{n}{LU62CONV} \PYG{n}{MODEENT} \PYG{n}{LOGMODE}\PYG{o}{=}\PYG{n}{LU62CONV}\PYG{p}{,}\PYG{n}{FMPROF}\PYG{o}{=}\PYG{n}{X}\PYG{l+s+s1}{\PYGZsq{}}\PYG{l+s+s1}{13}\PYG{l+s+s1}{\PYGZsq{}}\PYG{p}{,}\PYG{n}{TSPROF}\PYG{o}{=}\PYG{n}{X}\PYG{l+s+s1}{\PYGZsq{}}\PYG{l+s+s1}{07}\PYG{l+s+s1}{\PYGZsq{}}\PYG{p}{,} \PYG{n}{X}
  \PYG{n}{PRIPROT}\PYG{o}{=}\PYG{n}{X}\PYG{l+s+s1}{\PYGZsq{}}\PYG{l+s+s1}{B0}\PYG{l+s+s1}{\PYGZsq{}}\PYG{p}{,}\PYG{n}{SECPROT}\PYG{o}{=}\PYG{n}{X}\PYG{l+s+s1}{\PYGZsq{}}\PYG{l+s+s1}{B0}\PYG{l+s+s1}{\PYGZsq{}}\PYG{p}{,}\PYG{n}{COMPROT}\PYG{o}{=}\PYG{n}{X}\PYG{l+s+s1}{\PYGZsq{}}\PYG{l+s+s1}{D0B1}\PYG{l+s+s1}{\PYGZsq{}}\PYG{p}{,} \PYG{n}{X}
  \PYG{n}{RUSIZES}\PYG{o}{=}\PYG{n}{X}\PYG{l+s+s1}{\PYGZsq{}}\PYG{l+s+s1}{8686}\PYG{l+s+s1}{\PYGZsq{}}\PYG{p}{,}\PYG{n}{ENCR}\PYG{o}{=}\PYG{l+s+sa}{B}\PYG{l+s+s1}{\PYGZsq{}}\PYG{l+s+s1}{0000}\PYG{l+s+s1}{\PYGZsq{}}\PYG{p}{,}\PYG{n}{TYPE}\PYG{o}{=}\PYG{l+m+mi}{0}\PYG{p}{,} \PYG{n}{X}
  \PYG{n}{PSERVIC}\PYG{o}{=}\PYG{n}{X}\PYG{l+s+s1}{\PYGZsq{}}\PYG{l+s+s1}{060200000000000000000300}\PYG{l+s+s1}{\PYGZsq{}}
  \PYG{n}{MODEEND}
  \PYG{n}{END}
\end{sphinxVerbatim}

\sphinxAtStartPar
\sphinxstyleemphasis{VTAM logon mode table MODVIRT}

\index{VTAM Definitions@\spxentry{VTAM Definitions}!USSTAB for 3270 terninals@\spxentry{USSTAB for 3270 terninals}}\index{USSTAB for 3270 terninals@\spxentry{USSTAB for 3270 terninals}!VTAM Definitions@\spxentry{VTAM Definitions}}\ignorespaces 

\subsection{USSTAB Support}
\label{\detokenize{Installation_Guide:usstab-support}}\label{\detokenize{Installation_Guide:index-24}}

\subsubsection{USSTAB Support for 3270 terminals}
\label{\detokenize{Installation_Guide:usstab-support-for-3270-terminals}}
\sphinxAtStartPar
USSTAB support is provided by VIRTEL either as a JavaScript template or by loading the users existing VTAM MSG10 USS buffer.

\index{Support for USSTAB@\spxentry{Support for USSTAB}!Using Javascript@\spxentry{Using Javascript}}\index{Using Javascript@\spxentry{Using Javascript}!Support for USSTAB@\spxentry{Support for USSTAB}}\ignorespaces 

\subsubsection{USSTAB support using Javascript.}
\label{\detokenize{Installation_Guide:usstab-support-using-javascript}}\label{\detokenize{Installation_Guide:index-25}}
\sphinxAtStartPar
For further information on this options see the following Technical Newsletters:\sphinxhyphen{}
\begin{itemize}
\item {} 
\sphinxAtStartPar
TN201411 Building a VTAM USSTAB using Virtel’s Web Access Facility.

\end{itemize}

\sphinxAtStartPar
The latest versions of these newsletter documents can be found online at \sphinxurl{http://virtel.readthedocs.io/en/latest/}

\index{Support for USSTAB@\spxentry{Support for USSTAB}!VTAM MSG10 USS interface@\spxentry{VTAM MSG10 USS interface}}\index{VTAM MSG10 USS interface@\spxentry{VTAM MSG10 USS interface}!Support for USSTAB@\spxentry{Support for USSTAB}}\ignorespaces 

\subsubsection{USSTAB Support for VTAM MSG10}
\label{\detokenize{Installation_Guide:usstab-support-for-vtam-msg10}}\label{\detokenize{Installation_Guide:index-26}}
\sphinxAtStartPar
The USSMSG MSG10 support is implemented by the the VIRTEL VIR0021W USSTAB menu program. This program will interrogate the customers VTAM USSTAB MSG10 module and create an equivalent 3270 MAP. The MAP will be passed to the VIR0010 routine where it will subsequently be converted into a HTML template and served to the browser. The generated template will provide similar functionality to that of the VTAM USSMSG10, that being a presentation screen and support for USSCMD and USSPARM entries. This allows customers to maintain their USSTAB MSG10 presentation for both VTAM and VIRTEL users without modification. The customers assembled USSTAB module, normally found in USER.VTAMLIB or an equivalent library, must be made available to VIRTEL. This can be done by either copying the module to a VIRTEL steplib library or concatenating the USER.VTAMLIB library into the VIRTEL started procedure.

\sphinxAtStartPar
To support this interface a transaction must be added to the Entry point defining the USSTAB module to be used. Here is an example:\sphinxhyphen{}

\begin{sphinxVerbatim}[commandchars=\\\{\}]
    \PYG{n}{TRANSACT} \PYG{n}{ID}\PYG{o}{=}\PYG{n}{CLI}\PYG{o}{\PYGZhy{}}\PYG{l+m+mi}{16}\PYG{n}{A}\PYG{p}{,} \PYG{o}{\PYGZhy{}}
\PYG{n}{NAME}\PYG{o}{=}\PYG{n}{VTAMUSS}\PYG{p}{,} \PYG{o}{\PYGZhy{}}
\PYG{n}{DESC}\PYG{o}{=}\PYG{l+s+s1}{\PYGZsq{}}\PYG{l+s+s1}{Logon through USSTAB}\PYG{l+s+s1}{\PYGZsq{}}\PYG{p}{,} \PYG{o}{\PYGZhy{}}
\PYG{n}{APPL}\PYG{o}{=}\PYG{n}{VIR0021W}\PYG{p}{,} \PYG{o}{\PYGZhy{}}
\PYG{n}{TYPE}\PYG{o}{=}\PYG{l+m+mi}{2}\PYG{p}{,} \PYG{o}{\PYGZhy{}}
\PYG{n}{TERMINAL}\PYG{o}{=}\PYG{n}{CLVTA}\PYG{p}{,} \PYG{o}{\PYGZhy{}}
\PYG{n}{STARTUP}\PYG{o}{=}\PYG{l+m+mi}{1}\PYG{p}{,} \PYG{o}{\PYGZhy{}}
\PYG{n}{SECURITY}\PYG{o}{=}\PYG{l+m+mi}{1}\PYG{p}{,} \PYG{o}{\PYGZhy{}}
\PYG{n}{LOGMSG}\PYG{o}{=}\PYG{l+s+s1}{\PYGZsq{}}\PYG{l+s+s1}{usstab=USSVIRT}\PYG{l+s+s1}{\PYGZsq{}}
\end{sphinxVerbatim}

\sphinxAtStartPar
\sphinxincludegraphics{{image3}.png}

\sphinxAtStartPar
\sphinxstylestrong{System Symbolic support}

\sphinxAtStartPar
VIR0021W supports system symbolics within the MSG10 area and in the usstab= keyword. For example in the MSG10 buffer definition any area that has a symbolic will be translated using the values defined in the SYS1.PARMLIB member IEASYMXX. The z/OS command \sphinxstyleemphasis{D SYMBOLS} can be use to list the defined system symbolics.

\begin{sphinxVerbatim}[commandchars=\\\{\}]
\PYG{n}{DC} \PYG{n}{CL80}\PYG{l+s+s1}{\PYGZsq{}}\PYG{l+s+s1}{ This is system \PYGZam{}\PYGZam{}SYSNAME. running on SYSPLEX \PYGZam{}\PYGZam{}SYSPLEX.}\PYG{l+s+s1}{\PYGZsq{}}
\end{sphinxVerbatim}

\sphinxAtStartPar
would be translated and displayed as:\sphinxhyphen{}

\begin{sphinxVerbatim}[commandchars=\\\{\}]
\PYG{n}{DC} \PYG{n}{CL80}\PYG{l+s+s1}{\PYGZsq{}}\PYG{l+s+s1}{ This is system ZAMVS1 running on SYSPLEX PLEX1}\PYG{l+s+s1}{\PYGZsq{}}
\end{sphinxVerbatim}

\sphinxAtStartPar
In the usstab=keyword, in the transaction definition, a symbolic can form part of the usstab name. This enables different usstabs to be loaded depending on the system. For example the following would load USSTAB1A if the symbolic \&SYSCLONE. was defined as 1A.

\begin{sphinxVerbatim}[commandchars=\\\{\}]
\PYG{n}{usstab}\PYG{o}{=}\PYG{n}{usstab}\PYG{o}{\PYGZam{}}\PYG{n}{SYSCLONE}\PYG{o}{.}
\end{sphinxVerbatim}

\sphinxAtStartPar
\sphinxstylestrong{Feedback support}

\sphinxAtStartPar
The errmsg=y keyword can be appended to usstab=name. This additional keyword has been added to support feedback within the support module. If errmsg=y is coded then any error messages will be displayed on line 24. If the USSTAB uses row 24 for internal presentation purposes then the errmsg= keyword should not be used. Feedback will report on invalid application selection and invalid key usage. Any other error messqges will be reported to the Virtel SYSLOG. For example:\sphinxhyphen{}

\begin{sphinxVerbatim}[commandchars=\\\{\}]
\PYG{n}{usstab}\PYG{o}{=}\PYG{n}{usstab}\PYG{o}{\PYGZam{}}\PYG{n}{Sysclone}\PYG{o}{.}\PYG{p}{,}\PYG{n}{errmsg}\PYG{o}{=}\PYG{n}{y}
\end{sphinxVerbatim}

\sphinxAtStartPar
Will load a USSTAB module depending on the \&SYSCLONE. system symbolic value and report error messages to line 24.

\sphinxAtStartPar
\sphinxstylestrong{Default USSTAB module USSVIRT}
A default USSTAB module is shipped in the VIRTEL loadlib called USSVIRT. This USS module is loaded by default should the user’s USSTAB fail to load. The default USSTAB looks like:\sphinxhyphen{} The source is maintained in the SAMPLIB library.

\sphinxAtStartPar
\sphinxincludegraphics{{image4}.png}

\sphinxAtStartPar
\sphinxstylestrong{Constraints}

\sphinxAtStartPar
The user USSTAB MSG10 module that VIR0021W processes must adhere to certian constraints else results will be unpredicatble. The constraints are:\sphinxhyphen{}
\begin{itemize}
\item {} 
\sphinxAtStartPar
VIR0021W attempts to simulate the functionality of USSMSG10 processing. It doesn’t distinguished between PL1 and BAL code strings. So, for example, it will process either  LOGON APPLID(TSO) or LOGON APPLID=TSO. Both would be accepted.

\item {} 
\sphinxAtStartPar
Use only SBA Buffer addresses. Do not use displacement addresses such as AL2(((24\sphinxhyphen{}1)*80)+(80\sphinxhyphen{}1)).

\item {} 
\sphinxAtStartPar
The USSTAB module should only have ONE UNPROTECTED output field. This is used for the command area. It should not exceed 78 bytes. Use the tab key to check only one input area is defined.

\item {} 
\sphinxAtStartPar
Use extended attribute options rather than the basic field attributes to control colours. Some of the basic attributes imply unprotected fields.

\item {} 
\sphinxAtStartPar
The SCAN parameter must be included on the BUFFER statement:\sphinxhyphen{}

\end{itemize}

\begin{sphinxVerbatim}[commandchars=\\\{\}]
\PYG{n}{MSG10}    \PYG{n}{USSMSG}  \PYG{n}{MSG}\PYG{o}{=}\PYG{l+m+mi}{10}\PYG{p}{,}\PYG{n}{BUFFER}\PYG{o}{=}\PYG{p}{(}\PYG{n}{MSG10BUF}\PYG{p}{,}\PYG{n}{SCAN}\PYG{p}{)}
\end{sphinxVerbatim}
\begin{itemize}
\item {} 
\sphinxAtStartPar
The cursor position must be preceded by a SBA and terminated with a SF:\sphinxhyphen{}

\end{itemize}

\begin{sphinxVerbatim}[commandchars=\\\{\}]
\PYG{n}{DC}    \PYG{n}{X}\PYG{l+s+s1}{\PYGZsq{}}\PYG{l+s+s1}{115B5F}\PYG{l+s+s1}{\PYGZsq{}}                    \PYG{n}{R}\PYG{o}{=}\PYG{l+m+mi}{22}\PYG{p}{,}\PYG{n}{C}\PYG{o}{=}\PYG{l+m+mi}{80}
\PYG{n}{DC}    \PYG{n}{X}\PYG{l+s+s1}{\PYGZsq{}}\PYG{l+s+s1}{1D40}\PYG{l+s+s1}{\PYGZsq{}}                      \PYG{n}{UNPROTECTED} \PYG{n}{ON} \PYG{l+m+mi}{23}
\PYG{n}{DC}    \PYG{n}{X}\PYG{l+s+s1}{\PYGZsq{}}\PYG{l+s+s1}{13}\PYG{l+s+s1}{\PYGZsq{}}                        \PYG{n}{INSERTCURSOR}
\PYG{n}{DC}    \PYG{l+m+mi}{50}\PYG{n}{C}\PYG{l+s+s1}{\PYGZsq{}}\PYG{l+s+s1}{ }\PYG{l+s+s1}{\PYGZsq{}}                       \PYG{n}{Blank} \PYG{n}{command} \PYG{n}{area}
\PYG{n}{DC}    \PYG{n}{X}\PYG{l+s+s1}{\PYGZsq{}}\PYG{l+s+s1}{1D60}\PYG{l+s+s1}{\PYGZsq{}}                      \PYG{n}{End} \PYG{n}{of} \PYG{n}{command} \PYG{n}{line}
\end{sphinxVerbatim}

\index{VTAM Definitions@\spxentry{VTAM Definitions}!USSTAB for VIRTEL/PC@\spxentry{USSTAB for VIRTEL/PC}}\index{USSTAB for VIRTEL/PC@\spxentry{USSTAB for VIRTEL/PC}!VTAM Definitions@\spxentry{VTAM Definitions}}\ignorespaces 

\subsubsection{USSTAB Support for VIRTEL/PC}
\label{\detokenize{Installation_Guide:usstab-support-for-virtel-pc}}\label{\detokenize{Installation_Guide:index-27}}
\sphinxAtStartPar
For VIRTEL/PC it may be necessary to provide a customized USS table in the VTAM library. An example USS table is shown in the figure below. A USS table is not necessary for VIRTEL Web Access access.

\begin{sphinxVerbatim}[commandchars=\\\{\}]
      \PYG{n}{USSTAB}
\PYG{n}{USSCMD} \PYG{n}{CMD}\PYG{o}{=}\PYG{n}{MA}\PYG{p}{,}\PYG{n}{REP}\PYG{o}{=}\PYG{n}{LOGON}\PYG{p}{,}\PYG{n}{FORMAT}\PYG{o}{=}\PYG{n}{BAL}
\PYG{n}{USSPARM} \PYG{n}{PARM}\PYG{o}{=}\PYG{n}{APPLID}\PYG{p}{,}\PYG{n}{DEFAULT}\PYG{o}{=}\PYG{n}{VIRTEL2}
\PYG{n}{USSCMD} \PYG{n}{CMD}\PYG{o}{=}\PYG{n}{P}\PYG{p}{,}\PYG{n}{REP}\PYG{o}{=}\PYG{n}{LOGON}\PYG{p}{,}\PYG{n}{FORMAT}\PYG{o}{=}\PYG{n}{BAL}
\PYG{n}{USSPARM} \PYG{n}{PARM}\PYG{o}{=}\PYG{n}{APPLID}\PYG{p}{,}\PYG{n}{DEFAULT}\PYG{o}{=}\PYG{n}{VIRTEL2}
\PYG{n}{USSPARM} \PYG{n}{PARM}\PYG{o}{=}\PYG{n}{DATA}\PYG{p}{,}\PYG{n}{DEFAULT}\PYG{o}{=}\PYG{l+s+s1}{\PYGZsq{}}\PYG{l+s+s1}{PC}\PYG{l+s+s1}{\PYGZsq{}}
\PYG{n}{USSMSG} \PYG{n}{MSG}\PYG{o}{=}\PYG{l+m+mi}{00}\PYG{p}{,}\PYG{n}{BUFFER}\PYG{o}{=}\PYG{n}{MSG00}
\PYG{n}{USSMSG} \PYG{n}{MSG}\PYG{o}{=}\PYG{l+m+mi}{01}\PYG{p}{,}\PYG{n}{BUFFER}\PYG{o}{=}\PYG{n}{MSG02}
\PYG{n}{USSMSG} \PYG{n}{MSG}\PYG{o}{=}\PYG{l+m+mi}{02}\PYG{p}{,}\PYG{n}{BUFFER}\PYG{o}{=}\PYG{n}{MSG02}
\PYG{n}{USSMSG} \PYG{n}{MSG}\PYG{o}{=}\PYG{l+m+mi}{03}\PYG{p}{,}\PYG{n}{BUFFER}\PYG{o}{=}\PYG{n}{MSG03}
\PYG{n}{USSMSG} \PYG{n}{MSG}\PYG{o}{=}\PYG{l+m+mi}{04}\PYG{p}{,}\PYG{n}{BUFFER}\PYG{o}{=}\PYG{n}{MSG04}
\PYG{n}{USSMSG} \PYG{n}{MSG}\PYG{o}{=}\PYG{l+m+mi}{05}\PYG{p}{,}\PYG{n}{BUFFER}\PYG{o}{=}\PYG{n}{MSG02}
\PYG{n}{USSMSG} \PYG{n}{MSG}\PYG{o}{=}\PYG{l+m+mi}{06}\PYG{p}{,}\PYG{n}{BUFFER}\PYG{o}{=}\PYG{n}{MSG02}
\PYG{n}{USSMSG} \PYG{n}{MSG}\PYG{o}{=}\PYG{l+m+mi}{07}\PYG{p}{,}\PYG{n}{BUFFER}\PYG{o}{=}\PYG{n}{MSG04}
\PYG{n}{USSMSG} \PYG{n}{MSG}\PYG{o}{=}\PYG{l+m+mi}{08}\PYG{p}{,}\PYG{n}{BUFFER}\PYG{o}{=}\PYG{n}{MSG02}
\PYG{n}{USSMSG} \PYG{n}{MSG}\PYG{o}{=}\PYG{l+m+mi}{09}\PYG{p}{,}\PYG{n}{BUFFER}\PYG{o}{=}\PYG{n}{MSG02}
\PYG{n}{USSMSG} \PYG{n}{MSG}\PYG{o}{=}\PYG{l+m+mi}{10}\PYG{p}{,}\PYG{n}{BUFFER}\PYG{o}{=}\PYG{n}{MSG10}
\PYG{n}{USSMSG} \PYG{n}{MSG}\PYG{o}{=}\PYG{l+m+mi}{11}\PYG{p}{,}\PYG{n}{BUFFER}\PYG{o}{=}\PYG{n}{MSG10}
\PYG{n}{USSMSG} \PYG{n}{MSG}\PYG{o}{=}\PYG{l+m+mi}{12}\PYG{p}{,}\PYG{n}{BUFFER}\PYG{o}{=}\PYG{n}{MSG02}
\PYG{n}{MSG00} \PYG{n}{DC} \PYG{n}{Y}\PYG{p}{(}\PYG{n}{MSG0F}\PYG{o}{\PYGZhy{}}\PYG{o}{*}\PYG{o}{\PYGZhy{}}\PYG{l+m+mi}{2}\PYG{p}{)}
      \PYG{n}{DC} \PYG{n}{X}\PYG{l+s+s1}{\PYGZsq{}}\PYG{l+s+s1}{0D}\PYG{l+s+s1}{\PYGZsq{}}\PYG{p}{,}\PYG{n}{C}\PYG{l+s+s1}{\PYGZsq{}}\PYG{l+s+s1}{DEMANDE PRISE EN COMPTE}\PYG{l+s+s1}{\PYGZsq{}}\PYG{p}{,}\PYG{n}{X}\PYG{l+s+s1}{\PYGZsq{}}\PYG{l+s+s1}{0D}\PYG{l+s+s1}{\PYGZsq{}}
\PYG{n}{MSG0F} \PYG{n}{EQU} \PYG{o}{*}
\PYG{n}{MSG02} \PYG{n}{DC} \PYG{n}{Y}\PYG{p}{(}\PYG{n}{MSG2F}\PYG{o}{\PYGZhy{}}\PYG{o}{*}\PYG{o}{\PYGZhy{}}\PYG{l+m+mi}{2}\PYG{p}{)}
      \PYG{n}{DC} \PYG{n}{X}\PYG{l+s+s1}{\PYGZsq{}}\PYG{l+s+s1}{0D}\PYG{l+s+s1}{\PYGZsq{}}\PYG{p}{,}\PYG{n}{C}\PYG{l+s+s1}{\PYGZsq{}}\PYG{l+s+s1}{CHOIX NON PREVU}\PYG{l+s+s1}{\PYGZsq{}}\PYG{p}{,}\PYG{n}{X}\PYG{l+s+s1}{\PYGZsq{}}\PYG{l+s+s1}{0D}\PYG{l+s+s1}{\PYGZsq{}}
\PYG{n}{MSG2F} \PYG{n}{EQU} \PYG{o}{*}
\PYG{n}{MSG03} \PYG{n}{DC} \PYG{n}{Y}\PYG{p}{(}\PYG{n}{MSG3F}\PYG{o}{\PYGZhy{}}\PYG{o}{*}\PYG{o}{\PYGZhy{}}\PYG{l+m+mi}{2}\PYG{p}{)}
      \PYG{n}{DC} \PYG{n}{X}\PYG{l+s+s1}{\PYGZsq{}}\PYG{l+s+s1}{0D}\PYG{l+s+s1}{\PYGZsq{}}\PYG{p}{,}\PYG{n}{C}\PYG{l+s+s1}{\PYGZsq{}}\PYG{l+s+s1}{CODE RETOUR INCONNU}\PYG{l+s+s1}{\PYGZsq{}}\PYG{p}{,}\PYG{n}{X}\PYG{l+s+s1}{\PYGZsq{}}\PYG{l+s+s1}{0D}\PYG{l+s+s1}{\PYGZsq{}}
\PYG{n}{MSG3F} \PYG{n}{EQU} \PYG{o}{*}
\PYG{n}{MSG04} \PYG{n}{DC} \PYG{n}{Y}\PYG{p}{(}\PYG{n}{MSG4F}\PYG{o}{\PYGZhy{}}\PYG{o}{*}\PYG{o}{\PYGZhy{}}\PYG{l+m+mi}{2}\PYG{p}{)}
      \PYG{n}{DC} \PYG{n}{X}\PYG{l+s+s1}{\PYGZsq{}}\PYG{l+s+s1}{0D}\PYG{l+s+s1}{\PYGZsq{}}\PYG{p}{,}\PYG{n}{C}\PYG{l+s+s1}{\PYGZsq{}}\PYG{l+s+s1}{SERVEUR INDISPONIBLE}\PYG{l+s+s1}{\PYGZsq{}}\PYG{p}{,}\PYG{n}{X}\PYG{l+s+s1}{\PYGZsq{}}\PYG{l+s+s1}{0D}\PYG{l+s+s1}{\PYGZsq{}}
\PYG{n}{MSG4F} \PYG{n}{EQU} \PYG{o}{*}
\PYG{n}{MSG10} \PYG{n}{DC} \PYG{n}{Y}\PYG{p}{(}\PYG{n}{MSG10F}\PYG{o}{\PYGZhy{}}\PYG{o}{*}\PYG{o}{\PYGZhy{}}\PYG{l+m+mi}{2}\PYG{p}{)}
      \PYG{n}{DC} \PYG{n}{X}\PYG{l+s+s1}{\PYGZsq{}}\PYG{l+s+s1}{0C}\PYG{l+s+s1}{\PYGZsq{}}\PYG{p}{,}\PYG{n}{C}\PYG{l+s+s1}{\PYGZsq{}}\PYG{l+s+s1}{TAPEZ UN IDENTIFIANT PUIS ENVOI }\PYG{l+s+s1}{\PYGZsq{}}
      \PYG{n}{DC} \PYG{n}{X}\PYG{l+s+s1}{\PYGZsq{}}\PYG{l+s+s1}{0A}\PYG{l+s+s1}{\PYGZsq{}}\PYG{p}{,}\PYG{n}{C}\PYG{l+s+s1}{\PYGZsq{}}\PYG{l+s+s1}{ M POUR UN MINITEL }\PYG{l+s+s1}{\PYGZsq{}}
      \PYG{n}{DC} \PYG{n}{X}\PYG{l+s+s1}{\PYGZsq{}}\PYG{l+s+s1}{0A}\PYG{l+s+s1}{\PYGZsq{}}\PYG{p}{,}\PYG{n}{C}\PYG{l+s+s1}{\PYGZsq{}}\PYG{l+s+s1}{ P POUR UN PC }\PYG{l+s+s1}{\PYGZsq{}}
\PYG{n}{MSG10F} \PYG{n}{EQU} \PYG{o}{*}
      \PYG{n}{USSEND}
      \PYG{n}{END}
\end{sphinxVerbatim}

\sphinxAtStartPar
\sphinxstyleemphasis{VTAM USS table for Virtel/PC}

\index{CICS Definitions@\spxentry{CICS Definitions}!Defining Virtel to CICS@\spxentry{Defining Virtel to CICS}}\index{Defining Virtel to CICS@\spxentry{Defining Virtel to CICS}!CICS Definitions@\spxentry{CICS Definitions}}\ignorespaces 

\subsection{CICS Definitions}
\label{\detokenize{Installation_Guide:cics-definitions}}\label{\detokenize{Installation_Guide:index-28}}
\sphinxAtStartPar
When a VIRTEL Web Access terminal logs on via VIRTEL to CICS, the application relay LU represents the terminal as seen by CICS.The relay LU must therefore be referenced in the CICS CSD file, or alternatively configured by the AUTOINSTALL program of your site that will decide which TYPETERM to assign to which relay.


\subsubsection{VIRTEL CICS Sample definitions}
\label{\detokenize{Installation_Guide:virtel-cics-sample-definitions}}
\sphinxAtStartPar
The following example shows CSD definitions for VIRTEL Web Access terminals. The NETNAME parameter must match the “Relay” name specified in the definition of the VIRTEL terminals attached to the HTTP line. For more details, refer to the section entitled “Definition of an HTTP line” in the VIRTEL Configuration Reference documentation.

\begin{sphinxVerbatim}[commandchars=\\\{\}]
\PYG{o}{*} \PYG{n}{VIRTEL} \PYG{l+m+mi}{3270} \PYG{n}{TERMINALS} \PYG{n}{FOR} \PYG{n}{WEB2HOST}
\PYG{n}{DEFINE} \PYG{n}{TERMINAL}\PYG{p}{(}\PYG{n}{T000}\PYG{p}{)} \PYG{n}{GROUP}\PYG{p}{(}\PYG{n}{VIRTEL}\PYG{p}{)} \PYG{n}{TYPETERM}\PYG{p}{(}\PYG{n}{DFHLU2E2}\PYG{p}{)}
      \PYG{n}{NETNAME}\PYG{p}{(}\PYG{n}{RHTVT000}\PYG{p}{)} \PYG{n}{PRINTER}\PYG{p}{(}\PYG{n}{I000}\PYG{p}{)}
      \PYG{n}{DESC}\PYG{p}{(}\PYG{n}{VIRTEL} \PYG{n}{WEB} \PYG{n}{TO} \PYG{n}{HOST} \PYG{n}{TERMINAL}\PYG{p}{)}
\PYG{n}{DEFINE} \PYG{n}{TERMINAL}\PYG{p}{(}\PYG{n}{T001}\PYG{p}{)} \PYG{n}{GROUP}\PYG{p}{(}\PYG{n}{VIRTEL}\PYG{p}{)} \PYG{n}{TYPETERM}\PYG{p}{(}\PYG{n}{DFHLU2E2}\PYG{p}{)}
      \PYG{n}{NETNAME}\PYG{p}{(}\PYG{n}{RHTVT001}\PYG{p}{)} \PYG{n}{PRINTER}\PYG{p}{(}\PYG{n}{I001}\PYG{p}{)}
      \PYG{n}{DESC}\PYG{p}{(}\PYG{n}{VIRTEL} \PYG{n}{WEB} \PYG{n}{TO} \PYG{n}{HOST} \PYG{n}{TERMINAL}\PYG{p}{)}
\PYG{n}{DEFINE} \PYG{n}{TERMINAL}\PYG{p}{(}\PYG{n}{T002}\PYG{p}{)} \PYG{n}{GROUP}\PYG{p}{(}\PYG{n}{VIRTEL}\PYG{p}{)} \PYG{n}{TYPETERM}\PYG{p}{(}\PYG{n}{DFHLU2E2}\PYG{p}{)}
      \PYG{n}{NETNAME}\PYG{p}{(}\PYG{n}{RHTVT002}\PYG{p}{)} \PYG{n}{PRINTER}\PYG{p}{(}\PYG{n}{I002}\PYG{p}{)}
      \PYG{n}{DESC}\PYG{p}{(}\PYG{n}{VIRTEL} \PYG{n}{WEB} \PYG{n}{TO} \PYG{n}{HOST} \PYG{n}{TERMINAL}\PYG{p}{)}
\PYG{n}{DEFINE} \PYG{n}{TERMINAL}\PYG{p}{(}\PYG{n}{T003}\PYG{p}{)} \PYG{n}{GROUP}\PYG{p}{(}\PYG{n}{VIRTEL}\PYG{p}{)} \PYG{n}{TYPETERM}\PYG{p}{(}\PYG{n}{DFHLU2E2}\PYG{p}{)}
      \PYG{n}{NETNAME}\PYG{p}{(}\PYG{n}{RHTVT003}\PYG{p}{)} \PYG{n}{PRINTER}\PYG{p}{(}\PYG{n}{I003}\PYG{p}{)}
      \PYG{n}{DESC}\PYG{p}{(}\PYG{n}{VIRTEL} \PYG{n}{WEB} \PYG{n}{TO} \PYG{n}{HOST} \PYG{n}{TERMINAL}\PYG{p}{)}
\end{sphinxVerbatim}

\sphinxAtStartPar
\sphinxstyleemphasis{CICS definitions for VIRTEL Web Access terminals}

\phantomsection\label{\detokenize{Installation_Guide:vvrrig-virtct}}
\index{The Virtel TCT@\spxentry{The Virtel TCT}}\ignorespaces 

\chapter{The Virtel TCT}
\label{\detokenize{Installation_Guide:the-virtel-tct}}\label{\detokenize{Installation_Guide:index-29}}

\section{introduction}
\label{\detokenize{Installation_Guide:id5}}
\sphinxAtStartPar
All the general information necessary for VIRTEL to run is contained in a table known as the VIRTCT. After initialising the different  parameters, this table must be assembled and link edited with the name VIRTCTxx, where xx are the two characters that identify the VIRTCT at start up time to the system. This xx value will be contained in the parameter of the PARM operand of the VIRTEL start procedure in z/OS, or behind the EXEC card in the z/VSE environment.

\sphinxAtStartPar
The VIRTCT must be assembled before VIRTEL can be run. At the time of the assembly the VIRTEL macro library VIRT4XX.MACLIB must be on\sphinxhyphen{}line. Options RENT and REUS must not be specified when assembling the VIRTCT for an z/OS environment. The resulting phase or load module must be placed in the library containing the other phases or load modules required by VIRTEL.

\sphinxAtStartPar
For z/OS, a sample VIRTCT source member is provided in the VIRTCT01 member of the VIRTEL SAMPLIB, and the assembly and link\sphinxhyphen{}edit JCL is in member ASMTCT. For z/VSE, a sample VIRTCT with assembly and link\sphinxhyphen{}edit JCL is in the VIRTCT installation job.


\section{Parameters Of The VIRTCT}
\label{\detokenize{Installation_Guide:parameters-of-the-virtct}}
\sphinxAtStartPar
Some parameters have a default value taken by VIRTEL and do not need to be coded in your table.

\index{Virtel TCT@\spxentry{Virtel TCT}!ACCUEIL parameter@\spxentry{ACCUEIL parameter}}\index{ACCUEIL parameter@\spxentry{ACCUEIL parameter}!Virtel TCT@\spxentry{Virtel TCT}}\ignorespaces 

\subsection{ACCUEIL parameter}
\label{\detokenize{Installation_Guide:accueil-parameter}}\label{\detokenize{Installation_Guide:index-30}}
\begin{sphinxVerbatim}[commandchars=\\\{\}]
\PYG{n}{ACCUEIL}\PYG{o}{=}\PYG{p}{(}\PYG{n}{YES}\PYG{o}{/}\PYG{n}{NO}\PYG{p}{[}\PYG{p}{,}\PYG{n}{KEEP}\PYG{p}{]}\PYG{p}{)} \PYG{n}{Default}\PYG{o}{=}\PYG{n}{YES}
\end{sphinxVerbatim}

\sphinxAtStartPar
\sphinxstylestrong{YES} \sphinxhyphen{} Terminals not defined in VIRTEL may be connected in ACCUEIL mode. That means the terminals will have access to all functions, excepting dialogue with another application (relay). The maximum number of terminals accepted in ACCUEIL mode is a function of the parameter of the operand NBDYNAM.

\sphinxAtStartPar
\sphinxstylestrong{NO} \sphinxhyphen{} Terminals not defined in VIRTEL may not be connected.

\sphinxAtStartPar
\sphinxstylestrong{KEEP} \sphinxhyphen{} Allows the Multi\sphinxhyphen{}Session screen to be used as a dynamic USSTAB without the terminals being associated with the application relays (See the heading ‘Using the dynamic USSTAB’ in the ‘VIRTEL Multi\sphinxhyphen{}Session’ chapter only available in French)

\index{Virtel TCT@\spxentry{Virtel TCT}!ADDR1 parameter@\spxentry{ADDR1 parameter}}\index{ADDR1 parameter@\spxentry{ADDR1 parameter}!Virtel TCT@\spxentry{Virtel TCT}}\ignorespaces 

\subsection{ADDR1 parameter}
\label{\detokenize{Installation_Guide:addr1-parameter}}\label{\detokenize{Installation_Guide:index-31}}
\begin{sphinxVerbatim}[commandchars=\\\{\}]
\PYG{n}{ADDR1}\PYG{o}{=}\PYG{l+s+s1}{\PYGZsq{}}\PYG{l+s+s1}{ }\PYG{l+s+s1}{\PYGZsq{}} \PYG{n}{Default}\PYG{o}{=}\PYG{l+s+s1}{\PYGZsq{}}\PYG{l+s+s1}{ }\PYG{l+s+s1}{\PYGZsq{}}
\end{sphinxVerbatim}

\sphinxAtStartPar
The address line 1 of the client as specified in the key at the time of installation. This parameter is unique to each client and functions in relation to the following parameters ADDR2, COMPANY, LICENSE, EXPIRE and CODE

\index{Virtel TCT@\spxentry{Virtel TCT}!ADDR2 parameter@\spxentry{ADDR2 parameter}}\index{ADDR2 parameter@\spxentry{ADDR2 parameter}!Virtel TCT@\spxentry{Virtel TCT}}\ignorespaces 

\subsection{ADDR2 parameter}
\label{\detokenize{Installation_Guide:addr2-parameter}}\label{\detokenize{Installation_Guide:index-32}}
\begin{sphinxVerbatim}[commandchars=\\\{\}]
\PYG{n}{ADDR2}\PYG{o}{=}\PYG{l+s+s1}{\PYGZsq{}}\PYG{l+s+s1}{ }\PYG{l+s+s1}{\PYGZsq{}} \PYG{n}{Default}\PYG{o}{=}\PYG{l+s+s1}{\PYGZsq{}}\PYG{l+s+s1}{ }\PYG{l+s+s1}{\PYGZsq{}}
\end{sphinxVerbatim}

\sphinxAtStartPar
The address line 2 of the client as specified in the key at the time of installation. This parameter is unique to each client and functions in relation to the following parameters ADDR1, COMPANY, LICENSE, EXPIRE and CODE.

\index{Virtel TCT@\spxentry{Virtel TCT}!AIC parameter@\spxentry{AIC parameter}}\index{AIC parameter@\spxentry{AIC parameter}!Virtel TCT@\spxentry{Virtel TCT}}\ignorespaces 

\subsection{AIC parameter}
\label{\detokenize{Installation_Guide:aic-parameter}}\label{\detokenize{Installation_Guide:index-33}}
\begin{sphinxVerbatim}[commandchars=\\\{\}]
\PYG{n}{AIC}\PYG{o}{=}\PYG{n}{APPLID}\PYG{o}{/}\PYG{n}{TRANSACT} \PYG{n}{Default}\PYG{o}{=}\PYG{n}{APPLID}
\end{sphinxVerbatim}

\sphinxAtStartPar
This parameter determines the value returned by the APPLICATION\sphinxhyphen{}IS\sphinxhyphen{}CONNECTED condition of the CREATE\sphinxhyphen{}VARIABLEIF tag (see “Signon and password management” in the VIRTEL Web Access Guide). This in turn affects the window title of the VIRTEL Web Access screen. The following values are possible:

\sphinxAtStartPar
\sphinxstylestrong{APPLID} \sphinxhyphen{} The tag returns the VTAM applid of the host application.

\sphinxAtStartPar
\sphinxstylestrong{TRANSACT} \sphinxhyphen{} The tag returns the external name of the VIRTEL transaction used to access the host application.

\index{Virtel TCT@\spxentry{Virtel TCT}!ANNUL parameter@\spxentry{ANNUL parameter}}\index{ANNUL parameter@\spxentry{ANNUL parameter}!Virtel TCT@\spxentry{Virtel TCT}}\ignorespaces 

\subsection{ANNUL parameter}
\label{\detokenize{Installation_Guide:annul-parameter}}\label{\detokenize{Installation_Guide:index-34}}
\begin{sphinxVerbatim}[commandchars=\\\{\}]
\PYG{n}{ANNUL}\PYG{o}{=}\PYG{n}{xx} \PYG{n}{Default}\PYG{o}{=}\PYG{l+m+mi}{6}\PYG{n}{D} \PYG{p}{(}\PYG{n}{Clear}\PYG{p}{)}
\end{sphinxVerbatim}

\sphinxAtStartPar
\sphinxstylestrong{xx} \sphinxhyphen{} The 3270 AID function key which will be transmitted to the application when the user presses the {[}ANNULATION{]} key. This parameter allows the user to define a general parameter by default which may be modified in the definition of the sub\sphinxhyphen{}server nodes.
ANNUL=00 allows the cursor to be placed at the start of the field with erasure of the field.

\index{Virtel TCT@\spxentry{Virtel TCT}!APPLID parameter@\spxentry{APPLID parameter}}\index{APPLID parameter@\spxentry{APPLID parameter}!Virtel TCT@\spxentry{Virtel TCT}}\ignorespaces 

\subsection{APPLID parameter}
\label{\detokenize{Installation_Guide:index-35}}\label{\detokenize{Installation_Guide:id6}}
\begin{sphinxVerbatim}[commandchars=\\\{\}]
\PYG{n}{APPLID}\PYG{o}{=}\PYG{n}{nappl} \PYG{n}{Default}\PYG{o}{=}\PYG{n}{VIRTEL}
\end{sphinxVerbatim}

\sphinxAtStartPar
\sphinxstylestrong{nappl} \sphinxhyphen{} The name of the primary VIRTEL ACB.

\sphinxAtStartPar
The APPLID parameter specifies the label or ACBNAME parameter of the VTAM APPL for the primary VIRTEL ACB. The value specified here can be overridden in the VIRTEL startup JCL (see “Executing VIRTEL in an z/OS environment”, page 26 or “Executing VIRTEL in a z/VSE environment”, page 46 for details). When no primary VTAM ACB is required (for example, in the VIRTCT for a VIRTEL Batch job), then this parameter may be coded as APPLID=*NOAPPL*

\sphinxAtStartPar
If SYSPLUS=YES is specified, a ‘+’ character in the APPLID will be replaced by the value of the SYSCLONE system symbol. SYSCLONE is specified in the IEASYMxx member of SYS1.PARMLIB, and identifies the particular LPAR that VIRTEL is running on in a sysplex environment.

\index{Virtel TCT@\spxentry{Virtel TCT}!APPSTAT parameter@\spxentry{APPSTAT parameter}}\index{APPSTAT parameter@\spxentry{APPSTAT parameter}!Virtel TCT@\spxentry{Virtel TCT}}\ignorespaces 

\subsection{APPSTAT parameter}
\label{\detokenize{Installation_Guide:appstat-parameter}}\label{\detokenize{Installation_Guide:index-36}}
\begin{sphinxVerbatim}[commandchars=\\\{\}]
\PYG{n}{APPSTAT}\PYG{o}{=}\PYG{n}{YES}\PYG{o}{/}\PYG{n}{NO} \PYG{n}{Default}\PYG{o}{=}\PYG{n}{NO}
\end{sphinxVerbatim}

\sphinxAtStartPar
\sphinxstylestrong{YES} \sphinxhyphen{} The status of the applications (active or non active) is tested at the time of access to the VIRTEL Web Access Application Selection Menu and the VIRTEL Multi\sphinxhyphen{}Session screen. For VIRTEL Web Access the status of each application is indicated by a color (see “Application Selection Menu” in the VIRTEL Web Access Guide). For VIRTEL Multi\sphinxhyphen{}Session the test is based on the value contained in the “STATUS” field of the application definition screen. The function key allowing access to the application will only appear if the application is active.

\sphinxAtStartPar
\sphinxstylestrong{NO} \sphinxhyphen{} The function key allowing access to the application is always present.

\index{Virtel TCT@\spxentry{Virtel TCT}!ARBO parameter@\spxentry{ARBO parameter}}\index{ARBO parameter@\spxentry{ARBO parameter}!Virtel TCT@\spxentry{Virtel TCT}}\ignorespaces 

\subsection{ARBO parameter}
\label{\detokenize{Installation_Guide:arbo-parameter}}\label{\detokenize{Installation_Guide:index-37}}
\begin{sphinxVerbatim}[commandchars=\\\{\}]
\PYG{n}{ARBO}\PYG{o}{=}\PYG{n}{YES}\PYG{o}{/}\PYG{n}{NO} \PYG{n}{Default}\PYG{o}{=}\PYG{n}{NO}
\end{sphinxVerbatim}

\sphinxAtStartPar
\sphinxstylestrong{YES} \sphinxhyphen{} The program for managing the tree structure will function as a VIRTEL internal sub\sphinxhyphen{}application.

\sphinxAtStartPar
\sphinxstylestrong{NO} \sphinxhyphen{} The tree structure management software will not function.

\index{Virtel TCT@\spxentry{Virtel TCT}!BATCH1 parameter@\spxentry{BATCH1 parameter}}\index{BATCH1 parameter@\spxentry{BATCH1 parameter}!Virtel TCT@\spxentry{Virtel TCT}}\ignorespaces 

\subsection{BATCH1 parameter}
\label{\detokenize{Installation_Guide:batch1-parameter}}\label{\detokenize{Installation_Guide:index-38}}
\begin{sphinxVerbatim}[commandchars=\\\{\}]
\PYG{n}{BATCH1}\PYG{o}{=}\PYG{p}{(}\PYG{n}{indd}\PYG{p}{,}\PYG{n}{indcb}\PYG{p}{,}\PYG{n}{outdd}\PYG{p}{,}\PYG{n}{outdcb}\PYG{p}{)} \PYG{n}{Default}\PYG{o}{=}\PYG{n}{no} \PYG{n}{batch} \PYG{n}{connection}
\end{sphinxVerbatim}

\sphinxAtStartPar
This parameter defines the batch processing characteristics for all lines which specify type BATCH1.

\sphinxAtStartPar
\sphinxstylestrong{indd} \sphinxhyphen{} The batch input DD name (for example, SYSIN).

\sphinxAtStartPar
\sphinxstylestrong{indcb} \sphinxhyphen{} The label of the DCB macro defining the batch input file. This DCB macro must appear later in the VIRTCT (see {\hyperref[\detokenize{Installation_Guide:vvrrig-bookmark72}]{\sphinxcrossref{\DUrole{std,std-ref}{“Additional parameters for batch files”}}}}).

\sphinxAtStartPar
\sphinxstylestrong{outdd} \sphinxhyphen{} The batch output DD name (for example, SYSPRINT).

\sphinxAtStartPar
\sphinxstylestrong{outdcb} \sphinxhyphen{} The label of the DCB macro defining the batch output file. This DCB macro must appear later in the VIRTCT (see {\hyperref[\detokenize{Installation_Guide:vvrrig-bookmark72}]{\sphinxcrossref{\DUrole{std,std-ref}{“Additional parameters for batch files”}}}}).

\index{Virtel TCT@\spxentry{Virtel TCT}!BATCH2 parameter@\spxentry{BATCH2 parameter}}\index{BATCH2 parameter@\spxentry{BATCH2 parameter}!Virtel TCT@\spxentry{Virtel TCT}}\ignorespaces 

\subsection{BATCH2 parameter}
\label{\detokenize{Installation_Guide:batch2-parameter}}\label{\detokenize{Installation_Guide:index-39}}
\begin{sphinxVerbatim}[commandchars=\\\{\}]
\PYG{n}{BATCH2}\PYG{o}{=}\PYG{p}{(}\PYG{n}{indd}\PYG{p}{,}\PYG{n}{indcb}\PYG{p}{,}\PYG{n}{outdd}\PYG{p}{,}\PYG{n}{outdcb}\PYG{p}{)} \PYG{n}{Default}\PYG{o}{=}\PYG{n}{no} \PYG{l+m+mi}{2}\PYG{n}{nd} \PYG{n}{batch} \PYG{n}{connection}
\end{sphinxVerbatim}

\sphinxAtStartPar
This parameter defines the batch processing characteristics for all lines which specify type BATCH2. The subparameters are the same as those of the BATCH1 parameter.

\index{Virtel TCT@\spxentry{Virtel TCT}!BFVSAM parameter@\spxentry{BFVSAM parameter}}\index{BFVSAM parameter@\spxentry{BFVSAM parameter}!Virtel TCT@\spxentry{Virtel TCT}}\ignorespaces 

\subsection{BFVSAM parameter}
\label{\detokenize{Installation_Guide:bfvsam-parameter}}\label{\detokenize{Installation_Guide:index-40}}
\begin{sphinxVerbatim}[commandchars=\\\{\}]
\PYG{n}{BFVSAM}\PYG{o}{=}\PYG{n}{n} \PYG{n}{Default}\PYG{o}{=}\PYG{l+m+mi}{8192}
\end{sphinxVerbatim}

\sphinxAtStartPar
\sphinxstylestrong{n} \sphinxhyphen{} Size of VSAM buffer (“CI size”) used by VIRTEL for reading files such as GTVSAM. As a general rule, this value is calculated by VIRTEL and should not be modified. The size is normally 8192.

\index{Virtel TCT@\spxentry{Virtel TCT}!BUFDATA parameter@\spxentry{BUFDATA parameter}}\index{BUFDATA parameter@\spxentry{BUFDATA parameter}!Virtel TCT@\spxentry{Virtel TCT}}\ignorespaces 

\subsection{BUFDATA parameter}
\label{\detokenize{Installation_Guide:bufdata-parameter}}\label{\detokenize{Installation_Guide:index-41}}
\begin{sphinxVerbatim}[commandchars=\\\{\}]
\PYG{n}{BUFDATA}\PYG{o}{=}\PYG{n}{n} \PYG{n}{Default}\PYG{o}{=}\PYG{l+m+mi}{16}
\end{sphinxVerbatim}

\sphinxAtStartPar
\sphinxstylestrong{n} \sphinxhyphen{} The number of VSAM buffers in the pool allocated for file access.

\index{Virtel TCT@\spxentry{Virtel TCT}!BUFSIZE parameter@\spxentry{BUFSIZE parameter}}\index{BUFSIZE parameter@\spxentry{BUFSIZE parameter}!Virtel TCT@\spxentry{Virtel TCT}}\ignorespaces 

\subsection{BUFSIZE parameter}
\label{\detokenize{Installation_Guide:bufsize-parameter}}\label{\detokenize{Installation_Guide:index-42}}
\begin{sphinxVerbatim}[commandchars=\\\{\}]
\PYG{n}{BUFSIZE}\PYG{o}{=}\PYG{n}{n} \PYG{n}{Default}\PYG{o}{=}\PYG{l+m+mi}{8192}
\end{sphinxVerbatim}

\sphinxAtStartPar
\sphinxstylestrong{n} \sphinxhyphen{} The size of the largest VTAM message that may pass through VIRTEL. Generally this value should not be modified. The size is generally 8192.

\index{Virtel TCT@\spxentry{Virtel TCT}!CHARSET parameter@\spxentry{CHARSET parameter}}\index{CHARSET parameter@\spxentry{CHARSET parameter}!Virtel TCT@\spxentry{Virtel TCT}}\ignorespaces 

\subsection{CHARSET parameter}
\label{\detokenize{Installation_Guide:charset-parameter}}\label{\detokenize{Installation_Guide:index-43}}
\begin{sphinxVerbatim}[commandchars=\\\{\}]
\PYG{n}{CHARSET}\PYG{o}{=}\PYG{p}{(}\PYG{n}{charset1}\PYG{p}{,}\PYG{n}{charset2}\PYG{p}{,}\PYG{o}{.}\PYG{o}{.}\PYG{o}{.}\PYG{p}{)} \PYG{n}{Default}\PYG{o}{=}\PYG{n}{none}
\end{sphinxVerbatim}

\sphinxAtStartPar
The CHARSET parameter allows tables of non\sphinxhyphen{}standard character sets to be loaded into VIRTEL at startup time. DBCS tables, because of their size, are not loaded by default into VIRTEL and must be explicitly requested using this parameter. The standard and non\sphinxhyphen{}standard tables are used for EBCDIC \sphinxhyphen{} UTF\sphinxhyphen{}8 translation and can be specified by the SET\sphinxhyphen{}OUTPUT\sphinxhyphen{}ENCODING\sphinxhyphen{}UTF\sphinxhyphen{}8 tag and by the DEFUTF8 parameter of the VIRTCT. Refer to the description of the DEFUTF8 parameter 55 for the list of standard tables which are always loaded into VIRTEL.
charset

\sphinxAtStartPar
The following non\sphinxhyphen{}standard tables can be loaded:
\begin{itemize}
\item {} 
\sphinxAtStartPar
IBM933A: Korean host mixed

\item {} 
\sphinxAtStartPar
IBM1364: Korean host mixed extended

\item {} 
\sphinxAtStartPar
IBM1388: Chinese simplified SBCS et DBCS

\item {} 
\sphinxAtStartPar
IBM1390: Japanese Katakana\sphinxhyphen{}Kanji

\item {} 
\sphinxAtStartPar
IBM1399: Japanese Latin\sphinxhyphen{}Kanji

\item {} 
\sphinxAtStartPar
IBM0276: French Canadian

\item {} 
\sphinxAtStartPar
IBM0803: Hebrew Set A old code

\item {} 
\sphinxAtStartPar
IBM4899: Hebrew Set A old code including Euro and Sheqel

\item {} 
\sphinxAtStartPar
IBM0838: Thailand

\item {} 
\sphinxAtStartPar
IBM1160: Thailand with Euro sign

\end{itemize}

\index{Virtel TCT@\spxentry{Virtel TCT}!CODE parameter@\spxentry{CODE parameter}}\index{CODE parameter@\spxentry{CODE parameter}!Virtel TCT@\spxentry{Virtel TCT}}\ignorespaces 

\subsection{CODE parameter}
\label{\detokenize{Installation_Guide:code-parameter}}\label{\detokenize{Installation_Guide:index-44}}
\begin{sphinxVerbatim}[commandchars=\\\{\}]
\PYG{n}{CODE}\PYG{o}{=}\PYG{l+s+s1}{\PYGZsq{}}\PYG{l+s+s1}{xxxxxxxx}\PYG{l+s+s1}{\PYGZsq{}} \PYG{n}{Default}\PYG{o}{=}\PYG{l+s+s1}{\PYGZsq{}}\PYG{l+s+s1}{ }\PYG{l+s+s1}{\PYGZsq{}}
\end{sphinxVerbatim}

\sphinxAtStartPar
\sphinxstylestrong{xxxxxxx} \sphinxhyphen{} Is the code calculated for the client as it is specified in the installation key at the time of the installation. This code
is unique for each client and functions in relation to the following parameters: ADDR1, ADDR2, COMPANY, LICENSE, and EXPIRE.

\index{Virtel TCT@\spxentry{Virtel TCT}!COMPANY parameter@\spxentry{COMPANY parameter}}\index{COMPANY parameter@\spxentry{COMPANY parameter}!Virtel TCT@\spxentry{Virtel TCT}}\ignorespaces 

\subsection{COMPANY parameter}
\label{\detokenize{Installation_Guide:company-parameter}}\label{\detokenize{Installation_Guide:index-45}}
\begin{sphinxVerbatim}[commandchars=\\\{\}]
\PYG{n}{COMPANY}\PYG{o}{=}\PYG{l+s+s1}{\PYGZsq{}}\PYG{l+s+s1}{ }\PYG{l+s+s1}{\PYGZsq{}} \PYG{n}{Default}\PYG{o}{=}\PYG{l+s+s1}{\PYGZsq{}}\PYG{l+s+s1}{ }\PYG{l+s+s1}{\PYGZsq{}}
\end{sphinxVerbatim}

\sphinxAtStartPar
The name of the company as it is specified in the installation key at the time of the installation. This code is unique for
each client and functions in relation to the following parameters: ADDR1, ADDR2, LICENSE, EXPIRE and CODE.

\index{Virtel TCT@\spxentry{Virtel TCT}!COMPR3 parameter@\spxentry{COMPR3 parameter}}\index{COMPR3 parameter@\spxentry{COMPR3 parameter}!Virtel TCT@\spxentry{Virtel TCT}}\ignorespaces 

\subsection{COMPR3 parameter}
\label{\detokenize{Installation_Guide:compr3-parameter}}\label{\detokenize{Installation_Guide:index-46}}
\begin{sphinxVerbatim}[commandchars=\\\{\}]
\PYG{n}{COMPR3}\PYG{o}{=}\PYG{n}{NO}\PYG{o}{/}\PYG{n}{AUTO}\PYG{o}{/}\PYG{n}{FIXED} \PYG{n}{Default}\PYG{o}{=}\PYG{n}{NO}
\end{sphinxVerbatim}

\sphinxAtStartPar
\sphinxstylestrong{NO} \sphinxhyphen{} Level 3 compression for PC’s will not be used.

\sphinxAtStartPar
\sphinxstylestrong{AUTO} \sphinxhyphen{} Level 3 compression for PC’s will be used. VIRTEL will run in learning mode as well as processing screen types.

\sphinxAtStartPar
\sphinxstylestrong{FIXED} \sphinxhyphen{} Level 3 compression for PC’s will be used. VIRTEL will only run processing screen types.

\index{Virtel TCT@\spxentry{Virtel TCT}!CORRECT parameter@\spxentry{CORRECT parameter}}\index{CORRECT parameter@\spxentry{CORRECT parameter}!Virtel TCT@\spxentry{Virtel TCT}}\ignorespaces 

\subsection{CORRECT parameter}
\label{\detokenize{Installation_Guide:correct-parameter}}\label{\detokenize{Installation_Guide:index-47}}
\begin{sphinxVerbatim}[commandchars=\\\{\}]
\PYG{n}{CORRECT}\PYG{o}{=}\PYG{n}{xx} \PYG{n}{Default}\PYG{o}{=}\PYG{l+m+mi}{7}\PYG{n}{C} \PYG{p}{(}\PYG{n}{PF12}\PYG{p}{)}
\end{sphinxVerbatim}

\sphinxAtStartPar
\sphinxstylestrong{xx} \sphinxhyphen{} The 3270 AID function key which will be transmitted to the application when the user presses the {[}CORRECTION{]} key in a blank field.

\sphinxAtStartPar
\sphinxstylestrong{CORRECT=00} \sphinxhyphen{} Places the cursor at the start of the field without sending anything to the application.

\index{Virtel TCT@\spxentry{Virtel TCT}!COUNTRY parameter@\spxentry{COUNTRY parameter}}\index{COUNTRY parameter@\spxentry{COUNTRY parameter}!Virtel TCT@\spxentry{Virtel TCT}}\ignorespaces 

\subsection{COUNTRY parameter}
\label{\detokenize{Installation_Guide:country-parameter}}\label{\detokenize{Installation_Guide:index-48}}
\begin{sphinxVerbatim}[commandchars=\\\{\}]
\PYG{n}{COUNTRY}\PYG{o}{=}\PYG{n}{xxx} \PYG{n}{Default}\PYG{o}{=}\PYG{n}{FR}
\end{sphinxVerbatim}

\sphinxAtStartPar
\sphinxstylestrong{xxx} \sphinxhyphen{} Country name indicating which translation table is to be used for translation between EBCDIC and ASCII when UTF\sphinxhyphen{}8 is not specified.

\sphinxAtStartPar
Possible values are:


\begin{savenotes}\sphinxattablestart
\sphinxthistablewithglobalstyle
\centering
\begin{tabulary}{\linewidth}[t]{TTTT}
\sphinxtoprule
\sphinxstyletheadfamily 
\sphinxAtStartPar
Value
&\sphinxstyletheadfamily 
\sphinxAtStartPar
Country
&\sphinxstyletheadfamily 
\sphinxAtStartPar
EBCDIC Code Page
&\sphinxstyletheadfamily 
\sphinxAtStartPar
ASCII Code Page
\\
\sphinxmidrule
\sphinxtableatstartofbodyhook
\sphinxAtStartPar
ALBANIA
&
\sphinxAtStartPar
Albania
&
\sphinxAtStartPar
CP 500
&
\sphinxAtStartPar
8859\sphinxhyphen{}1
\\
\sphinxhline
\sphinxAtStartPar
AUSTRALIA
&
\sphinxAtStartPar
Australia
&
\sphinxAtStartPar
CP 037
&
\sphinxAtStartPar
8859\sphinxhyphen{}1
\\
\sphinxhline
\sphinxAtStartPar
BE or BELGIUM
&
\sphinxAtStartPar
Belgium
&
\sphinxAtStartPar
CP 500
&
\sphinxAtStartPar
8859\sphinxhyphen{}1
\\
\sphinxhline
\sphinxAtStartPar
BRAZIL
&
\sphinxAtStartPar
Brazil
&
\sphinxAtStartPar
CP 037
&
\sphinxAtStartPar
8859\sphinxhyphen{}1
\\
\sphinxhline
\sphinxAtStartPar
CANADA
&
\sphinxAtStartPar
Canada
&
\sphinxAtStartPar
CP 500
&
\sphinxAtStartPar
8859\sphinxhyphen{}1
\\
\sphinxhline
\sphinxAtStartPar
DENMARK
&
\sphinxAtStartPar
Denmark
&
\sphinxAtStartPar
CP 277
&
\sphinxAtStartPar
8859\sphinxhyphen{}1
\\
\sphinxhline
\sphinxAtStartPar
DE or GERMANY
&
\sphinxAtStartPar
Germany
&
\sphinxAtStartPar
CP 273
&
\sphinxAtStartPar
8859\sphinxhyphen{}1
\\
\sphinxhline
\sphinxAtStartPar
FI or FINLAND
&
\sphinxAtStartPar
Finland
&
\sphinxAtStartPar
CP 278
&
\sphinxAtStartPar
8859\sphinxhyphen{}1
\\
\sphinxhline
\sphinxAtStartPar
FR or FRANCE
&
\sphinxAtStartPar
France
&
\sphinxAtStartPar
CP 297
&
\sphinxAtStartPar
8859\sphinxhyphen{}1
\\
\sphinxhline
\sphinxAtStartPar
IS or ICELAND (IC)
&
\sphinxAtStartPar
Iceland
&
\sphinxAtStartPar
CP 871
&
\sphinxAtStartPar
8859\sphinxhyphen{}1
\\
\sphinxhline
\sphinxAtStartPar
IRELAND
&
\sphinxAtStartPar
Ireland
&
\sphinxAtStartPar
CP 285
&
\sphinxAtStartPar
8859\sphinxhyphen{}1
\\
\sphinxhline
\sphinxAtStartPar
IT or ITALY
&
\sphinxAtStartPar
Italy
&
\sphinxAtStartPar
CP 280
&
\sphinxAtStartPar
8859\sphinxhyphen{}1
\\
\sphinxhline
\sphinxAtStartPar
L2 or ISO\sphinxhyphen{}LATIN\sphinxhyphen{}2
&
\sphinxAtStartPar
Eastern Europe
&
\sphinxAtStartPar
CP 870
&
\sphinxAtStartPar
8859\sphinxhyphen{}2
\\
\sphinxhline
\sphinxAtStartPar
NETHERLAND
&
\sphinxAtStartPar
The Netherlands
&
\sphinxAtStartPar
CP 037
&
\sphinxAtStartPar
8859\sphinxhyphen{}1
\\
\sphinxhline
\sphinxAtStartPar
NO or NORWAY
&
\sphinxAtStartPar
Norway
&
\sphinxAtStartPar
CP 277
&
\sphinxAtStartPar
8859\sphinxhyphen{}1
\\
\sphinxhline
\sphinxAtStartPar
PORTUGAL
&
\sphinxAtStartPar
Portugal
&
\sphinxAtStartPar
CP 037
&
\sphinxAtStartPar
8859\sphinxhyphen{}1
\\
\sphinxhline
\sphinxAtStartPar
P2 or PC\sphinxhyphen{}LATIN\sphinxhyphen{}2
&
\sphinxAtStartPar
Eastern Europe
&
\sphinxAtStartPar
CP 870
&
\sphinxAtStartPar
CP 852
\\
\sphinxhline
\sphinxAtStartPar
ES or SPAIN (SP)
&
\sphinxAtStartPar
Spain
&
\sphinxAtStartPar
CP 284
&
\sphinxAtStartPar
8859\sphinxhyphen{}1
\\
\sphinxhline
\sphinxAtStartPar
SWEDEN
&
\sphinxAtStartPar
Sweden
&
\sphinxAtStartPar
CP 278
&
\sphinxAtStartPar
8859\sphinxhyphen{}1
\\
\sphinxhline
\sphinxAtStartPar
SWITZERLAND
&
\sphinxAtStartPar
Switzerland
&
\sphinxAtStartPar
CP 500
&
\sphinxAtStartPar
8859\sphinxhyphen{}1
\\
\sphinxhline
\sphinxAtStartPar
GB (UK)
&
\sphinxAtStartPar
United Kingdom
&
\sphinxAtStartPar
CP 285
&
\sphinxAtStartPar
8859\sphinxhyphen{}1
\\
\sphinxhline
\sphinxAtStartPar
US or USA
&
\sphinxAtStartPar
United States
&
\sphinxAtStartPar
CP 037
&
\sphinxAtStartPar
8859\sphinxhyphen{}1
\\
\sphinxbottomrule
\end{tabulary}
\sphinxtableafterendhook\par
\sphinxattableend\end{savenotes}

\begin{sphinxadmonition}{note}{Note:}
\sphinxAtStartPar
The values shown in parentheses in the table above are accepted for compatibility with previous versions of VIRTEL.
\end{sphinxadmonition}

\sphinxAtStartPar
The COUNTRY parameter is not used when displaying web pages which contain a \{\{\{SET\sphinxhyphen{}OUTPUT\sphinxhyphen{}ENCODING\sphinxhyphen{}UTF\sphinxhyphen{}8\}\}\} tag. In this case VIRTEL uses an EBCDIC\sphinxhyphen{}to\sphinxhyphen{}UTF\sphinxhyphen{}8 translate table determined by the “DEFUTF8 parameter”, page 0 or specified in the tag itself.

\index{Virtel TCT@\spxentry{Virtel TCT}!CRYPT1 parameter@\spxentry{CRYPT1 parameter}}\index{CRYPT1 parameter@\spxentry{CRYPT1 parameter}!Virtel TCT@\spxentry{Virtel TCT}}\ignorespaces 

\subsection{CRYPT1 parameter}
\label{\detokenize{Installation_Guide:crypt1-parameter}}\label{\detokenize{Installation_Guide:index-49}}
\begin{sphinxVerbatim}[commandchars=\\\{\}]
\PYG{n}{CRYPT1}\PYG{o}{=}\PYG{p}{(}\PYG{n}{name1}\PYG{p}{,}\PYG{p}{[}\PYG{n}{algs}\PYG{p}{]}\PYG{p}{,}\PYG{p}{[}\PYG{n}{algp}\PYG{p}{]}\PYG{p}{,}\PYG{p}{[}\PYG{n}{engine}\PYG{p}{]}\PYG{p}{,}\PYG{p}{[}\PYG{n}{encoding}\PYG{p}{]}\PYG{p}{,}\PYG{p}{[}\PYG{n}{chaining}\PYG{p}{]}\PYG{p}{,}\PYG{p}{[}\PYG{n}{padding}\PYG{p}{]}\PYG{p}{)}
\end{sphinxVerbatim}

\sphinxAtStartPar
\sphinxstylestrong{Default} = none

\sphinxAtStartPar
This parameter defines the characteristics of the encryption performed by VIRTEL for page templates which specify the cryptographic identifier name1.

\sphinxAtStartPar
\sphinxstylestrong{name1} \sphinxhyphen{} A name which serves to identify this set of encryption parameters. This name will be referenced in the PUBLIC\sphinxhyphen{}KEY and ENCRYPTION\sphinxhyphen{}PARAMETERS tags in the HTML page template which uses encrypted fields.

\sphinxAtStartPar
\sphinxstylestrong{algs} \sphinxhyphen{} The symmetric encryption algorithm to be used by VIRTEL for data encryption. The following values can be specified:

\sphinxAtStartPar
\sphinxstylestrong{NONE} \sphinxhyphen{} (default value) No encryption

\sphinxAtStartPar
\sphinxstylestrong{DES} \sphinxhyphen{} Data Encryption Standard (8 byte key)

\sphinxAtStartPar
\sphinxstylestrong{2TDEA} \sphinxhyphen{} Triple Data Encryption Algorithm, keying option 2 (16 byte key)

\sphinxAtStartPar
\sphinxstylestrong{3TDEA} \sphinxhyphen{} Triple Data Encryption Algorithm, keying option 3 (24 byte key)

\sphinxAtStartPar
\sphinxstylestrong{AES\sphinxhyphen{}128} \sphinxhyphen{} Advanced Encryption Standard, key size 128 bits (16 byte key)

\sphinxAtStartPar
\sphinxstylestrong{AES\sphinxhyphen{}192} \sphinxhyphen{} Advanced Encryption Standard, key size 192 bits (24 byte key)

\sphinxAtStartPar
\sphinxstylestrong{AES\sphinxhyphen{}256} \sphinxhyphen{} Advanced Encryption Standard, key size 256 bits (32 byte key)

\begin{sphinxadmonition}{note}{Note:}
\sphinxAtStartPar
In this version of VIRTEL, only NONE, DES, 2TDEA, and 3TDEA are supported
\end{sphinxadmonition}

\sphinxAtStartPar
\sphinxstylestrong{algp} \sphinxhyphen{} The asymmetric encryption algorithm to be used by VIRTEL for encryption of session keys. The following values can be specified:

\sphinxAtStartPar
\sphinxstylestrong{NONE} \sphinxhyphen{} (default value) No encryption

\sphinxAtStartPar
\sphinxstylestrong{RSA\sphinxhyphen{}512} \sphinxhyphen{} RSA public key encryption (512 bit key)

\sphinxAtStartPar
\sphinxstylestrong{RSA\sphinxhyphen{}1024} \sphinxhyphen{} RSA public key encryption (1024 bit key)

\sphinxAtStartPar
\sphinxstylestrong{RSA\sphinxhyphen{}2048} \sphinxhyphen{} RSA public key encryption (2048 bit key)

\sphinxAtStartPar
\sphinxstylestrong{RSA\sphinxhyphen{}4096} \sphinxhyphen{} RSA public key encryption (4096 bit key)

\begin{sphinxadmonition}{note}{Note:}
\sphinxAtStartPar
In this version of VIRTEL, only NONE, RSA\sphinxhyphen{}512, and RSA\sphinxhyphen{}1024 are supported.
\end{sphinxadmonition}

\sphinxAtStartPar
\sphinxstylestrong{engine} \sphinxhyphen{} The name of the encryption engine to be used. The following values can be specified:

\sphinxAtStartPar
\sphinxstylestrong{ICSF} \sphinxhyphen{} VIRTEL uses the Integrated Cryptographic Service Facility of z/OS

\sphinxAtStartPar
\sphinxstylestrong{NO\sphinxhyphen{}ENCRYPTION} \sphinxhyphen{} (default value) VIRTEL uses an internal null\sphinxhyphen{}encryption engine. In this case, NONE must be specified or defaulted for the cryptographic algorithms.

\sphinxAtStartPar
\sphinxstylestrong{encoding} \sphinxhyphen{} The representation which VIRTEL will use for encrypted text. The following values can be specified:

\sphinxAtStartPar
\sphinxstylestrong{HEX} \sphinxhyphen{} (default value) Encrypted data is represented in hexadecimal

\sphinxAtStartPar
\sphinxstylestrong{BASE64} \sphinxhyphen{} Encrypted data is represented in base64 format

\begin{sphinxadmonition}{note}{Note:}
\sphinxAtStartPar
In this version of VIRTEL, only HEX is supported
\end{sphinxadmonition}

\sphinxAtStartPar
\sphinxstylestrong{chaining} \sphinxhyphen{} The chaining method to be used for symmetric encryption. The following values can be specified:

\sphinxAtStartPar
\sphinxstylestrong{CBC} \sphinxhyphen{} (default value) Cipher block chaining will be used.

\sphinxAtStartPar
\sphinxstylestrong{ECB} \sphinxhyphen{} Electronic codebook will be used

\begin{sphinxadmonition}{note}{Note:}
\sphinxAtStartPar
In this version of VIRTEL, only CBC is supported
\end{sphinxadmonition}

\sphinxAtStartPar
\sphinxstylestrong{padding} \sphinxhyphen{} The padding method to be used for symmetric encryption. The following values can be specified:

\sphinxAtStartPar
\sphinxstylestrong{PKCS7} \sphinxhyphen{} (default value) Public Key Cryptographic Standard \#7 padding

\sphinxAtStartPar
\sphinxstylestrong{X9.23} \sphinxhyphen{} ANSI X9.23 padding method

\sphinxAtStartPar
\sphinxstylestrong{ISO10126} \sphinxhyphen{} Padding method using random padding bytes

\begin{sphinxadmonition}{note}{Note:}
\sphinxAtStartPar
In this version of VIRTEL, only PKCS7 is supported
\end{sphinxadmonition}

\index{Virtel TCT@\spxentry{Virtel TCT}!CRYPT2 parameter@\spxentry{CRYPT2 parameter}}\index{CRYPT2 parameter@\spxentry{CRYPT2 parameter}!Virtel TCT@\spxentry{Virtel TCT}}\ignorespaces 

\subsection{CRYPT2 parameter}
\label{\detokenize{Installation_Guide:crypt2-parameter}}\label{\detokenize{Installation_Guide:index-50}}
\begin{sphinxVerbatim}[commandchars=\\\{\}]
\PYG{n}{CRYPT2}\PYG{o}{=}\PYG{p}{(}\PYG{n}{name2}\PYG{p}{,}\PYG{p}{[}\PYG{n}{algs}\PYG{p}{]}\PYG{p}{,}\PYG{p}{[}\PYG{n}{algp}\PYG{p}{]}\PYG{p}{,}\PYG{p}{[}\PYG{n}{engine}\PYG{p}{]}\PYG{p}{,}\PYG{p}{[}\PYG{n}{encoding}\PYG{p}{]}\PYG{p}{,}\PYG{p}{[}\PYG{n}{chaining}\PYG{p}{]}\PYG{p}{,}\PYG{p}{[}\PYG{n}{padding}\PYG{p}{]}\PYG{p}{)}
\end{sphinxVerbatim}

\sphinxAtStartPar
Default=none

\sphinxAtStartPar
This parameter defines the characteristics of the encryption performed by VIRTEL for page templates which specify the cryptographic identifier name2. The subparameters are the same as those of CRYPT1.

\index{Virtel TCT@\spxentry{Virtel TCT}!DEFENTR parameter@\spxentry{DEFENTR parameter}}\index{DEFENTR parameter@\spxentry{DEFENTR parameter}!Virtel TCT@\spxentry{Virtel TCT}}\ignorespaces 

\subsection{DEFENTR parameter}
\label{\detokenize{Installation_Guide:defentr-parameter}}\label{\detokenize{Installation_Guide:index-51}}
\begin{sphinxVerbatim}[commandchars=\\\{\}]
\PYG{n}{DEFENTR}\PYG{o}{=}\PYG{p}{(}\PYG{n}{xxxxxxxx}\PYG{p}{,}\PYG{n}{yyyyyyyy}\PYG{p}{)} \PYG{n}{Default}\PYG{o}{=}\PYG{l+s+s1}{\PYGZsq{}}\PYG{l+s+s1}{ }\PYG{l+s+s1}{\PYGZsq{}}
\end{sphinxVerbatim}

\sphinxAtStartPar
\sphinxstylestrong{xxxxxxxx} \sphinxhyphen{} The name of the entry point taken by default at connection time by a 3270 terminal. This parameter may for example be used for 3270 connections functioning in ACCUEIL mode.

\sphinxAtStartPar
\sphinxstylestrong{yyyyyyyy} \sphinxhyphen{} The name of the default entry point for X25 asynchronous connections.

\index{Virtel TCT@\spxentry{Virtel TCT}!DEFUTF8 parameter@\spxentry{DEFUTF8 parameter}}\index{DEFUTF8 parameter@\spxentry{DEFUTF8 parameter}!Virtel TCT@\spxentry{Virtel TCT}}\ignorespaces 

\subsection{DEFUTF8 parameter}
\label{\detokenize{Installation_Guide:defutf8-parameter}}\label{\detokenize{Installation_Guide:index-52}}
\begin{sphinxVerbatim}[commandchars=\\\{\}]
\PYG{n}{DEFUTF8}\PYG{o}{=}\PYG{n}{xxxxxxxx} \PYG{n}{Default}\PYG{o}{=}\PYG{n}{IBM1147}
\end{sphinxVerbatim}

\sphinxAtStartPar
\sphinxstylestrong{xxxxxxx} \sphinxhyphen{} Name of the default character set for EBCDIC to UTF\sphinxhyphen{}8 translation. This character set is used when an HTML or XML page contains a SET\sphinxhyphen{}OUTPUT\sphinxhyphen{}ENCODING\sphinxhyphen{}UTF\sphinxhyphen{}8 tag without a character set name. Any one of the following values may be specified:

\sphinxAtStartPar
Table:


\begin{savenotes}\sphinxattablestart
\sphinxthistablewithglobalstyle
\centering
\begin{tabulary}{\linewidth}[t]{TT}
\sphinxtoprule
\sphinxstyletheadfamily 
\sphinxAtStartPar
Character set
&\sphinxstyletheadfamily 
\sphinxAtStartPar
Description
\\
\sphinxmidrule
\sphinxtableatstartofbodyhook
\sphinxAtStartPar
IBM0037
&
\sphinxAtStartPar
US EBCDIC (without Euro sign)
\\
\sphinxhline
\sphinxAtStartPar
IBM1047
&
\sphinxAtStartPar
Latin\sphinxhyphen{}1 Open Systems EBCDIC
\\
\sphinxhline
\sphinxAtStartPar
IBM1140
&
\sphinxAtStartPar
ECECP USA, Canada, Netherlands, Portugal, Brazil, Australia, New Zealand
\\
\sphinxhline
\sphinxAtStartPar
IBM1141
&
\sphinxAtStartPar
ECECP Austria, Germany
\\
\sphinxhline
\sphinxAtStartPar
IBM1142
&
\sphinxAtStartPar
ECECP Denmark, Norway
\\
\sphinxhline
\sphinxAtStartPar
IBM1143
&
\sphinxAtStartPar
ECECP Finland, Sweden
\\
\sphinxhline
\sphinxAtStartPar
IBM1144
&
\sphinxAtStartPar
ECECP Italy
\\
\sphinxhline
\sphinxAtStartPar
IBM1145
&
\sphinxAtStartPar
ECECP Spain, Latin America (Spanish)
\\
\sphinxhline
\sphinxAtStartPar
IBM1146
&
\sphinxAtStartPar
ECECP United Kingdom
\\
\sphinxhline
\sphinxAtStartPar
IBM1147
&
\sphinxAtStartPar
ECECP France UCS\sphinxhyphen{}2
\\
\sphinxhline
\sphinxAtStartPar
IBM1148
&
\sphinxAtStartPar
ECECP International 1
\\
\sphinxhline
\sphinxAtStartPar
IBM1149
&
\sphinxAtStartPar
ECECP Iceland
\\
\sphinxhline
\sphinxAtStartPar
IBM1153
&
\sphinxAtStartPar
Latin\sphinxhyphen{}2 \sphinxhyphen{} EBCDIC multilingual with euro
\\
\sphinxhline
\sphinxAtStartPar
IBM1154
&
\sphinxAtStartPar
Cyrillic multilingual with euro
\\
\sphinxhline
\sphinxAtStartPar
IBM1155
&
\sphinxAtStartPar
Turkey Latin 5 with euro
\\
\sphinxhline
\sphinxAtStartPar
IBM1156
&
\sphinxAtStartPar
Baltic multilingual with euro
\\
\sphinxhline
\sphinxAtStartPar
IBM1157
&
\sphinxAtStartPar
Estonia EBCDIC with euro
\\
\sphinxhline
\sphinxAtStartPar
IBM1158
&
\sphinxAtStartPar
Cyrillic Ukraine EBCDIC with euro
\\
\sphinxhline
\sphinxAtStartPar
IBM1159
&
\sphinxAtStartPar
T\sphinxhyphen{}Chinese host extended SBCS with euro
\\
\sphinxhline
\sphinxAtStartPar
IBM1160
&
\sphinxAtStartPar
IBM1160
\\
\sphinxhline
\sphinxAtStartPar
IBM1164
&
\sphinxAtStartPar
EBCDIC Vietnamese with euro
\\
\sphinxhline
\sphinxAtStartPar
IBM4971
&
\sphinxAtStartPar
Greek (including euro)
\\
\sphinxhline
\sphinxAtStartPar
IBM5123
&
\sphinxAtStartPar
Japanese Latin host extended SBCS (includes euro)
\\
\sphinxhline
\sphinxAtStartPar
IBM12712
&
\sphinxAtStartPar
Hebrew (max set including euro and new sheqel)
\\
\sphinxhline
\sphinxAtStartPar
IBM16804
&
\sphinxAtStartPar
Arabic (all presentation shapes) with euro
\\
\sphinxhline
\sphinxAtStartPar
IBM1137
&
\sphinxAtStartPar
Devanagari (Hindi) EBCDIC (based on Unicode character set)
\\
\sphinxbottomrule
\end{tabulary}
\sphinxtableafterendhook\par
\sphinxattableend\end{savenotes}

\sphinxAtStartPar
The values listed above are the names of the standard tables which are always available in VIRTEL. Additional tables
may be loaded at startup time by means of the “CHARSET parameter”.

\index{Virtel TCT@\spxentry{Virtel TCT}!DIRECT parameter@\spxentry{DIRECT parameter}}\index{DIRECT parameter@\spxentry{DIRECT parameter}!Virtel TCT@\spxentry{Virtel TCT}}\ignorespaces 

\subsection{DIRECT parameter}
\label{\detokenize{Installation_Guide:direct-parameter}}\label{\detokenize{Installation_Guide:index-53}}
\begin{sphinxVerbatim}[commandchars=\\\{\}]
\PYG{n}{DIRECT}\PYG{o}{=}\PYG{n}{xx} \PYG{n}{Default}\PYG{o}{=}\PYG{l+m+mi}{1}\PYG{n}{C} \PYG{p}{(}\PYG{n}{REPRO}\PYG{p}{)}
\end{sphinxVerbatim}

\sphinxAtStartPar
\sphinxstylestrong{xx} \sphinxhyphen{} Hex code of the character of the 3270 keyboard that will be used to switch directly from one session to another. If DIRECT=00 then this function will be disabled.

\index{Virtel TCT@\spxentry{Virtel TCT}!DONTSWA parameter@\spxentry{DONTSWA parameter}}\index{DONTSWA parameter@\spxentry{DONTSWA parameter}!Virtel TCT@\spxentry{Virtel TCT}}\ignorespaces 

\subsection{DONTSWA parameter (z/OS only)}
\label{\detokenize{Installation_Guide:dontswa-parameter-z-os-only}}\label{\detokenize{Installation_Guide:index-54}}
\begin{sphinxVerbatim}[commandchars=\\\{\}]
\PYG{n}{DONTSWA}\PYG{o}{=}\PYG{n}{YES}\PYG{o}{/}\PYG{n}{NO} \PYG{n}{Default}\PYG{o}{=}\PYG{n}{NO}
\end{sphinxVerbatim}

\sphinxAtStartPar
\sphinxstylestrong{YES} \sphinxhyphen{} VIRTEL will attempt to set itself non\sphinxhyphen{}swappable. This option is only available if VIRTEL is run from an APF\sphinxhyphen{}authorized library.
\sphinxstylestrong{NO} \sphinxhyphen{} VIRTEL remains swappable

\begin{sphinxadmonition}{note}{Note:}
\sphinxAtStartPar
When VIRTEL is executed via program VIR6000, it is always non\sphinxhyphen{}swappable
\end{sphinxadmonition}

\index{Virtel TCT@\spxentry{Virtel TCT}!EXIT1 parameter@\spxentry{EXIT1 parameter}}\index{EXIT1 parameter@\spxentry{EXIT1 parameter}!Virtel TCT@\spxentry{Virtel TCT}}\ignorespaces 

\subsection{EXIT1 parameter}
\label{\detokenize{Installation_Guide:exit1-parameter}}\label{\detokenize{Installation_Guide:index-55}}
\begin{sphinxVerbatim}[commandchars=\\\{\}]
\PYG{n}{EXIT1}\PYG{o}{=}\PYG{n}{xx} \PYG{n}{Default}\PYG{o}{=}\PYG{l+s+s1}{\PYGZsq{}}\PYG{l+s+s1}{ }\PYG{l+s+s1}{\PYGZsq{}}
\end{sphinxVerbatim}

\sphinxAtStartPar
\sphinxstylestrong{xx} \sphinxhyphen{} Is the name of the VIREXxx module that will be called to process an incoming call packet. This exit will only function for lines running in GATE mode.

\index{Virtel TCT@\spxentry{Virtel TCT}!EXIT2 parameter@\spxentry{EXIT2 parameter}}\index{EXIT2 parameter@\spxentry{EXIT2 parameter}!Virtel TCT@\spxentry{Virtel TCT}}\ignorespaces 

\subsection{EXIT2 parameter}
\label{\detokenize{Installation_Guide:exit2-parameter}}\label{\detokenize{Installation_Guide:index-56}}
\begin{sphinxVerbatim}[commandchars=\\\{\}]
\PYG{n}{EXIT2}\PYG{o}{=}\PYG{n}{xx} \PYG{n}{Default}\PYG{o}{=}\PYG{l+s+s1}{\PYGZsq{}}\PYG{l+s+s1}{ }\PYG{l+s+s1}{\PYGZsq{}}
\end{sphinxVerbatim}

\sphinxAtStartPar
\sphinxstylestrong{xx} \sphinxhyphen{} Is the name of the VIREXxx module that will be called when a sub\sphinxhyphen{}server node connects. If the line used is set to GATE mode this exit will process call packet CUD.

\index{Virtel TCT@\spxentry{Virtel TCT}!EXIT3 parameter@\spxentry{EXIT3 parameter}}\index{EXIT3 parameter@\spxentry{EXIT3 parameter}!Virtel TCT@\spxentry{Virtel TCT}}\ignorespaces 

\subsection{EXIT3 parameter}
\label{\detokenize{Installation_Guide:exit3-parameter}}\label{\detokenize{Installation_Guide:index-57}}
\begin{sphinxVerbatim}[commandchars=\\\{\}]
\PYG{n}{EXIT3}\PYG{o}{=}\PYG{n}{xx} \PYG{n}{Default}\PYG{o}{=}\PYG{l+s+s1}{\PYGZsq{}}\PYG{l+s+s1}{ }\PYG{l+s+s1}{\PYGZsq{}}
\end{sphinxVerbatim}

\sphinxAtStartPar
\sphinxstylestrong{xx} \sphinxhyphen{} Is the name of the VIREXxx module that will be called at connection time to a VTAM application from a multi\sphinxhyphen{}session screen.

\index{Virtel TCT@\spxentry{Virtel TCT}!EXIT4 parameter@\spxentry{EXIT4 parameter}}\index{EXIT4 parameter@\spxentry{EXIT4 parameter}!Virtel TCT@\spxentry{Virtel TCT}}\ignorespaces 

\subsection{EXIT4 parameter}
\label{\detokenize{Installation_Guide:exit4-parameter}}\label{\detokenize{Installation_Guide:index-58}}
\begin{sphinxVerbatim}[commandchars=\\\{\}]
\PYG{n}{EXIT4}\PYG{o}{=}\PYG{n}{xx} \PYG{n}{Default}\PYG{o}{=}\PYG{l+s+s1}{\PYGZsq{}}\PYG{l+s+s1}{ }\PYG{l+s+s1}{\PYGZsq{}}
\end{sphinxVerbatim}

\sphinxAtStartPar
\sphinxstylestrong{xx} \sphinxhyphen{} Is the name of the VIREXxx module that will be used to filter messages when a VTAM application is accessed either from the multi\sphinxhyphen{}session screen or from a sub\sphinxhyphen{}server node.

\index{Virtel TCT@\spxentry{Virtel TCT}!EXIT5 parameter@\spxentry{EXIT5 parameter}}\index{EXIT5 parameter@\spxentry{EXIT5 parameter}!Virtel TCT@\spxentry{Virtel TCT}}\ignorespaces 

\subsection{EXIT5 parameter}
\label{\detokenize{Installation_Guide:exit5-parameter}}\label{\detokenize{Installation_Guide:index-59}}
\begin{sphinxVerbatim}[commandchars=\\\{\}]
\PYG{n}{EXIT5}\PYG{o}{=}\PYG{n}{xx} \PYG{n}{Default}\PYG{o}{=}\PYG{l+s+s1}{\PYGZsq{}}\PYG{l+s+s1}{ }\PYG{l+s+s1}{\PYGZsq{}}
\end{sphinxVerbatim}

\sphinxAtStartPar
\sphinxstylestrong{xx} \sphinxhyphen{} Is the name of the VIREXxx module that will be called to process outgoing call packets.

\index{Virtel TCT@\spxentry{Virtel TCT}!EXIT6 parameter@\spxentry{EXIT6 parameter}}\index{EXIT6 parameter@\spxentry{EXIT6 parameter}!Virtel TCT@\spxentry{Virtel TCT}}\ignorespaces 

\subsection{EXIT6 parameter}
\label{\detokenize{Installation_Guide:exit6-parameter}}\label{\detokenize{Installation_Guide:index-60}}
\begin{sphinxVerbatim}[commandchars=\\\{\}]
\PYG{n}{EXIT6}\PYG{o}{=}\PYG{n}{xx} \PYG{n}{Default}\PYG{o}{=}\PYG{l+s+s1}{\PYGZsq{}}\PYG{l+s+s1}{ }\PYG{l+s+s1}{\PYGZsq{}}
\end{sphinxVerbatim}

\sphinxAtStartPar
\sphinxstylestrong{xx} \sphinxhyphen{} Is the name of the VIREXxx module that will be called to process messages bound for host applications.

\index{Virtel TCT@\spxentry{Virtel TCT}!EXIT7 parameter@\spxentry{EXIT7 parameter}}\index{EXIT7 parameter@\spxentry{EXIT7 parameter}!Virtel TCT@\spxentry{Virtel TCT}}\ignorespaces 

\subsection{EXIT7 parameter}
\label{\detokenize{Installation_Guide:exit7-parameter}}\label{\detokenize{Installation_Guide:index-61}}
\begin{sphinxVerbatim}[commandchars=\\\{\}]
\PYG{n}{EXIT7}\PYG{o}{=}\PYG{n}{xx} \PYG{n}{Default}\PYG{o}{=}\PYG{l+s+s1}{\PYGZsq{}}\PYG{l+s+s1}{ }\PYG{l+s+s1}{\PYGZsq{}}
\end{sphinxVerbatim}

\sphinxAtStartPar
\sphinxstylestrong{xx} \sphinxhyphen{} Is the name of the VIREXxx module that will be used to calculate the connection costs for external server calls.

\index{Virtel TCT@\spxentry{Virtel TCT}!EXIT8 parameter@\spxentry{EXIT8 parameter}}\index{EXIT8 parameter@\spxentry{EXIT8 parameter}!Virtel TCT@\spxentry{Virtel TCT}}\ignorespaces 

\subsection{EXIT8 parameter}
\label{\detokenize{Installation_Guide:exit8-parameter}}\label{\detokenize{Installation_Guide:index-62}}
\begin{sphinxVerbatim}[commandchars=\\\{\}]
\PYG{n}{EXIT8}\PYG{o}{=}\PYG{n}{xx} \PYG{n}{Default}\PYG{o}{=}\PYG{l+s+s1}{\PYGZsq{}}\PYG{l+s+s1}{ }\PYG{l+s+s1}{\PYGZsq{}}
\end{sphinxVerbatim}

\sphinxAtStartPar
\sphinxstylestrong{xx} \sphinxhyphen{} Is the name of the VIREXxx module that will be used to process the incoming call connection packet for the HTTP server.

\index{Virtel TCT@\spxentry{Virtel TCT}!EXPIRE parameter@\spxentry{EXPIRE parameter}}\index{EXPIRE parameter@\spxentry{EXPIRE parameter}!Virtel TCT@\spxentry{Virtel TCT}}\ignorespaces 

\subsection{EXPIRE parameter}
\label{\detokenize{Installation_Guide:expire-parameter}}\label{\detokenize{Installation_Guide:index-63}}
\begin{sphinxVerbatim}[commandchars=\\\{\}]
\PYG{n}{EXPIRE}\PYG{o}{=}\PYG{p}{(}\PYG{n}{YYYY}\PYG{p}{,}\PYG{n}{MM}\PYG{p}{,}\PYG{n}{JJ}\PYG{p}{)} \PYG{n}{Default}\PYG{o}{=}\PYG{p}{(}\PYG{l+m+mi}{2999}\PYG{p}{,}\PYG{l+m+mi}{12}\PYG{p}{,}\PYG{l+m+mi}{31}\PYG{p}{)}
\end{sphinxVerbatim}

\sphinxAtStartPar
\sphinxstylestrong{(YYYY,MM,JJ)} \sphinxhyphen{} Is the expiry date of the contract specified in the key at installation time. This parameter is unique for each client and functions in relation with the following parameters: ADDR1, ADDR2, COMPANY, LICENSE and CODE.

\index{Virtel TCT@\spxentry{Virtel TCT}!FASTC parameter@\spxentry{FASTC parameter}}\index{FASTC parameter@\spxentry{FASTC parameter}!Virtel TCT@\spxentry{Virtel TCT}}\ignorespaces 

\subsection{FASTC parameter}
\label{\detokenize{Installation_Guide:fastc-parameter}}\label{\detokenize{Installation_Guide:index-64}}
\begin{sphinxVerbatim}[commandchars=\\\{\}]
\PYG{n}{FASTC}\PYG{o}{=}\PYG{n}{YES}\PYG{o}{/}\PYG{n}{NO} \PYG{n}{Default}\PYG{o}{=}\PYG{n}{NO}
\end{sphinxVerbatim}

\sphinxAtStartPar
This parameter specifies whether VIRTEL will use the Fast Connect mode of NPSI for X25 communications.

\sphinxAtStartPar
\sphinxstylestrong{YES} \sphinxhyphen{} Indicates that Fast Connect mode will be used
\sphinxstylestrong{NO} \sphinxhyphen{} Indicates that Fast Connect mode will not be used.

\index{Virtel TCT@\spxentry{Virtel TCT}!FCAPT parameter@\spxentry{FCAPT parameter}}\index{FCAPT parameter@\spxentry{FCAPT parameter}!Virtel TCT@\spxentry{Virtel TCT}}\ignorespaces 

\subsection{FCAPT parameter}
\label{\detokenize{Installation_Guide:fcapt-parameter}}\label{\detokenize{Installation_Guide:index-65}}
\begin{sphinxVerbatim}[commandchars=\\\{\}]
\PYG{n}{FCAPT}\PYG{o}{=}\PYG{n}{xxxxxxx} \PYG{n}{Default}\PYG{o}{=} \PYG{p}{(}\PYG{n}{none}\PYG{p}{)}
\end{sphinxVerbatim}

\sphinxAtStartPar
\sphinxstylestrong{xxxxxxx} \sphinxhyphen{} Is the DD name of the file used to save screen images captured during an external server call. To enable the screen image capture facility, specify FCAPT=VIRCAPT and include a VIRCAPT DD/DLBL statement in the VIRTEL JCL procedure. If the FCAPT parameter is omitted, the screen image capture facility is disabled.

\index{Virtel TCT@\spxentry{Virtel TCT}!FCMP3 parameter@\spxentry{FCMP3 parameter}}\index{FCMP3 parameter@\spxentry{FCMP3 parameter}!Virtel TCT@\spxentry{Virtel TCT}}\ignorespaces 

\subsection{FCMP3 parameter}
\label{\detokenize{Installation_Guide:fcmp3-parameter}}\label{\detokenize{Installation_Guide:index-66}}
\begin{sphinxVerbatim}[commandchars=\\\{\}]
\PYG{n}{FCMP3}\PYG{o}{=}\PYG{n}{xxxxxxx} \PYG{n}{Default}\PYG{o}{=}\PYG{n}{VIRCMP3}
\end{sphinxVerbatim}

\sphinxAtStartPar
\sphinxstylestrong{xxxxxxx} \sphinxhyphen{} Indicates the DD name of the file containing the screen types used in level 3 compression. To enable the level 3 compression facility, specify FCMP3=VIRCMP3 and include a VIRCMP3 DD/DLBL statement in the VIRTEL JCL procedure. The COMPR3 parameter specifies the type of compression. If COMPR3=NO is specified then the FCMP3 parameter is ignored and the VIRCMP3 file is not required.

\index{Virtel TCT@\spxentry{Virtel TCT}!GATE parameter@\spxentry{GATE parameter}}\index{GATE parameter@\spxentry{GATE parameter}!Virtel TCT@\spxentry{Virtel TCT}}\ignorespaces 

\subsection{GATE parameter}
\label{\detokenize{Installation_Guide:gate-parameter}}\label{\detokenize{Installation_Guide:index-67}}
\begin{sphinxVerbatim}[commandchars=\\\{\}]
\PYG{n}{GATE}\PYG{o}{=}\PYG{n}{GENERAL}\PYG{o}{/}\PYG{n}{NO} \PYG{n}{Default}\PYG{o}{=}\PYG{n}{GENERAL}
\end{sphinxVerbatim}

\sphinxAtStartPar
\sphinxstylestrong{GENERAL} \sphinxhyphen{} Activates support for all types of terminal.
\sphinxstylestrong{NO} \sphinxhyphen{} Activates support for incoming calls only.

\index{Virtel TCT@\spxentry{Virtel TCT}!GMT parameter@\spxentry{GMT parameter}}\index{GMT parameter@\spxentry{GMT parameter}!Virtel TCT@\spxentry{Virtel TCT}}\ignorespaces 

\subsection{GMT parameter}
\label{\detokenize{Installation_Guide:gmt-parameter}}\label{\detokenize{Installation_Guide:index-68}}
\begin{sphinxVerbatim}[commandchars=\\\{\}]
\PYG{n}{GMT}\PYG{o}{=}\PYG{p}{(}\PYG{n}{x}\PYG{p}{,}\PYG{n}{y}\PYG{p}{)} \PYG{n}{Default}\PYG{o}{=}\PYG{p}{(}\PYG{l+m+mi}{0}\PYG{p}{,}\PYG{l+m+mi}{2}\PYG{p}{)}
\PYG{n}{GMT}\PYG{o}{=}\PYG{p}{(}\PYG{n}{x}\PYG{p}{,}\PYG{n}{SYSTZ}\PYG{p}{)}
\PYG{n}{GMT}\PYG{o}{=}\PYG{n}{SYSTZ}
\end{sphinxVerbatim}

\sphinxAtStartPar
This parameter indicates the timezone adjustments which VIRTEL must take into account in order to generate the correct standard conformant timestamps in SMTP and HTTP headers. This parameter is also used to generate
timestamps in local time for the VIRLOG and VIRSTAT files.

\sphinxAtStartPar
\sphinxstylestrong{x} \sphinxhyphen{} The first subparameter is the number of hours which must be added to the system TOD clock value to arrive at GMT. Negative values indicate that the TOD clock is ahead of GMT, positive values indicate that the TOD clock is behind GMT. For systems which run with TOD=GMT this subparameter is 0.

\sphinxAtStartPar
\sphinxstylestrong{y} \sphinxhyphen{} The second subparameter is the number of hours which must be added to GMT to arrive at the local time. Negative values indicate that local time is behind GMT (west), positive values indicate that local time is ahead of GMT (east).

\sphinxAtStartPar
For example, USA EASTERN DAYLIGHT SAVINGS TIME with the TOD clock set to GMT should be coded as GMT=(0,\sphinxhyphen{}4). If the TOD clock is set to CENTRAL EUROPEAN TIME (GMT+1) and the local time is EUROPEAN SUMMER TIME (GMT+2) then this parameter should be coded as GMT=(\sphinxhyphen{}1,+2). GMT=(\sphinxhyphen{}1,+1) indicates that both TOD clock and local time are CENTRAL EUROPEAN TIME.

\sphinxAtStartPar
To avoid the need to modify the GMT parameter when daylight savings time is in effect, you may specify GMT=SYSTZ or GMT=(x,SYSTZ)

\sphinxAtStartPar
\sphinxstylestrong{GMT=SYSTZ} \sphinxhyphen{} indicates that the TOD clock is set to GMT and that VIRTEL will obtain the timezone difference by inspecting the system local time offset. For z/OS the local time offset is specified in the CLOCKxx member of the system PARMLIB, which may be modified by the SET CLOCK command in the event of a transition between winter and summer time. For z/VSE the local time offset is specified by the SET ZONEDEF command in the \$IPL procedure.

\sphinxAtStartPar
\sphinxstylestrong{GMT=(x,SYSTZ)} \sphinxhyphen{} indicates that the TOD clock is set to GMT\sphinxhyphen{}x, and VIRTEL will use the system local time offset to calculate the timezone difference. In this case, x is the number of hours which must be added to the TOD clock value to arrive at GMT, and VIRTEL considers the local time to be GMT + w \textendash{} x where w is the system local time offset. GMT=SYSTZ is equivalent to GMT=(0,SYSTZ).

\index{Virtel TCT@\spxentry{Virtel TCT}!GRNAME parameter@\spxentry{GRNAME parameter}}\index{GRNAME parameter@\spxentry{GRNAME parameter}!Virtel TCT@\spxentry{Virtel TCT}}\ignorespaces 

\subsection{GRNAME parameter}
\label{\detokenize{Installation_Guide:grname-parameter}}\label{\detokenize{Installation_Guide:index-69}}
\begin{sphinxVerbatim}[commandchars=\\\{\}]
\PYG{n}{GRNAME}\PYG{o}{=}\PYG{n}{grname}            \PYG{n}{Default}\PYG{o}{=}\PYG{n}{none}
\end{sphinxVerbatim}

\sphinxAtStartPar
\sphinxstylestrong{grname} \sphinxhyphen{} The VTAM generic resource name for VIRTEL. If GRNAME is specified, VIRTEL will identify itself to VTAM using the specified generic resource name. The VTAM generic resources function allows the assignment of a generic resource name to a group of application programs that all provide the same function. VTAM automatically distributes sessions among these application programs rather than assigning all sessions to a single resource.

\begin{sphinxadmonition}{note}{Note:}
\sphinxAtStartPar
Use of generic resources requires a coupling facility structure.
\end{sphinxadmonition}

\index{Virtel TCT@\spxentry{Virtel TCT}!GTLOAD parameter@\spxentry{GTLOAD parameter}}\index{GTLOAD parameter@\spxentry{GTLOAD parameter}!Virtel TCT@\spxentry{Virtel TCT}}\ignorespaces 

\subsection{GTLOAD parameter}
\label{\detokenize{Installation_Guide:gtload-parameter}}\label{\detokenize{Installation_Guide:index-70}}
\begin{sphinxVerbatim}[commandchars=\\\{\}]
\PYG{n}{GTLOAD}\PYG{o}{=}\PYG{n}{nn}                \PYG{n}{Default}\PYG{o}{=}\PYG{l+m+mi}{0}
\end{sphinxVerbatim}

\sphinxAtStartPar
\sphinxstylestrong{nn} \sphinxhyphen{} Indicates the number of GTM map load modules.

\index{Virtel TCT@\spxentry{Virtel TCT}!GTPREF1 parameter@\spxentry{GTPREF1 parameter}}\index{GTPREF1 parameter@\spxentry{GTPREF1 parameter}!Virtel TCT@\spxentry{Virtel TCT}}\ignorespaces 

\subsection{GTPRFE1 parameter}
\label{\detokenize{Installation_Guide:gtprfe1-parameter}}\label{\detokenize{Installation_Guide:index-71}}
\begin{sphinxVerbatim}[commandchars=\\\{\}]
\PYG{n}{GTPRFE1}\PYG{o}{=}\PYG{p}{(}\PYG{n}{x1}\PYG{p}{,}\PYG{n}{x2}\PYG{p}{,}\PYG{o}{.}\PYG{o}{.}\PYG{p}{,}\PYG{n}{xn}\PYG{p}{)}    \PYG{n}{Default}\PYG{o}{=}\PYG{l+s+s1}{\PYGZsq{}}\PYG{l+s+s1}{ }\PYG{l+s+s1}{\PYGZsq{}}
\end{sphinxVerbatim}

\sphinxAtStartPar
\sphinxstylestrong{xn} \sphinxhyphen{} Indicates the base screen codes used in the \$\%F commands of GTM. Each code references one of the ‘ym’ prefixes defined in the GTPRFE2 parameter. The number of codes defined in GTPRFE1 may not exceed the number of prefixes defined in the GTPRFE2 parameter.

\index{Virtel TCT@\spxentry{Virtel TCT}!GTPREF2 parameter@\spxentry{GTPREF2 parameter}}\index{GTPREF2 parameter@\spxentry{GTPREF2 parameter}!Virtel TCT@\spxentry{Virtel TCT}}\ignorespaces 

\subsection{GTPRFE2 parameter}
\label{\detokenize{Installation_Guide:gtprfe2-parameter}}\label{\detokenize{Installation_Guide:index-72}}
\begin{sphinxVerbatim}[commandchars=\\\{\}]
\PYG{n}{GTPRFE2}\PYG{o}{=}\PYG{p}{(}\PYG{n}{y1}\PYG{p}{,}\PYG{n}{y2}\PYG{p}{,}\PYG{o}{.}\PYG{o}{.}\PYG{n}{ym}\PYG{p}{)}     \PYG{n}{Default}\PYG{o}{=}\PYG{l+s+s1}{\PYGZsq{}}\PYG{l+s+s1}{ }\PYG{l+s+s1}{\PYGZsq{}}
\end{sphinxVerbatim}

\sphinxAtStartPar
\sphinxstylestrong{ym} \sphinxhyphen{} Indicates base screen prefixes associated with the code ‘xn’ defined in the GTPRFE1 parameter. The number of prefixes defined in the GTPRFE2 parameter must equal the number of codes defined in GTPRFE1 + 1; the last position contains the prefix to be used if no code is specified in the \$\%F command or if the specified code does not exist.

\index{Virtel TCT@\spxentry{Virtel TCT}!GTVSAM parameter@\spxentry{GTVSAM parameter}}\index{GTVSAM parameter@\spxentry{GTVSAM parameter}!Virtel TCT@\spxentry{Virtel TCT}}\ignorespaces 

\subsection{GTVSAM parameter}
\label{\detokenize{Installation_Guide:gtvsam-parameter}}\label{\detokenize{Installation_Guide:index-73}}
\begin{sphinxVerbatim}[commandchars=\\\{\}]
\PYG{n}{GTVSAM}\PYG{o}{=}\PYG{p}{(}\PYG{n}{filename}\PYG{p}{,}\PYG{n}{keylen}\PYG{p}{,}\PYG{n}{rkp}\PYG{p}{,}\PYG{n}{acbcard}\PYG{p}{)}       \PYG{n}{Default}\PYG{o}{=}\PYG{l+s+s1}{\PYGZsq{}}\PYG{l+s+s1}{ }\PYG{l+s+s1}{\PYGZsq{}}
\end{sphinxVerbatim}

\sphinxAtStartPar
\sphinxstylestrong{filename} \sphinxhyphen{} Is the name of the VSAM file containing the GTM maps when these are contained in a VMO file.

\sphinxAtStartPar
\sphinxstylestrong{keylen} \sphinxhyphen{} length of the VSAM key

\sphinxAtStartPar
\sphinxstylestrong{rkp} \sphinxhyphen{} position relative to zero of the key in the record

\sphinxAtStartPar
\sphinxstylestrong{acbcard} \sphinxhyphen{} Name of the ACB macro referenced, if the VMO file is described by a UFILEn parameter in the VIRTCT.

\index{Virtel TCT@\spxentry{Virtel TCT}!GTVSKIP parameter@\spxentry{GTVSKIP parameter}}\index{GTVSKIP parameter@\spxentry{GTVSKIP parameter}!Virtel TCT@\spxentry{Virtel TCT}}\ignorespaces 

\subsection{GTVSKIP parameter}
\label{\detokenize{Installation_Guide:gtvskip-parameter}}\label{\detokenize{Installation_Guide:index-74}}
\begin{sphinxVerbatim}[commandchars=\\\{\}]
\PYG{n}{GTVSKIP}\PYG{o}{=}\PYG{n}{n}                \PYG{n}{Default}\PYG{o}{=}\PYG{l+s+s1}{\PYGZsq{}}\PYG{l+s+s1}{0}\PYG{l+s+s1}{\PYGZsq{}}
\end{sphinxVerbatim}

\sphinxAtStartPar
\sphinxstylestrong{n} \sphinxhyphen{} Is the displacement used to localise the data in the VSAM record being read.

\index{Virtel TCT@\spxentry{Virtel TCT}!GTVSKIP parameter@\spxentry{GTVSKIP parameter}}\index{GTVSKIP parameter@\spxentry{GTVSKIP parameter}!Virtel TCT@\spxentry{Virtel TCT}}\ignorespaces 

\subsection{GUIDE parameter}
\label{\detokenize{Installation_Guide:guide-parameter}}\label{\detokenize{Installation_Guide:index-75}}
\begin{sphinxVerbatim}[commandchars=\\\{\}]
\PYG{n}{GUIDE}\PYG{o}{=}\PYG{n}{xx}                 \PYG{n}{Default}\PYG{o}{=}\PYG{n}{F1} \PYG{p}{(}\PYG{n}{PF1}\PYG{p}{)}
\end{sphinxVerbatim}

\sphinxAtStartPar
\sphinxstylestrong{xx} \sphinxhyphen{} The 3270 AID function key which will be transmitted to the application when the user presses the {[}GUIDE{]} key. This parameter allows the definition of a general value by default that may be modified when defining the subserver nodes.

\sphinxAtStartPar
GUIDE=00 allows the {[}GUIDE{]} key to display a pad offering further choices.

\index{Virtel TCT@\spxentry{Virtel TCT}!HTFORWD parameter@\spxentry{HTFORWD parameter}}\index{HTFORWD parameter@\spxentry{HTFORWD parameter}!Virtel TCT@\spxentry{Virtel TCT}}\ignorespaces 

\subsection{HTFORWD parameter}
\label{\detokenize{Installation_Guide:htforwd-parameter}}\label{\detokenize{Installation_Guide:index-76}}
\begin{sphinxVerbatim}[commandchars=\\\{\}]
\PYG{n}{HTFORWD}\PYG{o}{=}\PYG{p}{(}\PYG{n}{proxy1}\PYG{p}{,}\PYG{n}{proxy2}\PYG{p}{,}\PYG{o}{.}\PYG{o}{.}\PYG{o}{.}\PYG{p}{)}        \PYG{n}{Default}\PYG{o}{=}\PYG{n}{none}
\end{sphinxVerbatim}

\sphinxAtStartPar
\sphinxstylestrong{(proxy1,…)} \sphinxhyphen{} Specifies the IP address(es) of one or more proxy servers which forward HTTP requests to VIRTEL on behalf of clients.

\sphinxAtStartPar
For all requests received from these proxies, VIRTEL obtains the client’s IP address from the iv\sphinxhyphen{}remote\sphinxhyphen{}address: or the X\sphinxhyphen{}Forwarded\sphinxhyphen{}For: HTTP header generated by the proxy. This function may also be activated on a per\sphinxhyphen{}line basis by specifying the proxy address in the “Calling DTE” field of a rule (see “Rules” in the VIRTEL Connectivity Reference manual).

\begin{sphinxadmonition}{note}{Note:}
\sphinxAtStartPar
IP addresses must include leading zeroes. For example, HTFORWD=(192.168.001.020,010.001.001.020)
\end{sphinxadmonition}

\index{Virtel TCT@\spxentry{Virtel TCT}!HTHEADR parameter@\spxentry{HTHEADR parameter}}\index{HTHEADR parameter@\spxentry{HTHEADR parameter}!Virtel TCT@\spxentry{Virtel TCT}}\ignorespaces 

\subsection{HTHEADR parameter}
\label{\detokenize{Installation_Guide:htheadr-parameter}}\label{\detokenize{Installation_Guide:index-77}}
\begin{sphinxVerbatim}[commandchars=\\\{\}]
\PYG{n}{HTHEADR}\PYG{o}{=}\PYG{p}{(}\PYG{n}{h1}\PYG{p}{,}\PYG{n}{h2}\PYG{p}{,}\PYG{o}{.}\PYG{o}{.}\PYG{o}{.}\PYG{p}{)}                \PYG{n}{Default}\PYG{o}{=}\PYG{n}{none}
\end{sphinxVerbatim}

\sphinxAtStartPar
\sphinxstylestrong{(h1,h2,…)} \sphinxhyphen{} Specifies the names of up to 5 additional HTTP headers whose value is to be made available to scenarios. The names must be specified in upper case in this parameter, although the headers in the HTTP request may be upper or lower case. Refer to the description of the COPY\$ SYSTEM\sphinxhyphen{}TO\sphinxhyphen{}VARIABLE instruction in the VIRTEL Web Access Guide for further details.

\index{Virtel TCT@\spxentry{Virtel TCT}!HTMINI parameter@\spxentry{HTMINI parameter}}\index{HTMINI parameter@\spxentry{HTMINI parameter}!Virtel TCT@\spxentry{Virtel TCT}}\ignorespaces 

\subsection{HTMINI parameter}
\label{\detokenize{Installation_Guide:htmini-parameter}}\label{\detokenize{Installation_Guide:index-78}}
\begin{sphinxVerbatim}[commandchars=\\\{\}]
\PYG{n}{HTMINI}\PYG{o}{=}\PYG{p}{(}\PYG{n+nb}{len}\PYG{p}{,}\PYG{n}{time}\PYG{p}{)}                  \PYG{n}{Default}\PYG{o}{=}\PYG{p}{(}\PYG{l+m+mi}{40}\PYG{p}{,}\PYG{l+m+mi}{100}\PYG{p}{)}
\end{sphinxVerbatim}

\sphinxAtStartPar
The HTMINI parameter allows control over messages sent by VIRTEL Web Access applications. Certain applications may send several 3270 messages which together make up a complete screen. VIRTEL attempts to combine such messages into a single transmission to the browser, in order to avoid the need for the user to press ENTER to retrieve each message sent by the application.

\sphinxAtStartPar
VIRTEL considers that a message is possibly incomplete if the following conditions are true:
\begin{itemize}
\item {} 
\sphinxAtStartPar
The flag “restore keyboard” flag is not set in the 3270 WCC

\item {} 
\sphinxAtStartPar
The “start printer” flag is not set in the 3270 WCC

\item {} 
\sphinxAtStartPar
The message length is less than or equal to len bytes

\item {} 
\sphinxAtStartPar
The message does not contain an “insert cursor” command {[}R.Bowler : “Text hidden because the presence of an insert cursor command no longer inhibits the operation of the HTMINI parameter (see updt2717 in Virtel 4.23)”{]}

\end{itemize}

\sphinxAtStartPar
After the arrival of a possibly incomplete message, VIRTEL waits for time hundreths of a second. If no other message has arrived during this interval, the possibly incomplete message is sent to the browser anyway. Otherwise, the possibly incomplete message is combined with the following message before sending it to the browser.

\index{Virtel TCT@\spxentry{Virtel TCT}!HTPARM parameter@\spxentry{HTPARM parameter}}\index{HTPARM parameter@\spxentry{HTPARM parameter}!Virtel TCT@\spxentry{Virtel TCT}}\ignorespaces 

\subsection{HTPARM parameter}
\label{\detokenize{Installation_Guide:htparm-parameter}}\label{\detokenize{Installation_Guide:index-79}}
\begin{sphinxVerbatim}[commandchars=\\\{\}]
\PYG{n}{HTPARM}\PYG{o}{=}\PYG{p}{(}\PYG{n}{n1}\PYG{p}{,}\PYG{n}{n2}\PYG{p}{)}                 \PYG{n}{Default}\PYG{o}{=}\PYG{p}{(}\PYG{l+m+mi}{30000}\PYG{p}{,}\PYG{l+m+mi}{4096000}\PYG{p}{)}
\end{sphinxVerbatim}

\sphinxAtStartPar
This parameter allows you to override various VIRTEL Web Access settings. If HTPARM is specified, then all subparameters must be coded. The sub\sphinxhyphen{}parameters are:

\sphinxAtStartPar
\sphinxstylestrong{n1} \sphinxhyphen{} HTTP segment size. Do not change from the default value of 30000 unless advised by VIRTEL support.

\sphinxAtStartPar
\sphinxstylestrong{n2} \sphinxhyphen{} Maximum file size (in bytes) allowed for an IND\$FILE transfer. The default value 4096000 permits a maximum transfer size of approximately 4MB. For upload, If the size is exceeded the user will see HTTP response code “413 Request Entity Too Large”. For download, if the size is exceeded the user will see error message “TRANS14 Error reading file from host: file transfer canceled”.

\index{Virtel TCT@\spxentry{Virtel TCT}!HTSETn parameter@\spxentry{HTSETn parameter}}\index{HTSETn parameter@\spxentry{HTSETn parameter}!Virtel TCT@\spxentry{Virtel TCT}}\ignorespaces 

\subsection{HTSET1 to HTSET4 parameters}
\label{\detokenize{Installation_Guide:htset1-to-htset4-parameters}}\label{\detokenize{Installation_Guide:index-80}}
\begin{sphinxVerbatim}[commandchars=\\\{\}]
\PYG{n}{HTSETx}\PYG{o}{=}\PYG{p}{(}\PYG{n}{option}\PYG{p}{,}\PYG{n}{option}\PYG{p}{,}\PYG{o}{.}\PYG{o}{.}\PYG{o}{.}\PYG{p}{)}          \PYG{n}{Default}\PYG{o}{=}\PYG{n}{none}
\end{sphinxVerbatim}

\sphinxAtStartPar
These parameters allow various HTML processing options to be set as defaults. Each parameter has the form HTSETx = (option, option, …) where option can take the values listed below:

\sphinxAtStartPar
\sphinxstylestrong{HTSET1} \sphinxhyphen{} MAXLENGTH, ID, BLANK\sphinxhyphen{}BINARY\sphinxhyphen{}ZEROES, HTML\sphinxhyphen{}ESCAPES, JAVASCRIPT\sphinxhyphen{}ESCAPES, XML\sphinxhyphen{}ESCAPES, AUTO\sphinxhyphen{}INCREMENTVARIABLES

\sphinxAtStartPar
\sphinxstylestrong{HTSET2} \sphinxhyphen{} NO\sphinxhyphen{}ADD\sphinxhyphen{}TO\sphinxhyphen{}CHECKBOX, NO\sphinxhyphen{}ADD\sphinxhyphen{}TO\sphinxhyphen{}LISTBOX, DO\sphinxhyphen{}NOT\sphinxhyphen{}IGNORE\sphinxhyphen{}BINARY\sphinxhyphen{}ZEROES

\sphinxAtStartPar
\sphinxstylestrong{HTSET3} \sphinxhyphen{} Reserved for future use

\sphinxAtStartPar
\sphinxstylestrong{HTSET4} \sphinxhyphen{} Reserved for future use

\sphinxAtStartPar
These processing options can be enabled or disabled within individual page templates via the SET\sphinxhyphen{}LOCAL\sphinxhyphen{}OPTIONS and UNSET\sphinxhyphen{}LOCAL\sphinxhyphen{}OPTIONS tags. Refer to the description of these tags in the VIRTEL Web Access Guide for the further details and for the meaning of each option.

\index{Virtel TCT@\spxentry{Virtel TCT}!HTVSAM parameter@\spxentry{HTVSAM parameter}}\index{HTVSAM parameter@\spxentry{HTVSAM parameter}!Virtel TCT@\spxentry{Virtel TCT}}\ignorespaces 

\subsection{HTVSAM parameter}
\label{\detokenize{Installation_Guide:htvsam-parameter}}\label{\detokenize{Installation_Guide:index-81}}
\begin{sphinxVerbatim}[commandchars=\\\{\}]
\PYG{n}{HTVSAM}\PYG{o}{=}\PYG{n}{xxxxxxx}                     \PYG{n}{Default}\PYG{o}{=} \PYG{p}{(}\PYG{n}{none}\PYG{p}{)}
\end{sphinxVerbatim}

\sphinxAtStartPar
\sphinxstylestrong{xxxxxxxx} \sphinxhyphen{} Indicates the DD name in the VIRTEL JCL procedure of the VSAM file used to store the names of the e\sphinxhyphen{}mail correspondents for VIRTEL Web Access applications or USERID information for the centralised parameter feature; UPARM= specified in the TCT. Installations using the VIRTEL Web Access feature must specify HTVSAM=VIRHTML and include a VIRHTML DD/DLBL statement in the VIRTEL JCL procedure. If no HTTP or SMTP lines are defined in the VIRTEL configuration, then the HTVSAM parameter may be omitted, and the VIRHTML file is not required.

\index{Virtel TCT@\spxentry{Virtel TCT}!IBERTEX parameter@\spxentry{IBERTEX parameter}}\index{IBERTEX parameter@\spxentry{IBERTEX parameter}!Virtel TCT@\spxentry{Virtel TCT}}\ignorespaces 

\subsection{IBERTEX parameter}
\label{\detokenize{Installation_Guide:ibertex-parameter}}\label{\detokenize{Installation_Guide:index-82}}
\begin{sphinxVerbatim}[commandchars=\\\{\}]
\PYG{n}{IBERTEX}\PYG{o}{=}\PYG{n}{YES}\PYG{o}{/}\PYG{n}{NO}                      \PYG{n}{Default}\PYG{o}{=}\PYG{n}{NO}
\end{sphinxVerbatim}

\sphinxAtStartPar
\sphinxstylestrong{YES} \sphinxhyphen{} Supports the CEPT1 (Spanish) protocol standard.

\sphinxAtStartPar
\sphinxstylestrong{NO} \sphinxhyphen{} Does not support the CEPT1 standard.

\index{Virtel TCT@\spxentry{Virtel TCT}!IGNLU parameter@\spxentry{IGNLU parameter}}\index{IGNLU parameter@\spxentry{IGNLU parameter}!Virtel TCT@\spxentry{Virtel TCT}}\ignorespaces 

\subsection{IGNLU parameter}
\label{\detokenize{Installation_Guide:ignlu-parameter}}\label{\detokenize{Installation_Guide:index-83}}
\begin{sphinxVerbatim}[commandchars=\\\{\}]
\PYG{n}{IGNLU}\PYG{o}{=}\PYG{p}{(}\PYG{n}{LuMch1}\PYG{p}{,}\PYG{n}{LuMch2}\PYG{p}{,}\PYG{o}{.}\PYG{o}{.}\PYG{o}{.}\PYG{p}{)}            \PYG{n}{Default}\PYG{o}{=}\PYG{l+s+s1}{\PYGZsq{}}\PYG{l+s+s1}{ }\PYG{l+s+s1}{\PYGZsq{}}
\end{sphinxVerbatim}

\sphinxAtStartPar
\sphinxstylestrong{LuMchx} \sphinxhyphen{} The IGNLU parameter contains a list of line names which are not to be activated at VIRTEL startup time.

\index{Virtel TCT@\spxentry{Virtel TCT}!LANG parameter@\spxentry{LANG parameter}}\index{LANG parameter@\spxentry{LANG parameter}!Virtel TCT@\spxentry{Virtel TCT}}\ignorespaces 

\subsection{LANG parameter}
\label{\detokenize{Installation_Guide:lang-parameter}}\label{\detokenize{Installation_Guide:index-84}}
\begin{sphinxVerbatim}[commandchars=\\\{\}]
\PYG{n}{LANG}\PYG{o}{=}\PYG{l+s+s1}{\PYGZsq{}}\PYG{l+s+s1}{E}\PYG{l+s+s1}{\PYGZsq{}}                             \PYG{n}{Default}\PYG{o}{=}\PYG{l+s+s1}{\PYGZsq{}}\PYG{l+s+s1}{ }\PYG{l+s+s1}{\PYGZsq{}}
\end{sphinxVerbatim}

\sphinxAtStartPar
Specifies the language in which the VIRTEL administration panels are displayed. The following values are possible:
\sphinxstylestrong{‘ ‘} \sphinxhyphen{} French language.
\sphinxstylestrong{‘E’} \sphinxhyphen{} English language.

\begin{sphinxadmonition}{note}{Note:}
\sphinxAtStartPar
The apostrophes are required.
\end{sphinxadmonition}

\index{Virtel TCT@\spxentry{Virtel TCT}!LICENCE parameter@\spxentry{LICENCE parameter}}\index{LICENCE parameter@\spxentry{LICENCE parameter}!Virtel TCT@\spxentry{Virtel TCT}}\ignorespaces 

\subsection{LICENCE parameter}
\label{\detokenize{Installation_Guide:licence-parameter}}\label{\detokenize{Installation_Guide:index-85}}
\begin{sphinxVerbatim}[commandchars=\\\{\}]
\PYG{n}{LICENCE}\PYG{o}{=}\PYG{l+s+s1}{\PYGZsq{}}\PYG{l+s+s1}{ }\PYG{l+s+s1}{\PYGZsq{}}                          \PYG{n}{Default}\PYG{o}{=}\PYG{l+s+s1}{\PYGZsq{}}\PYG{l+s+s1}{ }\PYG{l+s+s1}{\PYGZsq{}}
\end{sphinxVerbatim}

\sphinxAtStartPar
Is the number of the licence attributed to the client as it is specified in the installation key at the time of the installation. This code is unique for each client and functions in relation to the following parameters: ADDR1, ADDR2, COMPANY, EXPIRE and CODE.

\index{Virtel TCT@\spxentry{Virtel TCT}!LOCK parameter@\spxentry{LOCK parameter}}\index{LOCK parameter@\spxentry{LOCK parameter}!Virtel TCT@\spxentry{Virtel TCT}}\ignorespaces 

\subsection{LOCK parameter}
\label{\detokenize{Installation_Guide:lock-parameter}}\label{\detokenize{Installation_Guide:index-86}}
\begin{sphinxVerbatim}[commandchars=\\\{\}]
\PYG{n}{LOCK}\PYG{o}{=}\PYG{n}{n}                               \PYG{n}{Default}\PYG{o}{=}\PYG{l+m+mi}{32767}
\end{sphinxVerbatim}

\sphinxAtStartPar
\sphinxstylestrong{n} \sphinxhyphen{} Inactivity delay in minutes, after which a VIRTEL will lock a terminal and request the user to resubmit his password. When the LOCK is triggered, a VIR0510W message appears on the console.

\index{Virtel TCT@\spxentry{Virtel TCT}!LOG parameter@\spxentry{LOG parameter}}\index{LOG parameter@\spxentry{LOG parameter}!Virtel TCT@\spxentry{Virtel TCT}}\ignorespaces 

\subsection{LOG parameter}
\label{\detokenize{Installation_Guide:log-parameter}}\label{\detokenize{Installation_Guide:index-87}}
\begin{sphinxVerbatim}[commandchars=\\\{\}]
\PYG{n}{LOG}\PYG{o}{=}\PYG{n}{CONSOLE} \PYG{o}{|} \PYG{n}{SYSOUT} \PYG{o}{|} \PYG{p}{(}\PYG{n}{SYSOUT}\PYG{p}{,}\PYG{n}{class}\PYG{p}{)}\PYG{o}{|} \PYG{n}{LOGGER} \PYG{o}{|} \PYG{n}{FILE}             \PYG{n}{Default}\PYG{o}{=}\PYG{n}{CONSOLE}
\end{sphinxVerbatim}
\begin{description}
\sphinxlineitem{CONSOLE}
\sphinxAtStartPar
WTOs are written to the SYSTEM console.

\sphinxlineitem{SYSOUT or (SYSOUT,class)}
\sphinxAtStartPar
WTOs are written to the sysout dataset, eventually in a specific class.

\sphinxlineitem{LOGGER}
\sphinxAtStartPar
WTOs are written to Sysplex logger.

\sphinxlineitem{FILE}
\sphinxAtStartPar
Messages will be written to the DDNAMEs LOGFILEx|y               (Virtel 4.61)

\end{description}

\index{Virtel TCT@\spxentry{Virtel TCT}!LPKALIVE parameter@\spxentry{LPKALIVE parameter}}\index{LPKALIVE parameter@\spxentry{LPKALIVE parameter}!Virtel TCT@\spxentry{Virtel TCT}}\ignorespaces 

\subsection{LPKALIVE parameter}
\label{\detokenize{Installation_Guide:lpkalive-parameter}}\label{\detokenize{Installation_Guide:index-88}}
\begin{sphinxVerbatim}[commandchars=\\\{\}]
    \PYG{n}{LPKALIVE}\PYG{o}{=}\PYG{n}{nn}         \PYG{n}{LONG} \PYG{n}{POLL} \PYG{n}{KEEP} \PYG{n}{ALIVE} \PYG{n}{VALUE} \PYG{o+ow}{in} \PYG{n}{Seconds}\PYG{o}{.}
\PYG{n}{LPKALIVE}\PYG{o}{=}\PYG{l+m+mi}{0}          \PYG{l+m+mi}{0} \PYG{n}{means} \PYG{n}{inactive}

    \PYG{n}{It} \PYG{o+ow}{is} \PYG{n}{possible} \PYG{k}{for} \PYG{n}{some} \PYG{n}{intermediary} \PYG{n}{equipment} \PYG{o+ow}{or} \PYG{n}{firewalls} \PYG{n}{to} \PYG{k}{try} \PYG{n}{to} \PYG{n}{close} \PYG{n}{the} \PYG{n}{Virtel} \PYG{n}{Long} \PYG{n}{Poll} \PYG{n}{session}\PYG{p}{,} \PYG{n}{because} \PYG{o+ow}{is} \PYG{n}{appears} \PYG{k}{as} \PYG{n}{a} \PYG{n}{stalled} \PYG{n}{request}\PYG{o}{.} \PYG{n}{This} \PYG{n}{causes} \PYG{o+ow}{or} \PYG{n}{appears} \PYG{k}{as} \PYG{n}{a} \PYG{n}{hung} \PYG{n}{session}\PYG{o}{.}

    \PYG{n}{Example}\PYG{p}{:} \PYG{o}{\PYGZhy{}}
    \PYG{n}{LPKALIVE}\PYG{o}{=}\PYG{l+m+mi}{30}             \PYG{n}{A} \PYG{l+m+mi}{304} \PYG{n}{response} \PYG{n}{to} \PYG{n}{the} \PYG{n}{LongPoll} \PYG{n}{session} \PYG{n}{will} \PYG{n}{be} \PYG{n}{sent} \PYG{n}{by} \PYG{n}{Virtel} \PYG{l+m+mi}{30} \PYG{n}{seconds} \PYG{n}{after} \PYG{n}{receiving} \PYG{n}{a} \PYG{n}{LP}\PYG{o}{=}\PYG{l+m+mi}{0} \PYG{n}{request}\PYG{o}{.}

    \PYG{n}{This} \PYG{n}{parameter} \PYG{n}{should} \PYG{n}{only} \PYG{n}{be} \PYG{n}{included} \PYG{o+ow}{in} \PYG{n}{the} \PYG{n}{TCT} \PYG{k}{if} \PYG{n}{recommended} \PYG{n}{by} \PYG{n}{Syspertec}\PYG{o}{.}
\end{sphinxVerbatim}

\index{Virtel TCT@\spxentry{Virtel TCT}!MARK parameter@\spxentry{MARK parameter}}\index{MARK parameter@\spxentry{MARK parameter}!Virtel TCT@\spxentry{Virtel TCT}}\ignorespaces 

\subsection{MARK parameter}
\label{\detokenize{Installation_Guide:mark-parameter}}\label{\detokenize{Installation_Guide:index-89}}
\begin{sphinxVerbatim}[commandchars=\\\{\}]
\PYG{n}{MARK}\PYG{o}{=}\PYG{n}{xx}                              \PYG{n}{Default}\PYG{o}{=}\PYG{l+m+mi}{1}\PYG{n}{E}\PYG{p}{(}\PYG{n}{EndField}\PYG{p}{)}
\end{sphinxVerbatim}

\sphinxAtStartPar
\sphinxstylestrong{xx} \sphinxhyphen{} Code of the key enabling selection of fields in a Multi\sphinxhyphen{}Session copy / paste operation. The default key is ‘end of field’ : Shift PA2.

\index{Virtel TCT@\spxentry{Virtel TCT}!MAXAPPL parameter@\spxentry{MAXAPPL parameter}}\index{MAXAPPL parameter@\spxentry{MAXAPPL parameter}!Virtel TCT@\spxentry{Virtel TCT}}\ignorespaces 

\subsection{MAXAPPL parameter}
\label{\detokenize{Installation_Guide:maxappl-parameter}}\label{\detokenize{Installation_Guide:index-90}}
\begin{sphinxVerbatim}[commandchars=\\\{\}]
\PYG{n}{MAXAPPL}\PYG{o}{=}\PYG{n}{n}                            \PYG{n}{Default}\PYG{o}{=}\PYG{l+m+mi}{64}
\end{sphinxVerbatim}

\sphinxAtStartPar
\sphinxstylestrong{n} \sphinxhyphen{} The maximum number of applications or transactions that may appear in the VIRTEL Multi\sphinxhyphen{}Session screen. The maximum value allowed is 64.

\index{Virtel TCT@\spxentry{Virtel TCT}!MEMORY parameter@\spxentry{MEMORY parameter}}\index{MEMORY parameter@\spxentry{MEMORY parameter}!Virtel TCT@\spxentry{Virtel TCT}}\ignorespaces 

\subsection{MEMORY parameter}
\label{\detokenize{Installation_Guide:memory-parameter}}\label{\detokenize{Installation_Guide:index-91}}
\begin{sphinxVerbatim}[commandchars=\\\{\}]
\PYG{n}{MEMORY}\PYG{o}{=}\PYG{p}{(}\PYG{n}{BELOW}\PYG{o}{/}\PYG{n}{ABOVE}\PYG{p}{[}\PYG{p}{,}\PYG{n}{DEBUG}\PYG{p}{]}\PYG{p}{)} \PYG{n}{Default}\PYG{o}{=}\PYG{n}{BELOW}
\PYG{n}{MEMORY}\PYG{o}{=}\PYG{n}{NATIVE}\PYG{o}{/}\PYG{n}{TEST}
\end{sphinxVerbatim}

\sphinxAtStartPar
Indicates the type of memory management used by VIRTEL:

\sphinxAtStartPar
\sphinxstylestrong{BELOW} \sphinxhyphen{} Memory managed by VIRTEL, with memory obtained below the 16 megabyte line.

\sphinxAtStartPar
\sphinxstylestrong{ABOVE} \sphinxhyphen{} Memory is managed by VIRTEL, with memory obtained above the 16 megabyte line.

\sphinxAtStartPar
\sphinxstylestrong{NATIVE} \sphinxhyphen{} Memory managed by z/OS or z/VSE

\sphinxAtStartPar
\sphinxstylestrong{TEST} \sphinxhyphen{} NATIVE plus ability to track memory usage.

\sphinxAtStartPar
\sphinxstylestrong{DEBUG} \sphinxhyphen{} Turn on debug option. Only set on when instructed to by Virtel Support due to performance implications.

\begin{sphinxadmonition}{note}{Note:}
\sphinxAtStartPar
MEMORY=ABOVE is recommended under z/OS. MEMORY=(ABOVE,DEBUG) consumes more resources and is intended for debugging of memory corruption errors. NATIVE may produce a smaller real storage footprint for some HTML applications with very large numbers of terminals defined. TEST allows monitoring of memory usage by module via sub\sphinxhyphen{}application F4. TEST also produces a report of allocated memory via the output of the SNAP command.
\end{sphinxadmonition}

\index{Virtel TCT@\spxentry{Virtel TCT}!MQn parameter@\spxentry{MQn parameter}}\index{MQn parameter@\spxentry{MQn parameter}!Virtel TCT@\spxentry{Virtel TCT}}\ignorespaces 

\subsection{MQ1 parameter}
\label{\detokenize{Installation_Guide:mq1-parameter}}\label{\detokenize{Installation_Guide:index-92}}
\begin{sphinxVerbatim}[commandchars=\\\{\}]
\PYG{n}{MQ1}\PYG{o}{=}\PYG{p}{(}\PYG{n}{mqmname}\PYG{p}{,}\PYG{l+s+s1}{\PYGZsq{}}\PYG{l+s+s1}{prefix}\PYG{l+s+s1}{\PYGZsq{}}\PYG{p}{,}\PYG{p}{[}\PYG{n}{pgmname}\PYG{p}{]}\PYG{p}{)}         \PYG{n}{Default}\PYG{o}{=}\PYG{n}{no} \PYG{n}{MQ} \PYG{n}{connection}
\end{sphinxVerbatim}

\sphinxAtStartPar
This parameter defines the characteristics of the connection to the message\sphinxhyphen{}queue manager (MQSeries) used by all lines which specify type MQ1.

\sphinxAtStartPar
\sphinxstylestrong{mqmname} \sphinxhyphen{} The name of the message\sphinxhyphen{}queue manager (for example, CSQ1).

\sphinxAtStartPar
\sphinxstylestrong{prefix} \sphinxhyphen{} A prefix which VIRTEL will add to all queue names. The prefix must be specified in quotes.

\sphinxAtStartPar
\sphinxstylestrong{pgmname} \sphinxhyphen{} The name of the VIRTEL MQ interface program used for this connection. The following values can be specified: VIR0Q09 Interface program for MQSeries. This is the default.


\subsection{MQ2 parameter}
\label{\detokenize{Installation_Guide:mq2-parameter}}
\begin{sphinxVerbatim}[commandchars=\\\{\}]
\PYG{n}{MQ2}\PYG{o}{=}\PYG{p}{(}\PYG{n}{mqmname}\PYG{p}{,}\PYG{l+s+s1}{\PYGZsq{}}\PYG{l+s+s1}{prefix}\PYG{l+s+s1}{\PYGZsq{}}\PYG{p}{,}\PYG{p}{[}\PYG{n}{pgmname}\PYG{p}{]}\PYG{p}{)}          \PYG{n}{Default}\PYG{o}{=}\PYG{n}{no} \PYG{l+m+mi}{2}\PYG{n}{nd} \PYG{n}{MQ} \PYG{n}{connection}
\end{sphinxVerbatim}

\sphinxAtStartPar
This parameter defines the characteristics of the connection to the message\sphinxhyphen{}queue manager (MQSeries) used by all lines which specify type MQ2. The subparameters are the same as those of the MQ1 parameter.


\subsection{MQ3 parameter}
\label{\detokenize{Installation_Guide:mq3-parameter}}
\begin{sphinxVerbatim}[commandchars=\\\{\}]
\PYG{n}{MQ3}\PYG{o}{=}\PYG{p}{(}\PYG{n}{mqmname}\PYG{p}{,}\PYG{l+s+s1}{\PYGZsq{}}\PYG{l+s+s1}{prefix}\PYG{l+s+s1}{\PYGZsq{}}\PYG{p}{,}\PYG{p}{[}\PYG{n}{pgmname}\PYG{p}{]}\PYG{p}{)}          \PYG{n}{Default}\PYG{o}{=}\PYG{n}{no} \PYG{l+m+mi}{3}\PYG{n}{rd} \PYG{n}{MQ} \PYG{n}{connection}
\end{sphinxVerbatim}

\sphinxAtStartPar
This parameter defines the characteristics of the connection to the message\sphinxhyphen{}queue manager (MQSeries) used by all lines which specify type MQ3. The subparameters are the same as those of the MQ1 parameter.


\subsection{MQ4 parameter}
\label{\detokenize{Installation_Guide:mq4-parameter}}
\begin{sphinxVerbatim}[commandchars=\\\{\}]
\PYG{n}{MQ4}\PYG{o}{=}\PYG{p}{(}\PYG{n}{mqmname}\PYG{p}{,}\PYG{l+s+s1}{\PYGZsq{}}\PYG{l+s+s1}{prefix}\PYG{l+s+s1}{\PYGZsq{}}\PYG{p}{,}\PYG{p}{[}\PYG{n}{pgmname}\PYG{p}{]}\PYG{p}{)} \PYG{n}{Default}\PYG{o}{=}\PYG{n}{no} \PYG{l+m+mi}{4}\PYG{n}{th} \PYG{n}{MQ} \PYG{n}{connection}
\end{sphinxVerbatim}

\sphinxAtStartPar
This parameter defines the characteristics of the connection to the message\sphinxhyphen{}queue manager (MQSeries) used by all lines which specify type MQ4. The subparameters are the same as those of the MQ1 parameter.

\index{Virtel TCT@\spxentry{Virtel TCT}!MULTI parameter@\spxentry{MULTI parameter}}\index{MULTI parameter@\spxentry{MULTI parameter}!Virtel TCT@\spxentry{Virtel TCT}}\ignorespaces 

\subsection{MULTI parameter}
\label{\detokenize{Installation_Guide:multi-parameter}}\label{\detokenize{Installation_Guide:index-93}}
\begin{sphinxVerbatim}[commandchars=\\\{\}]
\PYG{n}{MULTI}\PYG{o}{=}\PYG{n}{YES}\PYG{o}{/}\PYG{n}{NO} \PYG{n}{Default}\PYG{o}{=}\PYG{n}{YES}
\end{sphinxVerbatim}

\sphinxAtStartPar
\sphinxstylestrong{YES} \sphinxhyphen{} Support for VIRTEL Multi\sphinxhyphen{}Session environment.
\sphinxstylestrong{NO} \sphinxhyphen{} No Multi\sphinxhyphen{}Session.

\index{Virtel TCT@\spxentry{Virtel TCT}!NBCVC parameter@\spxentry{NBCVC parameter}}\index{NBCVC parameter@\spxentry{NBCVC parameter}!Virtel TCT@\spxentry{Virtel TCT}}\ignorespaces 

\subsection{NBCVC parameter}
\label{\detokenize{Installation_Guide:nbcvc-parameter}}\label{\detokenize{Installation_Guide:index-94}}
\begin{sphinxVerbatim}[commandchars=\\\{\}]
\PYG{n}{NBCVC}\PYG{o}{=}\PYG{n}{n} \PYG{n}{Default}\PYG{o}{=}\PYG{l+m+mi}{8}
\end{sphinxVerbatim}

\sphinxAtStartPar
\sphinxstylestrong{n} = The number of logical channels that are available for processing by VIRTEL.

\index{Virtel TCT@\spxentry{Virtel TCT}!NBDYNAM parameter@\spxentry{NBDYNAM parameter}}\index{NBDYNAM parameter@\spxentry{NBDYNAM parameter}!Virtel TCT@\spxentry{Virtel TCT}}\ignorespaces 

\subsection{NBDYNAM parameter}
\label{\detokenize{Installation_Guide:nbdynam-parameter}}\label{\detokenize{Installation_Guide:index-95}}
\begin{sphinxVerbatim}[commandchars=\\\{\}]
\PYG{n}{NBDYNAM}\PYG{o}{=}\PYG{p}{(}\PYG{n}{t1}\PYG{p}{,}\PYG{n}{t2}\PYG{p}{)} \PYG{n}{Default}\PYG{o}{=}\PYG{p}{(}\PYG{l+m+mi}{20}\PYG{p}{,}\PYG{l+m+mi}{0}\PYG{p}{)}
\end{sphinxVerbatim}

\sphinxAtStartPar
\sphinxstylestrong{t1} \sphinxhyphen{} The number of 3270 terminals that may connect via a “dynamic terminal definition entry” (welcome mode).
\sphinxstylestrong{t2} \sphinxhyphen{} The number of non\sphinxhyphen{}3270 terminals that may connect via a “dynamic terminal definition”.

\index{Virtel TCT@\spxentry{Virtel TCT}!NBTERM parameter@\spxentry{NBTERM parameter}}\index{NBTERM parameter@\spxentry{NBTERM parameter}!Virtel TCT@\spxentry{Virtel TCT}}\ignorespaces 

\subsection{NBTERM parameter}
\label{\detokenize{Installation_Guide:nbterm-parameter}}\label{\detokenize{Installation_Guide:index-96}}
\begin{sphinxVerbatim}[commandchars=\\\{\}]
\PYG{n}{NBTERM}\PYG{o}{=}\PYG{n}{nbterm} \PYG{n}{Default}\PYG{o}{=}\PYG{l+m+mi}{500}
\end{sphinxVerbatim}

\sphinxAtStartPar
\sphinxstylestrong{nbterm} \sphinxhyphen{} Number of terminals envisaged running in VIRTEL. This parameter allows the user to estimate the maximum number events that may be waiting for service at any one time.

\index{Virtel TCT@\spxentry{Virtel TCT}!NUMTASK parameter@\spxentry{NUMTASK parameter}}\index{NUMTASK parameter@\spxentry{NUMTASK parameter}!Virtel TCT@\spxentry{Virtel TCT}}\ignorespaces 

\subsection{NUMTASK parameter}
\label{\detokenize{Installation_Guide:numtask-parameter}}\label{\detokenize{Installation_Guide:index-97}}
\begin{sphinxVerbatim}[commandchars=\\\{\}]
\PYG{n}{NUMTASK}\PYG{o}{=}\PYG{n}{nn} \PYG{n}{Default}\PYG{o}{=}\PYG{l+m+mi}{4}
\end{sphinxVerbatim}

\sphinxAtStartPar
\sphinxstylestrong{nn} \sphinxhyphen{} The number of primary tasks waiting events on the primary VIRTEL ACB.

\index{Virtel TCT@\spxentry{Virtel TCT}!OTMAPRM parameter@\spxentry{OTMAPRM parameter}}\index{OTMAPRM parameter@\spxentry{OTMAPRM parameter}!Virtel TCT@\spxentry{Virtel TCT}}\ignorespaces 

\subsection{OTMAPRM parameter}
\label{\detokenize{Installation_Guide:otmaprm-parameter}}\label{\detokenize{Installation_Guide:index-98}}
\begin{sphinxVerbatim}[commandchars=\\\{\}]
\PYG{n}{OTMAPRM}\PYG{o}{=}\PYG{p}{(}\PYG{n}{exitname}\PYG{p}{,}\PYG{n}{userid}\PYG{p}{,}\PYG{n}{group}\PYG{p}{,}\PYG{n}{password}\PYG{p}{,}\PYG{n}{applname}\PYG{p}{)} \PYG{n}{Default}\PYG{o}{=}\PYG{p}{(}\PYG{o}{*}\PYG{n}{SAMPLE}\PYG{o}{*}\PYG{p}{)}
\end{sphinxVerbatim}

\sphinxAtStartPar
This parameter defines the data which is passed to OTMA/IMSConnect in the header of a RESUME TPIPE request. All of the subparameters are optional. Missing subparameters are indicated by a comma.

\sphinxAtStartPar
\sphinxstylestrong{exitname} \sphinxhyphen{} The identifier of the OTMA exit routine. Typical values are \sphinxstyleemphasis{SAMPLE} or \sphinxstyleemphasis{SAMPL1}. If omitted, the default value is \sphinxstyleemphasis{SAMPLE}.
\sphinxstylestrong{userid, group, password, applname} \sphinxhyphen{} Security parameters which VIRTEL will place in the userid, group, password, and application name fields in the RESUME TPIPE header.

\index{Virtel TCT@\spxentry{Virtel TCT}!OSCORE parameter@\spxentry{OSCORE parameter}}\index{OSCORE parameter@\spxentry{OSCORE parameter}!Virtel TCT@\spxentry{Virtel TCT}}\ignorespaces 

\subsection{OSCORE parameter}
\label{\detokenize{Installation_Guide:oscore-parameter}}\label{\detokenize{Installation_Guide:index-99}}
\begin{sphinxVerbatim}[commandchars=\\\{\}]
\PYG{n}{OSCORE}\PYG{o}{=}\PYG{n}{n} \PYG{n}{Default}\PYG{o}{=}\PYG{l+m+mi}{384}
\end{sphinxVerbatim}

\sphinxAtStartPar
\sphinxstylestrong{n} \sphinxhyphen{} The number of kilobytes reserved for memory allocation by the operating system (e.g. for loading sub application modules). The default value of this parameter is calculated when this macro is assembled and is indicated by an MNOTE being issued. This value may optionally be reduced but a problem may then arise if all functions of the sub applications are used.

\index{Virtel TCT@\spxentry{Virtel TCT}!PACKET parameter@\spxentry{PACKET parameter}}\index{PACKET parameter@\spxentry{PACKET parameter}!Virtel TCT@\spxentry{Virtel TCT}}\ignorespaces 

\subsection{PACKET parameter}
\label{\detokenize{Installation_Guide:packet-parameter}}\label{\detokenize{Installation_Guide:index-100}}
\begin{sphinxVerbatim}[commandchars=\\\{\}]
\PYG{n}{PACKET}\PYG{o}{=}\PYG{n}{n} \PYG{n}{Default}\PYG{o}{=}\PYG{l+m+mi}{128}
\end{sphinxVerbatim}

\sphinxAtStartPar
\sphinxstylestrong{n} \sphinxhyphen{} The size of the packets used for transfer over the packet switched network.

\phantomsection\label{\detokenize{Installation_Guide:virtvrrig-passtck}}
\index{Virtel TCT@\spxentry{Virtel TCT}!PASSTCK parameter@\spxentry{PASSTCK parameter}}\index{PASSTCK parameter@\spxentry{PASSTCK parameter}!Virtel TCT@\spxentry{Virtel TCT}}\ignorespaces 

\subsection{PASSTCK parameter}
\label{\detokenize{Installation_Guide:passtck-parameter}}\label{\detokenize{Installation_Guide:index-101}}
\begin{sphinxVerbatim}[commandchars=\\\{\}]
\PYG{n}{PASSTCK}\PYG{o}{=}\PYG{n}{YES} \PYG{n}{Default}\PYG{o}{=}\PYG{n}{none}
\end{sphinxVerbatim}

\sphinxAtStartPar
This parameter activates PassTicket support in VIRTEL. The following values are possible:

\sphinxAtStartPar
\sphinxstylestrong{YES} \sphinxhyphen{} VIRTEL may generate PassTickets for VIRTEL transactions which specify 1 or 2 in the PassTicket field If the PASSTCK parameter is omitted, VIRTEL will not generate PassTickets.

\index{Virtel TCT@\spxentry{Virtel TCT}!PREZ900 parameter@\spxentry{PREZ900 parameter}}\index{PREZ900 parameter@\spxentry{PREZ900 parameter}!Virtel TCT@\spxentry{Virtel TCT}}\ignorespaces 

\subsection{PREZ900 parameter}
\label{\detokenize{Installation_Guide:prez900-parameter}}\label{\detokenize{Installation_Guide:index-102}}
\begin{sphinxVerbatim}[commandchars=\\\{\}]
\PYG{n}{PREZ900}\PYG{o}{=}\PYG{n}{YES}\PYG{o}{/}\PYG{n}{NO} \PYG{n}{Default}\PYG{o}{=}\PYG{n}{NO}
\end{sphinxVerbatim}

\sphinxAtStartPar
Allows VIRTEL to run on a pre\sphinxhyphen{}zSeries processor. Possible values are:

\sphinxAtStartPar
\sphinxstylestrong{YES} \sphinxhyphen{} Specifies that the processor is a 9672, MP2000, MP3000, IS/390, or P/390. VIRTEL will not use instructions which are only available on z900 or later processors.

\sphinxAtStartPar
\sphinxstylestrong{NO} \sphinxhyphen{} Specifies that VIRTEL may use all instructions available on z900 or later processors.

\begin{sphinxadmonition}{note}{Note:}
\sphinxAtStartPar
VIRTEL does not support 9672\sphinxhyphen{}G1, ES/9000, or any earlier processor.
\end{sphinxadmonition}

\index{Virtel TCT@\spxentry{Virtel TCT}!PRFSECU parameter@\spxentry{PRFSECU parameter}}\index{PRFSECU parameter@\spxentry{PRFSECU parameter}!Virtel TCT@\spxentry{Virtel TCT}}\ignorespaces 

\subsection{PRFSECU parameter}
\label{\detokenize{Installation_Guide:prfsecu-parameter}}\label{\detokenize{Installation_Guide:index-103}}
\begin{sphinxVerbatim}[commandchars=\\\{\}]
\PYG{n}{PRFSECU}\PYG{o}{=}\PYG{l+s+s1}{\PYGZsq{}}\PYG{l+s+s1}{xxxxxxxx}\PYG{l+s+s1}{\PYGZsq{}} \PYG{n}{Default}\PYG{o}{=}
\end{sphinxVerbatim}

\sphinxAtStartPar
\sphinxstylestrong{xxxxxxxx} \sphinxhyphen{} Indicates the maximum 8 character prefix associated with the resources defined in the security management system if using RACF, TOP SECRET or ACF2.

\index{Virtel TCT@\spxentry{Virtel TCT}!PWPROT parameter@\spxentry{PWPROT parameter}}\index{PWPROT parameter@\spxentry{PWPROT parameter}!Virtel TCT@\spxentry{Virtel TCT}}\ignorespaces 

\subsection{PWPROT parameter}
\label{\detokenize{Installation_Guide:pwprot-parameter}}\label{\detokenize{Installation_Guide:index-104}}
\begin{sphinxVerbatim}[commandchars=\\\{\}]
\PYG{n}{PWPROT}\PYG{o}{=}\PYG{n}{YES}\PYG{o}{/}\PYG{n}{NO} \PYG{n}{Default}\PYG{o}{=}\PYG{n}{NO}
\end{sphinxVerbatim}

\sphinxAtStartPar
\sphinxstylestrong{YES} \sphinxhyphen{} Supports protected field (DARK field) for 80 column terminal with PAD=INTEG coded. This parameter must also be specified in NPSI.

\sphinxAtStartPar
\sphinxstylestrong{NO} \sphinxhyphen{} No support for the protected field (DARK field) for 80 column terminal if PAD=INTEG.

\index{Virtel TCT@\spxentry{Virtel TCT}!RACAPPL parameter@\spxentry{RACAPPL parameter}}\index{RACAPPL parameter@\spxentry{RACAPPL parameter}!Virtel TCT@\spxentry{Virtel TCT}}\ignorespaces 

\subsection{RACAPPL parameter}
\label{\detokenize{Installation_Guide:racappl-parameter}}\label{\detokenize{Installation_Guide:index-105}}
\begin{sphinxVerbatim}[commandchars=\\\{\}]
\PYG{n}{RACAPPL}\PYG{o}{=}\PYG{n}{NONE}\PYG{o}{/}\PYG{n}{APPLID}\PYG{o}{/}\PYG{n}{GRNAME}\PYG{o}{/}\PYG{l+s+s1}{\PYGZsq{}}\PYG{l+s+s1}{name}\PYG{l+s+s1}{\PYGZsq{}} \PYG{n}{Default}\PYG{o}{=}\PYG{n}{NONE}
\end{sphinxVerbatim}

\sphinxAtStartPar
The RACAPPL parameter specifies the VIRTEL application name as it is known to RACF. When RACAPPL is present in the VIRTCT, VIRTEL will add an APPL= parameter to all RACROUTE VERIFY calls. RACF uses this application name for two purposes: (1) to validate that the user is authorized to access the VIRTEL application (if the RACF APPL class is active) and (2) to validate the user’s passticket using the PTKTDATA class (if the user signs on to VIRTEL using a passticket instead of a password). The possible values are:

\sphinxAtStartPar
\sphinxstylestrong{NONE} (or omitted) \sphinxhyphen{} VIRTEL will not use the APPL= parameter on RACROUTE VERIFY calls. In this case RACF will use the default application name (MVSxxxx where xxxx is the 4\sphinxhyphen{}character SMF identifier of the z/OS system)

\sphinxAtStartPar
\sphinxstylestrong{APPLID} \sphinxhyphen{} VIRTEL will use the VTAM APPLID of the VIRTEL started task (specified in the APPLID parameter of the VIRTCT) as the value of the APPL= parameter for RACF.

\sphinxAtStartPar
\sphinxstylestrong{GRNAME} \sphinxhyphen{} VIRTEL will use the VTAM generic resource name of the VIRTEL started task (specified in the GRNAME parameter of the VIRTCT) as the value of the APPL= parameter for RACF. This setting may be useful in a sysplex environment. It allows all VIRTEL STCs in the sysplex to present the same application name to RACF.

\sphinxAtStartPar
\sphinxstylestrong{‘name’} \sphinxhyphen{} VIRTEL will use the specified name as the value of the APPL= parameter for RACF. The name must be specified in single quotes.

\index{Virtel TCT@\spxentry{Virtel TCT}!RAPPL parameter@\spxentry{RAPPL parameter}}\index{RAPPL parameter@\spxentry{RAPPL parameter}!Virtel TCT@\spxentry{Virtel TCT}}\ignorespaces 

\subsection{RAPPL parameter}
\label{\detokenize{Installation_Guide:rappl-parameter}}\label{\detokenize{Installation_Guide:index-106}}
\begin{sphinxVerbatim}[commandchars=\\\{\}]
\PYG{n}{RAPPL}\PYG{o}{=}\PYG{n}{rappl} \PYG{n}{Default}\PYG{o}{=}\PYG{n}{USERA}
\end{sphinxVerbatim}

\sphinxAtStartPar
\sphinxstylestrong{rappl} \sphinxhyphen{} Name of the security management resource class which contains the applications resources for the Multi\sphinxhyphen{}Session function and for external servers. The entities in this resource class are external servers and VTAM applications. If resource \$\$ALLSRV is used, then all the servers defined in VIRTEL are authorised.

\index{Virtel TCT@\spxentry{Virtel TCT}!REALM parameter@\spxentry{REALM parameter}}\index{REALM parameter@\spxentry{REALM parameter}!Virtel TCT@\spxentry{Virtel TCT}}\ignorespaces 

\subsection{REALM parameter}
\label{\detokenize{Installation_Guide:realm-parameter}}\label{\detokenize{Installation_Guide:index-107}}
\begin{sphinxVerbatim}[commandchars=\\\{\}]
\PYG{n}{REALM}\PYG{o}{=}\PYG{n}{TRANSACT}\PYG{o}{/}\PYG{n}{APPLID}\PYG{o}{/}\PYG{n}{GRNAME} \PYG{n}{Default}\PYG{o}{=}\PYG{n}{TRANSACT}
\end{sphinxVerbatim}

\sphinxAtStartPar
This parameter specifies the name presented by VIRTEL to the browser in the HTTP basic security signon dialog. The possible values are:

\sphinxAtStartPar
\sphinxstylestrong{TRANSACT} \sphinxhyphen{} the external name of the VIRTEL transaction which is requesting security. This causes the browser to issue the signon prompt for each transaction the first time the transaction is requested in a browser session.

\sphinxAtStartPar
\sphinxstylestrong{APPLID} \sphinxhyphen{} the VTAM APPLID of the VIRTEL started task. With this setting VIRTEL presents the same realm name for all transactions, and thus the user sees only one signon prompt per browser session.

\sphinxAtStartPar
\sphinxstylestrong{GRNAME} \sphinxhyphen{} the VTAM generic resource name of the VIRTEL started task. This setting may be useful in a sysplex environment. It allows all VIRTEL STCs in the sysplex to present the same realm name to the browser.

\index{Virtel TCT@\spxentry{Virtel TCT}!REPET parameter@\spxentry{REPET parameter}}\index{REPET parameter@\spxentry{REPET parameter}!Virtel TCT@\spxentry{Virtel TCT}}\ignorespaces 

\subsection{REPET parameter}
\label{\detokenize{Installation_Guide:repet-parameter}}\label{\detokenize{Installation_Guide:index-108}}
\begin{sphinxVerbatim}[commandchars=\\\{\}]
\PYG{n}{REPET}\PYG{o}{=}\PYG{n}{xx} \PYG{n}{Default}\PYG{o}{=}\PYG{n}{F2} \PYG{p}{(}\PYG{n}{PF2}\PYG{p}{)}
\end{sphinxVerbatim}

\sphinxAtStartPar
\sphinxstylestrong{xx} \sphinxhyphen{} The 3270 AID function key which will be transmitted to the application when a user presses the {[}REPETITION{]} key. This parameter allows the definition of a general value by default which may be modified in the sub\sphinxhyphen{}server node definition. A value of 00 indicates that the {[}REPETITION{]} key will not be transmitted.

\index{Virtel TCT@\spxentry{Virtel TCT}!RESO parameter@\spxentry{RESO parameter}}\index{RESO parameter@\spxentry{RESO parameter}!Virtel TCT@\spxentry{Virtel TCT}}\ignorespaces 

\subsection{RESO parameter}
\label{\detokenize{Installation_Guide:reso-parameter}}\label{\detokenize{Installation_Guide:index-109}}
\begin{sphinxVerbatim}[commandchars=\\\{\}]
\PYG{n}{RESO}\PYG{o}{=}\PYG{n}{YES}\PYG{o}{/}\PYG{n}{NO} \PYG{n}{Default}\PYG{o}{=}\PYG{n}{YES}
\end{sphinxVerbatim}

\sphinxAtStartPar
\sphinxstylestrong{YES} \sphinxhyphen{} VIRTEL’s VTAM network management sub\sphinxhyphen{}application will be activated.

\sphinxAtStartPar
\sphinxstylestrong{NO} \sphinxhyphen{} The network management sub\sphinxhyphen{}application will not be used.

\index{Virtel TCT@\spxentry{Virtel TCT}!RETOUR parameter@\spxentry{RETOUR parameter}}\index{RETOUR parameter@\spxentry{RETOUR parameter}!Virtel TCT@\spxentry{Virtel TCT}}\ignorespaces 

\subsection{RETOUR parameter}
\label{\detokenize{Installation_Guide:retour-parameter}}\label{\detokenize{Installation_Guide:index-110}}
\begin{sphinxVerbatim}[commandchars=\\\{\}]
\PYG{n}{RETOUR}\PYG{o}{=}\PYG{n}{xx} \PYG{n}{Default}\PYG{o}{=}\PYG{l+m+mi}{00}
\end{sphinxVerbatim}

\sphinxAtStartPar
\sphinxstylestrong{xx} = The 3270 AID function key which will be transmitted to the application when the user presses the {[}RETURN{]} key. By default the {[}RETURN{]} key is not transmitted to the application but serves to set the cursor to the beginning of the preceding field. This parameter allows for the definition of a general value by default that may be modified in the definition of the sub\sphinxhyphen{}server nodes.

\index{Virtel TCT@\spxentry{Virtel TCT}!RNODE parameter@\spxentry{RNODE parameter}}\index{RNODE parameter@\spxentry{RNODE parameter}!Virtel TCT@\spxentry{Virtel TCT}}\ignorespaces 

\subsection{RNODE parameter}
\label{\detokenize{Installation_Guide:rnode-parameter}}\label{\detokenize{Installation_Guide:index-111}}
\begin{sphinxVerbatim}[commandchars=\\\{\}]
\PYG{n}{RNODE}\PYG{o}{=}\PYG{n}{rnode} \PYG{n}{Default}\PYG{o}{=}\PYG{n}{USERB}
\end{sphinxVerbatim}

\sphinxAtStartPar
\sphinxstylestrong{rnode} \sphinxhyphen{} The name of the security management resource class which contains the tree structure nodes, VIRTEL subapplication names, internal names of transactions associated with entry points, and directory names for file transfer.

\index{Virtel TCT@\spxentry{Virtel TCT}!RTERM parameter@\spxentry{RTERM parameter}}\index{RTERM parameter@\spxentry{RTERM parameter}!Virtel TCT@\spxentry{Virtel TCT}}\ignorespaces 

\subsection{RTERM parameter}
\label{\detokenize{Installation_Guide:rterm-parameter}}\label{\detokenize{Installation_Guide:index-112}}
\begin{sphinxVerbatim}[commandchars=\\\{\}]
\PYG{n}{RTERM}\PYG{o}{=}\PYG{k}{class}
\end{sphinxVerbatim}

\sphinxAtStartPar
\sphinxstylestrong{class} = The security resource class for terminals. This options forces RACF to validate the name of the LU specified on the ForceLUNAME parameter. For further details on setting an LU name with the ForcedLUNAME parameter. See the Virtel Connectivity Guide: ForcedLUNAME.

\sphinxAtStartPar
For example:\sphinxhyphen{}

\begin{sphinxVerbatim}[commandchars=\\\{\}]
\PYG{n}{RTERM}\PYG{o}{=}\PYG{n}{Facility}
\end{sphinxVerbatim}

\index{Virtel TCT@\spxentry{Virtel TCT}!SECUR parameter@\spxentry{SECUR parameter}}\index{SECUR parameter@\spxentry{SECUR parameter}!Virtel TCT@\spxentry{Virtel TCT}}\ignorespaces 

\subsection{SECUR parameter}
\label{\detokenize{Installation_Guide:secur-parameter}}\label{\detokenize{Installation_Guide:index-113}}
\begin{sphinxVerbatim}[commandchars=\\\{\}]
\PYG{n}{SECUR}\PYG{o}{=}\PYG{n}{NO}\PYG{o}{/}\PYG{n}{VIRTEL}\PYG{o}{/}\PYG{n}{RACF}\PYG{o}{/}\PYG{n}{TOPS}\PYG{o}{/}\PYG{n}{ACF2}\PYG{o}{/}\PYG{n}{RACROUTE}\PYG{o}{/}\PYG{n}{MIXEDCASE}\PYG{o}{/}\PYG{n}{PASSPHRASE}\PYG{o}{/}\PYG{p}{,}\PYG{n}{PASSPHRASELEN} \PYG{n}{Default}\PYG{o}{=}\PYG{n}{NO}
\end{sphinxVerbatim}

\sphinxAtStartPar
For the z/OS environment, the following options can be specified:

\sphinxAtStartPar
\sphinxstylestrong{NO} \sphinxhyphen{} No security software is used to control access.

\sphinxAtStartPar
\sphinxstylestrong{VIRTEL} \sphinxhyphen{} VIRTEL’s internal security management feature is used.

\sphinxAtStartPar
\sphinxstylestrong{(RACROUTE,RACF)} \sphinxhyphen{} IBM’s security management product is used (via SAF).

\sphinxAtStartPar
\sphinxstylestrong{(RACROUTE,TOPS)} \sphinxhyphen{} The TOP SECRET security management product is used (via SAF).

\sphinxAtStartPar
\sphinxstylestrong{(RACROUTE,ACF2)} \sphinxhyphen{} The ACF2 security management system is used (via SAF).

\begin{sphinxadmonition}{note}{Note:}
\sphinxAtStartPar
Passphrase support can be activated by coding PASSPHRASE as an option in the SECUR keyword. For example: \sphinxstylestrong{SECUR=(RACROUTE,{[}RACF|TOPS|ACF2{]},PASSPHRASE)}
You can also specify the minimum passphrase length using the Passphrase Length option. Default is 14 characters long.
The following sets the Passphrase length to 10 :
\sphinxstylestrong{SECURE=(RACROUTE,RACF,PASSPHRASE,10)}

\sphinxAtStartPar
Passwords and Passphrases cannot be intermixed. You cannot go from a password, length 8 or less, to a passphrase. Passphrases will always be mixed characters (Upper and Lower case) and will never be “UPPERCASED”.
\end{sphinxadmonition}

\sphinxAtStartPar
For the z/VSE environment, the following options can be specified:

\sphinxAtStartPar
\sphinxstylestrong{NO} \sphinxhyphen{} No security software is used to control access.

\sphinxAtStartPar
\sphinxstylestrong{VIRTEL} \sphinxhyphen{} VIRTEL’s internal security management feature is used.

\sphinxAtStartPar
\sphinxstylestrong{RACROUTE} \sphinxhyphen{} (for z/VSE Version 3 or later) VIRTEL uses the z/VSE Basic Security Manager (via SAF), or the External Security Manager if specified in the z/VSE IPL parameters.

\sphinxAtStartPar
The following options are retained for compatibility with previous versions:

\sphinxAtStartPar
\sphinxstylestrong{RACF} \sphinxhyphen{} RACF without SAF.

\sphinxAtStartPar
\sphinxstylestrong{TOPS} \sphinxhyphen{} TOP SECRET without SAF.

\sphinxAtStartPar
\sphinxstylestrong{ACF2} \sphinxhyphen{} ACF2 with ACFDIAG (Only for VM). For MVS, this is treated as (RACROUTE,ACF2).

\sphinxAtStartPar
\sphinxstylestrong{RACROUTE} \sphinxhyphen{} Multi product interface security (via SAF).

\sphinxAtStartPar
If MEMORY=ABOVE, RACF without SAF and TOPS without SAF are not supported.

\newpage

\sphinxAtStartPar
MIXEDCASE support prevents a password being automatically “UPPERCASED” prior to signon. This TCT option applies to Top Secret only. For example:\sphinxhyphen{}

\begin{sphinxVerbatim}[commandchars=\\\{\}]
\PYG{n}{SECUR}\PYG{o}{=}\PYG{p}{(}\PYG{n}{RACROUTE}\PYG{p}{,}\PYG{n}{TOPS}\PYG{p}{,}\PYG{n}{MIXEDCASE}\PYG{p}{)}\PYG{p}{,}      \PYG{o}{*}\PYG{n}{Setting} \PYG{n}{Mixedcase} \PYG{k}{with} \PYG{n}{TOPS}\PYG{o}{*}

\PYG{o+ow}{or}

\PYG{n}{SECUR}\PYG{o}{=}\PYG{p}{(}\PYG{n}{RACROUTE}\PYG{p}{,}\PYG{n}{TOPS}\PYG{p}{,}\PYG{n}{MIXEDCASE}\PYG{p}{,}\PYG{n}{PASSPHRASE}\PYG{p}{)}\PYG{p}{,} \PYG{o}{*}\PYG{n}{Setting} \PYG{n}{mixedcase} \PYG{o+ow}{and} \PYG{n}{passphrase} \PYG{n}{support} \PYG{k}{for} \PYG{n}{TOPS}\PYG{o}{*}
\end{sphinxVerbatim}

\index{Virtel TCT@\spxentry{Virtel TCT}!SILENCE parameter@\spxentry{SILENCE parameter}}\index{SILENCE parameter@\spxentry{SILENCE parameter}!Virtel TCT@\spxentry{Virtel TCT}}\ignorespaces 

\subsection{SILENCE parameter}
\label{\detokenize{Installation_Guide:silence-parameter}}\label{\detokenize{Installation_Guide:index-114}}
\begin{sphinxVerbatim}[commandchars=\\\{\}]
\PYG{n}{SILENCE}\PYG{o}{=}\PYG{n}{YES}\PYG{o}{/}\PYG{n}{NO} \PYG{n}{Default}\PYG{o}{=}\PYG{n}{NO}
\end{sphinxVerbatim}

\sphinxAtStartPar
\sphinxstylestrong{YES} \sphinxhyphen{} Inhibits the sending of asynchronous terminal connection and disconnection messages to the log (VIR0026W, VIR0028W, VIR0051I, VIR0052I, VIR0505I, VIR0507I, VIR1551I, VIRHT51I, VIRNA51I, VIRPF28I, VIRPF51I, VIRPF52I, VIRPF99I, VIRQ912W, VIRQ922W, VIRT912W, VIRT922W). Also, messages VIR0002W and VIR0914E will be suppressed on a REQSESS request.

\sphinxAtStartPar
\sphinxstylestrong{NO} \sphinxhyphen{} Enables the sending of asynchronous terminal connection and disconnection messages to the log.

\sphinxAtStartPar
The VIRTEL command SILENCE can be used to dynamically modify this parameter.

\index{Virtel TCT@\spxentry{Virtel TCT}!SNAPMSG parameter@\spxentry{SNAPMSG parameter}}\index{SNAPMSG parameter@\spxentry{SNAPMSG parameter}!Virtel TCT@\spxentry{Virtel TCT}}\ignorespaces 

\subsection{SNAPMSG parameter}
\label{\detokenize{Installation_Guide:snapmsg-parameter}}\label{\detokenize{Installation_Guide:index-115}}
\begin{sphinxVerbatim}[commandchars=\\\{\}]
\PYG{n}{SNAPMSG}\PYG{o}{=}\PYG{p}{(}\PYG{n}{message}\PYG{p}{,}\PYG{n}{search}\PYG{p}{,}\PYG{n}{action}\PYG{p}{)}
\end{sphinxVerbatim}

\sphinxAtStartPar
The SNAPMSG parameter allows a SNAP or DUMP to be taken whenever a particular message number is issued by VIRTEL. The command has an additional search field which can be used to identify a message with a particular character string, for example a specific return code. This feature is also avalable by using the SNAPMSG command from the console. (see “SNAPMSG command” in the VIRTEL Audit and Performance Reference manual.
\begin{description}
\sphinxlineitem{\sphinxstylestrong{Message}}
\sphinxAtStartPar
Any message that can be issued by Virtel.

\sphinxlineitem{\sphinxstylestrong{Search}}
\sphinxAtStartPar
Any seache criteria issued within the message. The search file is restricted to a maximu of 10 characters. Anything beyond will be ignored. Default search is none.

\sphinxlineitem{\sphinxstylestrong{Action}}
\sphinxAtStartPar
Possible values are S for SNAP or A for ABEND. Virtel will abend with a U0999 abend code, reason code 15 if the ABEND action is used. Default action is SNAP.

\end{description}

\index{Virtel TCT@\spxentry{Virtel TCT}!SNAPW parameter@\spxentry{SNAPW parameter}}\index{SNAPW parameter@\spxentry{SNAPW parameter}!Virtel TCT@\spxentry{Virtel TCT}}\ignorespaces 

\subsection{SNAPW parameter}
\label{\detokenize{Installation_Guide:snapw-parameter}}\label{\detokenize{Installation_Guide:index-116}}
\begin{sphinxVerbatim}[commandchars=\\\{\}]
\PYG{n}{SNAPW}\PYG{o}{=}\PYG{l+m+mi}{80}\PYG{o}{/}\PYG{l+m+mi}{132} \PYG{n}{Default}\PYG{o}{=}\PYG{l+m+mi}{132}
\end{sphinxVerbatim}

\sphinxAtStartPar
Indicates the default presentation format for SNAP and other dumps (80 or 132 columns). This parameter can be dynamically modified by the VIRTEL SNAPW command.


\subsection{SOMMR parameter}
\label{\detokenize{Installation_Guide:sommr-parameter}}
\begin{sphinxVerbatim}[commandchars=\\\{\}]
\PYG{n}{SOMMR}\PYG{o}{=}\PYG{n}{xx} \PYG{n}{Default}\PYG{o}{=}\PYG{l+m+mi}{00}
\end{sphinxVerbatim}

\sphinxAtStartPar
\sphinxstylestrong{xx} \sphinxhyphen{} The 3270 AID function key which will be transmitted to the application when the user presses the {[}SUMMARY{]} key.

\sphinxAtStartPar
By default, the {[}SUMMARY{]} key is not transmitted to the application but serves to return the user to the tree structure. This parameter allows for the definition of a default which may be modified in the sub\sphinxhyphen{}server node definition. Where the value specified is a ‘01’, use of the {[}SUMMARY{]} key sets the cursor on the first field to be entered in the current screen.

\index{Virtel TCT@\spxentry{Virtel TCT}!STATDSN parameter@\spxentry{STATDSN parameter}}\index{STATDSN parameter@\spxentry{STATDSN parameter}!Virtel TCT@\spxentry{Virtel TCT}}\ignorespaces 

\subsection{STATDSN parameter}
\label{\detokenize{Installation_Guide:statdsn-parameter}}\label{\detokenize{Installation_Guide:index-117}}
\begin{sphinxVerbatim}[commandchars=\\\{\}]
\PYG{n}{STATDSN}\PYG{o}{=}\PYG{p}{(}\PYG{n}{dsn1}\PYG{p}{,}\PYG{n}{dsn2}\PYG{p}{,}\PYG{o}{.}\PYG{o}{.}\PYG{o}{.}\PYG{p}{)} \PYG{n}{Default}\PYG{o}{=}\PYG{n}{none}
\end{sphinxVerbatim}

\sphinxAtStartPar
\sphinxstylestrong{dsn1,…} \sphinxhyphen{} Dataset names of the files to be used for recording statistics if the parameter STATS=MULTI is specified. From 2 to 10 datasets can be specified. The datasets must be cataloged.

\index{Virtel TCT@\spxentry{Virtel TCT}!STATS parameter@\spxentry{STATS parameter}}\index{STATS parameter@\spxentry{STATS parameter}!Virtel TCT@\spxentry{Virtel TCT}}\ignorespaces 

\subsection{STATS parameter}
\label{\detokenize{Installation_Guide:stats-parameter}}\label{\detokenize{Installation_Guide:index-118}}
\begin{sphinxVerbatim}[commandchars=\\\{\}]
\PYG{n}{STATS}\PYG{o}{=}\PYG{n}{YES}\PYG{o}{/}\PYG{n}{NO}\PYG{o}{/}\PYG{p}{(}\PYG{n}{MULTI}\PYG{p}{,}\PYG{n}{CONTINUE}\PYG{o}{/}\PYG{n}{TERMINATE}\PYG{p}{)} \PYG{o}{|} \PYG{n}{SMF} \PYG{o}{|} \PYG{p}{(}\PYG{n}{SMF}\PYG{p}{,}\PYG{n}{nnn}\PYG{p}{)} \PYG{n}{Default}\PYG{o}{=}\PYG{n}{YES}
\end{sphinxVerbatim}

\sphinxAtStartPar
\sphinxstylestrong{YES} \sphinxhyphen{} Statistics recording is active. Statistics will be written to a single file defined in the VIRSTAT DD or DLBL statement in the VIRTEL started task JCL. VIRTEL must be stopped periodically to allow the statistics to be copied to a history file. The VIRSTAT file is overwritten each time VIRTEL is started.

\sphinxAtStartPar
\sphinxstylestrong{NO} \sphinxhyphen{} Statistics will not be recorded.

\sphinxAtStartPar
\sphinxstylestrong{MULTI} \sphinxhyphen{} Statistics recording is active. Statistics are written to one of the datasets defined in the STATDSN parameter of the VIRTCT. VIRTEL rotates the datasets by switching automatically to the next dataset when the current dataset becomes full. A batch job can then be used to copy the statistics to an archive file without stopping VIRTEL. Message VIR0603I can be used by an automated operator to trigger the submission of the batch job. After copying the statistics, the batch job must empty the dataset by writing an EOF marker at the beginning, which allows VIRTEL to reuse the dataset (see member STATCOPY in the VIRTEL SAMPLIB).
The VIRTEL STAT command allows the console operator to display the status of the VIRSTATx datasets, or to force VIRTEL to switch to the next statistics dataset.

\sphinxAtStartPar
If all of the statistics datasets are full, there are two options:

\sphinxAtStartPar
\sphinxstylestrong{STATS=(MULTI,TERMINATE)} \sphinxhyphen{} VIRTEL terminates, to avoid the possibility of losing any further statistics.

\sphinxAtStartPar
\sphinxstylestrong{STATS=(MULTI,CONTINUE)} \sphinxhyphen{} (default) VIRTEL continues, without recording any further statistics. To restart statistics recording, run a STATCOPY batch job to empty at least one VIRSTATx dataset, then issue the STAT,I command.

\sphinxAtStartPar
The STATS=MULTI option is only available in the z/OS environment.

\sphinxAtStartPar
\sphinxstylestrong{SMF.} Statistics recording is active and are written into SMF. The VIRTEL STAT command allows the console operator to display the status of the VIRSTATx datasets, or to force VIRTEL to switch to the next statistics dataset. The SMF record format is the same as the current STATS record but prefixed by the standard SMF header. The options are:

\sphinxAtStartPar
\sphinxstylestrong{STATS=SMF} \sphinxhyphen{} The default SMF record number is 223.

\sphinxAtStartPar
\sphinxstylestrong{STATS=(SMF,nnn)} \sphinxhyphen{} The SMF record number used will be nnn. The specified number must be between 128 and 255. The STATS=SMF/(SMF,nnn) option is only available in the MVS environment.

\index{Virtel TCT@\spxentry{Virtel TCT}!STRNO parameter@\spxentry{STRNO parameter}}\index{STRNO parameter@\spxentry{STRNO parameter}!Virtel TCT@\spxentry{Virtel TCT}}\ignorespaces 

\subsection{STRNO parameter}
\label{\detokenize{Installation_Guide:strno-parameter}}\label{\detokenize{Installation_Guide:index-119}}
\begin{sphinxVerbatim}[commandchars=\\\{\}]
\PYG{n}{STRNO}\PYG{o}{=}\PYG{n}{n} \PYG{n}{Default}\PYG{o}{=}\PYG{l+m+mi}{8}
\end{sphinxVerbatim}

\sphinxAtStartPar
\sphinxstylestrong{n} \sphinxhyphen{} Number of concurrent accesses to VSAM files.

\index{Virtel TCT@\spxentry{Virtel TCT}!SUITE parameter@\spxentry{SUITE parameter}}\index{SUITE parameter@\spxentry{SUITE parameter}!Virtel TCT@\spxentry{Virtel TCT}}\ignorespaces 

\subsection{SUITE parameter}
\label{\detokenize{Installation_Guide:suite-parameter}}\label{\detokenize{Installation_Guide:index-120}}
\begin{sphinxVerbatim}[commandchars=\\\{\}]
\PYG{n}{SUITE}\PYG{o}{=}\PYG{n}{xx} \PYG{n}{Default}\PYG{o}{=}\PYG{l+m+mi}{00}
\end{sphinxVerbatim}

\sphinxAtStartPar
\sphinxstylestrong{xx} \sphinxhyphen{} The 3270 AID function key which will be transmitted to the application when the user presses the {[}SUITE{]} function key. By default the {[}SUITE{]} function key is not transmitted to the application but serves to set the cursor to the following field. This parameter allows the definition of a general value by default that may be modified in the definition of the sub server node.

\index{Virtel TCT@\spxentry{Virtel TCT}!SWAP parameter@\spxentry{SWAP parameter}}\index{SWAP parameter@\spxentry{SWAP parameter}!Virtel TCT@\spxentry{Virtel TCT}}\ignorespaces 

\subsection{SWAP parameter}
\label{\detokenize{Installation_Guide:swap-parameter}}\label{\detokenize{Installation_Guide:index-121}}
\begin{sphinxVerbatim}[commandchars=\\\{\}]
\PYG{n}{SWAP}\PYG{o}{=}\PYG{n}{Pnn} \PYG{n}{Default}\PYG{o}{=}\PYG{n}{P24}
\end{sphinxVerbatim}

\sphinxAtStartPar
\sphinxstylestrong{Pnn} \sphinxhyphen{} Identifies the 3270 function key that causes VIRTEL to return to the multi\sphinxhyphen{}session menu (for SNA terminals, the ATTN key also performs this function). This parameter may take the following parameter values P1 to P24, PA1, PA2, or CLR.

\index{Virtel TCT@\spxentry{Virtel TCT}!SYSPLUS parameter@\spxentry{SYSPLUS parameter}}\index{SYSPLUS parameter@\spxentry{SYSPLUS parameter}!Virtel TCT@\spxentry{Virtel TCT}}\ignorespaces 

\subsection{SYSPLUS parameter}
\label{\detokenize{Installation_Guide:sysplus-parameter}}\label{\detokenize{Installation_Guide:index-122}}
\begin{sphinxVerbatim}[commandchars=\\\{\}]
\PYG{n}{SYSPLUS}\PYG{o}{=}\PYG{n}{YES}\PYG{o}{/}\PYG{n}{NO} \PYG{n}{Default}\PYG{o}{=}\PYG{n}{NO}
\end{sphinxVerbatim}

\sphinxAtStartPar
\sphinxstylestrong{YES} \sphinxhyphen{} VIRTEL will retrieve certain system symbols from z/OS. Whenever the ‘+’ character appears in the APPLID parameter or in a terminal relay name, VIRTEL will replace the ‘+’ by the value of the SYSCLONE symbol.
\sphinxstylestrong{NO} \sphinxhyphen{} System symbols will not be retrieved, the ‘+’ character will not be substituted in LU names, and the xxx\sphinxhyphen{}SYMBOL functionality of the NAME\sphinxhyphen{}OF tag and the COPY\$ SYSTEM\sphinxhyphen{}TO\sphinxhyphen{}VARIABLE instruction is not active (see VIRTEL Web Access Guide).

\index{Virtel TCT@\spxentry{Virtel TCT}!TCPn parameter@\spxentry{TCPn parameter}}\index{TCPn parameter@\spxentry{TCPn parameter}!Virtel TCT@\spxentry{Virtel TCT}}\ignorespaces 

\subsection{TCP1 parameter}
\label{\detokenize{Installation_Guide:tcp1-parameter}}\label{\detokenize{Installation_Guide:index-123}}
\begin{sphinxVerbatim}[commandchars=\\\{\}]
\PYG{n}{TCP1}\PYG{o}{=}\PYG{n}{tcpname} \PYG{n}{Default}\PYG{o}{=}\PYG{n}{no} \PYG{n}{TCP}\PYG{o}{/}\PYG{n}{IP} \PYG{n}{connection}
\PYG{n}{TCP1}\PYG{o}{=}\PYG{p}{(}\PYG{p}{[}\PYG{n}{tcpname}\PYG{p}{]}\PYG{p}{,}\PYG{p}{,}\PYG{p}{[}\PYG{n}{DNS}\PYG{p}{]}\PYG{p}{,}\PYG{p}{[}\PYG{n}{maxsock}\PYG{p}{]}\PYG{p}{,}\PYG{p}{[}\PYG{n}{pgmname}\PYG{p}{]}\PYG{p}{,}\PYG{p}{[}\PYG{n}{adsname}\PYG{p}{]}\PYG{p}{)}
\end{sphinxVerbatim}

\sphinxAtStartPar
This parameter defines the characteristics of the connection to the TCP/IP stack used by all lines which specify type TCP1.

\sphinxAtStartPar
\sphinxstylestrong{tcpname} \sphinxhyphen{} The name of the TCP/IP stack. This name should match the TCPIPJOBNAME parameter in the TCPIP.TCPIP.DATA file of the TCP/IP stack, or the name of the TCP/IP started task itself if TCPIPJOBNAME is not specified. The value ANY indicates that a connection can be established which any TCP/IP stack. This parameter is ignored by the TCP/IP for z/VSE stack.

\sphinxAtStartPar
\sphinxstylestrong{DNS} \sphinxhyphen{}  Start the DNS subtask VIRDNS1. This subtask supports the use of an asynchronous GETNAMEINFO function from within a scenario. See COPY\$ NAME\sphinxhyphen{}OF\sphinxhyphen{}TERMINAL in the Virtel Users Guide for further information.

\sphinxAtStartPar
\sphinxstylestrong{maxsock} \sphinxhyphen{} In z/OS, this is the maximum number of sockets for each type TCP1 line defined in VIRTEL. If this subparameter is not specified, TCP/IP determines the number (50 by default). The maximum value allowed by VIRTEL is 65535. However, for customers using older versions of z/OS (z/OS V1R4 or earlier), the TCP/IP stack enforces an upper limit of 2000 on this subparameter. Also, the value of the MAXFILEPROC parameter in PARMLIB member BPXPRMxx must exceed the maxsock value. In z/VSE, this is the total maximum number of sockets for all VIRTEL lines of type TCP1. The TCP/IP for z/VSE stack currently ignores the value specified here, and uses a fixed value of 8001 instead.

\sphinxAtStartPar
\sphinxstylestrong{pgmname} \sphinxhyphen{} The name of the VIRTEL TCP/IP interface program used for this connection. The following values can be specified:
\begin{quote}

\sphinxAtStartPar
\sphinxstylestrong{VIR0T09} \sphinxhyphen{} Interface program using ASYNC=EXIT mode. This is the default for z/OS systems.

\sphinxAtStartPar
\sphinxstylestrong{VIR0T10} \sphinxhyphen{} Interface program using ASYNC=ECB mode. This is the default for z/VSE systems.
\end{quote}

\sphinxAtStartPar
\sphinxstylestrong{adsname} \sphinxhyphen{} The name which VIRTEL uses to identify itself to TCP/IP. The value * indicates that VIRTEL uses its VTAM APPLID as the address space identifier. The default value is blank, which means that TCP/IP will assign the name of the VIRTEL started task as the address space identifier. This parameter is ignored by the TCP/IP for z/VSE stack.


\subsection{TCP2 parameter}
\label{\detokenize{Installation_Guide:tcp2-parameter}}
\begin{sphinxVerbatim}[commandchars=\\\{\}]
\PYG{n}{TCP2}\PYG{o}{=}\PYG{n}{tcpname} \PYG{n}{Default}\PYG{o}{=}\PYG{n}{no} \PYG{l+m+mi}{2}\PYG{n}{nd} \PYG{n}{TCP}\PYG{o}{/}\PYG{n}{IP} \PYG{n}{connection}
\PYG{n}{TCP2}\PYG{o}{=}\PYG{p}{(}\PYG{p}{[}\PYG{n}{tcpname}\PYG{p}{]}\PYG{p}{,}\PYG{p}{,}\PYG{p}{[}\PYG{n}{DNS}\PYG{p}{]}\PYG{p}{,}\PYG{p}{[}\PYG{n}{maxsock}\PYG{p}{]}\PYG{p}{,}\PYG{p}{[}\PYG{n}{pgmname}\PYG{p}{]}\PYG{p}{,}\PYG{p}{[}\PYG{n}{adsname}\PYG{p}{]}\PYG{p}{)}
\end{sphinxVerbatim}

\sphinxAtStartPar
This parameter defines the characteristics of the connection to the TCP/IP stack used by all lines which specify type TCP2. The subparameters are the same as those of TCP1.

\index{Virtel TCT@\spxentry{Virtel TCT}!TIMEOUT parameter@\spxentry{TIMEOUT parameter}}\index{TIMEOUT parameter@\spxentry{TIMEOUT parameter}!Virtel TCT@\spxentry{Virtel TCT}}\ignorespaces 

\subsection{TIMEOUT parameter}
\label{\detokenize{Installation_Guide:timeout-parameter}}\label{\detokenize{Installation_Guide:index-124}}
\begin{sphinxVerbatim}[commandchars=\\\{\}]
\PYG{n}{TIMEOUT}\PYG{o}{=}\PYG{n}{n} \PYG{n}{Default}\PYG{o}{=}\PYG{l+m+mi}{5}
\end{sphinxVerbatim}

\sphinxAtStartPar
\sphinxstylestrong{n} \sphinxhyphen{} Indicates in minutes the time\sphinxhyphen{}out after which a terminal connected to an external server will be force disconnected if no line activity is seen. A value of 0 means that the terminal will not be disconnected even if no activity is detected. The value specified here applies only when the “User time out” field in the external server definition is set to zero (see “Parameters of the external server” in the VIRTEL Connectivity Reference manual).

\index{Virtel TCT@\spxentry{Virtel TCT}!TIMERQS parameter@\spxentry{TIMERQS parameter}}\index{TIMERQS parameter@\spxentry{TIMERQS parameter}!Virtel TCT@\spxentry{Virtel TCT}}\ignorespaces 

\subsection{TIMERQS parameter}
\label{\detokenize{Installation_Guide:timerqs-parameter}}\label{\detokenize{Installation_Guide:index-125}}
\begin{sphinxVerbatim}[commandchars=\\\{\}]
\PYG{n}{TIMERQS}\PYG{o}{=}\PYG{p}{(}\PYG{n}{n1}\PYG{p}{,}\PYG{n}{n2}\PYG{p}{,}\PYG{n}{n3}\PYG{p}{,}\PYG{n}{n4}\PYG{p}{)} \PYG{n}{Default}\PYG{o}{=}\PYG{p}{(}\PYG{l+m+mi}{5}\PYG{p}{,}\PYG{l+m+mi}{10}\PYG{p}{,}\PYG{l+m+mi}{5}\PYG{p}{,}\PYG{l+m+mi}{0}\PYG{p}{)}
\end{sphinxVerbatim}

\sphinxAtStartPar
This parameter indicates the timeout values (in seconds) used by VIRTEL when attempting to establish an outbound connection using the Application\sphinxhyphen{}to\sphinxhyphen{}Application API (FA29 structured field). If this parameter is specified, then all four sub\sphinxhyphen{}parameters must be coded. The sub\sphinxhyphen{}parameters are:

\sphinxAtStartPar
\sphinxstylestrong{n1} \sphinxhyphen{} Timeout for VTAM connections.

\sphinxAtStartPar
\sphinxstylestrong{n2} \sphinxhyphen{} Timeout for X25 connections.

\sphinxAtStartPar
\sphinxstylestrong{n3} \sphinxhyphen{} Timeout for TCP/IP connections.

\sphinxAtStartPar
\sphinxstylestrong{n4} \sphinxhyphen{} Reserved for future use.

\index{Virtel TCT@\spxentry{Virtel TCT}!TITREn parameter@\spxentry{TITREn parameter}}\index{TITREn parameter@\spxentry{TITREn parameter}!Virtel TCT@\spxentry{Virtel TCT}}\ignorespaces 

\subsection{TITRE1 parameter}
\label{\detokenize{Installation_Guide:titre1-parameter}}\label{\detokenize{Installation_Guide:index-126}}
\begin{sphinxVerbatim}[commandchars=\\\{\}]
\PYG{n}{TITRE1}\PYG{o}{=}\PYG{l+s+s1}{\PYGZsq{}}\PYG{l+s+s1}{ccccc}\PYG{l+s+s1}{\PYGZsq{}} \PYG{n}{Default}\PYG{o}{=}\PYG{l+s+s1}{\PYGZsq{}}\PYG{l+s+s1}{SYSPERTEC}\PYG{l+s+s1}{\PYGZsq{}}
\end{sphinxVerbatim}

\sphinxAtStartPar
\sphinxstylestrong{ccccc} \sphinxhyphen{} The first line of the Multi\sphinxhyphen{}Session menu screen.


\subsection{TITRE2 parameter}
\label{\detokenize{Installation_Guide:titre2-parameter}}
\begin{sphinxVerbatim}[commandchars=\\\{\}]
\PYG{n}{TITRE2}\PYG{o}{=}\PYG{l+s+s1}{\PYGZsq{}}\PYG{l+s+s1}{ccccc}\PYG{l+s+s1}{\PYGZsq{}} \PYG{n}{Default}\PYG{o}{=}\PYG{l+s+s1}{\PYGZsq{}}\PYG{l+s+s1}{ }\PYG{l+s+s1}{\PYGZsq{}}
\end{sphinxVerbatim}

\sphinxAtStartPar
\sphinxstylestrong{ccccc} \sphinxhyphen{} The second line of the Multi\sphinxhyphen{}Session menu screen.

\index{Virtel TCT@\spxentry{Virtel TCT}!TRACALL parameter@\spxentry{TRACALL parameter}}\index{TRACALL parameter@\spxentry{TRACALL parameter}!Virtel TCT@\spxentry{Virtel TCT}}\ignorespaces 

\subsection{TRACALL parameter}
\label{\detokenize{Installation_Guide:tracall-parameter}}\label{\detokenize{Installation_Guide:index-127}}
\begin{sphinxVerbatim}[commandchars=\\\{\}]
\PYG{n}{TRACALL}\PYG{o}{=}\PYG{p}{(}\PYG{n}{p1}\PYG{p}{,}\PYG{n}{p2}\PYG{p}{,}\PYG{o}{.}\PYG{o}{.}\PYG{o}{.}\PYG{p}{)} \PYG{n}{Default}\PYG{o}{=}\PYG{n}{none}
\end{sphinxVerbatim}

\sphinxAtStartPar
\sphinxstylestrong{p1,p2,…} \sphinxhyphen{} Additional categories of trace data to be included in the VIRTEL internal trace. One or more of the following values may be coded in any order:

\sphinxAtStartPar
\sphinxstylestrong{HTTP} \sphinxhyphen{} Additional trace data for HTTP server

\sphinxAtStartPar
\sphinxstylestrong{VSAM} \sphinxhyphen{} Additional trace data for VSAM I/O requests

\sphinxAtStartPar
\sphinxstylestrong{XM} \sphinxhyphen{} Additional trace data for Cross\sphinxhyphen{}Memory communication

\index{Virtel TCT@\spxentry{Virtel TCT}!TRACBIG parameter@\spxentry{TRACBIG parameter}}\index{TRACBIG parameter@\spxentry{TRACBIG parameter}!Virtel TCT@\spxentry{Virtel TCT}}\ignorespaces 

\subsection{TRACBIG parameter}
\label{\detokenize{Installation_Guide:tracbig-parameter}}\label{\detokenize{Installation_Guide:index-128}}
\begin{sphinxVerbatim}[commandchars=\\\{\}]
\PYG{n}{TRACBIG}\PYG{o}{=}\PYG{n}{n} \PYG{n}{Default}\PYG{o}{=}\PYG{l+m+mi}{40}
\end{sphinxVerbatim}

\sphinxAtStartPar
\sphinxstylestrong{n} \sphinxhyphen{} The number of entries reserved for the VIRTEL internal trace. The value indicated corresponds to n times 256 entries.

\index{Virtel TCT@\spxentry{Virtel TCT}!TRACEB parameter@\spxentry{TRACEB parameter}}\index{TRACEB parameter@\spxentry{TRACEB parameter}!Virtel TCT@\spxentry{Virtel TCT}}\ignorespaces 

\subsection{TRACEB parameter}
\label{\detokenize{Installation_Guide:traceb-parameter}}\label{\detokenize{Installation_Guide:index-129}}
\begin{sphinxVerbatim}[commandchars=\\\{\}]
\PYG{n}{TRACEB}\PYG{o}{=}\PYG{n}{nn} \PYG{n}{Default}\PYG{o}{=}\PYG{l+m+mi}{200}
\end{sphinxVerbatim}

\sphinxAtStartPar
\sphinxstylestrong{nn} \sphinxhyphen{} The number of 1K buffers reserved for buffer data associated with entries in the VIRTEL internal trace. From VIRTEL 4.20 onwards, trace data is allocated above the 16MB line if possible.

\index{Virtel TCT@\spxentry{Virtel TCT}!TRACEOJ parameter@\spxentry{TRACEOJ parameter}}\index{TRACEOJ parameter@\spxentry{TRACEOJ parameter}!Virtel TCT@\spxentry{Virtel TCT}}\ignorespaces 

\subsection{TRACEOJ parameter}
\label{\detokenize{Installation_Guide:traceoj-parameter}}\label{\detokenize{Installation_Guide:index-130}}
\begin{sphinxVerbatim}[commandchars=\\\{\}]
\PYG{n}{TRACEOJ}\PYG{o}{=}\PYG{n}{STANDARD}\PYG{o}{/}\PYG{n}{YES}\PYG{o}{/}\PYG{n}{NO} \PYG{n}{Default}\PYG{o}{=}\PYG{n}{NO}
\end{sphinxVerbatim}

\sphinxAtStartPar
\sphinxstylestrong{STANDARD} \sphinxhyphen{} An automatic SNAP of the VIRTEL internal trace table will be produced at the start of VIRTEL termination.

\sphinxAtStartPar
\sphinxstylestrong{YES} \sphinxhyphen{} An automatic SNAP of the VIRTEL internal trace table will be produced at the end of VIRTEL termination.

\sphinxAtStartPar
\sphinxstylestrong{NO} \sphinxhyphen{} No SNAP at VIRTEL termination.

\index{Virtel TCT@\spxentry{Virtel TCT}!TRACEON parameter@\spxentry{TRACEON parameter}}\index{TRACEON parameter@\spxentry{TRACEON parameter}!Virtel TCT@\spxentry{Virtel TCT}}\ignorespaces 

\subsection{TRACEON parameter}
\label{\detokenize{Installation_Guide:traceon-parameter}}\label{\detokenize{Installation_Guide:index-131}}
\begin{sphinxVerbatim}[commandchars=\\\{\}]
TRACEON= ON | OFF | (Y|N,Y|N,Y|N)

Default = ON    Equivalent VIT = YYN    Tracing ON

OFF             Equivalent VIT = NNN    Tracing OFF (Not recommended)

TRACEON=(n,n,n)         n = Y|N         Set Tracing options

TRACEON=(N,N,N)                 Tracing OFF
TRACEON=(Y,N,N)                 Minimal tracing, no data elements
TRACEON=(Y,Y,N)                 Full tracing with data, no archive [Default]
TRACEON=(Y,Y,Y)                 Full tracing with data and archive

Command Option:
The VIT tracing categories can be set through the F VIRTEL,TRACE command

F VIRTEL,TRACE,VIT=nnn                  nnn correspond to the three Y|N indicators.

Example:

F VIRTEL,TRACE,VIT=YYY                  Turn on full VIT tracing plus external buffer archive.

Ability to offload external trace buffers to a dataset.
With the external VIT trace facility comes the ability to offload the trace buffers to a dataset. This offload capability can be triggered when the maximum number of external trace buffers have been reached, as identified in message VIR0208I, or through an operator command:

F VIRTEL,TRACE,VIT=OFFLOAD.

Note: IF VIT is not equal to YYY then you will receive the message “VIR0068E INVALID COMMAND”. Offloading the VIT only applies to the external VIT data store.

Setting up for trace \PYGZdq{}OFFLOAD\PYGZdq{}.
The trace buffers are offloaded to a GDG dataset which means historical trace data can be kept. To set up the GDG see the below. This job can also be found in the SAMPLIB dataset as member DEFTRGDG.

//*
//* DEFINE THE TRACE GDG DATASET
//*
//DELETE   EXEC PGM=IDCAMS
//SYSPRINT DD SYSOUT=*
 DELETE VIRTEL.TRACE.GDG GDG
 DELETE VIRTEL.TRACE.GDG.DSCB NVSAM
 SET MAXCC=0
//ALLOC1   EXEC PGM=IEFBR14
//FILE     DD DSN=VIRTEL.TRACE.GDG.DSCB,
//            UNIT=3390,DISP=(NEW,CATLG),
//            SPACE=(TRK,(0,0)),VOL=SER=VVVVVV,
//            DCB=BLKSIZE=13300
//*
//ALLOC2   EXEC PGM=IDCAMS
//SYSPRINT  DD SYSOUT=*
//SYSIN     DD *
 DEF GDG(NAME(VIRTEL.TRACE.GDG) LIMIT(5) SCRATCH NOEMPTY)
/*
\end{sphinxVerbatim}

\index{Virtel TCT@\spxentry{Virtel TCT}!TRACTIM parameter@\spxentry{TRACTIM parameter}}\index{TRACTIM parameter@\spxentry{TRACTIM parameter}!Virtel TCT@\spxentry{Virtel TCT}}\ignorespaces 

\subsection{TRACTIM parameter}
\label{\detokenize{Installation_Guide:tractim-parameter}}\label{\detokenize{Installation_Guide:index-132}}
\begin{sphinxVerbatim}[commandchars=\\\{\}]
\PYG{n}{TRACTIM}\PYG{o}{=}\PYG{n}{CPU}\PYG{o}{/}\PYG{n}{LOCAL}\PYG{o}{/}\PYG{n}{TOD} \PYG{n}{Default}\PYG{o}{=}\PYG{n}{LOCAL}
\end{sphinxVerbatim}

\sphinxAtStartPar
VIRTEL uses the TOD clock to timestamp each entry in its internal trace table. This parameter specifies whether or not the SNAP command should adjust the timestamps to match the local time used in the system message log. Possible values are:

\sphinxAtStartPar
\sphinxstylestrong{CPU} \sphinxhyphen{} The last column in the SNAP trace, instead of the local time or TOD time in 1/10000 of a second, contains the total used CPU time for the current TCB in 1/10000 of a second.

\begin{sphinxadmonition}{note}{Note:}
\sphinxAtStartPar
CPU option only works on z/OS and on a machine having support for the ECTG (Extract CPU Time) instruction (i.e. Z9\sphinxhyphen{}109 and above).
\end{sphinxadmonition}

\sphinxAtStartPar
\sphinxstylestrong{LOCAL} \sphinxhyphen{} The SNAP command adjusts the timestamps in the internal trace table so that they display as local time. This is the recommended setting.

\sphinxAtStartPar
\sphinxstylestrong{TOD} \sphinxhyphen{} Timestamps are not adjusted for local time.

\index{Virtel TCT@\spxentry{Virtel TCT}!TRAN parameter@\spxentry{TRAN parameter}}\index{TRAN parameter@\spxentry{TRAN parameter}!Virtel TCT@\spxentry{Virtel TCT}}\ignorespaces 

\subsection{TRAN parameter}
\label{\detokenize{Installation_Guide:tran-parameter}}\label{\detokenize{Installation_Guide:index-133}}
\begin{sphinxVerbatim}[commandchars=\\\{\}]
\PYG{n}{TRAN}\PYG{o}{=}\PYG{n}{EVEN}\PYG{o}{/}\PYG{n}{ODD}\PYG{o}{/}\PYG{n}{NO} \PYG{n}{Default}\PYG{o}{=}\PYG{n}{Even}
\end{sphinxVerbatim}

\sphinxAtStartPar
This parameter should be coded in the same way as for the X25MCH macro in NPSI.

\index{Virtel TCT@\spxentry{Virtel TCT}!UFIELnn parameter@\spxentry{UFIELnn parameter}}\index{UFIELnn parameter@\spxentry{UFIELnn parameter}!Virtel TCT@\spxentry{Virtel TCT}}\ignorespaces 

\subsection{UFILE1 to UFILE20 parameters}
\label{\detokenize{Installation_Guide:ufile1-to-ufile20-parameters}}\label{\detokenize{Installation_Guide:v461ig-ufile1-to-ufile20-parameters}}\label{\detokenize{Installation_Guide:index-134}}
\begin{sphinxVerbatim}[commandchars=\\\{\}]
\PYG{n}{UFILEx}\PYG{o}{=}\PYG{p}{(}\PYG{n}{ddname}\PYG{p}{,}\PYG{n}{acbcard}\PYG{p}{,}\PYG{n}{rkp}\PYG{p}{,}\PYG{n}{keylen}\PYG{p}{,}\PYG{n}{mode}\PYG{p}{)} \PYG{n}{Default}\PYG{o}{=}\PYG{l+s+s1}{\PYGZsq{}}\PYG{l+s+s1}{ }\PYG{l+s+s1}{\PYGZsq{}}
\end{sphinxVerbatim}

\sphinxAtStartPar
These parameters define the VSAM files used by VIRTEL for HTML directories. Each parameter has the form UFILEx = (ddname, acbcard, rkp, keylen, mode) where:
\begin{itemize}
\item {} 
\sphinxAtStartPar
ddname is the DD name of the file as specified in the VIRTEL start procedure.

\item {} 
\sphinxAtStartPar
acbcard is the name of the ACB macro defining the access characteristics of the transfer file. This ACB macro must appear later in the VIRTCT (see “Additional parameters for VSAM files”, page 78).

\item {} 
\sphinxAtStartPar
rkp represents the relative position of the key in hexadecimal. This value must match the value specified in the DEFINE CLUSTER.

\item {} 
\sphinxAtStartPar
keylen represents the length of the key in hexadecimal. This value must match the value specified in the DEFINE CLUSTER.

\item {} 
\sphinxAtStartPar
mode represents one of the following values (in hexadecimal):
\begin{quote}

\sphinxAtStartPar
\sphinxstylestrong{00} \sphinxhyphen{} read\sphinxhyphen{}only mode
\sphinxstylestrong{01} \sphinxhyphen{} read/write mode
\sphinxstylestrong{05} \sphinxhyphen{} read\sphinxhyphen{}only mode or read/write mode, depending on the value of the “VSAMTYP parameter”. Seen the VIRTCT.
\end{quote}

\end{itemize}

\sphinxAtStartPar
The UFILEx parameters must be defined in sequence with no intervening gaps in the suffix number x.

\index{Virtel TCT@\spxentry{Virtel TCT}!UPARMS parameter@\spxentry{UPARMS parameter}}\index{UPARMS parameter@\spxentry{UPARMS parameter}!Virtel TCT@\spxentry{Virtel TCT}}\ignorespaces 

\subsection{UPARMS parameter}
\label{\detokenize{Installation_Guide:uparms-parameter}}\label{\detokenize{Installation_Guide:index-135}}
\sphinxAtStartPar
The keyword enables the centralized saving of user settings on the host. Based upon a userid key, user settings can be maintained in a mainframe repository identified bu the name specified in the UPARMS parameter. For example UPARMS=(USERPARM). Here, the value USERPARM would be a Virtel transaction that identfied the repository. A sub\sphinxhyphen{}directory definition of the same name is also required to aasociate the repositry with a physical VSAM file. See the sample definitions below. The ARBOLOAD JCL can be used to load sample definitions into the ARBO through the USERPARM=YES option.

\begin{sphinxVerbatim}[commandchars=\\\{\}]
\PYG{n}{UPARMS}\PYG{o}{=}\PYG{p}{(}\PYG{n}{name}\PYG{p}{)}
\end{sphinxVerbatim}

\sphinxAtStartPar
\sphinxstylestrong{name} must be a value name supported by a directory and transaction definition.

\sphinxAtStartPar
Sample ARBO definitions to support the USERPARM feature:\sphinxhyphen{}

\begin{sphinxVerbatim}[commandchars=\\\{\}]
\PYG{n}{SUBDIR}   \PYG{n}{ID}\PYG{o}{=}\PYG{n}{USERPARM}\PYG{p}{,}              \PYG{o}{/}\PYG{o}{*} \PYG{n}{Specified} \PYG{o+ow}{in} \PYG{n}{UPARMS} \PYG{n}{TCT} \PYG{n}{keyword}  \PYG{o}{*}\PYG{o}{/}
 \PYG{n}{DESC}\PYG{o}{=}\PYG{l+s+s1}{\PYGZsq{}}\PYG{l+s+s1}{USERPARM directory}\PYG{l+s+s1}{\PYGZsq{}}\PYG{p}{,}
 \PYG{n}{DDNAME}\PYG{o}{=}\PYG{n}{USERTRSF}\PYG{p}{,}              \PYG{o}{/}\PYG{o}{*} \PYG{n}{Note}\PYG{o}{.} \PYG{n}{Could} \PYG{n}{point} \PYG{n}{to} \PYG{n}{HTMLTRSF} \PYG{o}{*}\PYG{o}{/}
 \PYG{n}{KEY}\PYG{o}{=}\PYG{n}{UPRMS}\PYG{p}{,}
 \PYG{n}{NAMELEN}\PYG{o}{=}\PYG{l+m+mi}{0064}\PYG{p}{,}
 \PYG{n}{AUTHUP}\PYG{o}{=}\PYG{n}{X}\PYG{p}{,}
 \PYG{n}{AUTHDOWN}\PYG{o}{=}\PYG{n}{X}\PYG{p}{,}
 \PYG{n}{AUTHDEL}\PYG{o}{=}\PYG{n}{X}

\PYG{n}{TRANSACT} \PYG{n}{ID}\PYG{o}{=}\PYG{n}{XXX}\PYG{o}{\PYGZhy{}}\PYG{l+m+mi}{00}\PYG{n}{P}\PYG{p}{,}                 \PYG{o}{/}\PYG{o}{*} \PYG{n}{Required} \PYG{k}{for} \PYG{n}{each} \PYG{n}{EP} \PYG{o}{\PYGZhy{}} \PYG{n}{W2H}\PYG{p}{,} \PYG{n}{CLI}  \PYG{o}{*}\PYG{o}{/}
 \PYG{n}{DESC}\PYG{o}{=}\PYG{l+s+s1}{\PYGZsq{}}\PYG{l+s+s1}{User Parameters directory}\PYG{l+s+s1}{\PYGZsq{}}\PYG{p}{,}
 \PYG{n}{NAME}\PYG{o}{=}\PYG{n}{USERPARM}\PYG{p}{,}                  \PYG{o}{/}\PYG{o}{*} \PYG{n}{Specified} \PYG{o+ow}{in} \PYG{n}{UPARMS} \PYG{n}{TCT} \PYG{n}{keyword}  \PYG{o}{*}\PYG{o}{/}
 \PYG{n}{APPL}\PYG{o}{=}\PYG{n}{USERPARM}\PYG{p}{,}                  \PYG{o}{/}\PYG{o}{*} \PYG{n}{Specified} \PYG{o+ow}{in} \PYG{n}{UPARMS} \PYG{n}{TCT} \PYG{n}{keyword}  \PYG{o}{*}\PYG{o}{/}
 \PYG{n}{TYPE}\PYG{o}{=}\PYG{l+m+mi}{4}\PYG{p}{,}
 \PYG{n}{TERMINAL}\PYG{o}{=}\PYG{n}{DELOC}
\end{sphinxVerbatim}

\index{Virtel TCT@\spxentry{Virtel TCT}!VIRSECU parameter@\spxentry{VIRSECU parameter}}\index{VIRSECU parameter@\spxentry{VIRSECU parameter}!Virtel TCT@\spxentry{Virtel TCT}}\ignorespaces 

\subsection{VIRSECU parameter}
\label{\detokenize{Installation_Guide:virsecu-parameter}}\label{\detokenize{Installation_Guide:index-136}}
\begin{sphinxVerbatim}[commandchars=\\\{\}]
\PYG{n}{VIRSECU}\PYG{o}{=}\PYG{n}{YES}\PYG{o}{/}\PYG{n}{NO} \PYG{n}{Default}\PYG{o}{=}\PYG{n}{NO}
\end{sphinxVerbatim}

\sphinxAtStartPar
\sphinxstylestrong{YES} \sphinxhyphen{} The VIRTEL internal security sub\sphinxhyphen{}application is available. To use VIRTEL security, specify VIRSECU=YES with SECUR=VIRTEL. The combination SECUR=NO, VIRSECU=YES allows online definition of Virtel security without security being active.

\sphinxAtStartPar
\sphinxstylestrong{NO} \sphinxhyphen{} VIRTEL internal security is not available.

\index{Virtel TCT@\spxentry{Virtel TCT}!VIRSV1 parameter@\spxentry{VIRSV1 parameter}}\index{VIRSV1 parameter@\spxentry{VIRSV1 parameter}!Virtel TCT@\spxentry{Virtel TCT}}\ignorespaces 

\subsection{VIRSV1 parameter}
\label{\detokenize{Installation_Guide:virsv1-parameter}}\label{\detokenize{Installation_Guide:index-137}}
\begin{sphinxVerbatim}[commandchars=\\\{\}]
\PYG{n}{VIRSV1}\PYG{o}{=}\PYG{p}{(}\PYG{n}{vsvname}\PYG{p}{)} \PYG{n}{Default}\PYG{o}{=}\PYG{n}{none}
\end{sphinxVerbatim}

\sphinxAtStartPar
This parameter defines the characteristics of the interface to the VIRSV service request manager for service programs called from a scenario via the VIRSV\$ instruction.

\sphinxAtStartPar
\sphinxstylestrong{vsvname} \sphinxhyphen{} Name of the service request manager. Must be VIRSV.

\index{Virtel TCT@\spxentry{Virtel TCT}!VSAMTYP parameter@\spxentry{VSAMTYP parameter}}\index{VSAMTYP parameter@\spxentry{VSAMTYP parameter}!Virtel TCT@\spxentry{Virtel TCT}}\ignorespaces 

\subsection{VSAMTYP parameter}
\label{\detokenize{Installation_Guide:vsamtyp-parameter}}\label{\detokenize{Installation_Guide:index-138}}
\begin{sphinxVerbatim}[commandchars=\\\{\}]
\PYG{n}{VSAMTYP}\PYG{o}{=}\PYG{p}{[}\PYG{n}{READONLY}\PYG{o}{|}\PYG{n}{NORMAL}\PYG{o}{|}\PYG{n}{WRITER}\PYG{p}{]} \PYG{n}{Default}\PYG{o}{=}\PYG{n}{NORMAL}
\end{sphinxVerbatim}

\sphinxAtStartPar
\sphinxstylestrong{READONLY} \sphinxhyphen{} This parameter, if specified in the VIRTCT, allows the VIRTEL started task to be run in read\sphinxhyphen{}only mode for VSAM files,to be used in production mode, especially in a sysplex environment. Except for the VIRSWAP work file, all other VSAM files used by VIRTEL can be opened in read\sphinxhyphen{}only mode.

\sphinxAtStartPar
\sphinxstylestrong{WRITER} \sphinxhyphen{} This parameter, if specified in the VIRTCT, allows updates in a Virplex without necessitating the stop of other, read\sphinxhyphen{}only, Virtel instances.

\sphinxAtStartPar
\sphinxstylestrong{NORMAL} \sphinxhyphen{} By default VIRTEL opens files in read/write mode to allow the possibility of updating certain VSAM files, such as the VIRARBO file for example.

\sphinxAtStartPar
VSAMTYP=READONLY takes effect only if the appropriate values have been specified in the MACRF parameter of the ACB (see {\hyperref[\detokenize{Installation_Guide:v461ig-additional-parameters-for-vsam-files}]{\sphinxcrossref{\DUrole{std,std-ref}{“Additional parameters for VSAM files”,}}}}): and in the MODE subparameter of the UFILEx parameter of the VIRTCT (see {\hyperref[\detokenize{Installation_Guide:v461ig-ufile1-to-ufile20-parameters}]{\sphinxcrossref{\DUrole{std,std-ref}{“UFILE1 to UFILE20”,}}}}).

\begin{sphinxadmonition}{note}{Note:}
\sphinxAtStartPar
VSAMTYP=WRITER should be used to suppport the Virplex writer instance of Virtel. See the Virplex section in the Virtel Connectivity Guide.
\end{sphinxadmonition}

\index{Virtel TCT@\spxentry{Virtel TCT}!VTKEYS parameter@\spxentry{VTKEYS parameter}}\index{VTKEYS parameter@\spxentry{VTKEYS parameter}!Virtel TCT@\spxentry{Virtel TCT}}\ignorespaces 

\subsection{VTKEYS parameter}
\label{\detokenize{Installation_Guide:vtkeys-parameter}}\label{\detokenize{Installation_Guide:index-139}}
\begin{sphinxVerbatim}[commandchars=\\\{\}]
\PYG{n}{VTKEYS}\PYG{o}{=}\PYG{n}{xxxxxxxx} \PYG{n}{Default}\PYG{o}{=}\PYG{l+m+mi}{0}
\end{sphinxVerbatim}

\sphinxAtStartPar
\sphinxstylestrong{xxxxxxxx} \sphinxhyphen{} The name of a table added to the end of the VIRTCT allowing for redefinition of the function keys for VT100. Please refer to the member VTSAMPLE in SAMPLIB.

\index{Virtel TCT@\spxentry{Virtel TCT}!VTOVER parameter@\spxentry{VTOVER parameter}}\index{VTOVER parameter@\spxentry{VTOVER parameter}!Virtel TCT@\spxentry{Virtel TCT}}\ignorespaces 

\subsection{VTOVER parameter}
\label{\detokenize{Installation_Guide:index-140}}\label{\detokenize{Installation_Guide:id7}}
\begin{sphinxVerbatim}[commandchars=\\\{\}]
\PYG{n}{VTOVER}\PYG{o}{=}\PYG{n}{xxxxxxxx} \PYG{n}{Default}\PYG{o}{=}\PYG{n}{none}
\end{sphinxVerbatim}

\sphinxAtStartPar
\sphinxstylestrong{xxxxxxxx} \sphinxhyphen{} The name of a table added to the end of the VIRTCT allowing for dynamic override of certain parameters in the VIRTCT. Please refer to the section “Dynamic VIRTCT overrides”, page 81 for further details.

\index{Virtel TCT@\spxentry{Virtel TCT}!VWAVERS parameter@\spxentry{VWAVERS parameter}}\index{VWAVERS parameter@\spxentry{VWAVERS parameter}!Virtel TCT@\spxentry{Virtel TCT}}\ignorespaces 
\def\sphinxLiteralBlockLabel{\label{\detokenize{Installation_Guide:index-141}}}
\begin{sphinxVerbatim}[commandchars=\\\{\}]
\PYG{n}{VWAVERS}\PYG{o}{=}\PYG{l+s+s1}{\PYGZsq{}}\PYG{l+s+s1}{xxxxxxxxxxxx}\PYG{l+s+s1}{\PYGZsq{}}
\end{sphinxVerbatim}

\sphinxAtStartPar
Max of 12 characters. This text will replace the ‘VIRTEL \&VERS’ message in the HTTP 40x messages. It acts as security feature to hide the identity of Virtel as the issuing program. For example: \sphinxhyphen{}

\begin{sphinxVerbatim}[commandchars=\\\{\}]
\PYG{n}{VWAVERS}\PYG{o}{=}\PYG{l+s+s1}{\PYGZsq{}}\PYG{l+s+s1}{Apache V99}\PYG{l+s+s1}{\PYGZsq{}}
\end{sphinxVerbatim}

\index{Virtel TCT@\spxentry{Virtel TCT}!WARNING parameter@\spxentry{WARNING parameter}}\index{WARNING parameter@\spxentry{WARNING parameter}!Virtel TCT@\spxentry{Virtel TCT}}\ignorespaces 

\subsection{WARNING parameter}
\label{\detokenize{Installation_Guide:warning-parameter}}\label{\detokenize{Installation_Guide:index-142}}
\begin{sphinxVerbatim}[commandchars=\\\{\}]
\PYG{n}{WARNING}\PYG{o}{=}\PYG{n}{nn}\PYG{p}{,}
\end{sphinxVerbatim}

\sphinxAtStartPar
Where nn is the number of days prior to issuing a licence warning message. If not specified, no warning is given. Virtel may shutdown on the next restart if the licence key has expired.

\index{Virtel TCT@\spxentry{Virtel TCT}!XMn parameter@\spxentry{XMn parameter}}\index{XMn parameter@\spxentry{XMn parameter}!Virtel TCT@\spxentry{Virtel TCT}}\ignorespaces 

\subsection{XM1 parameter}
\label{\detokenize{Installation_Guide:xm1-parameter}}\label{\detokenize{Installation_Guide:index-143}}
\begin{sphinxVerbatim}[commandchars=\\\{\}]
\PYG{n}{XM1}\PYG{o}{=}\PYG{n}{xmname} \PYG{n}{Default}\PYG{o}{=}\PYG{n}{no} \PYG{n}{XM} \PYG{n}{connection}
\PYG{n}{XM1}\PYG{o}{=}\PYG{p}{(}\PYG{n}{xmname}\PYG{p}{,}\PYG{p}{,}\PYG{p}{,}\PYG{p}{,}\PYG{p}{[}\PYG{n}{pgmname}\PYG{p}{]}\PYG{p}{)}
\end{sphinxVerbatim}

\sphinxAtStartPar
This parameter defines the characteristics of the connection to the cross\sphinxhyphen{}memory manager (VIRXM) used by all lines which specify type XM1.

\begin{sphinxadmonition}{note}{Note:}
\sphinxAtStartPar
Cross\sphinxhyphen{}memory is supported only on z/OS systems (OS/390 or z/OS). The VIRXM product from Syspertec must also be installed.
\end{sphinxadmonition}

\sphinxAtStartPar
\sphinxstylestrong{xmname} \sphinxhyphen{} The name of the cross\sphinxhyphen{}memory manager started task (VIRXM).

\sphinxAtStartPar
\sphinxstylestrong{pgmname} \sphinxhyphen{} The name of the VIRTEL XM interface program used for this connection. The following values can be specified:
\begin{quote}

\sphinxAtStartPar
\sphinxstylestrong{VIR0X09} \sphinxhyphen{} Interface program for z/OS systems. This is the default.
\end{quote}


\subsection{XM2 parameter}
\label{\detokenize{Installation_Guide:xm2-parameter}}
\begin{sphinxVerbatim}[commandchars=\\\{\}]
\PYG{n}{XM2}\PYG{o}{=}\PYG{n}{xmname} \PYG{n}{Default}\PYG{o}{=}\PYG{n}{no} \PYG{l+m+mi}{2}\PYG{n}{nd} \PYG{n}{XM} \PYG{n}{connection}
\PYG{n}{XM2}\PYG{o}{=}\PYG{p}{(}\PYG{p}{[}\PYG{n}{xmname}\PYG{p}{]}\PYG{p}{,}\PYG{p}{,}\PYG{p}{,}\PYG{p}{,}\PYG{p}{[}\PYG{n}{pgmname}\PYG{p}{]}\PYG{p}{)}
\end{sphinxVerbatim}

\sphinxAtStartPar
This parameter defines the characteristics of the connection to the cross\sphinxhyphen{}memory manager (VIRXM) used by all lines which specify type XM2. The subparameters are the same as those of the XM1 parameter.

\index{Virtel TCT@\spxentry{Virtel TCT}!ZAPH parameter@\spxentry{ZAPH parameter}}\index{ZAPH parameter@\spxentry{ZAPH parameter}!Virtel TCT@\spxentry{Virtel TCT}}\ignorespaces 

\subsection{ZAPH parameter}
\label{\detokenize{Installation_Guide:zaph-parameter}}\label{\detokenize{Installation_Guide:index-144}}
\begin{sphinxVerbatim}[commandchars=\\\{\}]
\PYG{n}{ZAPH}\PYG{o}{=}\PYG{n}{xxxxxxxx} \PYG{n}{Default}\PYG{o}{=}\PYG{n}{none}
\end{sphinxVerbatim}

\sphinxAtStartPar
\sphinxstylestrong{xxxxxxxx} \sphinxhyphen{} The name of a table added to the end of the VIRTCT allowing for one or more patches to be applied at startup. Please refer to the section “Applying patches via the VIRTCT”, page 82 for further details.

\newpage

\index{VSAM Files@\spxentry{VSAM Files}!Additional Parameters@\spxentry{Additional Parameters}}\index{Additional Parameters@\spxentry{Additional Parameters}!VSAM Files@\spxentry{VSAM Files}}\ignorespaces 

\section{Additional Parameters For VSAM Files}
\label{\detokenize{Installation_Guide:additional-parameters-for-vsam-files}}\label{\detokenize{Installation_Guide:v461ig-additional-parameters-for-vsam-files}}\label{\detokenize{Installation_Guide:index-145}}
\sphinxAtStartPar
VIRTEL uses VSAM files for storing HTML pages and for VIRTEL/PC file transfer. These VSAM files must be defined in the VIRTCT by means of a parameter UFILEx and an ACBHx macro for each file. The ACB macros must be coded in the VIRTCT before the END card. The formats of these macros are as follows:

\begin{sphinxVerbatim}[commandchars=\\\{\}]
ACBHx ACB AM=VSAM,DDNAME=dddddddd,MACRF=(P1,P2,…....,Pn),STRNO=3
\end{sphinxVerbatim}
\begin{itemize}
\item {} 
\sphinxAtStartPar
The label ACBHx must match the acbname subparameter as specified in the UFILEx parameter of the VIRTCT.

\item {} 
\sphinxAtStartPar
dddddddd represents the ddname of the file as it is known in the VIRTEL start up procedure.

\item {} 
\sphinxAtStartPar
pn represents the authorisations granted to the transfer file. The permitted values are: SEQ,DIR,OUT,LSR (for read/write mode) or SEQ,DIR,IN,LSR (for read\sphinxhyphen{}only mode).

\item {} 
\sphinxAtStartPar
The value OUT should be omitted from MACRF if you want the mode (read\sphinxhyphen{}only or read/write) to be determined by the value of the VSAMTYP parameter in the VIRTCT (see “VSAMTYP parameter”, page 76).

\end{itemize}

\newpage
\phantomsection\label{\detokenize{Installation_Guide:vvrrig-bookmark72}}
\index{Batch@\spxentry{Batch}!Additional Parameters@\spxentry{Additional Parameters}}\index{Additional Parameters@\spxentry{Additional Parameters}!Batch@\spxentry{Batch}}\ignorespaces 

\section{Additional Parameters For Batch Files}
\label{\detokenize{Installation_Guide:additional-parameters-for-batch-files}}\label{\detokenize{Installation_Guide:index-146}}
\sphinxAtStartPar
VIRTEL uses sequential files for batch input and output when the BATCHx parameter is defined in the VIRTCT, and a batch line is present in the configuration. These sequential files must be defined in the VIRTCT by means of a pair of DCB/DCBE macros for each file. The DCB/DCBE macros must be coded in the VIRTCT before the END card. The formats of these macros are as follows:

\begin{sphinxVerbatim}[commandchars=\\\{\}]
\PYG{n}{label}  \PYG{n}{DCB} \PYG{n}{DDNAME}\PYG{o}{=}\PYG{n}{ddname}\PYG{p}{,}\PYG{n}{DCBE}\PYG{o}{=}\PYG{n}{labelx}\PYG{p}{,}\PYG{n}{DSORG}\PYG{o}{=}\PYG{n}{PS}\PYG{p}{,}                          \PYG{o}{*}
          \PYG{n}{LRECL}\PYG{o}{=}\PYG{n}{lrecl}\PYG{p}{,}\PYG{n}{RECFM}\PYG{o}{=}\PYG{n}{recfm}\PYG{p}{,}\PYG{n}{MACRF}\PYG{o}{=}\PYG{p}{(}\PYG{n}{macrf}\PYG{p}{)}
\PYG{n}{labelx} \PYG{n}{DCBE} \PYG{n}{EODAD}\PYG{o}{=}\PYG{l+m+mi}{0}\PYG{p}{,}\PYG{n}{RMODE31}\PYG{o}{=}\PYG{n}{BUFF}
\end{sphinxVerbatim}

\sphinxAtStartPar
where:

\sphinxAtStartPar
\sphinxstylestrong{label} \sphinxhyphen{} corresponds to the DCB label specified in the BATCHx parameter

\sphinxAtStartPar
\sphinxstylestrong{ddname} \sphinxhyphen{} corresponds to the DD name specified in the BATCHx parameter.

\sphinxAtStartPar
For input files:
\begin{itemize}
\item {} 
\sphinxAtStartPar
lrecl is 80, recfm is FB, macrf is GL.

\end{itemize}

\sphinxAtStartPar
For output files:
\begin{itemize}
\item {} 
\sphinxAtStartPar
lrecl is 133, recfm is FBA, macrf is PM.

\end{itemize}

\sphinxAtStartPar
\sphinxstylestrong{EODAD=0} \sphinxhyphen{} should be specified for input files.

\sphinxAtStartPar
The example below shows how to code DCB/DCBE macros when the BATCH1 parameter is specified as:

\begin{sphinxVerbatim}[commandchars=\\\{\}]
      \PYG{n}{BATCH1}\PYG{o}{=}\PYG{p}{(}\PYG{n}{SYSIN1}\PYG{p}{,}\PYG{n}{DCBI1}\PYG{p}{,}\PYG{n}{SYSOUT1}\PYG{p}{,}\PYG{n}{DCBO1}\PYG{p}{)}

\PYG{n}{DCBI1}   \PYG{n}{DCB} \PYG{n}{DDNAME}\PYG{o}{=}\PYG{n}{SYSIN1}\PYG{p}{,} \PYG{n}{SYSIN} \PYG{n}{DD}                       \PYG{o}{*}
                  \PYG{n}{DCBE}\PYG{o}{=}\PYG{n}{DCBI1X}\PYG{p}{,}                                  \PYG{o}{*}
                  \PYG{n}{LRECL}\PYG{o}{=}\PYG{l+m+mi}{80}\PYG{p}{,}                                     \PYG{o}{*}
                  \PYG{n}{DSORG}\PYG{o}{=}\PYG{n}{PS}\PYG{p}{,}                                     \PYG{o}{*}
                  \PYG{n}{RECFM}\PYG{o}{=}\PYG{n}{FB}\PYG{p}{,}                                     \PYG{o}{*}
                  \PYG{n}{MACRF}\PYG{o}{=}\PYG{p}{(}\PYG{n}{GL}\PYG{p}{)}
      \PYG{n}{DCBI1X}  \PYG{n}{DCBE} \PYG{n}{EODAD}\PYG{o}{=}\PYG{l+m+mi}{0}\PYG{p}{,}\PYG{n}{RMODE31}\PYG{o}{=}\PYG{n}{BUFF}
      \PYG{n}{DCBO1}   \PYG{n}{DCB} \PYG{n}{DDNAME}\PYG{o}{=}\PYG{n}{SYSOUT1}\PYG{p}{,} \PYG{n}{SYSPRINT} \PYG{n}{DD}                   \PYG{o}{*}
                  \PYG{n}{DCBE}\PYG{o}{=}\PYG{n}{DCBO1X}\PYG{p}{,}                                  \PYG{o}{*}
                  \PYG{n}{LRECL}\PYG{o}{=}\PYG{l+m+mi}{133}\PYG{p}{,}                                    \PYG{o}{*}
                  \PYG{n}{DSORG}\PYG{o}{=}\PYG{n}{PS}\PYG{p}{,}                                     \PYG{o}{*}
                  \PYG{n}{RECFM}\PYG{o}{=}\PYG{n}{FBA}\PYG{p}{,}                                    \PYG{o}{*}
                  \PYG{n}{MACRF}\PYG{o}{=}\PYG{p}{(}\PYG{n}{PM}\PYG{p}{)}
      \PYG{n}{DCBO1X} \PYG{n}{DCBE} \PYG{n}{RMODE31}\PYG{o}{=}\PYG{n}{BUFF}
\end{sphinxVerbatim}

\index{Virtel@\spxentry{Virtel}!Sharing files@\spxentry{Sharing files}}\index{Sharing files@\spxentry{Sharing files}!Virtel@\spxentry{Virtel}}\ignorespaces 

\subsection{How To Share VSAM Files Between Multiple Instances Of VIRTEL}
\label{\detokenize{Installation_Guide:how-to-share-vsam-files-between-multiple-instances-of-virtel}}\label{\detokenize{Installation_Guide:index-147}}
\sphinxAtStartPar
Some VSAM files are shareable between multiple instances of Virtel with the condition that a file can be opened in “write” mode by only one instance. File sharing can be implemented by modifying the corresponding UFILEx entry in the TCT and/or by using the VSAMTYP definition. Some files are not shareable, for example the statistics and swap files. These must be opened in read/write mode for each instance of Virtel.

\sphinxAtStartPar
For more detailed informations on this subject, see {\hyperref[\detokenize{Installation_Guide:v461ig-ufile1-to-ufile20-parameters}]{\sphinxcrossref{\DUrole{std,std-ref}{“UFILE1 to UFILE20”,}}}} and also {\hyperref[\detokenize{Installation_Guide:v461ig-additional-parameters-for-vsam-files}]{\sphinxcrossref{\DUrole{std,std-ref}{“Additional parameters for VSAM files”,}}}}.

\newpage


\section{Example Of The VIRTCT}
\label{\detokenize{Installation_Guide:example-of-the-virtct}}
\sphinxAtStartPar
An example of the VIRTCT is supplied in member VIRTCT01 in the VIRTEL SAMPLIB for z/OS, and in the installation job VIRTCT for z/VSE:

\begin{sphinxVerbatim}[commandchars=\\\{\}]
\PYG{n}{PRINT} \PYG{n}{GEN}
\PYG{n}{VIRTERM} \PYG{n}{TYPE}\PYG{o}{=}\PYG{n}{INITIAL}\PYG{p}{,}\PYG{n}{APPLID}\PYG{o}{=}\PYG{n}{VIRTEL}\PYG{p}{,} \PYG{o}{*}
        \PYG{n}{COMPR3}\PYG{o}{=}\PYG{n}{AUTO}\PYG{p}{,} \PYG{o}{*}
        \PYG{n}{LANG}\PYG{o}{=}\PYG{l+s+s1}{\PYGZsq{}}\PYG{l+s+s1}{ }\PYG{l+s+s1}{\PYGZsq{}}\PYG{p}{,} \PYG{n}{LANGUAGE} \PYG{n}{FOR} \PYG{n}{USER} \PYG{n}{MESSAGES} \PYG{o}{*}
        \PYG{n}{COUNTRY}\PYG{o}{=}\PYG{n}{FR}\PYG{p}{,} \PYG{n}{EBCDIC}\PYG{o}{\PYGZhy{}}\PYG{n}{ASCII} \PYG{n}{TRANSLATION} \PYG{o}{*}
        \PYG{n}{DEFUTF8}\PYG{o}{=}\PYG{n}{IBM1147}\PYG{p}{,} \PYG{n}{DEFAULT} \PYG{n}{OUTPUT} \PYG{n}{ENCODING} \PYG{n}{UTF}\PYG{o}{\PYGZhy{}}\PYG{l+m+mi}{8} \PYG{o}{*}
        \PYG{n}{CHARSET}\PYG{o}{=}\PYG{p}{,} \PYG{n}{UTF}\PYG{o}{\PYGZhy{}}\PYG{l+m+mi}{8}\PYG{p}{:} \PYG{n}{ADDITIONAL} \PYG{n}{CHARSETS} \PYG{o}{*}
        \PYG{n}{GMT}\PYG{o}{=}\PYG{n}{SYSTZ}\PYG{p}{,} \PYG{o}{*}
        \PYG{n}{CRYPT1}\PYG{o}{=}\PYG{p}{(}\PYG{n}{CRYPT3270}\PYG{p}{,}\PYG{n}{NONE}\PYG{p}{,}\PYG{n}{NONE}\PYG{p}{,}\PYG{n}{NO}\PYG{o}{\PYGZhy{}}\PYG{n}{ENCRYPTION}\PYG{p}{,}\PYG{n}{HEX}\PYG{p}{)}\PYG{p}{,} \PYG{o}{*}
        \PYG{n}{TCP1}\PYG{o}{=}\PYG{p}{(}\PYG{n}{TCPIP}\PYG{p}{,}\PYG{p}{,}\PYG{p}{,}\PYG{l+m+mi}{250}\PYG{p}{)}\PYG{p}{,} \PYG{o}{\PYGZlt{}}\PYG{o}{\PYGZhy{}}\PYG{o}{\PYGZhy{}}\PYG{o}{\PYGZhy{}}\PYG{o}{\PYGZhy{}}\PYG{o}{\PYGZhy{}}\PYG{o}{\PYGZhy{}} \PYG{o}{*}
        \PYG{n}{HTVSAM}\PYG{o}{=}\PYG{n}{VIRHTML}\PYG{p}{,} \PYG{o}{\PYGZlt{}}\PYG{o}{\PYGZhy{}}\PYG{o}{\PYGZhy{}}\PYG{o}{\PYGZhy{}}\PYG{o}{\PYGZhy{}}\PYG{o}{\PYGZhy{}}\PYG{o}{\PYGZhy{}} \PYG{o}{*}
        \PYG{n}{BUFSIZE}\PYG{o}{=}\PYG{l+m+mi}{20000}\PYG{p}{,} \PYG{o}{*}
        \PYG{n}{ACCUEIL}\PYG{o}{=}\PYG{n}{YES}\PYG{p}{,} \PYG{o}{*}
        \PYG{n}{DEFENTR}\PYG{o}{=}\PYG{n}{PC}\PYG{p}{,} \PYG{o}{*}
        \PYG{n}{FCAPT}\PYG{o}{=}\PYG{n}{VIRCAPT}\PYG{p}{,} \PYG{o}{*}
        \PYG{n}{RETOUR}\PYG{o}{=}\PYG{l+m+mi}{00}\PYG{p}{,} \PYG{o}{*}
        \PYG{n}{SUITE}\PYG{o}{=}\PYG{l+m+mi}{00}\PYG{p}{,} \PYG{o}{*}
        \PYG{n}{SOMMR}\PYG{o}{=}\PYG{l+m+mi}{00}\PYG{p}{,} \PYG{o}{*}
        \PYG{n}{CORRECT}\PYG{o}{=}\PYG{l+m+mi}{00}\PYG{p}{,} \PYG{o}{*}
        \PYG{n}{EXIT1}\PYG{o}{=}\PYG{p}{,} \PYG{o}{*}
        \PYG{n}{EXIT2}\PYG{o}{=}\PYG{p}{,} \PYG{o}{*}
        \PYG{n}{EXIT3}\PYG{o}{=}\PYG{p}{,} \PYG{o}{*}
        \PYG{n}{EXIT5}\PYG{o}{=}\PYG{p}{,} \PYG{o}{*}
        \PYG{n}{EXIT4}\PYG{o}{=}\PYG{p}{,} \PYG{o}{*}
        \PYG{n}{EXIT6}\PYG{o}{=}\PYG{p}{,} \PYG{o}{*}
        \PYG{n}{EXIT7}\PYG{o}{=}\PYG{p}{,} \PYG{o}{*}
        \PYG{n}{STATS}\PYG{o}{=}\PYG{n}{YES}\PYG{p}{,} \PYG{n}{OU} \PYG{p}{(}\PYG{n}{MULTI}\PYG{p}{,}\PYG{n}{CONTINUE}\PYG{o}{/}\PYG{n}{TERMINATE}\PYG{p}{)} \PYG{o}{*}
        \PYG{n}{STATDSN}\PYG{o}{=}\PYG{p}{(}\PYG{n}{VIRTEL}\PYG{o}{.}\PYG{n}{STATA}\PYG{p}{,}\PYG{n}{VIRTEL}\PYG{o}{.}\PYG{n}{STATB}\PYG{p}{)}\PYG{p}{,} \PYG{n}{SI} \PYG{n}{STATS}\PYG{o}{=}\PYG{n}{MULTI} \PYG{o}{*}
        \PYG{n}{FCMP3}\PYG{o}{=}\PYG{n}{VIRCMP3}\PYG{p}{,} \PYG{o}{*}
        \PYG{n}{APPSTAT}\PYG{o}{=}\PYG{n}{YES}\PYG{p}{,} \PYG{o}{*}
        \PYG{n}{DONTSWA}\PYG{o}{=}\PYG{n}{YES}\PYG{p}{,} \PYG{o}{*}
        \PYG{n}{NBDYNAM}\PYG{o}{=}\PYG{l+m+mi}{250}\PYG{p}{,} \PYG{o}{*}
        \PYG{n}{TRACEB}\PYG{o}{=}\PYG{l+m+mi}{200}\PYG{p}{,}\PYG{n}{TRACEON}\PYG{o}{=}\PYG{n}{YES}\PYG{p}{,}\PYG{n}{TRACBIG}\PYG{o}{=}\PYG{l+m+mi}{40}\PYG{p}{,} \PYG{o}{*}
        \PYG{n}{MULTI}\PYG{o}{=}\PYG{n}{YES}\PYG{p}{,}\PYG{n}{RESO}\PYG{o}{=}\PYG{n}{YES}\PYG{p}{,}\PYG{n}{ARBO}\PYG{o}{=}\PYG{n}{YES}\PYG{p}{,} \PYG{o}{*}
        \PYG{n}{VIRSECU}\PYG{o}{=}\PYG{n}{YES}\PYG{p}{,}\PYG{n}{SECUR}\PYG{o}{=}\PYG{n}{NO}\PYG{p}{,} \PYG{n}{VIRTEL}\PYG{p}{,} \PYG{p}{(}\PYG{n}{RACROUTE}\PYG{p}{,}\PYG{n}{RACF}\PYG{p}{)} \PYG{o}{*}
        \PYG{n}{RAPPL}\PYG{o}{=}\PYG{n}{VIRTSERV}\PYG{p}{,}\PYG{n}{RNODE}\PYG{o}{=}\PYG{n}{VIRTNODE}\PYG{p}{,} \PYG{o}{*}
        \PYG{n}{LOCK}\PYG{o}{=}\PYG{l+m+mi}{20000}\PYG{p}{,} \PYG{o}{*}
        \PYG{n}{TIMEOUT}\PYG{o}{=}\PYG{l+m+mi}{5}\PYG{p}{,} \PYG{o}{*}
        \PYG{n}{FASTC}\PYG{o}{=}\PYG{n}{NO}\PYG{p}{,} \PYG{o}{*}
        \PYG{n}{UFILE1}\PYG{o}{=}\PYG{p}{(}\PYG{n}{SAMPTRSF}\PYG{p}{,}\PYG{n}{ACBH1}\PYG{p}{,}\PYG{l+m+mi}{0}\PYG{p}{,}\PYG{l+m+mi}{10}\PYG{p}{,}\PYG{l+m+mi}{01}\PYG{p}{)}\PYG{p}{,} \PYG{o}{*}
        \PYG{n}{UFILE2}\PYG{o}{=}\PYG{p}{(}\PYG{n}{HTMLTRSF}\PYG{p}{,}\PYG{n}{ACBH2}\PYG{p}{,}\PYG{l+m+mi}{0}\PYG{p}{,}\PYG{l+m+mi}{10}\PYG{p}{,}\PYG{l+m+mi}{01}\PYG{p}{)}\PYG{p}{,} \PYG{o}{*}
        \PYG{n}{UFILE3}\PYG{o}{=}\PYG{p}{(}\PYG{n}{PLUGTRSF}\PYG{p}{,}\PYG{n}{ACBH3}\PYG{p}{,}\PYG{l+m+mi}{0}\PYG{p}{,}\PYG{l+m+mi}{10}\PYG{p}{,}\PYG{l+m+mi}{01}\PYG{p}{)}\PYG{p}{,} \PYG{o}{*}
        \PYG{n}{GATE}\PYG{o}{=}\PYG{n}{GENERAL}\PYG{p}{,} \PYG{o}{*}
        \PYG{n}{NBCVC}\PYG{o}{=}\PYG{l+m+mi}{32}\PYG{p}{,} \PYG{o}{*}
        \PYG{n}{VTKEYS}\PYG{o}{=}\PYG{n}{VTTABLE}\PYG{p}{,} \PYG{n}{VT100} \PYG{p}{:} \PYG{n}{KEY} \PYG{n}{REDEFINITION} \PYG{o}{*}
        \PYG{n}{MEMORY}\PYG{o}{=}\PYG{n}{ABOVE}\PYG{p}{,} \PYG{o}{*}
        \PYG{n}{COMPANY}\PYG{o}{=}\PYG{l+s+s1}{\PYGZsq{}}\PYG{l+s+s1}{VOTRE COMPAGNIE }\PYG{l+s+s1}{\PYGZsq{}}\PYG{p}{,} \PYG{n}{VOIR} \PYG{n}{LA} \PYG{n}{CLE} \PYG{o}{*}
        \PYG{n}{ADDR1}\PYG{o}{=}\PYG{l+s+s1}{\PYGZsq{}}\PYG{l+s+s1}{VOTRE ADRESSE 1 }\PYG{l+s+s1}{\PYGZsq{}}\PYG{p}{,} \PYG{n}{COMMUNIQUEE} \PYG{o}{*}
        \PYG{n}{ADDR2}\PYG{o}{=}\PYG{l+s+s1}{\PYGZsq{}}\PYG{l+s+s1}{VOTRE ADRESSE 2 }\PYG{l+s+s1}{\PYGZsq{}}\PYG{p}{,} \PYG{n}{PAR} \PYG{n}{SYSPERTEC} \PYG{o}{*}
        \PYG{n}{LICENCE}\PYG{o}{=}\PYG{l+s+s1}{\PYGZsq{}}\PYG{l+s+s1}{TYPE DE LICENCE }\PYG{l+s+s1}{\PYGZsq{}}\PYG{p}{,} \PYG{o}{*}
        \PYG{n}{EXPIRE}\PYG{o}{=}\PYG{p}{(}\PYG{l+m+mi}{2999}\PYG{p}{,}\PYG{l+m+mi}{12}\PYG{p}{,}\PYG{l+m+mi}{31}\PYG{p}{)}\PYG{p}{,} \PYG{o}{*}
        \PYG{n}{CODE}\PYG{o}{=}\PYG{l+s+s1}{\PYGZsq{}}\PYG{l+s+s1}{XXXXXXXX}\PYG{l+s+s1}{\PYGZsq{}}\PYG{p}{,} \PYG{o}{*}
        \PYG{n}{TITRE1}\PYG{o}{=}\PYG{l+s+s1}{\PYGZsq{}}\PYG{l+s+s1}{S Y S P E R T E C C O M M U N I C A T I O N }\PYG{l+s+s1}{\PYGZsq{}}\PYG{p}{,} \PYG{o}{*}
        \PYG{n}{TITRE2}\PYG{o}{=}\PYG{l+s+s1}{\PYGZsq{}}\PYG{l+s+s1}{=============== 4.2 ======================== }\PYG{l+s+s1}{\PYGZsq{}}
\PYG{o}{*} \PYG{o}{==}\PYG{o}{==}\PYG{o}{==}\PYG{o}{==}\PYG{o}{==}\PYG{o}{==}\PYG{o}{==}\PYG{o}{==}\PYG{o}{==}\PYG{o}{==}\PYG{o}{==}\PYG{o}{==}\PYG{o}{==}\PYG{o}{==}\PYG{o}{==}\PYG{o}{==}\PYG{o}{==}\PYG{o}{==}\PYG{o}{==}\PYG{o}{==}\PYG{o}{==}\PYG{o}{==}\PYG{o}{==}\PYG{o}{==}\PYG{o}{==}\PYG{o}{==}\PYG{o}{==}\PYG{o}{==}\PYG{o}{==}\PYG{o}{==}
\PYG{n}{VTTABLE} \PYG{n}{KTRANH} \PYG{p}{,} \PYG{n}{SAMPLE} \PYG{n}{VT} \PYG{n}{KEYS} \PYG{n}{TRANSLATION}
        \PYG{n}{KTRAN} \PYG{n}{KEY}\PYG{o}{=}\PYG{n}{D6D7}\PYG{p}{,}\PYG{n}{IS}\PYG{o}{=}\PYG{n}{F1}\PYG{p}{,}\PYG{n}{RETCODE}\PYG{o}{=}\PYG{l+m+mi}{1} \PYG{n}{PF1}
        \PYG{n}{KTRAN} \PYG{n}{KEY}\PYG{o}{=}\PYG{n}{D6D8}\PYG{p}{,}\PYG{n}{IS}\PYG{o}{=}\PYG{n}{F2}\PYG{p}{,}\PYG{n}{RETCODE}\PYG{o}{=}\PYG{l+m+mi}{1} \PYG{n}{PF2}
        \PYG{n}{KTRAN} \PYG{n}{KEY}\PYG{o}{=}\PYG{n}{D6D9}\PYG{p}{,}\PYG{n}{IS}\PYG{o}{=}\PYG{n}{F3}\PYG{p}{,}\PYG{n}{RETCODE}\PYG{o}{=}\PYG{l+m+mi}{1} \PYG{n}{PF3}
        \PYG{n}{KTRAN} \PYG{n}{KEY}\PYG{o}{=}\PYG{n}{D6E2}\PYG{p}{,}\PYG{n}{IS}\PYG{o}{=}\PYG{n}{F4}\PYG{p}{,}\PYG{n}{RETCODE}\PYG{o}{=}\PYG{l+m+mi}{1} \PYG{n}{PF4}
        \PYG{n}{KTRAN} \PYG{n}{KEY}\PYG{o}{=}\PYG{n}{D6E3}\PYG{p}{,}\PYG{n}{IS}\PYG{o}{=}\PYG{n}{F5}\PYG{p}{,}\PYG{n}{RETCODE}\PYG{o}{=}\PYG{l+m+mi}{1} \PYG{n}{PF5}
        \PYG{n}{KTRAN} \PYG{n}{KEY}\PYG{o}{=}\PYG{n}{D6E4}\PYG{p}{,}\PYG{n}{IS}\PYG{o}{=}\PYG{n}{F6}\PYG{p}{,}\PYG{n}{RETCODE}\PYG{o}{=}\PYG{l+m+mi}{1} \PYG{n}{PF6}
        \PYG{n}{KTRAN} \PYG{n}{KEY}\PYG{o}{=}\PYG{n}{D6E5}\PYG{p}{,}\PYG{n}{IS}\PYG{o}{=}\PYG{n}{F7}\PYG{p}{,}\PYG{n}{RETCODE}\PYG{o}{=}\PYG{l+m+mi}{1} \PYG{n}{PF7}
        \PYG{n}{KTRAN} \PYG{n}{KEY}\PYG{o}{=}\PYG{n}{D6E6}\PYG{p}{,}\PYG{n}{IS}\PYG{o}{=}\PYG{n}{F8}\PYG{p}{,}\PYG{n}{RETCODE}\PYG{o}{=}\PYG{l+m+mi}{1} \PYG{n}{PF8}
        \PYG{n}{KTRAN} \PYG{n}{KEY}\PYG{o}{=}\PYG{n}{D6E7}\PYG{p}{,}\PYG{n}{IS}\PYG{o}{=}\PYG{n}{F9}\PYG{p}{,}\PYG{n}{RETCODE}\PYG{o}{=}\PYG{l+m+mi}{1} \PYG{n}{PF9}
        \PYG{n}{KTRAN} \PYG{n}{KEY}\PYG{o}{=}\PYG{n}{D6E8}\PYG{p}{,}\PYG{n}{IS}\PYG{o}{=}\PYG{l+m+mi}{7}\PYG{n}{A}\PYG{p}{,}\PYG{n}{RETCODE}\PYG{o}{=}\PYG{l+m+mi}{1} \PYG{n}{PF10}
        \PYG{n}{KTRAN} \PYG{n}{KEY}\PYG{o}{=}\PYG{n}{D6E9}\PYG{p}{,}\PYG{n}{IS}\PYG{o}{=}\PYG{l+m+mi}{7}\PYG{n}{B}\PYG{p}{,}\PYG{n}{RETCODE}\PYG{o}{=}\PYG{l+m+mi}{1} \PYG{n}{PF11}
        \PYG{n}{KTRAN} \PYG{n}{KEY}\PYG{o}{=}\PYG{n}{D64A}\PYG{p}{,}\PYG{n}{IS}\PYG{o}{=}\PYG{l+m+mi}{7}\PYG{n}{C}\PYG{p}{,}\PYG{n}{RETCODE}\PYG{o}{=}\PYG{l+m+mi}{1} \PYG{n}{PF12}
        \PYG{n}{KTRAN} \PYG{n}{KEY}\PYG{o}{=}\PYG{n}{D6A7}\PYG{p}{,}\PYG{n}{IS}\PYG{o}{=}\PYG{l+m+mi}{6}\PYG{n}{D}\PYG{p}{,}\PYG{n}{RETCODE}\PYG{o}{=}\PYG{l+m+mi}{1} \PYG{n}{CLEAR}
        \PYG{n}{KTRAN} \PYG{n}{KEY}\PYG{o}{=}\PYG{l+m+mi}{4}\PYG{n}{AC8}\PYG{p}{,}\PYG{n}{IS}\PYG{o}{=}\PYG{l+m+mi}{6}\PYG{n}{D}\PYG{p}{,}\PYG{n}{RETCODE}\PYG{o}{=}\PYG{l+m+mi}{1} \PYG{n}{CLEAR}
\PYG{o}{*}
        \PYG{n}{KTRAN} \PYG{n}{KEY}\PYG{o}{=}\PYG{l+m+mi}{4}\PYG{n}{AD2}\PYG{p}{,}\PYG{n}{IS}\PYG{o}{=}\PYG{l+m+mi}{00}\PYG{p}{,}\PYG{n}{RETCODE}\PYG{o}{=}\PYG{l+m+mi}{2} \PYG{n}{ERASEOF}
        \PYG{n}{KTRAN} \PYG{n}{KEY}\PYG{o}{=}\PYG{l+m+mi}{4}\PYG{n}{AC1}\PYG{p}{,}\PYG{n}{IS}\PYG{o}{=}\PYG{l+m+mi}{00}\PYG{p}{,}\PYG{n}{RETCODE}\PYG{o}{=}\PYG{l+m+mi}{3} \PYG{n}{CURU}
        \PYG{n}{KTRAN} \PYG{n}{KEY}\PYG{o}{=}\PYG{l+m+mi}{4}\PYG{n}{AC2}\PYG{p}{,}\PYG{n}{IS}\PYG{o}{=}\PYG{l+m+mi}{00}\PYG{p}{,}\PYG{n}{RETCODE}\PYG{o}{=}\PYG{l+m+mi}{4} \PYG{n}{CURD}
        \PYG{n}{KTRAN} \PYG{n}{KEY}\PYG{o}{=}\PYG{l+m+mi}{4}\PYG{n}{AC3}\PYG{p}{,}\PYG{n}{IS}\PYG{o}{=}\PYG{l+m+mi}{00}\PYG{p}{,}\PYG{n}{RETCODE}\PYG{o}{=}\PYG{l+m+mi}{5} \PYG{n}{CURR}
        \PYG{n}{KTRAN} \PYG{n}{KEY}\PYG{o}{=}\PYG{l+m+mi}{4}\PYG{n}{AC4}\PYG{p}{,}\PYG{n}{IS}\PYG{o}{=}\PYG{l+m+mi}{00}\PYG{p}{,}\PYG{n}{RETCODE}\PYG{o}{=}\PYG{l+m+mi}{6} \PYG{n}{CURL}
\PYG{o}{*} \PYG{o}{==}\PYG{o}{==}\PYG{o}{==}\PYG{o}{==}\PYG{o}{==}\PYG{o}{==}\PYG{o}{==}\PYG{o}{==}\PYG{o}{==}\PYG{o}{==}\PYG{o}{==}\PYG{o}{==}\PYG{o}{==}\PYG{o}{==}\PYG{o}{==}\PYG{o}{==}\PYG{o}{==}\PYG{o}{==}\PYG{o}{==}\PYG{o}{==}\PYG{o}{==}\PYG{o}{==}\PYG{o}{==}\PYG{o}{==}\PYG{o}{==}\PYG{o}{==}\PYG{o}{==}\PYG{o}{==}\PYG{o}{==}\PYG{o}{==}
\PYG{n}{ACBH1} \PYG{n}{ACB} \PYG{n}{AM}\PYG{o}{=}\PYG{n}{VSAM}\PYG{p}{,}\PYG{n}{DDNAME}\PYG{o}{=}\PYG{n}{SAMPTRSF}\PYG{p}{,}\PYG{n}{MACRF}\PYG{o}{=}\PYG{p}{(}\PYG{n}{SEQ}\PYG{p}{,}\PYG{n}{DIR}\PYG{p}{,}\PYG{n}{OUT}\PYG{p}{,}\PYG{n}{LSR}\PYG{p}{)}\PYG{p}{,} \PYG{o}{*}
        \PYG{n}{STRNO}\PYG{o}{=}\PYG{l+m+mi}{3}
\PYG{n}{ACBH2} \PYG{n}{ACB} \PYG{n}{AM}\PYG{o}{=}\PYG{n}{VSAM}\PYG{p}{,}\PYG{n}{DDNAME}\PYG{o}{=}\PYG{n}{HTMLTRSF}\PYG{p}{,}\PYG{n}{MACRF}\PYG{o}{=}\PYG{p}{(}\PYG{n}{SEQ}\PYG{p}{,}\PYG{n}{DIR}\PYG{p}{,}\PYG{n}{OUT}\PYG{p}{,}\PYG{n}{LSR}\PYG{p}{)}\PYG{p}{,} \PYG{o}{*}
        \PYG{n}{STRNO}\PYG{o}{=}\PYG{l+m+mi}{3}
\PYG{n}{ACBH3} \PYG{n}{ACB} \PYG{n}{AM}\PYG{o}{=}\PYG{n}{VSAM}\PYG{p}{,}\PYG{n}{DDNAME}\PYG{o}{=}\PYG{n}{PLUGTRSF}\PYG{p}{,}\PYG{n}{MACRF}\PYG{o}{=}\PYG{p}{(}\PYG{n}{SEQ}\PYG{p}{,}\PYG{n}{DIR}\PYG{p}{,}\PYG{n}{OUT}\PYG{p}{,}\PYG{n}{LSR}\PYG{p}{)}\PYG{p}{,} \PYG{o}{*}
        \PYG{n}{STRNO}\PYG{o}{=}\PYG{l+m+mi}{3}
        \PYG{n}{END}
\end{sphinxVerbatim}

\sphinxAtStartPar
\sphinxstyleemphasis{Example VIRTCT}

\index{Virtel TCT@\spxentry{Virtel TCT}!Assembly@\spxentry{Assembly}}\index{Assembly@\spxentry{Assembly}!Virtel TCT@\spxentry{Virtel TCT}}\ignorespaces 

\section{Assembling The VIRTCT}
\label{\detokenize{Installation_Guide:index-148}}\label{\detokenize{Installation_Guide:id8}}
\sphinxAtStartPar
The VIRTCT must be assembled before starting VIRTEL for the first time. The VIRTEL macro library must be available to the assembler. In the z/OS environment, the VIRTCT must be link\sphinxhyphen{}edited with the NORENT and NOREUS options. The RENT and REUS options must NOT be specified in the z/OS environment. In the z/VSE environment, PRD1.MACLIB must be specified. The resulting phase or load module must be placed in a STEPLIB or SEARCH PHASE library available to the VIRTEL started task.

\index{Virtel TCT@\spxentry{Virtel TCT}!Assembly \sphinxhyphen{} Z/OS@\spxentry{Assembly \sphinxhyphen{} Z/OS}}\index{Assembly \sphinxhyphen{} Z/OS@\spxentry{Assembly \sphinxhyphen{} Z/OS}!Virtel TCT@\spxentry{Virtel TCT}}\ignorespaces 

\subsection{z/OS example}
\label{\detokenize{Installation_Guide:z-os-example}}\label{\detokenize{Installation_Guide:index-149}}
\sphinxAtStartPar
A sample job for assembling the VIRTCT is supplied in member ASMTCT of the VIRTEL SAMPLIB:

\begin{sphinxVerbatim}[commandchars=\\\{\}]
\PYG{o}{/}\PYG{o}{/}\PYG{n}{VIRTASM} \PYG{n}{JOB} \PYG{l+m+mi}{1}\PYG{p}{,}\PYG{n}{ASMTCT}\PYG{p}{,}\PYG{n}{MSGCLASS}\PYG{o}{=}\PYG{n}{X}\PYG{p}{,}\PYG{n}{CLASS}\PYG{o}{=}\PYG{n}{A}\PYG{p}{,}\PYG{n}{NOTIFY}\PYG{o}{=}\PYG{o}{\PYGZam{}}\PYG{n}{SYSUID}
\PYG{o}{/}\PYG{o}{/}\PYG{o}{*}\PYG{o}{\PYGZhy{}}\PYG{o}{\PYGZhy{}}\PYG{o}{\PYGZhy{}}\PYG{o}{\PYGZhy{}}\PYG{o}{\PYGZhy{}}\PYG{o}{\PYGZhy{}}\PYG{o}{\PYGZhy{}}\PYG{o}{\PYGZhy{}}\PYG{o}{\PYGZhy{}}\PYG{o}{\PYGZhy{}}\PYG{o}{\PYGZhy{}}\PYG{o}{\PYGZhy{}}\PYG{o}{\PYGZhy{}}\PYG{o}{\PYGZhy{}}\PYG{o}{\PYGZhy{}}\PYG{o}{\PYGZhy{}}\PYG{o}{\PYGZhy{}}\PYG{o}{\PYGZhy{}}\PYG{o}{\PYGZhy{}}\PYG{o}{\PYGZhy{}}\PYG{o}{\PYGZhy{}}\PYG{o}{\PYGZhy{}}\PYG{o}{\PYGZhy{}}\PYG{o}{\PYGZhy{}}\PYG{o}{\PYGZhy{}}\PYG{o}{\PYGZhy{}}\PYG{o}{\PYGZhy{}}\PYG{o}{\PYGZhy{}}\PYG{o}{\PYGZhy{}}\PYG{o}{\PYGZhy{}}\PYG{o}{\PYGZhy{}}\PYG{o}{\PYGZhy{}}\PYG{o}{\PYGZhy{}}\PYG{o}{\PYGZhy{}}\PYG{o}{\PYGZhy{}}\PYG{o}{\PYGZhy{}}\PYG{o}{\PYGZhy{}}\PYG{o}{\PYGZhy{}}\PYG{o}{\PYGZhy{}}\PYG{o}{\PYGZhy{}}\PYG{o}{\PYGZhy{}}\PYG{o}{\PYGZhy{}}\PYG{o}{\PYGZhy{}}\PYG{o}{\PYGZhy{}}\PYG{o}{\PYGZhy{}}\PYG{o}{\PYGZhy{}}\PYG{o}{\PYGZhy{}}\PYG{o}{\PYGZhy{}}\PYG{o}{\PYGZhy{}}\PYG{o}{\PYGZhy{}}\PYG{o}{\PYGZhy{}}\PYG{o}{\PYGZhy{}}\PYG{o}{\PYGZhy{}}\PYG{o}{\PYGZhy{}}\PYG{o}{\PYGZhy{}}\PYG{o}{\PYGZhy{}}\PYG{o}{\PYGZhy{}}\PYG{o}{\PYGZhy{}}\PYG{o}{\PYGZhy{}}\PYG{o}{\PYGZhy{}}\PYG{o}{\PYGZhy{}}\PYG{o}{\PYGZhy{}}\PYG{o}{*}
\PYG{o}{/}\PYG{o}{/}\PYG{o}{*} \PYG{n}{GENERATION} \PYG{n}{VIRTCT} \PYG{n}{ET} \PYG{n}{EXIT} \PYG{n}{DE} \PYG{n}{VIRTEL} \PYG{o}{*}
\PYG{o}{/}\PYG{o}{/}\PYG{o}{*}\PYG{o}{\PYGZhy{}}\PYG{o}{\PYGZhy{}}\PYG{o}{\PYGZhy{}}\PYG{o}{\PYGZhy{}}\PYG{o}{\PYGZhy{}}\PYG{o}{\PYGZhy{}}\PYG{o}{\PYGZhy{}}\PYG{o}{\PYGZhy{}}\PYG{o}{\PYGZhy{}}\PYG{o}{\PYGZhy{}}\PYG{o}{\PYGZhy{}}\PYG{o}{\PYGZhy{}}\PYG{o}{\PYGZhy{}}\PYG{o}{\PYGZhy{}}\PYG{o}{\PYGZhy{}}\PYG{o}{\PYGZhy{}}\PYG{o}{\PYGZhy{}}\PYG{o}{\PYGZhy{}}\PYG{o}{\PYGZhy{}}\PYG{o}{\PYGZhy{}}\PYG{o}{\PYGZhy{}}\PYG{o}{\PYGZhy{}}\PYG{o}{\PYGZhy{}}\PYG{o}{\PYGZhy{}}\PYG{o}{\PYGZhy{}}\PYG{o}{\PYGZhy{}}\PYG{o}{\PYGZhy{}}\PYG{o}{\PYGZhy{}}\PYG{o}{\PYGZhy{}}\PYG{o}{\PYGZhy{}}\PYG{o}{\PYGZhy{}}\PYG{o}{\PYGZhy{}}\PYG{o}{\PYGZhy{}}\PYG{o}{\PYGZhy{}}\PYG{o}{\PYGZhy{}}\PYG{o}{\PYGZhy{}}\PYG{o}{\PYGZhy{}}\PYG{o}{\PYGZhy{}}\PYG{o}{\PYGZhy{}}\PYG{o}{\PYGZhy{}}\PYG{o}{\PYGZhy{}}\PYG{o}{\PYGZhy{}}\PYG{o}{\PYGZhy{}}\PYG{o}{\PYGZhy{}}\PYG{o}{\PYGZhy{}}\PYG{o}{\PYGZhy{}}\PYG{o}{\PYGZhy{}}\PYG{o}{\PYGZhy{}}\PYG{o}{\PYGZhy{}}\PYG{o}{\PYGZhy{}}\PYG{o}{\PYGZhy{}}\PYG{o}{\PYGZhy{}}\PYG{o}{\PYGZhy{}}\PYG{o}{\PYGZhy{}}\PYG{o}{\PYGZhy{}}\PYG{o}{\PYGZhy{}}\PYG{o}{\PYGZhy{}}\PYG{o}{\PYGZhy{}}\PYG{o}{\PYGZhy{}}\PYG{o}{\PYGZhy{}}\PYG{o}{\PYGZhy{}}\PYG{o}{\PYGZhy{}}\PYG{o}{*}
\PYG{o}{/}\PYG{o}{/}\PYG{n}{ASMTCT} \PYG{n}{PROC} \PYG{n}{OUTC}\PYG{o}{=}\PYG{o}{*}\PYG{p}{,}\PYG{n}{WORK}\PYG{o}{=}\PYG{n}{SYSDA}\PYG{p}{,}
\PYG{o}{/}\PYG{o}{/} \PYG{n}{QUAL}\PYG{o}{=}\PYG{n}{yourqual}\PYG{o}{.}\PYG{n}{VIRTvrr}\PYG{p}{,}
\PYG{o}{/}\PYG{o}{/} \PYG{n}{MEMBER}\PYG{o}{=}\PYG{n}{VIRTCT01}
\PYG{o}{/}\PYG{o}{/}\PYG{o}{*}\PYG{o}{\PYGZhy{}}\PYG{o}{\PYGZhy{}}\PYG{o}{\PYGZhy{}}\PYG{o}{\PYGZhy{}}\PYG{o}{\PYGZhy{}}\PYG{o}{\PYGZhy{}}\PYG{o}{\PYGZhy{}}\PYG{o}{\PYGZhy{}}\PYG{o}{\PYGZhy{}}\PYG{o}{\PYGZhy{}}\PYG{o}{\PYGZhy{}}\PYG{o}{\PYGZhy{}}\PYG{o}{\PYGZhy{}}\PYG{o}{\PYGZhy{}}\PYG{o}{\PYGZhy{}}\PYG{o}{\PYGZhy{}}\PYG{o}{\PYGZhy{}}\PYG{o}{\PYGZhy{}}\PYG{o}{\PYGZhy{}}\PYG{o}{\PYGZhy{}}\PYG{o}{\PYGZhy{}}\PYG{o}{\PYGZhy{}}\PYG{o}{\PYGZhy{}}\PYG{o}{\PYGZhy{}}\PYG{o}{\PYGZhy{}}\PYG{o}{\PYGZhy{}}\PYG{o}{\PYGZhy{}}\PYG{o}{\PYGZhy{}}\PYG{o}{\PYGZhy{}}\PYG{o}{\PYGZhy{}}\PYG{o}{\PYGZhy{}}\PYG{o}{\PYGZhy{}}\PYG{o}{\PYGZhy{}}\PYG{o}{\PYGZhy{}}\PYG{o}{\PYGZhy{}}\PYG{o}{\PYGZhy{}}\PYG{o}{\PYGZhy{}}\PYG{o}{\PYGZhy{}}\PYG{o}{\PYGZhy{}}\PYG{o}{\PYGZhy{}}\PYG{o}{\PYGZhy{}}\PYG{o}{\PYGZhy{}}\PYG{o}{\PYGZhy{}}\PYG{o}{\PYGZhy{}}\PYG{o}{\PYGZhy{}}\PYG{o}{\PYGZhy{}}\PYG{o}{\PYGZhy{}}\PYG{o}{\PYGZhy{}}\PYG{o}{\PYGZhy{}}\PYG{o}{\PYGZhy{}}\PYG{o}{\PYGZhy{}}\PYG{o}{\PYGZhy{}}\PYG{o}{\PYGZhy{}}\PYG{o}{\PYGZhy{}}\PYG{o}{\PYGZhy{}}\PYG{o}{\PYGZhy{}}\PYG{o}{\PYGZhy{}}\PYG{o}{\PYGZhy{}}\PYG{o}{\PYGZhy{}}\PYG{o}{\PYGZhy{}}\PYG{o}{\PYGZhy{}}\PYG{o}{\PYGZhy{}}\PYG{o}{*}
\PYG{o}{/}\PYG{o}{/}\PYG{o}{*} \PYG{n}{ASSEMBLAGE} \PYG{o}{*}
\PYG{o}{/}\PYG{o}{/}\PYG{o}{*}\PYG{o}{\PYGZhy{}}\PYG{o}{\PYGZhy{}}\PYG{o}{\PYGZhy{}}\PYG{o}{\PYGZhy{}}\PYG{o}{\PYGZhy{}}\PYG{o}{\PYGZhy{}}\PYG{o}{\PYGZhy{}}\PYG{o}{\PYGZhy{}}\PYG{o}{\PYGZhy{}}\PYG{o}{\PYGZhy{}}\PYG{o}{\PYGZhy{}}\PYG{o}{\PYGZhy{}}\PYG{o}{\PYGZhy{}}\PYG{o}{\PYGZhy{}}\PYG{o}{\PYGZhy{}}\PYG{o}{\PYGZhy{}}\PYG{o}{\PYGZhy{}}\PYG{o}{\PYGZhy{}}\PYG{o}{\PYGZhy{}}\PYG{o}{\PYGZhy{}}\PYG{o}{\PYGZhy{}}\PYG{o}{\PYGZhy{}}\PYG{o}{\PYGZhy{}}\PYG{o}{\PYGZhy{}}\PYG{o}{\PYGZhy{}}\PYG{o}{\PYGZhy{}}\PYG{o}{\PYGZhy{}}\PYG{o}{\PYGZhy{}}\PYG{o}{\PYGZhy{}}\PYG{o}{\PYGZhy{}}\PYG{o}{\PYGZhy{}}\PYG{o}{\PYGZhy{}}\PYG{o}{\PYGZhy{}}\PYG{o}{\PYGZhy{}}\PYG{o}{\PYGZhy{}}\PYG{o}{\PYGZhy{}}\PYG{o}{\PYGZhy{}}\PYG{o}{\PYGZhy{}}\PYG{o}{\PYGZhy{}}\PYG{o}{\PYGZhy{}}\PYG{o}{\PYGZhy{}}\PYG{o}{\PYGZhy{}}\PYG{o}{\PYGZhy{}}\PYG{o}{\PYGZhy{}}\PYG{o}{\PYGZhy{}}\PYG{o}{\PYGZhy{}}\PYG{o}{\PYGZhy{}}\PYG{o}{\PYGZhy{}}\PYG{o}{\PYGZhy{}}\PYG{o}{\PYGZhy{}}\PYG{o}{\PYGZhy{}}\PYG{o}{\PYGZhy{}}\PYG{o}{\PYGZhy{}}\PYG{o}{\PYGZhy{}}\PYG{o}{\PYGZhy{}}\PYG{o}{\PYGZhy{}}\PYG{o}{\PYGZhy{}}\PYG{o}{\PYGZhy{}}\PYG{o}{\PYGZhy{}}\PYG{o}{\PYGZhy{}}\PYG{o}{\PYGZhy{}}\PYG{o}{\PYGZhy{}}\PYG{o}{*}
\PYG{o}{/}\PYG{o}{/}\PYG{n}{ASM} \PYG{n}{EXEC} \PYG{n}{PGM}\PYG{o}{=}\PYG{n}{ASMA90}\PYG{p}{,}\PYG{n}{REGION}\PYG{o}{=}\PYG{l+m+mi}{2048}\PYG{n}{K}\PYG{p}{,}
\PYG{o}{/}\PYG{o}{/} \PYG{n}{PARM}\PYG{o}{=}\PYG{l+s+s1}{\PYGZsq{}}\PYG{l+s+s1}{NOOBJECT,DECK,XREF(SHORT),NORENT,LIST}\PYG{l+s+s1}{\PYGZsq{}}
\PYG{o}{/}\PYG{o}{/}\PYG{n}{SYSPRINT} \PYG{n}{DD} \PYG{n}{SYSOUT}\PYG{o}{=}\PYG{o}{\PYGZam{}}\PYG{n}{OUTC}
\PYG{o}{/}\PYG{o}{/}\PYG{n}{SYSLIB} \PYG{n}{DD} \PYG{n}{DSN}\PYG{o}{=}\PYG{o}{\PYGZam{}}\PYG{n}{QUAL}\PYG{o}{.}\PYG{o}{.}\PYG{n}{MACLIB}\PYG{p}{,}\PYG{n}{DISP}\PYG{o}{=}\PYG{n}{SHR}
\PYG{o}{/}\PYG{o}{/} \PYG{n}{DD} \PYG{n}{DSN}\PYG{o}{=}\PYG{n}{SYS1}\PYG{o}{.}\PYG{n}{MACLIB}\PYG{p}{,}\PYG{n}{DISP}\PYG{o}{=}\PYG{n}{SHR}
\PYG{o}{/}\PYG{o}{/} \PYG{n}{DD} \PYG{n}{DSN}\PYG{o}{=}\PYG{n}{SYS1}\PYG{o}{.}\PYG{n}{MODGEN}\PYG{p}{,}\PYG{n}{DISP}\PYG{o}{=}\PYG{n}{SHR}
\PYG{o}{/}\PYG{o}{/}\PYG{n}{SYSUT1} \PYG{n}{DD} \PYG{n}{UNIT}\PYG{o}{=}\PYG{o}{\PYGZam{}}\PYG{n}{WORK}\PYG{p}{,}\PYG{n}{SPACE}\PYG{o}{=}\PYG{p}{(}\PYG{l+m+mi}{1700}\PYG{p}{,}\PYG{p}{(}\PYG{l+m+mi}{400}\PYG{p}{,}\PYG{l+m+mi}{400}\PYG{p}{)}\PYG{p}{)}
\PYG{o}{/}\PYG{o}{/}\PYG{n}{SYSUT2} \PYG{n}{DD} \PYG{n}{UNIT}\PYG{o}{=}\PYG{o}{\PYGZam{}}\PYG{n}{WORK}\PYG{p}{,}\PYG{n}{SPACE}\PYG{o}{=}\PYG{p}{(}\PYG{l+m+mi}{1700}\PYG{p}{,}\PYG{p}{(}\PYG{l+m+mi}{400}\PYG{p}{,}\PYG{l+m+mi}{400}\PYG{p}{)}\PYG{p}{)}
\PYG{o}{/}\PYG{o}{/}\PYG{n}{SYSUT3} \PYG{n}{DD} \PYG{n}{UNIT}\PYG{o}{=}\PYG{o}{\PYGZam{}}\PYG{n}{WORK}\PYG{p}{,}\PYG{n}{SPACE}\PYG{o}{=}\PYG{p}{(}\PYG{l+m+mi}{1700}\PYG{p}{,}\PYG{p}{(}\PYG{l+m+mi}{400}\PYG{p}{,}\PYG{l+m+mi}{400}\PYG{p}{)}\PYG{p}{)}
\PYG{o}{/}\PYG{o}{/}\PYG{n}{SYSPUNCH} \PYG{n}{DD} \PYG{n}{DSN}\PYG{o}{=}\PYG{o}{\PYGZam{}}\PYG{o}{\PYGZam{}}\PYG{n}{LOADSET}\PYG{p}{,}\PYG{n}{UNIT}\PYG{o}{=}\PYG{o}{\PYGZam{}}\PYG{n}{WORK}\PYG{p}{,}\PYG{n}{DISP}\PYG{o}{=}\PYG{p}{(}\PYG{p}{,}\PYG{n}{PASS}\PYG{p}{)}\PYG{p}{,}
\PYG{o}{/}\PYG{o}{/} \PYG{n}{SPACE}\PYG{o}{=}\PYG{p}{(}\PYG{l+m+mi}{400}\PYG{p}{,}\PYG{p}{(}\PYG{l+m+mi}{100}\PYG{p}{,}\PYG{l+m+mi}{100}\PYG{p}{)}\PYG{p}{)}
\PYG{o}{/}\PYG{o}{/}\PYG{n}{SYSIN} \PYG{n}{DD} \PYG{n}{DSN}\PYG{o}{=}\PYG{o}{\PYGZam{}}\PYG{n}{QUAL}\PYG{o}{.}\PYG{o}{.}\PYG{n}{CNTL}\PYG{p}{(}\PYG{o}{\PYGZam{}}\PYG{n}{MEMBER}\PYG{p}{)}\PYG{p}{,}\PYG{n}{DISP}\PYG{o}{=}\PYG{n}{SHR}
\PYG{o}{/}\PYG{o}{/}\PYG{o}{*}\PYG{o}{\PYGZhy{}}\PYG{o}{\PYGZhy{}}\PYG{o}{\PYGZhy{}}\PYG{o}{\PYGZhy{}}\PYG{o}{\PYGZhy{}}\PYG{o}{\PYGZhy{}}\PYG{o}{\PYGZhy{}}\PYG{o}{\PYGZhy{}}\PYG{o}{\PYGZhy{}}\PYG{o}{\PYGZhy{}}\PYG{o}{\PYGZhy{}}\PYG{o}{\PYGZhy{}}\PYG{o}{\PYGZhy{}}\PYG{o}{\PYGZhy{}}\PYG{o}{\PYGZhy{}}\PYG{o}{\PYGZhy{}}\PYG{o}{\PYGZhy{}}\PYG{o}{\PYGZhy{}}\PYG{o}{\PYGZhy{}}\PYG{o}{\PYGZhy{}}\PYG{o}{\PYGZhy{}}\PYG{o}{\PYGZhy{}}\PYG{o}{\PYGZhy{}}\PYG{o}{\PYGZhy{}}\PYG{o}{\PYGZhy{}}\PYG{o}{\PYGZhy{}}\PYG{o}{\PYGZhy{}}\PYG{o}{\PYGZhy{}}\PYG{o}{\PYGZhy{}}\PYG{o}{\PYGZhy{}}\PYG{o}{\PYGZhy{}}\PYG{o}{\PYGZhy{}}\PYG{o}{\PYGZhy{}}\PYG{o}{\PYGZhy{}}\PYG{o}{\PYGZhy{}}\PYG{o}{\PYGZhy{}}\PYG{o}{\PYGZhy{}}\PYG{o}{\PYGZhy{}}\PYG{o}{\PYGZhy{}}\PYG{o}{\PYGZhy{}}\PYG{o}{\PYGZhy{}}\PYG{o}{\PYGZhy{}}\PYG{o}{\PYGZhy{}}\PYG{o}{\PYGZhy{}}\PYG{o}{\PYGZhy{}}\PYG{o}{\PYGZhy{}}\PYG{o}{\PYGZhy{}}\PYG{o}{\PYGZhy{}}\PYG{o}{\PYGZhy{}}\PYG{o}{\PYGZhy{}}\PYG{o}{\PYGZhy{}}\PYG{o}{\PYGZhy{}}\PYG{o}{\PYGZhy{}}\PYG{o}{\PYGZhy{}}\PYG{o}{\PYGZhy{}}\PYG{o}{\PYGZhy{}}\PYG{o}{\PYGZhy{}}\PYG{o}{\PYGZhy{}}\PYG{o}{\PYGZhy{}}\PYG{o}{\PYGZhy{}}\PYG{o}{\PYGZhy{}}\PYG{o}{\PYGZhy{}}\PYG{o}{*}
\PYG{o}{/}\PYG{o}{/}\PYG{o}{*} \PYG{n}{LINKEDIT} \PYG{o}{*}
\PYG{o}{/}\PYG{o}{/}\PYG{o}{*}\PYG{o}{\PYGZhy{}}\PYG{o}{\PYGZhy{}}\PYG{o}{\PYGZhy{}}\PYG{o}{\PYGZhy{}}\PYG{o}{\PYGZhy{}}\PYG{o}{\PYGZhy{}}\PYG{o}{\PYGZhy{}}\PYG{o}{\PYGZhy{}}\PYG{o}{\PYGZhy{}}\PYG{o}{\PYGZhy{}}\PYG{o}{\PYGZhy{}}\PYG{o}{\PYGZhy{}}\PYG{o}{\PYGZhy{}}\PYG{o}{\PYGZhy{}}\PYG{o}{\PYGZhy{}}\PYG{o}{\PYGZhy{}}\PYG{o}{\PYGZhy{}}\PYG{o}{\PYGZhy{}}\PYG{o}{\PYGZhy{}}\PYG{o}{\PYGZhy{}}\PYG{o}{\PYGZhy{}}\PYG{o}{\PYGZhy{}}\PYG{o}{\PYGZhy{}}\PYG{o}{\PYGZhy{}}\PYG{o}{\PYGZhy{}}\PYG{o}{\PYGZhy{}}\PYG{o}{\PYGZhy{}}\PYG{o}{\PYGZhy{}}\PYG{o}{\PYGZhy{}}\PYG{o}{\PYGZhy{}}\PYG{o}{\PYGZhy{}}\PYG{o}{\PYGZhy{}}\PYG{o}{\PYGZhy{}}\PYG{o}{\PYGZhy{}}\PYG{o}{\PYGZhy{}}\PYG{o}{\PYGZhy{}}\PYG{o}{\PYGZhy{}}\PYG{o}{\PYGZhy{}}\PYG{o}{\PYGZhy{}}\PYG{o}{\PYGZhy{}}\PYG{o}{\PYGZhy{}}\PYG{o}{\PYGZhy{}}\PYG{o}{\PYGZhy{}}\PYG{o}{\PYGZhy{}}\PYG{o}{\PYGZhy{}}\PYG{o}{\PYGZhy{}}\PYG{o}{\PYGZhy{}}\PYG{o}{\PYGZhy{}}\PYG{o}{\PYGZhy{}}\PYG{o}{\PYGZhy{}}\PYG{o}{\PYGZhy{}}\PYG{o}{\PYGZhy{}}\PYG{o}{\PYGZhy{}}\PYG{o}{\PYGZhy{}}\PYG{o}{\PYGZhy{}}\PYG{o}{\PYGZhy{}}\PYG{o}{\PYGZhy{}}\PYG{o}{\PYGZhy{}}\PYG{o}{\PYGZhy{}}\PYG{o}{\PYGZhy{}}\PYG{o}{\PYGZhy{}}\PYG{o}{\PYGZhy{}}\PYG{o}{*}
\PYG{o}{/}\PYG{o}{/}\PYG{n}{LKED} \PYG{n}{EXEC} \PYG{n}{PGM}\PYG{o}{=}\PYG{n}{HEWL}\PYG{p}{,}\PYG{n}{REGION}\PYG{o}{=}\PYG{l+m+mi}{2048}\PYG{n}{K}\PYG{p}{,}\PYG{n}{COND}\PYG{o}{=}\PYG{p}{(}\PYG{l+m+mi}{7}\PYG{p}{,}\PYG{n}{LT}\PYG{p}{,}\PYG{n}{ASM}\PYG{p}{)}\PYG{p}{,}
\PYG{o}{/}\PYG{o}{/} \PYG{n}{PARM}\PYG{o}{=}\PYG{l+s+s1}{\PYGZsq{}}\PYG{l+s+s1}{LIST,LET,XREF,NORENT}\PYG{l+s+s1}{\PYGZsq{}}
\PYG{o}{/}\PYG{o}{/}\PYG{n}{SYSPRINT} \PYG{n}{DD} \PYG{n}{SYSOUT}\PYG{o}{=}\PYG{o}{\PYGZam{}}\PYG{n}{OUTC}
\PYG{o}{/}\PYG{o}{/}\PYG{n}{SYSLIB} \PYG{n}{DD} \PYG{n}{DSN}\PYG{o}{=}\PYG{o}{\PYGZam{}}\PYG{n}{QUAL}\PYG{o}{.}\PYG{o}{.}\PYG{n}{LOADLIB}\PYG{p}{,}\PYG{n}{DISP}\PYG{o}{=}\PYG{n}{SHR}
\PYG{o}{/}\PYG{o}{/}\PYG{n}{SYSUT1} \PYG{n}{DD} \PYG{n}{UNIT}\PYG{o}{=}\PYG{o}{\PYGZam{}}\PYG{n}{WORK}\PYG{p}{,}\PYG{n}{DCB}\PYG{o}{=}\PYG{n}{BLKSIZE}\PYG{o}{=}\PYG{l+m+mi}{1024}\PYG{p}{,}\PYG{n}{SPACE}\PYG{o}{=}\PYG{p}{(}\PYG{l+m+mi}{1024}\PYG{p}{,}\PYG{p}{(}\PYG{l+m+mi}{200}\PYG{p}{,}\PYG{l+m+mi}{20}\PYG{p}{)}\PYG{p}{)}
\PYG{o}{/}\PYG{o}{/}\PYG{n}{SYSLIN} \PYG{n}{DD} \PYG{n}{DSN}\PYG{o}{=}\PYG{o}{\PYGZam{}}\PYG{o}{\PYGZam{}}\PYG{n}{LOADSET}\PYG{p}{,}\PYG{n}{DISP}\PYG{o}{=}\PYG{p}{(}\PYG{n}{OLD}\PYG{p}{,}\PYG{n}{DELETE}\PYG{p}{)}
\PYG{o}{/}\PYG{o}{/}\PYG{n}{SYSLMOD} \PYG{n}{DD} \PYG{n}{DSN}\PYG{o}{=}\PYG{o}{\PYGZam{}}\PYG{n}{QUAL}\PYG{o}{.}\PYG{o}{.}\PYG{n}{LOADLIB}\PYG{p}{(}\PYG{o}{\PYGZam{}}\PYG{n}{MEMBER}\PYG{p}{)}\PYG{p}{,}\PYG{n}{DISP}\PYG{o}{=}\PYG{n}{SHR}
\PYG{o}{/}\PYG{o}{/} \PYG{n}{PEND}
\PYG{o}{/}\PYG{o}{/}\PYG{n}{VIRTASM} \PYG{n}{EXEC} \PYG{n}{ASMTCT}
\end{sphinxVerbatim}

\sphinxAtStartPar
\sphinxstyleemphasis{VIRTCT assembly in z/OS}

\index{Virtel TCT@\spxentry{Virtel TCT}!Assembly \sphinxhyphen{} z/VSE@\spxentry{Assembly \sphinxhyphen{} z/VSE}}\index{Assembly \sphinxhyphen{} z/VSE@\spxentry{Assembly \sphinxhyphen{} z/VSE}!Virtel TCT@\spxentry{Virtel TCT}}\ignorespaces 

\subsection{z/VSE example}
\label{\detokenize{Installation_Guide:z-vse-example}}\label{\detokenize{Installation_Guide:index-150}}
\sphinxAtStartPar
A sample job for assembling the VIRTCT is supplied on the installation tape:

\begin{sphinxVerbatim}[commandchars=\\\{\}]
* \PYGZdl{}\PYGZdl{} JOB JNM=VIRTCT,CLASS=0,DISP=D
* \PYGZdl{}\PYGZdl{} LST DISP=D,CLASS=Q
// JOB VIRTCT
// DLBL VIRTvrr,\PYGZsq{}VSE.VIRTvrr.LIBRARY\PYGZsq{},,VSAM,CAT=VSESPUC
// LIBDEF PHASE,CATALOG=VIRTvrr.SUBLIB
// LIBDEF SOURCE,SEARCH=(VIRTvrr.SUBLIB,PRD1.MACLIB)
// OPTION CATAL,NODECK,ALIGN
        PHASE VIRTCT01,*
// EXEC ASSEMBLY,SIZE=512K
* \PYGZdl{}\PYGZdl{} SLI ICCF=(VIRTCT01),LIB=0037
/*
// EXEC LNKEDT,SIZE=512K
/*
/\PYGZam{}
* \PYGZdl{}\PYGZdl{} EOJ
\end{sphinxVerbatim}

\sphinxAtStartPar
\sphinxstyleemphasis{VIRTCT assembly in z/VSE}

\index{Virtel TCT@\spxentry{Virtel TCT}!Dynamic Overrides@\spxentry{Dynamic Overrides}}\index{Dynamic Overrides@\spxentry{Dynamic Overrides}!Virtel TCT@\spxentry{Virtel TCT}}\ignorespaces 

\subsection{Dynamic VIRTCT Overrides}
\label{\detokenize{Installation_Guide:dynamic-virtct-overrides}}\label{\detokenize{Installation_Guide:index-151}}
\sphinxAtStartPar
Data may be passed to the VIRTEL procedure via the start command which allows the value of certain parameters in the VIRTCT (APPLID, MQ1, MQ2) to be modified. For example the VIRTEL started task procedure should contain the TCT, APPLID, and VTOVER parameters which are substituted into the PARM as shown below:

\begin{sphinxVerbatim}[commandchars=\\\{\}]
\PYG{o}{/}\PYG{o}{/}\PYG{n}{VIRTEL} \PYG{n}{EXEC} \PYG{n}{PGM}\PYG{o}{=}\PYG{n}{VIR0000}\PYG{p}{,}
\PYG{o}{/}\PYG{o}{/} \PYG{n}{TIME}\PYG{o}{=}\PYG{l+m+mi}{1440}\PYG{p}{,}\PYG{n}{REGION}\PYG{o}{=}\PYG{l+m+mi}{8192}\PYG{n}{K}\PYG{p}{,}
\PYG{o}{/}\PYG{o}{/} \PYG{n}{PARM}\PYG{o}{=}\PYG{p}{(}\PYG{o}{\PYGZam{}}\PYG{n}{TCT}\PYG{p}{,}\PYG{o}{\PYGZam{}}\PYG{n}{APPLID}\PYG{p}{,}\PYG{o}{\PYGZam{}}\PYG{n}{VTOVER}\PYG{p}{)}
\end{sphinxVerbatim}

\sphinxAtStartPar
When starting VIRTEL, you may specify values for the TCT, APPLID, and VTOVER on the start command as shown in the example below:

\begin{sphinxVerbatim}[commandchars=\\\{\}]
\PYG{n}{S} \PYG{n}{VIRTEL}\PYG{p}{,}\PYG{n}{TCT}\PYG{o}{=}\PYG{l+m+mi}{01}\PYG{p}{,}\PYG{n}{APPLID}\PYG{o}{=}\PYG{n}{AA4AVIRX}\PYG{p}{,}\PYG{n}{VTOVER}\PYG{o}{=}\PYG{l+s+s1}{\PYGZsq{}}\PYG{l+s+s1}{12345}\PYG{l+s+s1}{\PYGZsq{}}
\end{sphinxVerbatim}

\sphinxAtStartPar
After loading VIRTCT01, VIRTEL will then:
\begin{itemize}
\item {} 
\sphinxAtStartPar
replace the APPLID variable by the value AA4AVIRX

\item {} 
\sphinxAtStartPar
use the value of the VTOVER variable to replace the specified substitution characters \% in certain VIRTCT parameters by characters extracted from the VTOVER value. This functionality requires that the VIRTCT should contain:
\begin{itemize}
\item {} 
\sphinxAtStartPar
a new parameter VTOVER=VTDYNAM

\item {} 
\sphinxAtStartPar
a new table VTDYNAM consisting of macros VTOVERH and VTOVER

\item {} 
\sphinxAtStartPar
the presence of one or more \% characters in certain VIRTCT parameters which will be substituted by the corresponding characters from the VTOVER parameter specified on the start command.

\end{itemize}

\end{itemize}

\sphinxAtStartPar
For example, if the fourth character of the MQ1 and MQ2 Queue Manager name and the second three characters of the MQ2 Queue Name are variable, the parameters may be defined in the VIRTCT as shown below:

\begin{sphinxVerbatim}[commandchars=\\\{\}]
\PYG{n}{MQ1}\PYG{o}{=}\PYG{p}{(}\PYG{n}{CSQ}\PYG{o}{\PYGZpc{}}\PYG{p}{)}\PYG{p}{,}         \PYG{o}{\PYGZhy{}}\PYG{o}{\PYGZgt{}} \PYG{n}{wild} \PYG{n}{char} \PYG{o+ow}{in} \PYG{n}{MQ1} \PYG{n}{parm} \PYG{o}{*}
\PYG{n}{MQ2}\PYG{o}{=}\PYG{p}{(}\PYG{n}{CSQ}\PYG{o}{\PYGZpc{}}\PYG{p}{,}\PYG{l+s+s1}{\PYGZsq{}}\PYG{l+s+s1}{A}\PYG{l+s+si}{\PYGZpc{}\PYGZpc{}}\PYG{l+s+s1}{\PYGZpc{}}\PYG{l+s+s1}{\PYGZsq{}}\PYG{p}{)}\PYG{p}{,}  \PYG{o}{\PYGZhy{}}\PYG{o}{\PYGZgt{}} \PYG{n}{wild} \PYG{n}{char} \PYG{o+ow}{in} \PYG{n}{MQ2} \PYG{n}{parm} \PYG{o}{*}
\PYG{n}{VTOVER}\PYG{o}{=}\PYG{n}{VTDYNAM}\PYG{p}{,}     \PYG{o}{\PYGZhy{}}\PYG{o}{\PYGZgt{}} \PYG{n}{new} \PYG{n}{VIRTCT} \PYG{n}{parm} \PYG{o}{*}
\end{sphinxVerbatim}

\sphinxAtStartPar
At the end of the VIRTCT, define the VTDYNAM table with the rules for substituting data from the VTOVER parameter. For each parameter, specify the substitution character(s) to look for (TARGET), and the position relative to zero of the characters to be extracted from the VTOVER parameter (FROM), as shown in the example below:

\begin{sphinxVerbatim}[commandchars=\\\{\}]
\PYG{n}{VTDYNAM} \PYG{n}{VTOVERH}                    \PYG{o}{\PYGZhy{}}\PYG{o}{\PYGZgt{}} \PYG{n}{new} \PYG{n}{table} \PYG{n}{after} \PYG{n}{the} \PYG{n}{VIRTCT}
\PYG{n}{MQ1}     \PYG{n}{VTOVER} \PYG{n}{PARM}\PYG{o}{=}\PYG{n}{MQ1}\PYG{p}{,}           \PYG{n}{modify} \PYG{n}{MQ1}\PYG{p}{(}\PYG{l+m+mi}{1}\PYG{p}{)}                 \PYG{o}{*}
        \PYG{n}{TARGET}\PYG{o}{=}\PYG{l+s+s1}{\PYGZsq{}}\PYG{l+s+s1}{\PYGZpc{}}\PYG{l+s+s1}{\PYGZsq{}}\PYG{p}{,}                \PYG{n}{find} \PYG{o}{\PYGZpc{}} \PYG{n}{char}                   \PYG{o}{*}
        \PYG{n}{FROM}\PYG{o}{=}\PYG{l+m+mi}{0}\PYG{p}{,}                    \PYG{n}{replace} \PYG{o}{\PYGZpc{}} \PYG{k}{with} \PYG{n}{VTOVER}\PYG{p}{(}\PYG{l+m+mi}{0}\PYG{p}{)}      \PYG{o}{*}
        \PYG{n}{ERRORC}\PYG{o}{=}\PYG{l+m+mi}{12} \PYG{n}{Virtel}           \PYG{n}{RC} \PYG{k}{if} \PYG{n}{replace} \PYG{n}{failed}
\PYG{n}{MQ21}    \PYG{n}{VTOVER} \PYG{n}{PARM}\PYG{o}{=}\PYG{p}{(}\PYG{n}{MQ2}\PYG{p}{,}\PYG{l+m+mi}{1}\PYG{p}{)}\PYG{p}{,}\PYG{n}{TARGET}\PYG{o}{=}\PYG{l+s+s1}{\PYGZsq{}}\PYG{l+s+s1}{\PYGZpc{}}\PYG{l+s+s1}{\PYGZsq{}}\PYG{p}{,}\PYG{n}{FROM}\PYG{o}{=}\PYG{l+m+mi}{1}
\PYG{n}{MQ22}    \PYG{n}{VTOVER} \PYG{n}{PARM}\PYG{o}{=}\PYG{p}{(}\PYG{n}{MQ2}\PYG{p}{,}\PYG{l+m+mi}{2}\PYG{p}{)}\PYG{p}{,}\PYG{n}{TARGET}\PYG{o}{=}\PYG{l+s+s1}{\PYGZsq{}}\PYG{l+s+si}{\PYGZpc{}\PYGZpc{}}\PYG{l+s+s1}{\PYGZpc{}}\PYG{l+s+s1}{\PYGZsq{}}\PYG{p}{,}\PYG{n}{FROM}\PYG{o}{=}\PYG{l+m+mi}{2}
\end{sphinxVerbatim}

\sphinxAtStartPar
With these definitions and VTOVER=’12345’, the MQ1 and MQ2 parameters of the VIRTCT will have the values shown below:

\begin{sphinxVerbatim}[commandchars=\\\{\}]
\PYG{n}{MQ1}\PYG{o}{=}\PYG{n}{CSQ1}\PYG{p}{,}
\PYG{n}{MQ2}\PYG{o}{=}\PYG{p}{(}\PYG{n}{CSQ2}\PYG{p}{,}\PYG{l+s+s1}{\PYGZsq{}}\PYG{l+s+s1}{A345}\PYG{l+s+s1}{\PYGZsq{}}\PYG{p}{)}\PYG{p}{,}
\end{sphinxVerbatim}

\sphinxAtStartPar
If an error occurs during substitution, VIRTEL will issue message VIR0025E indicating the error code specified in the ERRORC parameter of the VTOVER macro.

\newpage

\index{Virtel TCT@\spxentry{Virtel TCT}!Applying patches@\spxentry{Applying patches}}\index{Applying patches@\spxentry{Applying patches}!Virtel TCT@\spxentry{Virtel TCT}}\ignorespaces 

\section{Applying Patches Via The VIRTCT}
\label{\detokenize{Installation_Guide:applying-patches-via-the-virtct}}\label{\detokenize{Installation_Guide:index-152}}
\sphinxAtStartPar
The “ZAPH parameter” of the VIRTCT allows one or more patches to be applied to the VIRTEL kernel after the resident modules have been loaded into memory at startup. This parameter is intended to be used only under the advice of Syspertec technical support personnel.

\sphinxAtStartPar
For example, if the VIRTCT contains the parameter:

\begin{sphinxVerbatim}[commandchars=\\\{\}]
\PYG{n}{ZAPH}\PYG{o}{=}\PYG{n}{MYPTFS}\PYG{p}{,} \PYG{o}{*}
\end{sphinxVerbatim}

\sphinxAtStartPar
then it refers to the table MYPTFS coded after the VIRTERM macro, for example:

\begin{sphinxVerbatim}[commandchars=\\\{\}]
\PYG{n}{MYPTFS}   \PYG{n}{ZAPH}
\PYG{n}{USER4611} \PYG{n}{ZAPD} \PYG{n}{VIR00TAB}\PYG{p}{,}\PYG{o}{+}\PYG{l+m+mi}{246}\PYG{n}{D}\PYG{p}{,}\PYG{n}{EC}\PYG{p}{,}\PYG{l+m+mi}{1}\PYG{n}{B}\PYG{p}{,}\PYG{l+s+s1}{\PYGZsq{}}\PYG{l+s+s1}{ USERMOD TO TRANSLATE TABLE }\PYG{l+s+s1}{\PYGZsq{}}
\PYG{n}{PTF4618}  \PYG{n}{ZAPD} \PYG{n}{VIR0011D}\PYG{p}{,}\PYG{o}{+}\PYG{l+m+mi}{092}\PYG{n}{A}\PYG{p}{,}\PYG{n}{C98C}\PYG{p}{,}\PYG{n}{C984}\PYG{p}{,}\PYG{l+s+s1}{\PYGZsq{}}\PYG{l+s+s1}{ TEMP FIX FOR SEND\PYGZdl{} }\PYG{l+s+s1}{\PYGZsq{}}
\end{sphinxVerbatim}

\sphinxAtStartPar
The format of each ZAPD instruction is as follows:

\begin{sphinxVerbatim}[commandchars=\\\{\}]
\PYG{n}{label} \PYG{n}{ZAPD} \PYG{n}{progname}\PYG{p}{,}\PYG{o}{+}\PYG{n}{offset}\PYG{p}{,}\PYG{n}{verify}\PYG{p}{,}\PYG{n}{replace}\PYG{p}{,}\PYG{l+s+s1}{\PYGZsq{}}\PYG{l+s+s1}{desc}\PYG{l+s+s1}{\PYGZsq{}}
\end{sphinxVerbatim}

\sphinxAtStartPar
\sphinxstylestrong{label} \sphinxhyphen{} PTF identifier for message VIR0066I

\sphinxAtStartPar
\sphinxstylestrong{progname} \sphinxhyphen{} program name

\sphinxAtStartPar
\sphinxstylestrong{offset} \sphinxhyphen{} offset into program

\sphinxAtStartPar
\sphinxstylestrong{verify} \sphinxhyphen{} verify value (hexadecimal digits)

\sphinxAtStartPar
\sphinxstylestrong{replace} \sphinxhyphen{} replacement value (hexadecimal digits)

\sphinxAtStartPar
\sphinxstylestrong{desc} \sphinxhyphen{} (optional) description for message VIR0066I

\index{VIRCONF Utility@\spxentry{VIRCONF Utility}}\ignorespaces 

\chapter{VIRCONF UTILITY}
\label{\detokenize{Installation_Guide:virconf-utility}}\label{\detokenize{Installation_Guide:index-153}}

\section{Introduction}
\label{\detokenize{Installation_Guide:id9}}
\sphinxAtStartPar
The VIRCONF utility program allows a batch job to manage the VIRARBO file, which is the main configuration file for VIRTEL. VIRCONF allows you to:
\begin{itemize}
\item {} 
\sphinxAtStartPar
Upload a new VIRARBO file using SYSIN cards.

\item {} 
\sphinxAtStartPar
Unload an existing VIRARBO file.

\item {} 
\sphinxAtStartPar
Add, replace, or suppress one or more definitions within an existing VIRARBO file

\item {} 
\sphinxAtStartPar
Create new définitions as SYSIN cards using an existing VIRARBO file

\item {} 
\sphinxAtStartPar
Scan a SYSIN cards file for checking the right syntax

\end{itemize}

\index{VIRCONF Utility@\spxentry{VIRCONF Utility}!Define and Upload(z/VSE)@\spxentry{Define and Upload(z/VSE)}}\index{Define and Upload(z/VSE)@\spxentry{Define and Upload(z/VSE)}!VIRCONF Utility@\spxentry{VIRCONF Utility}}\ignorespaces 

\section{JCL}
\label{\detokenize{Installation_Guide:jcl}}\label{\detokenize{Installation_Guide:index-154}}
\sphinxAtStartPar
Below are some JCL examples to define and upload a new VIRARBO file:\sphinxhyphen{}

\sphinxAtStartPar
z/VSE

\begin{sphinxVerbatim}[commandchars=\\\{\}]
* \PYGZdl{}\PYGZdl{} JOB JNM=VIRCONF,CLASS=0,DISP=D
* \PYGZdl{}\PYGZdl{} LST DISP=D,CLASS=V,DEST=(,SPTUSER)
// JOB VIRCONF DEFINE AND LOAD VIRARBO
// DLBL IJSYSUC,\PYGZsq{}VSESP.USER.CATALOG\PYGZsq{},,VSAM
// EXEC IDCAMS,SIZE=AUTO
        DEFINE CLUSTER(NAME(VIRTEL.TESTARBO.KSDS) \PYGZhy{}
              RECORDS(500 100) SHAREOPTIONS (2 3) \PYGZhy{}
              RECSZ (600 4089) KEYS (9 0) \PYGZhy{}
              VOLUMES (DOSRES) TO (99366))\PYGZhy{}
        DATA (NAME(VIRTEL.TESTARBO.KSDS.DATA)) \PYGZhy{}
        INDEX (NAME(VIRTEL.TESTARBO.KSDS.INDEX)) \PYGZhy{}
              CATALOG(VSESP.USER.CATALOG)
IF LASTCC NE 0 THEN CANCEL JOB
/*
// LIBDEF *,SEARCH=(VIRTvrr.SUBLIB)
// DLBL VIRARBO,\PYGZsq{}VIRTEL.TESTARBO.KSDS\PYGZsq{},,VSAM,CAT=VSESPUC
// EXEC VIRCONF,PARM=\PYGZsq{}LOAD\PYGZsq{}
        (insert sysin control statements here)
/*
/\PYGZam{}
* \PYGZdl{}\PYGZdl{} EOJ
\end{sphinxVerbatim}

\sphinxAtStartPar
\sphinxstyleemphasis{VIRCONF JCL in z/VSE to define and upload a new VIRARBO file}

\index{VIRCONF Utility@\spxentry{VIRCONF Utility}!Define and Upload(z/OS)@\spxentry{Define and Upload(z/OS)}}\index{Define and Upload(z/OS)@\spxentry{Define and Upload(z/OS)}!VIRCONF Utility@\spxentry{VIRCONF Utility}}\ignorespaces 
\sphinxAtStartPar
z/OS

\begin{sphinxVerbatim}[commandchars=\\\{\}]
\PYG{o}{/}\PYG{o}{/}\PYG{n}{VIRCONF} \PYG{n}{JOB} \PYG{n}{CLASS}\PYG{o}{=}\PYG{n}{A}\PYG{p}{,}\PYG{n}{MSGCLASS}\PYG{o}{=}\PYG{n}{X}\PYG{p}{,}\PYG{n}{MSGLEVEL}\PYG{o}{=}\PYG{p}{(}\PYG{l+m+mi}{1}\PYG{p}{,}\PYG{l+m+mi}{1}\PYG{p}{)}\PYG{p}{,}\PYG{n}{NOTIFY}\PYG{o}{=}\PYG{o}{\PYGZam{}}\PYG{n}{SYSUID}
\PYG{o}{/}\PYG{o}{/}\PYG{o}{*} \PYG{n}{THIS} \PYG{n}{JOB} \PYG{n}{DEFINES} \PYG{n}{AND} \PYG{n}{LOADS} \PYG{n}{A} \PYG{n}{NEW} \PYG{n}{ARBO} \PYG{n}{FILE}
\PYG{o}{/}\PYG{o}{/}\PYG{n}{DEFARBO} \PYG{n}{EXEC} \PYG{n}{PGM}\PYG{o}{=}\PYG{n}{IDCAMS}\PYG{p}{,}\PYG{n}{REGION}\PYG{o}{=}\PYG{l+m+mi}{2}\PYG{n}{M}
\PYG{o}{/}\PYG{o}{/}\PYG{n}{SYSPRINT} \PYG{n}{DD} \PYG{n}{SYSOUT}\PYG{o}{=}\PYG{o}{*}
        \PYG{n}{DEFINE} \PYG{n}{CLUSTER}\PYG{p}{(}\PYG{n}{NAME}\PYG{p}{(}\PYG{n}{VIRTEL}\PYG{o}{.}\PYG{n}{TEST}\PYG{o}{.}\PYG{n}{ARBO}\PYG{p}{)} \PYG{o}{\PYGZhy{}}
            \PYG{n}{KEYS}\PYG{p}{(}\PYG{l+m+mi}{9} \PYG{l+m+mi}{0}\PYG{p}{)} \PYG{n}{RECSZ}\PYG{p}{(}\PYG{l+m+mi}{100} \PYG{l+m+mi}{4089}\PYG{p}{)} \PYG{n}{FSPC}\PYG{p}{(}\PYG{l+m+mi}{10} \PYG{l+m+mi}{10}\PYG{p}{)} \PYG{o}{\PYGZhy{}}
            \PYG{n}{VOL}\PYG{p}{(}\PYG{n}{SPT001}\PYG{p}{)} \PYG{n}{REC}\PYG{p}{(}\PYG{l+m+mi}{250}\PYG{p}{,}\PYG{l+m+mi}{50}\PYG{p}{)} \PYG{n}{SHR}\PYG{p}{(}\PYG{l+m+mi}{2}\PYG{p}{)} \PYG{n}{SPEED}\PYG{p}{)} \PYG{o}{\PYGZhy{}}
        \PYG{n}{DATA} \PYG{p}{(}\PYG{n}{NAME}\PYG{p}{(}\PYG{n}{VIRTEL}\PYG{o}{.}\PYG{n}{TEST}\PYG{o}{.}\PYG{n}{ARBO}\PYG{o}{.}\PYG{n}{DATA}\PYG{p}{)} \PYG{n}{CISZ}\PYG{p}{(}\PYG{l+m+mi}{4096}\PYG{p}{)}\PYG{p}{)} \PYG{o}{\PYGZhy{}}
        \PYG{n}{INDEX} \PYG{p}{(}\PYG{n}{NAME}\PYG{p}{(}\PYG{n}{VIRTEL}\PYG{o}{.}\PYG{n}{TEST}\PYG{o}{.}\PYG{n}{ARBO}\PYG{o}{.}\PYG{n}{INDEX}\PYG{p}{)}\PYG{p}{)}
\PYG{o}{/}\PYG{o}{/}\PYG{n}{RELOAD} \PYG{n}{EXEC} \PYG{n}{PGM}\PYG{o}{=}\PYG{n}{VIRCONF}\PYG{p}{,}\PYG{n}{COND}\PYG{o}{=}\PYG{p}{(}\PYG{l+m+mi}{0}\PYG{p}{,}\PYG{n}{NE}\PYG{p}{,}\PYG{n}{DEFARBO}\PYG{p}{)}\PYG{p}{,}\PYG{n}{PARM}\PYG{o}{=}\PYG{n}{LOAD}
\PYG{o}{/}\PYG{o}{/}\PYG{n}{STEPLIB} \PYG{n}{DD} \PYG{n}{DSN}\PYG{o}{=}\PYG{n}{yourqual}\PYG{o}{.}\PYG{n}{VIRTvrr}\PYG{o}{.}\PYG{n}{LOADLIB}\PYG{p}{,}\PYG{n}{DISP}\PYG{o}{=}\PYG{n}{SHR}
\PYG{o}{/}\PYG{o}{/}\PYG{n}{SYSPRINT} \PYG{n}{DD} \PYG{n}{SYSOUT}\PYG{o}{=}\PYG{o}{*}
\PYG{o}{/}\PYG{o}{/}\PYG{n}{VIRARBO} \PYG{n}{DD} \PYG{n}{DSN}\PYG{o}{=}\PYG{n}{VIRTEL}\PYG{o}{.}\PYG{n}{TEST}\PYG{o}{.}\PYG{n}{ARBO}\PYG{p}{,}\PYG{n}{DISP}\PYG{o}{=}\PYG{n}{SHR}
\PYG{o}{/}\PYG{o}{/}\PYG{n}{SYSIN} \PYG{n}{DD} \PYG{n}{DSN}\PYG{o}{=}\PYG{o}{\PYGZam{}}\PYG{n}{SYSUID}\PYG{o}{.}\PYG{o}{.}\PYG{n}{VIRCONF}\PYG{o}{.}\PYG{n}{SYSIN}\PYG{p}{,}\PYG{n}{DISP}\PYG{o}{=}\PYG{n}{SHR}
\end{sphinxVerbatim}

\sphinxAtStartPar
\sphinxstyleemphasis{VIRCONF JCL in z/OS to define and upload a new VIRARBO file}

\sphinxAtStartPar
When VIRCONF is executed with PARM=LOAD, control cards are read from SYSIPT (z/VSE) or SYSIN (z/OS) and are loaded into the VIRARBO file.


\subsection{Updating a VIRARBO file}
\label{\detokenize{Installation_Guide:updating-a-virarbo-file}}
\sphinxAtStartPar
Below are some JCL examples to add, replace, or delete one or more definitions from an existing VIRARBO file:\sphinxhyphen{}

\index{VIRCONF Utility@\spxentry{VIRCONF Utility}!Update(z/VSE)@\spxentry{Update(z/VSE)}}\index{Update(z/VSE)@\spxentry{Update(z/VSE)}!VIRCONF Utility@\spxentry{VIRCONF Utility}}\ignorespaces 
\sphinxAtStartPar
z/VSE

\begin{sphinxVerbatim}[commandchars=\\\{\}]
* \PYGZdl{}\PYGZdl{} JOB JNM=VIRCONF,CLASS=0,DISP=D
* \PYGZdl{}\PYGZdl{} LST DISP=D,CLASS=V,DEST=(,SPTUSER)
// JOB VIRCONF UPDATE VIRARBO
// LIBDEF *,SEARCH=(VIRTvrr.SUBLIB)
// DLBL VIRARBO,\PYGZsq{}VIRTEL4.VIRARBO.KSDS\PYGZsq{},,VSAM,CAT=VSESPUC
// EXEC VIRCONF,PARM=\PYGZsq{}LOAD\PYGZsq{}
        LINE  ID=A\PYGZhy{}XOT,
              NAME=XOT\PYGZhy{}IP30,
              PARTNER=192.168.0.80:1998,
              LOCADDR=192.168.229.30:1998,
              DESC=\PYGZsq{}Connections via Cisco router\PYGZsq{},
              TERMINAL=XOTF,INOUT=3,TYPE=TCP1,PROTOCOL=XOT,
              WINSZ=3,PKTSZ=128,RETRY=10,TIMEOUT=10,ACTION=0
        RULE ID=AX200CFT,LINE=A\PYGZhy{}XOT,STATUS=ACTIVE,
              DESC=\PYGZdq{}XOT\PYGZhy{}\PYGZgt{}AntiPCNE\PYGZhy{}\PYGZgt{}CFT (CUD0=X\PYGZsq{}C0\PYGZsq{})\PYGZdq{},
              ENTRY=APCFT,CUD0=(BEGIN,C0)
        DELETE TYPE=RULE,ID=AX100CFT
/*
/\PYGZam{}
* \PYGZdl{}\PYGZdl{} EOJ
\end{sphinxVerbatim}

\sphinxAtStartPar
\sphinxstyleemphasis{VIRCONF JCL in z/VSE to update a VIRARBO file}

\index{VIRCONF Utility@\spxentry{VIRCONF Utility}!Update(z/OS)@\spxentry{Update(z/OS)}}\index{Update(z/OS)@\spxentry{Update(z/OS)}!VIRCONF Utility@\spxentry{VIRCONF Utility}}\ignorespaces 
\sphinxAtStartPar
z/OS

\begin{sphinxVerbatim}[commandchars=\\\{\}]
\PYG{o}{/}\PYG{o}{/}\PYG{n}{VIRCONF} \PYG{n}{JOB} \PYG{n}{CLASS}\PYG{o}{=}\PYG{n}{A}\PYG{p}{,}\PYG{n}{MSGCLASS}\PYG{o}{=}\PYG{n}{X}\PYG{p}{,}\PYG{n}{MSGLEVEL}\PYG{o}{=}\PYG{p}{(}\PYG{l+m+mi}{1}\PYG{p}{,}\PYG{l+m+mi}{1}\PYG{p}{)}\PYG{p}{,}\PYG{n}{NOTIFY}\PYG{o}{=}\PYG{o}{\PYGZam{}}\PYG{n}{SYSUID}
\PYG{o}{/}\PYG{o}{/}\PYG{o}{*} \PYG{n}{THIS} \PYG{n}{JOB} \PYG{n}{UPDATES} \PYG{n}{AN} \PYG{n}{ARBO} \PYG{n}{FILE}
\PYG{o}{/}\PYG{o}{/}\PYG{n}{UPDARBO} \PYG{n}{EXEC} \PYG{n}{PGM}\PYG{o}{=}\PYG{n}{VIRCONF}\PYG{p}{,}\PYG{n}{PARM}\PYG{o}{=}\PYG{n}{LOAD}
\PYG{o}{/}\PYG{o}{/}\PYG{n}{STEPLIB} \PYG{n}{DD} \PYG{n}{DSN}\PYG{o}{=}\PYG{n}{yourqual}\PYG{o}{.}\PYG{n}{VIRTvrr}\PYG{o}{.}\PYG{n}{LOADLIB}\PYG{p}{,}\PYG{n}{DISP}\PYG{o}{=}\PYG{n}{SHR}
\PYG{o}{/}\PYG{o}{/}\PYG{n}{SYSPRINT} \PYG{n}{DD} \PYG{n}{SYSOUT}\PYG{o}{=}\PYG{o}{*}
\PYG{o}{/}\PYG{o}{/}\PYG{n}{VIRARBO} \PYG{n}{DD} \PYG{n}{DSN}\PYG{o}{=}\PYG{n}{VIRTEL}\PYG{o}{.}\PYG{n}{TEST}\PYG{o}{.}\PYG{n}{ARBO}\PYG{p}{,}\PYG{n}{DISP}\PYG{o}{=}\PYG{n}{SHR}
\PYG{o}{/}\PYG{o}{/}\PYG{n}{SYSIN} \PYG{n}{DD} \PYG{o}{*}
        \PYG{n}{DELETE} \PYG{n}{TYPE}\PYG{o}{=}\PYG{n}{RULE}\PYG{p}{,}\PYG{n}{ID}\PYG{o}{=}\PYG{n}{R0000300}       \PYG{n}{Delete} \PYG{n}{rule}
        \PYG{n}{DELETE} \PYG{n}{TYPE}\PYG{o}{=}\PYG{n}{LINE}\PYG{p}{,}\PYG{n}{ID}\PYG{o}{=}\PYG{n}{X}\PYG{o}{\PYGZhy{}}\PYG{n}{HTTP}         \PYG{n}{Delete} \PYG{n}{line}
        \PYG{n}{DELETE} \PYG{n}{TYPE}\PYG{o}{=}\PYG{n}{TRANSACT}\PYG{p}{,}\PYG{n}{ID}\PYG{o}{=}\PYG{n}{INITV}\PYG{o}{\PYGZhy{}}\PYG{l+m+mi}{00}   \PYG{n}{Delete} \PYG{n}{transaction}
        \PYG{n}{DELETE} \PYG{n}{TYPE}\PYG{o}{=}\PYG{n}{ENTRY}\PYG{p}{,}\PYG{n}{ID}\PYG{o}{=}\PYG{n}{INITVTAM}      \PYG{n}{Delete} \PYG{n}{Entry} \PYG{n}{Point}
        \PYG{n}{DELETE} \PYG{n}{TYPE}\PYG{o}{=}\PYG{n}{SUBDIR}\PYG{p}{,}\PYG{n}{ID}\PYG{o}{=}\PYG{n}{EXC}\PYG{o}{\PYGZhy{}}\PYG{n}{DIR}      \PYG{n}{Delete} \PYG{n}{sub} \PYG{n}{directory}
        \PYG{n}{DELETE} \PYG{n}{TYPE}\PYG{o}{=}\PYG{n}{USER}\PYG{p}{,}\PYG{n}{ID}\PYG{o}{=}\PYG{n}{SAMPUSER}       \PYG{n}{Delete} \PYG{n}{User}
        \PYG{n}{USER} \PYG{n}{ID}\PYG{o}{=}\PYG{n}{BLOGGS}\PYG{p}{,}\PYG{n}{NAME}\PYG{o}{=}\PYG{l+s+s1}{\PYGZsq{}}\PYG{l+s+s1}{JOE BLOGGS}\PYG{l+s+s1}{\PYGZsq{}}\PYG{p}{,}\PYG{n}{DEPT}\PYG{o}{=}\PYG{n}{VIRTEL}\PYG{p}{,}\PYG{n}{PASSWORD}\PYG{o}{=}\PYG{n}{JOE}\PYG{p}{,}
            \PYG{n}{PROFILE}\PYG{o}{=}\PYG{p}{(}\PYG{n}{APPLICS}\PYG{p}{,}\PYG{n}{PC}\PYG{p}{,}\PYG{n}{REPERT}\PYG{p}{,}\PYG{n}{SECURITE}\PYG{p}{,}
            \PYG{n}{SERVEUR}\PYG{p}{,}\PYG{n}{SERVEXT}\PYG{p}{,}\PYG{n}{WEBMASTR}\PYG{p}{)}
\PYG{o}{/}\PYG{o}{*}
\end{sphinxVerbatim}

\sphinxAtStartPar
\sphinxstyleemphasis{VIRCONF JCL in z/OS to update a VIRARBO file}

\sphinxAtStartPar
Submitting VIRCONF with PARM=LOAD for an existing VIRARBO file allows definitions to be added, replaced, or deleted, while keeping existing definitions in the VIRARBO file. Using PARM=’LOAD,NOREPL’ parameter allows only new definitions to be added, while keeping existing definitions. In this case, VIRCONF will ignore any statement with the same name as existing definitions, returning a zero return code, except if another error was encountered.

\index{VIRCONF Utility@\spxentry{VIRCONF Utility}!Unloading the ARBO@\spxentry{Unloading the ARBO}}\index{Unloading the ARBO@\spxentry{Unloading the ARBO}!VIRCONF Utility@\spxentry{VIRCONF Utility}}\ignorespaces 

\subsection{Unloading a VIRARBO file}
\label{\detokenize{Installation_Guide:unloading-a-virarbo-file}}\label{\detokenize{Installation_Guide:index-158}}
\sphinxAtStartPar
Below are some JCL examples to obtain existing VIRARBO definitions in the form of control cards:\sphinxhyphen{}

\begin{sphinxVerbatim}[commandchars=\\\{\}]
* \PYGZdl{}\PYGZdl{} JOB JNM=VIRCONF,CLASS=0,DISP=D
* \PYGZdl{}\PYGZdl{} LST DISP=D,CLASS=V,DEST=(,SPTUSER)
* \PYGZdl{}\PYGZdl{} PUN DISP=D,CLASS=W,DEST=(,SPTUSER)
// JOB VIRCONF UNLOAD VIRARBO TO SYSPCH
// LIBDEF *,SEARCH=VIRTvrr.SUBLIB
// DLBL VIRARBO,\PYGZsq{}VIRTEL.TESTARBO.KSDS\PYGZsq{},,VSAM,CAT=VSESPUC
// EXEC VIRCONF,PARM=\PYGZsq{}UNLOAD\PYGZsq{}
/\PYGZam{}
* \PYGZdl{}\PYGZdl{} EOJ
\end{sphinxVerbatim}

\sphinxAtStartPar
\sphinxstyleemphasis{VIRCONF JCL in z/VSE to unload a VIRARBO file}

\begin{sphinxVerbatim}[commandchars=\\\{\}]
\PYG{o}{/}\PYG{o}{/}\PYG{n}{VIRCONF} \PYG{n}{JOB} \PYG{n}{CLASS}\PYG{o}{=}\PYG{n}{A}\PYG{p}{,}\PYG{n}{MSGCLASS}\PYG{o}{=}\PYG{n}{X}\PYG{p}{,}\PYG{n}{MSGLEVEL}\PYG{o}{=}\PYG{p}{(}\PYG{l+m+mi}{1}\PYG{p}{,}\PYG{l+m+mi}{1}\PYG{p}{)}\PYG{p}{,}\PYG{n}{NOTIFY}\PYG{o}{=}\PYG{o}{\PYGZam{}}\PYG{n}{SYSUID}
\PYG{o}{/}\PYG{o}{/}\PYG{o}{*} \PYG{n}{THIS} \PYG{n}{JOB} \PYG{n}{UNLOADS} \PYG{n}{AN} \PYG{n}{ARBO} \PYG{n}{FILE} \PYG{n}{TO} \PYG{n}{SYSPUNCH}
\PYG{o}{/}\PYG{o}{/}\PYG{n}{UNLOAD} \PYG{n}{EXEC} \PYG{n}{PGM}\PYG{o}{=}\PYG{n}{VIRCONF}\PYG{p}{,}\PYG{n}{PARM}\PYG{o}{=}\PYG{n}{UNLOAD}
\PYG{o}{/}\PYG{o}{/}\PYG{n}{STEPLIB} \PYG{n}{DD} \PYG{n}{DSN}\PYG{o}{=}\PYG{n}{yourqual}\PYG{o}{.}\PYG{n}{VIRTvrr}\PYG{o}{.}\PYG{n}{LOADLIB}\PYG{p}{,}\PYG{n}{DISP}\PYG{o}{=}\PYG{n}{SHR}
\PYG{o}{/}\PYG{o}{/}\PYG{n}{SYSPRINT} \PYG{n}{DD} \PYG{n}{SYSOUT}\PYG{o}{=}\PYG{o}{*}
\PYG{o}{/}\PYG{o}{/}\PYG{n}{VIRARBO} \PYG{n}{DD} \PYG{n}{DSN}\PYG{o}{=}\PYG{n}{VIRTEL}\PYG{o}{.}\PYG{n}{TEST}\PYG{o}{.}\PYG{n}{ARBO}\PYG{p}{,}\PYG{n}{DISP}\PYG{o}{=}\PYG{n}{SHR}\PYG{p}{,}\PYG{n}{AMP}\PYG{o}{=}\PYG{p}{(}\PYG{l+s+s1}{\PYGZsq{}}\PYG{l+s+s1}{RMODE31=NONE}\PYG{l+s+s1}{\PYGZsq{}}\PYG{p}{)}
\PYG{o}{/}\PYG{o}{/}\PYG{n}{SYSPUNCH} \PYG{n}{DD} \PYG{n}{DSN}\PYG{o}{=}\PYG{o}{\PYGZam{}}\PYG{n}{SYSUID}\PYG{o}{.}\PYG{o}{.}\PYG{n}{VIRCONF}\PYG{o}{.}\PYG{n}{SYSIN}\PYG{p}{,}\PYG{n}{DISP}\PYG{o}{=}\PYG{p}{(}\PYG{p}{,}\PYG{n}{CATLG}\PYG{p}{)}\PYG{p}{,}\PYG{n}{UNIT}\PYG{o}{=}\PYG{n}{SYSDA}\PYG{p}{,}
\PYG{o}{/}\PYG{o}{/} \PYG{n}{SPACE}\PYG{o}{=}\PYG{p}{(}\PYG{n}{TRK}\PYG{p}{,}\PYG{p}{(}\PYG{l+m+mi}{5}\PYG{p}{,}\PYG{l+m+mi}{1}\PYG{p}{)}\PYG{p}{)}\PYG{p}{,}\PYG{n}{DCB}\PYG{o}{=}\PYG{p}{(}\PYG{n}{RECFM}\PYG{o}{=}\PYG{n}{FB}\PYG{p}{,}\PYG{n}{LRECL}\PYG{o}{=}\PYG{l+m+mi}{80}\PYG{p}{,}\PYG{n}{BLKSIZE}\PYG{o}{=}\PYG{l+m+mi}{6080}\PYG{p}{)}
\end{sphinxVerbatim}

\sphinxAtStartPar
\sphinxstyleemphasis{VIRCONF JCL in z/OS to unload a VIRARBO file}

\sphinxAtStartPar
When VIRCONF is run with the PARM=UNLOAD parameter, the existing VIRARBO definitions are converted into control cards and are written to SYSPCH (z/VSE) or SYSPUNCH (z/OS). The created cards issued by VIRCONF may be edited and then reused with another VIRCONF job with the PARM=LOAD parameter.

\index{VIRCONF Utility@\spxentry{VIRCONF Utility}!Syntax checking@\spxentry{Syntax checking}}\index{Syntax checking@\spxentry{Syntax checking}!VIRCONF Utility@\spxentry{VIRCONF Utility}}\ignorespaces 

\subsection{Verify control card syntax}
\label{\detokenize{Installation_Guide:verify-control-card-syntax}}\label{\detokenize{Installation_Guide:index-159}}
\sphinxAtStartPar
Below are some JCL examples to verify the control card syntax:\sphinxhyphen{}

\begin{sphinxVerbatim}[commandchars=\\\{\}]
* \PYGZdl{}\PYGZdl{} JOB JNM=VIRCONF,CLASS=0,DISP=D
* \PYGZdl{}\PYGZdl{} LST DISP=D,CLASS=V,DEST=(,SPTUSER)
// JOB VIRCONF SYNTAX CHECK
// LIBDEF *,SEARCH=(VIRTvrr.SUBLIB)
// EXEC VIRCONF,PARM=\PYGZsq{}SCAN\PYGZsq{}
        (insert sysin control statements here)
/*
/\PYGZam{}
* \PYGZdl{}\PYGZdl{} EOJ
\end{sphinxVerbatim}

\sphinxAtStartPar
\sphinxstyleemphasis{VIRCONF JCL in z/VSE for syntax verification}

\begin{sphinxVerbatim}[commandchars=\\\{\}]
\PYG{o}{/}\PYG{o}{/}\PYG{n}{VIRCONF} \PYG{n}{JOB} \PYG{n}{CLASS}\PYG{o}{=}\PYG{n}{A}\PYG{p}{,}\PYG{n}{MSGCLASS}\PYG{o}{=}\PYG{n}{X}\PYG{p}{,}\PYG{n}{MSGLEVEL}\PYG{o}{=}\PYG{p}{(}\PYG{l+m+mi}{1}\PYG{p}{,}\PYG{l+m+mi}{1}\PYG{p}{)}\PYG{p}{,}\PYG{n}{NOTIFY}\PYG{o}{=}\PYG{o}{\PYGZam{}}\PYG{n}{SYSUID}
\PYG{o}{/}\PYG{o}{/}\PYG{o}{*} \PYG{n}{VIRCONF} \PYG{n}{SYNTAX} \PYG{n}{CHECK}
\PYG{o}{/}\PYG{o}{/}\PYG{n}{CONFCHK} \PYG{n}{EXEC} \PYG{n}{PGM}\PYG{o}{=}\PYG{n}{VIRCONF}\PYG{p}{,}\PYG{n}{PARM}\PYG{o}{=}\PYG{n}{SCAN}
\PYG{o}{/}\PYG{o}{/}\PYG{n}{STEPLIB} \PYG{n}{DD} \PYG{n}{DSN}\PYG{o}{=}\PYG{n}{yourqual}\PYG{o}{.}\PYG{n}{VIRTvrr}\PYG{o}{.}\PYG{n}{LOADLIB}\PYG{p}{,}\PYG{n}{DISP}\PYG{o}{=}\PYG{n}{SHR}
\PYG{o}{/}\PYG{o}{/}\PYG{n}{SYSPRINT} \PYG{n}{DD} \PYG{n}{SYSOUT}\PYG{o}{=}\PYG{o}{*}
\PYG{o}{/}\PYG{o}{/}\PYG{n}{SYSIN} \PYG{n}{DD} \PYG{o}{*}
        \PYG{p}{(}\PYG{n}{insert} \PYG{n}{sysin} \PYG{n}{control} \PYG{n}{statements} \PYG{n}{here}\PYG{p}{)}
\PYG{o}{/}\PYG{o}{*}
\end{sphinxVerbatim}

\sphinxAtStartPar
\sphinxstyleemphasis{VIRCONF JCL in z/OS for syntax verification}

\sphinxAtStartPar
Submitting the VIRCONF program with PARM=SCAN allows you to scan the SYSIPT (z/VSE) or SYSIN (z/OS) cards for potential syntax errors. There is no access to the VIRCONF file.

\index{VIRCONF Utility@\spxentry{VIRCONF Utility}!Multi\sphinxhyphen{}language Support@\spxentry{Multi\sphinxhyphen{}language Support}}\index{Multi\sphinxhyphen{}language Support@\spxentry{Multi\sphinxhyphen{}language Support}!VIRCONF Utility@\spxentry{VIRCONF Utility}}\ignorespaces 

\subsection{Multi\sphinxhyphen{}language support}
\label{\detokenize{Installation_Guide:multi-language-support}}\label{\detokenize{Installation_Guide:index-160}}
\sphinxAtStartPar
When uploading the VIRARBO file, VIRCONF may select one among several versions of a control card, based on the LANG=xx parameter defined in the JCL. In this way, the same SYSIN file may be used to generate several different language versions of the VIRARBO file. For example:

\begin{sphinxVerbatim}[commandchars=\\\{\}]
* \PYGZdl{}\PYGZdl{} JOB JNM=VIRCONF,CLASS=0,DISP=D
// JOB VIRCONF LOAD VIRARBO
// LIBDEF *,SEARCH=(VIRTvrr.SUBLIB)
// DLBL VIRARBO,\PYGZsq{}VIRTEL.TESTARBO.KSDS\PYGZsq{},,VSAM,CAT=VSESPUC
// EXEC VIRCONF,PARM=\PYGZsq{}LOAD,LANG=FR\PYGZsq{}
        TRANSACT ID=PC\PYGZhy{}0003, \PYGZhy{}
        (FR) NAME=\PYGZsq{}Entrée\PYGZsq{}, \PYGZhy{}
        (FR) DESC=\PYGZdq{}Gestion des points d\PYGZsq{}entrée\PYGZdq{}, \PYGZhy{}
        (EN) NAME=\PYGZsq{}Entry\PYGZsq{}, \PYGZhy{}
        (EN) DESC=\PYGZsq{}Entry point management\PYGZsq{}, \PYGZhy{}
        APPL=VIR0044, \PYGZhy{}
        TYPE=2, \PYGZhy{}
        STARTUP=1
/*
/\PYGZam{}
* \PYGZdl{}\PYGZdl{} EOJ
\end{sphinxVerbatim}

\sphinxAtStartPar
\sphinxstyleemphasis{VIRCONF JCL in z/VSE for multi\sphinxhyphen{}language upload}

\begin{sphinxVerbatim}[commandchars=\\\{\}]
\PYG{o}{/}\PYG{o}{/}\PYG{n}{VIRCONF} \PYG{n}{JOB} \PYG{n}{CLASS}\PYG{o}{=}\PYG{n}{A}\PYG{p}{,}\PYG{n}{MSGCLASS}\PYG{o}{=}\PYG{n}{X}\PYG{p}{,}\PYG{n}{MSGLEVEL}\PYG{o}{=}\PYG{p}{(}\PYG{l+m+mi}{1}\PYG{p}{,}\PYG{l+m+mi}{1}\PYG{p}{)}\PYG{p}{,}\PYG{n}{NOTIFY}\PYG{o}{=}\PYG{o}{\PYGZam{}}\PYG{n}{SYSUID}
\PYG{o}{/}\PYG{o}{/}\PYG{o}{*} \PYG{n}{LOAD} \PYG{n}{AN} \PYG{n}{ARBO} \PYG{n}{FILE} \PYG{n}{USING} \PYG{n}{MULTILINGUAL} \PYG{n}{SOURCE}
\PYG{o}{/}\PYG{o}{/}\PYG{n}{VIRCONF} \PYG{n}{EXEC} \PYG{n}{PGM}\PYG{o}{=}\PYG{n}{VIRCONF}\PYG{p}{,}\PYG{n}{PARM}\PYG{o}{=}\PYG{l+s+s1}{\PYGZsq{}}\PYG{l+s+s1}{LOAD,LANG=EN}\PYG{l+s+s1}{\PYGZsq{}}
\PYG{o}{/}\PYG{o}{/}\PYG{n}{STEPLIB} \PYG{n}{DD} \PYG{n}{DSN}\PYG{o}{=}\PYG{n}{yourqual}\PYG{o}{.}\PYG{n}{VIRTvrr}\PYG{o}{.}\PYG{n}{LOADLIB}\PYG{p}{,}\PYG{n}{DISP}\PYG{o}{=}\PYG{n}{SHR}
\PYG{o}{/}\PYG{o}{/}\PYG{n}{SYSPRINT} \PYG{n}{DD} \PYG{n}{SYSOUT}\PYG{o}{=}\PYG{o}{*}
\PYG{o}{/}\PYG{o}{/}\PYG{n}{VIRARBO} \PYG{n}{DD} \PYG{n}{DSN}\PYG{o}{=}\PYG{n}{VIRTEL}\PYG{o}{.}\PYG{n}{TEST}\PYG{o}{.}\PYG{n}{ARBO}\PYG{p}{,}\PYG{n}{DISP}\PYG{o}{=}\PYG{n}{SHR}
\PYG{o}{/}\PYG{o}{/}\PYG{n}{SYSIN} \PYG{n}{DD} \PYG{o}{*}
        \PYG{n}{TRANSACT} \PYG{n}{ID}\PYG{o}{=}\PYG{n}{PC}\PYG{o}{\PYGZhy{}}\PYG{l+m+mi}{0003}\PYG{p}{,} \PYG{o}{\PYGZhy{}}
        \PYG{p}{(}\PYG{n}{FR}\PYG{p}{)} \PYG{n}{NAME}\PYG{o}{=}\PYG{l+s+s1}{\PYGZsq{}}\PYG{l+s+s1}{Entrée}\PYG{l+s+s1}{\PYGZsq{}}\PYG{p}{,} \PYG{o}{\PYGZhy{}}
        \PYG{p}{(}\PYG{n}{FR}\PYG{p}{)} \PYG{n}{DESC}\PYG{o}{=}\PYG{l+s+s2}{\PYGZdq{}}\PYG{l+s+s2}{Gestion des points d}\PYG{l+s+s2}{\PYGZsq{}}\PYG{l+s+s2}{entrée}\PYG{l+s+s2}{\PYGZdq{}}\PYG{p}{,} \PYG{o}{\PYGZhy{}}
        \PYG{p}{(}\PYG{n}{EN}\PYG{p}{)} \PYG{n}{NAME}\PYG{o}{=}\PYG{l+s+s1}{\PYGZsq{}}\PYG{l+s+s1}{Entry}\PYG{l+s+s1}{\PYGZsq{}}\PYG{p}{,} \PYG{o}{\PYGZhy{}}
        \PYG{p}{(}\PYG{n}{EN}\PYG{p}{)} \PYG{n}{DESC}\PYG{o}{=}\PYG{l+s+s1}{\PYGZsq{}}\PYG{l+s+s1}{Entry point management}\PYG{l+s+s1}{\PYGZsq{}}\PYG{p}{,} \PYG{o}{\PYGZhy{}}
        \PYG{n}{APPL}\PYG{o}{=}\PYG{n}{VIR0044}\PYG{p}{,} \PYG{o}{\PYGZhy{}}
        \PYG{n}{TYPE}\PYG{o}{=}\PYG{l+m+mi}{2}\PYG{p}{,} \PYG{o}{\PYGZhy{}}
        \PYG{n}{STARTUP}\PYG{o}{=}\PYG{l+m+mi}{1}
\PYG{o}{/}\PYG{o}{*}
\end{sphinxVerbatim}

\sphinxAtStartPar
\sphinxstyleemphasis{VIRCONF JCL in z/OS for multi\sphinxhyphen{}language upload}

\newpage

\index{VIRCONF Utility@\spxentry{VIRCONF Utility}!Control Cards@\spxentry{Control Cards}}\index{Control Cards@\spxentry{Control Cards}!VIRCONF Utility@\spxentry{VIRCONF Utility}}\ignorespaces 

\section{VIRCONF Control Cards}
\label{\detokenize{Installation_Guide:virconf-control-cards}}\label{\detokenize{Installation_Guide:index-161}}

\subsection{VIRCONF control card syntax}
\label{\detokenize{Installation_Guide:virconf-control-card-syntax}}
\sphinxAtStartPar
The control card syntax for VIRCONF is similar to the syntax for JCL.
\begin{itemize}
\item {} 
\sphinxAtStartPar
Each instruction begins on a new card

\item {} 
\sphinxAtStartPar
One instruction consists of an “operation code”, followed by a blank space, followed by one or more parameters, followed by an optional comment

\item {} 
\sphinxAtStartPar
The parameters use the keyword=value form and parameters are separated with a comma

\item {} 
\sphinxAtStartPar
The parameters are ended by a blank character; anything after this first blank is treated as a comment.

\item {} 
\sphinxAtStartPar
A card beginning with a “*” is treated as a comment

\item {} 
\sphinxAtStartPar
A completely blank card is treated as a comment

\item {} 
\sphinxAtStartPar
A card beginning with two characters between parenthesis, for instance (EN), will be processed only if these two characters match the value of the LANG= parameter specified in the JCL PARM

\item {} 
\sphinxAtStartPar
Instructions must be coded between columns 1 to 71. Column 72, if non\sphinxhyphen{}blank, means that the instruction continues on the next card

\item {} 
\sphinxAtStartPar
If a parameter is terminated by a comma followed by a blank, the instruction continues at the first non\sphinxhyphen{}blank character of next card. A non\sphinxhyphen{}blank in column 72 is optional in this case

\item {} 
\sphinxAtStartPar
A character string between apostrophes or quotes which goes over column 71 may be continued on the next card, by putting a non\sphinxhyphen{}blank character in column 72 and by continuing the string starting on column 16 of the next card

\item {} 
\sphinxAtStartPar
Each instruction must have at least one “ID=” parameter which is used as a key to identify the described entity

\item {} 
\sphinxAtStartPar
The values of parameters may in general contain letters (A\sphinxhyphen{}Z in upper case), digits (0\sphinxhyphen{}9), and special characters(.+\&\$*\sphinxhyphen{}/\%\_?:@). Some parameters also allow values which contain other special characters, letters in lower case, and blank characters, and in this case the value must be coded as a character string enclosed in either quotes or apostrophe.

\end{itemize}

\sphinxAtStartPar
You can generate some examples by submitting a job using the PARM=UNLOAD parameter (see “Unloading a VIRARBO file”, page 86) for a specific VIRARBO file, for instance the one delivered as VIRARBO base in the standard installation process.

\index{VIRCONF Control Cards@\spxentry{VIRCONF Control Cards}!APPLIC statement@\spxentry{APPLIC statement}}\index{APPLIC statement@\spxentry{APPLIC statement}!VIRCONF Control Cards@\spxentry{VIRCONF Control Cards}}\ignorespaces 

\subsection{APPLIC}
\label{\detokenize{Installation_Guide:applic}}\label{\detokenize{Installation_Guide:index-162}}
\sphinxAtStartPar
This operation adds or replaces an APPLIC entity in the VIRARBO file. The parameters correspond to the various items
described under the heading “Applications Management” on page 123


\begin{savenotes}\sphinxattablestart
\sphinxthistablewithglobalstyle
\centering
\begin{tabulary}{\linewidth}[t]{TTT}
\sphinxtoprule
\sphinxstyletheadfamily 
\sphinxAtStartPar
Parameter
&\sphinxstyletheadfamily 
\sphinxAtStartPar
Item
&\sphinxstyletheadfamily 
\sphinxAtStartPar
Remarks
\\
\sphinxmidrule
\sphinxtableatstartofbodyhook
\sphinxAtStartPar
ID=
&
\sphinxAtStartPar
Name
&\\
\sphinxhline
\sphinxAtStartPar
DESC=
&
\sphinxAtStartPar
Description
&
\sphinxAtStartPar
Quotes allowed
\\
\sphinxhline
\sphinxAtStartPar
LOGON=
&
\sphinxAtStartPar
Logon
&
\sphinxAtStartPar
Quotes allowed
\\
\sphinxhline
\sphinxAtStartPar
STATUS=
&
\sphinxAtStartPar
Status
&
\sphinxAtStartPar
Quotes allowed
\\
\sphinxbottomrule
\end{tabulary}
\sphinxtableafterendhook\par
\sphinxattableend\end{savenotes}

\index{VIRCONF Control Cards@\spxentry{VIRCONF Control Cards}!DELETE statement@\spxentry{DELETE statement}}\index{DELETE statement@\spxentry{DELETE statement}!VIRCONF Control Cards@\spxentry{VIRCONF Control Cards}}\ignorespaces 

\subsection{DELETE}
\label{\detokenize{Installation_Guide:delete}}\label{\detokenize{Installation_Guide:index-163}}
\sphinxAtStartPar
This operation deletes an entity of the specified type from the VIRARBO file.


\begin{savenotes}\sphinxattablestart
\sphinxthistablewithglobalstyle
\centering
\begin{tabulary}{\linewidth}[t]{TTT}
\sphinxtoprule
\sphinxstyletheadfamily 
\sphinxAtStartPar
Parameter
&\sphinxstyletheadfamily 
\sphinxAtStartPar
Item
&\sphinxstyletheadfamily 
\sphinxAtStartPar
Remarks
\\
\sphinxmidrule
\sphinxtableatstartofbodyhook
\sphinxAtStartPar
TYPE=
&
\sphinxAtStartPar
Entity type
&
\sphinxAtStartPar
LINE, RULE, TERMINAL etc.
\\
\sphinxhline
\sphinxAtStartPar
ID=
&
\sphinxAtStartPar
Entity name
&
\sphinxAtStartPar
Quotes allowed
\\
\sphinxbottomrule
\end{tabulary}
\sphinxtableafterendhook\par
\sphinxattableend\end{savenotes}

\index{VIRCONF Control Cards@\spxentry{VIRCONF Control Cards}!DEPT statement@\spxentry{DEPT statement}}\index{DEPT statement@\spxentry{DEPT statement}!VIRCONF Control Cards@\spxentry{VIRCONF Control Cards}}\ignorespaces 

\subsection{DEPT}
\label{\detokenize{Installation_Guide:dept}}\label{\detokenize{Installation_Guide:index-164}}
\sphinxAtStartPar
This operation adds or replaces a DEPT entity in the VIRARBO file. The parameters correspond to the various items
described under the heading “Create a department” or “Profile lent to a department”.


\begin{savenotes}\sphinxattablestart
\sphinxthistablewithglobalstyle
\centering
\begin{tabulary}{\linewidth}[t]{TTT}
\sphinxtoprule
\sphinxstyletheadfamily 
\sphinxAtStartPar
Parameter
&\sphinxstyletheadfamily 
\sphinxAtStartPar
Item
&\sphinxstyletheadfamily 
\sphinxAtStartPar
Remarks
\\
\sphinxmidrule
\sphinxtableatstartofbodyhook
\sphinxAtStartPar
ID=
&
\sphinxAtStartPar
Department
&\\
\sphinxhline
\sphinxAtStartPar
DESC=
&
\sphinxAtStartPar
Description
&
\sphinxAtStartPar
Quotes allowed
\\
\sphinxhline
\sphinxAtStartPar
OWNER=
&
\sphinxAtStartPar
Responsible
&\\
\sphinxhline
\sphinxAtStartPar
PROFILE
&
\sphinxAtStartPar
Lent profiles list
&
\sphinxAtStartPar
Separated by commas, the whole between parenthesis
\\
\sphinxbottomrule
\end{tabulary}
\sphinxtableafterendhook\par
\sphinxattableend\end{savenotes}

\index{VIRCONF Control Cards@\spxentry{VIRCONF Control Cards}!ENTRY statement@\spxentry{ENTRY statement}}\index{ENTRY statement@\spxentry{ENTRY statement}!VIRCONF Control Cards@\spxentry{VIRCONF Control Cards}}\ignorespaces 

\subsection{ENTRY}
\label{\detokenize{Installation_Guide:entry}}\label{\detokenize{Installation_Guide:index-165}}
\sphinxAtStartPar
This operation adds or replaces an ENTRY entity in the VIRARBO file. The parameters correspond to the various items
described under the heading “Parameters of the Entry Point”.


\begin{savenotes}\sphinxattablestart
\sphinxthistablewithglobalstyle
\centering
\begin{tabulary}{\linewidth}[t]{TTT}
\sphinxtoprule
\sphinxstyletheadfamily 
\sphinxAtStartPar
Parameter
&\sphinxstyletheadfamily 
\sphinxAtStartPar
Item
&\sphinxstyletheadfamily 
\sphinxAtStartPar
Remarks
\\
\sphinxmidrule
\sphinxtableatstartofbodyhook
\sphinxAtStartPar
ID=
&
\sphinxAtStartPar
Name
&\\
\sphinxhline
\sphinxAtStartPar
DESC=
&
\sphinxAtStartPar
Description
&
\sphinxAtStartPar
Quotes allowed
\\
\sphinxhline
\sphinxAtStartPar
TRANSACT=
&
\sphinxAtStartPar
Transaction
&\\
\sphinxhline
\sphinxAtStartPar
ARBO=
&
\sphinxAtStartPar
Arborescence
&\\
\sphinxhline
\sphinxAtStartPar
ENDPAGE=
&
\sphinxAtStartPar
Last Page
&\\
\sphinxhline
\sphinxAtStartPar
TRANSP=
&
\sphinxAtStartPar
Transparency
&\\
\sphinxhline
\sphinxAtStartPar
TIMEOUT=
&
\sphinxAtStartPar
Responsible
&
\sphinxAtStartPar
Numeric
\\
\sphinxhline
\sphinxAtStartPar
EMUL=
&
\sphinxAtStartPar
Emulation
&\\
\sphinxhline
\sphinxAtStartPar
SCENDIR=
&
\sphinxAtStartPar
Directory for
Scenarios
&\\
\sphinxhline
\sphinxAtStartPar
SIGNON=
&
\sphinxAtStartPar
Signon Program
&\\
\sphinxhline
\sphinxAtStartPar
MENU=
&
\sphinxAtStartPar
Menu Program
&\\
\sphinxhline
\sphinxAtStartPar
IDENT=
&
\sphinxAtStartPar
Identification or
scenario or program
&\\
\sphinxhline
\sphinxAtStartPar
COMPR3=
&
\sphinxAtStartPar
Type 3 compression
&\\
\sphinxhline
\sphinxAtStartPar
IDREQ=
&
\sphinxAtStartPar
Mandatory
Identification
&\\
\sphinxhline
\sphinxAtStartPar
SWAP=
&
\sphinxAtStartPar
3270 Swap key
&\\
\sphinxhline
\sphinxAtStartPar
EXTCOLOR=
&
\sphinxAtStartPar
Extended colors
&\\
\sphinxbottomrule
\end{tabulary}
\sphinxtableafterendhook\par
\sphinxattableend\end{savenotes}

\index{VIRCONF Control Cards@\spxentry{VIRCONF Control Cards}!INDEX statement@\spxentry{INDEX statement}}\index{INDEX statement@\spxentry{INDEX statement}!VIRCONF Control Cards@\spxentry{VIRCONF Control Cards}}\ignorespaces 

\subsection{INDEX}
\label{\detokenize{Installation_Guide:index}}\label{\detokenize{Installation_Guide:index-166}}
\sphinxAtStartPar
This operation adds or replaces an INDEX entity in the VIRARBO file. The parameters correspond to the various items described under the heading “Updating the keywords”.


\begin{savenotes}\sphinxattablestart
\sphinxthistablewithglobalstyle
\centering
\begin{tabulary}{\linewidth}[t]{TTT}
\sphinxtoprule
\sphinxstyletheadfamily 
\sphinxAtStartPar
Parameter
&\sphinxstyletheadfamily 
\sphinxAtStartPar
Item
&\sphinxstyletheadfamily 
\sphinxAtStartPar
Remarks
\\
\sphinxmidrule
\sphinxtableatstartofbodyhook
\sphinxAtStartPar
ID=
&
\sphinxAtStartPar
Keyword
&
\sphinxAtStartPar
Quotes allowed
\\
\sphinxhline
\sphinxAtStartPar
TARGET=
&
\sphinxAtStartPar
Target Node \sphinxhyphen{}
Choice
&
\sphinxAtStartPar
{[}1{]} ; Quotes allowed
First 6 characters: Target Node.
Characters 7\sphinxhyphen{}8: Choice
\\
\sphinxbottomrule
\end{tabulary}
\sphinxtableafterendhook\par
\sphinxattableend\end{savenotes}

\begin{sphinxadmonition}{note}{Note:}
\sphinxAtStartPar
{[}1{]} the target node name has less than 6 characters, it must be padded with blanks and enclosed in quotes.
\end{sphinxadmonition}

\index{VIRCONF Control Cards@\spxentry{VIRCONF Control Cards}!LINE statement@\spxentry{LINE statement}}\index{LINE statement@\spxentry{LINE statement}!VIRCONF Control Cards@\spxentry{VIRCONF Control Cards}}\ignorespaces 

\subsection{LINE}
\label{\detokenize{Installation_Guide:line}}\label{\detokenize{Installation_Guide:index-167}}
\sphinxAtStartPar
This operation adds or replaces a LINE entity in the VIRARBO file. The parameters correspond to the various items described under the heading “Parameters of the line”.


\begin{savenotes}\sphinxattablestart
\sphinxthistablewithglobalstyle
\centering
\begin{tabulary}{\linewidth}[t]{TTT}
\sphinxtoprule
\sphinxstyletheadfamily 
\sphinxAtStartPar
Parameter
&\sphinxstyletheadfamily 
\sphinxAtStartPar
Item
&\sphinxstyletheadfamily 
\sphinxAtStartPar
Remarks
\\
\sphinxmidrule
\sphinxtableatstartofbodyhook
\sphinxAtStartPar
ID=
&
\sphinxAtStartPar
Internal name
&\\
\sphinxhline
\sphinxAtStartPar
NAME=
&
\sphinxAtStartPar
External name
&\\
\sphinxhline
\sphinxAtStartPar
PARTNER=
&
\sphinxAtStartPar
Remote ident
&\\
\sphinxhline
\sphinxAtStartPar
LOCADDR=
&
\sphinxAtStartPar
Local ident
&\\
\sphinxhline
\sphinxAtStartPar
DESC=
&
\sphinxAtStartPar
Description
&
\sphinxAtStartPar
Quotes allowed
\\
\sphinxhline
\sphinxAtStartPar
TERMINAL=
&
\sphinxAtStartPar
Prefix
&\\
\sphinxhline
\sphinxAtStartPar
ENTRY=
&
\sphinxAtStartPar
Entry Point
&\\
\sphinxhline
\sphinxAtStartPar
TYPE=
&
\sphinxAtStartPar
Line Type
&\\
\sphinxhline
\sphinxAtStartPar
INOUT=
&
\sphinxAtStartPar
Possible Calls
&\\
\sphinxhline
\sphinxAtStartPar
COND=
&
\sphinxAtStartPar
Startup
prerequisite
&
\sphinxAtStartPar
Quotes allowed
\\
\sphinxhline
\sphinxAtStartPar
PROTOCOL=
&
\sphinxAtStartPar
Protocol program
&\\
\sphinxhline
\sphinxAtStartPar
SECURITY=
&
\sphinxAtStartPar
Security program
&\\
\sphinxhline
\sphinxAtStartPar
TIMEOUT=
&
\sphinxAtStartPar
Time out
&
\sphinxAtStartPar
Numeric
\\
\sphinxhline
\sphinxAtStartPar
ACTION=
&
\sphinxAtStartPar
Action if
time out
&
\sphinxAtStartPar
Numeric
\\
\sphinxhline
\sphinxAtStartPar
WINSZ=
&
\sphinxAtStartPar
Window
&
\sphinxAtStartPar
Numeric
\\
\sphinxhline
\sphinxAtStartPar
PKTSZ=
&
\sphinxAtStartPar
Packet
&
\sphinxAtStartPar
Numeric
\\
\sphinxhline
\sphinxAtStartPar
PAD=
&
\sphinxAtStartPar
Pad
&
\sphinxAtStartPar
Numeric
\\
\sphinxhline
\sphinxAtStartPar
TRAN=
&
\sphinxAtStartPar
Tran
&\\
\sphinxhline
\sphinxAtStartPar
RETRY=
&
\sphinxAtStartPar
Retries
&
\sphinxAtStartPar
Numeric
\\
\sphinxhline
\sphinxAtStartPar
DELAY=
&
\sphinxAtStartPar
Delay
&
\sphinxAtStartPar
Numeric
\\
\sphinxhline
\sphinxAtStartPar
UNIQUEP=
&
\sphinxAtStartPar
Unique Partner {[}1{]}
&
\sphinxAtStartPar
Y or N
\\
\sphinxhline
\sphinxAtStartPar
SHAREDA=
&
\sphinxAtStartPar
Shared address
time out
&
\sphinxAtStartPar
Y or N
\\
\sphinxbottomrule
\end{tabulary}
\sphinxtableafterendhook\par
\sphinxattableend\end{savenotes}

\begin{sphinxadmonition}{note}{Note:}
\sphinxAtStartPar
{[}1{]} This parameter is available only in VIRCONF
\end{sphinxadmonition}

\index{VIRCONF Control Cards@\spxentry{VIRCONF Control Cards}!NODE statement@\spxentry{NODE statement}}\index{NODE statement@\spxentry{NODE statement}!VIRCONF Control Cards@\spxentry{VIRCONF Control Cards}}\ignorespaces 

\subsection{NODE}
\label{\detokenize{Installation_Guide:node}}\label{\detokenize{Installation_Guide:index-168}}
\sphinxAtStartPar
This operation adds or replaces a NODE entity in the VIRARBO file. The parameters correspond to the various items described under the heading ”Defining a native node”.


\begin{savenotes}\sphinxattablestart
\sphinxthistablewithglobalstyle
\centering
\begin{tabulary}{\linewidth}[t]{TTT}
\sphinxtoprule
\sphinxstyletheadfamily 
\sphinxAtStartPar
Parameter
&\sphinxstyletheadfamily 
\sphinxAtStartPar
Item
&\sphinxstyletheadfamily 
\sphinxAtStartPar
Remarks
\\
\sphinxmidrule
\sphinxtableatstartofbodyhook
\sphinxAtStartPar
ID=
&
\sphinxAtStartPar
Name of node
&
\sphinxAtStartPar
Quotes allowed
\\
\sphinxhline
\sphinxAtStartPar
PAGE=
&
\sphinxAtStartPar
Generic of associated Pages
&
\sphinxAtStartPar
Quotes allowed
\\
\sphinxhline
\sphinxAtStartPar
GUIDE=
&
\sphinxAtStartPar
Generic of associated guides
&
\sphinxAtStartPar
Quotes allowed
\\
\sphinxhline
\sphinxAtStartPar
CHILD=
&
\sphinxAtStartPar
Generic of children
&
\sphinxAtStartPar
Quotes allowed
\\
\sphinxbottomrule
\end{tabulary}
\sphinxtableafterendhook\par
\sphinxattableend\end{savenotes}

\index{VIRCONF Control Cards@\spxentry{VIRCONF Control Cards}!PC statement@\spxentry{PC statement}}\index{PC statement@\spxentry{PC statement}!VIRCONF Control Cards@\spxentry{VIRCONF Control Cards}}\ignorespaces 

\subsection{PC}
\label{\detokenize{Installation_Guide:pc}}\label{\detokenize{Installation_Guide:index-169}}
\sphinxAtStartPar
This operation adds or replaces a PC entity in the VIRARBO file. The parameters correspond to the various items described under the heading “PC management”.


\begin{savenotes}\sphinxattablestart
\sphinxthistablewithglobalstyle
\centering
\begin{tabulary}{\linewidth}[t]{TTT}
\sphinxtoprule
\sphinxstyletheadfamily 
\sphinxAtStartPar
Parameter
&\sphinxstyletheadfamily 
\sphinxAtStartPar
Item
&\sphinxstyletheadfamily 
\sphinxAtStartPar
Remarks
\\
\sphinxmidrule
\sphinxtableatstartofbodyhook
\sphinxAtStartPar
ID=
&
\sphinxAtStartPar
PC Name
&\\
\sphinxhline
\sphinxAtStartPar
DESC=
&
\sphinxAtStartPar
Description
&
\sphinxAtStartPar
Quotes allowed
\\
\sphinxhline
\sphinxAtStartPar
DISABLE=
&
\sphinxAtStartPar
Block
&
\sphinxAtStartPar
X=Blocked connections
\\
\sphinxhline
\sphinxAtStartPar
SUBDIR=
&
\sphinxAtStartPar
Assoc. Directory
&\\
\sphinxhline
\sphinxAtStartPar
PASSCODE=
&
\sphinxAtStartPar
Password
&
\sphinxAtStartPar
Quotes allowed
\\
\sphinxbottomrule
\end{tabulary}
\sphinxtableafterendhook\par
\sphinxattableend\end{savenotes}

\index{VIRCONF Control Cards@\spxentry{VIRCONF Control Cards}!PROFILE statement@\spxentry{PROFILE statement}}\index{PROFILE statement@\spxentry{PROFILE statement}!VIRCONF Control Cards@\spxentry{VIRCONF Control Cards}}\ignorespaces 

\subsection{PROFILE}
\label{\detokenize{Installation_Guide:profile}}\label{\detokenize{Installation_Guide:index-170}}
\sphinxAtStartPar
This operation adds or replaces one entity with PROFILE entity in the VIRARBO file. The parameters correspond to the various items described under the heading “Defining a profile”.


\begin{savenotes}\sphinxattablestart
\sphinxthistablewithglobalstyle
\centering
\begin{tabulary}{\linewidth}[t]{TTT}
\sphinxtoprule
\sphinxstyletheadfamily 
\sphinxAtStartPar
Parameter
&\sphinxstyletheadfamily 
\sphinxAtStartPar
Item
&\sphinxstyletheadfamily 
\sphinxAtStartPar
Remarks
\\
\sphinxmidrule
\sphinxtableatstartofbodyhook
\sphinxAtStartPar
ID=
&
\sphinxAtStartPar
Profile
&\\
\sphinxhline
\sphinxAtStartPar
DESC=
&
\sphinxAtStartPar
Description
&
\sphinxAtStartPar
Quotes allowed
\\
\sphinxhline
\sphinxAtStartPar
DEPT=
&
\sphinxAtStartPar
Department
&
\sphinxAtStartPar
Quotes allowed
\\
\sphinxhline
\sphinxAtStartPar
TYPE=
&
\sphinxAtStartPar
List of given resources
&
\sphinxAtStartPar
Separated by commas, and surrounded by parenthesesUnused field
\\
\sphinxbottomrule
\end{tabulary}
\sphinxtableafterendhook\par
\sphinxattableend\end{savenotes}

\index{VIRCONF Control Cards@\spxentry{VIRCONF Control Cards}!RESOURCE statement@\spxentry{RESOURCE statement}}\index{RESOURCE statement@\spxentry{RESOURCE statement}!VIRCONF Control Cards@\spxentry{VIRCONF Control Cards}}\ignorespaces 

\subsection{RESOURCE}
\label{\detokenize{Installation_Guide:resource}}\label{\detokenize{Installation_Guide:index-171}}
\sphinxAtStartPar
This operation adds or replaces a RESOURCE entity in the VIRARBO file. The parameters correspond to the various items described under the heading “Defining a resource”.


\begin{savenotes}\sphinxattablestart
\sphinxthistablewithglobalstyle
\centering
\begin{tabulary}{\linewidth}[t]{TTT}
\sphinxtoprule
\sphinxstyletheadfamily 
\sphinxAtStartPar
Parameter
&\sphinxstyletheadfamily 
\sphinxAtStartPar
Item
&\sphinxstyletheadfamily 
\sphinxAtStartPar
Remarks
\\
\sphinxmidrule
\sphinxtableatstartofbodyhook
\sphinxAtStartPar
ID=
&
\sphinxAtStartPar
Resource
&\\
\sphinxhline
\sphinxAtStartPar
DESC=
&
\sphinxAtStartPar
Description
&
\sphinxAtStartPar
Quotes allowed
\\
\sphinxhline
\sphinxAtStartPar
DEPT=
&
\sphinxAtStartPar
Department
&\\
\sphinxhline
\sphinxAtStartPar
TYPE=
&
\sphinxAtStartPar
Resource Type
&
\sphinxAtStartPar
Unused field
\\
\sphinxbottomrule
\end{tabulary}
\sphinxtableafterendhook\par
\sphinxattableend\end{savenotes}

\index{VIRCONF Control Cards@\spxentry{VIRCONF Control Cards}!RULE statement@\spxentry{RULE statement}}\index{RULE statement@\spxentry{RULE statement}!VIRCONF Control Cards@\spxentry{VIRCONF Control Cards}}\ignorespaces 

\subsection{RULE}
\label{\detokenize{Installation_Guide:rule}}\label{\detokenize{Installation_Guide:index-172}}
\sphinxAtStartPar
This operation adds or replaces a RULE entity in the VIRARBO file. The parameters correspond to the various items described under the heading “Parameters of the rule”.


\begin{savenotes}\sphinxattablestart
\sphinxthistablewithglobalstyle
\centering
\begin{tabulary}{\linewidth}[t]{TTT}
\sphinxtoprule
\sphinxstyletheadfamily 
\sphinxAtStartPar
Parameter
&\sphinxstyletheadfamily 
\sphinxAtStartPar
Item
&\sphinxstyletheadfamily 
\sphinxAtStartPar
Remarks
\\
\sphinxmidrule
\sphinxtableatstartofbodyhook
\sphinxAtStartPar
ID=
&
\sphinxAtStartPar
Rule name
&\\
\sphinxhline
\sphinxAtStartPar
RULESET=
&
\sphinxAtStartPar
Ruleset name
&\\
\sphinxhline
\sphinxAtStartPar
STATUS=
&
\sphinxAtStartPar
Status
&\\
\sphinxhline
\sphinxAtStartPar
DESC=
&
\sphinxAtStartPar
Description
&\\
\sphinxhline
\sphinxAtStartPar
ENTRY=
&
\sphinxAtStartPar
Entry point
&
\sphinxAtStartPar
Quotes allowed
\\
\sphinxhline
\sphinxAtStartPar
PARAM=
&
\sphinxAtStartPar
Parameter
&\\
\sphinxhline
\sphinxAtStartPar
TRACE=
&
\sphinxAtStartPar
Trace
&\\
\sphinxhline
\sphinxAtStartPar
IPADDR=
&
\sphinxAtStartPar
IP Subnet
&\\
\sphinxhline
\sphinxAtStartPar
NETMASK=
&
\sphinxAtStartPar
MASK
&\\
\sphinxhline
\sphinxAtStartPar
HTTPHOST=
&
\sphinxAtStartPar
Host
&
\sphinxAtStartPar
{[}1{]} ; Quotes allowed
\\
\sphinxhline
\sphinxAtStartPar
EMAIL=
&
\sphinxAtStartPar
eMail
&
\sphinxAtStartPar
{[}1{]} ; Quotes allowed
\\
\sphinxhline
\sphinxAtStartPar
CALLER=
&
\sphinxAtStartPar
Calling DTE
&
\sphinxAtStartPar
{[}1{]} ; Numeric IP address or string =HTFORWD
\\
\sphinxhline
\sphinxAtStartPar
CALLED=
&
\sphinxAtStartPar
Called
&
\sphinxAtStartPar
{[}1{]} ; Numeric
\\
\sphinxhline
\sphinxAtStartPar
CUD0=
&
\sphinxAtStartPar
CUD0 (Hexadecimal)
&
\sphinxAtStartPar
{[}1{]} ; Hexadecimal
\\
\sphinxhline
\sphinxAtStartPar
USERDATA=
&
\sphinxAtStartPar
User data
&
\sphinxAtStartPar
{[}1{]} ; Quotes allowed
\\
\sphinxhline
\sphinxAtStartPar
DAYS=
&
\sphinxAtStartPar
Days
&
\sphinxAtStartPar
{[}1{]} ; Seven ‘X’ char. or blank, defining the days
from Sunday to Saturday.
\\
\sphinxhline
\sphinxAtStartPar
TIME=
&
\sphinxAtStartPar
Start time
&
\sphinxAtStartPar
{[}1{]} ; Twelve nemeric chars. with the pattern:\sphinxhyphen{}
HHMMSSHHMMSS
\\
\sphinxbottomrule
\end{tabulary}
\sphinxtableafterendhook\par
\sphinxattableend\end{savenotes}

\begin{sphinxadmonition}{note}{Note:}
\sphinxAtStartPar
{[}1{]} The conditions are coded in the form keyword=(condition,value) where condition is one of the following:
\end{sphinxadmonition}

\sphinxAtStartPar
\sphinxstylestrong{IGNORE}   \sphinxhyphen{} Ignore

\sphinxAtStartPar
\sphinxstylestrong{EQUAL}    \sphinxhyphen{} Is

\sphinxAtStartPar
\sphinxstylestrong{NOTEQ}    \sphinxhyphen{} Is not

\sphinxAtStartPar
\sphinxstylestrong{BEGIN}    \sphinxhyphen{} Starts with

\sphinxAtStartPar
\sphinxstylestrong{NOTBEGIN} \sphinxhyphen{} Does not

\sphinxAtStartPar
\sphinxstylestrong{END}      \sphinxhyphen{} Ends with

\sphinxAtStartPar
\sphinxstylestrong{NOTEND}   \sphinxhyphen{} Does not

\begin{sphinxadmonition}{note}{Note:}
\sphinxAtStartPar
{[}2{]} For compatibility with earlier versions, VIRCONF also accepts the parameter LINE= as a synonym of RULESET=
\end{sphinxadmonition}

\index{VIRCONF Control Cards@\spxentry{VIRCONF Control Cards}!SERVER statement@\spxentry{SERVER statement}}\index{SERVER statement@\spxentry{SERVER statement}!VIRCONF Control Cards@\spxentry{VIRCONF Control Cards}}\ignorespaces 

\subsection{SERVER}
\label{\detokenize{Installation_Guide:server}}\label{\detokenize{Installation_Guide:index-173}}
\sphinxAtStartPar
This operation adds or replaces a SERVER entity in the VIRARBO file. The parameters correspond to the various items described under the heading “Parameters of external server”.


\begin{savenotes}\sphinxattablestart
\sphinxthistablewithglobalstyle
\centering
\begin{tabulary}{\linewidth}[t]{TTT}
\sphinxtoprule
\sphinxstyletheadfamily 
\sphinxAtStartPar
Parameter
&\sphinxstyletheadfamily 
\sphinxAtStartPar
Item
&\sphinxstyletheadfamily 
\sphinxAtStartPar
Remarks
\\
\sphinxmidrule
\sphinxtableatstartofbodyhook
\sphinxAtStartPar
ID=
&
\sphinxAtStartPar
Name
&\\
\sphinxhline
\sphinxAtStartPar
DESC=
&
\sphinxAtStartPar
Description
&
\sphinxAtStartPar
Quotes allowed
\\
\sphinxhline
\sphinxAtStartPar
DIALNO=
&
\sphinxAtStartPar
Number
&
\sphinxAtStartPar
Numeric or ‘=’
\\
\sphinxhline
\sphinxAtStartPar
USERDATA=
&
\sphinxAtStartPar
Data
&
\sphinxAtStartPar
Quotes allowed
\\
\sphinxhline
\sphinxAtStartPar
LINE=
&
\sphinxAtStartPar
Line number
&\\
\sphinxhline
\sphinxAtStartPar
LINE2=
&
\sphinxAtStartPar
Backup line
&\\
\sphinxhline
\sphinxAtStartPar
CALLER=
&
\sphinxAtStartPar
Caller
&
\sphinxAtStartPar
Numeric or ‘=’ or ‘*’
\\
\sphinxhline
\sphinxAtStartPar
EMUL=
&
\sphinxAtStartPar
Emulation
&\\
\sphinxhline
\sphinxAtStartPar
CHARSET=
&
\sphinxAtStartPar
Character Set
&\\
\sphinxhline
\sphinxAtStartPar
TIMEOUT=
&
\sphinxAtStartPar
Server time out
&
\sphinxAtStartPar
Numeric
\\
\sphinxhline
\sphinxAtStartPar
DELAY=
&
\sphinxAtStartPar
Inactivity delay
&
\sphinxAtStartPar
Numeric
\\
\sphinxhline
\sphinxAtStartPar
ACTION=
&
\sphinxAtStartPar
Cut off warning
&
\sphinxAtStartPar
Numeric
\\
\sphinxhline
\sphinxAtStartPar
LEVEL=
&
\sphinxAtStartPar
Price level
&\\
\sphinxhline
\sphinxAtStartPar
SECRET=
&
\sphinxAtStartPar
Secret
&\\
\sphinxhline
\sphinxAtStartPar
FACILITY=
&
\sphinxAtStartPar
Facilities
&
\sphinxAtStartPar
Hexadecimal or ‘=’
\\
\sphinxhline
\sphinxAtStartPar
CUD0=
&
\sphinxAtStartPar
CUD0 (Hex)
&
\sphinxAtStartPar
Hexadecimal
\\
\sphinxhline
\sphinxAtStartPar
TIOA=
&
\sphinxAtStartPar
TIOA at startup
&
\sphinxAtStartPar
Quotes allowed
\\
\sphinxbottomrule
\end{tabulary}
\sphinxtableafterendhook\par
\sphinxattableend\end{savenotes}

\index{VIRCONF Control Cards@\spxentry{VIRCONF Control Cards}!SSERV statement@\spxentry{SSERV statement}}\index{SSERV statement@\spxentry{SSERV statement}!VIRCONF Control Cards@\spxentry{VIRCONF Control Cards}}\ignorespaces 

\subsection{SSERV}
\label{\detokenize{Installation_Guide:sserv}}\label{\detokenize{Installation_Guide:index-174}}
\sphinxAtStartPar
This operation adds or replaces an SSERV entity in the VIRARBO file. The parameters correspond to the various items described under the heading “Defining a sub\sphinxhyphen{}server node”.


\begin{savenotes}\sphinxattablestart
\sphinxthistablewithglobalstyle
\centering
\begin{tabulary}{\linewidth}[t]{TTT}
\sphinxtoprule
\sphinxstyletheadfamily 
\sphinxAtStartPar
Parameter
&\sphinxstyletheadfamily 
\sphinxAtStartPar
Item
&\sphinxstyletheadfamily 
\sphinxAtStartPar
Remarks
\\
\sphinxmidrule
\sphinxtableatstartofbodyhook
\sphinxAtStartPar
ID=
&
\sphinxAtStartPar
SubServer node name
&
\sphinxAtStartPar
Quotes allowed
\\
\sphinxhline
\sphinxAtStartPar
SERVER=
&
\sphinxAtStartPar
SubServer name
&\\
\sphinxhline
\sphinxAtStartPar
TRANSACT=
&
\sphinxAtStartPar
Transaction name
&
\sphinxAtStartPar
{[}1{]}; Quotes allowed
\\
\sphinxhline
\sphinxAtStartPar
CONTROL=
&
\sphinxAtStartPar
Control program
&\\
\sphinxhline
\sphinxAtStartPar
TIOA=
&
\sphinxAtStartPar
TIOA
&
\sphinxAtStartPar
{[}1{]}; Quotes allowed
\\
\sphinxhline
\sphinxAtStartPar
TRANSLAT=
&
\sphinxAtStartPar
Transaltion type
&\\
\sphinxhline
\sphinxAtStartPar
RESTRICT=
&
\sphinxAtStartPar
Node with reserved
access
&
\sphinxAtStartPar
{[}2{]}; Quotes allowed
\\
\sphinxhline
\sphinxAtStartPar
PFRETOUR=
&
\sphinxAtStartPar
Return key
&
\sphinxAtStartPar
{[}3{]}; Quotes allowed
\\
\sphinxhline
\sphinxAtStartPar
PFGUIDE=
&
\sphinxAtStartPar
Guide key
&
\sphinxAtStartPar
{[}3{]}; Quotes allowed
\\
\sphinxhline
\sphinxAtStartPar
PFSOMMR=
&
\sphinxAtStartPar
Summary key
&
\sphinxAtStartPar
{[}3{]}; Quotes allowed
\\
\sphinxhline
\sphinxAtStartPar
PFSUITE=
&
\sphinxAtStartPar
Next key
&
\sphinxAtStartPar
{[}3{]}; Quotes allowed
\\
\sphinxhline
\sphinxAtStartPar
PFREPET=
&
\sphinxAtStartPar
Repeat key
&
\sphinxAtStartPar
{[}3{]}; Quotes allowed
\\
\sphinxhline
\sphinxAtStartPar
PFANNUL=
&
\sphinxAtStartPar
Cancel key
&
\sphinxAtStartPar
{[}3{]}; Quotes allowed
\\
\sphinxhline
\sphinxAtStartPar
PFCORREC=
&
\sphinxAtStartPar
Correct key
&
\sphinxAtStartPar
{[}3{]}; Quotes allowed
\\
\sphinxbottomrule
\end{tabulary}
\sphinxtableafterendhook\par
\sphinxattableend\end{savenotes}

\begin{sphinxadmonition}{note}{Note:}
\sphinxAtStartPar
{[}1{]}  Specify either TRANSACT or TIOA, but not both

\sphinxAtStartPar
{[}2{]}  *=Yes

\sphinxAtStartPar
{[}3{]}  The key name must be coded according to the table below:
\end{sphinxadmonition}

\sphinxAtStartPar
Key table:


\begin{savenotes}\sphinxattablestart
\sphinxthistablewithglobalstyle
\centering
\begin{tabulary}{\linewidth}[t]{TTTTTT}
\sphinxtoprule
\sphinxstyletheadfamily 
\sphinxAtStartPar
Key Name
&\sphinxstyletheadfamily 
\sphinxAtStartPar
Code
&\sphinxstyletheadfamily 
\sphinxAtStartPar
Key Name
&\sphinxstyletheadfamily 
\sphinxAtStartPar
Code
&\sphinxstyletheadfamily 
\sphinxAtStartPar
Key Name
&\sphinxstyletheadfamily 
\sphinxAtStartPar
Code
\\
\sphinxmidrule
\sphinxtableatstartofbodyhook
\sphinxAtStartPar
Enter
&
\sphinxAtStartPar
“’”
&
\sphinxAtStartPar
PF12
&
\sphinxAtStartPar
‘@’
&
\sphinxAtStartPar
PF24
&
\sphinxAtStartPar
‘\textless{}’
\\
\sphinxhline
\sphinxAtStartPar
PF1
&
\sphinxAtStartPar
1
&
\sphinxAtStartPar
PF13
&
\sphinxAtStartPar
A
&
\sphinxAtStartPar
PA1
&
\sphinxAtStartPar
‘\%’
\\
\sphinxhline
\sphinxAtStartPar
PF2
&
\sphinxAtStartPar
2
&
\sphinxAtStartPar
PF14
&
\sphinxAtStartPar
B
&
\sphinxAtStartPar
PA2
&
\sphinxAtStartPar
‘\textgreater{}’
\\
\sphinxhline
\sphinxAtStartPar
PF3
&
\sphinxAtStartPar
3
&
\sphinxAtStartPar
PF15
&
\sphinxAtStartPar
C
&
\sphinxAtStartPar
PA3
&
\sphinxAtStartPar
‘,’
\\
\sphinxhline
\sphinxAtStartPar
PF4
&
\sphinxAtStartPar
4
&
\sphinxAtStartPar
PF16
&
\sphinxAtStartPar
D
&
\sphinxAtStartPar
Clear
&
\sphinxAtStartPar
‘\_’
\\
\sphinxhline
\sphinxAtStartPar
PF5
&
\sphinxAtStartPar
5
&
\sphinxAtStartPar
PF17
&
\sphinxAtStartPar
E
&&\\
\sphinxhline
\sphinxAtStartPar
PF6
&
\sphinxAtStartPar
6
&
\sphinxAtStartPar
PF18
&
\sphinxAtStartPar
F
&&\\
\sphinxhline
\sphinxAtStartPar
PF7
&
\sphinxAtStartPar
7
&
\sphinxAtStartPar
PF19
&
\sphinxAtStartPar
G
&&\\
\sphinxhline
\sphinxAtStartPar
PF8
&
\sphinxAtStartPar
8
&
\sphinxAtStartPar
PF20
&
\sphinxAtStartPar
H
&&\\
\sphinxhline
\sphinxAtStartPar
PF9
&
\sphinxAtStartPar
9
&
\sphinxAtStartPar
PF21
&
\sphinxAtStartPar
I
&&\\
\sphinxhline
\sphinxAtStartPar
PF10
&
\sphinxAtStartPar
“:”
&
\sphinxAtStartPar
PF22
&
\sphinxAtStartPar
‘ç’
&&\\
\sphinxhline
\sphinxAtStartPar
PF11
&
\sphinxAtStartPar
“\#”
&
\sphinxAtStartPar
PF23
&
\sphinxAtStartPar
‘\sphinxhyphen{}’
&&\\
\sphinxbottomrule
\end{tabulary}
\sphinxtableafterendhook\par
\sphinxattableend\end{savenotes}

\index{VIRCONF Control Cards@\spxentry{VIRCONF Control Cards}!SUBDIR statement@\spxentry{SUBDIR statement}}\index{SUBDIR statement@\spxentry{SUBDIR statement}!VIRCONF Control Cards@\spxentry{VIRCONF Control Cards}}\ignorespaces 

\subsection{SUBDIR}
\label{\detokenize{Installation_Guide:subdir}}\label{\detokenize{Installation_Guide:index-175}}
\sphinxAtStartPar
This operation adds or replaces a SUBDIR entity in the VIRARBO file. The parameters correspond to the various items described under the heading “Parameter of directory”.


\begin{savenotes}\sphinxattablestart
\sphinxthistablewithglobalstyle
\centering
\begin{tabulary}{\linewidth}[t]{TTT}
\sphinxtoprule
\sphinxstyletheadfamily 
\sphinxAtStartPar
Parameter
&\sphinxstyletheadfamily 
\sphinxAtStartPar
Item
&\sphinxstyletheadfamily 
\sphinxAtStartPar
Remarks
\\
\sphinxmidrule
\sphinxtableatstartofbodyhook
\sphinxAtStartPar
ID=
&
\sphinxAtStartPar
Name
&\\
\sphinxhline
\sphinxAtStartPar
DESC=
&
\sphinxAtStartPar
Description
&
\sphinxAtStartPar
Quotes allowed
\\
\sphinxhline
\sphinxAtStartPar
FSTYPE=
&
\sphinxAtStartPar
Type
&\\
\sphinxhline
\sphinxAtStartPar
DDNAME=
&
\sphinxAtStartPar
DD name
&\\
\sphinxhline
\sphinxAtStartPar
KEY=
&
\sphinxAtStartPar
Keyword
&\\
\sphinxhline
\sphinxAtStartPar
NAMELEN=
&
\sphinxAtStartPar
Size of Names
&\\
\sphinxhline
\sphinxAtStartPar
CASELEN=
&
\sphinxAtStartPar
Case
&\\
\sphinxhline
\sphinxAtStartPar
AUTHUP=
&
\sphinxAtStartPar
Copy Up
&
\sphinxAtStartPar
X=Copy into directory is allowed
\\
\sphinxhline
\sphinxAtStartPar
AUTHDOWN=
&
\sphinxAtStartPar
Copy Down
&
\sphinxAtStartPar
X=Copy into directory is allowed
\\
\sphinxhline
\sphinxAtStartPar
AUTHDEL=
&
\sphinxAtStartPar
Delete
&
\sphinxAtStartPar
X=File deletion is allowed
\\
\sphinxbottomrule
\end{tabulary}
\sphinxtableafterendhook\par
\sphinxattableend\end{savenotes}

\index{VIRCONF Control Cards@\spxentry{VIRCONF Control Cards}!TERMINAL statement@\spxentry{TERMINAL statement}}\index{TERMINAL statement@\spxentry{TERMINAL statement}!VIRCONF Control Cards@\spxentry{VIRCONF Control Cards}}\ignorespaces 

\subsection{TERMINAL}
\label{\detokenize{Installation_Guide:terminal}}\label{\detokenize{Installation_Guide:index-176}}
\sphinxAtStartPar
This operation adds or replaces a TERMINAL entity in the VIRARBO file. The parameters correspond to the various items described under the heading “Parameter of the terminal”.


\begin{savenotes}\sphinxattablestart
\sphinxthistablewithglobalstyle
\centering
\begin{tabulary}{\linewidth}[t]{TTT}
\sphinxtoprule
\sphinxstyletheadfamily 
\sphinxAtStartPar
Parameter
&\sphinxstyletheadfamily 
\sphinxAtStartPar
Item
&\sphinxstyletheadfamily 
\sphinxAtStartPar
Remarks
\\
\sphinxmidrule
\sphinxtableatstartofbodyhook
\sphinxAtStartPar
ID=
&
\sphinxAtStartPar
Terminal
&\\
\sphinxhline
\sphinxAtStartPar
RELAY=
&
\sphinxAtStartPar
Relay
&
\sphinxAtStartPar
Quotes allowed
\\
\sphinxhline
\sphinxAtStartPar
POOL=
&
\sphinxAtStartPar
Pool Name
&\\
\sphinxhline
\sphinxAtStartPar
DESC=
&
\sphinxAtStartPar
Description
&
\sphinxAtStartPar
Quotes allowed
\\
\sphinxhline
\sphinxAtStartPar
ENTRY=
&
\sphinxAtStartPar
Entry Point
&\\
\sphinxhline
\sphinxAtStartPar
RELAY2=
&
\sphinxAtStartPar
2nd Relay
&
\sphinxAtStartPar
Quotes allowed
\\
\sphinxhline
\sphinxAtStartPar
TYPE=
&
\sphinxAtStartPar
Type
&\\
\sphinxhline
\sphinxAtStartPar
COMPRESS=
&
\sphinxAtStartPar
Compression
&\\
\sphinxhline
\sphinxAtStartPar
INOUT=
&
\sphinxAtStartPar
Possible Calls
&
\sphinxAtStartPar
Numeric
\\
\sphinxhline
\sphinxAtStartPar
STATS=
&
\sphinxAtStartPar
Write Stats to
&\\
\sphinxhline
\sphinxAtStartPar
REPEAT=
&
\sphinxAtStartPar
Repeat
&
\sphinxAtStartPar
Numeric
\\
\sphinxbottomrule
\end{tabulary}
\sphinxtableafterendhook\par
\sphinxattableend\end{savenotes}

\index{VIRCONF Control Cards@\spxentry{VIRCONF Control Cards}!TRANSACT statement@\spxentry{TRANSACT statement}}\index{TRANSACT statement@\spxentry{TRANSACT statement}!VIRCONF Control Cards@\spxentry{VIRCONF Control Cards}}\ignorespaces 

\subsection{TRANSACT}
\label{\detokenize{Installation_Guide:transact}}\label{\detokenize{Installation_Guide:index-177}}
\sphinxAtStartPar
This operation adds or replaces a TRANSACT entity in the VIRARBO file. The parameters correspond to the various items described under the heading “Parameter of the transaction”.


\begin{savenotes}\sphinxattablestart
\sphinxthistablewithglobalstyle
\centering
\begin{tabulary}{\linewidth}[t]{TTT}
\sphinxtoprule
\sphinxstyletheadfamily 
\sphinxAtStartPar
Parameter
&\sphinxstyletheadfamily 
\sphinxAtStartPar
Item
&\sphinxstyletheadfamily 
\sphinxAtStartPar
Remarks
\\
\sphinxmidrule
\sphinxtableatstartofbodyhook
\sphinxAtStartPar
ID=
&
\sphinxAtStartPar
Internal name
&\\
\sphinxhline
\sphinxAtStartPar
NAME=
&
\sphinxAtStartPar
External name
&
\sphinxAtStartPar
Quotes allowed
\\
\sphinxhline
\sphinxAtStartPar
DESC=
&
\sphinxAtStartPar
Description
&
\sphinxAtStartPar
Quotes allowed
\\
\sphinxhline
\sphinxAtStartPar
APPL=
&
\sphinxAtStartPar
Application
&
\sphinxAtStartPar
Quotes allowed
\\
\sphinxhline
\sphinxAtStartPar
ALIAS=
&
\sphinxAtStartPar
Alias
&\\
\sphinxhline
\sphinxAtStartPar
PASSTCKT=
&
\sphinxAtStartPar
Passticket
&
\sphinxAtStartPar
Numeric
\\
\sphinxhline
\sphinxAtStartPar
RACFNAME=
&
\sphinxAtStartPar
Passticket Name
&\\
\sphinxhline
\sphinxAtStartPar
TYPE=
&
\sphinxAtStartPar
Application Type
&
\sphinxAtStartPar
Numeric
\\
\sphinxhline
\sphinxAtStartPar
TERMINAL=
&
\sphinxAtStartPar
Psuedo\sphinxhyphen{}terminals
&\\
\sphinxhline
\sphinxAtStartPar
LOGMODE=
&
\sphinxAtStartPar
Logmode
&\\
\sphinxhline
\sphinxAtStartPar
STARTUP=
&
\sphinxAtStartPar
How started
prerequisite
&
\sphinxAtStartPar
Numeric
\\
\sphinxhline
\sphinxAtStartPar
SECURITY=
&
\sphinxAtStartPar
Security
&
\sphinxAtStartPar
Numeric
\\
\sphinxhline
\sphinxAtStartPar
TRANSL=
&
\sphinxAtStartPar
Translation or H4W
Commands
&\\
\sphinxhline
\sphinxAtStartPar
LOGMSG=
&
\sphinxAtStartPar
Logon message or
check URL prefix
&
\sphinxAtStartPar
Quotes allowed
\\
\sphinxhline
\sphinxAtStartPar
TIOASTA=
&
\sphinxAtStartPar
TIOA at logon
&
\sphinxAtStartPar
Quotes allowed
\\
\sphinxhline
\sphinxAtStartPar
TIOAEND=
&
\sphinxAtStartPar
TIOA at logoff
&
\sphinxAtStartPar
Quotes allowed
\\
\sphinxhline
\sphinxAtStartPar
EXITSA=
&
\sphinxAtStartPar
Initial Scenario
&\\
\sphinxhline
\sphinxAtStartPar
EXITEND=
&
\sphinxAtStartPar
Final Scenario
&\\
\sphinxhline
\sphinxAtStartPar
EXITMSGI=
&
\sphinxAtStartPar
Input scenario
&\\
\sphinxhline
\sphinxAtStartPar
EXITMSGO=
&
\sphinxAtStartPar
Output scenario
&\\
\sphinxbottomrule
\end{tabulary}
\sphinxtableafterendhook\par
\sphinxattableend\end{savenotes}

\index{VIRCONF Control Cards@\spxentry{VIRCONF Control Cards}!UPDATE statement@\spxentry{UPDATE statement}}\index{UPDATE statement@\spxentry{UPDATE statement}!VIRCONF Control Cards@\spxentry{VIRCONF Control Cards}}\ignorespaces 

\subsection{UPDATE}
\label{\detokenize{Installation_Guide:update}}\label{\detokenize{Installation_Guide:index-178}}
\sphinxAtStartPar
This operation updates one or more parameters of an entity in the VIRARBO file.


\begin{savenotes}\sphinxattablestart
\sphinxthistablewithglobalstyle
\centering
\begin{tabulary}{\linewidth}[t]{TTT}
\sphinxtoprule
\sphinxstyletheadfamily 
\sphinxAtStartPar
Parameter
&\sphinxstyletheadfamily 
\sphinxAtStartPar
Item
&\sphinxstyletheadfamily 
\sphinxAtStartPar
Remarks
\\
\sphinxmidrule
\sphinxtableatstartofbodyhook
\sphinxAtStartPar
TYPE=
&
\sphinxAtStartPar
Entity type
&
\sphinxAtStartPar
LINE, RULE, TERMINAL, etc.
\\
\sphinxhline
\sphinxAtStartPar
ID=
&
\sphinxAtStartPar
Entity name
&
\sphinxAtStartPar
Quotes allowed
\\
\sphinxhline
\sphinxAtStartPar
Param=
&
\sphinxAtStartPar
According to Entity type
&
\sphinxAtStartPar
See the precedding description of entity.
\\
\sphinxbottomrule
\end{tabulary}
\sphinxtableafterendhook\par
\sphinxattableend\end{savenotes}

\index{VIRCONF Control Cards@\spxentry{VIRCONF Control Cards}!USER statement@\spxentry{USER statement}}\index{USER statement@\spxentry{USER statement}!VIRCONF Control Cards@\spxentry{VIRCONF Control Cards}}\ignorespaces 

\subsection{USER}
\label{\detokenize{Installation_Guide:user}}\label{\detokenize{Installation_Guide:index-179}}
\sphinxAtStartPar
This operation adds or replaces a USER entity in the VIRARBO file. The parameters correspond to the various items described under the heading “Managing users”.


\begin{savenotes}\sphinxattablestart
\sphinxthistablewithglobalstyle
\centering
\begin{tabulary}{\linewidth}[t]{TTT}
\sphinxtoprule
\sphinxstyletheadfamily 
\sphinxAtStartPar
Parameter
&\sphinxstyletheadfamily 
\sphinxAtStartPar
Item
&\sphinxstyletheadfamily 
\sphinxAtStartPar
Remarks
\\
\sphinxmidrule
\sphinxtableatstartofbodyhook
\sphinxAtStartPar
ID=
&
\sphinxAtStartPar
User name
&\\
\sphinxhline
\sphinxAtStartPar
NAME=
&
\sphinxAtStartPar
Description
&
\sphinxAtStartPar
Quotes allowed
\\
\sphinxhline
\sphinxAtStartPar
DEPT=
&
\sphinxAtStartPar
Department
&\\
\sphinxhline
\sphinxAtStartPar
PASSWORD=
&
\sphinxAtStartPar
Password (Clear)
&
\sphinxAtStartPar
{[}1{]}
\\
\sphinxhline
\sphinxAtStartPar
PASSCODE=
&
\sphinxAtStartPar
Password (Encrypted)
&
\sphinxAtStartPar
{[}1{]}; Quotes allowed
\\
\sphinxhline
\sphinxAtStartPar
ADMIN=
&
\sphinxAtStartPar
Administrator
&
\sphinxAtStartPar
1=Yes, 0=No
\\
\sphinxhline
\sphinxAtStartPar
ASSIST=
&
\sphinxAtStartPar
Assist(O\sphinxhyphen{}N)
&
\sphinxAtStartPar
1=Yes, 0=No
\\
\sphinxhline
\sphinxAtStartPar
PROFILE=
&
\sphinxAtStartPar
Profiles assigned
&
\sphinxAtStartPar
List of profiles assigned
\\
\sphinxbottomrule
\end{tabulary}
\sphinxtableafterendhook\par
\sphinxattableend\end{savenotes}

\begin{sphinxadmonition}{note}{Note:}
\sphinxAtStartPar
{[}1{]} Indicate either PASSWORD (readable password), or PASSCODE (encrypted password). PASSWORD allows you to specify the password explicitly; if present, it must be coded after the ID parameter. PASSCODE is generated by the UNLOAD function; this allows to unload and reload the USER records without exposing the password in readable format.

\sphinxAtStartPar
{[}2{]} The PARM=’UNLOAD,PLAINTXT’ function unloads the password in readable format (PASSWORD instead of PASSCODE). To use this function, the user that submits this VIRCONF job must be RACF authorized (or by another securiyy tool) with READ access for the VIRTEL.PASSWORD.DECRYPT resource in the FACILITY class.
\end{sphinxadmonition}
\phantomsection\label{\detokenize{Installation_Guide:vvrrig-security}}
\index{Security@\spxentry{Security}}\ignorespaces 

\chapter{Security}
\label{\detokenize{Installation_Guide:security}}\label{\detokenize{Installation_Guide:index-180}}
\sphinxAtStartPar
Perform the following steps to activate RACF security for VIRTEL in the z/OS environment.

\index{Security@\spxentry{Security}!Setting up the TCT parameters@\spxentry{Setting up the TCT parameters}}\index{Setting up the TCT parameters@\spxentry{Setting up the TCT parameters}!Security@\spxentry{Security}}\ignorespaces 

\section{Modify the VIRTCT}
\label{\detokenize{Installation_Guide:modify-the-virtct}}\label{\detokenize{Installation_Guide:index-181}}
\sphinxAtStartPar
In the VIRTCTxx member of the VIRTEL CNTL library,
\begin{itemize}
\item {} 
\sphinxAtStartPar
VIRTCTUS = English language.

\item {} 
\sphinxAtStartPar
VIRTCTFR = French language.

\end{itemize}

\sphinxAtStartPar
replace the default parameters:

\begin{sphinxVerbatim}[commandchars=\\\{\}]
\PYG{n}{SECUR}\PYG{o}{=}\PYG{n}{NO}\PYG{p}{,}\PYG{n}{RAPPL}\PYG{o}{=}\PYG{n}{VIRTSERV}\PYG{p}{,}\PYG{n}{RNODE}\PYG{o}{=}\PYG{n}{VIRTNODE}
\end{sphinxVerbatim}

\sphinxAtStartPar
with the following parameters:

\begin{sphinxVerbatim}[commandchars=\\\{\}]
\PYG{n}{SECUR}\PYG{o}{=}\PYG{p}{(}\PYG{n}{RACROUTE}\PYG{p}{,}\PYG{n}{RACF}\PYG{p}{)}\PYG{p}{,}
\PYG{n}{RAPPL}\PYG{o}{=}\PYG{n}{FACILITY}\PYG{p}{,}\PYG{n}{RNODE}\PYG{o}{=}\PYG{n}{FACILITY}\PYG{p}{,}\PYG{n}{PRFSECU}\PYG{o}{=}\PYG{n}{VIRTEL}\PYG{p}{,}
\end{sphinxVerbatim}

\sphinxAtStartPar
This tells VIRTEL that its security definitions are stored in the FACILITY class, and the resource names are prefixed by “VIRTEL.”. You can choose your own prefix for each VIRTEL. Multiple VIRTEL started tasks can share the same resource name prefix if their security definitions are identical. You can also choose the class name, but it must already be defined in RACF with the correct attributes. It is recommended to use the FACILITY class which is standard in RACF.

\sphinxAtStartPar
Having updated the VIRTCTxx source member, reassemble and relink the VIRTCT into VIRTEL LOADLIB using the sample JCL in member ASMTCT of the VIRTEL CNTL library. Be sure to specify the correct member name MEMBER=VIRTCTxx in the job. Stop and start VIRTEL to pick up the new VIRTCT.

\newpage

\index{Security@\spxentry{Security}!RACF definitions@\spxentry{RACF definitions}}\index{RACF definitions@\spxentry{RACF definitions}!Security@\spxentry{Security}}\ignorespaces 

\section{Add RACF definitions}
\label{\detokenize{Installation_Guide:add-racf-definitions}}\label{\detokenize{Installation_Guide:index-182}}
\sphinxAtStartPar
The following RACF definitions are the minimum you need to get started. They simply authorize the VIRTEL administrator (you) to do everything. In this job, replace your userid by the administrator’s RACF userid or group name. This JCL can be found in member RACFDEF in the VIRTEL SAMPLIB.

\begin{sphinxVerbatim}[commandchars=\\\{\}]
\PYG{o}{/}\PYG{o}{/}\PYG{n}{VIRTRACF} \PYG{n}{JOB} \PYG{l+m+mi}{1}\PYG{p}{,}\PYG{n}{RACFDEF}\PYG{p}{,}\PYG{n}{MSGCLASS}\PYG{o}{=}\PYG{n}{X}\PYG{p}{,}\PYG{n}{CLASS}\PYG{o}{=}\PYG{n}{A}\PYG{p}{,}\PYG{n}{NOTIFY}\PYG{o}{=}\PYG{o}{\PYGZam{}}\PYG{n}{SYSUID}
\PYG{o}{/}\PYG{o}{/}\PYG{o}{*}\PYG{o}{\PYGZhy{}}\PYG{o}{\PYGZhy{}}\PYG{o}{\PYGZhy{}}\PYG{o}{\PYGZhy{}}\PYG{o}{\PYGZhy{}}\PYG{o}{\PYGZhy{}}\PYG{o}{\PYGZhy{}}\PYG{o}{\PYGZhy{}}\PYG{o}{\PYGZhy{}}\PYG{o}{\PYGZhy{}}\PYG{o}{\PYGZhy{}}\PYG{o}{\PYGZhy{}}\PYG{o}{\PYGZhy{}}\PYG{o}{\PYGZhy{}}\PYG{o}{\PYGZhy{}}\PYG{o}{\PYGZhy{}}\PYG{o}{\PYGZhy{}}\PYG{o}{\PYGZhy{}}\PYG{o}{\PYGZhy{}}\PYG{o}{\PYGZhy{}}\PYG{o}{\PYGZhy{}}\PYG{o}{\PYGZhy{}}\PYG{o}{\PYGZhy{}}\PYG{o}{\PYGZhy{}}\PYG{o}{\PYGZhy{}}\PYG{o}{\PYGZhy{}}\PYG{o}{\PYGZhy{}}\PYG{o}{\PYGZhy{}}\PYG{o}{\PYGZhy{}}\PYG{o}{\PYGZhy{}}\PYG{o}{\PYGZhy{}}\PYG{o}{\PYGZhy{}}\PYG{o}{\PYGZhy{}}\PYG{o}{\PYGZhy{}}\PYG{o}{\PYGZhy{}}\PYG{o}{\PYGZhy{}}\PYG{o}{\PYGZhy{}}\PYG{o}{\PYGZhy{}}\PYG{o}{\PYGZhy{}}\PYG{o}{\PYGZhy{}}\PYG{o}{\PYGZhy{}}\PYG{o}{\PYGZhy{}}\PYG{o}{\PYGZhy{}}\PYG{o}{\PYGZhy{}}\PYG{o}{\PYGZhy{}}\PYG{o}{\PYGZhy{}}\PYG{o}{\PYGZhy{}}\PYG{o}{\PYGZhy{}}\PYG{o}{\PYGZhy{}}\PYG{o}{\PYGZhy{}}\PYG{o}{\PYGZhy{}}\PYG{o}{\PYGZhy{}}\PYG{o}{\PYGZhy{}}\PYG{o}{\PYGZhy{}}\PYG{o}{\PYGZhy{}}\PYG{o}{\PYGZhy{}}\PYG{o}{\PYGZhy{}}\PYG{o}{*}
\PYG{o}{/}\PYG{o}{/}\PYG{o}{*} \PYG{n}{RACF} \PYG{p}{:} \PYG{n}{AUTHORIZATIONS} \PYG{n}{FOR} \PYG{n}{VIRTEL} \PYG{o}{*}
\PYG{o}{/}\PYG{o}{/}\PYG{o}{*} \PYG{n}{Replace} \PYG{l+s+s1}{\PYGZsq{}}\PYG{l+s+s1}{youruserid}\PYG{l+s+s1}{\PYGZsq{}} \PYG{n}{by} \PYG{n}{your} \PYG{n}{VIRTEL} \PYG{n}{administrator} \PYG{n+nb}{id} \PYG{o}{*}
    \PYG{o}{/}\PYG{o}{/}\PYG{o}{*}\PYG{o}{\PYGZhy{}}\PYG{o}{\PYGZhy{}}\PYG{o}{\PYGZhy{}}\PYG{o}{\PYGZhy{}}\PYG{o}{\PYGZhy{}}\PYG{o}{\PYGZhy{}}\PYG{o}{\PYGZhy{}}\PYG{o}{\PYGZhy{}}\PYG{o}{\PYGZhy{}}\PYG{o}{\PYGZhy{}}\PYG{o}{\PYGZhy{}}\PYG{o}{\PYGZhy{}}\PYG{o}{\PYGZhy{}}\PYG{o}{\PYGZhy{}}\PYG{o}{\PYGZhy{}}\PYG{o}{\PYGZhy{}}\PYG{o}{\PYGZhy{}}\PYG{o}{\PYGZhy{}}\PYG{o}{\PYGZhy{}}\PYG{o}{\PYGZhy{}}\PYG{o}{\PYGZhy{}}\PYG{o}{\PYGZhy{}}\PYG{o}{\PYGZhy{}}\PYG{o}{\PYGZhy{}}\PYG{o}{\PYGZhy{}}\PYG{o}{\PYGZhy{}}\PYG{o}{\PYGZhy{}}\PYG{o}{\PYGZhy{}}\PYG{o}{\PYGZhy{}}\PYG{o}{\PYGZhy{}}\PYG{o}{\PYGZhy{}}\PYG{o}{\PYGZhy{}}\PYG{o}{\PYGZhy{}}\PYG{o}{\PYGZhy{}}\PYG{o}{\PYGZhy{}}\PYG{o}{\PYGZhy{}}\PYG{o}{\PYGZhy{}}\PYG{o}{\PYGZhy{}}\PYG{o}{\PYGZhy{}}\PYG{o}{\PYGZhy{}}\PYG{o}{\PYGZhy{}}\PYG{o}{\PYGZhy{}}\PYG{o}{\PYGZhy{}}\PYG{o}{\PYGZhy{}}\PYG{o}{\PYGZhy{}}\PYG{o}{\PYGZhy{}}\PYG{o}{\PYGZhy{}}\PYG{o}{\PYGZhy{}}\PYG{o}{\PYGZhy{}}\PYG{o}{\PYGZhy{}}\PYG{o}{\PYGZhy{}}\PYG{o}{\PYGZhy{}}\PYG{o}{\PYGZhy{}}\PYG{o}{\PYGZhy{}}\PYG{o}{\PYGZhy{}}\PYG{o}{\PYGZhy{}}\PYG{o}{\PYGZhy{}}\PYG{o}{*}
    \PYG{o}{/}\PYG{o}{/}\PYG{n}{STEP1} \PYG{n}{EXEC} \PYG{n}{PGM}\PYG{o}{=}\PYG{n}{IKJEFT1A}\PYG{p}{,}\PYG{n}{DYNAMNBR}\PYG{o}{=}\PYG{l+m+mi}{20}
    \PYG{o}{/}\PYG{o}{/}\PYG{n}{SYSTSPRT} \PYG{n}{DD} \PYG{n}{SYSOUT}\PYG{o}{=}\PYG{o}{*}
    \PYG{o}{/}\PYG{o}{/}\PYG{n}{SYSTSIN} \PYG{n}{DD} \PYG{o}{*}
    \PYG{o}{/}\PYG{o}{*}\PYG{o}{\PYGZhy{}}\PYG{o}{\PYGZhy{}}\PYG{o}{\PYGZhy{}}\PYG{o}{\PYGZhy{}}\PYG{o}{\PYGZhy{}}\PYG{o}{\PYGZhy{}}\PYG{o}{\PYGZhy{}}\PYG{o}{\PYGZhy{}}\PYG{o}{\PYGZhy{}}\PYG{o}{\PYGZhy{}}\PYG{o}{\PYGZhy{}}\PYG{o}{\PYGZhy{}}\PYG{o}{\PYGZhy{}}\PYG{o}{\PYGZhy{}}\PYG{o}{\PYGZhy{}}\PYG{o}{\PYGZhy{}}\PYG{o}{\PYGZhy{}}\PYG{o}{\PYGZhy{}}\PYG{o}{\PYGZhy{}}\PYG{o}{\PYGZhy{}}\PYG{o}{\PYGZhy{}}\PYG{o}{\PYGZhy{}}\PYG{o}{\PYGZhy{}}\PYG{o}{\PYGZhy{}}\PYG{o}{\PYGZhy{}}\PYG{o}{\PYGZhy{}}\PYG{o}{\PYGZhy{}}\PYG{o}{\PYGZhy{}}\PYG{o}{\PYGZhy{}}\PYG{o}{\PYGZhy{}}\PYG{o}{\PYGZhy{}}\PYG{o}{\PYGZhy{}}\PYG{o}{\PYGZhy{}}\PYG{o}{\PYGZhy{}}\PYG{o}{\PYGZhy{}}\PYG{o}{\PYGZhy{}}\PYG{o}{\PYGZhy{}}\PYG{o}{\PYGZhy{}}\PYG{o}{\PYGZhy{}}\PYG{o}{\PYGZhy{}}\PYG{o}{\PYGZhy{}}\PYG{o}{\PYGZhy{}}\PYG{o}{\PYGZhy{}}\PYG{o}{\PYGZhy{}}\PYG{o}{\PYGZhy{}}\PYG{o}{\PYGZhy{}}\PYG{o}{\PYGZhy{}}\PYG{o}{\PYGZhy{}}\PYG{o}{\PYGZhy{}}\PYG{o}{\PYGZhy{}}\PYG{o}{\PYGZhy{}}\PYG{o}{\PYGZhy{}}\PYG{o}{\PYGZhy{}}\PYG{o}{\PYGZhy{}}\PYG{o}{\PYGZhy{}}\PYG{o}{*}\PYG{o}{/}
    \PYG{o}{/}\PYG{o}{*} \PYG{n}{BY} \PYG{n}{DEFAULT} \PYG{n}{DISALLOW} \PYG{n}{EVERYTHING} \PYG{n}{TO} \PYG{n}{GENERAL} \PYG{n}{USERS} \PYG{o}{*}\PYG{o}{/}
    \PYG{o}{/}\PYG{o}{*} \PYG{n}{BUT} \PYG{n}{ALLOW} \PYG{n}{EVERYTHING} \PYG{n}{TO} \PYG{n}{youruserid} \PYG{o}{*}\PYG{o}{/}
    \PYG{o}{/}\PYG{o}{*}\PYG{o}{\PYGZhy{}}\PYG{o}{\PYGZhy{}}\PYG{o}{\PYGZhy{}}\PYG{o}{\PYGZhy{}}\PYG{o}{\PYGZhy{}}\PYG{o}{\PYGZhy{}}\PYG{o}{\PYGZhy{}}\PYG{o}{\PYGZhy{}}\PYG{o}{\PYGZhy{}}\PYG{o}{\PYGZhy{}}\PYG{o}{\PYGZhy{}}\PYG{o}{\PYGZhy{}}\PYG{o}{\PYGZhy{}}\PYG{o}{\PYGZhy{}}\PYG{o}{\PYGZhy{}}\PYG{o}{\PYGZhy{}}\PYG{o}{\PYGZhy{}}\PYG{o}{\PYGZhy{}}\PYG{o}{\PYGZhy{}}\PYG{o}{\PYGZhy{}}\PYG{o}{\PYGZhy{}}\PYG{o}{\PYGZhy{}}\PYG{o}{\PYGZhy{}}\PYG{o}{\PYGZhy{}}\PYG{o}{\PYGZhy{}}\PYG{o}{\PYGZhy{}}\PYG{o}{\PYGZhy{}}\PYG{o}{\PYGZhy{}}\PYG{o}{\PYGZhy{}}\PYG{o}{\PYGZhy{}}\PYG{o}{\PYGZhy{}}\PYG{o}{\PYGZhy{}}\PYG{o}{\PYGZhy{}}\PYG{o}{\PYGZhy{}}\PYG{o}{\PYGZhy{}}\PYG{o}{\PYGZhy{}}\PYG{o}{\PYGZhy{}}\PYG{o}{\PYGZhy{}}\PYG{o}{\PYGZhy{}}\PYG{o}{\PYGZhy{}}\PYG{o}{\PYGZhy{}}\PYG{o}{\PYGZhy{}}\PYG{o}{\PYGZhy{}}\PYG{o}{\PYGZhy{}}\PYG{o}{\PYGZhy{}}\PYG{o}{\PYGZhy{}}\PYG{o}{\PYGZhy{}}\PYG{o}{\PYGZhy{}}\PYG{o}{\PYGZhy{}}\PYG{o}{\PYGZhy{}}\PYG{o}{\PYGZhy{}}\PYG{o}{\PYGZhy{}}\PYG{o}{\PYGZhy{}}\PYG{o}{\PYGZhy{}}\PYG{o}{\PYGZhy{}}\PYG{o}{*}\PYG{o}{/}
            \PYG{n}{RDEF} \PYG{n}{FACILITY} \PYG{n}{VIRTEL}\PYG{o}{.}\PYG{o}{*} \PYG{n}{UACC}\PYG{p}{(}\PYG{n}{NONE}\PYG{p}{)}
            \PYG{n}{PE} \PYG{n}{VIRTEL}\PYG{o}{.}\PYG{o}{*} \PYG{n}{CL}\PYG{p}{(}\PYG{n}{FACILITY}\PYG{p}{)} \PYG{n}{RESET} \PYG{n}{ACC}\PYG{p}{(}\PYG{n}{READ}\PYG{p}{)} \PYG{n}{ID}\PYG{p}{(}\PYG{n}{youruserid}\PYG{p}{)}
    \PYG{o}{/}\PYG{o}{*}\PYG{o}{\PYGZhy{}}\PYG{o}{\PYGZhy{}}\PYG{o}{\PYGZhy{}}\PYG{o}{\PYGZhy{}}\PYG{o}{\PYGZhy{}}\PYG{o}{\PYGZhy{}}\PYG{o}{\PYGZhy{}}\PYG{o}{\PYGZhy{}}\PYG{o}{\PYGZhy{}}\PYG{o}{\PYGZhy{}}\PYG{o}{\PYGZhy{}}\PYG{o}{\PYGZhy{}}\PYG{o}{\PYGZhy{}}\PYG{o}{\PYGZhy{}}\PYG{o}{\PYGZhy{}}\PYG{o}{\PYGZhy{}}\PYG{o}{\PYGZhy{}}\PYG{o}{\PYGZhy{}}\PYG{o}{\PYGZhy{}}\PYG{o}{\PYGZhy{}}\PYG{o}{\PYGZhy{}}\PYG{o}{\PYGZhy{}}\PYG{o}{\PYGZhy{}}\PYG{o}{\PYGZhy{}}\PYG{o}{\PYGZhy{}}\PYG{o}{\PYGZhy{}}\PYG{o}{\PYGZhy{}}\PYG{o}{\PYGZhy{}}\PYG{o}{\PYGZhy{}}\PYG{o}{\PYGZhy{}}\PYG{o}{\PYGZhy{}}\PYG{o}{\PYGZhy{}}\PYG{o}{\PYGZhy{}}\PYG{o}{\PYGZhy{}}\PYG{o}{\PYGZhy{}}\PYG{o}{\PYGZhy{}}\PYG{o}{\PYGZhy{}}\PYG{o}{\PYGZhy{}}\PYG{o}{\PYGZhy{}}\PYG{o}{\PYGZhy{}}\PYG{o}{\PYGZhy{}}\PYG{o}{\PYGZhy{}}\PYG{o}{\PYGZhy{}}\PYG{o}{\PYGZhy{}}\PYG{o}{\PYGZhy{}}\PYG{o}{\PYGZhy{}}\PYG{o}{\PYGZhy{}}\PYG{o}{\PYGZhy{}}\PYG{o}{\PYGZhy{}}\PYG{o}{\PYGZhy{}}\PYG{o}{\PYGZhy{}}\PYG{o}{\PYGZhy{}}\PYG{o}{\PYGZhy{}}\PYG{o}{\PYGZhy{}}\PYG{o}{\PYGZhy{}}\PYG{o}{*}\PYG{o}{/}
    \PYG{o}{/}\PYG{o}{*} \PYG{n}{ALLOW} \PYG{n}{EVERYONE} \PYG{n}{TO} \PYG{n}{USE} \PYG{n}{THE} \PYG{l+m+mi}{3270} \PYG{n}{LOGOFF} \PYG{n}{TRANSACTION} \PYG{o}{*}\PYG{o}{/}
    \PYG{o}{/}\PYG{o}{*}\PYG{o}{\PYGZhy{}}\PYG{o}{\PYGZhy{}}\PYG{o}{\PYGZhy{}}\PYG{o}{\PYGZhy{}}\PYG{o}{\PYGZhy{}}\PYG{o}{\PYGZhy{}}\PYG{o}{\PYGZhy{}}\PYG{o}{\PYGZhy{}}\PYG{o}{\PYGZhy{}}\PYG{o}{\PYGZhy{}}\PYG{o}{\PYGZhy{}}\PYG{o}{\PYGZhy{}}\PYG{o}{\PYGZhy{}}\PYG{o}{\PYGZhy{}}\PYG{o}{\PYGZhy{}}\PYG{o}{\PYGZhy{}}\PYG{o}{\PYGZhy{}}\PYG{o}{\PYGZhy{}}\PYG{o}{\PYGZhy{}}\PYG{o}{\PYGZhy{}}\PYG{o}{\PYGZhy{}}\PYG{o}{\PYGZhy{}}\PYG{o}{\PYGZhy{}}\PYG{o}{\PYGZhy{}}\PYG{o}{\PYGZhy{}}\PYG{o}{\PYGZhy{}}\PYG{o}{\PYGZhy{}}\PYG{o}{\PYGZhy{}}\PYG{o}{\PYGZhy{}}\PYG{o}{\PYGZhy{}}\PYG{o}{\PYGZhy{}}\PYG{o}{\PYGZhy{}}\PYG{o}{\PYGZhy{}}\PYG{o}{\PYGZhy{}}\PYG{o}{\PYGZhy{}}\PYG{o}{\PYGZhy{}}\PYG{o}{\PYGZhy{}}\PYG{o}{\PYGZhy{}}\PYG{o}{\PYGZhy{}}\PYG{o}{\PYGZhy{}}\PYG{o}{\PYGZhy{}}\PYG{o}{\PYGZhy{}}\PYG{o}{\PYGZhy{}}\PYG{o}{\PYGZhy{}}\PYG{o}{\PYGZhy{}}\PYG{o}{\PYGZhy{}}\PYG{o}{\PYGZhy{}}\PYG{o}{\PYGZhy{}}\PYG{o}{\PYGZhy{}}\PYG{o}{\PYGZhy{}}\PYG{o}{\PYGZhy{}}\PYG{o}{\PYGZhy{}}\PYG{o}{\PYGZhy{}}\PYG{o}{\PYGZhy{}}\PYG{o}{\PYGZhy{}}\PYG{o}{*}\PYG{o}{/}
            \PYG{n}{RDEF} \PYG{n}{FACILITY} \PYG{n}{VIRTEL}\PYG{o}{.}\PYG{n}{PC}\PYG{o}{\PYGZhy{}}\PYG{l+m+mi}{0020} \PYG{n}{UACC}\PYG{p}{(}\PYG{n}{READ}\PYG{p}{)} \PYG{o}{/}\PYG{o}{*} \PYG{n}{LOGOFF} \PYG{o}{*}\PYG{o}{/}
    \PYG{o}{/}\PYG{o}{*}\PYG{o}{\PYGZhy{}}\PYG{o}{\PYGZhy{}}\PYG{o}{\PYGZhy{}}\PYG{o}{\PYGZhy{}}\PYG{o}{\PYGZhy{}}\PYG{o}{\PYGZhy{}}\PYG{o}{\PYGZhy{}}\PYG{o}{\PYGZhy{}}\PYG{o}{\PYGZhy{}}\PYG{o}{\PYGZhy{}}\PYG{o}{\PYGZhy{}}\PYG{o}{\PYGZhy{}}\PYG{o}{\PYGZhy{}}\PYG{o}{\PYGZhy{}}\PYG{o}{\PYGZhy{}}\PYG{o}{\PYGZhy{}}\PYG{o}{\PYGZhy{}}\PYG{o}{\PYGZhy{}}\PYG{o}{\PYGZhy{}}\PYG{o}{\PYGZhy{}}\PYG{o}{\PYGZhy{}}\PYG{o}{\PYGZhy{}}\PYG{o}{\PYGZhy{}}\PYG{o}{\PYGZhy{}}\PYG{o}{\PYGZhy{}}\PYG{o}{\PYGZhy{}}\PYG{o}{\PYGZhy{}}\PYG{o}{\PYGZhy{}}\PYG{o}{\PYGZhy{}}\PYG{o}{\PYGZhy{}}\PYG{o}{\PYGZhy{}}\PYG{o}{\PYGZhy{}}\PYG{o}{\PYGZhy{}}\PYG{o}{\PYGZhy{}}\PYG{o}{\PYGZhy{}}\PYG{o}{\PYGZhy{}}\PYG{o}{\PYGZhy{}}\PYG{o}{\PYGZhy{}}\PYG{o}{\PYGZhy{}}\PYG{o}{\PYGZhy{}}\PYG{o}{\PYGZhy{}}\PYG{o}{\PYGZhy{}}\PYG{o}{\PYGZhy{}}\PYG{o}{\PYGZhy{}}\PYG{o}{\PYGZhy{}}\PYG{o}{\PYGZhy{}}\PYG{o}{\PYGZhy{}}\PYG{o}{\PYGZhy{}}\PYG{o}{\PYGZhy{}}\PYG{o}{\PYGZhy{}}\PYG{o}{\PYGZhy{}}\PYG{o}{\PYGZhy{}}\PYG{o}{\PYGZhy{}}\PYG{o}{\PYGZhy{}}\PYG{o}{\PYGZhy{}}\PYG{o}{*}\PYG{o}{/}
    \PYG{o}{/}\PYG{o}{*} \PYG{n}{REFRESH} \PYG{n}{THE} \PYG{n}{RACF} \PYG{n}{PROFILES} \PYG{o}{*}\PYG{o}{/}
    \PYG{o}{/}\PYG{o}{*}\PYG{o}{\PYGZhy{}}\PYG{o}{\PYGZhy{}}\PYG{o}{\PYGZhy{}}\PYG{o}{\PYGZhy{}}\PYG{o}{\PYGZhy{}}\PYG{o}{\PYGZhy{}}\PYG{o}{\PYGZhy{}}\PYG{o}{\PYGZhy{}}\PYG{o}{\PYGZhy{}}\PYG{o}{\PYGZhy{}}\PYG{o}{\PYGZhy{}}\PYG{o}{\PYGZhy{}}\PYG{o}{\PYGZhy{}}\PYG{o}{\PYGZhy{}}\PYG{o}{\PYGZhy{}}\PYG{o}{\PYGZhy{}}\PYG{o}{\PYGZhy{}}\PYG{o}{\PYGZhy{}}\PYG{o}{\PYGZhy{}}\PYG{o}{\PYGZhy{}}\PYG{o}{\PYGZhy{}}\PYG{o}{\PYGZhy{}}\PYG{o}{\PYGZhy{}}\PYG{o}{\PYGZhy{}}\PYG{o}{\PYGZhy{}}\PYG{o}{\PYGZhy{}}\PYG{o}{\PYGZhy{}}\PYG{o}{\PYGZhy{}}\PYG{o}{\PYGZhy{}}\PYG{o}{\PYGZhy{}}\PYG{o}{\PYGZhy{}}\PYG{o}{\PYGZhy{}}\PYG{o}{\PYGZhy{}}\PYG{o}{\PYGZhy{}}\PYG{o}{\PYGZhy{}}\PYG{o}{\PYGZhy{}}\PYG{o}{\PYGZhy{}}\PYG{o}{\PYGZhy{}}\PYG{o}{\PYGZhy{}}\PYG{o}{\PYGZhy{}}\PYG{o}{\PYGZhy{}}\PYG{o}{\PYGZhy{}}\PYG{o}{\PYGZhy{}}\PYG{o}{\PYGZhy{}}\PYG{o}{\PYGZhy{}}\PYG{o}{\PYGZhy{}}\PYG{o}{\PYGZhy{}}\PYG{o}{\PYGZhy{}}\PYG{o}{\PYGZhy{}}\PYG{o}{\PYGZhy{}}\PYG{o}{\PYGZhy{}}\PYG{o}{\PYGZhy{}}\PYG{o}{\PYGZhy{}}\PYG{o}{\PYGZhy{}}\PYG{o}{\PYGZhy{}}\PYG{o}{*}\PYG{o}{/}
            \PYG{n}{SETR} \PYG{n}{REFRESH} \PYG{n}{RACLIST}\PYG{p}{(}\PYG{n}{FACILITY}\PYG{p}{)}
    \PYG{o}{/}\PYG{o}{/}
\end{sphinxVerbatim}

\sphinxAtStartPar
\sphinxstyleemphasis{RACFDEF : JCL to add RACF definitions}

\sphinxAtStartPar
Later you can refine the definitions so that other VIRTEL users can use VIRTEL transactions (such as secured VIRTEL Web Access transactions). The following example allows DEMOGRP to use transaction CLI\sphinxhyphen{}10:

\begin{sphinxVerbatim}[commandchars=\\\{\}]
\PYG{o}{/}\PYG{o}{/}\PYG{n}{VIRTRACF} \PYG{n}{JOB} \PYG{l+m+mi}{1}\PYG{p}{,}\PYG{n}{RACFDEF}\PYG{p}{,}\PYG{n}{MSGCLASS}\PYG{o}{=}\PYG{n}{X}\PYG{p}{,}\PYG{n}{CLASS}\PYG{o}{=}\PYG{n}{A}\PYG{p}{,}\PYG{n}{NOTIFY}\PYG{o}{=}\PYG{o}{\PYGZam{}}\PYG{n}{SYSUID}
    \PYG{o}{/}\PYG{o}{/}\PYG{o}{*}\PYG{o}{\PYGZhy{}}\PYG{o}{\PYGZhy{}}\PYG{o}{\PYGZhy{}}\PYG{o}{\PYGZhy{}}\PYG{o}{\PYGZhy{}}\PYG{o}{\PYGZhy{}}\PYG{o}{\PYGZhy{}}\PYG{o}{\PYGZhy{}}\PYG{o}{\PYGZhy{}}\PYG{o}{\PYGZhy{}}\PYG{o}{\PYGZhy{}}\PYG{o}{\PYGZhy{}}\PYG{o}{\PYGZhy{}}\PYG{o}{\PYGZhy{}}\PYG{o}{\PYGZhy{}}\PYG{o}{\PYGZhy{}}\PYG{o}{\PYGZhy{}}\PYG{o}{\PYGZhy{}}\PYG{o}{\PYGZhy{}}\PYG{o}{\PYGZhy{}}\PYG{o}{\PYGZhy{}}\PYG{o}{\PYGZhy{}}\PYG{o}{\PYGZhy{}}\PYG{o}{\PYGZhy{}}\PYG{o}{\PYGZhy{}}\PYG{o}{\PYGZhy{}}\PYG{o}{\PYGZhy{}}\PYG{o}{\PYGZhy{}}\PYG{o}{\PYGZhy{}}\PYG{o}{\PYGZhy{}}\PYG{o}{\PYGZhy{}}\PYG{o}{\PYGZhy{}}\PYG{o}{\PYGZhy{}}\PYG{o}{\PYGZhy{}}\PYG{o}{\PYGZhy{}}\PYG{o}{\PYGZhy{}}\PYG{o}{\PYGZhy{}}\PYG{o}{\PYGZhy{}}\PYG{o}{\PYGZhy{}}\PYG{o}{\PYGZhy{}}\PYG{o}{\PYGZhy{}}\PYG{o}{\PYGZhy{}}\PYG{o}{\PYGZhy{}}\PYG{o}{\PYGZhy{}}\PYG{o}{\PYGZhy{}}\PYG{o}{\PYGZhy{}}\PYG{o}{\PYGZhy{}}\PYG{o}{\PYGZhy{}}\PYG{o}{\PYGZhy{}}\PYG{o}{\PYGZhy{}}\PYG{o}{\PYGZhy{}}\PYG{o}{\PYGZhy{}}\PYG{o}{\PYGZhy{}}\PYG{o}{\PYGZhy{}}\PYG{o}{\PYGZhy{}}\PYG{o}{\PYGZhy{}}\PYG{o}{\PYGZhy{}}\PYG{o}{*}
    \PYG{o}{/}\PYG{o}{/}\PYG{o}{*} \PYG{n}{RACF} \PYG{p}{:} \PYG{n}{AUTHORIZATIONS} \PYG{n}{FOR} \PYG{n}{VIRTEL} \PYG{o}{*}
    \PYG{o}{/}\PYG{o}{/}\PYG{o}{*}\PYG{o}{\PYGZhy{}}\PYG{o}{\PYGZhy{}}\PYG{o}{\PYGZhy{}}\PYG{o}{\PYGZhy{}}\PYG{o}{\PYGZhy{}}\PYG{o}{\PYGZhy{}}\PYG{o}{\PYGZhy{}}\PYG{o}{\PYGZhy{}}\PYG{o}{\PYGZhy{}}\PYG{o}{\PYGZhy{}}\PYG{o}{\PYGZhy{}}\PYG{o}{\PYGZhy{}}\PYG{o}{\PYGZhy{}}\PYG{o}{\PYGZhy{}}\PYG{o}{\PYGZhy{}}\PYG{o}{\PYGZhy{}}\PYG{o}{\PYGZhy{}}\PYG{o}{\PYGZhy{}}\PYG{o}{\PYGZhy{}}\PYG{o}{\PYGZhy{}}\PYG{o}{\PYGZhy{}}\PYG{o}{\PYGZhy{}}\PYG{o}{\PYGZhy{}}\PYG{o}{\PYGZhy{}}\PYG{o}{\PYGZhy{}}\PYG{o}{\PYGZhy{}}\PYG{o}{\PYGZhy{}}\PYG{o}{\PYGZhy{}}\PYG{o}{\PYGZhy{}}\PYG{o}{\PYGZhy{}}\PYG{o}{\PYGZhy{}}\PYG{o}{\PYGZhy{}}\PYG{o}{\PYGZhy{}}\PYG{o}{\PYGZhy{}}\PYG{o}{\PYGZhy{}}\PYG{o}{\PYGZhy{}}\PYG{o}{\PYGZhy{}}\PYG{o}{\PYGZhy{}}\PYG{o}{\PYGZhy{}}\PYG{o}{\PYGZhy{}}\PYG{o}{\PYGZhy{}}\PYG{o}{\PYGZhy{}}\PYG{o}{\PYGZhy{}}\PYG{o}{\PYGZhy{}}\PYG{o}{\PYGZhy{}}\PYG{o}{\PYGZhy{}}\PYG{o}{\PYGZhy{}}\PYG{o}{\PYGZhy{}}\PYG{o}{\PYGZhy{}}\PYG{o}{\PYGZhy{}}\PYG{o}{\PYGZhy{}}\PYG{o}{\PYGZhy{}}\PYG{o}{\PYGZhy{}}\PYG{o}{\PYGZhy{}}\PYG{o}{\PYGZhy{}}\PYG{o}{\PYGZhy{}}\PYG{o}{\PYGZhy{}}\PYG{o}{*}
    \PYG{o}{/}\PYG{o}{/}\PYG{n}{STEP1} \PYG{n}{EXEC} \PYG{n}{PGM}\PYG{o}{=}\PYG{n}{IKJEFT1A}\PYG{p}{,}\PYG{n}{DYNAMNBR}\PYG{o}{=}\PYG{l+m+mi}{20}
    \PYG{o}{/}\PYG{o}{/}\PYG{n}{SYSTSPRT} \PYG{n}{DD} \PYG{n}{SYSOUT}\PYG{o}{=}\PYG{o}{*}
    \PYG{o}{/}\PYG{o}{/}\PYG{n}{SYSTSIN} \PYG{n}{DD} \PYG{o}{*}
    \PYG{o}{/}\PYG{o}{*}\PYG{o}{\PYGZhy{}}\PYG{o}{\PYGZhy{}}\PYG{o}{\PYGZhy{}}\PYG{o}{\PYGZhy{}}\PYG{o}{\PYGZhy{}}\PYG{o}{\PYGZhy{}}\PYG{o}{\PYGZhy{}}\PYG{o}{\PYGZhy{}}\PYG{o}{\PYGZhy{}}\PYG{o}{\PYGZhy{}}\PYG{o}{\PYGZhy{}}\PYG{o}{\PYGZhy{}}\PYG{o}{\PYGZhy{}}\PYG{o}{\PYGZhy{}}\PYG{o}{\PYGZhy{}}\PYG{o}{\PYGZhy{}}\PYG{o}{\PYGZhy{}}\PYG{o}{\PYGZhy{}}\PYG{o}{\PYGZhy{}}\PYG{o}{\PYGZhy{}}\PYG{o}{\PYGZhy{}}\PYG{o}{\PYGZhy{}}\PYG{o}{\PYGZhy{}}\PYG{o}{\PYGZhy{}}\PYG{o}{\PYGZhy{}}\PYG{o}{\PYGZhy{}}\PYG{o}{\PYGZhy{}}\PYG{o}{\PYGZhy{}}\PYG{o}{\PYGZhy{}}\PYG{o}{\PYGZhy{}}\PYG{o}{\PYGZhy{}}\PYG{o}{\PYGZhy{}}\PYG{o}{\PYGZhy{}}\PYG{o}{\PYGZhy{}}\PYG{o}{\PYGZhy{}}\PYG{o}{\PYGZhy{}}\PYG{o}{\PYGZhy{}}\PYG{o}{\PYGZhy{}}\PYG{o}{\PYGZhy{}}\PYG{o}{\PYGZhy{}}\PYG{o}{\PYGZhy{}}\PYG{o}{\PYGZhy{}}\PYG{o}{\PYGZhy{}}\PYG{o}{\PYGZhy{}}\PYG{o}{\PYGZhy{}}\PYG{o}{\PYGZhy{}}\PYG{o}{\PYGZhy{}}\PYG{o}{\PYGZhy{}}\PYG{o}{\PYGZhy{}}\PYG{o}{\PYGZhy{}}\PYG{o}{\PYGZhy{}}\PYG{o}{\PYGZhy{}}\PYG{o}{\PYGZhy{}}\PYG{o}{\PYGZhy{}}\PYG{o}{\PYGZhy{}}\PYG{o}{*}\PYG{o}{/}
    \PYG{o}{/}\PYG{o}{*} \PYG{n}{ALLOW} \PYG{n}{DEMOGRP} \PYG{n}{TO} \PYG{n}{USE} \PYG{n}{THE} \PYG{n}{CLI}\PYG{o}{\PYGZhy{}}\PYG{l+m+mi}{10} \PYG{p}{(}\PYG{n}{CICS}\PYG{p}{)} \PYG{n}{TRANSACTION}    \PYG{o}{*}\PYG{o}{/}
    \PYG{o}{/}\PYG{o}{*} \PYG{n}{AND} \PYG{n}{THE} \PYG{n}{APPLIST} \PYG{n}{TRANSACTION} \PYG{n}{CLI}\PYG{o}{\PYGZhy{}}\PYG{l+m+mi}{90}                    \PYG{o}{*}\PYG{o}{/}
    \PYG{o}{/}\PYG{o}{*}\PYG{o}{\PYGZhy{}}\PYG{o}{\PYGZhy{}}\PYG{o}{\PYGZhy{}}\PYG{o}{\PYGZhy{}}\PYG{o}{\PYGZhy{}}\PYG{o}{\PYGZhy{}}\PYG{o}{\PYGZhy{}}\PYG{o}{\PYGZhy{}}\PYG{o}{\PYGZhy{}}\PYG{o}{\PYGZhy{}}\PYG{o}{\PYGZhy{}}\PYG{o}{\PYGZhy{}}\PYG{o}{\PYGZhy{}}\PYG{o}{\PYGZhy{}}\PYG{o}{\PYGZhy{}}\PYG{o}{\PYGZhy{}}\PYG{o}{\PYGZhy{}}\PYG{o}{\PYGZhy{}}\PYG{o}{\PYGZhy{}}\PYG{o}{\PYGZhy{}}\PYG{o}{\PYGZhy{}}\PYG{o}{\PYGZhy{}}\PYG{o}{\PYGZhy{}}\PYG{o}{\PYGZhy{}}\PYG{o}{\PYGZhy{}}\PYG{o}{\PYGZhy{}}\PYG{o}{\PYGZhy{}}\PYG{o}{\PYGZhy{}}\PYG{o}{\PYGZhy{}}\PYG{o}{\PYGZhy{}}\PYG{o}{\PYGZhy{}}\PYG{o}{\PYGZhy{}}\PYG{o}{\PYGZhy{}}\PYG{o}{\PYGZhy{}}\PYG{o}{\PYGZhy{}}\PYG{o}{\PYGZhy{}}\PYG{o}{\PYGZhy{}}\PYG{o}{\PYGZhy{}}\PYG{o}{\PYGZhy{}}\PYG{o}{\PYGZhy{}}\PYG{o}{\PYGZhy{}}\PYG{o}{\PYGZhy{}}\PYG{o}{\PYGZhy{}}\PYG{o}{\PYGZhy{}}\PYG{o}{\PYGZhy{}}\PYG{o}{\PYGZhy{}}\PYG{o}{\PYGZhy{}}\PYG{o}{\PYGZhy{}}\PYG{o}{\PYGZhy{}}\PYG{o}{\PYGZhy{}}\PYG{o}{\PYGZhy{}}\PYG{o}{\PYGZhy{}}\PYG{o}{\PYGZhy{}}\PYG{o}{\PYGZhy{}}\PYG{o}{\PYGZhy{}}\PYG{o}{*}\PYG{o}{/}
            \PYG{n}{RDEF} \PYG{n}{FACILITY} \PYG{n}{VIRTEL}\PYG{o}{.}\PYG{n}{CLI}\PYG{o}{\PYGZhy{}}\PYG{l+m+mi}{10} \PYG{n}{UACC}\PYG{p}{(}\PYG{n}{NONE}\PYG{p}{)} \PYG{o}{/}\PYG{o}{*} \PYG{n}{CICS} \PYG{o}{*}\PYG{o}{/}
            \PYG{n}{PE} \PYG{n}{VIRTEL}\PYG{o}{.}\PYG{n}{CLI}\PYG{o}{\PYGZhy{}}\PYG{l+m+mi}{10} \PYG{n}{CL}\PYG{p}{(}\PYG{n}{FACILITY}\PYG{p}{)} \PYG{n}{ACC}\PYG{p}{(}\PYG{n}{READ}\PYG{p}{)} \PYG{n}{ID}\PYG{p}{(}\PYG{n}{DEMOGRP}\PYG{p}{)}
            \PYG{n}{RDEF} \PYG{n}{FACILITY} \PYG{n}{VIRTEL}\PYG{o}{.}\PYG{n}{CLI}\PYG{o}{\PYGZhy{}}\PYG{l+m+mi}{90} \PYG{n}{UACC}\PYG{p}{(}\PYG{n}{NONE}\PYG{p}{)} \PYG{o}{/}\PYG{o}{*} \PYG{n}{APPLIST} \PYG{o}{*}\PYG{o}{/}
            \PYG{n}{PE} \PYG{n}{VIRTEL}\PYG{o}{.}\PYG{n}{CLI}\PYG{o}{\PYGZhy{}}\PYG{l+m+mi}{90} \PYG{n}{CL}\PYG{p}{(}\PYG{n}{FACILITY}\PYG{p}{)} \PYG{n}{ACC}\PYG{p}{(}\PYG{n}{READ}\PYG{p}{)} \PYG{n}{ID}\PYG{p}{(}\PYG{n}{DEMOGRP}\PYG{p}{)}
    \PYG{o}{/}\PYG{o}{*}\PYG{o}{\PYGZhy{}}\PYG{o}{\PYGZhy{}}\PYG{o}{\PYGZhy{}}\PYG{o}{\PYGZhy{}}\PYG{o}{\PYGZhy{}}\PYG{o}{\PYGZhy{}}\PYG{o}{\PYGZhy{}}\PYG{o}{\PYGZhy{}}\PYG{o}{\PYGZhy{}}\PYG{o}{\PYGZhy{}}\PYG{o}{\PYGZhy{}}\PYG{o}{\PYGZhy{}}\PYG{o}{\PYGZhy{}}\PYG{o}{\PYGZhy{}}\PYG{o}{\PYGZhy{}}\PYG{o}{\PYGZhy{}}\PYG{o}{\PYGZhy{}}\PYG{o}{\PYGZhy{}}\PYG{o}{\PYGZhy{}}\PYG{o}{\PYGZhy{}}\PYG{o}{\PYGZhy{}}\PYG{o}{\PYGZhy{}}\PYG{o}{\PYGZhy{}}\PYG{o}{\PYGZhy{}}\PYG{o}{\PYGZhy{}}\PYG{o}{\PYGZhy{}}\PYG{o}{\PYGZhy{}}\PYG{o}{\PYGZhy{}}\PYG{o}{\PYGZhy{}}\PYG{o}{\PYGZhy{}}\PYG{o}{\PYGZhy{}}\PYG{o}{\PYGZhy{}}\PYG{o}{\PYGZhy{}}\PYG{o}{\PYGZhy{}}\PYG{o}{\PYGZhy{}}\PYG{o}{\PYGZhy{}}\PYG{o}{\PYGZhy{}}\PYG{o}{\PYGZhy{}}\PYG{o}{\PYGZhy{}}\PYG{o}{\PYGZhy{}}\PYG{o}{\PYGZhy{}}\PYG{o}{\PYGZhy{}}\PYG{o}{\PYGZhy{}}\PYG{o}{\PYGZhy{}}\PYG{o}{\PYGZhy{}}\PYG{o}{\PYGZhy{}}\PYG{o}{\PYGZhy{}}\PYG{o}{\PYGZhy{}}\PYG{o}{\PYGZhy{}}\PYG{o}{\PYGZhy{}}\PYG{o}{\PYGZhy{}}\PYG{o}{\PYGZhy{}}\PYG{o}{\PYGZhy{}}\PYG{o}{\PYGZhy{}}\PYG{o}{\PYGZhy{}}\PYG{o}{*}\PYG{o}{/}
    \PYG{o}{/}\PYG{o}{*} \PYG{n}{REFRESH} \PYG{n}{THE} \PYG{n}{RACF} \PYG{n}{PROFILES} \PYG{o}{*}\PYG{o}{/}
    \PYG{o}{/}\PYG{o}{*}\PYG{o}{\PYGZhy{}}\PYG{o}{\PYGZhy{}}\PYG{o}{\PYGZhy{}}\PYG{o}{\PYGZhy{}}\PYG{o}{\PYGZhy{}}\PYG{o}{\PYGZhy{}}\PYG{o}{\PYGZhy{}}\PYG{o}{\PYGZhy{}}\PYG{o}{\PYGZhy{}}\PYG{o}{\PYGZhy{}}\PYG{o}{\PYGZhy{}}\PYG{o}{\PYGZhy{}}\PYG{o}{\PYGZhy{}}\PYG{o}{\PYGZhy{}}\PYG{o}{\PYGZhy{}}\PYG{o}{\PYGZhy{}}\PYG{o}{\PYGZhy{}}\PYG{o}{\PYGZhy{}}\PYG{o}{\PYGZhy{}}\PYG{o}{\PYGZhy{}}\PYG{o}{\PYGZhy{}}\PYG{o}{\PYGZhy{}}\PYG{o}{\PYGZhy{}}\PYG{o}{\PYGZhy{}}\PYG{o}{\PYGZhy{}}\PYG{o}{\PYGZhy{}}\PYG{o}{\PYGZhy{}}\PYG{o}{\PYGZhy{}}\PYG{o}{\PYGZhy{}}\PYG{o}{\PYGZhy{}}\PYG{o}{\PYGZhy{}}\PYG{o}{\PYGZhy{}}\PYG{o}{\PYGZhy{}}\PYG{o}{\PYGZhy{}}\PYG{o}{\PYGZhy{}}\PYG{o}{\PYGZhy{}}\PYG{o}{\PYGZhy{}}\PYG{o}{\PYGZhy{}}\PYG{o}{\PYGZhy{}}\PYG{o}{\PYGZhy{}}\PYG{o}{\PYGZhy{}}\PYG{o}{\PYGZhy{}}\PYG{o}{\PYGZhy{}}\PYG{o}{\PYGZhy{}}\PYG{o}{\PYGZhy{}}\PYG{o}{\PYGZhy{}}\PYG{o}{\PYGZhy{}}\PYG{o}{\PYGZhy{}}\PYG{o}{\PYGZhy{}}\PYG{o}{\PYGZhy{}}\PYG{o}{\PYGZhy{}}\PYG{o}{\PYGZhy{}}\PYG{o}{\PYGZhy{}}\PYG{o}{\PYGZhy{}}\PYG{o}{\PYGZhy{}}\PYG{o}{*}\PYG{o}{/}
            \PYG{n}{SETR} \PYG{n}{REFRESH} \PYG{n}{RACLIST}\PYG{p}{(}\PYG{n}{FACILITY}\PYG{p}{)}
    \PYG{o}{/}\PYG{o}{/}
\end{sphinxVerbatim}

\sphinxAtStartPar
\sphinxstyleemphasis{RACFDEF : JCL to update RACF definitions}

\newpage

\index{Security@\spxentry{Security}!Virtel Administrators@\spxentry{Virtel Administrators}}\index{Virtel Administrators@\spxentry{Virtel Administrators}!Security@\spxentry{Security}}\ignorespaces 

\section{Virtel Administrators}
\label{\detokenize{Installation_Guide:virtel-administrators}}\label{\detokenize{Installation_Guide:index-183}}
\sphinxAtStartPar
Virtel Administrators have access to all the features of Virtel and are rsponsible for the administration of the product. For example this includes defining transactions and maintaining macros in the DDI central repository. Virtel uses the security subsystem to protect transactions. The following job shows an example of setting up the security profiles for Administrators in group SPGPTECH. This will enable them to control DDI and macro administration:\sphinxhyphen{}

\begin{sphinxVerbatim}[commandchars=\\\{\}]
\PYG{o}{/}\PYG{o}{/}\PYG{o}{*}\PYG{o}{\PYGZhy{}}\PYG{o}{\PYGZhy{}}\PYG{o}{\PYGZhy{}}\PYG{o}{\PYGZhy{}}\PYG{o}{\PYGZhy{}}\PYG{o}{\PYGZhy{}}\PYG{o}{\PYGZhy{}}\PYG{o}{\PYGZhy{}}\PYG{o}{\PYGZhy{}}\PYG{o}{\PYGZhy{}}\PYG{o}{\PYGZhy{}}\PYG{o}{\PYGZhy{}}\PYG{o}{\PYGZhy{}}\PYG{o}{\PYGZhy{}}\PYG{o}{\PYGZhy{}}\PYG{o}{\PYGZhy{}}\PYG{o}{\PYGZhy{}}\PYG{o}{\PYGZhy{}}\PYG{o}{\PYGZhy{}}\PYG{o}{\PYGZhy{}}\PYG{o}{\PYGZhy{}}\PYG{o}{\PYGZhy{}}\PYG{o}{\PYGZhy{}}\PYG{o}{\PYGZhy{}}\PYG{o}{\PYGZhy{}}\PYG{o}{\PYGZhy{}}\PYG{o}{\PYGZhy{}}\PYG{o}{\PYGZhy{}}\PYG{o}{\PYGZhy{}}\PYG{o}{\PYGZhy{}}\PYG{o}{\PYGZhy{}}\PYG{o}{\PYGZhy{}}\PYG{o}{\PYGZhy{}}\PYG{o}{\PYGZhy{}}\PYG{o}{\PYGZhy{}}\PYG{o}{\PYGZhy{}}\PYG{o}{\PYGZhy{}}\PYG{o}{\PYGZhy{}}\PYG{o}{\PYGZhy{}}\PYG{o}{\PYGZhy{}}\PYG{o}{\PYGZhy{}}\PYG{o}{\PYGZhy{}}\PYG{o}{\PYGZhy{}}\PYG{o}{\PYGZhy{}}\PYG{o}{\PYGZhy{}}\PYG{o}{\PYGZhy{}}\PYG{o}{\PYGZhy{}}\PYG{o}{\PYGZhy{}}\PYG{o}{\PYGZhy{}}\PYG{o}{\PYGZhy{}}\PYG{o}{\PYGZhy{}}\PYG{o}{\PYGZhy{}}\PYG{o}{\PYGZhy{}}\PYG{o}{\PYGZhy{}}\PYG{o}{\PYGZhy{}}\PYG{o}{\PYGZhy{}}\PYG{o}{\PYGZhy{}}\PYG{o}{*}
\PYG{o}{/}\PYG{o}{/}\PYG{o}{*} \PYG{n}{RACF} \PYG{p}{:} \PYG{n}{AUTHORIZATIONS} \PYG{n}{FOR} \PYG{n}{VIRTEL} \PYG{n}{DDI} \PYG{o}{*}
\PYG{o}{/}\PYG{o}{/}\PYG{o}{*}\PYG{o}{\PYGZhy{}}\PYG{o}{\PYGZhy{}}\PYG{o}{\PYGZhy{}}\PYG{o}{\PYGZhy{}}\PYG{o}{\PYGZhy{}}\PYG{o}{\PYGZhy{}}\PYG{o}{\PYGZhy{}}\PYG{o}{\PYGZhy{}}\PYG{o}{\PYGZhy{}}\PYG{o}{\PYGZhy{}}\PYG{o}{\PYGZhy{}}\PYG{o}{\PYGZhy{}}\PYG{o}{\PYGZhy{}}\PYG{o}{\PYGZhy{}}\PYG{o}{\PYGZhy{}}\PYG{o}{\PYGZhy{}}\PYG{o}{\PYGZhy{}}\PYG{o}{\PYGZhy{}}\PYG{o}{\PYGZhy{}}\PYG{o}{\PYGZhy{}}\PYG{o}{\PYGZhy{}}\PYG{o}{\PYGZhy{}}\PYG{o}{\PYGZhy{}}\PYG{o}{\PYGZhy{}}\PYG{o}{\PYGZhy{}}\PYG{o}{\PYGZhy{}}\PYG{o}{\PYGZhy{}}\PYG{o}{\PYGZhy{}}\PYG{o}{\PYGZhy{}}\PYG{o}{\PYGZhy{}}\PYG{o}{\PYGZhy{}}\PYG{o}{\PYGZhy{}}\PYG{o}{\PYGZhy{}}\PYG{o}{\PYGZhy{}}\PYG{o}{\PYGZhy{}}\PYG{o}{\PYGZhy{}}\PYG{o}{\PYGZhy{}}\PYG{o}{\PYGZhy{}}\PYG{o}{\PYGZhy{}}\PYG{o}{\PYGZhy{}}\PYG{o}{\PYGZhy{}}\PYG{o}{\PYGZhy{}}\PYG{o}{\PYGZhy{}}\PYG{o}{\PYGZhy{}}\PYG{o}{\PYGZhy{}}\PYG{o}{\PYGZhy{}}\PYG{o}{\PYGZhy{}}\PYG{o}{\PYGZhy{}}\PYG{o}{\PYGZhy{}}\PYG{o}{\PYGZhy{}}\PYG{o}{\PYGZhy{}}\PYG{o}{\PYGZhy{}}\PYG{o}{\PYGZhy{}}\PYG{o}{\PYGZhy{}}\PYG{o}{\PYGZhy{}}\PYG{o}{\PYGZhy{}}\PYG{o}{\PYGZhy{}}\PYG{o}{*}
\PYG{o}{/}\PYG{o}{/}\PYG{n}{STEP1} \PYG{n}{EXEC} \PYG{n}{PGM}\PYG{o}{=}\PYG{n}{IKJEFT01}\PYG{p}{,}\PYG{n}{DYNAMNBR}\PYG{o}{=}\PYG{l+m+mi}{20}
\PYG{o}{/}\PYG{o}{/}\PYG{n}{SYSTSPRT} \PYG{n}{DD} \PYG{n}{SYSOUT}\PYG{o}{=}\PYG{o}{*}
\PYG{o}{/}\PYG{o}{/}\PYG{n}{SYSTSIN} \PYG{n}{DD} \PYG{o}{*}
\PYG{o}{/}\PYG{o}{*}\PYG{o}{\PYGZhy{}}\PYG{o}{\PYGZhy{}}\PYG{o}{\PYGZhy{}}\PYG{o}{\PYGZhy{}}\PYG{o}{\PYGZhy{}}\PYG{o}{\PYGZhy{}}\PYG{o}{\PYGZhy{}}\PYG{o}{\PYGZhy{}}\PYG{o}{\PYGZhy{}}\PYG{o}{\PYGZhy{}}\PYG{o}{\PYGZhy{}}\PYG{o}{\PYGZhy{}}\PYG{o}{\PYGZhy{}}\PYG{o}{\PYGZhy{}}\PYG{o}{\PYGZhy{}}\PYG{o}{\PYGZhy{}}\PYG{o}{\PYGZhy{}}\PYG{o}{\PYGZhy{}}\PYG{o}{\PYGZhy{}}\PYG{o}{\PYGZhy{}}\PYG{o}{\PYGZhy{}}\PYG{o}{\PYGZhy{}}\PYG{o}{\PYGZhy{}}\PYG{o}{\PYGZhy{}}\PYG{o}{\PYGZhy{}}\PYG{o}{\PYGZhy{}}\PYG{o}{\PYGZhy{}}\PYG{o}{\PYGZhy{}}\PYG{o}{\PYGZhy{}}\PYG{o}{\PYGZhy{}}\PYG{o}{\PYGZhy{}}\PYG{o}{\PYGZhy{}}\PYG{o}{\PYGZhy{}}\PYG{o}{\PYGZhy{}}\PYG{o}{\PYGZhy{}}\PYG{o}{\PYGZhy{}}\PYG{o}{\PYGZhy{}}\PYG{o}{\PYGZhy{}}\PYG{o}{\PYGZhy{}}\PYG{o}{\PYGZhy{}}\PYG{o}{\PYGZhy{}}\PYG{o}{\PYGZhy{}}\PYG{o}{\PYGZhy{}}\PYG{o}{\PYGZhy{}}\PYG{o}{\PYGZhy{}}\PYG{o}{\PYGZhy{}}\PYG{o}{\PYGZhy{}}\PYG{o}{\PYGZhy{}}\PYG{o}{\PYGZhy{}}\PYG{o}{\PYGZhy{}}\PYG{o}{\PYGZhy{}}\PYG{o}{\PYGZhy{}}\PYG{o}{\PYGZhy{}}\PYG{o}{\PYGZhy{}}\PYG{o}{\PYGZhy{}}\PYG{o}{*}\PYG{o}{/}
\PYG{o}{/}\PYG{o}{*} \PYG{n}{Setup} \PYG{k}{for} \PYG{n}{DDI} \PYG{o}{*}\PYG{o}{/}
\PYG{o}{/}\PYG{o}{*}\PYG{o}{\PYGZhy{}}\PYG{o}{\PYGZhy{}}\PYG{o}{\PYGZhy{}}\PYG{o}{\PYGZhy{}}\PYG{o}{\PYGZhy{}}\PYG{o}{\PYGZhy{}}\PYG{o}{\PYGZhy{}}\PYG{o}{\PYGZhy{}}\PYG{o}{\PYGZhy{}}\PYG{o}{\PYGZhy{}}\PYG{o}{\PYGZhy{}}\PYG{o}{\PYGZhy{}}\PYG{o}{\PYGZhy{}}\PYG{o}{\PYGZhy{}}\PYG{o}{\PYGZhy{}}\PYG{o}{\PYGZhy{}}\PYG{o}{\PYGZhy{}}\PYG{o}{\PYGZhy{}}\PYG{o}{\PYGZhy{}}\PYG{o}{\PYGZhy{}}\PYG{o}{\PYGZhy{}}\PYG{o}{\PYGZhy{}}\PYG{o}{\PYGZhy{}}\PYG{o}{\PYGZhy{}}\PYG{o}{\PYGZhy{}}\PYG{o}{\PYGZhy{}}\PYG{o}{\PYGZhy{}}\PYG{o}{\PYGZhy{}}\PYG{o}{\PYGZhy{}}\PYG{o}{\PYGZhy{}}\PYG{o}{\PYGZhy{}}\PYG{o}{\PYGZhy{}}\PYG{o}{\PYGZhy{}}\PYG{o}{\PYGZhy{}}\PYG{o}{\PYGZhy{}}\PYG{o}{\PYGZhy{}}\PYG{o}{\PYGZhy{}}\PYG{o}{\PYGZhy{}}\PYG{o}{\PYGZhy{}}\PYG{o}{\PYGZhy{}}\PYG{o}{\PYGZhy{}}\PYG{o}{\PYGZhy{}}\PYG{o}{\PYGZhy{}}\PYG{o}{\PYGZhy{}}\PYG{o}{\PYGZhy{}}\PYG{o}{\PYGZhy{}}\PYG{o}{\PYGZhy{}}\PYG{o}{\PYGZhy{}}\PYG{o}{\PYGZhy{}}\PYG{o}{\PYGZhy{}}\PYG{o}{\PYGZhy{}}\PYG{o}{\PYGZhy{}}\PYG{o}{\PYGZhy{}}\PYG{o}{\PYGZhy{}}\PYG{o}{\PYGZhy{}}\PYG{o}{*}\PYG{o}{/}
  \PYG{n}{RDEF} \PYG{n}{FACILITY} \PYG{n}{VIRTEL}\PYG{o}{.}\PYG{n}{W2H}\PYG{o}{\PYGZhy{}}\PYG{l+m+mi}{03}\PYG{n}{G} \PYG{n}{UACC}\PYG{p}{(}\PYG{n}{NONE}\PYG{p}{)} \PYG{o}{/}\PYG{o}{*} \PYG{n}{W2H} \PYG{o}{*}\PYG{o}{/}
  \PYG{n}{RDEF} \PYG{n}{FACILITY} \PYG{n}{VIRTEL}\PYG{o}{.}\PYG{n}{W2H}\PYG{o}{\PYGZhy{}}\PYG{l+m+mi}{03}\PYG{n}{U} \PYG{n}{UACC}\PYG{p}{(}\PYG{n}{NONE}\PYG{p}{)} \PYG{o}{/}\PYG{o}{*} \PYG{n}{W2H} \PYG{o}{*}\PYG{o}{/}
  \PYG{n}{RDEF} \PYG{n}{FACILITY} \PYG{n}{VIRTEL}\PYG{o}{.}\PYG{n}{W2H}\PYG{o}{\PYGZhy{}}\PYG{l+m+mi}{03}\PYG{n}{A} \PYG{n}{UACC}\PYG{p}{(}\PYG{n}{NONE}\PYG{p}{)} \PYG{o}{/}\PYG{o}{*} \PYG{n}{W2H} \PYG{o}{*}\PYG{o}{/}
  \PYG{n}{RDEF} \PYG{n}{FACILITY} \PYG{n}{VIRTEL}\PYG{o}{.}\PYG{n}{CLI}\PYG{o}{\PYGZhy{}}\PYG{l+m+mi}{03}\PYG{n}{G} \PYG{n}{UACC}\PYG{p}{(}\PYG{n}{NONE}\PYG{p}{)} \PYG{o}{/}\PYG{o}{*} \PYG{n}{CLI} \PYG{o}{*}\PYG{o}{/}
  \PYG{n}{RDEF} \PYG{n}{FACILITY} \PYG{n}{VIRTEL}\PYG{o}{.}\PYG{n}{CLI}\PYG{o}{\PYGZhy{}}\PYG{l+m+mi}{03}\PYG{n}{U} \PYG{n}{UACC}\PYG{p}{(}\PYG{n}{NONE}\PYG{p}{)} \PYG{o}{/}\PYG{o}{*} \PYG{n}{CLI} \PYG{o}{*}\PYG{o}{/}
  \PYG{n}{RDEF} \PYG{n}{FACILITY} \PYG{n}{VIRTEL}\PYG{o}{.}\PYG{n}{CLI}\PYG{o}{\PYGZhy{}}\PYG{l+m+mi}{03}\PYG{n}{A} \PYG{n}{UACC}\PYG{p}{(}\PYG{n}{NONE}\PYG{p}{)} \PYG{o}{/}\PYG{o}{*} \PYG{n}{CLI} \PYG{o}{*}\PYG{o}{/}
  \PYG{n}{RDEF} \PYG{n}{FACILITY} \PYG{n}{VIRTEL}\PYG{o}{.}\PYG{n}{W2H}\PYG{o}{\PYGZhy{}}\PYG{l+m+mi}{07} \PYG{n}{UACC}\PYG{p}{(}\PYG{n}{NONE}\PYG{p}{)} \PYG{o}{/}\PYG{o}{*} \PYG{n}{W2H} \PYG{o}{*}\PYG{o}{/}
  \PYG{n}{RDEF} \PYG{n}{FACILITY} \PYG{n}{VIRTEL}\PYG{o}{.}\PYG{n}{W2H}\PYG{o}{\PYGZhy{}}\PYG{l+m+mi}{66} \PYG{n}{UACC}\PYG{p}{(}\PYG{n}{NONE}\PYG{p}{)} \PYG{o}{/}\PYG{o}{*} \PYG{n}{W2H} \PYG{o}{*}\PYG{o}{/}
  \PYG{n}{RDEF} \PYG{n}{FACILITY} \PYG{n}{VIRTEL}\PYG{o}{.}\PYG{n}{W2H}\PYG{o}{\PYGZhy{}}\PYG{l+m+mi}{80}\PYG{n}{U} \PYG{n}{UACC}\PYG{p}{(}\PYG{n}{NONE}\PYG{p}{)} \PYG{o}{/}\PYG{o}{*} \PYG{n}{W2H} \PYG{o}{*}\PYG{o}{/}
  \PYG{n}{RDEF} \PYG{n}{FACILITY} \PYG{n}{VIRTEL}\PYG{o}{.}\PYG{n}{W2H}\PYG{o}{\PYGZhy{}}\PYG{l+m+mi}{80}\PYG{n}{G} \PYG{n}{UACC}\PYG{p}{(}\PYG{n}{NONE}\PYG{p}{)} \PYG{o}{/}\PYG{o}{*} \PYG{n}{W2H} \PYG{o}{*}\PYG{o}{/}
  \PYG{n}{RDEF} \PYG{n}{FACILITY} \PYG{n}{VIRTEL}\PYG{o}{.}\PYG{n}{W2H}\PYG{o}{\PYGZhy{}}\PYG{l+m+mi}{80}\PYG{n}{A} \PYG{n}{UACC}\PYG{p}{(}\PYG{n}{NONE}\PYG{p}{)} \PYG{o}{/}\PYG{o}{*} \PYG{n}{W2H} \PYG{o}{*}\PYG{o}{/}
  \PYG{n}{RDEF} \PYG{n}{FACILITY} \PYG{n}{VIRTEL}\PYG{o}{.}\PYG{n}{USR}\PYG{o}{\PYGZhy{}}\PYG{n}{DIR} \PYG{n}{UACC}\PYG{p}{(}\PYG{n}{NONE}\PYG{p}{)} \PYG{o}{/}\PYG{o}{*} \PYG{n}{W2H} \PYG{o}{*}\PYG{o}{/}
  \PYG{n}{RDEF} \PYG{n}{FACILITY} \PYG{n}{VIRTEL}\PYG{o}{.}\PYG{n}{GRP}\PYG{o}{\PYGZhy{}}\PYG{n}{DIR} \PYG{n}{UACC}\PYG{p}{(}\PYG{n}{NONE}\PYG{p}{)} \PYG{o}{/}\PYG{o}{*} \PYG{n}{W2H} \PYG{o}{*}\PYG{o}{/}
  \PYG{n}{RDEF} \PYG{n}{FACILITY} \PYG{n}{VIRTEL}\PYG{o}{.}\PYG{n}{GLB}\PYG{o}{\PYGZhy{}}\PYG{n}{DIR} \PYG{n}{UACC}\PYG{p}{(}\PYG{n}{NONE}\PYG{p}{)} \PYG{o}{/}\PYG{o}{*} \PYG{n}{W2H} \PYG{o}{*}\PYG{o}{/}
  \PYG{n}{PE} \PYG{n}{VIRTEL}\PYG{o}{.}\PYG{n}{W2H}\PYG{o}{\PYGZhy{}}\PYG{l+m+mi}{03}\PYG{n}{G} \PYG{n}{CL}\PYG{p}{(}\PYG{n}{FACILITY}\PYG{p}{)} \PYG{n}{RESET}
  \PYG{n}{PE} \PYG{n}{VIRTEL}\PYG{o}{.}\PYG{n}{W2H}\PYG{o}{\PYGZhy{}}\PYG{l+m+mi}{03}\PYG{n}{U} \PYG{n}{CL}\PYG{p}{(}\PYG{n}{FACILITY}\PYG{p}{)} \PYG{n}{RESET}
  \PYG{n}{PE} \PYG{n}{VIRTEL}\PYG{o}{.}\PYG{n}{W2H}\PYG{o}{\PYGZhy{}}\PYG{l+m+mi}{03}\PYG{n}{A} \PYG{n}{CL}\PYG{p}{(}\PYG{n}{FACILITY}\PYG{p}{)} \PYG{n}{RESET}
  \PYG{n}{PE} \PYG{n}{VIRTEL}\PYG{o}{.}\PYG{n}{CLI}\PYG{o}{\PYGZhy{}}\PYG{l+m+mi}{03}\PYG{n}{G} \PYG{n}{CL}\PYG{p}{(}\PYG{n}{FACILITY}\PYG{p}{)} \PYG{n}{RESET}
  \PYG{n}{PE} \PYG{n}{VIRTEL}\PYG{o}{.}\PYG{n}{CLI}\PYG{o}{\PYGZhy{}}\PYG{l+m+mi}{03}\PYG{n}{U} \PYG{n}{CL}\PYG{p}{(}\PYG{n}{FACILITY}\PYG{p}{)} \PYG{n}{RESET}
  \PYG{n}{PE} \PYG{n}{VIRTEL}\PYG{o}{.}\PYG{n}{CLI}\PYG{o}{\PYGZhy{}}\PYG{l+m+mi}{03}\PYG{n}{A} \PYG{n}{CL}\PYG{p}{(}\PYG{n}{FACILITY}\PYG{p}{)} \PYG{n}{RESET}
  \PYG{n}{PE} \PYG{n}{VIRTEL}\PYG{o}{.}\PYG{n}{W2H}\PYG{o}{\PYGZhy{}}\PYG{l+m+mi}{07} \PYG{n}{CL}\PYG{p}{(}\PYG{n}{FACILITY}\PYG{p}{)} \PYG{n}{RESET}
  \PYG{n}{PE} \PYG{n}{VIRTEL}\PYG{o}{.}\PYG{n}{W2H}\PYG{o}{\PYGZhy{}}\PYG{l+m+mi}{66} \PYG{n}{CL}\PYG{p}{(}\PYG{n}{FACILITY}\PYG{p}{)} \PYG{n}{RESET}
  \PYG{n}{PE} \PYG{n}{VIRTEL}\PYG{o}{.}\PYG{n}{W2H}\PYG{o}{\PYGZhy{}}\PYG{l+m+mi}{80}\PYG{n}{U} \PYG{n}{CL}\PYG{p}{(}\PYG{n}{FACILITY}\PYG{p}{)} \PYG{n}{RESET}
  \PYG{n}{PE} \PYG{n}{VIRTEL}\PYG{o}{.}\PYG{n}{W2H}\PYG{o}{\PYGZhy{}}\PYG{l+m+mi}{80}\PYG{n}{G} \PYG{n}{CL}\PYG{p}{(}\PYG{n}{FACILITY}\PYG{p}{)} \PYG{n}{RESET}
  \PYG{n}{PE} \PYG{n}{VIRTEL}\PYG{o}{.}\PYG{n}{W2H}\PYG{o}{\PYGZhy{}}\PYG{l+m+mi}{80}\PYG{n}{A} \PYG{n}{CL}\PYG{p}{(}\PYG{n}{FACILITY}\PYG{p}{)} \PYG{n}{RESET}
  \PYG{n}{PE} \PYG{n}{VIRTEL}\PYG{o}{.}\PYG{n}{USR}\PYG{o}{\PYGZhy{}}\PYG{n}{DIR} \PYG{n}{CL}\PYG{p}{(}\PYG{n}{FACILITY}\PYG{p}{)} \PYG{n}{RESET}
  \PYG{n}{PE} \PYG{n}{VIRTEL}\PYG{o}{.}\PYG{n}{GRP}\PYG{o}{\PYGZhy{}}\PYG{n}{DIR} \PYG{n}{CL}\PYG{p}{(}\PYG{n}{FACILITY}\PYG{p}{)} \PYG{n}{RESET}
  \PYG{n}{PE} \PYG{n}{VIRTEL}\PYG{o}{.}\PYG{n}{GLB}\PYG{o}{\PYGZhy{}}\PYG{n}{DIR} \PYG{n}{CL}\PYG{p}{(}\PYG{n}{FACILITY}\PYG{p}{)} \PYG{n}{RESET}
  \PYG{n}{PE} \PYG{n}{VIRTEL}\PYG{o}{.}\PYG{n}{W2H}\PYG{o}{\PYGZhy{}}\PYG{l+m+mi}{07} \PYG{n}{CL}\PYG{p}{(}\PYG{n}{FACILITY}\PYG{p}{)} \PYG{n}{ACC}\PYG{p}{(}\PYG{n}{READ}\PYG{p}{)} \PYG{n}{ID}\PYG{p}{(}\PYG{n}{SPGPTECH}\PYG{p}{)}
  \PYG{n}{PE} \PYG{n}{VIRTEL}\PYG{o}{.}\PYG{n}{W2H}\PYG{o}{\PYGZhy{}}\PYG{l+m+mi}{66} \PYG{n}{CL}\PYG{p}{(}\PYG{n}{FACILITY}\PYG{p}{)} \PYG{n}{ACC}\PYG{p}{(}\PYG{n}{READ}\PYG{p}{)} \PYG{n}{ID}\PYG{p}{(}\PYG{n}{SPGPTECH}\PYG{p}{)}
  \PYG{n}{PE} \PYG{n}{VIRTEL}\PYG{o}{.}\PYG{n}{W2H}\PYG{o}{\PYGZhy{}}\PYG{l+m+mi}{03}\PYG{n}{G} \PYG{n}{CL}\PYG{p}{(}\PYG{n}{FACILITY}\PYG{p}{)} \PYG{n}{ACC}\PYG{p}{(}\PYG{n}{READ}\PYG{p}{)} \PYG{n}{ID}\PYG{p}{(}\PYG{n}{SPGPTECH}\PYG{p}{)}
  \PYG{n}{PE} \PYG{n}{VIRTEL}\PYG{o}{.}\PYG{n}{W2H}\PYG{o}{\PYGZhy{}}\PYG{l+m+mi}{03}\PYG{n}{U} \PYG{n}{CL}\PYG{p}{(}\PYG{n}{FACILITY}\PYG{p}{)} \PYG{n}{ACC}\PYG{p}{(}\PYG{n}{READ}\PYG{p}{)} \PYG{n}{ID}\PYG{p}{(}\PYG{n}{SPGPTECH}\PYG{p}{)}
  \PYG{n}{PE} \PYG{n}{VIRTEL}\PYG{o}{.}\PYG{n}{W2H}\PYG{o}{\PYGZhy{}}\PYG{l+m+mi}{03}\PYG{n}{A} \PYG{n}{CL}\PYG{p}{(}\PYG{n}{FACILITY}\PYG{p}{)} \PYG{n}{ACC}\PYG{p}{(}\PYG{n}{READ}\PYG{p}{)} \PYG{n}{ID}\PYG{p}{(}\PYG{n}{SPGPTECH}\PYG{p}{)}
  \PYG{n}{PE} \PYG{n}{VIRTEL}\PYG{o}{.}\PYG{n}{CLI}\PYG{o}{\PYGZhy{}}\PYG{l+m+mi}{03}\PYG{n}{G} \PYG{n}{CL}\PYG{p}{(}\PYG{n}{FACILITY}\PYG{p}{)} \PYG{n}{ACC}\PYG{p}{(}\PYG{n}{READ}\PYG{p}{)} \PYG{n}{ID}\PYG{p}{(}\PYG{n}{SPGPTECH}\PYG{p}{)}
  \PYG{n}{PE} \PYG{n}{VIRTEL}\PYG{o}{.}\PYG{n}{CLI}\PYG{o}{\PYGZhy{}}\PYG{l+m+mi}{03}\PYG{n}{U} \PYG{n}{CL}\PYG{p}{(}\PYG{n}{FACILITY}\PYG{p}{)} \PYG{n}{ACC}\PYG{p}{(}\PYG{n}{READ}\PYG{p}{)} \PYG{n}{ID}\PYG{p}{(}\PYG{n}{SPGPTECH}\PYG{p}{)}
  \PYG{n}{PE} \PYG{n}{VIRTEL}\PYG{o}{.}\PYG{n}{CLI}\PYG{o}{\PYGZhy{}}\PYG{l+m+mi}{03}\PYG{n}{A} \PYG{n}{CL}\PYG{p}{(}\PYG{n}{FACILITY}\PYG{p}{)} \PYG{n}{ACC}\PYG{p}{(}\PYG{n}{READ}\PYG{p}{)} \PYG{n}{ID}\PYG{p}{(}\PYG{n}{SPGPTECH}\PYG{p}{)}
  \PYG{n}{PE} \PYG{n}{VIRTEL}\PYG{o}{.}\PYG{n}{W2H}\PYG{o}{\PYGZhy{}}\PYG{l+m+mi}{80}\PYG{n}{U} \PYG{n}{CL}\PYG{p}{(}\PYG{n}{FACILITY}\PYG{p}{)} \PYG{n}{ACC}\PYG{p}{(}\PYG{n}{READ}\PYG{p}{)} \PYG{n}{ID}\PYG{p}{(}\PYG{n}{SPGPTECH}\PYG{p}{)}
  \PYG{n}{PE} \PYG{n}{VIRTEL}\PYG{o}{.}\PYG{n}{W2H}\PYG{o}{\PYGZhy{}}\PYG{l+m+mi}{80}\PYG{n}{G} \PYG{n}{CL}\PYG{p}{(}\PYG{n}{FACILITY}\PYG{p}{)} \PYG{n}{ACC}\PYG{p}{(}\PYG{n}{READ}\PYG{p}{)} \PYG{n}{ID}\PYG{p}{(}\PYG{n}{SPGPTECH}\PYG{p}{)}
  \PYG{n}{PE} \PYG{n}{VIRTEL}\PYG{o}{.}\PYG{n}{W2H}\PYG{o}{\PYGZhy{}}\PYG{l+m+mi}{80}\PYG{n}{A} \PYG{n}{CL}\PYG{p}{(}\PYG{n}{FACILITY}\PYG{p}{)} \PYG{n}{ACC}\PYG{p}{(}\PYG{n}{READ}\PYG{p}{)} \PYG{n}{ID}\PYG{p}{(}\PYG{n}{SPGPTECH}\PYG{p}{)}
  \PYG{n}{PE} \PYG{n}{VIRTEL}\PYG{o}{.}\PYG{n}{USR}\PYG{o}{\PYGZhy{}}\PYG{n}{DIR} \PYG{n}{CL}\PYG{p}{(}\PYG{n}{FACILITY}\PYG{p}{)} \PYG{n}{ACC}\PYG{p}{(}\PYG{n}{READ}\PYG{p}{)} \PYG{n}{ID}\PYG{p}{(}\PYG{n}{SPGPTECH}\PYG{p}{)}
  \PYG{n}{PE} \PYG{n}{VIRTEL}\PYG{o}{.}\PYG{n}{GRP}\PYG{o}{\PYGZhy{}}\PYG{n}{DIR} \PYG{n}{CL}\PYG{p}{(}\PYG{n}{FACILITY}\PYG{p}{)} \PYG{n}{ACC}\PYG{p}{(}\PYG{n}{READ}\PYG{p}{)} \PYG{n}{ID}\PYG{p}{(}\PYG{n}{SPGPTECH}\PYG{p}{)}
  \PYG{n}{PE} \PYG{n}{VIRTEL}\PYG{o}{.}\PYG{n}{GLB}\PYG{o}{\PYGZhy{}}\PYG{n}{DIR} \PYG{n}{CL}\PYG{p}{(}\PYG{n}{FACILITY}\PYG{p}{)} \PYG{n}{ACC}\PYG{p}{(}\PYG{n}{READ}\PYG{p}{)} \PYG{n}{ID}\PYG{p}{(}\PYG{n}{SPGPTECH}\PYG{p}{)}
  \PYG{o}{/}\PYG{o}{*}\PYG{o}{\PYGZhy{}}\PYG{o}{\PYGZhy{}}\PYG{o}{\PYGZhy{}}\PYG{o}{\PYGZhy{}}\PYG{o}{\PYGZhy{}}\PYG{o}{\PYGZhy{}}\PYG{o}{\PYGZhy{}}\PYG{o}{\PYGZhy{}}\PYG{o}{\PYGZhy{}}\PYG{o}{\PYGZhy{}}\PYG{o}{\PYGZhy{}}\PYG{o}{\PYGZhy{}}\PYG{o}{\PYGZhy{}}\PYG{o}{\PYGZhy{}}\PYG{o}{\PYGZhy{}}\PYG{o}{\PYGZhy{}}\PYG{o}{\PYGZhy{}}\PYG{o}{\PYGZhy{}}\PYG{o}{\PYGZhy{}}\PYG{o}{\PYGZhy{}}\PYG{o}{\PYGZhy{}}\PYG{o}{\PYGZhy{}}\PYG{o}{\PYGZhy{}}\PYG{o}{\PYGZhy{}}\PYG{o}{\PYGZhy{}}\PYG{o}{\PYGZhy{}}\PYG{o}{\PYGZhy{}}\PYG{o}{\PYGZhy{}}\PYG{o}{\PYGZhy{}}\PYG{o}{\PYGZhy{}}\PYG{o}{\PYGZhy{}}\PYG{o}{\PYGZhy{}}\PYG{o}{\PYGZhy{}}\PYG{o}{\PYGZhy{}}\PYG{o}{\PYGZhy{}}\PYG{o}{\PYGZhy{}}\PYG{o}{\PYGZhy{}}\PYG{o}{\PYGZhy{}}\PYG{o}{\PYGZhy{}}\PYG{o}{\PYGZhy{}}\PYG{o}{\PYGZhy{}}\PYG{o}{\PYGZhy{}}\PYG{o}{\PYGZhy{}}\PYG{o}{\PYGZhy{}}\PYG{o}{\PYGZhy{}}\PYG{o}{\PYGZhy{}}\PYG{o}{\PYGZhy{}}\PYG{o}{\PYGZhy{}}\PYG{o}{\PYGZhy{}}\PYG{o}{\PYGZhy{}}\PYG{o}{\PYGZhy{}}\PYG{o}{\PYGZhy{}}\PYG{o}{\PYGZhy{}}\PYG{o}{\PYGZhy{}}\PYG{o}{\PYGZhy{}}\PYG{o}{*}\PYG{o}{/}
  \PYG{o}{/}\PYG{o}{*} \PYG{n}{REFRESH} \PYG{n}{THE} \PYG{n}{RACF} \PYG{n}{PROFILES} \PYG{o}{*}\PYG{o}{/}
  \PYG{o}{/}\PYG{o}{*}\PYG{o}{\PYGZhy{}}\PYG{o}{\PYGZhy{}}\PYG{o}{\PYGZhy{}}\PYG{o}{\PYGZhy{}}\PYG{o}{\PYGZhy{}}\PYG{o}{\PYGZhy{}}\PYG{o}{\PYGZhy{}}\PYG{o}{\PYGZhy{}}\PYG{o}{\PYGZhy{}}\PYG{o}{\PYGZhy{}}\PYG{o}{\PYGZhy{}}\PYG{o}{\PYGZhy{}}\PYG{o}{\PYGZhy{}}\PYG{o}{\PYGZhy{}}\PYG{o}{\PYGZhy{}}\PYG{o}{\PYGZhy{}}\PYG{o}{\PYGZhy{}}\PYG{o}{\PYGZhy{}}\PYG{o}{\PYGZhy{}}\PYG{o}{\PYGZhy{}}\PYG{o}{\PYGZhy{}}\PYG{o}{\PYGZhy{}}\PYG{o}{\PYGZhy{}}\PYG{o}{\PYGZhy{}}\PYG{o}{\PYGZhy{}}\PYG{o}{\PYGZhy{}}\PYG{o}{\PYGZhy{}}\PYG{o}{\PYGZhy{}}\PYG{o}{\PYGZhy{}}\PYG{o}{\PYGZhy{}}\PYG{o}{\PYGZhy{}}\PYG{o}{\PYGZhy{}}\PYG{o}{\PYGZhy{}}\PYG{o}{\PYGZhy{}}\PYG{o}{\PYGZhy{}}\PYG{o}{\PYGZhy{}}\PYG{o}{\PYGZhy{}}\PYG{o}{\PYGZhy{}}\PYG{o}{\PYGZhy{}}\PYG{o}{\PYGZhy{}}\PYG{o}{\PYGZhy{}}\PYG{o}{\PYGZhy{}}\PYG{o}{\PYGZhy{}}\PYG{o}{\PYGZhy{}}\PYG{o}{\PYGZhy{}}\PYG{o}{\PYGZhy{}}\PYG{o}{\PYGZhy{}}\PYG{o}{\PYGZhy{}}\PYG{o}{\PYGZhy{}}\PYG{o}{\PYGZhy{}}\PYG{o}{\PYGZhy{}}\PYG{o}{\PYGZhy{}}\PYG{o}{\PYGZhy{}}\PYG{o}{\PYGZhy{}}\PYG{o}{\PYGZhy{}}\PYG{o}{*}\PYG{o}{/}
  \PYG{n}{SETR} \PYG{n}{REFRESH} \PYG{n}{RACLIST}\PYG{p}{(}\PYG{n}{FACILITY}\PYG{p}{)}
\PYG{o}{/}\PYG{o}{*}
\PYG{o}{/}\PYG{o}{/}
\end{sphinxVerbatim}

\sphinxAtStartPar
An administrator would have READ access to all profiles whereas a user may only have access to the some of the profiles.

\begin{sphinxadmonition}{note}{Note:}
\sphinxAtStartPar
If you are implementing the Virtel USERPARM feature you should also add the following profiles:

\begin{sphinxVerbatim}[commandchars=\\\{\}]
\PYG{n}{RDEF} \PYG{n}{FACILITY} \PYG{n}{VIRTEL}\PYG{o}{.}\PYG{n}{USERPARM} \PYG{n}{UACC}\PYG{p}{(}\PYG{n}{READ}\PYG{p}{)} \PYG{o}{/}\PYG{o}{*} \PYG{n}{Allow} \PYG{n+nb}{all} \PYG{n}{users} \PYG{n}{access} \PYG{n}{to} \PYG{n}{USERPARM} \PYG{o}{*}\PYG{o}{/}
\end{sphinxVerbatim}
\end{sphinxadmonition}

\sphinxAtStartPar
For more information about protecting VIRTEL Web Access resources, refer to the Security section of the the VIRTEL User Guide.

\index{Security@\spxentry{Security}!ACF Security@\spxentry{ACF Security}}\index{ACF Security@\spxentry{ACF Security}!Security@\spxentry{Security}}\ignorespaces 

\section{How to activate ACF2 Security}
\label{\detokenize{Installation_Guide:how-to-activate-acf2-security}}\label{\detokenize{Installation_Guide:index-184}}
\sphinxAtStartPar
Perform the following steps to activate ACF2 security for VIRTEL in the z/OS environment.


\subsection{Modify the VIRTCT}
\label{\detokenize{Installation_Guide:id10}}
\sphinxAtStartPar
In the VIRTCTxx member of the VIRTEL CNTL library, replace the default parameters:

\begin{sphinxVerbatim}[commandchars=\\\{\}]
\PYG{n}{SECUR}\PYG{o}{=}\PYG{n}{NO}\PYG{p}{,}\PYG{n}{RAPPL}\PYG{o}{=}\PYG{n}{VIRTSERV}\PYG{p}{,}\PYG{n}{RNODE}\PYG{o}{=}\PYG{n}{VIRTNODE}
\end{sphinxVerbatim}

\sphinxAtStartPar
with the following parameters:

\begin{sphinxVerbatim}[commandchars=\\\{\}]
\PYG{n}{SECUR}\PYG{o}{=}\PYG{p}{(}\PYG{n}{RACROUTE}\PYG{p}{,}\PYG{n}{ACF2}\PYG{p}{)}\PYG{p}{,}
\PYG{n}{RAPPL}\PYG{o}{=}\PYG{n}{VIRTAPPL}\PYG{p}{,}\PYG{n}{RNODE}\PYG{o}{=}\PYG{n}{VIRTNODE}\PYG{p}{,}
\end{sphinxVerbatim}

\sphinxAtStartPar
This tells VIRTEL that the security definitions for calls to external servers are stored in the VIRTAPPL resource class, and that the security definitions for access to VIRTEL transactions, directories, and nodes are stored in the VIRTNODE resource class. You can choose your own resource class names for each VIRTEL. Multiple VIRTEL started tasks can share the same resource class names if their security definitions are identical.

\sphinxAtStartPar
Having updated the VIRTCTxx source member, reassemble and relink the VIRTCT into VIRTEL LOADLIB using the sample JCL in member ASMTCT of the VIRTEL CNTL library. Stop and start VIRTEL to pick up the new VIRTCT.


\subsection{Determine the ACF2 resource type}
\label{\detokenize{Installation_Guide:determine-the-acf2-resource-type}}
\sphinxAtStartPar
ACF2 maps each 8\sphinxhyphen{}character SAF resource class name to a 3\sphinxhyphen{}character ACF2 resource type. By default, the resource type is the first three characters of the resource class name, so classes VIRTAPPL and VIRTNODE both map to resource type VIR. You can use the ACF2 CLASMAP record to translate the resource classes to different resource types if required.


\subsection{Add ACF2 definitions}
\label{\detokenize{Installation_Guide:add-acf2-definitions}}
\sphinxAtStartPar
A example job to add VIRTEL definitions for ACF2 can be found in member ACF2DEF in the VIRTEL SAMPLIB. The commands in this job are explained in the following paragraphs.


\subsubsection{Create OMVS segment for VIRTEL}
\label{\detokenize{Installation_Guide:create-omvs-segment-for-virtel}}
\begin{sphinxVerbatim}[commandchars=\\\{\}]
\PYG{n}{SET} \PYG{n}{PROFILE}\PYG{p}{(}\PYG{n}{VIRTSTC}\PYG{p}{)} \PYG{n}{DIV}\PYG{p}{(}\PYG{n}{OMVS}\PYG{p}{)}
\PYG{n}{INSERT} \PYG{n}{VIRTSTC} \PYG{n}{UID}\PYG{p}{(}\PYG{n}{nn}\PYG{p}{)} \PYG{n}{HOME}\PYG{p}{(}\PYG{l+s+s1}{\PYGZsq{}}\PYG{l+s+s1}{/}\PYG{l+s+s1}{\PYGZsq{}}\PYG{p}{)} \PYG{n}{PROGRAM}\PYG{p}{(}\PYG{l+s+s1}{\PYGZsq{}}\PYG{l+s+s1}{/bin/sh}\PYG{l+s+s1}{\PYGZsq{}}\PYG{p}{)}
\end{sphinxVerbatim}

\sphinxAtStartPar
\sphinxstyleemphasis{ACF2DEF : ACF2 commands to create OMVS segment for VIRTEL}

\sphinxAtStartPar
This command allows VIRTEL to access the TCP/IP stack.


\subsubsection{Add permissions for VIRTEL administrators}
\label{\detokenize{Installation_Guide:add-permissions-for-virtel-administrators}}
\begin{sphinxVerbatim}[commandchars=\\\{\}]
\PYGZdl{}KEY(********) TYPE(VIR) UID(******** admin\PYGZhy{}group\PYGZhy{}name) SERVICE(READ)
\end{sphinxVerbatim}

\sphinxAtStartPar
\sphinxstyleemphasis{ACF2DEF : ACF2 command to grant administrator permissions}

\sphinxAtStartPar
This command permits users in group admin\sphinxhyphen{}group\sphinxhyphen{}name to access all VIRTEL transactions and administrator functions.

\index{Security@\spxentry{Security}!ACF Security \sphinxhyphen{} Adding Users@\spxentry{ACF Security \sphinxhyphen{} Adding Users}}\index{ACF Security \sphinxhyphen{} Adding Users@\spxentry{ACF Security \sphinxhyphen{} Adding Users}!Security@\spxentry{Security}}\ignorespaces 

\subsubsection{Add permissions for VIRTEL general users}
\label{\detokenize{Installation_Guide:add-permissions-for-virtel-general-users}}\label{\detokenize{Installation_Guide:index-185}}
\begin{sphinxVerbatim}[commandchars=\\\{\}]
\PYGZdl{}KEY(W2H\PYGZhy{}10) TYPE(VIR) UID(******** user\PYGZhy{}group\PYGZhy{}name) SERVICE(READ)
\PYGZdl{}KEY(CLI\PYGZhy{}****) TYPE(VIR) UID(******** user\PYGZhy{}group\PYGZhy{}name) SERVICE(READ)
\end{sphinxVerbatim}

\sphinxAtStartPar
\sphinxstyleemphasis{ACF2DEF : ACF2 commands to grant general user permissions}

\sphinxAtStartPar
These commands permit users in group user\sphinxhyphen{}group\sphinxhyphen{}name to access specific VIRTEL transactions.

\sphinxAtStartPar
Resource W2H\sphinxhyphen{}10 permits specific access to the CICS Web Access transaction on port 41001. Resource CLI\sphinxhyphen{}** is a generic resource which permits access to customer\sphinxhyphen{}defined transactions (internal name CLI\sphinxhyphen{}nn) on port 41002 and to the directory CLI\sphinxhyphen{}DIR.

\sphinxAtStartPar
8.2.3.4. Allow everyone to use the 3270 LOGOFF transactions
\begin{description}
\sphinxlineitem{::}
\sphinxAtStartPar
\$KEY(PC\sphinxhyphen{}0020) TYPE(VIR) UID(\sphinxstylestrong{****} \sphinxstylestrong{****}) SERVICE(READ)

\end{description}

\sphinxAtStartPar
\sphinxstyleemphasis{ACF2DEF : ACF2 command to permit access to 3270 Logoff transaction}

\sphinxAtStartPar
This command permits all users to use the 3270 Logoff transaction, whose internal name is PC\sphinxhyphen{}0020.

\newpage

\index{Security@\spxentry{Security}!Top Secret Security platform.@\spxentry{Top Secret Security platform.}}\index{Top Secret Security platform.@\spxentry{Top Secret Security platform.}!Security@\spxentry{Security}}\ignorespaces 

\section{How To Activate Top Secret (TSS) Security Perform}
\label{\detokenize{Installation_Guide:how-to-activate-top-secret-tss-security-perform}}\label{\detokenize{Installation_Guide:index-186}}
\sphinxAtStartPar
Perform the following steps to activate TSS security for VIRTEL in the z/OS environment.


\subsection{Modify the TCT}
\label{\detokenize{Installation_Guide:modify-the-tct}}
\sphinxAtStartPar
In the VIRTCTxx member of the VIRTEL CNTL library, replace the default parameters:

\begin{sphinxVerbatim}[commandchars=\\\{\}]
\PYG{n}{SECUR}\PYG{o}{=}\PYG{n}{NO}\PYG{p}{,}\PYG{n}{RAPPL}\PYG{o}{=}\PYG{n}{VIRTSERV}\PYG{p}{,}\PYG{n}{RNODE}\PYG{o}{=}\PYG{n}{VIRTNODE}
\end{sphinxVerbatim}

\sphinxAtStartPar
with the following parameters:

\begin{sphinxVerbatim}[commandchars=\\\{\}]
\PYG{n}{SECUR}\PYG{o}{=}\PYG{p}{(}\PYG{n}{RACROUTE}\PYG{p}{,}\PYG{n}{TOPS}\PYG{p}{)}\PYG{p}{,}
\PYG{n}{RAPPL}\PYG{o}{=}\PYG{n}{VIRTAPPL}\PYG{p}{,}\PYG{n}{RNODE}\PYG{o}{=}\PYG{n}{VIRTNODE}\PYG{p}{,}
\end{sphinxVerbatim}

\sphinxAtStartPar
This tells VIRTEL that the security definitions for calls to external servers are stored in the VIRTAPPL resource class, and that the security definitions for access to VIRTEL transactions, directories, and nodes are stored in the VIRTNODE resource class. You can choose your own resource class names for each VIRTEL.  Multiple VIRTEL started tasks can share the same resource class names if their security definitions are identical.

\sphinxAtStartPar
Having updated the VIRTCTxx source member, reassemble and relink the VIRTCT into VIRTEL LOADLIB using the sample JCL in member ASMTCT of the VIRTEL CNTL library. Stop and start VIRTEL to pick up the new VIRTCT.

\index{Security@\spxentry{Security}!Adding TSS definitions@\spxentry{Adding TSS definitions}}\index{Adding TSS definitions@\spxentry{Adding TSS definitions}!Security@\spxentry{Security}}\ignorespaces 

\subsection{Add TSS definitions}
\label{\detokenize{Installation_Guide:add-tss-definitions}}\label{\detokenize{Installation_Guide:index-187}}
\sphinxAtStartPar
A example job to add VIRTEL definitions for TSS can be found in member TOPSDEF in the VIRTEL SAMPLIB. The commands in this job are explained in the following paragraphs.

\sphinxAtStartPar
8.3.2.1 Add TSS definitions

\begin{sphinxVerbatim}[commandchars=\\\{\}]
\PYG{n}{TSS} \PYG{n}{MODIFY} \PYG{p}{(}\PYG{n}{FACILITY}\PYG{p}{(}\PYG{n}{USERnn}\PYG{o}{=}\PYG{n}{NAME}\PYG{o}{=}\PYG{n}{VIRTFAC}\PYG{p}{)}\PYG{p}{)}
\PYG{n}{TSS} \PYG{n}{MODIFY} \PYG{p}{(}\PYG{n}{FACILITY}\PYG{p}{(}\PYG{n}{VIRTFAC}\PYG{o}{=}\PYG{n}{PGM}\PYG{o}{=}\PYG{n}{VIR}\PYG{p}{)}\PYG{p}{)}
\PYG{n}{TSS} \PYG{n}{MODIFY} \PYG{p}{(}\PYG{n}{FACILITY}\PYG{p}{(}\PYG{n}{VIRTFAC}\PYG{o}{=}\PYG{n}{ACTIVE}\PYG{p}{)}\PYG{p}{)}
\PYG{n}{TSS} \PYG{n}{MODIFY} \PYG{p}{(}\PYG{n}{FACILITY}\PYG{p}{(}\PYG{n}{VIRTFAC}\PYG{o}{=}\PYG{n}{ASUBM}\PYG{p}{)}\PYG{p}{)}
\PYG{n}{TSS} \PYG{n}{MODIFY} \PYG{p}{(}\PYG{n}{FACILITY}\PYG{p}{(}\PYG{n}{VIRTFAC}\PYG{o}{=}\PYG{n}{AUTHINIT}\PYG{p}{)}\PYG{p}{)}
\PYG{n}{TSS} \PYG{n}{MODIFY} \PYG{p}{(}\PYG{n}{FACILITY}\PYG{p}{(}\PYG{n}{VIRTFAC}\PYG{o}{=}\PYG{n}{DEFACID}\PYG{p}{(}\PYG{o}{*}\PYG{n}{NONE}\PYG{o}{*}\PYG{p}{)}\PYG{p}{)}\PYG{p}{)}
\PYG{n}{TSS} \PYG{n}{MODIFY} \PYG{p}{(}\PYG{n}{FACILITY}\PYG{p}{(}\PYG{n}{VIRTFAC}\PYG{o}{=}\PYG{n}{LUMSG}\PYG{p}{)}\PYG{p}{)}
\PYG{n}{TSS} \PYG{n}{MODIFY} \PYG{p}{(}\PYG{n}{FACILITY}\PYG{p}{(}\PYG{n}{VIRTFAC}\PYG{o}{=}\PYG{n}{MODE}\PYG{o}{=}\PYG{n}{FAIL}\PYG{p}{)}\PYG{p}{)}
\PYG{n}{TSS} \PYG{n}{MODIFY} \PYG{p}{(}\PYG{n}{FACILITY}\PYG{p}{(}\PYG{n}{VIRTFAC}\PYG{o}{=}\PYG{n}{MULTIUSER}\PYG{p}{)}\PYG{p}{)}
\PYG{n}{TSS} \PYG{n}{MODIFY} \PYG{p}{(}\PYG{n}{FACILITY}\PYG{p}{(}\PYG{n}{VIRTFAC}\PYG{o}{=}\PYG{n}{NOABEND}\PYG{p}{)}\PYG{p}{)}
\PYG{n}{TSS} \PYG{n}{MODIFY} \PYG{p}{(}\PYG{n}{FACILITY}\PYG{p}{(}\PYG{n}{VIRTFAC}\PYG{o}{=}\PYG{n}{NOAUDIT}\PYG{p}{)}\PYG{p}{)}
\PYG{n}{TSS} \PYG{n}{MODIFY} \PYG{p}{(}\PYG{n}{FACILITY}\PYG{p}{(}\PYG{n}{VIRTFAC}\PYG{o}{=}\PYG{n}{NOPROMPT}\PYG{p}{)}\PYG{p}{)}
\PYG{n}{TSS} \PYG{n}{MODIFY} \PYG{p}{(}\PYG{n}{FACILITY}\PYG{p}{(}\PYG{n}{VIRTFAC}\PYG{o}{=}\PYG{n}{NORES}\PYG{p}{)}\PYG{p}{)}
\PYG{n}{TSS} \PYG{n}{MODIFY} \PYG{p}{(}\PYG{n}{FACILITY}\PYG{p}{(}\PYG{n}{VIRTFAC}\PYG{o}{=}\PYG{n}{NOTSOC}\PYG{p}{)}\PYG{p}{)}
\PYG{n}{TSS} \PYG{n}{MODIFY} \PYG{p}{(}\PYG{n}{FACILITY}\PYG{p}{(}\PYG{n}{VIRTFAC}\PYG{o}{=}\PYG{n}{NOXDEF}\PYG{p}{)}\PYG{p}{)}
\PYG{n}{TSS} \PYG{n}{MODIFY} \PYG{p}{(}\PYG{n}{FACILITY}\PYG{p}{(}\PYG{n}{VIRTFAC}\PYG{o}{=}\PYG{n}{RNDPW}\PYG{p}{)}\PYG{p}{)}
\PYG{n}{TSS} \PYG{n}{MODIFY} \PYG{p}{(}\PYG{n}{FACILITY}\PYG{p}{(}\PYG{n}{VIRTFAC}\PYG{o}{=}\PYG{n}{SHRPRF}\PYG{p}{)}\PYG{p}{)}
\PYG{n}{TSS} \PYG{n}{MODIFY} \PYG{p}{(}\PYG{n}{FACILITY}\PYG{p}{(}\PYG{n}{VIRTFAC}\PYG{o}{=}\PYG{n}{SIGN}\PYG{p}{(}\PYG{n}{M}\PYG{p}{)}\PYG{p}{)}\PYG{p}{)}
\PYG{n}{TSS} \PYG{n}{MODIFY} \PYG{p}{(}\PYG{n}{FACILITY}\PYG{p}{(}\PYG{n}{VIRTFAC}\PYG{o}{=}\PYG{n}{STMSG}\PYG{p}{)}\PYG{p}{)}
\PYG{n}{TSS} \PYG{n}{MODIFY} \PYG{p}{(}\PYG{n}{FACILITY}\PYG{p}{(}\PYG{n}{VIRTFAC}\PYG{o}{=}\PYG{n}{WARNPW}\PYG{p}{)}\PYG{p}{)}
\end{sphinxVerbatim}

\sphinxAtStartPar
\sphinxstyleemphasis{TOPSDEF : TSS commands to create VIRTEL facility}

\sphinxAtStartPar
VIRTFAC is the VIRTEL facility name. You may choose your own name, but you must replace VIRTFAC in all of the following commands by the name you chose. Replace USERnn by the name of an unused user facility (for example, USER34).


\subsubsection{Create VIRTEL division and department}
\label{\detokenize{Installation_Guide:create-virtel-division-and-department}}
\begin{sphinxVerbatim}[commandchars=\\\{\}]
\PYG{n}{TSS} \PYG{n}{CREATE}\PYG{p}{(}\PYG{n}{VIRTDIV}\PYG{p}{)} \PYG{n}{NAME}\PYG{p}{(}\PYG{l+s+s1}{\PYGZsq{}}\PYG{l+s+s1}{VIRTEL DIVISION}\PYG{l+s+s1}{\PYGZsq{}}\PYG{p}{)} \PYG{n}{TYPE}\PYG{p}{(}\PYG{n}{DIVISION}\PYG{p}{)}
\PYG{n}{TSS} \PYG{n}{CREATE}\PYG{p}{(}\PYG{n}{VIRTDEP}\PYG{p}{)} \PYG{n}{NAME}\PYG{p}{(}\PYG{l+s+s1}{\PYGZsq{}}\PYG{l+s+s1}{VIRTEL DEPT}\PYG{l+s+s1}{\PYGZsq{}}\PYG{p}{)} \PYG{n}{TYPE}\PYG{p}{(}\PYG{n}{DEPARTMENT}\PYG{p}{)} \PYG{o}{+}
        \PYG{n}{DIVISION}\PYG{p}{(}\PYG{n}{VIRTDIV}\PYG{p}{)}
\end{sphinxVerbatim}

\sphinxAtStartPar
\sphinxstyleemphasis{TOPSDEF : TSS commands to create VIRTEL division and department}

\sphinxAtStartPar
A division and department are created to contain the VIRTEL resources. You can choose your own names, or you can use an existing division and department. If you choose to use different names then the following commands must be modified accordingly.


\subsubsection{Create ACID for the VIRTEL started task}
\label{\detokenize{Installation_Guide:create-acid-for-the-virtel-started-task}}
\begin{sphinxVerbatim}[commandchars=\\\{\}]
\PYG{n}{TSS} \PYG{n}{CREATE}\PYG{p}{(}\PYG{n}{VIRTSTC}\PYG{p}{)} \PYG{n}{NAME}\PYG{p}{(}\PYG{l+s+s1}{\PYGZsq{}}\PYG{l+s+s1}{VIRTEL STC}\PYG{l+s+s1}{\PYGZsq{}}\PYG{p}{)} \PYG{n}{TYPE}\PYG{p}{(}\PYG{n}{USER}\PYG{p}{)} \PYG{o}{+}
        \PYG{n}{FAC}\PYG{p}{(}\PYG{n}{BATCH}\PYG{p}{,}\PYG{n}{STC}\PYG{p}{)} \PYG{n}{PASSWORD}\PYG{p}{(}\PYG{n}{NOPW}\PYG{p}{,}\PYG{l+m+mi}{0}\PYG{p}{)} \PYG{n}{DEPARTMENT}\PYG{p}{(}\PYG{n}{VIRTDEP}\PYG{p}{)} \PYG{o}{+}
        \PYG{n}{MASTFAC}\PYG{p}{(}\PYG{n}{VIRTFAC}\PYG{p}{)} \PYG{n}{NODSNCHK} \PYG{n}{NOVOLCHK}
\end{sphinxVerbatim}

\sphinxAtStartPar
\sphinxstyleemphasis{TOPSDEF : TSS commands to create ACID for VIRTEL started task}

\sphinxAtStartPar
An ACID named VIRTSTC is defined in the BATCH and STC facilities to allow VIRTEL to execute in both batch and started task modes. It has no password and it belongs to department VIRTDEP.

\sphinxAtStartPar
The definition in the BATCH facility is not compulsory and is only required if VIRTEL might be executed as a batch job.


\subsubsection{Assign VIRTEL procedure name to the ACID}
\label{\detokenize{Installation_Guide:assign-virtel-procedure-name-to-the-acid}}
\begin{sphinxVerbatim}[commandchars=\\\{\}]
\PYG{n}{TSS} \PYG{n}{ADDTO}\PYG{p}{(}\PYG{n}{STC}\PYG{p}{)} \PYG{n}{PROCNAME}\PYG{p}{(}\PYG{n}{VIRTEL}\PYG{p}{)} \PYG{n}{ACID}\PYG{p}{(}\PYG{n}{VIRTSTC}\PYG{p}{)}
\end{sphinxVerbatim}

\sphinxAtStartPar
\sphinxstyleemphasis{TOPSDEF : TSS commands to associate ACID with VIRTEL started task}

\sphinxAtStartPar
This command associates the VIRTEL started task with the VIRTSTC ACID. VIRTEL is the name of the started task procedure in the system or user PROCLIB.


\subsubsection{Create OMVS segment for VIRTEL}
\label{\detokenize{Installation_Guide:id11}}
\begin{sphinxVerbatim}[commandchars=\\\{\}]
\PYG{n}{TSS} \PYG{n}{ADDTO}\PYG{p}{(}\PYG{n}{VIRTSTC}\PYG{p}{)} \PYG{n}{UID}\PYG{p}{(}\PYG{n}{nn}\PYG{p}{)} \PYG{n}{DFLTGRP}\PYG{p}{(}\PYG{n}{OMVSGRP}\PYG{p}{)} \PYG{n}{GROUP}\PYG{p}{(}\PYG{n}{OMVSGRP}\PYG{p}{)} \PYG{o}{+}
        \PYG{n}{OMVSPGM}\PYG{p}{(}\PYG{l+s+s1}{\PYGZsq{}}\PYG{l+s+s1}{/bin/sh}\PYG{l+s+s1}{\PYGZsq{}}\PYG{p}{)} \PYG{n}{HOME}\PYG{p}{(}\PYG{l+s+s1}{\PYGZsq{}}\PYG{l+s+s1}{/}\PYG{l+s+s1}{\PYGZsq{}}\PYG{p}{)}
\end{sphinxVerbatim}

\sphinxAtStartPar
\sphinxstyleemphasis{TOPSDEF : TSS commands to create OMVS segment for VIRTEL}

\sphinxAtStartPar
This command allows VIRTEL to access the TCP/IP stack. The name of the group (OMVSGRP in this example) should be adapted according to your naming conventions.


\subsubsection{Define VIRTEL resource classes in the RDT}
\label{\detokenize{Installation_Guide:define-virtel-resource-classes-in-the-rdt}}
\begin{sphinxVerbatim}[commandchars=\\\{\}]
\PYG{n}{TSS} \PYG{n}{ADDTO}\PYG{p}{(}\PYG{n}{RDT}\PYG{p}{)} \PYG{n}{RESCLASS}\PYG{p}{(}\PYG{n}{VIRTAPPL}\PYG{p}{)}
\PYG{n}{TSS} \PYG{n}{ADDTO}\PYG{p}{(}\PYG{n}{RDT}\PYG{p}{)} \PYG{n}{RESCLASS}\PYG{p}{(}\PYG{n}{VIRTNODE}\PYG{p}{)}
\end{sphinxVerbatim}

\sphinxAtStartPar
\sphinxstyleemphasis{TOPSDEF : TSS commands to define VIRTEL resource classes}

\sphinxAtStartPar
VIRTEL uses two resource classes for security management.
\begin{quote}

\sphinxAtStartPar
The first class, whose name must match the RAPPL parameter of the VIRTCT, contains the names of VTAM applications used by VIRTEL Multi\sphinxhyphen{}Session, and the names of external servers used by VIRTEL Outgoing Calls (Videotex).
\end{quote}

\sphinxAtStartPar
The second class, whose name must match the RNODE parameter of the  VIRTCT, contains node names for VIRTEL Incoming Calls, the names of sub\sphinxhyphen{}applications and directories for VIRTEL administration, and the internal names of transactions associated with entry points for VIRTEL Web Access.


\subsubsection{Attach resources to VIRTEL department}
\label{\detokenize{Installation_Guide:attach-resources-to-virtel-department}}
\begin{sphinxVerbatim}[commandchars=\\\{\}]
TSS ADDTO(VIRTDEP) VIRTAPPL(AE) Annuaire électronique
TSS ADDTO(VIRTDEP) VIRTAPPL(SNCF) Serveur SNCF
TSS ADDTO(VIRTDEP) VIRTAPPL(\PYGZdl{}\PYGZdl{}ALLSRV) Authorize all servers
TSS ADDTO(VIRTDEP) VIRTNODE(\PYGZdl{}\PYGZdl{}ARBO\PYGZdl{}\PYGZdl{}) Arborescence (admin.)
TSS ADDTO(VIRTDEP) VIRTNODE(\PYGZdl{}\PYGZdl{}UTIL\PYGZdl{}\PYGZdl{}) Users
TSS ADDTO(VIRTDEP) VIRTNODE(\PYGZdl{}\PYGZdl{}APPL\PYGZdl{}\PYGZdl{}) Applications
TSS ADDTO(VIRTDEP) VIRTNODE(\PYGZdl{}\PYGZdl{}CMP3\PYGZdl{}\PYGZdl{}) Compression
TSS ADDTO(VIRTDEP) VIRTNODE(\PYGZdl{}\PYGZdl{}GLOG\PYGZdl{}\PYGZdl{}) Entry points
TSS ADDTO(VIRTDEP) VIRTNODE(\PYGZdl{}\PYGZdl{}LINE\PYGZdl{}\PYGZdl{}) Lines
TSS ADDTO(VIRTDEP) VIRTNODE(\PYGZdl{}\PYGZdl{}PCPC\PYGZdl{}\PYGZdl{}) Intelligent terminals
TSS ADDTO(VIRTDEP) VIRTNODE(\PYGZdl{}\PYGZdl{}RESO\PYGZdl{}\PYGZdl{}) Network management
TSS ADDTO(VIRTDEP) VIRTNODE(\PYGZdl{}\PYGZdl{}SECU\PYGZdl{}\PYGZdl{}) Virtel security
TSS ADDTO(VIRTDEP) VIRTNODE(\PYGZdl{}\PYGZdl{}SERV\PYGZdl{}\PYGZdl{}) External servers
TSS ADDTO(VIRTDEP) VIRTNODE(\PYGZdl{}\PYGZdl{}TERM\PYGZdl{}\PYGZdl{}) Terminals
TSS ADDTO(VIRTDEP) VIRTNODE(PC) Administration transactions
TSS ADDTO(VIRTDEP) VIRTNODE(PC\PYGZhy{}0020) Logoff transaction
TSS ADDTO(VIRTDEP) VIRTNODE(SERVEUR) Transaction SERVEUR
TSS ADDTO(VIRTDEP) VIRTNODE(W2H) Web Access transactions
TSS ADDTO(VIRTDEP) VIRTNODE(W2H\PYGZhy{}10) Web Access CICS transaction
TSS ADDTO(VIRTDEP) VIRTNODE(CLI) Client transactions
\end{sphinxVerbatim}

\sphinxAtStartPar
\sphinxstyleemphasis{TOPSDEF : TSS commands to define VIRTEL resources}

\sphinxAtStartPar
AE and SNCF are examples of external servers defined for VIRTEL Outgoing Calls (Videotex). VIRTEL permits access to an external server if the user is authorized to the corresponding resource name. Users who are authorized to the resource named \$\$ALLSRV may access all servers.

\sphinxAtStartPar
The resources named \$\$xxxx\$\$ are used to grant access to the various VIRTEL administrator functions. Refer to the VIRTEL Connectivity Reference manual for more details.

\sphinxAtStartPar
The resource named PC is a generic resource which permits access to the VIRTEL administrator 3270 interface transactions, whose internal name is PC\sphinxhyphen{}nnnn. The resource PC\sphinxhyphen{}0020 permits specific access to the 3270 Logoff transaction.

\sphinxAtStartPar
Resource W2H is a generic resource which permits access to VIRTEL Web Access transactions (internal name W2H\sphinxhyphen{}nn) and to the directory W2H\sphinxhyphen{}DIR. The resource W2H\sphinxhyphen{}10 permits specific access to the CICS Web Access transaction.

\sphinxAtStartPar
Resource CLI is a generic resource which permits access to customer\sphinxhyphen{}defined transactions (internal name CLI\sphinxhyphen{}nn) and to the directory CLI\sphinxhyphen{}DIR.


\subsubsection{Set Facility to resident}
\label{\detokenize{Installation_Guide:set-facility-to-resident}}
\sphinxAtStartPar
Set the facility to resident using the following command.

\begin{sphinxVerbatim}[commandchars=\\\{\}]
\PYG{n}{TSS} \PYG{n}{MODIFY} \PYG{n}{FAC}\PYG{p}{(}\PYG{n}{VIRTFAC}\PYG{o}{=}\PYG{n}{RES}\PYG{p}{)}
\end{sphinxVerbatim}


\subsubsection{Create administrator profile}
\label{\detokenize{Installation_Guide:create-administrator-profile}}
\begin{sphinxVerbatim}[commandchars=\\\{\}]
TSS CREATE(VIRTADP) NAME(\PYGZsq{}VIRTEL ADMINISTRATOR\PYGZsq{}) +
        TYPE(PROFILE) DEPARTMENT(VIRTDEP)
TSS ADDTO(VIRTADP) FACILITY(VIRTFAC)
TSS PERMIT(VIRTADP) VIRTAPPL(AE)
TSS PERMIT(VIRTADP) VIRTAPPL(SNCF)
TSS PERMIT(VIRTADP) VIRTAPPL(\PYGZdl{}\PYGZdl{}ALLSRV)
TSS PERMIT(VIRTADP) VIRTNODE(\PYGZdl{}\PYGZdl{}ARBO\PYGZdl{}\PYGZdl{})
TSS PERMIT(VIRTADP) VIRTNODE(\PYGZdl{}\PYGZdl{}UTIL\PYGZdl{}\PYGZdl{})
TSS PERMIT(VIRTADP) VIRTNODE(\PYGZdl{}\PYGZdl{}APPL\PYGZdl{}\PYGZdl{})
TSS PERMIT(VIRTADP) VIRTNODE(\PYGZdl{}\PYGZdl{}CMP3\PYGZdl{}\PYGZdl{})
TSS PERMIT(VIRTADP) VIRTNODE(\PYGZdl{}\PYGZdl{}GLOG\PYGZdl{}\PYGZdl{})
TSS PERMIT(VIRTADP) VIRTNODE(\PYGZdl{}\PYGZdl{}LINE\PYGZdl{}\PYGZdl{})
TSS PERMIT(VIRTADP) VIRTNODE(\PYGZdl{}\PYGZdl{}PCPC\PYGZdl{}\PYGZdl{})
TSS PERMIT(VIRTADP) VIRTNODE(\PYGZdl{}\PYGZdl{}RESO\PYGZdl{}\PYGZdl{})
TSS PERMIT(VIRTADP) VIRTNODE(\PYGZdl{}\PYGZdl{}SECU\PYGZdl{}\PYGZdl{})
TSS PERMIT(VIRTADP) VIRTNODE(\PYGZdl{}\PYGZdl{}SERV\PYGZdl{}\PYGZdl{})
TSS PERMIT(VIRTADP) VIRTNODE(\PYGZdl{}\PYGZdl{}TERM\PYGZdl{}\PYGZdl{})
TSS PERMIT(VIRTADP) VIRTNODE(PC(G))
TSS PERMIT(VIRTADP) VIRTNODE(SERVEUR)
TSS PERMIT(VIRTADP) VIRTNODE(W2H(G))
TSS PERMIT(VIRTADP) VIRTNODE(CLI(G))
\end{sphinxVerbatim}

\sphinxAtStartPar
\sphinxstyleemphasis{TOPSDEF : TSS commands to create VIRTEL administrator profile}

\sphinxAtStartPar
The VIRTEL administrator profile is named VIRTADP. You may choose a different name if required. In this example the administrator is granted access to all of the VIRTEL administration functions as well as to transactions PC\sphinxhyphen{}nnnn, W2H\sphinxhyphen{} nn and CLI\sphinxhyphen{}nn, and to directories W2H\sphinxhyphen{}DIR and CLI\sphinxhyphen{}DIR.

\begin{sphinxadmonition}{note}{Note:}
\sphinxAtStartPar
If you are implementing the Virtel USERPARM feature you should also add the following profiles:

\begin{sphinxVerbatim}[commandchars=\\\{\}]
\PYG{n}{TSS} \PYG{n}{ADDTO}\PYG{p}{(}\PYG{n}{VIRTDEP}\PYG{p}{)}  \PYG{n}{VIRTNODE}\PYG{p}{(}\PYG{n}{USERPARM}\PYG{p}{)}
\PYG{n}{TSS} \PYG{n}{PERMIT}\PYG{p}{(}\PYG{n}{VIRTADP}\PYG{p}{)} \PYG{n}{VIRTNODE}\PYG{p}{(}\PYG{n}{USERPARM}\PYG{p}{)}
\PYG{n}{TSS} \PYG{n}{PERMIT}\PYG{p}{(}\PYG{n}{VIRTUSP}\PYG{p}{)} \PYG{n}{VIRTNODE}\PYG{p}{(}\PYG{n}{USERPARM}\PYG{p}{)}    \PYG{o}{\PYGZlt{}}\PYG{o}{\PYGZhy{}} \PYG{n+nb}{all} \PYG{n}{Virtel} \PYG{n}{users} \PYG{n}{should} \PYG{n}{have} \PYG{n}{this} \PYG{n}{rule}\PYG{o}{.}
\end{sphinxVerbatim}
\end{sphinxadmonition}


\subsubsection{Create user profile}
\label{\detokenize{Installation_Guide:create-user-profile}}
\begin{sphinxVerbatim}[commandchars=\\\{\}]
\PYG{n}{TSS} \PYG{n}{CREATE}\PYG{p}{(}\PYG{n}{VIRTUSP}\PYG{p}{)} \PYG{n}{NAME}\PYG{p}{(}\PYG{l+s+s1}{\PYGZsq{}}\PYG{l+s+s1}{VIRTEL USER}\PYG{l+s+s1}{\PYGZsq{}}\PYG{p}{)} \PYG{o}{+}
        \PYG{n}{TYPE}\PYG{p}{(}\PYG{n}{PROFILE}\PYG{p}{)} \PYG{n}{DEPARTMENT}\PYG{p}{(}\PYG{n}{VIRTDEP}\PYG{p}{)}
\PYG{n}{TSS} \PYG{n}{ADDTO}\PYG{p}{(}\PYG{n}{VIRTUSP}\PYG{p}{)} \PYG{n}{FACILITY}\PYG{p}{(}\PYG{n}{VIRTFAC}\PYG{p}{)}
\PYG{n}{TSS} \PYG{n}{PERMIT}\PYG{p}{(}\PYG{n}{VIRTUSP}\PYG{p}{)} \PYG{n}{VIRTAPPL}\PYG{p}{(}\PYG{n}{AE}\PYG{p}{)}
\PYG{n}{TSS} \PYG{n}{PERMIT}\PYG{p}{(}\PYG{n}{VIRTUSP}\PYG{p}{)} \PYG{n}{VIRTAPPL}\PYG{p}{(}\PYG{n}{SNCF}\PYG{p}{)}
\PYG{n}{TSS} \PYG{n}{PERMIT}\PYG{p}{(}\PYG{n}{VIRTUSP}\PYG{p}{)} \PYG{n}{VIRTNODE}\PYG{p}{(}\PYG{n}{W2H}\PYG{o}{\PYGZhy{}}\PYG{l+m+mi}{10}\PYG{p}{)}
\end{sphinxVerbatim}

\sphinxAtStartPar
\sphinxstyleemphasis{TOPSDEF : TSS commands to create VIRTEL user profile}

\sphinxAtStartPar
The VIRTEL general user profile is named VIRTUSP. You may choose a different name if required. In this example the general user is granted access to external servers AE and SNCF, as well as to transaction W2H\sphinxhyphen{}10.


\subsubsection{Allow everyone to use the 3270 LOGOFF transaction}
\label{\detokenize{Installation_Guide:allow-everyone-to-use-the-3270-logoff-transaction}}
\begin{sphinxVerbatim}[commandchars=\\\{\}]
\PYG{n}{TSS} \PYG{n}{PERMIT}\PYG{p}{(}\PYG{n}{ALL}\PYG{p}{)} \PYG{n}{VIRTNODE}\PYG{p}{(}\PYG{n}{PC}\PYG{o}{\PYGZhy{}}\PYG{l+m+mi}{0020}\PYG{p}{)}
\end{sphinxVerbatim}

\sphinxAtStartPar
\sphinxstyleemphasis{TOPSDEF : TSS command to permit access to 3270 Logoff transaction}

\sphinxAtStartPar
This command permits all users to use the 3270 Logoff transaction, whose internal name is PC\sphinxhyphen{}0020.


\subsubsection{Define VIRTEL general users}
\label{\detokenize{Installation_Guide:define-virtel-general-users}}
\begin{sphinxVerbatim}[commandchars=\\\{\}]
\PYG{n}{TSS} \PYG{n}{ADDTO}\PYG{p}{(}\PYG{n}{userid1}\PYG{p}{)} \PYG{n}{PROFILE}\PYG{p}{(}\PYG{n}{VIRTUSP}\PYG{p}{)}
\PYG{n}{TSS} \PYG{n}{ADDTO}\PYG{p}{(}\PYG{n}{userid2}\PYG{p}{)} \PYG{n}{PROFILE}\PYG{p}{(}\PYG{n}{VIRTUSP}\PYG{p}{)}
\end{sphinxVerbatim}

\sphinxAtStartPar
\sphinxstyleemphasis{TOPSDEF : TSS command to add general users}

\sphinxAtStartPar
These commands define userid1 and userid2 as VIRTEL general users by adding the VIRTEL user profile to their ACID.

\sphinxAtStartPar
8.3.2.12. Define VIRTEL administrators

\begin{sphinxVerbatim}[commandchars=\\\{\}]
\PYG{n}{TSS} \PYG{n}{ADDTO}\PYG{p}{(}\PYG{n}{admin1}\PYG{p}{)} \PYG{n}{PROFILE}\PYG{p}{(}\PYG{n}{VIRTADP}\PYG{p}{)}
\PYG{n}{TSS} \PYG{n}{ADDTO}\PYG{p}{(}\PYG{n}{admin2}\PYG{p}{)} \PYG{n}{PROFILE}\PYG{p}{(}\PYG{n}{VIRTADP}\PYG{p}{)}
\end{sphinxVerbatim}

\sphinxAtStartPar
\sphinxstyleemphasis{TOPSDEF : TSS command to add administrators}

\sphinxAtStartPar
These commands define admin1 and admin2 as VIRTEL administrators by adding the VIRTEL administrator profile to their ACID.

\index{Virtel@\spxentry{Virtel}!Library Authorization.@\spxentry{Library Authorization.}}\index{Library Authorization.@\spxentry{Library Authorization.}!Virtel@\spxentry{Virtel}}\ignorespaces 

\subsubsection{Authorize the VIRTEL LOADLIB}
\label{\detokenize{Installation_Guide:authorize-the-virtel-loadlib}}\label{\detokenize{Installation_Guide:index-188}}
\sphinxAtStartPar
The VIRTEL load library should normally be APF\sphinxhyphen{}authorized. If this is not the case, NOAUTH should be specified in the VIRTFAC facility.


\chapter{Appendix}
\label{\detokenize{Installation_Guide:appendix}}
\index{Virtel Modules@\spxentry{Virtel Modules}}\ignorespaces 

\section{Appendix A. Virtel Modules}
\label{\detokenize{Installation_Guide:appendix-a-virtel-modules}}\label{\detokenize{Installation_Guide:index-189}}

\subsection{Virtel Modules}
\label{\detokenize{Installation_Guide:virtel-modules}}
\sphinxAtStartPar
The functionality of VIRTEL is divided into components known as “modules”. The following is a list of the VIRTEL modules:

\sphinxAtStartPar
\sphinxstylestrong{Kernel modules}

\begin{sphinxVerbatim}[commandchars=\\\{\}]
\PYG{n}{VIR0000} \PYG{n}{System} \PYG{n}{initialisation}
\PYG{n}{VIR0001} \PYG{n}{VSAM} \PYG{n}{access} \PYG{n}{routines}
\PYG{n}{VIR0002} \PYG{n}{Console} \PYG{n}{command} \PYG{n}{processing}
\PYG{n}{VIR0004} \PYG{n}{SNAP} \PYG{n}{trace}\PYG{o}{/}\PYG{n}{dump} \PYG{n}{formatter}
\PYG{n}{VIR0005} \PYG{n}{Terminal} \PYG{o+ow}{and} \PYG{n}{line} \PYG{n}{management}\PYG{p}{,} \PYG{n}{also} \PYG{n}{X25} \PYG{n}{call} \PYG{n}{packet} \PYG{n}{management}
\PYG{n}{VIR0006} \PYG{n}{Statistics} \PYG{n}{I}\PYG{o}{/}\PYG{n}{O} \PYG{n}{subtask}
\PYG{n}{VIR0007} \PYG{n}{Abend} \PYG{n}{recovery} \PYG{n}{routine}
\PYG{n}{VIR00080} \PYG{n}{Security} \PYG{n}{functions} \PYG{k}{for} \PYG{n}{no} \PYG{n}{security}
\PYG{n}{VIR00081} \PYG{n}{Security} \PYG{n}{functions} \PYG{k}{for} \PYG{n}{Virtel} \PYG{n}{security}
\PYG{n}{VIR00082} \PYG{n}{Security} \PYG{n}{functions} \PYG{k}{for} \PYG{n}{TOP}\PYG{o}{\PYGZhy{}}\PYG{n}{SECRET} \PYG{n}{without} \PYG{n}{RACROUTE}
\PYG{n}{VIR00083} \PYG{n}{Security} \PYG{n}{functions} \PYG{k}{for} \PYG{n}{RACF} \PYG{n}{without} \PYG{n}{RACROUTE}
\PYG{n}{VIR00084} \PYG{n}{Security} \PYG{n}{functions} \PYG{k}{for} \PYG{n}{TOP}\PYG{o}{\PYGZhy{}}\PYG{n}{SECRET} \PYG{k}{with} \PYG{n}{RACROUTE}
\PYG{n}{VIR00085} \PYG{n}{Security} \PYG{n}{functions} \PYG{k}{for} \PYG{n}{ACF2} \PYG{k}{with} \PYG{n}{ACFDIAG} \PYG{p}{(}\PYG{k}{for} \PYG{n}{VM}\PYG{p}{)}
\PYG{n}{VIR00086} \PYG{n}{Security} \PYG{n}{functions} \PYG{k}{for} \PYG{n}{ACF2} \PYG{o+ow}{or} \PYG{n}{RACF} \PYG{k}{with} \PYG{n}{RACROUTE}
\PYG{n}{VIR0009} \PYG{n}{VTAM} \PYG{n}{interface} \PYG{n}{module}
\end{sphinxVerbatim}

\sphinxAtStartPar
\sphinxstylestrong{Communication modules}

\begin{sphinxVerbatim}[commandchars=\\\{\}]
\PYG{n}{VIR0C12}         \PYG{n}{Web}\PYG{o}{\PYGZhy{}}\PYG{n}{to}\PYG{o}{\PYGZhy{}}\PYG{n}{Host} \PYG{n}{Interface} \PYG{n}{CGI}
\PYG{n}{VIR0U12}         \PYG{n}{Web}\PYG{o}{\PYGZhy{}}\PYG{n}{to}\PYG{o}{\PYGZhy{}}\PYG{n}{Host} \PYG{n}{utility} \PYG{n}{functions}
\PYG{n}{VIR0V12}         \PYG{n}{Web}\PYG{o}{\PYGZhy{}}\PYG{n}{to}\PYG{o}{\PYGZhy{}}\PYG{n}{Host} \PYG{n}{utility} \PYG{n}{functions}
\PYG{n}{VIR0X12}         \PYG{n}{X25} \PYG{n}{by} \PYG{n}{structured} \PYG{n}{field}
\PYG{n}{VIR0011A}        \PYG{l+m+mi}{3270} \PYG{n}{multisession} \PYG{n}{processing}
\PYG{n}{VIR0011B}        \PYG{l+m+mi}{3270} \PYG{n}{Compr2} \PYG{n}{Minitel} \PYG{l+m+mi}{40} \PYG{n}{colonnes}
\PYG{n}{VIR0011C}        \PYG{l+m+mi}{3270} \PYG{n}{Compr2} \PYG{n}{Minitel} \PYG{l+m+mi}{80} \PYG{n}{colonnes}
\PYG{n}{VIR0011D}        \PYG{n}{Web}\PYG{o}{\PYGZhy{}}\PYG{n}{to}\PYG{o}{\PYGZhy{}}\PYG{n}{Host} \PYG{l+m+mi}{3270} \PYG{n}{scenario} \PYG{n}{processing}
\PYG{n}{VIR0012}         \PYG{n}{Minitel} \PYG{l+m+mi}{3270} \PYG{n}{emulation}
\PYG{n}{VIR0013}         \PYG{l+m+mi}{3270} \PYG{n}{receive} \PYG{n}{processing}
\PYG{n}{VIR0014}         \PYG{n}{Minitel} \PYG{n}{Support}
\PYG{n}{VIR0014A}        \PYG{n}{LU1} \PYG{n}{Term} \PYG{n}{Support}
\PYG{n}{VIR0015}         \PYG{n}{Multi}\PYG{o}{\PYGZhy{}}\PYG{n}{session} \PYG{l+m+mi}{3270} \PYG{o}{+} \PYG{n}{File} \PYG{n}{Transfer}
\PYG{n}{VIR0016}         \PYG{n}{Terminal} \PYG{l+m+mi}{24}\PYG{n}{x80} \PYG{n}{Support}
\PYG{n}{VIR0018}         \PYG{n}{VT100} \PYG{n}{Support}
\PYG{n}{VIR0019}         \PYG{n}{Inverse} \PYG{l+m+mi}{3270} \PYG{n}{emulation}
\PYG{n}{VIR0034}         \PYG{n}{CEPT1} \PYG{n}{Protocol} \PYG{n}{Support}
\PYG{n}{VIR0035}         \PYG{n}{VIRTEL}\PYG{o}{/}\PYG{n}{PC}
\PYG{n}{VIR0039}         \PYG{n}{Interface} \PYG{n}{LECAM}
\end{sphinxVerbatim}

\sphinxAtStartPar
\sphinxstylestrong{Transaction modules}

\begin{sphinxVerbatim}[commandchars=\\\{\}]
\PYG{n}{VIR0020}         \PYG{n}{Multi}\PYG{o}{\PYGZhy{}}\PYG{n}{session} \PYG{n}{signon} \PYG{p}{(}\PYG{n}{pre}\PYG{o}{\PYGZhy{}}\PYG{n}{transaction} \PYG{n}{version}\PYG{p}{)}
\PYG{n}{VIR0020A}        \PYG{n}{Multi}\PYG{o}{\PYGZhy{}}\PYG{n}{session} \PYG{n}{signon}
\PYG{n}{VIR0020B}        \PYG{n}{Multi}\PYG{o}{\PYGZhy{}}\PYG{n}{session} \PYG{n}{signon} \PYG{k}{with} \PYG{n}{userid}\PYG{o}{+}\PYG{n}{password} \PYG{o+ow}{in} \PYG{n}{logon} \PYG{n}{data}
\PYG{n}{VIR0020C}        \PYG{n}{Multi}\PYG{o}{\PYGZhy{}}\PYG{n}{session} \PYG{n}{signon} \PYG{k}{with} \PYG{n}{userid} \PYG{o+ow}{in} \PYG{n}{logon} \PYG{n}{data}
\PYG{n}{VIR0020H}        \PYG{n}{Basic} \PYG{n}{authentication} \PYG{k}{for} \PYG{n}{HTTP}
\PYG{n}{VIR0020L}        \PYG{n}{Multi}\PYG{o}{\PYGZhy{}}\PYG{n}{session} \PYG{n}{signon} \PYG{p}{(}\PYG{n}{Minitel} \PYG{l+m+mi}{40} \PYG{n}{columns}\PYG{p}{)}
\PYG{n}{VIR0020M}        \PYG{n}{Multi}\PYG{o}{\PYGZhy{}}\PYG{n}{session} \PYG{n}{signon} \PYG{p}{(}\PYG{n}{Minitel} \PYG{l+m+mi}{40} \PYG{n}{columns}\PYG{p}{)}
\PYG{n}{VIR0020P}        \PYG{n}{Multi}\PYG{o}{\PYGZhy{}}\PYG{n}{session} \PYG{n}{signon} \PYG{p}{(}\PYG{n}{Minitel} \PYG{l+m+mi}{40} \PYG{n}{columns}\PYG{p}{)}
\PYG{n}{VIR0021}         \PYG{n}{Multi}\PYG{o}{\PYGZhy{}}\PYG{n}{session} \PYG{n}{menu}
\PYG{n}{VIR0022}         \PYG{n}{Virtel} \PYG{n}{administration}\PYG{p}{:} \PYG{n}{main} \PYG{n}{menu}
\PYG{n}{VIR0023}         \PYG{n}{Virtel} \PYG{n}{administration}\PYG{p}{:} \PYG{n}{terminals}
\PYG{n}{VIR0025S}        \PYG{n}{List} \PYG{n}{of} \PYG{n}{external} \PYG{n}{servers} \PYG{o+ow}{in} \PYG{n}{structured} \PYG{n}{field}
\PYG{n}{VIR0025T}        \PYG{n}{Call} \PYG{n}{named} \PYG{n}{external} \PYG{n}{server}
\PYG{n}{VIR0027}         \PYG{n}{Virtel} \PYG{n}{administration}\PYG{p}{:} \PYG{n}{CVC} \PYG{n}{status} \PYG{n}{display}
\PYG{n}{VIR0029}         \PYG{l+m+mi}{3174} \PYG{n}{emulation} \PYG{p}{(}\PYG{n}{LLC3} \PYG{n}{client}\PYG{p}{)}
\PYG{n}{VIR0031}         \PYG{n}{Virtel} \PYG{n}{administration}\PYG{p}{:} \PYG{n}{external} \PYG{n}{servers}
\PYG{n}{VIR0041B}        \PYG{n}{Page} \PYG{n}{upload} \PYG{n}{by} \PYG{n}{SMTP}
\PYG{n}{VIR0041C}        \PYG{n}{Page} \PYG{n}{upload} \PYG{n}{by} \PYG{n}{HTTP}
\PYG{n}{VIR0042}         \PYG{n}{Virtel} \PYG{n}{administration}\PYG{p}{:} \PYG{n}{directories}
\PYG{n}{VIR0043}         \PYG{n}{Virtel} \PYG{n}{administration}\PYG{p}{:} \PYG{n}{directory} \PYG{n}{contents}
\PYG{n}{VIR0044}         \PYG{n}{Virtel} \PYG{n}{administration}\PYG{p}{:} \PYG{n}{entry} \PYG{n}{points}
\PYG{n}{VIR0045}         \PYG{n}{Virtel} \PYG{n}{administration}\PYG{p}{:} \PYG{n}{transactions}
\PYG{n}{VIR0046}         \PYG{n}{Virtel} \PYG{n}{administration}\PYG{p}{:} \PYG{n}{lines}
\PYG{n}{VIR0047}         \PYG{n}{Virtel} \PYG{n}{administration}\PYG{p}{:} \PYG{n}{rules}
\PYG{n}{VIR0048}         \PYG{n}{Virtel} \PYG{n}{administration}\PYG{p}{:} \PYG{n}{line} \PYG{n}{summary} \PYG{n}{display}
\PYG{n}{VIR0049}         \PYG{n}{Virtel} \PYG{n}{administration}\PYG{p}{:} \PYG{n}{line} \PYG{n}{summary} \PYG{n}{expanded} \PYG{n}{display}
\end{sphinxVerbatim}

\sphinxAtStartPar
\sphinxstylestrong{Utility function modules}

\begin{sphinxVerbatim}[commandchars=\\\{\}]
\PYG{n}{VIR0017}         \PYG{n}{Utility} \PYG{n}{functions}
\PYG{n}{VIR0B17}         \PYG{n}{Utility} \PYG{n}{functions}
\PYG{n}{VIR0C17}         \PYG{n}{Utility} \PYG{n}{functions}
\end{sphinxVerbatim}

\sphinxAtStartPar
\sphinxstylestrong{Line interface modules}

\begin{sphinxVerbatim}[commandchars=\\\{\}]
\PYG{n}{VIR0062}         \PYG{n}{APPC} \PYG{n}{LU6}\PYG{l+m+mf}{.2} \PYG{n}{interface} \PYG{n}{module}
\PYG{n}{VIR0I09}         \PYG{n}{VTAM} \PYG{n}{interface} \PYG{n}{module} \PYG{p}{(}\PYG{n}{bis}\PYG{p}{)}
\PYG{n}{VIR0T09}         \PYG{n}{TCP}\PYG{o}{/}\PYG{n}{IP} \PYG{n}{interface} \PYG{n}{module} \PYG{p}{(}\PYG{n}{HPNS} \PYG{n}{EXIT} \PYG{n}{mode}\PYG{p}{)}
\PYG{n}{VIR0T10}         \PYG{n}{TCP}\PYG{o}{/}\PYG{n}{IP} \PYG{n}{interface} \PYG{n}{module} \PYG{p}{(}\PYG{n}{HPNS} \PYG{n}{ECB} \PYG{n}{mode}\PYG{p}{)}
\end{sphinxVerbatim}

\sphinxAtStartPar
\sphinxstylestrong{Terminal      interface modules}

\begin{sphinxVerbatim}[commandchars=\\\{\}]
\PYG{n}{VIR0I19}         \PYG{n}{Interface} \PYG{n}{Pseudo} \PYG{p}{(}\PYG{n}{AntiPCNE}\PYG{o}{/}\PYG{n}{AntiGATE}\PYG{o}{/}\PYG{n}{AntiFASTC}\PYG{p}{)}
\PYG{n}{VIR0T19}         \PYG{n}{Interface} \PYG{n}{Pseudo} \PYG{p}{(}\PYG{n}{TCP}\PYG{o}{/}\PYG{n}{IP}\PYG{p}{)}
\end{sphinxVerbatim}

\newpage

\sphinxAtStartPar
\sphinxstylestrong{Protocol modules}

\begin{sphinxVerbatim}[commandchars=\\\{\}]
\PYG{n}{VIRHTTP}         \PYG{n}{HTTP} \PYG{n}{protocol} \PYG{n}{module}
\PYG{n}{VIRSMTP}         \PYG{n}{SMTP} \PYG{n}{protocol} \PYG{n}{module}
\PYG{n}{VIRXOT}          \PYG{n}{XOT} \PYG{n}{protocol} \PYG{n}{module}
\PYG{n}{VIRXTP}          \PYG{n}{XTP} \PYG{n}{protocol} \PYG{n}{module}
\PYG{n}{VIR0PASS}        \PYG{n}{VIRPASS} \PYG{n}{protocol} \PYG{n}{module}
\PYG{n}{VIR00IE}         \PYG{n}{AntiPCNE} \PYG{n}{protocol} \PYG{n}{module}
\PYG{n}{VIR00IF}         \PYG{n}{AntiFASTC} \PYG{n}{protocol} \PYG{n}{module}
\PYG{n}{VIR00IG}         \PYG{n}{AntiGATE} \PYG{n}{protocol} \PYG{n}{module}
\PYG{n}{VIR0715}         \PYG{n}{APPC1} \PYG{n}{protocol} \PYG{n}{module}
\PYG{n}{VIR0815}         \PYG{n}{APPC2} \PYG{n}{protocol} \PYG{n}{module}
\PYG{n}{VIR0S15}         \PYG{n}{Structured} \PYG{n}{fields} \PYG{n}{protocol} \PYG{n}{module}
\end{sphinxVerbatim}

\sphinxAtStartPar
The VIRTEL product contains support for the base kernel and all modules. The functionality of each module is activated either by setting specific parameters in the VIRTCT or by the activation of appropriate configuration definitions in the VIRARBO file.

\begin{DUlineblock}{0em}
\item[] \sphinxstyleemphasis{Please refer to your license agreement for the particular terms and conditions under which you are authorised to use the various VIRTEL modules.}
\end{DUlineblock}

\newpage


\section{Appendix B. VSE ICCF Editor commands}
\label{\detokenize{Installation_Guide:appendix-b-vse-iccf-editor-commands}}
\sphinxAtStartPar
\sphinxstyleemphasis{PFKs}

\begin{sphinxVerbatim}[commandchars=\\\{\}]
\PYG{n}{F6}                      \PYG{n}{Find}
\PYG{n}{F7}                      \PYG{n}{Scroll} \PYG{n}{back}
\PYG{n}{F8}                      \PYG{n}{Scroll} \PYG{n}{forward}
\PYG{n}{F9}                      \PYG{n}{Top} \PYG{n}{of} \PYG{n}{file}
\PYG{n}{F10}                     \PYG{n}{Scroll} \PYG{n}{left}
\PYG{n}{F11}                     \PYG{n}{Scroll} \PYG{n}{right}
\PYG{n}{F12}                     \PYG{n}{End} \PYG{n}{of} \PYG{n}{file}
\end{sphinxVerbatim}

\sphinxAtStartPar
\sphinxstyleemphasis{Locate commands}

\begin{sphinxVerbatim}[commandchars=\\\{\}]
\PYG{n}{L} \PYG{n}{string}        \PYG{n}{Find} \PYG{n+nb}{next}
\PYG{n}{LU} \PYG{n}{string}       \PYG{n}{Find} \PYG{n}{previous}
\end{sphinxVerbatim}

\sphinxAtStartPar
\sphinxstyleemphasis{Global Change}

\begin{sphinxVerbatim}[commandchars=\\\{\}]
\PYG{n}{C}\PYG{o}{/}\PYG{n}{oldstring}\PYG{o}{/}\PYG{n}{newstring}\PYG{o}{/}\PYG{o}{*} \PYG{n}{G}       \PYG{n}{Global} \PYG{n}{change}
\end{sphinxVerbatim}

\sphinxAtStartPar
\sphinxstyleemphasis{Cursor positioning}

\begin{sphinxVerbatim}[commandchars=\\\{\}]
\PYG{n}{N} \PYG{n}{n}                     \PYG{n}{Scroll} \PYG{n}{forward} \PYG{n}{n} \PYG{n}{lines}
\PYG{n}{U} \PYG{n}{n}                     \PYG{n}{Scroll} \PYG{n}{back} \PYG{n}{n} \PYG{n}{lines}
\end{sphinxVerbatim}

\sphinxAtStartPar
\sphinxstyleemphasis{Command Recall}

\sphinxAtStartPar
Prefix a command with \& to make it stay in the entry area.

\sphinxAtStartPar
\sphinxstyleemphasis{Line commands}

\begin{sphinxVerbatim}[commandchars=\\\{\}]
\PYG{n}{An}                      \PYG{n}{Insert} \PYG{n}{n} \PYG{n}{blank} \PYG{n}{lines}
\PYG{n}{Dn}                      \PYG{n}{Delete} \PYG{n}{n} \PYG{n}{lines}
\PYG{n}{Cn}                      \PYG{n}{Copy} \PYG{n}{n} \PYG{n}{lines} \PYG{n}{to} \PYG{n}{scratchpad}
\PYG{n}{Mn}                      \PYG{n}{Move} \PYG{n}{n} \PYG{n}{lines} \PYG{n}{to} \PYG{n}{scratchpad}
\PYG{n}{I}                       \PYG{n}{Insert} \PYG{n}{scratchpad} \PYG{n}{after} \PYG{n}{this} \PYG{n}{line}
\PYG{l+s+s2}{\PYGZdq{}}\PYG{l+s+s2}{n                      Duplicate this line n times}
\end{sphinxVerbatim}


\section{Trademarks}
\label{\detokenize{Installation_Guide:trademarks}}
\sphinxAtStartPar
SysperTec, the SysperTec logo, syspertec.com and VIRTEL are trademarks or registered trademarks of SysperTec
Communication Group, registered in France and other countries.

\sphinxAtStartPar
IBM, VTAM, CICS, IMS, RACF, DB2, z/OS, WebSphere, MQSeries, System z are trademarks or registered trademarks of
International Business Machines Corp., registered in United States and other countries.

\sphinxAtStartPar
Adobe, Acrobat, PostScript and all Adobe\sphinxhyphen{}based trademarks are either registered trademarks or trademarks of Adobe
Systems Incorporated in the United States and other countries.

\sphinxAtStartPar
Microsoft, Windows, Windows NT, and the Windows logo are trademarks of Microsoft Corporation in the United States
and other countries.

\sphinxAtStartPar
UNIX is a registered trademark of The Open Group in the United States and other countries.
Java and all Java\sphinxhyphen{}based trademarks and logos are trademarks or registered trademarks of Oracle and/or its affiliates.

\sphinxAtStartPar
Linux is a trademark of Linus Torvalds in the United States, other countries, or both.

\sphinxAtStartPar
Other company, product, or service names may be trademarks or service names of others.



\renewcommand{\indexname}{Index}
\printindex
\end{document}