%% Generated by Sphinx.
\def\sphinxdocclass{report}
\documentclass[letterpaper,10pt,english]{sphinxmanual}
\ifdefined\pdfpxdimen
   \let\sphinxpxdimen\pdfpxdimen\else\newdimen\sphinxpxdimen
\fi \sphinxpxdimen=.75bp\relax
\ifdefined\pdfimageresolution
    \pdfimageresolution= \numexpr \dimexpr1in\relax/\sphinxpxdimen\relax
\fi
%% let collapsible pdf bookmarks panel have high depth per default
\PassOptionsToPackage{bookmarksdepth=5}{hyperref}

\PassOptionsToPackage{booktabs}{sphinx}
\PassOptionsToPackage{colorrows}{sphinx}

\PassOptionsToPackage{warn}{textcomp}


\usepackage{cmap}

\usepackage{amsmath,amssymb,amstext}
\usepackage{babel}





\usepackage[Bjarne]{fncychap}
\usepackage[,numfigreset=1,mathnumfig]{sphinx}

\fvset{fontsize=auto}
\usepackage{geometry}


% Include hyperref last.
\usepackage{hyperref}
% Fix anchor placement for figures with captions.
\usepackage{hypcap}% it must be loaded after hyperref.
% Set up styles of URL: it should be placed after hyperref.
\urlstyle{same}

\addto\captionsenglish{\renewcommand{\contentsname}{Table of Contents:}}

\usepackage{sphinxmessages}
\setcounter{tocdepth}{2}

\usepackage{fontspec} \setmonofont{Courier New}

\title{Virtel Getting Started}
\date{Jun 23, 2024}
\release{4.62}
\author{Syspertec Communication}
\newcommand{\sphinxlogo}{\vbox{}}
\renewcommand{\releasename}{Release}
\makeindex
\begin{document}

\ifdefined\shorthandoff
  \ifnum\catcode`\=\string=\active\shorthandoff{=}\fi
  \ifnum\catcode`\"=\active\shorthandoff{"}\fi
\fi

\pagestyle{empty}
\sphinxmaketitle
\pagestyle{plain}
\sphinxtableofcontents
\pagestyle{normal}
\phantomsection\label{\detokenize{Getting_Started::doc}}


\sphinxAtStartPar
\sphinxincludegraphics[scale=0.5]{{logo_virtel_web}.png}

\sphinxAtStartPar
Version : 4.62

\sphinxAtStartPar
Release Date : 23/06/2024. Publication Date : 17/02/2024

\sphinxAtStartPar
Virtel SAS

\sphinxAtStartPar
302, Bureaux de la Colline 92213 Saint\sphinxhyphen{}Cloud Cedex Tél. : +33 (0) 1 46 02 60 42

\sphinxAtStartPar
\sphinxhref{https://www.virtelweb.com/}{www.virtelweb.com}

\sphinxAtStartPar
NOTICE
\begin{quote}

\sphinxAtStartPar
Reproduction, transfer, distribution, or storage, in any form, of all or any part of
the contents of this document, except by prior authorization of SysperTec
Communication, is prohibited.

\sphinxAtStartPar
Every possible effort has been made by SysperTec Group to ensure that this document
is complete and relevant. In no case can SysperTec Group be held responsible for
any damages, direct or indirect, caused by errors or omissions in this document.

\sphinxAtStartPar
As SysperTec Group uses a continuous development methodology; the information
contained in this document may be subject to change without notice. Nothing in this
document should be construed in any manner as conferring a right to use, in whole or in
part, the products or trademarks quoted herein.

\sphinxAtStartPar
“SysperTec Group” and “VIRTEL” are registered trademarks. Names of other products
and companies mentioned in this document may be trademarks or registered trademarks of
their respective owners.
\end{quote}


\chapter{What is Virtel?}
\label{\detokenize{Getting_Started:what-is-virtel}}
\sphinxAtStartPar
Virtel is a host\sphinxhyphen{}based protocol converter that runs as a started task on the mainframe. At the core of Virtel is the Virtel Engine which sits between host applications and external environments such as the web or another external server. Virtel supports the following standard protocols \sphinxhyphen{} TCP/IP, SMTP, HTTP/S, SOAP, MQ\sphinxhyphen{}SERIES, SNA, 3270, ICAL (IMS) and the inherited protocols \sphinxhyphen{} X25, XOT, XTP, LU 6.2 to interface between host applications and external services.

\sphinxAtStartPar
Virtel provides three models:
\begin{itemize}
\item {} 
\sphinxAtStartPar
Web Access (VWA)

\item {} 
\sphinxAtStartPar
Web Modernization (VWM)

\item {} 
\sphinxAtStartPar
Web Integration (VWI)

\end{itemize}

\newpage

\sphinxAtStartPar
\sphinxstylestrong{VIRTEL Web Access} is a set of functions which provides access to mainframe 3270 legacy applications via the user’s browser window. In the VWA model the Virtel Engine comprises of two components, a HTTP server and a VTAM component serving the back\sphinxhyphen{}end VTAM legacy applications, pertaining to operate as an LU2 device(s), as shown in the following diagram:

\sphinxAtStartPar
\sphinxincludegraphics{{VWA-architecture}.png}
\sphinxstyleemphasis{VIRTEL Web Access}

\newpage

\sphinxAtStartPar
\sphinxstylestrong{VIRTEL Web Modernization} allows the presentation of 3270 host applications to be modified, without modifying the application itself. The presentation can be adapted to a format (HTML, XML, etc.) suited to the requester, while hiding the details of navigation within the 3270 transactions.

\sphinxAtStartPar
The modernization process involves customizing the Virtel templates in order to change the way in which Virtel presents the host application data to the end\sphinxhyphen{}user. Virtel comes with two tools called \sphinxstyleemphasis{Virtel Screen Redesigner} and \sphinxstyleemphasis{Virtel Studio}, to assist developers with the modernization process.

\sphinxAtStartPar
The following diagram illustrates the VWM architecture:

\sphinxAtStartPar
\sphinxincludegraphics{{VWM-architecture}.png}
\sphinxstyleemphasis{VIRTEL Web Modernization}

\newpage

\sphinxAtStartPar
\sphinxstylestrong{VIRTEL Web Integration} allows a host application to take maximum control of its web interface, for example through web services.

\sphinxAtStartPar
VWI enables an application to create a dynamic dialog between its transactions and web applications through the creation of interactive bidirectional dialogs across the Internet between host (CICS, IMS, Ideal, Natural, etc.) and server\sphinxhyphen{}based applications using XML/HTML web services or other communication procedures.

\sphinxAtStartPar
Virtel provides a proprietary scripting language, that is used in \sphinxstyleemphasis{Virtel Scenarios} to implement these dynamic dialogs, as shown in the example below:

\sphinxAtStartPar
\sphinxincludegraphics{{VWI-architecture}.png}
\sphinxstyleemphasis{VIRTEL Web Integration}


\chapter{Release Notes}
\label{\detokenize{Getting_Started:release-notes}}

\section{Hardware and Software Requirements}
\label{\detokenize{Getting_Started:hardware-and-software-requirements}}
\sphinxAtStartPar
Virtel requires the following hardware and software:


\begin{savenotes}\sphinxattablestart
\sphinxthistablewithglobalstyle
\centering
\begin{tabular}[t]{*{3}{\X{1}{3}}}
\sphinxtoprule
\sphinxstyletheadfamily 
\sphinxAtStartPar
Component
&\sphinxstyletheadfamily 
\sphinxAtStartPar
Server side
&\sphinxstyletheadfamily 
\sphinxAtStartPar
Client side
\\
\sphinxmidrule
\sphinxtableatstartofbodyhook
\sphinxAtStartPar
Web Access

\sphinxAtStartPar
Web Modernization
&
\sphinxAtStartPar
Any supported IBM zSystem

\sphinxAtStartPar
Any supported version of z/OS
or VSEn
&
\sphinxAtStartPar
Any supported version of one of

\sphinxAtStartPar
the following browsers under any

\sphinxAtStartPar
operating system (JavaScript

\sphinxAtStartPar
enabled)
\begin{itemize}
\item {} 
\sphinxAtStartPar
Edge

\item {} 
\sphinxAtStartPar
Firefox

\item {} 
\sphinxAtStartPar
Chrome

\item {} 
\sphinxAtStartPar
Safari

\item {} 
\sphinxAtStartPar
Opera

\end{itemize}
\\
\sphinxhline
\sphinxAtStartPar
Virtel Screen
Redesigner
&&
\sphinxAtStartPar
Microsoft Windows
\\
\sphinxhline
\sphinxAtStartPar
Virtel Studio
&&
\sphinxAtStartPar
Microsoft Windows
Java 11
\\
\sphinxbottomrule
\end{tabular}
\sphinxtableafterendhook\par
\sphinxattableend\end{savenotes}

\sphinxAtStartPar
On the server side, the following elements are also required:
\begin{itemize}
\item {} 
\sphinxAtStartPar
Authorized Library: Virtel initialization program VIR6000 must run from an APF\sphinxhyphen{}authorized library.

\item {} 
\sphinxAtStartPar
High\sphinxhyphen{}Level Assembler: High\sphinxhyphen{}Level Assembler is required to assemble the VIRTCTxx configuration module and Virtel Scenarios

\item {} 
\sphinxAtStartPar
The REGION specified in the Virtel startup JCL depends on the number of terminals defined in the configuration, see the Installation Guide for more details.

\item {} 
\sphinxAtStartPar
Support for the cryptographic functions of VIRTEL requires ICSF Version HCR7740 or later.

\item {} 
\sphinxAtStartPar
TCP/IP server stack

\end{itemize}

\sphinxAtStartPar
\sphinxstylestrong{Space requirements:}


\begin{savenotes}\sphinxattablestart
\sphinxthistablewithglobalstyle
\centering
\begin{tabulary}{\linewidth}[t]{TTTT}
\sphinxtoprule
\sphinxstyletheadfamily 
\sphinxAtStartPar
Dataset
&\sphinxstyletheadfamily 
\sphinxAtStartPar
DSORG
&\sphinxstyletheadfamily 
\sphinxAtStartPar
CI/BLKSIZE
&\sphinxstyletheadfamily 
\sphinxAtStartPar
Size (TRKS)
\\
\sphinxmidrule
\sphinxtableatstartofbodyhook
\sphinxAtStartPar
ARBO
&
\sphinxAtStartPar
VS
&
\sphinxAtStartPar
4096
&
\sphinxAtStartPar
IX=1  DATA=1
\\
\sphinxhline
\sphinxAtStartPar
CNTL
&
\sphinxAtStartPar
PO
&
\sphinxAtStartPar
3120
&
\sphinxAtStartPar
75
\\
\sphinxhline
\sphinxAtStartPar
HTML
&
\sphinxAtStartPar
VS
&
\sphinxAtStartPar
32768
&
\sphinxAtStartPar
IX=1  DATA=3
\\
\sphinxhline
\sphinxAtStartPar
HTML.TRSF
&
\sphinxAtStartPar
VS
&
\sphinxAtStartPar
32768
&
\sphinxAtStartPar
IX=5  DATA=75
\\
\sphinxhline
\sphinxAtStartPar
LOADLIB
&
\sphinxAtStartPar
PO
&
\sphinxAtStartPar
32760
&
\sphinxAtStartPar
150
\\
\sphinxhline
\sphinxAtStartPar
MACLIB
&
\sphinxAtStartPar
PO
&
\sphinxAtStartPar
3120
&
\sphinxAtStartPar
30
\\
\sphinxhline
\sphinxAtStartPar
SAMP.TRSF
&
\sphinxAtStartPar
VS
&
\sphinxAtStartPar
32768
&
\sphinxAtStartPar
IX=5  DATA=275
\\
\sphinxhline
\sphinxAtStartPar
SAMPLIB
&
\sphinxAtStartPar
PO
&
\sphinxAtStartPar
3120
&
\sphinxAtStartPar
150
\\
\sphinxhline
\sphinxAtStartPar
SCRNAPI.MACLIB
&
\sphinxAtStartPar
PO
&
\sphinxAtStartPar
3120
&
\sphinxAtStartPar
20
\\
\sphinxhline
\sphinxAtStartPar
SERVLIB
&
\sphinxAtStartPar
PO\sphinxhyphen{}E
&
\sphinxAtStartPar
4096
&
\sphinxAtStartPar
45
\\
\sphinxhline
\sphinxAtStartPar
STAT
&
\sphinxAtStartPar
PS
&
\sphinxAtStartPar
12400
&
\sphinxAtStartPar
15
\\
\sphinxhline
\sphinxAtStartPar
SWAP
&
\sphinxAtStartPar
VS
&
\sphinxAtStartPar
8192
&
\sphinxAtStartPar
IX=1  DATA=1
\\
\sphinxbottomrule
\end{tabulary}
\sphinxtableafterendhook\par
\sphinxattableend\end{savenotes}


\section{What’s new in this release?}
\label{\detokenize{Getting_Started:what-s-new-in-this-release}}
\begin{sphinxadmonition}{note}{Note:}
\sphinxAtStartPar
For further details see the Virtel Technical Newsletter TN202403: Whats new in Virtel 4.62.
\end{sphinxadmonition}


\chapter{Before you install the product}
\label{\detokenize{Getting_Started:before-you-install-the-product}}

\section{Prepare to download the product}
\label{\detokenize{Getting_Started:prepare-to-download-the-product}}
\sphinxAtStartPar
The Virtel products and PTFs can be downloaded from the SysperTec FTP server, at \sphinxurl{https://ftp-group.syspertec.com}

\sphinxAtStartPar
Credentials to access this server can be requested from SysperTec support at \sphinxurl{https://ftp-group.syspertec.com/request/}


\section{Obtain your product license key}
\label{\detokenize{Getting_Started:obtain-your-product-license-key}}
\sphinxAtStartPar
To start the Virtel STC, you will need a valid product license key, that will be supplied by SysperTec support (please contact \sphinxhref{mailto:support@syspertec.com}{support@syspertec.com}). This key needs to be pasted into the VIRTCTxx configuration file before it is assembled. See the Virtel Installation Guide for more details on this process.


\section{Access your documentation}
\label{\detokenize{Getting_Started:access-your-documentation}}
\sphinxAtStartPar
Virtel documentation (including the detailed installation guide) is available in PDF format on the SysperTec FTP server at \sphinxurl{https://ftp-group.syspertec.com} or in HTML format at \sphinxurl{https://virtel.readthedocs.io}


\section{Security planning}
\label{\detokenize{Getting_Started:security-planning}}
\sphinxAtStartPar
To provide secure HTTP (https) sessions between the mainframe and client browsers, VIRTEL uses the system TLS services:
\begin{itemize}
\item {} 
\sphinxAtStartPar
On z/OS, the IBM Application Transparent Transport Layer Security (AT\sphinxhyphen{}TLS) feature of z/OS Communication Server: AT\sphinxhyphen{}TLS allows socket applications to access encrypted sessions by invoking system SSL within the transport layer of the TCP/IP stack. The Policy Agent decides which connections are to use AT\sphinxhyphen{}TLS, and provides system SSL configuration for those connections. The application continues to send and receive clear text over the socket, but data sent over the network is protected by system SSL. Setup for AT\sphinxhyphen{}TLS is performed outside of Virtel. This process is described in the Installation Guide.

\item {} 
\sphinxAtStartPar
On z/VSE, the system TLS feature of the selected TCP/IP stack (BSI/CSI).

\end{itemize}

\sphinxAtStartPar
Virtel can also be interfaced with RACF to protect access to specific applications or features. This is detailed in the Installation Guide.

\sphinxAtStartPar
A new component called Virtel SSO is now available to interface Virtel with OIDC and SAML identity servers for Single\sphinxhyphen{}Sign\sphinxhyphen{}On. Please contact SysperTec Support for more details.


\chapter{Quick Installation}
\label{\detokenize{Getting_Started:quick-installation}}

\section{z/OS}
\label{\detokenize{Getting_Started:z-os}}
\sphinxAtStartPar
Here are the instructions to quick install and start Virtel on z/OS:
\begin{enumerate}
\sphinxsetlistlabels{\arabic}{enumi}{enumii}{}{.}%
\item {} 
\sphinxAtStartPar
After unzipping virtel462mvs.zip run job \$ALOCDSU to create the TRANSFER.XMIT file.

\item {} 
\sphinxAtStartPar
Upload the virtel462mvs.xmit file to the TRANSFER.XMIT file IN BINARY MODE.

\item {} 
\sphinxAtStartPar
Edit the job \$RESTDSU specifying the high\sphinxhyphen{}level qualifiers and SMS or volume serial information for the VIRTEL datasets, then run the job \$RESTDSU to create the VIRTEL datasets

\item {} 
\sphinxAtStartPar
Apply the PTFs in the allptfs\sphinxhyphen{}mvs462.txt file using job ZAPJCL in the VIRTEL CNTL library. If allptfs\sphinxhyphen{}mvs462.txt doesn’t exist skip this step.

\item {} 
\sphinxAtStartPar
Use the SETPROG APF command to add the VIRTEL LOADLIB to your system APF authorized program library list:

\begin{sphinxVerbatim}[commandchars=\\\{\}]
\PYG{n}{SETPROG} \PYG{n}{APF}\PYG{p}{,}\PYG{n}{ADD}\PYG{p}{,}\PYG{n}{DSN}\PYG{o}{=}\PYG{n}{yourqual}\PYG{o}{.}\PYG{n}{VIRT462}\PYG{o}{.}\PYG{n}{LOADLIB}\PYG{p}{,}\PYG{n}{VOL}\PYG{o}{=}\PYG{n}{volser}
\end{sphinxVerbatim}

\item {} 
\sphinxAtStartPar
Edit member VIRTCT01 in the VIRTEL CNTL library:
\begin{enumerate}
\sphinxsetlistlabels{\alph}{enumii}{enumiii}{(}{)}%
\item {} 
\sphinxAtStartPar
set the APPLID= parameter to the VTAM ACBNAME you will use to log
on to VIRTEL (the suggested value is APPLID=VIRTEL).

\item {} 
\sphinxAtStartPar
the TCP1= parameter must match the jobname of your z/OS
TCP/IP stack (the suggested value TCPIP is usually correct)

\item {} 
\sphinxAtStartPar
if you prefer VIRTEL to display English language panels, then set
the following parameters:

\begin{sphinxVerbatim}[commandchars=\\\{\}]
\PYG{n}{LANG}\PYG{o}{=}\PYG{l+s+s1}{\PYGZsq{}}\PYG{l+s+s1}{E}\PYG{l+s+s1}{\PYGZsq{}}\PYG{p}{,}                                               \PYG{o}{*}
\PYG{n}{COUNTRY}\PYG{o}{=}\PYG{n}{xxxx}\PYG{p}{,}                                           \PYG{o}{*}
\PYG{n}{DEFUTF8}\PYG{o}{=}\PYG{n}{IBMnnnn}\PYG{p}{,}                                        \PYG{o}{*}
\end{sphinxVerbatim}

\sphinxAtStartPar
(xxxx and nnnn depend on your country, see below).

\sphinxAtStartPar
Users in France should leave these parameters unchanged, as the
default is French language with codepage 1147.

\item {} 
\sphinxAtStartPar
set the COMPANY ADDR1 ADDR2 LICENCE EXPIRE CODE parameters using
the license key supplied to you by Syspertec.

\item {} 
\sphinxAtStartPar
Run the job ASMTCT in the VIRTEL CNTL library to assemble VIRTCT01
into the VIRTEL LOADLIB.

\end{enumerate}

\end{enumerate}

\begin{sphinxadmonition}{note}{Note:}
\sphinxAtStartPar
COUNTRY can be:

\sphinxAtStartPar
FR or FRANCE, US or USA, PORTUGAL, BRAZIL, AUSTRALIA,
NETHERLAND, BE or BELGIUM, SWITZERLAND, CANADA, ALBANIA,
NO ou NORWAY, DENMARK, DE or GERMANY, FI or FINLAND,
SWEDEN, IT or ITALY, SP or SPAIN, UK, IRELAND, IC or ICELAND.
If your country is not listed, specify COUNTRY=US
\end{sphinxadmonition}

\begin{sphinxadmonition}{note}{Note:}
\sphinxAtStartPar
DEFUTF8 is your default EBCDIC codepage.

\sphinxAtStartPar
Check the Virtel Installation Guide to see which SBCS and DBCS codepages are available in VIRTEL.
\end{sphinxadmonition}
\begin{enumerate}
\sphinxsetlistlabels{\arabic}{enumi}{enumii}{}{.}%
\setcounter{enumi}{6}
\item {} 
\sphinxAtStartPar
Edit member ARBOLOAD in the VIRTEL CNTL library:
\begin{enumerate}
\sphinxsetlistlabels{\alph}{enumii}{enumiii}{(}{)}%
\item {} 
\sphinxAtStartPar
change LANG=EN to LANG=FR if French language is desired

\item {} 
\sphinxAtStartPar
set LOAD= the name of your VIRTEL LOADLIB

\item {} 
\sphinxAtStartPar
set SAMP= the name of your VIRTEL SAMPLIB

\item {} 
\sphinxAtStartPar
set ARBO= the name of your VIRTEL ARBO file

\item {} 
\sphinxAtStartPar
set VTAMLST= the name of a your VIRTEL CNTL library. The job will create a sample VTAMLST member in this library.

\item {} 
\sphinxAtStartPar
CHANGE ALL ‘DBDCCICS’ ‘xxxxxx’ where xxxxxx is the APPLID of your CICS system.

\item {} 
\sphinxAtStartPar
if you plan to run Virtel Screen Redesigner, set VSR=YES

\item {} 
\sphinxAtStartPar
if you changed the APPLID of VIRTEL in step 6 from its default value VIRTEL, then you must also change the ACBNAME=parameter in step VTAMDEF near the end of the ARBOLOAD job. The value of ACBNAME= in ARBOLOAD must match the value of APPLID= in VIRTCT01. (i) if you plan to use Virtel Web Access for iPad, set IPAD=YES

\item {} 
\sphinxAtStartPar
Submit the job ARBOLOAD. This creates your VIRTEL CONFIGURATION (the ARBO file) and a sample VTAMLST member VIRTAPPL.

\end{enumerate}

\end{enumerate}

\begin{sphinxadmonition}{note}{Note:}
\sphinxAtStartPar
If you need to rerun the ARBOLOAD job, you must change PARM=’LOAD,NOREPL’ to PARM=’LOAD,REPL’.

\sphinxAtStartPar
If you wish to completely start over from the beginning, you can run the job ARBOBASE to delete and reinitialize the ARBO file, followed by a rerun of the ARBOLOAD job.
\end{sphinxadmonition}
\begin{enumerate}
\sphinxsetlistlabels{\arabic}{enumi}{enumii}{}{.}%
\setcounter{enumi}{7}
\item {} 
\sphinxAtStartPar
Submit the job ASMMOD from the VIRTEL CNTL library:

\sphinxAtStartPar
This job assembles the VIRTEL logon mode table (MODVIRT) into your SYS1.VTAMLIB dataset.  You will need to set the QUAL=parameter to match the high\sphinxhyphen{}level qualifiers of your SAMPLIB dataset.

\item {} 
\sphinxAtStartPar
Copy the VIRTAPPL member (created by the ARBOLOAD job in step 8) from the VIRTEL CNTL library into your SYS1.VTAMLST dataset.
\begin{description}
\sphinxlineitem{Now activate the VTAMLST member using this command::}
\sphinxAtStartPar
V NET,ACT,ID=VIRTAPPL

\end{description}

\item {} 
\sphinxAtStartPar
Edit the procedure VIRTEL4 in your VIRTEL CNTL library so that the high\sphinxhyphen{}level qualifiers match the names you used when you loaded the files in step 3.  Copy the procedure to your system PROCLIB, renaming it as VIRTEL.

\item {} 
\sphinxAtStartPar
Ask your security administrator to create a userid for the VIRTEL started task, and to authorize this userid to access the datasets you created in step 3. This userid must also have an OMVS segment which authorizes VIRTEL to use TCP/IP. Your security administrator can use the job RACFSTC in the VIRTEL SAMPLIB as an example.

\item {} 
\sphinxAtStartPar
Start VIRTEL

\item {} 
\sphinxAtStartPar
You can now logon to VIRTEL from a 3270 terminal using the APPLID specified in the VIRTCT01, and you can display the VIRTEL Web Access menu in your web browser using the following URL:

\sphinxAtStartPar
\sphinxurl{http://nnn.nnn.nnn.nnn:41001}

\sphinxAtStartPar
where nnn.nnn.nnn.nnn is the IP address of your z/OS system.

\item {} 
\sphinxAtStartPar
The supplied system is configured with security disabled. If you wish, you can activate external security using RACF, ACF2, or TOP SECRET; please refer to separate documentation.

\item {} 
\sphinxAtStartPar
Apply any “update” maintenance (virtel462updtnnnn.zip) according to the instructions in the Readme\sphinxhyphen{}updtnnnn.txt file in the virtel462updtnnnn.zip if available. Skip this step if no zip file is available.

\end{enumerate}


\section{VSEn}
\label{\detokenize{Getting_Started:vsen}}\begin{enumerate}
\sphinxsetlistlabels{\arabic}{enumi}{enumii}{}{.}%
\item {} 
\sphinxAtStartPar
Virtel is provided as an AWS tape file. Load the installation jobs into the POWER READER QUEUE using a S RDR,cuu command.

\item {} 
\sphinxAtStartPar
Define the VIRTvrr.SUBLIB sublibrary using the VIRTLIB job

\item {} 
\sphinxAtStartPar
Load the CIL and SSL libraries using the VIRTCIL and VIRTSSL jobs

\item {} 
\sphinxAtStartPar
Define the Virtel runtime files, using the VIRTVS job

\item {} 
\sphinxAtStartPar
Customize and assemble the VIRTCT:
\begin{enumerate}
\sphinxsetlistlabels{\alph}{enumii}{enumiii}{(}{)}%
\item {} 
\sphinxAtStartPar
set the APPLID= parameter to the VTAM ACBNAME you will use to log on to VIRTEL (the suggested value is APPLID=VIRTEL).

\item {} 
\sphinxAtStartPar
if you prefer VIRTEL to display English language panels, then set the following parameters:

\begin{sphinxVerbatim}[commandchars=\\\{\}]
\PYG{n}{LANG}\PYG{o}{=}\PYG{l+s+s1}{\PYGZsq{}}\PYG{l+s+s1}{E}\PYG{l+s+s1}{\PYGZsq{}}\PYG{p}{,}                                               \PYG{o}{*}
\PYG{n}{COUNTRY}\PYG{o}{=}\PYG{n}{xxxx}\PYG{p}{,}                                           \PYG{o}{*}
\PYG{n}{DEFUTF8}\PYG{o}{=}\PYG{n}{IBMnnnn}\PYG{p}{,}                                        \PYG{o}{*}
\end{sphinxVerbatim}

\sphinxAtStartPar
(xxxx and nnnn depend on your country, see below).

\sphinxAtStartPar
Users in France should leave these parameters unchanged, as the default is French language with codepage 1147.

\item {} 
\sphinxAtStartPar
set the COMPANY ADDR1 ADDR2 LICENCE EXPIRE CODE parameters using the license key supplied to you by Syspertec.

\item {} 
\sphinxAtStartPar
Run the job ASMTCT  to assemble the TCT table into the VIRTEL LOADLIB.

\end{enumerate}

\item {} 
\sphinxAtStartPar
Assemble the VTAM mode table using the VIRMOD job

\item {} 
\sphinxAtStartPar
Update the VIRARBO file (ARBOLOAD) using the VIRCONF job

\item {} 
\sphinxAtStartPar
Define the VTAM application relays using the VIRTAPPL job

\item {} 
\sphinxAtStartPar
Define the VIRTEL start procedure

\item {} 
\sphinxAtStartPar
Start VIRTEL

\item {} 
\sphinxAtStartPar
You can now logon to VIRTEL from a 3270 terminal using the APPLID specified in the VIRTCT01, and you can display the VIRTEL Web Access menu in your web browser using the following URL:
\begin{quote}

\sphinxAtStartPar
\sphinxurl{http://nnn.nnn.nnn.nnn:41001}
\end{quote}

\sphinxAtStartPar
where nnn.nnn.nnn.nnn is the IP address of your z/OS system.

\item {} 
\sphinxAtStartPar
Apply any “update” maintenance (virtel462updtnnnn.zip) according to the instructions in the Readme\sphinxhyphen{}updtnnnn.txt file in the virtel462updtnnnn.zip if available. Skip this step if no zip file is available.

\end{enumerate}


\section{Accessing SysperTec support}
\label{\detokenize{Getting_Started:accessing-syspertec-support}}
\sphinxAtStartPar
To contact SysperTec support, please send an email to \sphinxhref{mailto:support@syspertec.com}{support@syspertec.com}. If you have the necessary credentials, you can also open an issue at \sphinxurl{https://support.syspertec.com}


\chapter{Appendix}
\label{\detokenize{Getting_Started:appendix}}

\section{Trademarks}
\label{\detokenize{Getting_Started:trademarks}}
\sphinxAtStartPar
SysperTec, the SysperTec logo, syspertec.com and VIRTEL are trademarks or registered trademarks of SysperTec
Group, registered in France and other countries.

\sphinxAtStartPar
IBM, VTAM, CICS, IMS, RACF, DB2, MVS, WebSphere, MQSeries, System z are trademarks or registered trademarks of
International Business Machines Corp., registered in United States and other countries.

\sphinxAtStartPar
Adobe, Acrobat, PostScript and all Adobe\sphinxhyphen{}based trademarks are either registered trademarks or trademarks of Adobe
Systems Incorporated in the United States and other countries.

\sphinxAtStartPar
Microsoft, Windows, Windows NT, and the Windows logo are trademarks of Microsoft Corporation in the United States
and other countries.

\sphinxAtStartPar
UNIX is a registered trademark of The Open Group in the United States and other countries.
Java and all Java\sphinxhyphen{}based trademarks and logos are trademarks or registered trademarks of Oracle and/or its affiliates.

\sphinxAtStartPar
Linux is a trademark of Linus Torvalds in the United States, other countries, or both.

\sphinxAtStartPar
Other company, product, or service names may be trademarks or service names of others.


\section{Open Source Software}
\label{\detokenize{Getting_Started:open-source-software}}
\sphinxAtStartPar
The current VIRTEL Web Access product uses the following open source software:
\begin{itemize}
\item {} \begin{description}
\sphinxlineitem{jQuery}
\sphinxAtStartPar
Under MIT license \sphinxhyphen{} \sphinxurl{https://jquery.org/license/}

\end{description}

\item {} \begin{description}
\sphinxlineitem{StoreJson}
\sphinxAtStartPar
Under MIT license \sphinxhyphen{} \sphinxurl{https://github.com/marcuswestin/store.js/commit/baf3d41b7092f0bacd441b768a77650199c25fa7}

\end{description}

\item {} \begin{description}
\sphinxlineitem{jQuery\_UI}
\sphinxAtStartPar
Under MIT license \sphinxhyphen{} \sphinxurl{http://en.wikipedia.org/wiki/JQuery\_UI}

\end{description}

\end{itemize}



\renewcommand{\indexname}{Index}
\printindex
\end{document}