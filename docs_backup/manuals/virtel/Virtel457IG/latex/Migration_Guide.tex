%% Generated by Sphinx.
\def\sphinxdocclass{report}
\documentclass[letterpaper,10pt,english]{sphinxmanual}
\ifdefined\pdfpxdimen
   \let\sphinxpxdimen\pdfpxdimen\else\newdimen\sphinxpxdimen
\fi \sphinxpxdimen=.75bp\relax



\usepackage{cmap}

\usepackage{amsmath,amssymb,amstext}
\usepackage{babel}

\usepackage[Bjarne]{fncychap}
\usepackage[dontkeepoldnames]{sphinx}

\usepackage{geometry}

% Include hyperref last.
\usepackage{hyperref}
% Fix anchor placement for figures with captions.
\usepackage{hypcap}% it must be loaded after hyperref.
% Set up styles of URL: it should be placed after hyperref.
\urlstyle{same}
\addto\captionsenglish{\renewcommand{\contentsname}{Table of Contents:}}

\addto\captionsenglish{\renewcommand{\figurename}{Fig.}}
\addto\captionsenglish{\renewcommand{\tablename}{Table}}
\addto\captionsenglish{\renewcommand{\literalblockname}{Listing}}

\addto\captionsenglish{\renewcommand{\literalblockcontinuedname}{continued from previous page}}
\addto\captionsenglish{\renewcommand{\literalblockcontinuesname}{continues on next page}}

\addto\extrasenglish{\def\pageautorefname{page}}

\setcounter{tocdepth}{1}


% Enable unicode and use Courier New to ensure the card suit
% characters that are part of the 'random' module examples
% appear properly in the PDF output.
\usepackage{fontspec}
\setmonofont{Courier New}


\title{Virtel Migration Guide}
\date{Jul 26, 2017}
\release{4.57}
\author{Syspertec Communications}
\newcommand{\sphinxlogo}{\vbox{}}
\renewcommand{\releasename}{Release}
\makeindex

\begin{document}

\maketitle
\sphinxtableofcontents
\phantomsection\label{\detokenize{Migration_Guide::doc}}


\sphinxstylestrong{VIRTEL Migration Guide}

Version : 4.57

Release Date : 31 Jul 2017 Publication Date : 24/07/2017

Syspertec Communication

196, Bureaux de la Colline 92213 Saint-Cloud Cedex Tél. : +33 (0) 1 46 02 60 42

\sphinxhref{http://www.syspertec.com/}{www.syspertec.com}

NOTICE
\begin{quote}

Reproduction, transfer, distribution, or storage, in any form, of all or any part of
the contents of this document, except by prior authorization of SysperTec
Communication, is prohibited.

Every possible effort has been made by SysperTec Communication to ensure that this document
is complete and relevant. In no case can SysperTec Communication be held responsible for
any damages, direct or indirect, caused by errors or omissions in this document.

As SysperTec Communication uses a continuous development methodology; the information
contained in this document may be subject to change without notice. Nothing in this
document should be construed in any manner as conferring a right to use, in whole or in
part, the products or trademarks quoted herein.

“SysperTec Communication” and “VIRTEL” are registered trademarks. Names of other products
and companies mentioned in this document may be trademarks or registered trademarks of
their respective owners.
\end{quote}

\newpage

\index{Migrating to Virtel V4.57}

\chapter{Migrating to Virtel V4.57}
\label{\detokenize{Migration_Guide:v457mi-introduction}}\label{\detokenize{Migration_Guide:index-0}}\label{\detokenize{Migration_Guide:migrating-to-virtel-v4-57}}\label{\detokenize{Migration_Guide:migration-guide-v4-57}}\label{\detokenize{Migration_Guide:virtel457mi}}

\section{Overview}
\label{\detokenize{Migration_Guide:overview}}
The upgrade procedure detailed below will restore a complete “new” system on the mainframe.

Once restored you will have to integrate all the new librairies (sources or binaries), but keep ALL existing VSAM files except SAMP.TRSF. This VSAM file must be upgraded as part of the upgrade process.

Also, before migration, you must ensure that:
- The current SAMP.TRSF file does not contain any other directories other than W2H-DIR, SCE-DIR, DYN-DIR, IPA-DIR, VSR-DIR, VTX-DIR, YUI-DIR or DOC-DIR. Some of the directories in this list may not exist on your system.
- Directories which are not in this list must be removed from the current SAMP.TRSF file. You must transfer those directories and their cotents to another .TRSF file.
- Similarly, you must ensure that you have not saved in any these directories. If this is the case, you must either back up your private files to another directory hosted on another VSAM file and restore them after taking into account the new SAMP.TRSF file.

\newpage

\index{Upgrade Procedure}

\section{Upgrade Procedure}
\label{\detokenize{Migration_Guide:upgrade-procedure}}\label{\detokenize{Migration_Guide:index-1}}
Download from our file server at \sphinxurl{http://ftp.syspertec.com}
- /VIRTEL 4.57/Produits/virtel457mvs.zip
- /VIRTEL 4.57/PTFS/allptfs-mvs457.txt
- /VIRTEL 4.57/PTFS/virtel457updtnnnn.zip
The procedure for upgrading from a previous release of VIRTEL (version 4.00 or later) is as follows:

\begin{sphinxadmonition}{note}{Note:}
You can keep your existing VIRARBO file for your existing configuration.
\end{sphinxadmonition}
\begin{enumerate}
\item {} 
Upload and unpack the virtel456mvs.xmit file (job \$RESTDSU)

\item {} 
Apply PTFs using AMASPZAP,PARM=IGNIDRFULL (job ZAPJCL from the new VIRT457.SAMPLIB)

\item {} 
Copy your VIRTCTnn from the old VIRTnnn.CNTL library to the new VIRT457.CNTL

\item {} 
Reassemble your VIRTCTnn module using the ASMTCT job in VIRT457.CNTL

\item {} 
If you have any scenario or user exit modules, copy them to the VIRT456.CNTL library and reassemble them using the ASMSCEN and ASMEXIT jobs respectively.

\item {} 
Add the new VIRT456.LOADLIB library to the system APF list in the MVS PARMLIB and use the SETPROG command to authorize the VIRT456.LOADLIB library.

\item {} 
Edit your VIRTEL procedure in the MVS PROCLIB, to ensure that:

\end{enumerate}
\begin{itemize}
\item {} 
The STEPLIB and DFHRPL DD statements reference the new VIRTxxx.LOADLIB

\item {} 
The SERVLIB DD statement references the new VIRTxxx.SERVLIB

\item {} 
The SAMPTRSF DD statement references the new VIRTxxx.SAMP.TRSF

\end{itemize}
\begin{enumerate}
\setcounter{enumi}{7}
\item {} 
Apply recommandations as shown below in “Specifics points to be checked”.

\item {} 
Stop and restart VIRTEL version 4.57

\item {} 
If you have modified the default values for the VIRTEL Web Access Settings (as described in the VIRTEL Web Access Guide), upload your customize w2hparm.js file into the CLI-DIR directory and ensure that your W2H-03P and CLI-03P transactions specify APPL=CLI-DIR. Use job CUSTCSS in VIRT457.SAMPLIB to create these transactions if necessary.

\item {} 
Apply any “update” maintenance (virtel456updtnnnn.zip) according to the instructions in the Readme-updtnnnn.txt file in the zip.

\end{enumerate}

\index{Check list and precautions}\index{All versions!Check list and precautions}\index{Check list and precautions!All versions}

\chapter{Check list and precautions}
\label{\detokenize{Migration_Guide:check-list-and-precautions}}\label{\detokenize{Migration_Guide:index-2}}
Before migration, you must check to following:-


\section{All versions}
\label{\detokenize{Migration_Guide:all-versions}}\begin{itemize}
\item {} 
YOU MUST save your current ARBO file content using a VIRCONF UNLOAD process. See “VIRCONF - Unloading a VIRARBO file” in your CURRENT Virtelxxx\_Installation\_User\_Guide.

\item {} 
If you using some “custom” files (“custom.js”, “customs.css” or custom “w2hparm.js”) YOU MUST check that they are stored in a directory other than W2H-DIR. If not you must store your “custom” files in another directory (CLI-DIR for example).

\item {} 
Ensure that the directory on which “custom” file are stored is not on VIRTxxx.SAMP.TRSF VSAM file. VIRTxxx.SAMP.TRSF is completely replaced as part of the migration procedure. Any customized elements will be lost or overwritten.

\end{itemize}

\index{Version older than 4.53!Check list and precautions}\index{Check list and precautions!Version older than 4.53}

\section{Upgrading from a version prior to 4.53}
\label{\detokenize{Migration_Guide:upgrading-from-a-version-prior-to-4-53}}\label{\detokenize{Migration_Guide:index-3}}
\begin{sphinxadmonition}{danger}{Danger:}
YOU MUST:
\end{sphinxadmonition}
\begin{itemize}
\item {} 
Add (or modify) the BFVSAM parameter in the VIRTCT to specify BFVSAM=32768. An “OPEN ERROR DC” on the SAMP.TRSF file will occur at STC startup if this is notdone.

\item {} 
Check the content of the transaction with external name = “Administ” - W2H-20 - present under the WEB2HOST entry point and enter “ADMINVWM” in the “Output Scenario” field if the current value is different.

\end{itemize}

For performance YOU CAN:
\begin{itemize}
\item {} 
Modify the Shareoptions of the ARBO File from (4 3) to (2 3). To do this, use job ALTERSHR from the new VIRT456.SAMPLIB.

\item {} 
Increase the CISIZE of the data part of ALL existing TRSF file (except for the delivered SAMP.TRSF which is already correctly defined) to specify, CIZISE(32768) or CISIZE(16384) on the DATA part of the files. Use the job TRSFREOR from VIRT457.SAMPLIB.

\end{itemize}

\index{Version 4.53!Check list and precautions}\index{Check list and precautions!Version 4.53}

\section{Upgrading from a version prior to 4.54}
\label{\detokenize{Migration_Guide:index-4}}\label{\detokenize{Migration_Guide:upgrading-from-a-version-prior-to-4-54}}
In addition to the above instructions
- If the directory SCE-DIR does not exist in your environment, YOU MUST run the job MIGR454A found in the SAMPLIB library.

\index{Version 4.54!Check list and precautions}\index{Check list and precautions!Version 4.54}

\section{Upgrading from a version prior to 4.55}
\label{\detokenize{Migration_Guide:index-5}}\label{\detokenize{Migration_Guide:upgrading-from-a-version-prior-to-4-55}}\begin{itemize}
\item {} 
If you have developed JavaScript code which manipulates the status bar located at the bottom of the 3270 screen, YOU MUST refer to “Example: Modifying the text of the status bar” in “VIRTEL Web Access” document and review your changes.

\item {} 
You must modify the definition of the WEB2HOST entry point to reference the SCE-DIR in the field named DIRECTORY FOR SCENARIO.

\item {} 
If your are using Virtel Web Access (VWA) features, you must add the following parameter in your VIRTCT: HTSET1=(OPTION-DEFAULT-COMPATIBILITY).

\end{itemize}

\index{Version 4.55!Check list and precautions}\index{Check list and precautions!Version 4.55}

\section{Upgrading from a version prior to 4.56}
\label{\detokenize{Migration_Guide:upgrading-from-a-version-prior-to-4-56}}\label{\detokenize{Migration_Guide:index-6}}
None.

\index{Version 4.56!Check list and precautions}\index{Check list and precautions!Version 4.56}

\section{Upgrading from a version prior to 4.57}
\label{\detokenize{Migration_Guide:upgrading-from-a-version-prior-to-4-57}}\label{\detokenize{Migration_Guide:index-7}}
None

\newpage

\index{Running under VSE}

\chapter{Running under VSE}
\label{\detokenize{Migration_Guide:running-under-vse}}\label{\detokenize{Migration_Guide:index-8}}\begin{itemize}
\item {} 
Send a request to \sphinxhref{mailto:support@syspertec.com}{support@syspertec.com} for any missing JCL described above.

\item {} 
You MUST modify the VIRTEL startup JCL to increase the SIZE parameter from 40K to 80K (// EXEC VIR0000,SIZE=80K,DSPACE=2M).

\end{itemize}

\newpage

\index{What's new in this release.}

\chapter{What’s new in this release}
\label{\detokenize{Migration_Guide:what-s-new-in-this-release}}\label{\detokenize{Migration_Guide:index-9}}
\sphinxstyleemphasis{VIRTEL Web Access:}
\begin{itemize}
\item {} 
Initial PASSPHRASE support.

\item {} 
Bidirectional presentation support.

\item {} 
Enhanced font stretch mode. Optimization of screen size.

\item {} 
Additional RACHECK support for ForceLUNAME

\item {} 
Licence warning feature

\item {} 
Enhancements to USSMSG10 support module

\item {} 
Enhancements to Virtel Web Macro interface (VWM)
\begin{itemize}
\item {} 
Keyboard mapping enhancements

\end{itemize}

\item {} 
Enhancements to Virtel Dynamic Directories Interface (DDI)
\begin{itemize}
\item {} 
Hotkey support for DDI macros

\item {} 
New refresh options.

\end{itemize}

\end{itemize}

\sphinxstyleemphasis{VIRTEL Web Modernisation \& Integration:}
\begin{itemize}
\item {} 
Enhancements to COPY\$ NAME-OF. Support for TERMINAL, GROUP and RELAY items.

\item {} 
Enhancements to DEFAULT-FILED-WITH-CURSOR statement.

\end{itemize}

\sphinxstyleemphasis{Miscellaneous:}
\begin{itemize}
\item {} 
Customizable HELP solution.

\item {} 
DNS access to resolve IP address or DNS name.

\item {} 
TCT option to support mixed case passwords

\item {} 
Additional language support

\item {} 
Batch export/import of RAW TRSF files.

\end{itemize}



\renewcommand{\indexname}{Index}
\printindex
\end{document}